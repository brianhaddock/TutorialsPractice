\NormalFont\ShortTitle{2 Chronicles}
{\MT 2 Chronicles

\par }\ChapOne{1}{\SH The Lord Gives Solomon Wisdom
\par }{\PP \VerseOne{1}Solomon
son
of David
solidified
his royal
authority, for the
{\ND{Lord}}
his God
was with
him and magnified him greatly.
\par }{\PP \VS{2}Solomon
addressed
all
Israel,
including those who commanded units
of a thousand
and a hundred,
the judges,
and all
the leaders
of all
Israel
who were heads
of families.
\VS{3}Solomon
and the entire
assembly
went to the worship center
in Gibeon,
for
the tent
where
they met
God
was located there,
which
Moses
the
{\ND{Lord}}’s
servant
had made
in the wilderness.
\VS{4}(Now
David
had brought up
the ark
of God
from Kiriath Jearim
to the place he
had prepared
for it, for
he
had pitched
a tent
for it in Jerusalem.
\VS{5}But the bronze
altar
made
by Bezalel
son
of Uri,
son
of Hur,
was in front
of the
{\ND{Lord}}’s
tabernacle.
Solomon
and the entire assembly
prayed to him there.)
\VS{6}Solomon
went up
to the bronze
altar
before
the {\ND{Lord}}
which
was at the meeting
tent,
and he offered
up a thousand
burnt sacrifices.
\par }{\PP \VS{7}That night
God
appeared
to Solomon
and said
to him, “Tell me
what
I should give you.”
\VS{8}Solomon
replied
to God,
“You
demonstrated
great
loyalty
to my father
David
and have made
me
king
in his place.
\VS{9}Now,

{\ND{Lord}}
God,
may your promise
to
my father
David
be realized,
for
you
have made me king
over
a great nation
as numerous
as the dust
of the earth.
\VS{10}Now
give
me wisdom
and discernment
so I can effectively lead
this
nation.
Otherwise
no one
is able to make judicial decisions
for this
great
nation of yours.”
\par }{\PP \VS{11}God
said
to Solomon,
“Because
you desire
this,
and did not
ask
for riches,
wealth,
and honor,
or for vengeance on your
enemies,
and because you did not ask
for long
life,
but requested
wisdom
and discernment
so you can make judicial decisions
for my people
over
whom
I have made you king,
\VS{12}you are granted wisdom
and discernment.
Furthermore I am giving
you riches,
wealth,
and honor
surpassing
that
of any king
before
or after you.”
\par }{\PP \VS{13}Solomon
left the meeting
tent
at the worship center
in Gibeon
and went to Jerusalem,
where he reigned
over
Israel.
\par }{\SH Solomon’s Wealth
\par }{\PP \VS{14}Solomon
accumulated
chariots
and horses.
He had
1,400
chariots
and 12,000
horses.
He kept
them in assigned cities
and in
Jerusalem.
\VS{15}The king
made
silver
and gold
as plentiful in Jerusalem
as stones;
cedar
was
as plentiful
as sycamore fig trees
are in the lowlands.
\VS{16}Solomon
acquired
his horses
from Egypt
and from Que;
the king’s
traders
purchased them from Que.
\VS{17}They paid
600
silver pieces
for each chariot
from Egypt,
and 150
silver pieces for each horse.
They also
sold
chariots and horses to all
the kings
of the Hittites
and to the kings
of Syria.

\par }\Chap{2}{\PP \VerseOne{1} Solomon
ordered
a temple
to be built
to honor
the {\ND{Lord}}, as well as a royal
palace for himself.
\VS{2}Solomon
had
70,000
common laborers
and 80,000
stonecutters
in the hills,
in addition
to 3,600
supervisors.
\par }{\PP \VS{3}Solomon
sent
a message to
King
Huram
of Tyre: “Help me as
you did
my father
David,
when you sent
him cedar
logs for the construction
of his palace.
\VS{4}Look,
I
am ready to build
a temple
to honor
the
{\ND{Lord}}
my God
and to dedicate
it to him in order to burn
fragrant
incense
before
him, to set out the bread that is regularly
displayed,
and to offer burnt sacrifices
each morning
and evening,
and on Sabbaths,
new moon
festivals,
and at other times appointed by the
{\ND{Lord}}
our God.
This
is something Israel
must do on
a permanent basis.
\VS{5}I
will build
a great
temple,
for
our God
is greater
than all
gods.
\VS{6}Of course,
who can
really build
a temple
for him, since
the sky
and the highest
heavens
cannot
contain
him? Who
am
I that
I should build
him a temple! It will really
be only
a place to offer
sacrifices before him.
\par }{\PP \VS{7}“Now
send
me a man
who is skilled
in working
with gold,
silver,
bronze,
and iron,
as well as purple,
crimson,
and violet
colored fabrics, and who knows
how to engrave.
He will work with
my skilled craftsmen
here in Jerusalem
and Judah,
whom
my father
David
provided.
\VS{8}Send
me cedars,
evergreens,
and algum
trees
from Lebanon,
for
I
know
your servants
are adept
at cutting
down
trees
in Lebanon.
My servants
will work with
your servants
\VS{9}to supply
me with large quantities
of timber,
for
I am
building
a great,
magnificent
temple.
\VS{10}Look,
I will pay your servants
who
cut
the timber
20,000 kors
of ground wheat,
20,000
kors
of barley,
120,000
gallons
of wine,
and 120,000
gallons
of olive oil.”
\par }{\PP \VS{11}King
Huram
of Tyre
sent
this letter
to
Solomon: “Because the
{\ND{Lord}}
loves
his people,
he has made
you their king.”
\VS{12}Huram
also said,
“Worthy of praise
is the
{\ND{Lord}}
God
of Israel,
who
made
the sky
and the earth! He has
given
David
a wise
son
who has discernment
and insight
and will build
a temple
for the
{\ND{Lord}}, as well as a royal
palace for himself.
\VS{13}Now
I am sending
you Huram
Abi,
a skilled
and capable
man,
\VS{14}whose mother
is a Danite
and whose father
is a Tyrian.
He knows
how to work
with gold,
silver,
bronze,
iron,
stones,
and wood,
as well as purple,
violet,
white,
and crimson
fabrics. He knows how to do
all
kinds of engraving
and understands
any
design
given
to him. He will work with
your skilled craftsmen
and the skilled craftsmen
of my lord
David
your father.
\VS{15}Now
let my lord
send
to his servants
the wheat,
barley,
olive oil,
and wine
he has promised;
\VS{16}we
will get
all
the timber
you need
from
Lebanon
and bring
it in raft-like
bundles by sea
to Joppa.
You
can then haul it on up
to Jerusalem.”
\par }{\PP \VS{17}Solomon
took a census
of all
the male
resident foreigners
in the land
of Israel,
after
the census
his father
David
had taken.
There were
153,600 in all.
\VS{18}He designated
70,000
as common laborers,
80,000
as stonecutters
in the hills,
and 3,600
as supervisors
to make sure the people
completed the work.

\par }\Chap{3}{\PP \VerseOne{1}Solomon
began
building
the
{\ND{Lord}}’s
temple
in Jerusalem
on Mount
Moriah,
where
the
{\ND{Lord}} had
appeared
to his father
David.
This was the place
that
David
prepared
at the threshing floor
of Ornan
the Jebusite.
\VS{2}He began
building
on the second
day of the second
month
of the fourth
year
of his reign.
\par }{\PP \VS{3}Solomon
laid
the foundation
for God’s
temple;
its length
(determined according to the old
standard
of measure) was 90 feet,
and its width
30 feet.
\VS{4}The porch
in front
of the main hall was 30 feet
long,
corresponding
to the width
of the temple,
and its height
was 30 feet.
He plated
the inside
with pure
gold.
\VS{5}He paneled
the main hall
with boards
made from evergreen
trees
and plated
it with fine
gold,
decorated
with palm trees
and chains.
\VS{6}He decorated
the temple
with precious
stones;
the gold
he used came from Parvaim.
\VS{7}He overlaid
the temple’s rafters,
thresholds,
walls
and doors
with gold;
he carved decorative
cherubim
on
the walls.
\par }{\PP \VS{8}He made
the most
holy place;
its length
was 30 feet,
corresponding
to the width
of the temple,
and its width
30 feet.
He plated
it with 600
talents
of fine
gold.
\VS{9}The gold
nails
weighed
50
shekels;
he also plated
the upper areas
with gold.
\VS{10}In the most
holy place
he made
two
images
of cherubim
and plated
them with
gold.
\VS{11}The combined wing
span of the cherubs
was 30 feet.
One
of the first
cherub’s wings
was seven and one-half feet
long
and touched
one wall
of the temple;
its other
wing
was also seven and one-half feet
long and touched
one of the second cherub’s
wings.
\VS{12}Likewise one
of the second cherub’s
wings
was seven and one-half feet
long and touched
the other wall
of the temple;
its other
wing
was also seven and one-half feet
long and touched
one of the first
cherub’s
wings.
\VS{13}The combined wingspan
of these
cherubim
was 30 feet.
They
stood
upright,
facing
inward.
\VS{14}He made
the curtain
out of violet,
purple,
crimson,
and white fabrics,
and embroidered
on
it decorative cherubim.
\par }{\PP \VS{15}In front of
the temple
he made
two
pillars
which
had a combined length
of 52½ feet,
with each having a plated capital
seven and one-half feet high.
\VS{16}He made
ornamental chains
and put
them on
top
of the pillars.
He also made
one hundred
pomegranate-shaped ornaments
and arranged
them within the chains.
\VS{17}He set up
the
pillars
in
front
of the temple,
one
on the right side
and the other
on the left.
He named
the one on the right
Jachin,
and the one on the left
Boaz.

\par }\Chap{4}{\PP \VerseOne{1}He made
a bronze
altar,
30 feet
long,
30 feet
wide,
and 15 feet
high.
\VS{2}He also made
the
big
bronze basin
called “The Sea.”
It measured 15 feet
from rim
to
rim,
was circular
in shape,
and stood seven and one-half feet
high.
Its circumference
was 45 feet.
\VS{3}Images
of bulls
were under
it all the way
around,
ten
every eighteen inches
all the way around.
The bulls
were in two
rows
and had been cast
with “The Sea.”
\VS{4}“The Sea” stood
on
top
of twelve
bulls.
Three
faced
northward,
three
westward,
three
southward,
and three
eastward.
“The Sea”
was placed
on
top
of them, and they all
faced outward.
\VS{5}It was four fingers
thick
and its rim
was like that
of a cup
shaped
like a lily
blossom.
It could hold
18,000 gallons.
\VS{6}He made
ten
washing basins;
he put
five
on the south
side and five
on the north
side. In them they rinsed
the items used
for burnt sacrifices;
the priests
washed
in “The Sea.”
\par }{\PP \VS{7}He made
ten
gold
lampstands
according
to specifications
and put
them in the temple,
five
on the right
and five
on the left.
\VS{8}He made
ten
tables
and set
them in the temple,
five
on the right
and five
on the left.
He also made
one hundred
gold
bowls.
\VS{9}He made
the courtyard
of the priests
and the large
enclosure
and its doors;
he plated
their doors
with bronze.
\VS{10}He put
“The Sea”
on the south
side,
in the southeast corner.
\par }{\PP \VS{11}Huram
Abi made
the pots,
shovels,
and bowls.
He
finished
all the work
on God’s
temple
he had been assigned
by King
Solomon.
\VS{12}He made the two
pillars,
the two
bowl-shaped
tops
of the pillars,
the latticework
for the bowl-shaped
tops
of the two pillars,
\VS{13}the four
hundred
pomegranate-shaped
ornaments for the latticework of the two
pillars (each latticework
had two
rows
of these ornaments
at the bowl-shaped
top
of the pillar),
\VS{14}the ten movable stands with their ten basins,
\VS{15}the big bronze basin
called “The Sea”
with its twelve
bulls
underneath,
\VS{16}and the pots,
shovels,
and meat forks.
All
the items
King
Solomon
assigned
Huram
Abi
to make for the
{\ND{Lord}}’s
temple
were made from polished
bronze.
\VS{17}The king
had them cast
in earthen
foundries
in the region
of the Jordan
between
Succoth
and Zarethan.
\VS{18}Solomon
made
so many
of these
items
they did not
weigh
the bronze.
\par }{\PP \VS{19}Solomon
also
made
these items
for God’s
temple: the gold
altar,
the tables
on
which the Bread
of the Presence was kept,
\VS{20}the pure
gold
lampstands
and their lamps
which burned
as specified
at the entrance to the inner sanctuary,
\VS{21}the pure
gold
flower-shaped
ornaments, lamps,
and tongs,
\VS{22}the pure gold trimming
shears, basins,
pans,
and censers,
and the gold
door
sockets for the inner
sanctuary
(the most
holy
place) and for the doors
of the main hall
of the temple.

\par }\Chap{5}{\PP \VerseOne{1}When Solomon
had finished
constructing
the
{\ND{Lord}}’s
temple,
he
put
the holy
items that belonged to his father
David
(the
silver,
gold,
and all
the other
articles) in the treasuries
of God’s
temple.
\par }{\SH Solomon Moves the Ark into the Temple
\par }{\PP \VS{2}Then
Solomon
convened
Israel’s
elders
– all
the leaders
of the Israelite
tribes
and families
– in Jerusalem,
so they could witness the transferal
of the ark
of the covenant
of the {\ND{Lord}}
from
the City
of David
(that
is, Zion).
\VS{3}All
the men
of Israel
assembled
before the king
during the festival
in the seventh
month.
\VS{4}When all
Israel’s
elders
had arrived,
the Levites
lifted
the ark.
\VS{5}The
priests
and Levites
carried
the ark,
the
tent
where God appeared
to his people, and all
the holy
items
in the tent.
\VS{6}Now King
Solomon
and all
the Israelites
who had assembled
with him went on
ahead
of the ark
and sacrificed
more sheep
and cattle
than could
be counted or numbered.
\par }{\PP \VS{7}The priests
brought
the ark
of the covenant
of the {\ND{Lord}}
to
its assigned place
in the inner sanctuary
of the temple,
in
the most
holy
place under
the wings
of the cherubs.
\VS{8}The cherubs’
wings
extended
over
the place
where the ark
sat; the cherubs
overshadowed
the ark
and its poles.
\VS{9}The poles
were so long
their ends
extending out from the ark
were visible
from
in front
of the inner sanctuary,
but they could not
be seen
from beyond that point.
They have remained
there
to this
very day.
\VS{10}There was nothing
in the ark
except
the two
tablets
Moses
had
placed
there in Horeb.
(It was there that
the {\ND{Lord}}
made an agreement
with
the Israelites
after he brought them out
of the land of Egypt.)
\par }{\PP \VS{11}The priests
left
the holy
place. All
the priests
who
participated had
consecrated
themselves, no
matter which division
they represented.
\VS{12}All
the Levites
who were musicians,
including
Asaph,
Heman,
Jeduthun,
and their sons
and relatives,
wore
linen.
They played cymbals
and stringed instruments
as they stood
east
of the altar.
They were accompanied
by 120
priests
who blew
trumpets.
\VS{13}The trumpeters
and musicians
played
together, praising
and giving thanks
to the
{\ND{Lord}}. Accompanied
by trumpets,
cymbals,
and other instruments,
they loudly praised
the {\ND{Lord}}, singing: “Certainly
he is good;
certainly
his loyal love
endures!” Then a cloud
filled
the
{\ND{Lord}}’s
temple.
\VS{14}The priests
could
not
carry out their duties
because
of the cloud;
the
{\ND{Lord}}’s
splendor
filled
God’s
temple.

\par }\Chap{6}{\PP \VerseOne{1}Then
Solomon
said,
“The
{\ND{Lord}}
has said
that he lives
in thick darkness.
\VS{2}O
{\ND{Lord}}, I
have built
a lofty
temple
for you, a place
where you can live
permanently.”
\VS{3}Then the king
turned
around and pronounced a blessing
over the whole
Israelite
assembly
as they stood there.
\VS{4}He said,
“The
{\ND{Lord}}
God
of Israel
is worthy
of praise because he has fulfilled
what
he promised
my father
David.
\VS{5}He told David, ‘Since
the day
I brought
my people
out of the land
of Egypt,
I have not
chosen
a city
from all
the tribes
of Israel
to build
a temple
in which
to live.
Nor
did I choose
a man
as leader
of
my people
Israel.
\VS{6}But now I have chosen
Jerusalem
as a place
to live,
and I have chosen
David
to lead
my people
Israel.’
\VS{7}Now
my father
David
had a strong desire
to build
a temple
to honor
the {\ND{Lord}}
God
of Israel.
\VS{8}The
{\ND{Lord}}
told
my father
David,
‘It is right
for
you to have
a strong desire
to build
a temple
to honor me.
\VS{9}But
you
will not
build
the temple;
your very own
son
will build
the temple
for my honor.’
\VS{10}The
{\ND{Lord}}
has kept
the promise
he made. I have
taken my father
David’s
place
and have occupied
the throne
of Israel,
as
the {\ND{Lord}}
promised.
I have built
this temple
for the honor
of the {\ND{Lord}}
God
of Israel
\VS{11}and set
up in it a place
for the ark
containing
the covenant
the {\ND{Lord}}
made
with
the Israelites.”
\par }{\PP \VS{12}He stood
before
the altar
of the {\ND{Lord}}
in front
of the entire
assembly
of Israel
and spread
out his hands.
\VS{13}Solomon
had
made
a bronze
platform
and had placed
it in the middle
of the enclosure.
It was seven and one-half feet
long,
seven and one-half feet
wide,
and four and one-half feet
high.
He stood
on
it and then got down
on
his knees
in front
of the entire
assembly
of Israel.
He spread out
his hands
toward the sky,
\VS{14}and prayed: “O
{\ND{Lord}}
God
of Israel,
there is no
god
like
you in heaven
or on earth! You maintain
covenantal
loyalty
to your servants
who obey
you with sincerity.
\VS{15}You have
kept
your word to your servant,
my father
David;
this
very day
you have
fulfilled
what
you promised.
\VS{16}Now,
O
{\ND{Lord}}
God
of Israel,
keep
the promise you made to your servant,
my father
David,
when
you said,
‘You will never
fail
to have a successor
ruling
before
me on
the throne
of Israel,
provided
that your descendants
watch
their step
and obey
my law
as
you have
done.’
\VS{17}Now,
O
{\ND{Lord}}
God
of Israel,
may the promise
you made to your servant
David
be realized.
\par }{\PP \VS{18}“God
does not really
live
with
humankind
on
the earth! Look,
if the sky
and the highest
heaven
cannot
contain
you, how much
less
this
temple
I have
built!
\VS{19}But
respond favorably
to your servant’s
prayer
and his request for help,
O
{\ND{Lord}}
my God.
Answer the desperate
prayer
your servant
is presenting to you.
\VS{20}Night
and day
may
you watch
over
this
temple,
the place
where
you promised
you would live.
May you answer
your servant’s
prayer
for
this
place.
\VS{21}Respond
to
the requests
of your servant
and your people
Israel
for this
place.
Hear
from
your heavenly
dwelling
place
and respond
favorably
and forgive.
\par }{\PP \VS{22}“When
someone
is accused of sinning
against his neighbor
and the latter pronounces
a curse
on the alleged offender
before
your altar
in this
temple,
\VS{23}listen
from
heaven
and make
a just decision
about your servants’
claims.
Condemn the guilty
party,
declare
the other innocent,
and give
both
of them what they deserve.
\par }{\PP \VS{24}“If
your people
Israel
are defeated
by an enemy
because
they sinned
against you, then if they come back
to you, renew
their allegiance
to you, and pray
for your help
before
you in this
temple,
\VS{25}then
listen
from
heaven,
forgive
the sin
of your people
Israel,
and bring them back
to
the land
you gave
to them
and their ancestors.
\par }{\PP \VS{26}“The time will come when the skies
are shut up
tightly and no
rain
falls because
your people sinned against
you. When they direct their prayers
toward
this
place,
renew
their allegiance
to you, and turn
away from their sin
because
you punish them,
\VS{27}then
listen
from heaven
and forgive
the sin
of your servants,
your people
Israel.
Certainly
you will then teach
them
the right
way
to
live and send
rain
on
your land
that
you have given
your people
to possess.
\par }{\PP \VS{28}“The time will come
when
the land
suffers from a famine,
a plague,
blight,
and disease,
or a locust
invasion,
or when
their enemy
lays siege
to the cities
of the land,
or when some other type of plague
or epidemic occurs.
\VS{29}When
all
your people
Israel
pray
and ask for help,
as
they acknowledge
their intense pain
and spread out
their hands
toward
this
temple,
\VS{30}then listen
from
your heavenly
dwelling
place,
forgive
their sin, and act
favorably toward each
one based on
your evaluation
of their motives.
(Indeed
you
are the only
one who can correctly evaluate
the motives
of all people.)
\VS{31}Then
they will honor
you by obeying
you throughout
their
lifetimes
as they
live
on
the land
you gave
to our ancestors.
\par }{\PP \VS{32}“Foreigners,
who
do not
belong to your people
Israel,
will come
from a distant
land
because
of your great
reputation
and your ability
to accomplish mighty
deeds; they will come
and direct
their prayers
toward this
temple.
\VS{33}Then listen
from
your heavenly
dwelling
place
and answer
all
the prayers
of the foreigners.
Then all
the nations
of the earth
will acknowledge
your reputation,
obey
you
like your people
Israel
do, and recognize
that
this
temple
I built belongs to you.
\par }{\PP \VS{34}“When
you direct
your people
to march out and fight
their enemies,
and they direct
their prayers to
you toward
this chosen
city
and this temple
I built
for your honor,
\VS{35}then listen
from
heaven
to their prayers
for help
and vindicate them.
\par }{\PP \VS{36}“The time will come when
your people will sin
against you (for
there is no
one
who
is sinless!) and you will be angry
at them and deliver
them over
to their enemies,
who will take
them as prisoners
to
their land,
whether far away
or
close by.
\VS{37}When your people come
to
their senses
in the land
where
they are held prisoner,
they will repent
and beg for your mercy
in the land
of their imprisonment,
admitting,
‘We have sinned
and gone astray
, we have done evil!’
\VS{38}When they return
to
you with all
their heart
and being
in the land
where they are held
prisoner
and direct
their prayers
toward
the land
you gave
to their ancestors,
your chosen
city,
and the temple
I built
for your honor,
\VS{39}then listen
from
your heavenly
dwelling
place
to their prayers
for help,
vindicate
them, and forgive
your sinful
people.
\par }{\PP \VS{40}“Now,
my God,
may
you be attentive
and responsive
to the prayers
offered in this
place.
\VS{41}Now
ascend,
O
{\ND{Lord}}
God,
to your resting place,
you
and the ark
of your strength! May your priests,
O
{\ND{Lord}}
God,
experience
your deliverance! May your loyal followers
rejoice
in the prosperity you give!
\VS{42}O
{\ND{Lord}}
God,
do not
reject
your chosen ones! Remember
the faithful
promises you made to your servant
David!”

\par }\Chap{7}{\PP \VerseOne{1}When Solomon
finished
praying,
fire
came down
from heaven
and consumed
the burnt offering
and the sacrifices,
and the
{\ND{Lord}}’s
splendor
filled
the temple.
\VS{2}The priests
were unable
to enter
the
{\ND{Lord}}’s
temple
because
the
{\ND{Lord}}’s
splendor
filled
the
{\ND{Lord}}’s
temple.
\VS{3}When all
the Israelites
saw
the fire
come down
and the
{\ND{Lord}}’s
splendor
over
the temple,
they got on
their knees
with their faces
downward toward the pavement.
They worshiped
and gave thanks
to the
{\ND{Lord}}, saying, “Certainly
he is good;
certainly
his loyal love
endures!”
\par }{\PP \VS{4}The king
and all
the people
were presenting sacrifices
to the
{\ND{Lord}}.
\VS{5}King
Solomon
sacrificed
22,000
cattle
and 120,000
sheep.
Then the king
and all
the people
dedicated
God’s
temple.
\VS{6}The priests
stood
in their assigned spots,
along with the Levites
who had
the musical instruments used
for praising
the {\ND{Lord}}. (These were
the ones
King
David
made
for
giving thanks to the
{\ND{Lord}} and which were used by David when he offered praise, saying, “Certainly his loyal love endures.”) Opposite the Levites, the priests were blowing the trumpets, while all Israel stood there.
\VS{7}Solomon
consecrated
the
middle
of the courtyard
that
is in front
of the
{\ND{Lord}}’s
temple.
He offered burnt sacrifices,
grain offerings,
and the
fat
from the peace offerings
there,
because
the bronze
altar
that
Solomon
had made
was
too small
to hold
all these offerings.
\VS{8}At that time
Solomon
and all
Israel
with
him celebrated
a festival
for seven
days.
This great
assembly
included
people from Lebo Hamath
in the north to
the Brook
of Egypt in the south.
\VS{9}On
the eighth
day
they held an assembly,
for
they had dedicated
the altar
for seven
days
and celebrated the festival
for seven
more days.
\VS{10}On
the twenty-third
day
of the seventh
month,
Solomon sent
the
people
home.
They left happy
and contented
because of the good
the {\ND{Lord}}
had
done
for David,
Solomon,
and his people
Israel.
\par }{\SH The Lord Gives Solomon a Promise and a Warning
\par }{\PP \VS{11}After Solomon
finished
building the
{\ND{Lord}}’s
temple
and the royal
palace,
and accomplished
all
his
plans
for the
{\ND{Lord}}’s
temple
and his royal palace,
\VS{12}the {\ND{Lord}}
appeared
to
Solomon
at night
and said
to him: “I have answered
your prayer
and chosen
this
place
to be my temple
where sacrifices are to be made.
\VS{13}When I
close
up the sky
so that it
doesn’t
rain,
or command
locusts
to devour
the land’s
vegetation, or
send
a plague
among my people,
\VS{14}if my people,
who
belong to me,
humble
themselves, pray,
seek
to please me,
and repudiate
their sinful practices,
then I
will respond
from
heaven,
forgive
their sin,
and heal
their land.
\VS{15}Now
I will be
attentive
and responsive
to the prayers
offered in this
place.
\VS{16}Now
I have chosen
and consecrated
this
temple
by
making it my permanent
home; I will be
constantly
present
there.
\VS{17}You
must
serve
me as
your father
David
did. Do
everything
I commanded
and obey
my rules
and regulations.
\VS{18}Then I will establish
your dynasty,
just as
I promised
your father
David,
‘You will not
fail
to have a successor
ruling over
Israel.’
\par }{\PP \VS{19}“But if
you
people
ever turn
away from me, fail
to obey the regulations and rules
I instructed you to keep, and decide
to serve
and worship
other
gods,
\VS{20}then I will remove
you from my land
I have
given
you, I will abandon this
temple
I have
consecrated
with my presence,
and I will
make you an object of mockery
and ridicule
among all
the nations.
\VS{21}As for this
temple,
which
was once majestic,
everyone
who passes
by it will be shocked
and say,
‘Why
did
the {\ND{Lord}}
do this
to this land
and this
temple?’
\VS{22}Others will then answer, ‘Because they abandoned
the {\ND{Lord}}
God
of their ancestors,
who
led
them out of Egypt.
They embraced
other
gods
whom they worshiped
and served.
That is why
he brought
all
this
disaster
down on them.’ ”

\par }\Chap{8}{\PP \VerseOne{1}After
twenty
years,
during which
Solomon
built
the
{\ND{Lord}}’s
temple
and his royal palace,
\VS{2}Solomon
rebuilt
the cities
that
Huram
had given
him
and settled
Israelites
there.
\VS{3}Solomon
went
to Hamath Zobah
and seized it.
\VS{4}He built
up Tadmor
in the wilderness
and all
the storage
cities
he had
built
in Hamath.
\VS{5}He made
upper
Beth Horon
and lower
Beth Horon
fortified
cities
with walls
and barred
gates,
\VS{6}and built up Baalath,
all
the storage
cities
that
belonged
to him,
and all
the cities
where chariots
and horses
were kept. He
built
whatever
he wanted
in Jerusalem,
Lebanon,
and throughout
his entire
kingdom.
\par }{\PP \VS{7}Now several
non-Israelite
peoples
were left
in the land after the conquest of Joshua, including
the Hittites,
Amorites,
Perizzites,
Hivites,
and Jebusites.
\VS{8}Their descendants
remained
in the land
(the Israelites
were unable
to wipe
them out). Solomon
conscripted
them for his work crews
and they continue
in that
role to
this
very day.
\VS{9}Solomon
did not
assign
Israelites
to these work
crews;
the Israelites served as his soldiers,
officers,
charioteers,
and commanders
of his chariot
forces.
\VS{10}These
men worked for Solomon
as supervisors;
there were a total of 250
of them who were in charge
of the people.
\par }{\PP \VS{11}Solomon
moved Pharaoh’s
daughter
up
from the City
of David
to the palace
he had
built
for her, for
he said,
“My wife
must not
live
in the palace
of King
David
of Israel,
for
the places where
the ark
of the {\ND{Lord}}
has entered
are
holy.”
\par }{\PP \VS{12}Then
Solomon
offered
burnt sacrifices
to the
{\ND{Lord}}
on
the altar
of the {\ND{Lord}}
which
he had built
in front
of the temple’s porch.
\VS{13}He observed
the daily requirements
for sacrifices
that
Moses
had specified
for Sabbaths,
new moon
festivals,
and the three
annual
celebrations
– the Feast
of Unleavened
Bread, the Feast
of Weeks,
and the Feast
of Temporary Shelters.
\VS{14}As
his father
David
had
decreed,
Solomon appointed
the divisions
of the priests
to do their assigned tasks,
the Levitical
orders
to lead worship and help the priests
with their daily
tasks, and the divisions
of the gatekeepers
to serve at their assigned gates.
This
was what
David
the man
of God
had ordered.
\VS{15}They did not
neglect
any
detail
of the king’s
orders
pertaining
to the priests,
Levites,
and treasuries.
\par }{\PP \VS{16}All
the work
ordered
by Solomon
was completed, from
the day
the foundation
of the
{\ND{Lord}}’s
temple
was laid until
it was finished;
the
{\ND{Lord}}’s
temple
was completed.
\par }{\PP \VS{17}Then
Solomon
went
to Ezion Geber
and to
Elat
on
the coast
in the land
of Edom.
\VS{18}Huram
sent
him ships and some of his sailors,
men
who were well acquainted
with the sea.
They sailed
with
Solomon’s
men
to Ophir,
and took
from there
450
talents
of gold,
which they brought back
to
King
Solomon.

\par }\Chap{9}{\PP \VerseOne{1}When the queen
of Sheba
heard
about
Solomon,
she came
to challenge
him
with difficult questions.
She arrived in Jerusalem
with a great
display of pomp,
bringing with her camels
carrying
spices,
a very large quantity
of gold,
and precious
gems.
She visited
Solomon
and discussed
with
him everything
that
was
on
her mind.
\VS{2}Solomon
answered
all
her questions;
there was no
question
too complex
for the king.
\VS{3}When
the queen
of Sheba
saw
for herself Solomon’s
extensive wisdom,
the palace
he had
built,
\VS{4}the food
in his banquet hall,
his servants
and attendants
in their robes,
his cupbearers
in their robes,
and his burnt sacrifices
which
he presented
in the
{\ND{Lord}}’s
temple,
she was amazed.
\VS{5}She said
to
the king,
“The report
I heard
in my own country
about your wise sayings
and insight
was true!
\VS{6}I did not
believe
these things
until
I came
and saw
them with my own eyes.
Indeed,
I didn’t
hear
even half
the story! Your wisdom
surpasses
what was reported to me.
\VS{7}Your attendants,
who stand
before
you at all times
and hear
your wise sayings,
are truly
happy!
\VS{8}May
the
{\ND{Lord}}
your God
be praised
because
he favored
you by placing
you on
his throne
as the one ruling
on his behalf! Because of your God’s
love
for Israel
and his lasting
commitment
to them, he made
you king
over
them so you could make
just and right decisions.”
\VS{9}She gave
the king
120
talents
of gold
and a very
large quantity
of spices
and precious
gems.
The quantity
of spices
the queen
of Sheba
gave
King
Solomon
has never
been matched.
\VS{10}(Huram’s
servants,
aided by Solomon’s
servants,
brought
gold
from Ophir,
as well as fine
timber
and precious
gems.
\VS{11}With the
timber
the king
made
steps
for the
{\ND{Lord}}’s
temple
and royal
palace
as well as stringed instruments
for the musicians.
No one
had seen
anything like them
in the land
of Judah
prior to that. )
\VS{12}King
Solomon
gave
the queen
of Sheba
everything
she requested,
more than what
she had
brought
him.
Then
she
left and returned to her homeland
with her attendants.
\par }{\SH Solomon’s Wealth
\par }{\PP \VS{13}Solomon
received
666
talents
of gold
per
year,
\VS{14}besides
what he collected
from the merchants
and traders.
All
the Arabian
kings
and the governors
of the land
also brought
gold
and silver
to Solomon.
\VS{15}King
Solomon
made
two hundred
large shields
of hammered
gold;
600
measures of hammered
gold
were used
for each
shield.
\VS{16}He also made three
hundred
small shields
of hammered
gold;
300
measures of gold
were used
for each
of those shields.
The king
placed
them in the Palace
of the Lebanon
Forest.
\par }{\PP \VS{17}The king
made
a large
throne
decorated with ivory
and overlaid
it with pure
gold.
\VS{18}There were six
steps
leading up to the throne,
and a gold
footstool
was attached
to the throne.
The throne had two
armrests with a statue of a lion
standing
on each side.
\VS{19}There
were twelve
statues
of lions
on
the six
steps,
one lion at each end of each step. There was nothing
like
it in any
other kingdom.
\par }{\PP \VS{20}All
of King
Solomon’s
cups
were made of gold,
and all
the household items
in the Palace
of the Lebanon
Forest
were made of pure
gold.
There were no
silver
items, for silver was not
considered
very valuable
in Solomon’s
time.
\VS{21}The king
had a fleet of large merchant ships
manned by
Huram’s
men
that
sailed
the sea. Once
every three
years
the fleet
came
into port
with cargoes
of gold,
silver,
ivory,
apes,
and peacocks.
\par }{\PP \VS{22}King
Solomon
was wealthier
and wiser
than any
of the kings
of the earth.
\VS{23}All
the kings
of the earth
wanted
to visit Solomon
to see him
display
his God-given
wisdom.
\VS{24}Year
after
year
visitors
brought
their gifts,
which included items
of silver,
items
of gold,
clothes,
perfume,
spices,
horses,
and mules.
\par }{\PP \VS{25}Solomon
had 4,000
stalls
for his chariot
horses
and 12,000
horses.
He kept
them in assigned cities
and in
Jerusalem.
\VS{26}He ruled
all
the kingdoms
from
the Euphrates
River to
the land
of the Philistines
as far
as the border
of Egypt.
\VS{27}The king
made
silver
as plentiful in Jerusalem
as stones;
cedar
was
as plentiful
as sycamore fig trees
are in the lowlands .
\VS{28}Solomon
acquired
horses
from Egypt
and from all
the lands.
\par }{\SH Solomon’s Reign Ends
\par }{\PP \VS{29}The rest
of the events
of Solomon’s
reign, from start
to finish,
are
recorded
in
the Annals
of Nathan
the Prophet,
the Prophecy
of Ahijah
the Shilonite,
and the Vision
of Iddo
the Seer
pertaining
to Jeroboam
son
of Nebat.
\VS{30}Solomon
ruled
over
all
Israel
from Jerusalem
for forty
years.
\VS{31}Then Solomon
passed away
and was buried
in the city
of his father
David.
His son
Rehoboam
replaced
him as king.

\par }\Chap{10}{\PP \VerseOne{1}Rehoboam
traveled
to Shechem,
for
all
Israel
had gathered
in Shechem
to make Rehoboam king.
\VS{2}When
Jeroboam
son
of Nebat
heard
the news, he was
still in Egypt,
where
he had fled
from King
Solomon.
Jeroboam
returned
from Egypt.
\VS{3}They sent
for him and Jeroboam
and all
Israel
came
and spoke
to
Rehoboam,
saying,
\VS{4}“Your father
made us work
too hard! Now
if you lighten
the demands he made and don’t make
us work as
hard,
we will serve you.”
\VS{5}He said
to them,
“Go away
for three
days,
then return
to me.”
So the people
went away.
\par }{\PP \VS{6}King
Rehoboam
consulted
with
the older
advisers who had
served
his father
Solomon
when
he had been alive.
He asked
them, “How
do you
advise
me to answer
these
people?”
\VS{7}They said
to
him, “If
you are fair
to these
people,
grant their request, and are cordial
to them,
they will be
your servants
from this time forward.”
\VS{8}But Rehoboam rejected
their advice
and consulted
the young advisers
who served him, with
whom
he had
grown up.
\VS{9}He asked
them, “How
do you
advise
me to respond
to these
people
who
said
to me,
‘Lessen
the demands
your father
placed
on us’?”
\VS{10}The young advisers
with
whom
Rehoboam had grown up
said
to him, “Say
this
to these people
who
have said
to
you, ‘Your father
made us work hard, but now lighten
our
burden’ – say this to them: ‘I am a lot harsher than my father!
\VS{11}My father
imposed
heavy
demands
on
you; I
will make
them even heavier.
My father
punished
you with ordinary whips;
I
will punish you with whips that really sting your flesh.’ ”
\par }{\PP \VS{12}Jeroboam
and all
the people
reported to Rehoboam
on
the third
day,
just as
the king
had ordered when he said,
“Return
to
me on
the third
day.”
\VS{13}The king
responded
to the people harshly.
He
rejected
the advice
of the older men
\VS{14}and followed
the advice
of the younger ones.
He said,
“My father imposed heavy
demands
on
you; I
will make them even heavier.
My father
punished you
with ordinary whips;
I
will punish you
with whips
that really sting your flesh.”
\VS{15}The king
refused
to listen
to
the people,
because
God
was
instigating
this turn of events
so that
he might bring to pass the prophetic
announcement
he had made through
Ahijah
the Shilonite
to
Jeroboam
son
of Nebat.
\par }{\PP \VS{16}When all
Israel
saw that
the king
refused
to listen
to them,
the people
answered
the king,
“We
have no portion
in David
– no
share
in the son
of Jesse! Return to your homes,
O Israel! Now,
look
after your own dynasty,
O David!” So all
Israel
returned
to their homes.
\VS{17}(Rehoboam
continued to rule
over the Israelites
who lived
in the cities
of Judah.)
\VS{18}King
Rehoboam
sent
Hadoram,
the supervisor
of the work crews,
out after them, but the Israelites
stoned
him to death.
King
Rehoboam
managed
to jump into
his chariot
and escape
to Jerusalem.
\VS{19}So Israel
has been in rebellion against
the Davidic
dynasty
to this
very day.

\par }\Chap{11}{\PP \VerseOne{1}When Rehoboam
arrived
in Jerusalem,
he summoned
180,000
skilled
warriors
from Judah
and Benjamin
to attack
Israel
and restore
the kingdom
to Rehoboam.
\VS{2}But the
{\ND{Lord}}
told
Shemaiah
the prophet,
\VS{3}“Say
this to
King
Rehoboam
son
of Solomon
of Judah
and to
all
the Israelites
in Judah
and Benjamin,
\VS{4}‘The

{\ND{Lord}}
says
this: “Do not
attack
and make war
with
your brothers.
Each
of you go
home,
for
I have caused
this
to happen.” ’” They obeyed
the

{\ND{Lord}}
and called off the attack
against Jeroboam.
\par }{\SH Rehoboam’s Reign
\par }{\PP \VS{5}Rehoboam
lived
in Jerusalem;
he built
up these fortified
cities
throughout Judah:
\VS{6}Bethlehem,
Etam,
Tekoa,
\VS{7}Beth Zur,
Soco,
Adullam,
\VS{8}Gath,
Mareshah,
Ziph,
\VS{9}Adoraim,
Lachish,
Azekah,
\VS{10}Zorah,
Aijalon,
and Hebron.
These were the fortified
cities
in Judah
and Benjamin.
\VS{11}He fortified
these cities
and placed
officers
in them,
as
well as storehouses
of food,
olive oil,
and wine.
\VS{12}In each city
there were shields
and spears;
he strongly
fortified them. Judah
and Benjamin belonged to him.
\par }{\PP \VS{13}The priests
and Levites
who
lived throughout
Israel
supported
him, no matter where they resided.
\VS{14}The
Levites
even
left
their pasturelands
and their property
behind
and came
to Judah
and Jerusalem,
for
Jeroboam
and his sons
prohibited
them from serving as the
{\ND{Lord}}’s priests.
\VS{15}Jeroboam appointed
his own priests
to serve at the worship centers
and to lead in the worship of the goat
idols and calf
idols he had
made.
\VS{16}Those
among all
the Israelite
tribes
who were determined
to worship the
{\ND{Lord}}
God
of Israel
followed them
to Jerusalem
to sacrifice
to the
{\ND{Lord}}
God
of their ancestors.
\VS{17}They supported
the kingdom
of Judah
and were loyal
to Rehoboam
son
of Solomon
for
three
years;
they followed
the edicts
of David
and Solomon
for three
years.
\par }{\PP \VS{18}Rehoboam
married
Mahalath
the daughter
of David’s
son
Jerimoth and of Abihail,
the daughter
of Jesse’s
son
Eliab.
\VS{19}She bore
him sons
named Jeush,
Shemariah,
and Zaham.
\VS{20}He later
married
Maacah
the daughter
of Absalom.
She bore
to him Abijah,
Attai,
Ziza,
and Shelomith.
\VS{21}Rehoboam
loved
Maacah
daughter
of Absalom
more than his other
wives
and concubines.
He had
eighteen
wives
and sixty
concubines;
he fathered
twenty-eight
sons
and sixty
daughters.
\par }{\PP \VS{22}Rehoboam
appointed
Abijah
son
of Maacah
as the leader over
his brothers,
for
he intended
to name him his successor.
\VS{23}He wisely
placed
some
of his many
sons
throughout
the regions
of Judah
and Benjamin
in the various
fortified
cities.
He supplied
them with abundant
provisions
and acquired
many
wives for them.

\par }\Chap{12}{\PP \VerseOne{1}After Rehoboam’s
rule
was
established
and solidified,
he and all
Israel
rejected
the law
of the {\ND{Lord}}.
\VS{2}Because
they were unfaithful
to the
{\ND{Lord}},
in King
Rehoboam’s
fifth
year,
King
Shishak
of Egypt
attacked Jerusalem.
\VS{3}He had 1,200
chariots,
60,000
horsemen,
and an innumerable number
of soldiers
who accompanied
him from
Egypt,
including Libyans,
Sukkites,
and Cushites.
\VS{4}He captured
the fortified
cities
of Judah
and marched
against Jerusalem.
\par }{\PP \VS{5}Shemaiah
the prophet
visited
Rehoboam
and the leaders
of Judah
who
were assembled
in
Jerusalem
because
of Shishak.
He said
to them, “This is what
the {\ND{Lord}}
says: ‘You
have rejected
me, so I
have rejected
you and will hand
you over to Shishak.’ ”
\VS{6}The leaders
of Israel
and the king
humbled
themselves and said,
“The
{\ND{Lord}}
is just.”
\VS{7}When the
{\ND{Lord}}
saw
that
they humbled
themselves, he gave this message
to
Shemaiah: “They have humbled
themselves, so I will not
destroy
them. I will deliver
them
soon.
My anger
will
not
be unleashed
against Jerusalem
through
Shishak.
\VS{8}Yet
they will become
his subjects,
so they can experience
how serving
me differs from serving
the surrounding
nations.”
\par }{\PP \VS{9}King
Shishak
of Egypt
attacked
Jerusalem
and took
away the treasures
of the
{\ND{Lord}}’s
temple
and of the royal
palace;
he took
everything,
including the gold
shields
that
Solomon
had made.
\VS{10}King
Rehoboam
made
bronze
shields
to replace
them and assigned
them to the officers
of the royal guard
who protected
the entrance
to the royal
palace.
\VS{11}Whenever
the king
visited
the
{\ND{Lord}}’s
temple,
the royal guards
carried
them and then brought
them back
to
the guardroom.
\par }{\PP \VS{12}So when Rehoboam humbled
himself, the
{\ND{Lord}}
relented
from
his anger
and did not
annihilate
him; Judah
experienced
some good
things.
\VS{13}King
Rehoboam
solidified
his rule
in Jerusalem;
he was forty-one
years
old when he became king
and he ruled
for seventeen
years
in Jerusalem,
the city
the {\ND{Lord}}
chose
from
all
the tribes
of Israel
to be his home.
Rehoboam’s mother
was an Ammonite
named
Naamah.
\VS{14}He did
evil
because
he was not
determined
to follow
the {\ND{Lord}}.
\par }{\PP \VS{15}The events
of Rehoboam’s
reign, from start
to finish,
are
recorded
in the Annals
of Shemaiah
the prophet
and of Iddo
the seer
that include genealogical records.
\VS{16}Then Rehoboam
passed away
and was buried
in the City
of David.
His son
Abijah
replaced
him as king.

\par }\Chap{13}{\PP \VerseOne{1}In the eighteenth
year
of the reign of King
Jeroboam,
Abijah
became king
over
Judah.
\VS{2}He ruled
for three
years
in Jerusalem.
His mother
was Michaiah,
the daughter
of Uriel
from
Gibeah.
\par }{\PP There was
war
between
Abijah
and Jeroboam.
\VS{3}Abijah
launched
the attack
with
400,000
well-trained
warriors,
while Jeroboam
deployed
against him 800,000
well-trained
warriors.
\par }{\PP \VS{4}Abijah
ascended
Mount
Zemaraim,
in
the Ephraimite
hill country,
and said: “Listen
to me, Jeroboam
and all
Israel!
\VS{5}Don’t
you realize
that
the {\ND{Lord}}
God
of Israel
has given
David
and his dynasty lasting
dominion
over
Israel
by a formal
agreement?
\VS{6}Jeroboam
son
of Nebat,
a servant
of Solomon
son
of David,
rose up
and rebelled
against
his master.
\VS{7}Lawless
good-for-nothing
men
gathered around
him and conspired
against
Rehoboam
son
of Solomon,
when Rehoboam
was
an inexperienced
young man
and could not
resist them.
\VS{8}Now
you
are declaring
that you will resist
the
{\ND{Lord}}’s
rule
through
the Davidic
dynasty.
You
have a huge
army, and bring with
you the gold
calves
that
Jeroboam
made
for you as gods.
\VS{9}But
you banished
the
{\ND{Lord}}’s
priests,
Aaron’s
descendants,
and the Levites,
and appointed your own priests
just
as the surrounding
nations
do! Anyone
who comes
to consecrate
himself with a young
bull
or seven
rams
becomes
a priest
of these fake gods!
\VS{10}But as for us,
the {\ND{Lord}}
is our God
and we have not
rejected
him. Aaron’s
descendants
serve
as the
{\ND{Lord}}’s
priests
and the Levites
assist them with the work.
\VS{11}They offer
burnt sacrifices
to the
{\ND{Lord}}
every morning
and every
evening,
along with fragrant
incense.
They arrange the Bread
of the Presence
on
a ritually clean
table
and light the lamps
on the gold
lampstand
every
evening.
Certainly
we
are observing
the {\ND{Lord}}
our God’s
regulations, but you
have rejected him.
\VS{12}Now look,
God
is with
us as our leader.
His priests
are ready to blow
the trumpets
to signal the attack
against
you. You Israelites,
don’t
fight
against
the {\ND{Lord}}
God
of your ancestors,
for
you will not
win!”
\par }{\PP \VS{13}Now Jeroboam
had sent
some men to ambush
the Judahite army from behind.
The main army was
in front
of the Judahite
army; the ambushers
were behind it.
\VS{14}The men of Judah
turned
around and realized
they were being attacked
from the front
and the rear.
So they cried out
for help to the
{\ND{Lord}}. The priests
blew
their trumpets,
\VS{15}and the men
of Judah
gave the battle
cry.
As the men
of Judah
gave
the battle cry,
the
{\ND{Lord}} struck
down Jeroboam
and all
Israel
before
Abijah
and Judah.
\VS{16}The Israelites
fled
from before
the Judahite
army, and God
handed
them over to the men of Judah.
\VS{17}Abijah
and his army
thoroughly
defeated
them; 500,000
well-trained
Israelite
men
fell
dead.
\VS{18}That day
the Israelites
were defeated;
the men
of Judah
prevailed
because
they relied
on
the {\ND{Lord}}
God
of their ancestors.
\par }{\PP \VS{19}Abijah
chased
Jeroboam;
he seized
from
him these cities: Bethel
and its surrounding towns,
Jeshanah
and its surrounding towns,
and Ephron
and its surrounding towns.
\VS{20}Jeroboam
did not
regain
power during
the reign of Abijah.
The
{\ND{Lord}}
struck
him down and he died.
\VS{21}Abijah’s
power
grew;
he had fourteen
wives
and fathered
twenty-two
sons
and sixteen
daughters.
\par }{\PP \VS{22}The rest
of the events of Abijah’s
reign, including his deeds
and sayings,
are recorded
in the writings
of the prophet
Iddo.

\par }\Chap{14}{\PP \VerseOne{1}Abijah
passed away
and was buried
in the City
of David.
His son
Asa
replaced
him as king.
During
his reign the land
had rest
for ten
years.
\par }{\SH Asa’s Religious and Military Accomplishments
\par }{\PP \VS{2}Asa
did
what the
{\ND{Lord}}
his God
desired
and approved.
\VS{3}He removed
the pagan
altars
and the high places,
smashed
the sacred pillars,
and cut down
the Asherah poles.
\VS{4}He ordered
Judah
to seek
the {\ND{Lord}}
God
of their ancestors
and to observe
his law
and commands.
\VS{5}He removed
the high places
and the incense altars
from all
the cities
of Judah.
The kingdom
had rest under his rule.
\par }{\PP \VS{6}He built
fortified
cities
throughout Judah,
for
the land
was at rest
and there was no
war
during
those years;
the {\ND{Lord}}
gave him peace.
\VS{7}He said
to the people of Judah: “Let’s
build
these
cities
and fortify
them with walls,
towers,
and barred
gates. The land
remains ours because we
have followed
the

{\ND{Lord}}
our God
and he has made
us secure
on all sides.”
So they built
the cities and prospered.
\par }{\PP \VS{8}Asa
had an army
of 300,000
men from Judah,
equipped
with large shields
and spears.
He also had 280,000
men from Benjamin
who carried
small shields
and were adept
archers;
they were all
skilled
warriors.
\VS{9}Zerah
the Cushite
marched
against
them with an army
of 1,000,000
men and 300
chariots.
He arrived
at
Mareshah,
\VS{10}and Asa
went out
to oppose
him. They deployed
for battle
in the Valley
of Zephathah
near Mareshah.
\par }{\PP \VS{11}Asa
prayed
to
the {\ND{Lord}}
his God: “O
{\ND{Lord}}, there is no
one but you
who can help
the weak when
they are vastly
outnumbered.
Help
us, O
{\ND{Lord}}
our God,
for
we rely
on you and have marched
on your behalf against
this
huge
army. O
{\ND{Lord}}
our God,
don’t
let men
prevail
against
you!”
\VS{12}The
{\ND{Lord}}
struck
down the Cushites
before
Asa
and Judah.
The Cushites
fled,
\VS{13}and Asa
and his army
chased
them as far
as Gerar.
The Cushites
were wiped out;
they
were shattered
before
the {\ND{Lord}}
and his army.
The men of Judah carried
off a huge amount
of plunder.
\VS{14}They defeated
all
the cities
surrounding
Gerar,
for
the {\ND{Lord}}
caused
them to panic.
The men of Judah looted
all
the cities,
for
they contained a huge amount
of goods.
\VS{15}They also
attacked
the tents
of the herdsmen in charge of the livestock.
They carried
off many
sheep
and camels
and then returned
to Jerusalem.

\par }\Chap{15}{\PP \VerseOne{1}God’s
Spirit
came upon
Azariah
son
of Oded.
\VS{2}He met Asa
and told
him, “Listen
to me, Asa
and all
Judah
and Benjamin! The
{\ND{Lord}}
is with
you when
you
are loyal
to him. If
you
seek
him, he will respond
to you, but if
you reject
him, he will reject you.
\VS{3}For a long
time Israel
had no
true
God,
or
priest
to instruct
them, or
law.
\VS{4}Because of their distress,
they turned back
to the
{\ND{Lord}}
God
of Israel.
They sought
him and he responded to them.
\VS{5}In those
days no
one could travel
safely,
for
total
chaos
had overtaken
all
the people
of the surrounding lands.
\VS{6}One nation
was crushed
by another,
and one city
by another, for
God
caused
them to be in great turmoil.
\VS{7}But as for you,
be strong
and don’t
get discouraged,
for
your work
will be
rewarded.”
\par }{\PP \VS{8}When Asa
heard
these
words
and the prophecy
of Oded
the prophet,
he was encouraged.
He removed
the detestable idols
from the entire
land
of Judah
and Benjamin
and from
the cities
he had
seized
in the Ephraimite
hill country.
He repaired
the altar
of the {\ND{Lord}}
in front of
the porch
of the
{\ND{Lord}}’s temple.
\par }{\PP \VS{9}He
assembled
all
Judah
and Benjamin,
as well as the settlers
from Ephraim,
Manasseh,
and Simeon
who had come to live
with
them. Many people
from Israel
had come there to live when they saw
that
the {\ND{Lord}}
his God
was with him.
\VS{10}They assembled
in Jerusalem
in the third
month
of the fifteenth
year
of Asa’s
reign.
\VS{11}At that time
they sacrificed
to the
{\ND{Lord}}
some
of the plunder
they had brought
back, including 700
head of cattle
and 7,000
sheep.
\VS{12}They solemnly agreed
to seek
the {\ND{Lord}}
God
of their ancestors
with their whole
heart
and being.
\VS{13}Anyone
who
would not
seek
the
{\ND{Lord}}
God
of Israel
would be executed,
whether
they were young
or old,
male
or female.
\VS{14}They swore
their allegiance to the
{\ND{Lord}}, shouting
their approval loudly
and sounding
trumpets
and horns.
\VS{15}All
Judah
was happy
about the oath,
because
they made the vow
with their whole
heart.
They willingly
sought
the {\ND{Lord}}
and he responded
to them. He made them secure
on every side.
\par }{\PP \VS{16}King
Asa
also
removed
Maacah
his grandmother
from her position as queen mother
because she had
made
a loathsome
Asherah pole.
Asa
cut down
her Asherah pole
and crushed
and burned
it in the Kidron
Valley.
\VS{17}The high places
were not
eliminated from
Israel,
yet
Asa
was wholeheartedly
devoted
to the
{\ND{Lord}} throughout
his lifetime.
\VS{18}He brought
the holy
items
that his father
and he had
made into God’s
temple,
including the silver,
gold,
and other articles.
\par }{\SH Asa’s Failures
\par }{\PP \VS{19}There was
no
more war
until
the thirty-fifth
year
of Asa’s
reign.

\par }\Chap{16}{\PP \VerseOne{1}In the thirty-sixth
year
of Asa’s
reign,
King
Baasha
of Israel
attacked Judah,
and he established
Ramah
as a military outpost to prevent
anyone from leaving
or entering
the land of King
Asa
of Judah.
\VS{2}Asa
took all the silver
and gold
that was left in the treasuries
of the
{\ND{Lord}}’s
temple
and of the royal
palace
and sent
it to
King
Ben Hadad
of Syria,
ruler
in Damascus, along with this message:
\VS{3}“I want to make a treaty
with
you, like the one our fathers
made. See,
I have sent
you silver
and gold.
Break
your treaty
with
King
Baasha
of Israel,
so he will retreat
from my land.”
\VS{4}Ben Hadad
accepted
King
Asa’s
offer and ordered
his army
commanders
to
attack
the cities
of Israel.
They conquered
Ijon,
Dan,
Abel Maim,
and all
the storage
cities
of Naphtali.
\VS{5}When
Baasha
heard
the news, he stopped
fortifying
Ramah
and abandoned
the project.
\VS{6}King
Asa
ordered
all
the men of Judah
to carry
away the stones
and wood
that
Baasha
had used to build
Ramah.
He used the materials to build
up Geba
and Mizpah.
\par }{\PP \VS{7}At that time
Hanani
the prophet
visited
King
Asa
of Judah
and said
to him: “Because you relied
on
the king
of Syria
and did not
rely
on
the
{\ND{Lord}}
your God,
the army
of the king
of Syria
has escaped
from your hand.
\VS{8}Did not
the Cushites
and Libyans
have a huge
army
with chariots
and a very
large number
of horsemen? But when
you relied
on
the {\ND{Lord}}, he handed
them over to you!
\VS{9}Certainly
the {\ND{Lord}}
watches
the whole
earth
carefully and is ready to strengthen
those who are devoted
to him.
You have acted foolishly
in
this
matter; from
now
on you will
have war.
\VS{10}Asa
was so angry
at the prophet,
he put
him in jail.
Asa
also
oppressed
some
of the people
at that time.
\par }{\SH Asa’s Reign Ends
\par }{\PP \VS{11}The events
of Asa’s
reign, from start
to finish,
are recorded
in the Scroll
of the Kings
of Judah
and Israel.
\VS{12}In the thirty-ninth
year
of his reign,
Asa
developed a foot
disease.
Though his disease
was severe,
he did not
seek
the {\ND{Lord}}, but only the doctors.
\VS{13}Asa
passed away
in
the forty-first
year
of his reign.
\VS{14}He was buried
in the tomb
he had
carved
out in the City
of David.
They laid
him to rest on a bier
covered with
spices
and assorted
mixtures of ointments.
They made
a huge bonfire to honor him.

\par }\Chap{17}{\PP \VerseOne{1}His son
Jehoshaphat
replaced
him as king and solidified
his rule
over
Israel.
\VS{2}He placed
troops
in all
of Judah’s
fortified
cities
and posted
garrisons
throughout the land
of Judah
and in the cities
of Ephraim
that his father
Asa
had
seized.
\par }{\PP \VS{3}The
{\ND{Lord}}
was with
Jehoshaphat
because
he followed
in his ancestor
David’s
footsteps
at the beginning
of his reign. He did not
seek
the Baals,
\VS{4}but instead
sought
the God
of his ancestors
and obeyed his commands,
unlike
the Israelites.
\VS{5}The
{\ND{Lord}}
made his kingdom
secure;
all
Judah
brought tribute
to Jehoshaphat,
and he became
very wealthy
and greatly
respected.
\VS{6}He was committed
to following
the {\ND{Lord}}; he even
removed
the high places
and Asherah poles
from Judah.
\par }{\PP \VS{7}In the third
year
of his reign
he sent
his officials
Ben-Hail,
Obadiah,
Zechariah,
Nethanel,
and Micaiah
to teach
in the cities
of Judah.
\VS{8}They were accompanied by
the Levites
Shemaiah,
Nethaniah,
Zebadiah,
Asahel,
Shemiramoth,
Jehonathan,
Adonijah,
Tobijah,
and Tob-Adonijah,
and by
the priests
Elishama
and Jehoram.
\VS{9}They taught
throughout Judah,
taking with
them the scroll
of the law
of the {\ND{Lord}}. They traveled
to all
the cities
of Judah
and taught
the people.
\par }{\PP \VS{10}The
{\ND{Lord}}
put fear
into all
the kingdoms
surrounding
Judah;
they did not
make war
with
Jehoshaphat.
\VS{11}Some
of the Philistines
brought
Jehoshaphat
tribute,
including a load
of silver.
The Arabs
brought
him 7,700
rams
and 7,700
goats
from their flocks.
\par }{\PP \VS{12}Jehoshaphat’s
power kept increasing.
He built
fortresses
and storage
cities
throughout Judah.
\VS{13}He had many
supplies
stored in the cities
of Judah
and an army of skilled
warriors
stationed in Jerusalem.
\VS{14}These
were their divisions
by families:
\par }{\PP There were a thousand
officers
from Judah.
Adnah
the commander
led 300,000
skilled
warriors,
\VS{15}Jehochanan
the commander
led 280,000,
\VS{16}and Amasiah
son
of Zikri,
who volunteered
to serve the
{\ND{Lord}}, led 200,000
skilled warriors.
\par }{\PP \VS{17}From
Benjamin,
Eliada,
a skilled
warrior,
led 200,000
men who were equipped
with
bows
and shields,
\VS{18}and Jehozabad
led 180,000
trained warriors.
\par }{\PP \VS{19}These
were the ones who served
the king,
besides
those whom
the king
placed
in the fortified
cities
throughout
Judah.

\par }\Chap{18}{\PP \VerseOne{1}Jehoshaphat
was very wealthy
and greatly
respected.
He made an alliance by marriage
with Ahab,
\VS{2}and after several
years
went down
to visit
Ahab
in Samaria.
Ahab
slaughtered
many
sheep
and cattle
to honor Jehoshaphat and those who
came with him.
He persuaded
him to
join
in an attack
against Ramoth
Gilead.
\VS{3}King
Ahab
of Israel
said
to
Jehoshaphat,
“Will you go
with
me to attack Ramoth
Gilead?” Jehoshaphat replied
to the king
of Israel, “I will support you; my
army
is at your disposal
and will support you
in battle.”
\VS{4}Then Jehoshaphat
added, “First seek
an oracle
from the
{\ND{Lord}}.”
\VS{5}So the king
of Israel
assembled
400
prophets
and asked
them, “Should we attack
Ramoth
Gilead
or
not?” They said,
“Attack! God
will hand
it over to the king.”
\VS{6}But Jehoshaphat
asked,
“Is there not
a prophet
of the {\ND{Lord}}
still
here,
that we may ask him?”
\VS{7}The king
of Israel
answered
Jehoshaphat,
“There is still
one
man
through whom we can seek
the
{\ND{Lord}}’s
will. But I
despise
him because
he does not
prophesy
prosperity
for
me, but always
disaster.
His name is
Micaiah
son
of Imlah.
Jehoshaphat
said,
“The king
should not
say
such things!”
\VS{8}The king
of Israel
summoned
an officer
and said,
“Quickly
bring Micaiah
son
of Imlah.”
\par }{\PP \VS{9}Now the king
of Israel
and King
Jehoshaphat
of Judah
were sitting
on
their respective
thrones,
dressed
in
their royal
robes,
at the threshing floor
at the entrance
of the gate
of Samaria.
All
the prophets
were prophesying
before them.
\VS{10}Zedekiah
son
of Kenaanah
made iron
horns
and said,
“This is what
the {\ND{Lord}}
says,
‘With these
you will gore
Syria
until
they are destroyed!’ ”
\VS{11}All
the prophets
were prophesying
the same,
saying,
“Attack
Ramoth
Gilead! You will succeed;
the {\ND{Lord}}
will hand
it over to the king!”
\VS{12}Now the messenger
who
went
to summon
Micaiah
said
to him,
“Look,
the prophets
are in complete agreement that
the king
will
succeed. Your words
must agree with theirs;
you must predict success!”
\VS{13}But Micaiah
said,
“As certainly as the
{\ND{Lord}}
lives,
I will say
what
my God
tells me to say!”
\par }{\PP \VS{14}Micaiah came
before
the king
and the king
asked
him,
“Micaiah,
should
we attack Ramoth
Gilead
or
not?” He answered
him, “Attack! You will succeed;
they will be handed over to you.”
\VS{15}The king
said
to him,
“How
many times
must I
make you solemnly
promise in the name
of the {\ND{Lord}}
to tell
me
only
the truth?”
\VS{16}Micaiah replied,
“I saw
all
Israel
scattered
on
the mountains
like sheep
that
have no
shepherd.
Then the
{\ND{Lord}}
said,
‘They have no
master.
They
should go
home
in peace.’ ”
\VS{17}The king
of Israel
said
to
Jehoshaphat,
“Didn’t
I tell
you he does not
prophesy
prosperity
for
me, but
disaster?”
\VS{18}Micaiah said,
“That being
the case, hear
the word
of the {\ND{Lord}}: I saw
the {\ND{Lord}}
sitting
on
his throne,
with all
the heavenly
assembly
standing
on
his right
and on his left.
\VS{19}The
{\ND{Lord}}
said,
‘Who
will deceive
King
Ahab
of Israel,
so
he will attack
Ramoth
Gilead
and die
there?’ One
said
this
and another that.
\VS{20}Then
a spirit
stepped
forward and stood
before
the {\ND{Lord}}. He said,
‘I
will deceive
him.’ The
{\ND{Lord}}
asked
him, ‘How?’
\VS{21}He replied,
‘I will go out
and be
a lying
spirit
in the mouths
of all
his prophets.’
The
{\ND{Lord}} said,
‘Deceive
and overpower him. Go out
and do
as you have proposed.’
\VS{22}So now,
look,
the {\ND{Lord}}
has placed
a lying
spirit
in the mouths
of all these
prophets
of yours;
but the
{\ND{Lord}}
has decreed
disaster for you.”
\VS{23}Zedekiah
son
of Kenaanah
approached,
hit
Micaiah
on
the jaw,
and said,
“Which
way
did the
{\ND{Lord}}’s
spirit
go when
he went from me to speak
to you?”
\VS{24}Micaiah
replied,
“Look,
you will see
in the day
when
you go
into an inner room
to hide.”
\VS{25}Then the king
of Israel
said,
“Take
Micaiah
and return
him to
Amon
the city
official
and Joash
the king’s
son.
\VS{26}Say,
‘This is what
the king
says: “Put
this
man in prison.
Give
him only a little
bread
and water
until
I return
safely.” ’ ”
\VS{27}Micaiah
said,
“If
you really
do return
safely,
then
the {\ND{Lord}}
has not
spoken
through me!” Then he added, “Take note,
all
you people.”
\par }{\PP \VS{28}The king
of Israel
and King
Jehoshaphat
of Judah
attacked
Ramoth
Gilead.
\VS{29}The king
of Israel
said
to
Jehoshaphat,
“I will disguise
myself and then enter
the battle;
but you
wear
your royal attire.”
So the king
of Israel
disguised
himself and they entered
the battle.
\VS{30}Now the king
of Syria
had ordered
his chariot
commanders,
“Do not
fight
common soldiers or
high ranking
officers; fight only
the king
of Israel!”
\VS{31}When
the chariot
commanders
saw
Jehoshaphat,
they
said,
“He must
be the king
of Israel!” So they turned
and attacked
him, but Jehoshaphat
cried
out. The
{\ND{Lord}}
helped
him; God
lured
them away from him.
\VS{32}When
the chariot
commanders
realized
he was not
the king
of Israel,
they turned
away
from him.
\VS{33}Now an archer
shot
an arrow
at random
and it struck
the king
of Israel
between
the plates
of his armor.
The king ordered
his charioteer,
“Turn
around
and take
me from
the battle line,
for
I am wounded.”
\VS{34}While the battle
raged
throughout the day,
the king
stood
propped up in his chariot
opposite
the Syrians.
He died
in the evening
as
the sun
was setting.

\par }\Chap{19}{\PP \VerseOne{1}When King
Jehoshaphat
of Judah
returned
home
safely
to Jerusalem,
\VS{2}the prophet Jehu
son
of Hanani
confronted
him; he said
to
King
Jehoshaphat,
“Is it right to help
the wicked
and be an ally
of those who oppose
the {\ND{Lord}}? Because you have done this
the {\ND{Lord}}
is angry with you!
\VS{3}Nevertheless
you have done some good
things; you removed
the Asherah poles
from
the land
and you were determined
to follow
the
{\ND{Lord}}.”
\par }{\SH Jehoshaphat Appoints Judges
\par }{\PP \VS{4}Jehoshaphat
lived
in Jerusalem.
He went
out
among the people
from Beer Sheba
to the hill country
of Ephraim
and encouraged
them to
follow the
{\ND{Lord}}
God
of their ancestors.
\VS{5}He appointed
judges
throughout
the land
and in each of the fortified
cities
of Judah.
\VS{6}He told
the judges,
“Be careful
what
you
do,
for
you are not
judging
for men,
but for
the {\ND{Lord}}, who will be with
you when you make judicial
decisions.
\VS{7}Respect
the {\ND{Lord}}
and make
careful
decisions, for
the {\ND{Lord}}
our God
disapproves
of injustice,
partiality,
and bribery.”
\par }{\PP \VS{8}In Jerusalem
Jehoshaphat
appointed
some
Levites,
priests,
and Israelite
family
leaders
to judge
on behalf of the
{\ND{Lord}}
and to settle disputes
among the residents of Jerusalem.
\VS{9}He commanded
them: “Carry
out your duties with respect
for the
{\ND{Lord}}, with honesty,
and with pure motives.
\VS{10}Whenever
your countrymen
who live
in the cities
bring
a case
before you (whether
it involves a violent
crime
or other matters related to the law,
commandments,
rules,
and regulations), warn
them that they must not
sin against the
{\ND{Lord}}. If you fail to do so, God will be
angry
with
you and your colleagues;
but if you obey,
you will be free
of guilt.
\VS{11}You will report
to Amariah
the chief
priest
in all
matters
pertaining to the
{\ND{Lord}}’s
law, and to Zebadiah
son
of Ishmael,
the
leader
of the
family
of Judah,
in all
matters
pertaining to the
king.
The
Levites
will serve as officials
before
you.
Confidently
carry
out your duties! May the
{\ND{Lord}}
be with
those who do
well!”

\par }\Chap{20}{\PP \VerseOne{1}Later
the Moabites
and Ammonites,
along with
some of the Meunites,
attacked
Jehoshaphat.
\VS{2}Messengers arrived
and reported
to Jehoshaphat,
“A huge
army is attacking
you from the other side
of the Dead Sea,
from the direction of Edom.
Look,
they are in Hazezon Tamar
(that
is, En Gedi).”
\VS{3}Jehoshaphat
was afraid,
so
he decided to seek
the
{\ND{Lord}}’s
advice. He decreed that all
Judah
should observe
a fast.
\VS{4}The people of Judah
assembled
to ask
for the
{\ND{Lord}}’s
help; they came
from all
the cities
of Judah
to ask
for the
{\ND{Lord}}’s help.
\par }{\PP \VS{5}Jehoshaphat
stood
before the assembly
of Judah
and Jerusalem
at the
{\ND{Lord}}’s
temple,
in front
of the new
courtyard.
\VS{6}He prayed: “O
{\ND{Lord}}
God
of our ancestors,
you
are the God
who lives in heaven
and rules
over all
the kingdoms
of the nations.
You possess strength
and power;
no
one can stand
against you.
\VS{7}Our God,
you
drove
out the inhabitants
of this
land
before
your people
Israel
and gave
it as a permanent
possession to the descendants
of your friend
Abraham.
\VS{8}They settled
down in it and built
in it a temple
to honor
you, saying,
\VS{9}‘If
disaster
comes
on
us in the form of military attack,
judgment,
plague,
or famine,
we will stand
in front
of this
temple
before
you, for
you are present
in this
temple.
We will cry out
to
you for help in our distress,
so that you will hear
and deliver us.’
\VS{10}Now
the Ammonites,
Moabites,
and men from Mount
Seir
are coming! When Israel
came
from the land
of Egypt,
you did not
allow
them to invade
these lands. They bypassed
them and did not
destroy them.
\VS{11}Look
how they
are repaying
us! They come
to drive
us out
of our allotted land which
you assigned to us!
\VS{12}Our God,
will you not
judge
them? For
we are powerless
against
this
huge
army
that attacks
us! We
don’t
know
what
we should do;
we look
to
you for help.”
\par }{\PP \VS{13}All
the men of Judah
were standing
before
the {\ND{Lord}}, along
with their infants,
wives,
and children.
\VS{14}Then in the midst
of the assembly,
the
{\ND{Lord}}’s
Spirit
came upon Jachaziel
son
of Zechariah,
son
of Benaiah,
son
of Jeiel,
son
of Mattaniah,
a Levite
and descendant
of Asaph.
\VS{15}He said: “Pay attention,
all
you people of Judah,
residents
of Jerusalem,
and King
Jehoshaphat! This is what
the {\ND{Lord}}
says to you: ‘Don’t
be afraid
and don’t
panic
because
of this
huge
army! For
the battle
is not
yours, but
God’s.
\VS{16}Tomorrow
march down
against
them as they come
up
the Ascent
of Ziz.
You will find
them
at the end
of the ravine
in front
of the Desert
of Jeruel.
\VS{17}You will not
fight
in this
battle. Take your positions,
stand,
and watch
the

{\ND{Lord}}
deliver
you, O Judah
and Jerusalem.
Don’t
be afraid
and don’t
panic! Tomorrow
march out
toward
them; the
{\ND{Lord}}
is with you!’ ”
\par }{\PP \VS{18}Jehoshaphat
bowed
down with his face
toward the ground,
and all
the people of Judah
and the residents
of Jerusalem
fell
down before
the {\ND{Lord}}
and worshiped him.
\VS{19}Then
some
Levites,
from
the Kohathites
and Korahites,
got up and loudly
praised
the {\ND{Lord}}
God
of Israel.
\par }{\PP \VS{20}Early
the next morning
they marched
out to the Desert
of Tekoa.
When they were ready to march,
Jehoshaphat
stood up
and said: “Listen
to me, you people of Judah
and residents
of Jerusalem! Trust
in the
{\ND{Lord}}
your God
and you will be safe! Trust
in the message of his prophets
and you will win.”
\VS{21}He met with
the people
and appointed
musicians
to play before
the {\ND{Lord}}
and praise
his majestic
splendor. As they marched
ahead
of the warriors
they said: “Give thanks
to the
{\ND{Lord}}, for
his loyal love
endures.”
\par }{\PP \VS{22}When
they began
to shout
and praise,
the {\ND{Lord}}
suddenly attacked
the Ammonites,
Moabites,
and men from Mount
Seir
who were invading
Judah,
and they were defeated.
\VS{23}The Ammonites
and Moabites
attacked
the men
from Mount
Seir
and annihilated
them. When they had finished off
the men
of Seir,
they attacked
and destroyed
one
another.
\VS{24}When the men of Judah
arrived
at
the observation post
overlooking
the desert
and looked
at
the huge
army, they saw dead bodies
on the ground;
there were no
survivors!
\VS{25}Jehoshaphat
and his men
went
to gather
the
plunder;
they found
a huge amount
of supplies,
clothing
and valuable
items.
They carried away
everything they could.
There was so
much plunder,
it took them
three
days
to haul
it off.
\par }{\PP \VS{26}On the fourth
day
they assembled
in the Valley
of Berachah,
where
they praised
the {\ND{Lord}}. So that
place
is called
the Valley
of Berachah
to
this very day.
\VS{27}Then all
the men
of Judah
and Jerusalem
returned
joyfully
to
Jerusalem
with Jehoshaphat
leading
them; the
{\ND{Lord}}
had given them reason to rejoice
over their enemies.
\VS{28}They entered
Jerusalem
to the sound of stringed instruments
and trumpets
and proceeded to
the temple
of the {\ND{Lord}}.
\VS{29}All
the kingdoms
of the surrounding lands
were afraid
of God
when they heard
how the
{\ND{Lord}}
had
fought
against
Israel’s
enemies.
\VS{30}Jehoshaphat’s
kingdom
enjoyed
peace;
his God
made him secure on every side.
\par }{\SH Jehoshaphat’s Reign Ends
\par }{\PP \VS{31}Jehoshaphat
reigned
over
Judah.
He was thirty-five
years
old
when he became king
and he reigned
for twenty-five
years
in Jerusalem.
His mother
was Azubah,
the daughter
of Shilhi.
\VS{32}He followed
in his father
Asa’s
footsteps
and was careful
to do what
the {\ND{Lord}}
approved.
\VS{33}However,
the high places
were not
eliminated;
the people
were still
not
devoted
to the God
of their ancestors.
\par }{\PP \VS{34}The rest
of the events
of Jehoshaphat’s
reign, from start
to finish,
are recorded
in the Annals
of Jehu
son
of Hanani
which
are included
in Scroll
of the Kings
of Israel.
\par }{\PP \VS{35}Later
King
Jehoshaphat
of Judah
made an alliance
with
King
Ahaziah
of Israel,
who did
evil.
\VS{36}They agreed
to make
large seagoing merchant ships;
they built
the ships
in Ezion Geber.
\VS{37}Eliezer
son
of Dodavahu
from Mareshah
prophesied
against
Jehoshaphat,
“Because you made an alliance
with
Ahaziah,
the {\ND{Lord}}
will shatter
what
you have made.” The ships
were wrecked
and unable
to go
to
sea.

\par }\Chap{21}{\PP \VerseOne{1}Jehoshaphat
passed away
and was buried
with
his ancestors
in the City
of David.
His son
Jehoram
replaced
him as king.
\par }{\SH Jehoram’s Reign
\par }{\PP \VS{2}His brothers,
Jehoshaphat’s
sons,
were Azariah,
Jechiel,
Zechariah,
Azariahu,
Michael,
and Shephatiah.
All
of these
were sons
of King
Jehoshaphat
of Israel.
\VS{3}Their
father
gave
them
many
presents,
including silver,
gold,
and other precious
items, along with
fortified
cities
in Judah.
But he gave
the kingdom
to Jehoram
because
he was
the firstborn.
\par }{\PP \VS{4}Jehoram
took control
of his father’s
kingdom
and became powerful.
Then he killed
all
his brothers,
as well as
some of the officials
of Israel.
\VS{5}Jehoram
was thirty-two
years
old when he became king and he reigned
for eight
years
in Jerusalem.
\VS{6}He followed
in the footsteps
of the kings
of Israel,
just
as Ahab’s
dynasty
had done,
for
he married
Ahab’s
daughter.
He did
evil
in the sight
of the
{\ND{Lord}}.
\VS{7}But the
{\ND{Lord}}
was unwilling
to destroy
David’s
dynasty
because
of the promise
he had
made
to give
David
a perpetual
dynasty.
\par }{\PP \VS{8}During
Jehoram’s reign Edom
freed
themselves from
Judah’s
control
and set up
their own king.
\VS{9}Jehoram
crossed
over to Zair with
his officers
and all
his chariots.
The Edomites,
who had surrounded
him,
attacked
at night
and defeated
him and his chariot
officers.
\VS{10}So Edom
has remained free
from
Judah’s
control
to
this very day.
At that same time
Libnah
also rebelled
and freed themselves from
Judah’s control
because
Jehoram rejected
the {\ND{Lord}}
God
of his ancestors.
\VS{11}He
also
built
high places
on the hills
of Judah;
he encouraged the
residents
of Jerusalem
to be unfaithful
to the
{\ND{Lord}} and led Judah away from the
{\ND{Lord}}.
\par }{\PP \VS{12}Jehoram received
this letter
from
Elijah
the prophet: “This is what
the {\ND{Lord}}
God
of your ancestor
David
says: ‘You have
not
followed
in the footsteps
of your father
Jehoshaphat
and of King
Asa
of Judah,
\VS{13}but have instead followed
in the footsteps
of the kings
of Israel.
You encouraged the
people of Judah
and the
residents
of Jerusalem
to be unfaithful
to the
{\ND{Lord}}, just as the family
of Ahab
does in Israel. You also
killed
your brothers,
members
of your father’s
family,
who were better
than you.
\VS{14}So look,
the {\ND{Lord}}
is about to severely
afflict
your people,
your sons,
your wives,
and all
you own.
\VS{15}And you
will get a serious,
chronic
intestinal
disease
which will cause
your intestines
to come out.”
\par }{\PP \VS{16}The
{\ND{Lord}}
stirred
up against
Jehoram
the Philistines
and the Arabs
who lived beside
the Cushites.
\VS{17}They attacked
Judah
and swept through
it. They carried
off everything
they found
in the royal
palace,
including
his sons
and wives.
None
of his sons
was left,
except
for
his youngest,
Ahaziah.
\VS{18}After
all
this
happened, the
{\ND{Lord}}
afflicted
him with an incurable
intestinal
disease.
\VS{19}After
about two
years his intestines
came out
because of the disease,
so that he died
a very painful death.
His people
did not
make
a bonfire
to honor him, as they had done for his ancestors.
\par }{\PP \VS{20}Jehoram was thirty-two
years old when he became
king and he reigned
eight
years
in Jerusalem.
No
one regretted
his death; he was buried
in the City
of David,
but not
in the royal
tombs.

\par }\Chap{22}{\PP \VerseOne{1}The residents
of Jerusalem
made his youngest
son
Ahaziah
king
in his place,
for
the raiding party
that invaded
the city with the Arabs
had killed
all
the older sons.
So Ahaziah
son
of Jehoram
became king
of Judah.
\VS{2}Ahaziah
was twenty-two
years
old when he became king
and he reigned
for one
year
in Jerusalem.
His mother
was Athaliah,
the granddaughter
of Omri.
\VS{3}He followed
in the footsteps
of Ahab’s
dynasty,
for
his mother
gave him evil
advice.
\VS{4}He did
evil
in the sight
of the
{\ND{Lord}}
like Ahab’s
dynasty
because,
after
his father’s
death,
they gave him advice
that led to his destruction.
\VS{5}He followed
their advice
and joined
Ahab’s
son
King
Joram
of Israel
in a battle
against
King
Hazael
of Syria
at Ramoth
Gilead
in which the Syrians
defeated
Joram.
\VS{6}Joram
returned
to Jezreel
to recover
from the wounds
he received
from the Syrians in Ramah
when he fought
against King
Hazael
of Syria.
Ahaziah
son
of King
Jehoram
of Judah
went down
to visit
Joram
son
of Ahab
in Jezreel,
because
he had been wounded.
\par }{\PP \VS{7}God
brought
about Ahaziah’s downfall
through his visit
to Joram.
When Ahaziah
arrived,
he went out
with
Joram
to
meet Jehu
son
of Nimshi,
whom
the {\ND{Lord}}
had commissioned
to wipe out
Ahab’s
family.
\VS{8}While
Jehu
was dishing out punishment
to
Ahab’s
family,
he discovered
the officials
of Judah
and the sons
of Ahaziah’s
relatives
who were serving
Ahaziah
and killed them.
\VS{9}He looked
for Ahaziah,
who was captured
while hiding
in Samaria.
They brought
him to
Jehu
and then executed
him. They did give him a burial,
for
they reasoned, “He is the son
of Jehoshaphat,
who
sought
the

{\ND{Lord}}
with his whole
heart.”
There was no
one in Ahaziah’s
family
strong
enough to rule
in his place.
\par }{\SH Athaliah is Eliminated
\par }{\PP \VS{10}When Athaliah
the mother
of Ahaziah
saw
that
her son
was dead,
she was determined
to destroy the entire
royal
line
of Judah.
\VS{11}So
Jehoshabeath,
the
daughter
of King
Jehoram, took Ahaziah’s
son
Joash
and sneaked
him away
from
the rest of the royal
descendants
who were to be executed.
She hid him and his nurse
in the room
where the bed
covers were stored. So Jehoshabeath
the daughter
of King
Jehoram,
wife
of Jehoiada
the priest
and sister
of Ahaziah,
hid him from
Athaliah
so she could not
execute him.
\VS{12}He remained
in hiding
in God’s
temple
for six
years,
while Athaliah
was ruling
over
the land.

\par }\Chap{23}{\PP \VerseOne{1}In the seventh
year
Jehoiada
made
a bold
move. He made a pact
with
the officers
of the units of hundreds: Azariah
son
of Jehoram,
Ishmael
son
of Jehochanan,
Azariah
son
of Obed,
Maaseiah
son
of Adaiah,
and Elishaphat
son
of Zikri.
\VS{2}They traveled throughout
Judah
and assembled
the Levites
from all
the cities
of Judah,
as well as the Israelite
family
leaders.
\par }{\PP They came
to
Jerusalem,
\VS{3}and the whole
assembly
made
a covenant
with
the king
in the temple
of God.
Jehoiada said
to them,
“The king’s
son
will rule,
just
as the
{\ND{Lord}}
promised
David’s
descendants.
\VS{4}This
is what
you must do.
One third
of you priests
and Levites
who are on duty
during the Sabbath
will guard
the doors.
\VS{5}Another third
of you will be stationed at the royal
palace
and still another third
at the Foundation
Gate.
All
the others
will stand in the courtyards
of the
{\ND{Lord}}’s
temple.
\VS{6}No
one must enter
the
{\ND{Lord}}’s
temple
except
the priests
and Levites
who are
on duty.
They
may enter
because
they are
ceremonially
pure. All
the others
should
carry out their assigned service
to the
{\ND{Lord}}.
\VS{7}The Levites
must surround
the king.
Each
of you must hold his weapon
in his hand.
Whoever tries to
enter
the temple
must be killed.
You must accompany
the king
wherever he goes.”
\par }{\PP \VS{8}The Levites
and all
the men of Judah
did
just
as Jehoiada
the priest
ordered.
Each
of them took
his men,
those who were on duty
during the Sabbath
as well as those who were off duty
on the Sabbath.
Jehoiada
the priest
did not
release
his divisions from their duties.
\VS{9}Jehoiada
the priest
gave
to the officers
of the units of hundreds
King
David’s
spears
and shields
that were kept in God’s
temple.
\VS{10}He placed
the men
at their posts, each
holding his weapon
in his hand.
They lined up
from the south
side
of the temple
to the north
side
and stood near the altar
and the temple,
surrounding
the king.
\VS{11}Jehoiada
and his sons
led out
the
king’s
son
and placed
on
him the crown
and the royal insignia.
They proclaimed him king
and poured
olive oil on his head. They declared,
“Long live
the king!”
\par }{\PP \VS{12}When Athaliah
heard
the royal guard
shouting
and praising
the king,
she joined
the crowd
at
the
{\ND{Lord}}’s
temple.
\VS{13}Then she saw
the king
standing
by
his pillar
at the entrance.
The officers
and trumpeters
stood beside
the king
and all
the people
of the land
were celebrating
and blowing
trumpets,
and the musicians
with various
instruments
were leading
the celebration.
Athaliah
tore
her clothes
and yelled,
“Treason! Treason!”
\VS{14}Jehoiada
the priest
sent out the officers
of the units of hundreds,
who
were in charge of the army,
and ordered
them,
“Bring
her outside
the temple
to
the guards.
Put
the sword
to anyone who follows her.” The priest
gave this order because
he had decided she should not
be executed
in the
{\ND{Lord}}’s
temple.
\VS{15}They seized
her and took
her into
the precincts
of the royal
palace
through the horses’
entrance.
There
they executed her.
\par }{\PP \VS{16}Jehoiada
then drew up
a covenant
stipulating that he, all
the people,
and the king
should be
loyal to the
{\ND{Lord}}.
\VS{17}All
the people
went
and demolished
the temple
of Baal.
They smashed
its altars
and idols.
They killed
Mattan
the priest
of Baal
in front
of the altars.
\VS{18}Jehoiada
then assigned
the duties
of the
{\ND{Lord}}’s
temple
to the priests,
the Levites
whom
David
had assigned
to the
{\ND{Lord}}’s
temple.
They were responsible for offering
burnt sacrifices
to the
{\ND{Lord}}
with joy
and music,
according to the law
of Moses
and the edict
of David.
\VS{19}He posted
guards
at the gates
of the
{\ND{Lord}}’s
temple,
so no
one who was ceremonially unclean
in any
way could
enter.
\VS{20}He summoned
the officers
of the units of hundreds,
the nobles,
the rulers
of the people,
and all
the people
of land,
and he then led the king
down
from the
{\ND{Lord}}’s
temple.
They entered
the royal
palace
through
the Upper
Gate
and seated
the king
on
the royal
throne.
\VS{21}All
the people
of the land
celebrated,
for the city
had rest
now that they had killed
Athaliah.

\par }\Chap{24}{\PP \VerseOne{1}Joash
was seven
years
old when he began to reign. He reigned
for forty
years
in Jerusalem.
His mother
was Zibiah,
who was from Beer Sheba.
\VS{2}Joash
did what the
{\ND{Lord}}
approved
throughout
the lifetime
of Jehoiada
the priest.
\VS{3}Jehoiada
chose
two
wives
for him who gave
him sons
and daughters.
\par }{\PP \VS{4}Joash
was determined
to repair
the
{\ND{Lord}}’s
temple.
\VS{5}He assembled
the priests
and Levites
and ordered
them, “Go out
to the cities
of Judah
and collect
the annual
quota of silver
from all
Israel
for repairs
on the temple
of your God.
Be quick
about it!” But the Levites
delayed.
\par }{\PP \VS{6}So
the king
summoned
Jehoiada
the chief
priest, and said
to him, “Why
have you not
made
the Levites
collect from Judah
and Jerusalem
the
tax
authorized by Moses
the
{\ND{Lord}}’s
servant
and by the assembly
of Israel
at the tent containing the tablets of the law?”
\VS{7}(Wicked
Athaliah
and her sons
had broken into
God’s
temple
and used all
the holy
items of the
{\ND{Lord}}’s
temple
in their worship
of the Baals.)
\VS{8}The king
ordered
a chest
to be made and placed
outside
the gate
of the
{\ND{Lord}}’s
temple.
\VS{9}An
edict
was sent throughout Judah
and Jerusalem
requiring
the people to bring
to the
{\ND{Lord}}
the tax
that Moses,
God’s
servant,
imposed on
Israel
in the wilderness.
\VS{10}All
the officials
and all
the people
gladly
brought
their silver and threw
it into the chest
until
it was full.
\VS{11}Whenever the Levites
brought
the chest
to
the royal
accountant
and they saw
there was a lot of
silver,
the royal
scribe
and the accountant
of the high
priest
emptied
the chest
and then took
it back
to
its place.
They went through
this routine
every day
and collected
a large amount
of silver.
\par }{\PP \VS{12}The king
and Jehoiada
gave it
to
the construction
foremen
assigned
to the
{\ND{Lord}}’s
temple.
They hired
carpenters
and craftsmen
to repair
the
{\ND{Lord}}’s
temple,
as well
as those skilled in working
with iron
and bronze
to restore
the
{\ND{Lord}}’s
temple.
\VS{13}They worked
hard and made
the repairs.
They followed
the measurements specified
for God’s
temple
and restored it.
\VS{14}When they were finished,
they brought
the rest
of the silver
to the king
and Jehoiada.
They used it to make
items
for the
{\ND{Lord}}’s
temple,
including items
used in the temple service
and for burnt sacrifices, pans,
and various other
gold
and silver
items. Throughout
Jehoiada’s
lifetime,
burnt sacrifices
were offered regularly
in the
{\ND{Lord}}’s
temple.
\par }{\PP \VS{15}Jehoiada
grew
old
and died
at
the age
of 130.
\VS{16}He was buried
in the City
of David
with
the kings,
because
he had accomplished
good
in Israel
and for God
and his temple.
\par }{\PP \VS{17}After
Jehoiada
died,
the officials
of Judah
visited
the king
and declared
their loyalty to him.
The king
listened to their advice.
\VS{18}They abandoned
the temple
of the {\ND{Lord}}
God
of their ancestors,
and worshiped
the Asherah poles
and idols.
Because of this
sinful activity,
God was angry
with Judah
and Jerusalem.
\VS{19}The
{\ND{Lord}}
sent
prophets
among them to lead them back
to
him. They warned
the people, but they would not
pay attention.
\VS{20}God’s
Spirit
energized
Zechariah
son
of Jehoiada
the priest.
He stood
up before the people
and said
to them, “This is what
God
says: ‘Why
are you
violating
the commands
of the {\ND{Lord}}? You will not
be prosperous! Because
you have rejected
the

{\ND{Lord}}, he has rejected
you!’ ”
\VS{21}They plotted
against
him and by royal
decree
stoned
him to death in the courtyard
of the
{\ND{Lord}}’s
temple.
\VS{22}King
Joash
disregarded
the loyalty
his father
Jehoiada
had
shown
him
and killed
Jehoiada’s son.
As Zechariah was dying,
he said,
“May
the {\ND{Lord}}
take notice
and seek vengeance!”
\par }{\PP \VS{23}At the beginning
of the year
the Syrian
army
attacked
Joash and invaded
Judah
and Jerusalem.
They wiped out
all
the leaders
of the people
and sent
all
the plunder
they gathered to the king
of Damascus.
\VS{24}Even though
the invading
Syrian
army
was relatively weak,
the {\ND{Lord}}
handed over
to them Judah’s very
large
army,
for
the people of Judah had abandoned
the {\ND{Lord}}
God
of their ancestors.
The Syrians gave
Joash
what he deserved.
\VS{25}When
they withdrew,
they left
Joash badly wounded.
His servants
plotted against
him because of what he had done to the son
of Jehoiada
the priest.
They murdered
him on
his bed.
Thus he died
and was buried
in the City
of David,
but not
in the tombs
of the kings.
\VS{26}The conspirators
were
Zabad
son
of Shimeath
(an Ammonite
woman) and Jehozabad
son
of Shimrith
(a Moabite woman).
\par }{\PP \VS{27}The list of Joash’s sons,
the many
prophetic oracles
pertaining to him, and the account of his building project
on
God’s
temple
are included
in the record of
the Scroll
of the Kings.
His son
Amaziah
replaced him as king.

\par }\Chap{25}{\PP \VerseOne{1}Amaziah
was twenty-five
years
old when he began to reign, and he reigned
for twenty-nine
years
in Jerusalem.
His mother
was Jehoaddan,
who was from Jerusalem.
\VS{2}He did
what the
{\ND{Lord}}
approved,
but
not
with wholehearted devotion.
\par }{\PP \VS{3}When
he had
secured
control of the kingdom,
he executed
the servants
who had assassinated
his father.
\VS{4}However, he did not execute
their sons.
He obeyed the
{\ND{Lord}}’s
commandment
as recorded
in the law
scroll
of Moses, “Fathers
must not be executed
for
what
their sons
do,
and sons
must not
be executed
for
what their fathers
do. A man
must be executed
only for
his own sin.”
\par }{\PP \VS{5}Amaziah
assembled
the people of Judah
and assigned
them by families
to the commanders
of units of a thousand
and the commanders
of units of a hundred
for all
Judah
and Benjamin.
He counted
those twenty
years
old and up
and discovered
there were 300,000
young men
of fighting age
equipped
with spears
and shields.
\VS{6}He hired
100,000
Israelite
warriors
for a hundred
talents
of silver.
\par }{\PP \VS{7}But a prophet
visited
him
and said: “O king,
the Israelite
troops
must not
go
with
you, for
the
{\ND{Lord}}
is not
with
Israel
or any
of the Ephraimites.
\VS{8}Even
if
you
go
and fight bravely
in battle,
God
will defeat
you before
the enemy.
God
is
capable
of helping
or defeating.”
\VS{9}Amaziah
asked
the prophet: “But what
should I do
about the hundred
talents
of silver I paid
the Israelite
troops?” The prophet
replied,
“The
{\ND{Lord}}
is capable
of giving
you more than that.”
\VS{10}So Amaziah
dismissed
the troops
that had
come
to him
from Ephraim
and sent
them home.
They were very
angry
at Judah
and returned
home
incensed.
\VS{11}Amaziah
boldly
led
his army
to the Valley
of Salt,
where he defeated
10,000
Edomites.
\VS{12}The men of Judah
captured
10,000
men
alive.
They took
them to the top
of a cliff
and threw
them over.
All
the captives fell to
their death.
\VS{13}Now the troops
Amaziah
had dismissed
and had
not allowed
to fight in
the battle
raided
the cities
of Judah
from Samaria
to
Beth Horon.
They
killed
3,000
people and carried off
a large amount
of plunder.
\par }{\PP \VS{14}When Amaziah
returned
from defeating
the Edomites,
he brought
back the gods
of the people
of Seir
and made them his personal gods.
He bowed down
before
them and offered
them sacrifices.
\VS{15}The
{\ND{Lord}}
was angry
at Amaziah
and sent
a prophet
to him,
who said,
“Why
are you following
these
gods
that
could
not
deliver
their own people
from your power?”
\VS{16}While
he was speaking,
Amaziah
said
to him,
“Did we
appoint you
to be a royal
counselor? Stop
prophesying or else
you will be killed!” So the prophet
stopped,
but added, “I know
that
the
{\ND{Lord}} has
decided
to destroy
you, because
you have done
this
thing and refused
to listen
to my advice.”
\par }{\PP \VS{17}After King
Amaziah
of Judah
consulted
with his advisers, he sent
this message to
the king
of Israel,
Joash
son
of Jehoahaz,
the son
of Jehu,
“Come, face
me on the battlefield.”
\VS{18}King
Joash
of Israel
sent
this message
back to
King
Amaziah
of Judah,
“A thorn
bush in Lebanon
sent
this message to
a cedar
in Lebanon,
‘Give
your daughter
to my son
as a wife.’
Then
a wild
animal
of Lebanon
came by and trampled
down the
thorn bush.
\VS{19}You defeated
Edom
and it
has gone to your head. Gloat
over your success,
but stay
in your palace.
Why
bring calamity
on
yourself? Why bring down
yourself
and Judah
along with you?”
\par }{\PP \VS{20}But Amaziah
did not
heed
the warning, for
God
wanted
to hand
them over to Joash because
they followed
the gods
of Edom.
\VS{21}So King
Joash
of Israel
attacked.
He and King
Amaziah
of Judah
faced
each other on the battlefield
in Beth Shemesh
of Judah.
\VS{22}Judah
was defeated
by Israel,
and each man
ran
back home.
\VS{23}King
Joash
of Israel
captured
King
Amaziah
of Judah,
son
of Joash
son
of Jehoahaz,
in Beth Shemesh
and brought
him to Jerusalem.
He broke down
the wall
of Jerusalem
from the Gate
of Ephraim
to
the Corner Gate
– a distance
of about six hundred feet.
\VS{24}He took away all
the gold
and silver,
all
the items
found
in God’s
temple
that were in the care of Obed-Edom,
the riches
in the
royal
palace,
and some hostages.
Then he went back
to Samaria.
\par }{\PP \VS{25}King
Amaziah
son
of Joash
of Judah
lived
for fifteen
years
after
the death
of King
Joash
son
of Jehoahaz
of Israel.
\VS{26}The rest
of the events
of Amaziah’s
reign, from start
to finish,
are recorded
in the Scroll
of the Kings
of Judah
and Israel.
\VS{27}From the time
Amaziah
turned
from following
the {\ND{Lord}}, conspirators
plotted against
him in
Jerusalem,
so he fled
to Lachish.
But they sent
assassins after
him
and they killed
him there.
\VS{28}His body was carried
back by horses,
and he was buried
in Jerusalem with
his ancestors
in the City of David.

\par }\Chap{26}{\PP \VerseOne{1}All
the people
of Judah
took
Uzziah,
who
was sixteen
years
old,
and made
him king
in
his father
Amaziah’s
place.
\VS{2}Uzziah
built up
Elat
and restored
it to Judah
after
King
Amaziah had passed away.
\par }{\PP \VS{3}Uzziah
was sixteen
years
old when he began to reign, and he reigned
for fifty-two
years
in Jerusalem.
His mother’s
name
was Jecholiah,
who was from
Jerusalem.
\VS{4}He did
what the
{\ND{Lord}}
approved,
just
as his father
Amaziah
had
done.
\VS{5}He followed
God
during
the lifetime
of Zechariah,
who taught him
how to honor God.
As long
as he followed
the {\ND{Lord}}, God
caused him to succeed.
\par }{\PP \VS{6}Uzziah attacked
the Philistines
and broke down
the walls
of Gath,
Jabneh,
and Ashdod.
He built
cities
in the region of Ashdod
and throughout Philistine territory.
\VS{7}God
helped
him in his campaigns against
the Philistines,
the Arabs
living
in Gur Baal,
and the Meunites.
\VS{8}The Ammonites
paid
tribute
to Uzziah
and his fame
reached
the border
of Egypt,
for
he grew
in power.
\par }{\PP \VS{9}Uzziah
built
and fortified towers
in Jerusalem
at the Corner
Gate,
Valley
Gate, and at the Angle.
\VS{10}He built
towers
in the desert
and dug
many
cisterns,
for
he owned many
herds
in the lowlands
and on the plain.
He had workers in the fields
and vineyards
in the hills
and in Carmel,
for
he loved
agriculture.
\par }{\PP \VS{11}Uzziah
had an army of skilled warriors
trained
for battle.
They were organized
by divisions
according to the muster
rolls made by Jeiel
the scribe
and Maaseiah
the officer
under
the authority
of Hananiah,
a royal
official.
\VS{12}The total
number
of family
leaders
who led warriors
was 2,600.
\VS{13}They commanded
an army
of 307,500
skilled and able warriors
who were ready
to defend
the king
against
his enemies.
\VS{14}Uzziah
supplied
shields,
spears,
helmets,
breastplates,
bows,
and slingstones
for the entire
army.
\VS{15}In Jerusalem
he made
war machines
carefully designed
to shoot
arrows
and large
stones
from the towers
and corners
of the walls. He became very
famous,
for
he received tremendous
support
and became powerful.
\par }{\PP \VS{16}But once he became powerful,
his pride
destroyed
him. He disobeyed
the {\ND{Lord}}
his God.
He entered
the
{\ND{Lord}}’s
temple
to offer incense
on
the incense
altar.
\VS{17}Azariah
the priest
and eighty
other brave
priests
of the {\ND{Lord}}
followed him in.
\VS{18}They confronted
King
Uzziah
and said
to him, “It is not proper
for you, Uzziah,
to offer incense
to the
{\ND{Lord}}. That
is the responsibility of the priests,
the descendants
of Aaron,
who are consecrated
to offer incense.
Leave
the sanctuary,
for
you have disobeyed
and the
{\ND{Lord}}
God
will not
honor you!”
\VS{19}Uzziah,
who had an incense
censer
in his hand,
became angry.
While he was ranting and raving
at the priests,
a skin disease
appeared
on his forehead
right there in front
of the priests
in the
{\ND{Lord}}’s
temple
near
the incense
altar.
\VS{20}When
Azariah
the high
priest
and the other
priests
looked at
him, there was a skin disease
on his forehead.
They hurried
him out
of there;
even
the king himself
wanted to leave
quickly
because
the {\ND{Lord}}
had afflicted him.
\VS{21}King
Uzziah
suffered from a skin disease
until
the day
he died.
He lived
in separate
quarters,
afflicted by a skin disease
and banned from
the
{\ND{Lord}}’s
temple.
His son
Jotham
was in charge
of the palace
and ruled
over the
people
of the land.
\par }{\PP \VS{22}The rest
of the events
of Uzziah’s
reign, from start
to finish,
were recorded
by the prophet
Isaiah
son
of Amoz.
\VS{23}Uzziah
passed away
and was buried
near his ancestors
in a cemetery
belonging to the kings.
(This was because
he had a skin disease.) His son
Jotham
replaced
him as king.

\par }\Chap{27}{\PP \VerseOne{1}Jotham
was twenty-five
years
old
when he began to reign,
and he reigned
for sixteen
years
in Jerusalem.
His mother
was Jerusha
the daughter
of Zadok.
\VS{2}He did
what the
{\ND{Lord}}
approved,
just as
his father
Uzziah
had done.
(He did not,
however,
have the audacity
to enter
the temple.) Yet the people
were still
sinning.
\par }{\PP \VS{3}He
built
the
Upper
Gate
to the
{\ND{Lord}}’s
temple
and did a lot of work on the wall
in the area known as Ophel.
\VS{4}He built
cities
in the hill country
of Judah
and fortresses
and towers
in the forests.
\par }{\PP \VS{5}He
launched a military campaign
against
the king
of the Ammonites
and defeated
them. That
year
the Ammonites
paid
him 100
talents
of silver,
10,000
kors
of wheat,
and 10,000
kors of barley.
The Ammonites
also paid
this
same amount of annual tribute the next two
years.
\par }{\PP \VS{6}Jotham
grew powerful
because
he was determined
to please the
{\ND{Lord}}
his God.
\VS{7}The rest
of the events
of Jotham’s
reign, including all
his military
campaigns
and his accomplishments, are recorded
in the scroll
of the kings
of Israel
and Judah.
\VS{8}He was twenty-five
years
old
when
he began to reign,
and he reigned
for sixteen
years
in Jerusalem.
\VS{9}Jotham
passed away
and was buried
in the City
of David.
His son
Ahaz
replaced
him as king.

\par }\Chap{28}{\PP \VerseOne{1}Ahaz
was twenty
years
old when he began to reign, and he reigned
for sixteen
years
in Jerusalem.
He did not
do
what pleased
the {\ND{Lord}}, in contrast to his ancestor
David.
\VS{2}He followed
in the footsteps
of the kings
of Israel;
he also
made
images
of the Baals.
\VS{3}He
offered
sacrifices in the Valley
of Ben-Hinnom
and passed his sons
through the fire,
a horrible
sin practiced by the nations
whom
the {\ND{Lord}}
drove
out before
the Israelites.
\VS{4}He offered sacrifices
and burned incense
on the high places,
on
the hills,
and under
every
green
tree.
\par }{\PP \VS{5}The
{\ND{Lord}}
his God
handed
him over to the king
of Syria.
The Syrians defeated
him and deported
many captives
to Damascus.
He was also
handed
over to the king
of Israel,
who thoroughly defeated him.
\VS{6}In one
day
King Pekah
son
of Remaliah
of Israel killed
120,000
warriors
in Judah,
because they had abandoned
the {\ND{Lord}}
God
of their ancestors.
\VS{7}Zikri,
an Ephraimite
warrior,
killed
the king’s
son
Maaseiah,
Azrikam,
the supervisor
of the palace,
and Elkanah,
the king’s
second-in-command.
\VS{8}The Israelites
seized
from their brothers
200,000
wives,
sons,
and daughters.
They
also
carried off
a huge amount
of plunder
and took
it
back
to Samaria.
\par }{\PP \VS{9}Oded,
a prophet
of the {\ND{Lord}}, was there.
He went
to meet
the army
as they arrived
in Samaria
and said
to them: “Look,
because the
{\ND{Lord}}
God
of your ancestors
was angry
with
Judah
he handed
them over
to you. You have killed
them so
mercilessly
that
God
has taken notice.
\VS{10}And now
you
are planning
to enslave
the people
of Judah
and Jerusalem.
Yet are you not
also
guilty
before the
{\ND{Lord}}
your God?
\VS{11}Now
listen
to me! Send back
those
you have
seized
from your brothers,
for
the
{\ND{Lord}}
is very
angry
at you!”
\VS{12}So some
of the Ephraimite
family leaders,
Azariah
son
of Jehochanan,
Berechiah
son
of Meshillemoth,
Jechizkiah
son
of Shallum,
and Amasa
son
of Hadlai
confronted
those returning
from
the battle.
\VS{13}They said
to them, “Don’t
bring
those
captives
here! Are you
planning on making
us even more
sinful
and guilty
before
the {\ND{Lord}}? Our guilt
is already great
and the
{\ND{Lord}} is very
angry
at
Israel.”
\VS{14}So
the soldiers released
the captives
and the plunder
before
the officials
and the entire
assembly.
\VS{15}Men
were assigned
to take
the prisoners
and find
clothes
among the plunder
for those who were naked.
So they clothed
them, supplied them with sandals,
gave them food
and drink,
and provided them with oil
to rub on their skin. They put
the ones
who couldn’t walk
on donkeys.
They brought
them back to their brothers
at Jericho,
the city
of the date
palm trees, and then returned
to Samaria.
\par }{\PP \VS{16}At that time
King
Ahaz
asked
the king
of Assyria
for help.
\VS{17}The Edomites
had again
invaded
and defeated
Judah
and carried
off captives.
\VS{18}The Philistines
had raided
the cities
of Judah
in the lowlands
and the Negev.
They captured
and settled
in Beth Shemesh,
Aijalon,
Gederoth,
Soco
and its surrounding villages,
Timnah
and its surrounding villages,
and Gimzo
and its surrounding villages.
\VS{19}The
{\ND{Lord}}
humiliated
Judah
because
of King
Ahaz
of Israel,
for
he encouraged
Judah
to sin and was very unfaithful
to the
{\ND{Lord}}.
\VS{20}King
Tiglath-pileser
of Assyria
came,
but he gave him more trouble than support.
\VS{21}Ahaz
gathered
riches
from the
{\ND{Lord}}’s
temple,
the royal
palace,
and the officials
and gave
them to the king
of Assyria,
but that did not
help.
\par }{\PP \VS{22}During his time
of trouble
King
Ahaz
was even more
unfaithful
to the
{\ND{Lord}}.
\VS{23}He offered sacrifices
to the gods
of Damascus
whom he thought had defeated
him. He reasoned, “Since
the gods
of the kings
of Damascus
helped
them, I will sacrifice
to them so they will help
me.” But they
caused
him and all
Israel
to stumble.
\VS{24}Ahaz
gathered
the items
in God’s
temple
and removed
them. He shut
the doors
of the
{\ND{Lord}}’s
temple
and erected
altars
on every
street corner
in Jerusalem.
\VS{25}In every
city
throughout Judah
he set up high places
to offer
sacrifices to other
gods.
He angered
the {\ND{Lord}}
God
of his ancestors.
\par }{\PP \VS{26}The rest
of the events
of Ahaz’s
reign, including his accomplishments from start
to finish,
are recorded
in the Scroll
of the Kings
of Judah
and Israel.
\VS{27}Ahaz
passed away
and was buried
in the City of David;
they did not
bring
him to the tombs
of the kings
of Israel.
His son
Hezekiah
replaced
him as king.

\par }\Chap{29}{\PP \VerseOne{1}Hezekiah
was twenty-five
years
old
when he began to reign,
and he reigned
twenty-nine
years
in Jerusalem.
His mother
was Abijah,
the daughter
of Zechariah.
\VS{2}He did
what the
{\ND{Lord}}
approved,
just
as his ancestor
David
had
done.
\par }{\PP \VS{3}In the first
month
of the first
year
of his reign,
he opened
the doors
of the
{\ND{Lord}}’s
temple
and repaired them.
\VS{4}He brought
in the priests
and Levites
and assembled
them in the square
on the east side.
\VS{5}He said
to them: “Listen
to me, you Levites! Now
consecrate
yourselves, so you can consecrate
the
temple
of the {\ND{Lord}}
God
of your ancestors! Remove
from
the sanctuary
what is ceremonially unclean!
\VS{6}For
our fathers
were unfaithful;
they did
what is evil
in the sight
of the
{\ND{Lord}}
our God
and abandoned
him! They turned away
from the
{\ND{Lord}}’s
dwelling place
and rejected him.
\VS{7}They closed
the doors
of the temple porch
and put out
the
lamps;
they did not
offer
incense
or
burnt sacrifices
in the sanctuary
of the God
of Israel.
\VS{8}The
{\ND{Lord}}
was angry
at Judah
and Jerusalem
and made
them an appalling object
of horror
at which people hiss out
their scorn,
as you
can see
with your own eyes.
\VS{9}Look,
our fathers
died
violently
and our
sons,
daughters,
and wives
were carried off
because
of this.
\VS{10}Now
I intend
to make
a covenant
with the
{\ND{Lord}}
God
of Israel,
so
that he may relent
from
his raging
anger.
\VS{11}My sons,
do not
be negligent
now,
for
the
{\ND{Lord}}
has chosen
you to serve
in his
presence
and offer sacrifices.”
\par }{\PP \VS{12}The following
Levites
prepared to carry out the king’s orders:
\par }{\PP From
the Kohathites: Mahath
son
of Amasai
and Joel
son
of Azariah;
\par }{\PP from
the Merarites: Kish
son
of Abdi
and Azariah
son
of Jehallelel;
\par }{\PP from
the Gershonites: Joah
son
of Zimmah
and Eden
son
of Joah;
\par }{\PP \VS{13}from
the descendants
of Elizaphan: Shimri
and Jeiel;
\par }{\PP from
the descendants
of Asaph: Zechariah
and Mattaniah;
\par }{\PP \VS{14}from
the descendants
of Heman: Jehiel
and Shimei;
\par }{\PP from
the descendants
of Jeduthun: Shemaiah
and Uzziel.
\par }{\PP \VS{15}They assembled
their brothers
and consecrated
themselves. Then they went
in to purify
the
{\ND{Lord}}’s
temple,
just as the king
had ordered,
in accordance with the word
of the {\ND{Lord}}.
\VS{16}The priests
then entered
the
{\ND{Lord}}’s
temple
to purify
it; they brought out
to the
courtyard
of the
{\ND{Lord}}’s
temple
every
ceremonially unclean thing
they discovered
inside.
The Levites
took
them out
to the Kidron
Valley.
\VS{17}On the first
day of the first
month
they
began
consecrating;
by the eighth
day
of the month
they reached
the porch
of the
{\ND{Lord}}’s
temple.
For eight
more days
they consecrated
the
{\ND{Lord}}’s
temple.
On the sixteenth
day
of the first
month
they were
finished.
\VS{18}They went
to
King
Hezekiah
and said: “We have purified
the
entire
temple
of the {\ND{Lord}}, including the
altar
of burnt sacrifice
and all
its equipment,
and the
table
for the Bread of the Presence
and all
its equipment.
\VS{19}We
have prepared
and consecrated
all
the items
that
King
Ahaz
removed during his reign
when he acted unfaithfully.
They are in front
of the altar
of the {\ND{Lord}}.”
\par }{\PP \VS{20}Early
the next morning King
Hezekiah
assembled
the city
officials
and went up
to the
{\ND{Lord}}’s
temple.
\VS{21}They brought
seven
bulls,
seven
rams,
seven
lambs,
and seven
goats
as a sin offering
for
the kingdom,
the sanctuary,
and Judah.
The king told
the priests,
the descendants
of Aaron,
to offer
burnt sacrifices on
the altar
of the {\ND{Lord}}.
\VS{22}They slaughtered
the bulls,
and the priests
took
the
blood
and splashed
it on the altar.
Then they slaughtered
the rams
and splashed
the blood
on the altar;
next they slaughtered
the lambs
and splashed
the blood
on the altar.
\VS{23}Finally
they brought the goats
for the sin offering
before
the king
and the assembly,
and they placed
their hands
on them.
\VS{24}Then the priests
slaughtered
them. They offered
their blood
as a sin
offering on
the altar
to make atonement
for
all
Israel,
because
the king
had decreed that the burnt sacrifice
and sin offering
were for
all
Israel.
\par }{\PP \VS{25}King Hezekiah stationed
the Levites
in the
{\ND{Lord}}’s
temple
with cymbals
and stringed instruments,
just as David,
Gad
the king’s
prophet,
and Nathan
the prophet
had ordered.
(The
{\ND{Lord}}
had actually given these orders
through
his prophets.)
\VS{26}The Levites
had David’s
musical
instruments
and the priests
had trumpets.
\VS{27}Hezekiah
ordered
the burnt sacrifice
to be offered
on the altar.
As they began
to offer the sacrifice,
they also began
to sing
to the
{\ND{Lord}},
accompanied by the trumpets
and the musical
instruments of King
David
of Israel.
\VS{28}The entire
assembly
worshiped,
as the singers
sang
and the trumpeters
played.
They continued until
the burnt sacrifice
was completed.
\par }{\PP \VS{29}When
the sacrifices
were completed,
the king
and all
who were
with
him bowed down and worshiped.
\VS{30}King
Hezekiah
and the officials
told
the Levites
to praise
the {\ND{Lord}}, using the psalms
of David
and Asaph
the prophet.
So they joyfully
offered praise
and bowed
down and worshiped.
\VS{31}Hezekiah
said,
“Now
you have consecrated
yourselves to the
{\ND{Lord}}. Come
and bring
sacrifices
and thank offerings
to the
{\ND{Lord}}’s
temple.”
So the assembly
brought
sacrifices
and thank offerings,
and whoever
desired to do so brought burnt sacrifices.
\par }{\PP \VS{32}The assembly
brought
a total
of 70
bulls,
100
rams,
and 200
lambs
as burnt sacrifices
to the
{\ND{Lord}},
\VS{33}and 600
bulls
and 3,000
sheep
were consecrated.
\VS{34}But
there were
not
enough
priests
to skin all
the animals, so their brothers,
the Levites,
helped
them until
the work
was finished
and the priests
could consecrate
themselves. (The Levites
had been more conscientious
about
consecrating themselves than
the priests.)
\VS{35}There was a large number
of burnt sacrifices,
as well as
fat
from the peace offerings
and drink offerings
that accompanied the burnt sacrifices.
So
the service
of the
{\ND{Lord}}’s
temple
was reinstituted.
\VS{36}Hezekiah
and all
the people
were happy
about what
God
had
done
for them, for
it had been done quickly.

\par }\Chap{30}{\PP \VerseOne{1}Hezekiah
sent
messages throughout
Israel
and Judah;
he even
wrote
letters
to Ephraim
and Manasseh,
summoning them to come
to the
{\ND{Lord}}’s
temple
in Jerusalem
and observe
a Passover celebration
for the
{\ND{Lord}}
God
of Israel.
\VS{2}The king,
his
officials,
and the entire
assembly
in Jerusalem
decided
to observe
the Passover
in the second
month.
\VS{3}They were unable
to observe
it at the regular time
because
not
enough
priests
had consecrated
themselves and the people
had not
assembled
in Jerusalem.
\VS{4}The proposal
seemed
appropriate
to the king
and the entire
assembly.
\VS{5}So they sent an edict
throughout
Israel
from Beer Sheba
to
Dan,
summoning the people to come
and observe
a Passover
for the
{\ND{Lord}}
God
of Israel
in Jerusalem,
for
they had not
observed it
on a nationwide
scale as
prescribed in the law.
\VS{6}Messengers
delivered the letters
from the king
and his officials
throughout
Israel
and Judah.
\par }{\PP This royal
edict
read: “O Israelites,
return
to
the {\ND{Lord}}
God
of Abraham,
Isaac,
and Israel,
so he may return
to
you who
have been spared
from the kings
of Assyria.
\VS{7}Don’t
be
like your fathers
and brothers
who
were unfaithful
to the
{\ND{Lord}}
God
of their ancestors,
provoking him to destroy
them, as
you
can see.
\VS{8}Now,
don’t
be stubborn
like
your fathers! Submit
to the
{\ND{Lord}}
and come
to his sanctuary
which
he has permanently
consecrated.
Serve
the

{\ND{Lord}}
your God
so
that he might relent
from
his raging
anger.
\VS{9}For
if you return
to the
{\ND{Lord}}, your brothers
and sons
will be shown mercy
by their captors
and return
to this
land.
The
{\ND{Lord}}
your God
is merciful
and compassionate;
he will not
reject
you if
you return
to him.”
\par }{\PP \VS{10}The messengers journeyed
from city
to city
through
the land
of Ephraim
and Manasseh
as far
as Zebulun,
but people mocked and ridiculed them.
\VS{11}But
some men
from Asher,
Manasseh,
and Zebulun
humbled
themselves and came
to Jerusalem.
\VS{12}In Judah
God
moved the people
to unite
and carry
out the edict
the king
and the officers
had issued at the
{\ND{Lord}}’s
command.
\VS{13}A huge
crowd
assembled
in Jerusalem
to observe
the Feast
of Unleavened Bread
in the second
month.
\VS{14}They removed
the altars
in Jerusalem;
they also
removed
all
the incense
altars
and threw
them into the Kidron
Valley.
\par }{\PP \VS{15}They slaughtered
the Passover lamb
on the fourteenth
day of the second
month.
The priests
and Levites
were ashamed,
so they consecrated
themselves and brought
burnt sacrifices
to the
{\ND{Lord}}’s
temple.
\VS{16}They stood
at their posts according to the regulations
outlined
in the law
of Moses,
the man
of God.
The priests
were splashing
the
blood
as the Levites
handed it to them.
\VS{17}Because
many
in the assembly
had not
consecrated
themselves, the Levites
slaughtered
the Passover lambs
of all
who were ceremonially unclean
and could not consecrate
their sacrifice to the
{\ND{Lord}}.
\VS{18}The majority
of the many
people
from Ephraim,
Manasseh,
Issachar,
and Zebulun
were ceremonially
unclean, yet
they ate
the Passover
in violation of what is prescribed
in the law. For
Hezekiah
prayed
for
them, saying: “May the
{\ND{Lord}}, who is good,
forgive
\VS{19}everyone
who has determined
to follow
God,
the {\ND{Lord}}
God
of his ancestors,
even if he is not
ceremonially clean according to the standards
of the temple.”
\VS{20}The
{\ND{Lord}}
responded favorably
to
Hezekiah
and forgave
the people.
\par }{\PP \VS{21}The Israelites
who were
in Jerusalem
observed the Feast
of Unleavened
Bread for seven
days
with great
joy.
The Levites
and priests
were praising
the {\ND{Lord}}
every
day
with all
their might.
\VS{22}Hezekiah
expressed
his appreciation
to
all
the Levites,
who demonstrated
great
skill in serving
the {\ND{Lord}}. They feasted
for the seven
days
of the festival,
and were making peace offerings
and giving thanks
to the
{\ND{Lord}}
God
of their ancestors.
\par }{\PP \VS{23}The entire
assembly
then decided
to celebrate
for seven
more days;
so
they joyfully
celebrated
for seven
more days.
\VS{24}King
Hezekiah
of Judah
supplied
1,000
bulls
and 7,000
sheep
for the assembly,
while the officials
supplied
them with 1,000
bulls
and 10,000
sheep.
Many
priests
consecrated themselves.
\VS{25}The celebration
included the entire
assembly
of Judah,
the priests,
the Levites,
the entire
assembly
of those who came
from Israel,
the resident foreigners
who came
from the land
of Israel,
and the residents
of Judah.
\VS{26}There was
a great
celebration
in Jerusalem,
unlike
anything that
had occurred
in Jerusalem
since
the time
of King
Solomon
son
of David
of Israel.
\VS{27}The priests
and Levites
got
up and pronounced blessings
on the
people.
The
{\ND{Lord}} responded
favorably
to them as their prayers
reached
his holy
dwelling place
in heaven.

\par }\Chap{31}{\PP \VerseOne{1}When
all
this
was over, the Israelites
who were
in the cities
of Judah
went out
and smashed
the sacred pillars,
cut down
the Asherah poles,
and demolished
all
the
high places
and altars
throughout
Judah,
Benjamin,
Ephraim,
and Manasseh.
Then all
the Israelites
returned
to their own
homes in their cities.
\par }{\SH The People Contribute to the Temple
\par }{\PP \VS{2}Hezekiah
appointed
the divisions
of the priests
and Levites
to do their assigned
tasks –
to offer burnt sacrifices
and present offerings
and to serve,
give thanks,
and offer praise
in the gates
of the
{\ND{Lord}}’s
sanctuary.
\par }{\PP \VS{3}The king
contributed some
of
what he owned
for burnt sacrifices,
including the morning
and evening
burnt sacrifices
and the burnt sacrifices
made on Sabbaths,
new moon
festivals,
and at other appointed times prescribed
in the law
of the {\ND{Lord}}.
\VS{4}He ordered
the people
living
in Jerusalem
to contribute
the portion prescribed
for the priests
and Levites
so they might
be obedient
to the law
of the {\ND{Lord}}.
\VS{5}When the edict
was issued,
the Israelites
freely contributed
the initial portion
of their grain,
wine,
olive oil,
honey,
and all
the produce
of their fields.
They brought a tenth
of everything,
which added up
to a huge amount.
\VS{6}The Israelites
and people of Judah
who lived
in the cities
of Judah
also
contributed
a tenth
of their cattle
and sheep,
as well as a tenth
of the holy
items consecrated
to the
{\ND{Lord}}
their God.
They brought
them and placed
them in many heaps.
\VS{7}In the third
month
they began
piling
their contributions in heaps
and finished
in the seventh
month.
\VS{8}When Hezekiah
and the officials
came
and saw
the heaps,
they praised
the {\ND{Lord}}
and pronounced blessings on his people
Israel.
\par }{\PP \VS{9}When Hezekiah
asked
the priests
and Levites
about the heaps,
\VS{10}Azariah,
the head
priest
from the family
of Zadok,
said
to him, “Since the contributions
began
arriving
in the
{\ND{Lord}}’s
temple,
we have had plenty
to eat
and have a large quantity
left over.
For
the {\ND{Lord}}
has blessed
his people,
and this
large amount
remains.”
\VS{11}Hezekiah
ordered
that storerooms
be prepared
in the
{\ND{Lord}}’s
temple.
When
this was done,
\VS{12}they brought
in the contributions,
tithes,
and consecrated items
that had been offered. Konaniah,
a Levite,
was in charge of all this, assisted by
his brother
Shimei.
\VS{13}Jehiel,
Azaziah,
Nahath,
Asahel,
Jerimoth,
Jozabad,
Eliel,
Ismakiah,
Mahath,
and Benaiah
worked under
the supervision
of Konaniah
and his brother
Shimei,
as directed
by King
Hezekiah
and Azariah,
the supervisor
of God’s
temple.
\par }{\PP \VS{14}Kore
son
of Imnah,
a Levite
and the guard
on the east
side, was in charge
of the voluntary
offerings made to God
and disbursed
the contributions
made to the
{\ND{Lord}}
and the consecrated
items.
\VS{15}In
the cities
of the priests,
Eden,
Miniamin,
Jeshua,
Shemaiah,
Amariah,
and Shecaniah
faithfully
assisted him in making disbursements
to their fellow priests
according to their divisions,
regardless of age.
\VS{16}They made disbursements to all
the males
three
years
old
and up who
were listed in the genealogical records
– to all
who would enter
the
{\ND{Lord}}’s
temple
to serve on
a daily
basis
and fulfill
their duties
as assigned to their divisions.
\VS{17}They made disbursements to the priests
listed in the genealogical records
by their families,
and to the Levites
twenty
years
old and up, according to their duties
as assigned to their divisions,
\VS{18}and to all
the infants,
wives,
sons,
and daughters
of the entire
assembly
listed in the genealogical records,
for
they faithfully
consecrated themselves.
\VS{19}As for the descendants
of Aaron,
the priests
who lived in the outskirts
of all
their cities,
men
were assigned
to disburse
portions
to every
male
among the priests
and to every
Levite
listed in the genealogical records.
\par }{\PP \VS{20}This
is what
Hezekiah
did
throughout
Judah.
He did
what the
{\ND{Lord}}
his God
considered good
and right
and faithful.
\VS{21}He wholeheartedly
and successfully reinstituted
service
in God’s
temple
and obedience
to the law,
in order
to follow
his God.

\par }\Chap{32}{\PP \VerseOne{1}After
these
faithful
deeds were accomplished,
King
Sennacherib
of Assyria
invaded
Judah.
He besieged
the fortified
cities,
intending
to
seize them.
\VS{2}When Hezekiah
saw
that
Sennacherib
had invaded
and intended
to attack
Jerusalem,
\VS{3}he consulted
with
his advisers and military officers
about stopping
up the springs
outside
the city,
and they supported him.
\VS{4}A large
number of people
gathered
together and stopped
up all
the springs
and the
stream
that flowed
through the district. They reasoned, “Why
should the kings
of Assyria
come
and find
plenty
of water?”
\VS{5}Hezekiah energetically
rebuilt
every
broken
wall.
He erected
towers
and an outer
wall,
and fortified
the terrace
of the City
of David.
He made
many
weapons
and shields.
\par }{\PP \VS{6}He appointed
military
officers
over
the army
and assembled
them in the square
at the city
gate.
He encouraged
them, saying,
\VS{7}“Be strong
and brave! Don’t
be afraid
and don’t
panic
because
of the king
of Assyria
and this huge army
that
is with
him! We have with
us one who is stronger
than those who are with him.
\VS{8}He has with
him mere
human strength,
but
the
{\ND{Lord}}
our God
is with us to help
us and fight
our battles!” The army
was encouraged
by
the words
of King
Hezekiah
of Judah.
\par }{\PP \VS{9}Afterward
King
Sennacherib
of Assyria,
while attacking Lachish
with all
his military might,
sent
his messengers
to Jerusalem.
The message was for King
Hezekiah
of Judah
and all
the people of Judah
who
were in Jerusalem. It read:
\VS{10}“This is what
King
Sennacherib
of Assyria
says: ‘Why
are you
so confident
that you remain
in Jerusalem
while it is under siege?
\VS{11}Hezekiah
says, “The
{\ND{Lord}}
our God
will rescue
us from the power
of the
king
of Assyria.”
But he is misleading
you
and you will
die
of hunger
and thirst!
\VS{12}Hezekiah
is the one
who eliminated
the
{\ND{Lord}}’s high places
and altars
and then told
Judah
and Jerusalem,
“At
one
altar
you must worship
and offer sacrifices.”
\VS{13}Are you not
aware
of what
I
and my predecessors
have done
to all
the nations
of the surrounding lands? Have the gods
of the surrounding lands
actually been able
to rescue
their lands
from my power?
\VS{14}Who
among all
the gods
of these
nations
whom
my predecessors
annihilated
was able
to rescue
his people
from my power?
\VS{15}Now
don’t
let Hezekiah
deceive
you or
mislead
you like this.
Don’t
believe
him, for
no
god
of any
nation
or kingdom
has been able
to rescue
his people
from my power
or the power
of my predecessors.
So how can your gods
rescue
you from my power?’ ”
\par }{\PP \VS{16}Sennacherib’s
servants
further insulted the
{\ND{Lord}}
God
and his servant
Hezekiah.
\VS{17}He wrote
letters
mocking
the {\ND{Lord}}
God
of Israel
and insulting him with these words: “The gods
of the surrounding
nations
could
not
rescue
their people
from my power.
Neither can Hezekiah’s
god
rescue
his people
from my power.”
\VS{18}They called
out loudly
in the Judahite dialect
to
the people
of Jerusalem
who
were on
the wall,
trying to scare
and terrify
them so
they could seize
the
city.
\VS{19}They talked
about the God
of Jerusalem
as if he were one
of the
man-made
gods
of the nations
of the earth.
\par }{\PP \VS{20}King
Hezekiah
and the prophet
Isaiah
son
of Amoz
prayed
about this
and cried
out to heaven.
\VS{21}The
{\ND{Lord}}
sent
a messenger
and he wiped out
all
the soldiers,
princes,
and officers
in the army
of the king
of Assyria.
So Sennacherib returned
home
humiliated.
When he entered
the temple
of his god,
some of his own sons
struck
him down
with the sword.
\VS{22}The
{\ND{Lord}}
delivered
Hezekiah
and the residents
of Jerusalem
from the power
of King
Sennacherib
of Assyria
and from
all
the other nations. He made them secure
on every side.
\VS{23}Many
were bringing
presents
to the
{\ND{Lord}}
in Jerusalem
and precious
gifts to King
Hezekiah
of Judah.
From that time on he was respected
by all
the nations.
\par }{\SH Hezekiah’s Shortcomings and Accomplishments
\par }{\PP \VS{24}In those
days
Hezekiah
was stricken
with a terminal
illness. He prayed
to
the {\ND{Lord}}, who answered
him and gave
him a sign confirming that he would be healed.
\VS{25}But
Hezekiah
was ungrateful;
he had a proud
attitude, provoking
God to be
angry
at him,
as well as Judah
and Jerusalem.
\VS{26}But then Hezekiah
and the residents
of Jerusalem
humbled
themselves and abandoned their pride,
and the
{\ND{Lord}}
was not
angry
with them for the rest
of Hezekiah’s reign.
\par }{\PP \VS{27}Hezekiah
was very wealthy
and greatly
respected. He made
storehouses
for his silver,
gold,
precious
stones,
spices,
and all
his other valuable
possessions.
\VS{28}He made storerooms
for the harvest
of grain,
wine,
and olive oil,
and stalls
for all
his various kinds
of livestock
and his flocks.
\VS{29}He built
royal cities
and owned
a large number
of sheep
and cattle,
for
God
gave
him a huge
amount
of possessions.
\par }{\PP \VS{30}Hezekiah
dammed
up the
source of the waters
of the Upper
Gihon
and directed
them down to the west side
of the City
of David.
Hezekiah
succeeded
in all
that he did.
\VS{31}So
when the envoys
arrived from the Babylonian
officials
to visit
him and inquire
about the sign
that
occurred
in the land,
God
left
him alone to test
him, in order to know
his true motives.
\par }{\PP \VS{32}The rest
of the events
of Hezekiah’s
reign, including his faithful
deeds, are recorded
in the vision
of the prophet
Isaiah
son
of Amoz,
included in the Scroll
of the Kings
of Judah
and Israel.
\VS{33}Hezekiah
passed away
and was buried
on the ascent
of the tombs
of the descendants
of David.
All
the people of Judah
and the residents
of Jerusalem
buried
him with great honor.
His son
Manasseh
replaced
him as king.

\par }\Chap{33}{\PP \VerseOne{1}Manasseh
was twelve
years
old when he became king, and he reigned
for fifty-five
years
in Jerusalem.
\VS{2}He did
evil
in the sight
of the
{\ND{Lord}}
and committed the same horrible
sins practiced by the nations
whom
the {\ND{Lord}}
drove
out ahead
of the Israelites.
\VS{3}He rebuilt
the
high places
that
his father
Hezekiah
had
destroyed;
he set
up altars
for the Baals
and made
Asherah poles.
He bowed
down to all
the stars
in the sky
and worshiped them.
\VS{4}He built
altars
in the
{\ND{Lord}}’s
temple,
about which
the {\ND{Lord}}
had
said,
“Jerusalem
will be
my
permanent
home.”
\VS{5}In the two
courtyards
of the
{\ND{Lord}}’s
temple
he built
altars
for all
the stars
in the sky.
\VS{6}He
passed
his sons
through the
fire
in the Valley
of Ben-Hinnom
and practiced divination,
omen
reading,
and sorcery. He
set up a ritual pit to conjure
up underworld spirits and appointed magicians
to supervise
it. He did
a great amount of evil
in the sight
of the {\ND{Lord}}
and angered him.
\VS{7}He put
an idolatrous
image
he had
made
in God’s
temple,
about which God
had
said
to
David
and to
his son
Solomon,
“This
temple
in Jerusalem,
which
I have chosen
out of all
the tribes
of Israel,
will be my permanent
home.
\VS{8}I will not
make
Israel
again
leave
the
land
I gave
to their ancestors,
provided
that they carefully
obey
all
I commanded
them, the whole
law,
the rules
and regulations
given
to Moses.”
\VS{9}But Manasseh
misled
the people of Judah
and the residents
of Jerusalem
so
that they sinned
more than
the nations
whom
the {\ND{Lord}}
had destroyed
ahead
of the Israelites.
\par }{\PP \VS{10}The
{\ND{Lord}}
confronted
Manasseh
and his people,
but they paid no
attention.
\VS{11}So the
{\ND{Lord}}
brought
against
them the commanders
of the army
of the king
of Assyria.
They seized
Manasseh,
put hooks
in his nose, bound
him with bronze chains,
and carried
him away
to Babylon.
\VS{12}In his pain
Manasseh
asked the
{\ND{Lord}}
his God
for mercy and truly
humbled
himself before
the God
of his ancestors.
\VS{13}When he prayed
to
the
{\ND{Lord}}, the
{\ND{Lord}} responded
to
him and answered favorably
his cry for mercy.
The
{\ND{Lord}} brought him back
to Jerusalem
to his kingdom.
Then Manasseh
realized
that
the {\ND{Lord}}
is the true God.
\par }{\PP \VS{14}After
this
Manasseh built up
the outer
wall
of the City
of David
on the west side
of the Gihon
in the valley
to the entrance
of the Fish
Gate
and all around
the terrace;
he made
it much
higher.
He placed
army
officers
in all
the fortified
cities
in Judah.
\par }{\PP \VS{15}He removed
the
foreign
gods
and images
from the
{\ND{Lord}}’s
temple
and all
the altars
he had
built
on the hill
of the
{\ND{Lord}}’s
temple
and in Jerusalem;
he threw
them outside
the city.
\VS{16}He erected
the
altar
of the {\ND{Lord}}
and offered
on
it peace offerings
and thank offerings.
He told
the people of Judah
to serve
the

{\ND{Lord}}
God
of Israel.
\VS{17}The people
continued
to offer sacrifices
at the high places,
but only
to the
{\ND{Lord}}
their God.
\par }{\PP \VS{18}The rest
of the events
of Manasseh’s
reign, including his prayer
to
his God
and the words
the prophets
spoke
to him
in the name
of the {\ND{Lord}}
God
of Israel,
are recorded
in the Annals
of the Kings
of Israel.
\VS{19}The Annals
of the Prophets
include
his prayer,
give an account of how
the
{\ND{Lord}} responded
to it, record
all
his sins
and unfaithful acts,
and identify the sites
where
he built
high places
and erected
Asherah poles
and idols
before
he humbled himself.
\VS{20}Manasseh
passed away
and was buried
in his palace.
His son
Amon
replaced
him as king.
\par }{\SH Amon’s Reign
\par }{\PP \VS{21}Amon
was twenty-two
years
old
when he became king,
and he reigned
for two
years
in Jerusalem.
\VS{22}He did
evil
in the sight
of the
{\ND{Lord}}, just
like his father
Manasseh
had
done.
He offered sacrifices
to all
the idols
his
father
Manasseh
had made, and worshiped them.
\VS{23}He did not
humble
himself before
the {\ND{Lord}}
as his father
Manasseh
had done. Amon
was guilty
of great sin.
\VS{24}His servants
conspired
against
him and killed
him in his palace.
\VS{25}The people
of the land
executed
all
who had conspired
against
King
Amon,
and they made his son
Josiah
king
in his place.

\par }\Chap{34}{\PP \VerseOne{1}Josiah
was eight
years
old when he became king, and he reigned
for thirty-one
years
in Jerusalem.
\VS{2}He did
what the
{\ND{Lord}}
approved
and followed
in his ancestor
David’s
footsteps;
he did not
deviate
to the right
or the left.
\par }{\PP \VS{3}In the eighth
year
of his reign,
while
he was still
young,
he began
to seek
the God
of his ancestor
David.
In his twelfth
year
he began
ridding
Judah
and Jerusalem
of the high places,
Asherah poles,
idols,
and images.
\VS{4}He ordered
the altars
of the Baals
to be torn down, and broke the incense altars
that
were above
them. He smashed
the Asherah poles,
idols
and images,
crushed
them up and sprinkled
the dust over
the tombs
of those who had sacrificed to them.
\VS{5}He burned
the bones
of the pagan priests
on
their altars;
he purified
Judah
and Jerusalem.
\VS{6}In the cities
of Manasseh,
Ephraim,
and Simeon,
as far
as Naphtali,
and in the ruins
around them,
\VS{7}he tore down
the altars
and Asherah poles,
demolished
the idols,
and smashed
all
the incense altars
throughout
the land
of Israel.
Then he returned
to Jerusalem.
\par }{\PP \VS{8}In the eighteenth
year
of his reign,
he continued his policy of purifying
the land
and the
temple.
He sent
Shaphan
son
of Azaliah,
Maaseiah
the city
official,
and Joah
son
of Joahaz
the secretary
to repair
the
temple
of the {\ND{Lord}}
his God.
\VS{9}They went
to
Hilkiah
the high
priest
and gave
him the silver
that had been brought
to God’s
temple.
The Levites
who
guarded
the door
had collected
it from the people of Manasseh
and Ephraim
and from all
who were left
in Israel,
as well as from all
the people of Judah
and Benjamin
and the residents
of Jerusalem.
\VS{10}They handed
it over
to the construction
foremen
assigned
to the
{\ND{Lord}}’s
temple.
They in turn paid
the temple
workers
to restore and repair it.
\VS{11}They gave
money to the craftsmen
and builders
to buy
chiseled
stone
and wood
for the braces
and rafters
of the buildings
that
the kings
of Judah
had allowed to fall into disrepair.
\VS{12}The men
worked
faithfully.
Their supervisors
were
Jahath
and Obadiah
(Levites
descended
from
Merari), as well as Zechariah
and Meshullam
(descendants
of Kohath). The Levites,
all
of whom were skilled
musicians,
\VS{13}supervised
the laborers
and all
the foremen
on their various jobs.
Some of the Levites
were scribes,
officials,
and guards.
\par }{\PP \VS{14}When they took out
the silver
that had been brought
to the
{\ND{Lord}}’s
temple,
Hilkiah
the priest
found
the law
scroll
the {\ND{Lord}}
had given
to Moses.
\VS{15}Hilkiah
informed
Shaphan
the scribe,
“I found
the law
scroll
in the
{\ND{Lord}}’s
temple.”
Hilkiah
gave
the scroll
to
Shaphan.
\VS{16}Shaphan
brought
the scroll
to
the king
and reported, “Your servants
are
doing
everything
assigned to them.
\VS{17}They melted
down the silver
in the
{\ND{Lord}}’s
temple
and handed it
over
to the supervisors
of the construction
foremen.”
\VS{18}Then Shaphan
the scribe
told the king,
“Hilkiah
the priest
has given
me a scroll.”
Shaphan
read
it out loud before
the king.
\VS{19}When
the king
heard
the words
of the law
scroll, he tore
his clothes.
\VS{20}The king
ordered
Hilkiah,
Ahikam
son
of Shaphan,
Abdon
son
of Micah,
Shaphan
the scribe,
and Asaiah
the king’s
servant,
\VS{21}“Go,
seek
an oracle from the

{\ND{Lord}}
for
me
and those who remain
in Israel
and Judah.
Find out
about the words
of this scroll
that
has been discovered.
For
the
{\ND{Lord}}’s
fury
has
been ignited
against us, because
our ancestors
have
not
obeyed
the
word
of the {\ND{Lord}}
by doing
all
that this
scroll
instructs!”
\par }{\PP \VS{22}So Hilkiah
and the others sent by the king
went
to Huldah
the prophetess,
the wife
of Shallum
son
of Tokhath,
the son
of Hasrah,
the supervisor
of the wardrobe.
(She lived in
Jerusalem
in the Mishneh district.) They stated their business,
\VS{23}and she said
to them: “This is what
the {\ND{Lord}}
God
of Israel
says: ‘Say
this to the man
who
sent
you to me:
\VS{24}“This is what
the {\ND{Lord}}
says: ‘I am
about to bring
disaster
on
this
place
and its residents,
the details
of which are recorded
in the scroll
which
they read
before
the king
of Judah.
\VS{25}This will happen because
they have abandoned
me and offered sacrifices
to other
gods,
angering
me with all
the idols
they have made.
My anger
will ignite
against this
place
and will not be extinguished!’ ”
\VS{26}Say
this
to
the king
of Judah,
who sent
you to seek
an oracle from the
{\ND{Lord}}: “This is what
the

{\ND{Lord}}
God
of Israel
says concerning the words
you have
heard:
\VS{27}‘You displayed a sensitive
spirit and humbled
yourself before
God
when you heard
his words
concerning
this
place
and its residents.
You humbled
yourself before
me, tore
your clothes
and wept
before
me, and I
have heard
you,’ says
the {\ND{Lord}}.
\VS{28}‘Therefore I will allow
you to
die
and be buried
in peace.
You will not
have to
witness
all
the disaster
I
will bring
on
this
place
and its residents.’ ”’ ” Then they reported back
to the king.
\par }{\PP \VS{29}The king
summoned
all
the leaders
of Judah
and Jerusalem.
\VS{30}The king
went up
to the
{\ND{Lord}}’s
temple,
accompanied by all
the people
of Judah,
the residents
of Jerusalem,
the priests,
and the Levites.
All
the people
were there, from the oldest
to
the youngest.
He read
aloud
all
the words
of the scroll
of the covenant
that had been discovered
in the
{\ND{Lord}}’s
temple.
\VS{31}The king
stood
by
his pillar and renewed
the
covenant
before
the {\ND{Lord}}, agreeing
to follow
the {\ND{Lord}}
and to obey
his commandments,
laws,
and rules
with all
his heart
and being,
by carrying out
the
terms
of this covenant
recorded
on
this
scroll.
\VS{32}He made
all
who were
in Jerusalem
and Benjamin
agree to it.
The residents
of Jerusalem
acted
in accordance with the covenant
of God,
the God
of their ancestors.
\VS{33}Josiah
removed
all
the detestable
idols from all
the areas
belonging to the
Israelites
and encouraged
all
who were
in Israel
to worship
the

{\ND{Lord}}
their God.
Throughout
the rest
of his reign
they did not
turn aside
from following
the {\ND{Lord}}
God
of their ancestors.

\par }\Chap{35}{\PP \VerseOne{1}Josiah
observed
a Passover festival
for the
{\ND{Lord}}
in Jerusalem.
They slaughtered
the Passover lambs
on the fourteenth
day of the first
month.
\VS{2}He appointed
the priests
to fulfill their duties
and encouraged them
to carry out their service
in the
{\ND{Lord}}’s
temple.
\VS{3}He told
the Levites,
who instructed
all
Israel
about things consecrated
to the
{\ND{Lord}}, “Place
the holy
ark
in the temple
which
King
Solomon
son
of David
of Israel
built.
Don’t
carry
it on your shoulders.
Now
serve
the

{\ND{Lord}}
your God
and his people
Israel!
\VS{4}Prepare
yourselves by your families
according to your divisions,
as instructed
by King
David
of Israel
and his son
Solomon.
\VS{5}Stand in
the sanctuary
and, together with the Levites,
represent the family
divisions
of your countrymen.
\VS{6}Slaughter
the Passover lambs,
consecrate
yourselves, and make preparations
for your countrymen
to do
what
the {\ND{Lord}}
commanded through
Moses.”
\par }{\PP \VS{7}From his own
royal
flocks
and herds, Josiah
supplied
the people
with 30,000
lambs
and goats
for the Passover sacrifice,
as well as 3,000
cattle.
\VS{8}His officials
also willingly contributed
to the people,
priests,
and Levites.
Hilkiah,
Zechariah,
and Jehiel,
the leaders
of God’s
temple,
supplied
2,600
Passover sacrifices
and 300
cattle.
\VS{9}Konaniah
and his brothers
Shemaiah
and Nethanel,
along with Hashabiah,
Jeiel,
and Jozabad,
the officials
of the Levites,
supplied
the Levites
with 5,000 Passover sacrifices
and 500
cattle.
\VS{10}Preparations
were made,
and the priests
stood
at their posts and the Levites
in their divisions
as prescribed
by the king.
\VS{11}They slaughtered
the Passover lambs
and the priests
splashed
the blood, while the Levites
skinned the animals.
\VS{12}They reserved
the burnt offerings
and the cattle
for the family
divisions
of the people
to present
to the
{\ND{Lord}}, as prescribed
in the scroll
of Moses.
\VS{13}They cooked
the Passover sacrifices
over the open fire
as prescribed
and cooked
the consecrated
offerings in pots, kettles,
and pans.
They quickly
served them to all
the people.
\VS{14}Afterward
they made preparations
for themselves
and for the priests,
because
the priests,
the descendants
of Aaron,
were offering
burnt sacrifices
and fat
portions until
evening.
The Levites
made preparations
for themselves
and for the priests,
the descendants
of Aaron.
\VS{15}The musicians,
the descendants
of Asaph,
manned
their posts, as prescribed by
David,
Asaph,
Heman,
and Jeduthun
the king’s
prophet.
The guards
at the various
gates
did not
need to leave
their posts,
for
their fellow
Levites
made preparations for them.
\VS{16}So all
the preparations
for the
{\ND{Lord}}’s
service
were made that day,
as the Passover
was observed
and the burnt sacrifices
were offered
on
the altar
of the {\ND{Lord}}, as prescribed
by King
Josiah.
\VS{17}So
the Israelites
who were present
observed the Passover
at that time,
as well as the Feast
of Unleavened
Bread for seven
days.
\VS{18}A Passover
like
this had
not
been
observed
in Israel
since the days
of Samuel
the prophet.
None
of the kings
of Israel
had observed
a Passover
like the one celebrated
by Josiah,
the priests,
the Levites,
all
the people of Judah
and Israel
who were
there, and the residents
of Jerusalem.
\VS{19}This
Passover
was observed
in the eighteenth
year
of Josiah’s
reign.
\par }{\SH Josiah’s Reign Ends
\par }{\PP \VS{20}After
Josiah
had done
all
this
for the temple,
King
Necho
of Egypt
marched up
to do battle at Carchemish
on
the Euphrates
River. Josiah
marched
out
to oppose him.
\VS{21}Necho sent
messengers
to
him, saying,
“Why
are you opposing me, O king
of Judah? I am not
attacking you
today,
but
the kingdom
with which I am at war.
God
told
me to hurry.
Stop
opposing God,
who
is with
me, or else he will destroy you.”
\VS{22}But Josiah
did not
turn
back from
him; he disguised
himself for
battle.
He did not
take seriously
the words
of Necho
which he had received from God;
he went
to fight
him in the Plain
of Megiddo.
\VS{23}Archers
shot
King
Josiah;
the king
ordered
his servants,
“Take me out of this chariot, for
I am seriously
wounded.”
\VS{24}So
his servants
took
him out
of
the chariot,
put him in
another chariot
that
he owned,
and brought
him to Jerusalem,
where he died.
He was buried
in the tombs
of his ancestors;
all
the people of Judah
and Jerusalem
mourned
Josiah.
\VS{25}Jeremiah
composed
laments
for Josiah
which all
the male
and female singers
use to mourn
Josiah
to this very
day.
It
has become customary
in Israel
to sing these; they are recorded
in the Book of Laments.
\par }{\PP \VS{26}The rest
of the events
of Josiah’s
reign, including the faithful
acts he did in obedience to what is written
in the law
of the {\ND{Lord}}
\VS{27}and his accomplishments,
from start
to finish,
are recorded
in the Scroll
of the Kings
of Israel
and Judah.

\par }\Chap{36}{\PP \VerseOne{1}The people
of the land
took
Jehoahaz
son
of Josiah
and made
him king
in his father’s
place
in Jerusalem.
\VS{2}Jehoahaz
was twenty-three
years
old when he became king, and he reigned
three
months
in Jerusalem.
\VS{3}The king
of Egypt
prevented
him from ruling in Jerusalem
and imposed
on the land
a special tax of one hundred
talents
of silver
and a talent
of gold.
\VS{4}The king
of Egypt
made Jehoahaz’s brother
Eliakim
king
over
Judah
and Jerusalem,
and changed
his name
to Jehoiakim.
Necho
seized his brother
Jehoahaz
and took
him to Egypt.
\par }{\SH Jehoiakim’s Reign
\par }{\PP \VS{5}Jehoiakim
was twenty-five
years
old when he became king, and he reigned
for eleven
years
in Jerusalem.
He did
evil
in the sight
of the
{\ND{Lord}}
his God.
\VS{6}King
Nebuchadnezzar
of Babylon
attacked
him, bound
him with bronze chains,
and carried
him away
to Babylon.
\VS{7}Nebuchadnezzar
took
some of the items
in the
{\ND{Lord}}’s
temple
to Babylon
and put
them in his palace
there.
\par }{\PP \VS{8}The rest
of the events
of Jehoiakim’s
reign, including the horrible
sins he committed
and his shortcomings,
are recorded
in the Scroll
of the Kings
of Israel
and Judah.
His son
Jehoiachin
replaced him as king.
\par }{\SH Jehoiachin’s Reign
\par }{\PP \VS{9}Jehoiachin
was eighteen
years
old when he became king, and he reigned
three
months
and ten
days
in Jerusalem.
He did
evil
in the sight
of the
{\ND{Lord}}.
\VS{10}At the beginning
of the year
King
Nebuchadnezzar
ordered
him to be brought
to Babylon,
along with
the valuable
items
in the
{\ND{Lord}}’s
temple.
In his place he made
his relative
Zedekiah
king over
Judah
and Jerusalem.
\par }{\SH Zedekiah’s Reign
\par }{\PP \VS{11}Zedekiah
was twenty-one
years
old
when he became king,
and he ruled
for eleven
years
in Jerusalem.
\VS{12}He did
evil
in the sight
of the
{\ND{Lord}}
his God.
He did not
humble
himself before
Jeremiah
the prophet,
the
{\ND{Lord}}’s
spokesman.
\VS{13}He also
rebelled
against King
Nebuchadnezzar,
who had
made him vow allegiance
in the name of God.
He was stubborn
and obstinate,
and refused to return
to
the {\ND{Lord}}
God
of Israel.
\VS{14}All
the leaders
of the priests
and people
became
more
unfaithful
and committed the same horrible
sins practiced by the nations.
They defiled
the
{\ND{Lord}}’s
temple
which
he had consecrated
in Jerusalem.
\par }{\SH The Babylonians Destroy Jerusalem
\par }{\PP \VS{15}The
{\ND{Lord}}
God
of their ancestors
continually warned
them through
his messengers,
for
he felt
compassion
for
his people
and his dwelling place.
\VS{16}But they mocked
God’s
messengers,
despised
his warnings,
and ridiculed
his prophets.
Finally
the {\ND{Lord}}
got
very angry
at his people
and there was no one
who could prevent his judgment.
\VS{17}He brought
against
them
the king
of the Babylonians,
who slaughtered
their young men
in their temple.
He did not
spare
young men
or women,
or even the old
and aging.
God handed
everyone
over to him.
\VS{18}He carried away to Babylon
all
the items
in God’s
temple,
whether large
or small,
as well as what was in the treasuries
of the
{\ND{Lord}}’s
temple
and in the treasuries
of the king
and his officials.
\VS{19}They burned down
the
{\ND{Lord}}’s
temple
and tore
down the wall
of Jerusalem.
They burned
all
its fortified buildings
and destroyed
all
its valuable
items.
\VS{20}He deported
to
Babylon
all who escaped
the sword.
They served
him and his sons
until
the Persian
kingdom rose to power.
\VS{21}This took place to fulfill
the
{\ND{Lord}}’s
message
delivered
through Jeremiah.
The land
experienced
its sabbatical
years;
it remained
desolate
for seventy
years,
as prophesied.
\par }{\SH Cyrus Allows the Exiles to Go Home
\par }{\PP \VS{22}In the first
year
of the reign of King
Cyrus
of Persia,
in fulfillment
of the promise he delivered through
Jeremiah,
the {\ND{Lord}}
moved
King
Cyrus
of Persia
to issue
a written decree
throughout
his kingdom.
\VS{23}It read: “This is what
King
Cyrus
of Persia
says: ‘The
{\ND{Lord}}
God
of the heavens
has given
to me all
the kingdoms
of the earth.
He has appointed
me to build
for him a temple
in Jerusalem
in Judah.
May the
{\ND{Lord}}
your God
energize
you who
belong to his people,
so you may be able to go back there!”
\par }