\NormalFont\ShortTitle{Esther}
{\MT Esther

\par }\ChapOne{1}{\SH The King Throws a Lavish Party
\par }{\PP \VerseOne{1}The following events happened
in the days
of Ahasuerus.
(I am referring to that
Ahasuerus
who used to rule over
a hundred
and twenty-seven
provinces
extending all the way
from India
to Ethiopia. )
\VS{2}In those
days,
as King
Ahasuerus
sat
on
his royal
throne
in Susa
the citadel,
\VS{3}in the third
year
of his reign
he provided
a banquet
for all
his officials
and his servants.
The army
of Persia
and Media
was present, as well as the nobles
and the officials
of the provinces.
\par }{\PP \VS{4}He displayed
the riches
of his royal
glory
and the splendor
of his majestic
greatness
for a lengthy period of time –
a hundred
and eighty
days, to be exact!
\VS{5}When those days
were
completed,
the king
then provided
a seven-day banquet
for all
the people
who were present
in Susa
the citadel,
for those of highest
standing to the most lowly.
It was held in the court
located in the garden
of the royal
palace.
\VS{6}The furnishings included linen
and purple
curtains hung
by cords
of the finest linen
and purple wool
on
silver
rings,
alabaster
columns,
gold
and silver
couches
displayed on
a floor
made of valuable stones
of alabaster,
mother-of-pearl,
and mineral stone.
\VS{7}Drinks
were served
in golden
containers,
all of which
differed
from one another. Royal
wine
was available in abundance
at the king’s
expense.
\VS{8}There were no
restrictions
on the drinking,
for
the king
had instructed
all
of his supervisors
that they should do
as everyone
so desired.
\VS{9}Queen
Vashti
also
gave a banquet
for the women
in King
Ahasuerus’
royal
palace.
\par }{\SH Queen Vashti is Removed from Her Royal Position
\par }{\PP \VS{10}On the seventh
day,
as King
Ahasuerus
was feeling
the effects
of the wine,
he ordered
Mehuman,
Biztha,
Harbona,
Bigtha,
Abagtha,
Zethar,
and Carcas,
the seven
eunuchs
who attended him,
\VS{11}to bring
Queen
Vashti
into the king’s
presence
wearing her royal
high turban.
He wanted to show
the people
and the officials
her beauty,
for
she was
very attractive.
\VS{12}But Queen
Vashti
refused
to come
at the king’s
bidding conveyed
through
the eunuchs.
Then
the king
became extremely
angry,
and his rage
consumed him.
\par }{\PP \VS{13}The king
then
inquired of the wise
men who were discerners
of the times
– for
it was the royal
custom
to confer
with all
those who were proficient
in laws
and legalities.
\VS{14}Those who were closest
to
him were Carshena,
Shethar,
Admatha,
Tarshish,
Meres,
Marsena,
and Memucan.
These men were the seven
officials
of Persia
and Media
who saw
the king
on a regular basis
and had the most prominent
offices
in the kingdom.
\VS{15}The king asked, “By law,
what
should be done
to Queen
Vashti
in light of the fact that
she has not
obeyed the instructions
of King
Ahasuerus
conveyed through
the eunuchs?”
\par }{\PP \VS{16}Memucan
then replied
to the king
and the officials,
“The wrong
of Queen
Vashti
is not
against
the king
alone,
but
against
all
the officials
and all
the people
who
are throughout all
the provinces
of King
Ahasuerus.
\VS{17}For
the matter
concerning the queen
will spread
to all
the women,
leading them to treat
their husbands
with contempt,
saying,
‘When King
Ahasuerus
gave orders
to bring
Queen
Vashti
into his presence,
she would not
come.’
\VS{18}And this
very day
the noble ladies
of Persia
and Media
who
have heard
the
matter
concerning the queen
will respond in the same way to all
the royal
officials,
and there will be more than enough
contempt
and anger!
\VS{19}If
the king
is so inclined,
let a royal
edict
go forth
from him,
and let it be written
in the laws
of Persia
and Media
that cannot
be repealed,
that
Vashti
may not
come
into the presence
of King
Ahasuerus,
and let the king
convey
her royalty
to another
who is more deserving
than she.
\VS{20}And let the king’s
decision
which
he
will enact
be disseminated throughout all
his kingdom,
vast
though
it is. Then all
the women
will give
honor
to
their husbands,
from the most prominent
to the lowly.”
\par }{\PP \VS{21}The matter
seemed
appropriate
to the king
and the officials.
So
the king
acted
on the advice
of Memucan.
\VS{22}He sent
letters
throughout
all
the royal
provinces,
to
each
province
according to its own script
and to
each
people
according to its own language,
that every
man
should be
ruling
his family
and should be speaking
the language
of his own people.

\par }\Chap{2}{\PP \VerseOne{1}When
these
things
had been accomplished
and the rage
of King
Ahasuerus
had diminished, he remembered
Vashti
and what she had
done
and what had
been decided
against
her.
\VS{2}The king’s
servants
who attended
him said,
“Let a search
be conducted in the king’s
behalf for attractive
young
women.
\VS{3}And let the king
appoint
officers
throughout all
the provinces
of his kingdom
to gather
all
the attractive
young
women
to
Susa
the citadel,
to
the harem
under the authority
of Hegai,
the king’s
eunuch
who oversees
the women,
and let him provide
whatever cosmetics they desire.
\VS{4}Let the young
woman whom
the king
finds most attractive become queen
in place
of Vashti.”
This seemed like a good
idea
to the king,
so
he acted
accordingly.
\par }{\PP \VS{5}Now there happened to be
a Jewish
man
in Susa
the citadel
whose name
was Mordecai.
He was the son
of Jair,
the son
of Shimei,
the son
of Kish,
a Benjaminite,
\VS{6}who had
been taken into exile
from Jerusalem
with
the captives
who had
been carried into exile
with
Jeconiah
king
of Judah,
whom
Nebuchadnezzar
king
of Babylon
had taken into exile.
\VS{7}Now
he was acting as the guardian
of Hadassah
(that
is, Esther), the daughter
of his uncle,
for
neither
her father
nor her mother
was alive. This young woman
was very attractive
and had a beautiful
figure.
When her father
and mother
died,
Mordecai
had raised
her as if she were his own daughter.
\par }{\PP \VS{8}It so happened
that when
the king’s
edict
and his law
became known
many
young
women were taken to
Susa
the citadel
to
be placed under the authority
of Hegai.
Esther
also was taken
to
the royal
palace
to
be under the authority
of Hegai,
who
was overseeing
the women.
\VS{9}This young woman
pleased
him, and she found
favor
with him. He quickly
provided
her with her cosmetics
and her rations;
he also provided
her with the seven
specially chosen
young women
who were from the palace.
He then transferred
her and her young women
to the best
quarters in the harem.
\par }{\PP \VS{10}Now Esther
had not
disclosed
her people
or her lineage,
for
Mordecai
had instructed
her not to do so.
\VS{11}And day
after day
Mordecai
used to walk
back and forth in front
of the court
of the harem
in order to learn
how
Esther
was doing
and what
might happen to her.
\par }{\PP \VS{12}At the end
of the twelve
months
that were required for the women,
when the turn
of each young woman
arrived
to go to
King
Ahasuerus
– for
in this way
they had to fulfill
their time
of cosmetic treatment: six
months
with oil
of myrrh,
and six
months
with perfume
and various ointments
used by women –
\VS{13}the woman
would go
to
the king
in the following way: Whatever
she asked
for would be provided
for her to take
with
her from
the harem
to
the royal
palace.
\VS{14}In the evening
she
went,
and in the morning
she
returned
to
a separate part of the harem,
to the authority of Shaashgaz
the king’s
eunuch
who
was overseeing
the concubines.
She would not
go
back
to
the king
unless
the king
was pleased
with her and she was requested
by name.
\par }{\PP \VS{15}When it
became the turn
of Esther
daughter
of Abihail
the uncle
of Mordecai
(who had
raised her
as if
she were his own daughter
) to go
to
the king,
she did not
request
anything
except what
Hegai
the king’s
eunuch,
who was
overseer
of the women,
had recommended. Yet Esther
met
with the approval
of all
who saw her.
\VS{16}Then Esther
was taken
to
King
Ahasuerus
at his royal
residence
in the tenth
month
(that is,
the month
of Tebeth) in the seventh
year
of his reign.
\VS{17}And the king
loved
Esther
more than all
the other women,
and she met
with his loving approval
more than all
the other young women.
So he placed
the royal
high turban
on her head
and appointed her queen
in place
of Vashti.
\VS{18}Then the king
prepared
a large
banquet
for all
his officials
and his servants
– it was actually Esther’s
banquet.
He also set aside
a holiday
for the provinces,
and he provided
for offerings
at the king’s
expense.
\par }{\SH Mordecai Learns of a Plot against the King
\par }{\PP \VS{19}Now when the young women
were being gathered
again,
Mordecai
was sitting
at the king’s
gate.
\VS{20}Esther
was still not
divulging
her lineage
or her people,
just as
Mordecai
had instructed
her. Esther
continued to do
whatever Mordecai
said,
just as
she had done when
he was raising her.
\par }{\PP \VS{21}In those
days
while
Mordecai
was
sitting
at
the king’s
gate,
Bigthan
and Teresh,
two
of the king’s
eunuchs
who protected
the entrance,
became angry
and plotted to assassinate
King
Ahasuerus.
\VS{22}When Mordecai
learned
of the conspiracy,
he informed
Queen
Esther,
and Esther
told
the king
in Mordecai’s
behalf.
\VS{23}The
king then had
the matter
investigated
and, finding
it to be so, had the two
conspirators hanged
on
a gallows.
It was then recorded
in the daily chronicles
in the king’s
presence.

\par }\Chap{3}{\PP \VerseOne{1}Some time later
King
Ahasuerus
promoted
Haman
the son
of Hammedatha,
the Agagite,
exalting
him and setting
his position
above
that of all
the officials
who
were with him.
\VS{2}As a result, all
the king’s
servants
who
were at the king’s
gate
were bowing
and paying
homage to Haman,
for
the king
had so
commanded.
However, Mordecai
did not bow,
nor did he pay him homage.
\par }{\PP \VS{3}Then the servants
of the king
who
were at the king’s
gate
asked Mordecai,
“Why
are you
violating
the king’s
commandment?”
\VS{4}And after
they had spoken
to him
day
after day
without his paying any attention
to
them, they informed
Haman
to see
whether this
attitude
on Mordecai’s
part would be permitted. Furthermore, he had disclosed to them
that
he was
a Jew.
\par }{\PP \VS{5}When Haman
saw
that
Mordecai
was not
bowing
or paying homage
to him, he
was filled
with rage.
\VS{6}But
the thought of striking out against Mordecai
alone
was repugnant
to him, for
he had been informed
of the
identity of Mordecai’s
people.
So Haman
sought
to destroy
all
the Jews
(that
is, the people
of Mordecai) who were in all
the kingdom
of Ahasuerus.
\par }{\PP \VS{7}In the first
month
(that is,
the month
of Nisan), in the twelfth
year
of King
Ahasuerus’
reign,
{\IT{
pur}}
(that is, the lot) was cast
before
Haman
in order to determine a day
and a month.
It turned out to be the twelfth
month
(that is,
the month
of Adar).
\par }{\PP \VS{8}Then Haman
said
to King
Ahasuerus,
“There
is a particular
people
that is dispersed and spread
among
the inhabitants
throughout all
the provinces
of your kingdom
whose laws
differ from
those of all
other peoples.
Furthermore, they do
not observe the king’s
laws.
It is not
appropriate
for the king
to provide a haven
for them.
\VS{9}If
the king
is so inclined,
let an edict be issued
to destroy
them. I will pay
ten
thousand
talents
of silver
to
be conveyed
to
the king’s
treasuries
for the officials
who carry out
this business.”
\par }{\PP \VS{10}So the king
removed
his signet ring
from his hand
and gave
it to Haman
the son
of Hammedatha,
the Agagite,
who was hostile
toward the Jews.
\VS{11}The king
replied
to Haman,
“Keep your money,
and do
with those people
whatever
you wish.”
\par }{\PP \VS{12}So the royal
scribes
were summoned
in the first
month,
on the thirteenth
day
of the month. Everything
Haman
commanded
was written
to
the king’s
satraps
and governors
who
were in
every
province
and to
the officials
of every
people,
province
by province
according to its script
and people
by people
according to its language.
In the name
of King
Ahasuerus
it was written
and sealed
with the king’s
signet ring.
\VS{13}Letters
were sent
by
the runners
to
all
the king’s
provinces
stating that they should destroy,
kill,
and annihilate
all
the Jews,
from
youth
to elderly,
both
women
and children,
on a
particular day,
namely the thirteenth
day of the twelfth
month
(that
is, the month
of Adar), and to loot
and plunder their possessions.
\VS{14}A copy
of this edict
was to be presented
as law
throughout
every
province;
it was to be made known
to all
the inhabitants,
so that they would be
prepared
for this
day.
\VS{15}The messengers
scurried
forth
with the king’s
order. The edict
was issued
in Susa
the citadel.
While the king
and Haman
sat
down to drink,
the city
of Susa
was in an uproar!

\par }\Chap{4}{\PP \VerseOne{1}Now when Mordecai
became aware
of all
that
had been done,
he
tore
his garments
and put on
sackcloth
and ashes.
He went out
into
the city,
crying out
in a loud
and bitter
voice.
\VS{2}But he went
no further
than the king’s
gate,
for
no
one was permitted to enter
the king’s
gate
clothed
in sackcloth.
\VS{3}Throughout each
and every
province
where
the king’s
edict
and law were announced there was considerable
mourning
among the Jews,
along with fasting,
weeping,
and sorrow.
Sackcloth
and ashes
were characteristic
of many.
\VS{4}When Esther’s
female
attendants and her eunuchs
came
and informed
her about Mordecai’s behavior, the queen
was overcome
with anguish.
Although she sent
garments
for Mordecai
to put on
so that he could remove
his sackcloth,
he would not
accept them.
\VS{5}So Esther
called
for Hathach,
one of the king’s
eunuchs
who had
been placed at
her service,
and instructed
him to find out the cause and reason for Mordecai’s behavior.
\VS{6}So Hathach
went to
Mordecai
at the plaza
of the city
in front
of the king’s
gate.
\VS{7}Then Mordecai
related
to him everything
that had happened
to him, even the specific amount
of money
that
Haman
had offered to pay
to the king’s
treasuries
for
the Jews
to be destroyed.
\VS{8}He also gave him a written
copy
of the law
that had
been disseminated
in Susa
for their destruction so
that he could show
it to Esther
and talk to her about it. He also gave instructions
that she should go
to
the
king
to implore
him and petition
him on
behalf of
her people.
\VS{9}So Hathach
returned
and related
Mordecai’s
instructions
to Esther.
\par }{\PP \VS{10}Then Esther
replied to Hathach
with instructions
for Mordecai:
\VS{11}“All
the servants
of the king
and the people
of the king’s
provinces
know
that
there is only
one
law
applicable
to any
man
or woman
who
comes
uninvited
to
the king
in the inner
court
– that
person will be put to death, unless the king
extends to him the gold
scepter,
permitting him to be spared.
Now I
have not
been invited
to come
to
the king
for some thirty
days!”
\par }{\PP \VS{12}When Esther’s
reply was conveyed
to Mordecai,
\VS{13}he
said
to take back this answer
to
Esther:
\VS{14}“Don’t imagine that because
you are part of the king’s household you will be the one Jew who
will escape.
If
you keep quiet
at this
time,
liberation
and protection for the Jews
will appear
from another
source,
while you
and your father’s
household
perish.
It may very well be that you have achieved
royal
status for such a time
as this!”
\par }{\PP \VS{15}Then Esther
sent this reply
to
Mordecai:
\VS{16}“Go,
assemble
all
the Jews
who are found
in Susa
and fast
in
my behalf. Don’t
eat
and don’t
drink
for three
days,
night
or day.
My female
attendants and I
will also
fast
in the same
way. Afterward
I will go
to
the king,
even though it violates the law.
If I perish,
I perish!”
\par }{\PP \VS{17}So Mordecai
set out to do
everything
that
Esther
had instructed him.

\par }\Chap{5}{\PP \VerseOne{1}It so happened
that on the third
day
Esther
put on
her royal attire
and stood
in the inner
court
of the palace,
opposite
the king’s
quarters.
The king
was sitting
on
his royal
throne
in the palace,
opposite
the entrance.
\VS{2}When
the king
saw
Queen
Esther
standing
in the court,
she met
with his approval.
The king
extended
to Esther
the
gold
scepter
that
was in his hand,
and Esther
approached
and touched
the end of the scepter.
\par }{\PP \VS{3}The king
said
to her, “What
is on your mind, Queen
Esther? What
is your request? Even
as much as
half
the kingdom
will be given to you!”
\par }{\PP \VS{4}Esther
replied,
“If
the king
is so inclined,
let the king
and Haman
come
today
to
the banquet
that
I have
prepared for him.”
\VS{5}The king
replied,
“Find Haman
quickly
so that we can do
as Esther
requests.”
\par }{\PP So the king
and Haman
went
to
the banquet
that
Esther
had prepared.
\VS{6}While at the banquet
of wine,
the king
said
to Esther,
“What
is your request? It shall be given
to you. What
is your petition? Ask for as
much as
half
the kingdom,
and it shall be done!”
\par }{\PP \VS{7}Esther
responded, “My request
and my petition is this:
\VS{8}If
I have found
favor
in the king’s
sight
and if
the king
is inclined
to grant
my request
and perform
my petition,
let the king
and Haman
come
tomorrow
to
the banquet
that
I will prepare
for them. At that time I will do
as
the king wishes.
\par }{\SH Haman Expresses His Hatred of Mordecai
\par }{\PP \VS{9}Now Haman
went forth
that day
pleased
and very
much encouraged. But
when Haman
saw
Mordecai
at the king’s
gate,
and he did not
rise
nor
tremble
in his presence, Haman
was filled
with rage
toward Mordecai.
\VS{10}But Haman
restrained
himself and went
on to
his home.
\par }{\PP He then sent
for his friends
to join
him, along
with his wife
Zeresh.
\VS{11}Haman
then recounted
to them
his fabulous
wealth,
his many
sons,
and how the king
had
magnified
him and exalted
him over
the king’s
other officials
and servants.
\VS{12}Haman
said,
“Furthermore,
Queen
Esther
invited only me to
accompany
the king
to
the banquet
that
she prepared! And also
tomorrow
I
am invited
along with
the king.
\VS{13}Yet all
of this
fails
to satisfy
me so long as
I
have to see
Mordecai
the Jew
sitting
at the king’s
gate.”
\par }{\PP \VS{14}Haman’s wife
Zeresh
and all
his friends
said
to him, “Have a gallows
seventy-five feet
high
built, and in the morning
tell
the king
that Mordecai
should be hanged
on
it. Then go
with
the king
to
the banquet
contented.”
\par }{\PP It seemed like a good
idea
to Haman,
so he had the gallows
built.

\par }\Chap{6}{\PP \VerseOne{1}Throughout that
night
the king
was unable
to sleep,
so he asked
for the
book
containing the historical
records
to be brought.
As the records were
being read
in the king’s
presence,
\VS{2}it was found
written
that
Mordecai
had disclosed
that Bigthana
and Teresh,
two
of the king’s
eunuchs
who guarded
the entrance,
had plotted
to assassinate
King
Ahasuerus.
\par }{\PP \VS{3}The king
asked,
“What
great
honor
was bestowed on Mordecai
because
of this?” The king’s
attendants
who served
him responded, “Not
a thing
was done
for him.”
\par }{\PP \VS{4}Then the king
said,
“Who
is that in the courtyard?” Now Haman
had come
to the outer
courtyard
of the palace
to suggest that the king
hang
Mordecai
on
the gallows
that he had
constructed for him.
\VS{5}The king’s
attendants
said
to him,
“It is
Haman
who is standing
in the courtyard.”
The king
said,
“Let him enter.”
\par }{\PP \VS{6}So Haman
came
in, and the king
said
to him, “What
should be done
for the man
whom
the king
wishes
to honor?” Haman
thought
to himself, “Who
is it that the king
would want
to honor
more
than me?”
\VS{7}So Haman
said
to
the king,
“For the man
whom
the king
wishes
to honor,
\VS{8}let them bring
royal
attire
which
the king
himself has worn
and a horse
on which
the king
himself has
ridden
– one bearing
the royal
insignia!
\VS{9}Then let this clothing
and this horse
be given
to one
of the king’s
noble
officials.
Let him then clothe
the man
whom
the king
wishes
to honor,
and let him lead
him about through the plaza
of the city
on
the horse,
calling
before
him, ‘So
shall it be done
to the man
whom
the king
wishes
to honor!’ ”
\par }{\PP \VS{10}The king
then said
to Haman,
“Go quickly! Take
the
clothing
and the
horse,
just
as you have described,
and do
as you just indicated
to Mordecai
the Jew
who sits
at the king’s
gate.
Don’t
neglect a single
thing
of all
that
you have said.”
\par }{\PP \VS{11}So
Haman
took
the clothing
and the
horse,
and he clothed
Mordecai.
He led him about on
the horse
throughout the plaza
of the city,
calling
before
him, “So shall
it be done
to the man
whom
the king
wishes
to honor!”
\par }{\PP \VS{12}Then Mordecai
again
sat at the king’s
gate,
while Haman
hurried
away to his home,
mournful
and with a veil
over his head.
\VS{13}Haman
then related
to his wife
Zeresh
and to all
his friends
everything
that had
happened
to him. These wise
men, along with his wife
Zeresh,
said
to him, “If
indeed this Mordecai
before
whom
you have begun
to fall
is Jewish, you will not
prevail
against him. No, you will surely
fall
before him!”
\par }{\PP \VS{14}While they were still
speaking
with
him, the king’s
eunuchs
arrived.
They quickly
brought
Haman
to
the banquet
that Esther
had
prepared.

\par }\Chap{7}{\PP \VerseOne{1}So the king
and Haman
came
to dine
with
Queen
Esther.
\VS{2}On the second
day
of the banquet
of wine
the king
asked Esther,
“What
is your request,
Queen
Esther? It shall be granted
to you. And what
is your petition? Ask up to
half
the kingdom,
and it shall be done!”
\par }{\PP \VS{3}Queen
Esther
replied,
“If
I have met
with your approval,
O king,
and if
the king
is so inclined,
grant
me my life
as my request,
and my people
as my petition.
\VS{4}For
we have been sold –
both I
and my people
– to destruction
and to slaughter
and to annihilation! If
we had simply been sold
as male and female
slaves,
I would have remained silent,
for
such distress
would not
have been sufficient
for troubling
the king.”
\par }{\PP \VS{5}Then King
Ahasuerus
responded
to Queen
Esther,
“Who
is this
individual? Where
is this
person to be found who is presumptuous
enough to act
in this way?”
\par }{\PP \VS{6}Esther
replied,
“The oppressor
and enemy
is this
evil
Haman!”
\par }{\PP Then Haman
became terrified
in the presence
of the king
and queen.
\VS{7}In rage
the king
arose
from the banquet
of wine
and withdrew to
the palace
garden.
Meanwhile, Haman
stood
to beg
Queen
Esther
for his life,
for
he realized
that
the king
had now determined a catastrophic
end for
him.
\par }{\PP \VS{8}When the king
returned
from the palace
garden
to
the banquet
of wine,
Haman
was throwing
himself down
on
the couch
where
Esther
was lying. The king
exclaimed,
“Will he also
attempt to
rape
the
queen
while
I am still
in the building!”
\par }{\PP As these words
left
the king’s
mouth,
they covered
Haman’s
face.
\VS{9}Harbona,
one
of the king’s
eunuchs,
said,
“Indeed,
there is
the gallows
that
Haman
made
for Mordecai,
who
spoke
out
in the king’s
behalf.
It stands
near Haman’s
home
and is seventy-five feet
high.”
\par }{\PP The king
said,
“Hang
him on it!”
\VS{10}So they hanged
Haman
on
the very gallows
that he had
prepared
for Mordecai.
The king’s
rage
then abated.

\par }\Chap{8}{\PP \VerseOne{1}On that same
day
King
Ahasuerus
gave
the estate
of Haman,
that adversary
of the Jews,
to Queen
Esther.
Now Mordecai
had come
before
the king,
for
Esther
had revealed how
he was
related to her.
\VS{2}The
king
then removed
his signet ring
(the very one he had
taken
back from Haman) and gave
it
to Mordecai.
And Esther
designated
Mordecai
to be in charge
of Haman’s
estate.
\par }{\PP \VS{3}Then Esther
again
spoke
with the king,
falling
at his feet.
She wept
and begged
him for mercy, that he might nullify
the evil
of Haman
the Agagite
which
he had intended
against
the Jews.
\VS{4}When the king
extended
to Esther
the gold
scepter,
she
arose
and stood
before
the king.
\par }{\PP \VS{5}She said,
“If
the king
is so inclined and if
I have met
with his approval
and if the matter
is agreeable
to the king
and if I am
attractive to him, let an edict be written
rescinding those recorded
intentions
of Haman
the son
of Hammedatha,
the Agagite,
which
he wrote
in order to destroy
the Jews
who
are throughout all
the king’s
provinces.
\VS{6}For
how
can
I watch
the calamity
that
will befall
my people,
and how
can
I watch
the destruction
of my relatives?”
\par }{\PP \VS{7}King
Ahasuerus
replied
to Queen
Esther
and to Mordecai
the Jew,
“Look,
I have already given
Haman’s
estate
to Esther,
and he has been hanged
on
the gallows
because
he took hostile
action
against the Jews.
\VS{8}Now you
write
in
the king’s
name
whatever
in your opinion
is appropriate
concerning the Jews
and seal
it with the king’s
signet ring.
Any decree
that
is written
in the king’s
name
and sealed
with the king’s
signet ring
cannot
be rescinded.
\par }{\PP \VS{9}The king’s
scribes
were quickly
summoned
– in the third
month
(that is,
the month
of Sivan), on the twenty-third
day. They wrote out
everything
that
Mordecai
instructed
to
the Jews
and to
the satraps
and the governors
and the officials
of the provinces
all the way from
India
to
Ethiopia –
a hundred
and twenty-seven
provinces in all – to each province in its own script and to each people in their own language, and to the Jews according to their own script and their own language.
\VS{10}Mordecai wrote
in the name
of King
Ahasuerus
and sealed
it with the king’s
signet ring.
He then sent
letters
by couriers
on horses,
who rode
royal
horses that were very swift.
\par }{\PP \VS{11}The king
thereby allowed
the Jews
who were in
every
city
to assemble
and to stand up
for
themselves
– to destroy,
to kill,
and to annihilate
any
army
of whatever people
or province
that should become their adversaries,
including their women
and children,
and to confiscate their property.
\VS{12}This
was to take place on a certain
day
throughout all
the provinces
of King
Ahasuerus
– namely, on the thirteenth
day of the twelfth
month
(that is,
the month
of Adar).
\VS{13}A copy
of the edict
was to be presented
as law
throughout each
and every
province
and made known
to all
peoples,
so that the Jews
might be prepared
on that day
to avenge
themselves from their enemies.
\par }{\PP \VS{14}The couriers
who were riding
the royal horses
went forth
with the king’s
edict
without delay.
And the law
was presented
in Susa
the citadel as well.
\par }{\PP \VS{15}Now Mordecai
went out
from the king’s
presence
in purple
and white
royal
attire,
with a large
golden
crown
and a purple
linen
mantle.
The city
of Susa
shouted
with joy.
\VS{16}For the Jews
there was
radiant
happiness
and joyous
honor.
\VS{17}Throughout every
province
and throughout every
city
where
the king’s
edict
and his law
arrived,
the Jews
experienced happiness
and joy,
banquets
and holidays. Many
of the resident peoples
pretended to be Jews,
because
the fear
of the Jews
had overcome
them.

\par }\Chap{9}{\PP \VerseOne{1}In the twelfth
month
(that is,
the month
of Adar), on its thirteenth
day,
the edict
of the king
and his law
were to be executed.
It was on this day
that
the enemies
of the Jews
had supposed
that
they would gain power over them. But contrary to expectations,
the Jews
gained power over their
enemies.
\VS{2}The Jews
assembled
themselves in their cities
throughout all
the provinces
of King
Ahasuerus
to strike out against those who were seeking
their harm.
No
one
was able to stand
before
them, for
dread
of them fell
on
all
the peoples.
\VS{3}All
the officials
of the provinces,
the satraps,
the governors
and those who performed
the king’s
business
were assisting
the
Jews,
for
the dread
of Mordecai
had fallen
on them.
\VS{4}Mordecai
was of high rank
in the king’s
palace,
and word about him
was spreading
throughout all
the provinces.
His influence continued to become greater
and greater.
\par }{\PP \VS{5}The Jews
struck
all
their enemies
with the sword,
bringing death
and destruction,
and they did
as they pleased
with their enemies.
\VS{6}In Susa
the citadel
the Jews
killed
and destroyed
five
hundred
men.
\VS{7}In addition, they also killed Parshandatha,
Dalphon,
Aspatha,
\VS{8}Poratha,
Adalia,
Aridatha,
\VS{9}Parmashta,
Arisai,
Aridai,
and Vaizatha,
\VS{10}the ten
sons
of Haman
son
of Hammedatha,
the enemy
of the Jews.
But they did not
confiscate
their property.
\par }{\PP \VS{11}On that same day
the number
of those killed
in Susa
the citadel
was brought
to the king’s attention.
\VS{12}Then the king
said
to Queen
Esther,
“In Susa
the citadel
the Jews
have killed
and destroyed
five
hundred
men
and the ten
sons
of Haman! What
then have they done
in the rest
of the king’s
provinces? What
is your request? It shall be given
to you. What
other
petition
do you have? It shall be done.”
\par }{\PP \VS{13}Esther
replied,
“If
the king
is
so
inclined,
let
the Jews
who are in Susa
be permitted to act
tomorrow
also
according to today’s law,
and let them
hang
the ten
sons
of Haman
on
the gallows.”
\par }{\PP \VS{14}So the king
issued orders for this
to be done.
A law
was passed in Susa,
and the ten
sons
of Haman
were hanged.
\VS{15}The Jews
who were in Susa
then assembled
on the fourteenth
day
of the month
of Adar,
and they killed
three
hundred
men
in Susa.
But they did not
confiscate
their property.
\par }{\PP \VS{16}The rest
of the Jews
who were throughout the provinces
of the king
assembled
in order to stand up
for themselves
and to have rest
from their enemies.
They killed
seventy-five
thousand
of their adversaries,
but they did not
confiscate
their property.
\VS{17}All of this
happened on the thirteenth
day
of the month
of Adar.
They then rested
on the fourteenth
day and made
it a day
for banqueting
and happiness.
\par }{\SH The Origins of the Feast of Purim
\par }{\PP \VS{18}But the Jews
who were in Susa
assembled
on the thirteenth
and fourteenth
days, and rested
on the fifteenth,
making it
a day
for banqueting
and happiness.
\VS{19}This is why
the Jews
who are in the rural country – those who live in rural cities – set aside the fourteenth day of the month of Adar as a holiday for happiness, banqueting, holiday, and sending gifts to one another.
\par }{\PP \VS{20}Mordecai
wrote
these
matters
down and sent
letters
to
all
the Jews
who were throughout all
the provinces
of King
Ahasuerus,
both near
and far,
\VS{21}to have
them
observe
the fourteenth
and the fifteenth
day
of the month
of Adar
each year
\VS{22}as the time
when
the Jews
gave themselves rest
from their enemies
– the month
when
their
trouble
was turned
to happiness
and their mourning
to a holiday.
These were to be
days
of banqueting,
happiness,
sending
gifts
to one
another,
and providing
for the poor.
\par }{\PP \VS{23}So the Jews
committed themselves to continue what they had
begun
to do
and to what
Mordecai
had
written
to them.
\VS{24}For
Haman
the son
of Hammedatha,
the Agagite,
the enemy
of all
the Jews,
had devised
plans against
the Jews
to destroy
them. He had cast
{\IT{
pur}}
(that
is, the lot) in order to afflict
and destroy them.
\VS{25}But when the matter came
to the king’s attention, the king
gave
written orders
that Haman’s
evil
intentions
that he had
devised
against
the
Jews
should fall on
his own head.
He and his sons
were hanged
on
the gallows.
\VS{26}For
this reason
these
days
are known
as
{\IT{
Purim}}, after
the name
of
{\IT{
pur}}.
\VS{27}Therefore, because of the account found in this letter and what they had faced in this regard and what had happened to them, the Jews
established
as binding
on themselves, their descendants,
and all
who joined their company
that they should observe
these
two
days
without
fail,
just as written
and at the appropriate time
on an annual basis.
\VS{28}These
days
were to be remembered
and to be celebrated
in every
generation
and in every family,
every
province,
and every city.
The Jews
were not to fail
to observe
these
days
of Purim;
the remembrance
of them was not to cease
among
their descendants.
\par }{\PP \VS{29}So Queen
Esther,
the daughter
of Abihail,
and Mordecai
the Jew
wrote
with
full
authority
to confirm
this
second
letter
about Purim.
\VS{30}Letters
were sent
to
all
the Jews
in the hundred
and twenty-seven
provinces
of the empire
of Ahasuerus
– words
of true
peace –
\VS{31}to establish
these
days
of Purim
in their proper times,
just
as Mordecai
the Jew
and Queen
Esther
had
established,
and just as
they had
established both for themselves
and their descendants,
matters
pertaining to fasting
and lamentation.
\VS{32}Esther’s
command
established
these matters
of Purim,
and the matter
was officially
recorded.

\par }\Chap{10}{\PP \VerseOne{1}King
Ahasuerus
then imposed
forced labor
on
the land
and on the coastlands
of the sea.
\VS{2}Now all
the actions
carried out under his authority
and his great achievements,
along with an exact statement
concerning the greatness
of Mordecai,
whom
the king
promoted,
are they
not
written
in the Book
of the Chronicles
of the Kings
of Media
and Persia?
\VS{3}Mordecai
the Jew
was second
only to King
Ahasuerus.
He was the highest-ranking
Jew,
and he was admired by his numerous
relatives.
He worked enthusiastically
for
the good
of his people
and was an advocate
for
the welfare
of all
his descendants.
\par }