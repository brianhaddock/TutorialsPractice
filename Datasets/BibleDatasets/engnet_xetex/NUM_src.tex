\NormalFont\ShortTitle{Numbers}
{\MT Numbers

\par }\ChapOne{1}{\SH Organizing the Census of the Israelites
\par }{\PP \VerseOne{1}Now
the {\ND{Lord}}
spoke to
Moses
in the tent
of meeting
in the wilderness
of Sinai
on the first
day of the second
month
of the second
year
after
the Israelites departed
from the land
of Egypt.
He said:
\VS{2}“Take
a census
of the entire
Israelite
community
by their clans
and families,
counting
the name
of every
individual
male.
\VS{3}You
and Aaron
are to number
all
in Israel
who can serve
in the army,
those who are
twenty
years
old or older,
by their divisions.
\VS{4}And to help you there is to be
a man
from each
tribe,
each man
the head
of his family.
\VS{5}Now these
are the names
of the men
who
are to help you:
\par }{\Q from Reuben,
Elizur
son
of Shedeur;
\par }{\Q \VS{6}from Simeon,
Shelumiel
son
of Zurishaddai;
\par }{\Q \VS{7}from Judah,
Nahshon
son
of Amminadab;
\par }{\Q \VS{8}from Issachar,
Nethanel
son
of Zuar;
\par }{\Q \VS{9}from Zebulun,
Eliab
son
of Helon;
\par }{\Q \VS{10}from the sons
of Joseph:
\par }{\Q from Ephraim,
Elishama
son
of Ammihud;
\par }{\Q from Manasseh,
Gamaliel
son
of Pedahzur;
\par }{\Q \VS{11}from Benjamin,
Abidan
son
of Gideoni;
\par }{\Q \VS{12}from Dan,
Ahiezer
son
of Ammishaddai;
\par }{\Q \VS{13}from Asher,
Pagiel
son
of Ocran;
\par }{\Q \VS{14}from Gad,
Eliasaph
son
of Deuel;
\par }{\Q \VS{15}from Naphtali,
Ahira
son
of Enan.”
\par }{\SH The Census of the Tribes
\par }{\PP \VS{16}These
were the ones chosen
from the community,
leaders
of their ancestral
tribes.
They were the heads
of the thousands
of Israel.
\par }{\PP \VS{17}So
Moses
and Aaron
took
these
men
who had
been mentioned specifically
by name,
\VS{18}and they assembled
the entire
community
together on the first
day of the second
month.
Then the people recorded their ancestry
by
their clans
and families,
and the men
who were twenty
years
old or older
were listed
by name
individually,
\VS{19}just
as the
{\ND{Lord}}
had commanded
Moses.
And so he numbered
them in the wilderness
of Sinai.
\par }{\PP \VS{20}And they were as follows:
\par }{\PP The descendants
of Reuben,
the firstborn son
of Israel: According to the records
of their clans
and families,
all
the males
twenty
years
old or older
who could serve
in the army
were listed
by name
individually.
\VS{21}Those of them who were numbered
from the tribe
of Reuben
were 46,500.
\par }{\PP \VS{22}From the descendants
of Simeon: According to the records
of their clans
and families,
all
the males
numbered
of them twenty
years
old or older
who could serve
in the army
were listed
by name
individually.
\VS{23}Those of them who were numbered
from the tribe
of Simeon
were 59,300.
\VS{24}\par }{\PP From the descendants
of Gad: According to the records
of their clans
and families,
all
the males
twenty
years
old or older
who could serve
in the army
were listed
by name.
\VS{25}Those of them who were numbered
from the tribe
of Gad
were 45,650.
\par }{\PP \VS{26}From the descendants
of Judah: According to the records
of their clans
and families,
all
the males
twenty
years
old or older
who could serve
in the army
were listed
by name.
\VS{27}Those of them who were numbered
from the tribe
of Judah
were 74,600.
\par }{\PP \VS{28}From the descendants
of Issachar: According to the records
of their clans
and families,
all
the males
twenty
years
old or older
who could serve
in the army
were listed
by name.
\VS{29}Those of them who were numbered
from the tribe
of Issachar
were 54,400.
\par }{\PP \VS{30}From the descendants
of Zebulun: According to the records
of their clans
and families,
all
the males
twenty
years
old or older
who could serve
in the army
were listed
by name.
\VS{31}Those of them who were numbered
from the tribe
of Zebulun
were 57,400.
\par }{\PP \VS{32}From the sons
of Joseph:
\par }{\PP From the descendants
of Ephraim: According to the records
of their clans
and families,
all
the males
twenty
years
old or older
who could serve
in the army
were listed
by name.
\VS{33}Those of them who were numbered
from the tribe
of Ephraim
were 40,500.
\VS{34}From the descendants
of Manasseh: According to the records
of their clans
and families,
all
the males
twenty
years
old or older
who could serve
in the army
were listed
by name.
\VS{35}Those of them who were numbered
from the tribe
of Manasseh
were 32,200.
\par }{\PP \VS{36}From the descendants
of Benjamin: According to the records
of their clans
and families,
all
the males
twenty
years
old or older
who could serve
in the army
were listed
by name.
\VS{37}Those of them who were numbered
from the tribe
of Benjamin
were 35,400.
\par }{\PP \VS{38}From the descendants
of Dan: According to the records
of their clans
and families,
all
the males
twenty
years
old or older
who could serve
in the army
were listed
by name.
\VS{39}Those of them who were numbered
from the tribe
of Dan
were 62,700.
\par }{\PP \VS{40}From the descendants
of Asher: According to the records
of their clans
and families,
all
the males
twenty
years
old or older
who could serve
in the army
were listed
by name.
\VS{41}Those of them who were numbered
from the tribe
of Asher
were 41,500.
\par }{\PP \VS{42}From the descendants
of Naphtali: According to the records
of their clans
and families,
all
the males
twenty
years
old or older
who could serve
in the army
were listed
by name.
\VS{43}Those of them who were numbered
from the tribe
of Naphtali
were 53,400.
\par }{\PP \VS{44}These
were the men
whom
Moses
and Aaron
numbered
along with the twelve
leaders
of Israel,
each
of whom was
from his own family.
\VS{45}All
the Israelites
who were twenty
years
old or older,
who could serve
in Israel’s
army,
were numbered
according to their families.
\VS{46}And all
those numbered
totaled 603,550.
\par }{\SH The Exemption of the Levites
\par }{\PP \VS{47}But the Levites,
according to the tribe
of their fathers,
were not
numbered
among them.
\VS{48}The
{\ND{Lord}}
had said
to
Moses,
\VS{49}“Only
the tribe
of Levi
you must not
number
or
count
with the other
Israelites.
\VS{50}But appoint
the
Levites
over
the tabernacle
of the testimony,
over
all
its furnishings
and over
everything
in it.
They
must carry
the
tabernacle
and all
its furnishings;
and they
must attend
to it and camp
around
it.
\VS{51}Whenever the tabernacle
is to move,
the Levites
must take it down,
and whenever the tabernacle
is to be reassembled,
the Levites
must set it up.
Any unauthorized person
who approaches
it must be killed.
\par }{\PP \VS{52}“The Israelites
will camp
according to their divisions,
each man
in his camp,
and each man
by his standard.
\VS{53}But the Levites
must camp
around
the tabernacle
of the testimony,
so that the
{\ND{Lord}}’s anger
will not
fall on
the Israelite
community.
The Levites
are responsible
for the care
of the tabernacle
of the testimony.”
\par }{\PP \VS{54}The Israelites
did
according to all
that
the {\ND{Lord}}
commanded
Moses –
that is what
they did.

\par }\Chap{2}{\PP \VerseOne{1}The
{\ND{Lord}}
spoke
to
Moses
and to
Aaron:
\VS{2}“Every one
of the Israelites
must camp under his standard
with the emblems
of his family;
they must camp
at some distance
around
the tent
of meeting.
\par }{\SH The Tribes on the East
\par }{\PP \VS{3}“Now those who will be camping
on
the east,
toward the sunrise,
are the divisions
of the camp
of Judah
under their standard.
The leader
of the people of Judah
is Nahshon
son
of Amminadab.
\VS{4}Those numbered
in his division
are 74,600.
\VS{5}Those who will be camping
next to them are the tribe
of Issachar.
The leader
of the people of Issachar
is Nethanel
son
of Zuar.
\VS{6}Those numbered
in his division
are 54,400.
\VS{7}Next will be the tribe
of Zebulun.
The leader
of the people of Zebulun
is Eliab
son
of Helon.
\VS{8}Those numbered
in his division
are 57,400.
\VS{9}All
those numbered
of the camp
of Judah,
according to their divisions,
are 186,400.
They will travel
at the front.
\par }{\SH The Tribes on the South
\par }{\PP \VS{10}“On the south
will be the divisions
of the camp
of Reuben
under their standard.
The leader
of the people of Reuben
is Elizur
son
of Shedeur.
\VS{11}Those numbered
in his division
are 46,500.
\VS{12}Those
who will be camping
next to them are the tribe
of Simeon.
The leader
of the people of Simeon
is Shelumiel
son
of Zurishaddai.
\VS{13}Those numbered
in his division
are 59,300.
\VS{14}Next will be the tribe
of Gad.
The leader
of the people of Gad
is Eliasaph
son
of Deuel.
\VS{15}Those numbered
in his division
are 45,650.
\VS{16}All
those numbered
of the camp
of Reuben,
according to their divisions,
are 151,450.
They will travel
second.
\par }{\SH The Tribe in the Center
\par }{\PP \VS{17}“Then
the tent
of meeting
with the camp
of the Levites
will travel
in the middle
of the camps.
They will travel
in the same
order as
they camped,
each
in his own place under
his standard.
\par }{\SH The Tribes on the West
\par }{\PP \VS{18}“On the west
will be the divisions
of the camp
of Ephraim
under their standard.
The leader
of the people of Ephraim
is Elishama
son
of Amihud.
\VS{19}Those numbered
in his division
are 40,500.
\VS{20}Next to them will be the tribe
of Manasseh.
The leader
of the people of Manasseh
is Gamaliel
son
of Pedahzur.
\VS{21}Those numbered
in his division
are 32,200.
\VS{22}Next will be the tribe
of Benjamin.
The leader
of the people of Benjamin
is Abidan
son
of Gideoni.
\VS{23}Those numbered
in his division
are 35,400.
\VS{24}All
those numbered
of the camp
of Ephraim,
according to their divisions,
are 108,100.
They will travel
third.
\par }{\SH The Tribes on the North
\par }{\PP \VS{25}“On the north
will be the divisions
of the camp
of Dan,
under their standards.
The leader
of the people of Dan
is Ahiezer
son
of Ammishaddai.
\VS{26}Those numbered
in his division
are 62,700.
\VS{27}Those
who will be camping
next to them are the tribe
of Asher.
The leader
of the people of Asher
is Pagiel
son
of Ocran.
\VS{28}Those numbered
in his division
are 41,500.
\VS{29}Next will be the tribe
of Naphtali.
The leader
of the people of Naphtali
is Ahira
son
of Enan.
\VS{30}Those numbered
in his division
are 53,400.
\VS{31}All
those numbered
of the camp
of Dan
are 157,600.
They will travel
last,
under their standards.”
\par }{\SH Summary
\par }{\PP \VS{32}These
are the Israelites,
numbered
according to their families.
All
those numbered
in the camps,
by their divisions,
are 603,550.
\VS{33}But the Levites
were not
numbered
among
the other Israelites,
as
the
{\ND{Lord}}
commanded
Moses.
\par }{\PP \VS{34}So
the Israelites
did
according to all
that
the {\ND{Lord}}
commanded
Moses;
that is the way
they camped
under their standards,
and that is the way
they traveled,
each
with his clan
and family.


\par }\Chap{3}{\PP \VerseOne{1}Now these
are the records
of Aaron
and Moses
when
the {\ND{Lord}}
spoke
with
Moses
on Mount
Sinai.
\VS{2}These
are the names
of the sons
of Aaron: Nadab,
the firstborn,
and Abihu,
Eleazar,
and Ithamar.
\VS{3}These
are the names
of the sons
of Aaron,
the anointed
priests,
whom
he consecrated
to minister as priests.
\par }{\PP \VS{4}Nadab
and Abihu
died
before
the {\ND{Lord}}
when they offered
strange
fire
before
the {\ND{Lord}}
in the wilderness
of Sinai,
and they had
no
children.
So Eleazar
and Ithamar
ministered as priests
in
the presence
of Aaron
their father.
\par }{\SH The Assignment of the Levites
\par }{\PP \VS{5}The
{\ND{Lord}}
spoke
to
Moses:
\VS{6}“Bring
the tribe
of Levi
near,
and present
them
before
Aaron
the priest,
that they may serve him.
\VS{7}They are responsible
for his needs
and the
needs
of the whole
community
before
the tent
of meeting,
by attending
to the service
of the tabernacle.
\VS{8}And they are responsible
for all
the furnishings
of the tent
of meeting,
and for the needs
of the Israelites,
as they serve
in the
tabernacle.
\VS{9}You are to assign
the Levites
to Aaron
and his sons;
they will
be assigned
exclusively
to him out of all the Israelites.
\VS{10}So you are to appoint
Aaron
and his sons,
and they will be responsible
for their priesthood;
but the unauthorized
person who comes near
must be put to death.”
\par }{\PP \VS{11}Then the
{\ND{Lord}}
spoke
to
Moses:
\VS{12}“Look,
I myself
have taken
the Levites
from among
the Israelites
instead
of every
firstborn
who opens
the womb
among the Israelites.
So
the Levites belong to me,
\VS{13}because
all
the firstborn
are mine. When
I destroyed
all
the firstborn
in the land
of Egypt,
I set apart
for myself all
the firstborn
in Israel,
both man
and beast.
They belong
to me. I am
the {\ND{Lord}}.”
\par }{\SH The Numbering of the Levites
\par }{\PP \VS{14}Then
the {\ND{Lord}}
spoke
to
Moses
in the wilderness
of Sinai:
\VS{15}“Number
the Levites
by their clans
and their families;
every
male
from a month
old
and upward
you are to number.”
\VS{16}So Moses
numbered
them according to the word of the
{\ND{Lord}}, just
as he had been commanded.
\par }{\SH The Summary of Families
\par }{\PP \VS{17}These
were the sons
of Levi
by their names: Gershon,
Kohath,
and Merari.
\par }{\PP \VS{18}These
are the names
of the sons
of Gershon
by their families: Libni
and Shimei.
\VS{19}The sons
of Kohath
by their families
were: Amram,
Izhar,
Hebron,
and Uzziel.
\VS{20}The sons
of Merari
by their families
were Mahli
and Mushi.
These
are the families
of the Levites
by their clans.
\par }{\SH The Numbering of the Gershonites
\par }{\PP \VS{21}From Gershon
came the family
of the Libnites
and the family
of the Shimeites;
these
were
the families
of the Gershonites.
\VS{22}Those of them who were numbered,
counting
every
male
from a month
old and upward,
were 7,500.
\VS{23}The families
of the Gershonites
were to camp
behind
the tabernacle
toward the west.
\VS{24}Now the leader
of the clan
of the Gershonites
was Eliasaph
son
of Lael.
\par }{\PP \VS{25}And the responsibilities
of the Gershonites
in the tent
of meeting
included the tabernacle,
the tent
with its covering,
the curtain
at the entrance
of the tent
of meeting,
\VS{26}the hangings
of the courtyard,
the curtain
at the entrance
to the courtyard
that
surrounded
the
tabernacle
and the altar,
and their ropes,
plus all
the service
connected with these things.
\par }{\SH The Numbering of the Kohathites
\par }{\PP \VS{27}From Kohath
came the family
of the Amramites,
the family
of the Izharites,
the family
of the Hebronites,
and the family
of the Uzzielites;
these
were
the families
of the Kohathites.
\VS{28}Counting
every
male
from a month
old and upward,
there were 8,600.
They were responsible
for the care
of the sanctuary.
\VS{29}The families
of the Kohathites
were to camp
on
the south
side
of the tabernacle.
\VS{30}Now the leader
of the clan
of the families
of the Kohathites
was Elizaphan
son
of Uzziel.
\par }{\PP \VS{31}Their responsibilities
included the ark,
the table,
the lampstand,
the altars,
and the utensils
of the sanctuary
with which
they ministered,
the curtain,
and all
their service.
\VS{32}Now the head of all the Levitical
leaders
was Eleazar
son
of Aaron
the priest.
He was appointed over
those who
were responsible
for the sanctuary.
\par }{\SH The Numbering of Merari
\par }{\PP \VS{33}From Merari
came the family
of the Mahlites
and the family
of the Mushites;
these
were
the families
of Merari.
\VS{34}Those of them who were numbered,
counting
every
male
from a month
old
and upward,
were 6,200.
\VS{35}Now the leader
of the clan
of the families
of Merari
was Zuriel
son
of Abihail.
These were to camp
on
the north
side
of the tabernacle.
\par }{\PP \VS{36}The appointed
responsibilities
of the Merarites
included the frames
of the tabernacle,
its crossbars,
its posts,
its sockets,
its utensils,
plus all
the service connected with these things,
\VS{37}and the pillars
of the courtyard
all around,
with their sockets,
their pegs,
and their ropes.
\par }{\PP \VS{38}But those who were to camp
in front
of the tabernacle
on the east,
in front
of the tent
of meeting,
were Moses,
Aaron,
and his sons.
They were responsible
for the needs
of the sanctuary
and for the needs
of the Israelites,
but the unauthorized
person who approached
was to be put to death.
\VS{39}All
who were numbered
of the Levites,
whom
Moses
and Aaron
numbered
by
the word
of the {\ND{Lord}}, according to their families,
every
male
from a month
old and upward,
were 22,000.
\par }{\SH The Substitution for the Firstborn
\par }{\PP \VS{40}Then the
{\ND{Lord}}
said to
Moses,
“Number
all
the firstborn
males
of the Israelites
from a month
old and upward,
and take
the number
of their names.
\VS{41}And take
the
Levites for me – I am the
{\ND{Lord}} – instead of all the firstborn males among the Israelites, and the livestock of the Levites instead of all the firstborn of the livestock of the Israelites.”
\VS{42}So Moses
numbered
all
the firstborn males
among the Israelites,
as
the {\ND{Lord}}
had commanded him.
\VS{43}And all
the firstborn
males,
by the number
of the names
from a month
old and upward,
totaled 22,273.
\par }{\PP \VS{44}Then the
{\ND{Lord}}
spoke
to
Moses:
\VS{45}“Take
the
Levites
instead
of all
the firstborn males
among the Israelites,
and the
livestock
of the Levites
instead
of their livestock.
And the Levites
will be
mine. I am
the {\ND{Lord}}.
\VS{46}And for the redemption
of the 273
firstborn males
of the Israelites
who
exceed
the number of the Levites,
\VS{47}collect
five
shekels
for each one individually;
you are to collect
this amount in the currency of the sanctuary
shekel
(this shekel
is twenty
gerahs).
\VS{48}And give
the money
for the redemption
of the excess
number of them to Aaron
and his sons.”
\par }{\PP \VS{49}So
Moses
took
the redemption
money
from those who
were in excess
of those redeemed
by the Levites.
\VS{50}From the firstborn males
of the Israelites
he collected
the money,
1,365
shekels,
according to the sanctuary shekel.
\VS{51}Moses
gave
the redemption
money
to Aaron
and his sons,
according
to the word of the
{\ND{Lord}}, as
the {\ND{Lord}}
had
commanded
Moses.


\par }\Chap{4}{\PP \VerseOne{1}Then the
{\ND{Lord}}
spoke
to
Moses
and Aaron:
\VS{2}“Take
a census
of the Kohathites
from among
the Levites,
by their families
and by their clans,
\VS{3}from thirty
years
old and upward
to
fifty
years
old, all
who enter
the company
to do
the work
in the tent
of meeting.
\VS{4}This
is the service
of the Kohathites
in the tent
of meeting,
relating to the most
holy things.
\VS{5}When it is time for the camp
to journey,
Aaron
and his sons
must come
and take down
the
screening
curtain
and cover
the
ark
of the testimony with it.
\VS{6}Then they must put
over
it a covering
of fine
leather
and spread
over that a cloth
entirely
of blue,
and then they must insert
its poles.
\par }{\PP \VS{7}“On
the table
of the presence
they must spread
a blue
cloth,
and put
on
it the dishes,
the pans,
the bowls,
and the
pitchers
for pouring,
and the Bread
of the Presence
must be
on
it continually.
\VS{8}They must spread
over
them a scarlet
cloth,
and cover
the same with a covering
of fine
leather;
and they must insert
its poles.
\par }{\PP \VS{9}“They must take
a blue
cloth
and cover
the lampstand
of the light,
with its lamps,
its wick-trimmers,
its trays,
and all
its oil
vessels,
with which
they service it.
\VS{10}Then they must put
it with all
its utensils
in a covering
of fine
leather,
and put
it on
a carrying beam.
\par }{\PP \VS{11}“They must spread
a blue
cloth
on
the gold
altar,
and cover
it with a covering
of fine
leather;
and they must insert
its poles.
\VS{12}Then they must take
all
the utensils
of the service,
with which
they serve
in the sanctuary,
put
them in
a blue
cloth,
cover
them with a covering
of fine
leather,
and put
them on
a carrying beam.
\VS{13}Also, they must take away the ashes
from the altar
and spread
a purple
cloth
over it.
\VS{14}Then they must place
on
it all
its implements with
which
they serve
there – the trays, the meat forks, the shovels, the basins, and all the utensils of the altar – and they must spread on it a covering of fine leather, and then insert its poles.
\par }{\PP \VS{15}“When Aaron
and his sons
have finished
covering
the
sanctuary
and all
the furnishings
of the sanctuary,
when the camp
is ready
to journey,
then
the Kohathites
will come
to carry
them; but they must not
touch
any holy thing,
or they will die.
These
are the responsibilities
of the Kohathites
with the tent
of meeting.
\par }{\PP \VS{16}“The appointed responsibility
of Eleazar
son
of Aaron
the priest
is for the oil
for the light,
and the spiced
incense,
and the daily
grain offering,
and the anointing
oil;
he also has the appointed responsibility
over all
the tabernacle
with all
that
is in it, over the sanctuary
and over all its furnishings.”
\par }{\PP \VS{17}Then the
{\ND{Lord}}
spoke
to
Moses
and Aaron:
\VS{18}“Do not
allow the tribe
of the families
of the Kohathites
to be cut off from among
the Levites;
\VS{19}but in order that they will live
and not
die
when they approach
the
most
holy things,
do
this for them: Aaron
and his sons
will go
in and appoint
each
man
to his service
and his responsibility.
\VS{20}But
the Kohathites are not
to go
in to watch
while
the holy
things are being covered, or they will die.”
\par }{\SH The Service of the Gershonites
\par }{\PP \VS{21}Then the
{\ND{Lord}}
spoke
to
Moses:
\VS{22}“Also
take
a census
of the Gershonites,
by their clans and by their
families.
\VS{23}You must
number them
from thirty
years
old and upward
to
fifty
years
old, all
who enter
the company
to do the work
of the tent
of meeting.
\VS{24}This
is the service
of the families
of Gershonites,
as they serve
and carry it.
\VS{25}They must carry
the
curtains
for the tabernacle
and the
tent
of meeting
with its covering,
the covering
of fine leather
that
is over
it, the curtains
for the entrance
of the tent
of meeting,
\VS{26}the
hangings
for the courtyard,
the curtain
for the entrance
of the gate
of the court,
which
is around
the
tabernacle
and the altar,
and their ropes,
along
with all
the furnishings
for their service
and everything
that
is made
for them. So they are to serve.
\par }{\PP \VS{27}“All
the service
of the Gershonites,
whether carrying loads
or for any
of their work,
will be at the direction of Aaron
and his sons.
You will assign them
all
their tasks
as their responsibility.
\VS{28}This
is the service
of the families
of the Gershonites
concerning the tent
of meeting.
Their responsibilities
will be under the authority
of Ithamar
son
of Aaron
the priest.
\par }{\SH The Service of the Merarites
\par }{\PP \VS{29}“As for the sons
of Merari,
you are to number
them by their families
and by their clans.
\VS{30}You must number them from thirty
years
old and upward
to
fifty
years
old, all
who enter
the
company
to do the work
of the tent
of meeting.
\VS{31}This
is what they are responsible
to carry
as their entire
service
in the tent
of meeting: the frames
of the tabernacle,
its crossbars,
its posts,
its sockets,
\VS{32}and the posts
of the surrounding
courtyard
with their sockets,
tent pegs,
and ropes,
along with all
their furnishings
and everything
for their service.
You are to assign
by names
the
items
that each man is responsible
to carry.
\VS{33}This
is the service
of the families
of the Merarites,
their entire
service
concerning the tent
of meeting,
under the authority
of Ithamar
son
of Aaron
the priest.”
\par }{\SH Summary
\par }{\PP \VS{34}So Moses
and Aaron
and the leaders
of the community
numbered
the Kohathites
by their families
and by clans,
\VS{35}from thirty
years
old and upward
to
fifty
years
old, everyone
who entered
the company
for the work
in the tent
of meeting;
\VS{36}and those of them numbered
by their families
were 2,750.
\VS{37}These
were those numbered
from the families
of the Kohathites,
everyone
who served
in the tent
of meeting,
whom
Moses
and Aaron
numbered
according to the word of the
{\ND{Lord}}
by the authority
of Moses.
\par }{\PP \VS{38}Those numbered
from the Gershonites,
by their families
and by their clans,
\VS{39}from thirty
years
old and upward
to
fifty
years
old, everyone
who entered
the company
for the work
in the tent
of meeting –
\VS{40}those of them numbered
by their families,
by their clans,
were 2,630.
\VS{41}These
were those numbered
from the families
of the Gershonites,
everyone
who served
in the tent
of meeting,
whom
Moses
and Aaron
numbered
according to the word of the
{\ND{Lord}}.
\par }{\PP \VS{42}Those numbered
from the families
of the Merarites,
by their families,
by their clans,
\VS{43}from thirty
years
old and upward
to
fifty
years
old, everyone
who entered
the company
for the work
in the tent
of meeting –
\VS{44}those of them numbered
by their families
were 3,200.
\VS{45}These
are those numbered
from the families
of the Merarites,
whom
Moses
and Aaron
numbered
according to the word of the
{\ND{Lord}}
by the authority
of Moses.
\par }{\PP \VS{46}All
who were numbered
of the Levites,
whom
Moses,
Aaron,
and the leaders
of Israel
numbered
by their families
and by their clans,
\VS{47}from thirty
years
old and upward
to
fifty
years
old, everyone
who entered
to do the work
of service
and the work
of carrying relating
to the tent
of meeting –
\VS{48}those of them numbered
were 8,580.
\VS{49}According
to the word
of the {\ND{Lord}}
they were numbered,
by the
authority
of Moses,
each
according
to his service
and according
to what he was to carry.
Thus were they numbered
by him, as the
{\ND{Lord}}
had
commanded
Moses.


\par }\Chap{5}{\PP \VerseOne{1}Then the
{\ND{Lord}}
spoke
to
Moses:
\VS{2}“Command
the Israelites
to expel
from
the camp
every
leper,
everyone
who has a discharge,
and whoever
becomes defiled
by a corpse.
\VS{3}You must expel both men
and women;
you must put them
outside
the camp,
so
that they will not
defile
their camps,
among
which
I
live.”
\VS{4}So
the Israelites
did
so, and expelled
them
outside
the camp.
As the
{\ND{Lord}}
had
spoken
to
Moses,
so
the Israelites
did.
\par }{\SH Restitution for Sin
\par }{\PP \VS{5}Then the
{\ND{Lord}}
spoke
to
Moses:
\VS{6}“Tell
the Israelites,
‘When a man
or
a woman
commits
any
sin
that people
commit, thereby
breaking faith
with the
{\ND{Lord}}, and that person
is
found guilty,
\VS{7}then he must confess
his sin
that
he has committed
and must make full reparation,
add
one fifth
to it, and give
it to whomever
he wronged.
\VS{8}But if
the individual
has no
close relative
to whom reparation
can be made for the wrong,
the reparation
for the wrong
must be paid to the
{\ND{Lord}}
for the priest,
in addition
to the ram
of atonement
by which
atonement
is made for him.
\VS{9}Every
offering
of all
the Israelites’
holy
things that
they bring
to the priest
will be his.
\VS{10}Every man’s
holy
things will be
his; whatever
any man
gives
the priest
will be his.’ ”
\par }{\SH The Jealousy Ordeal
\par }{\PP \VS{11}The
{\ND{Lord}}
spoke
to
Moses:
\VS{12}“Speak
to
the Israelites
and tell
them, ‘If
any man’s
wife
goes astray
and behaves
unfaithfully toward him,
\VS{13}and a man
has sexual relations
with
her without
her husband
knowing
it, and it is hidden
that she
has defiled
herself, since there was no
witness
against her, nor
was she caught –
\VS{14}and if jealous feelings come
over
him
and he becomes
suspicious
of his wife,
when
she
is defiled;
or
if
jealous feelings come
over
him
and he becomes suspicious
of his wife,
when
she
is not
defiled –
\VS{15}then
the man
must bring
his wife
to
the priest,
and he must bring
the
offering
required
for her,
one tenth
of an ephah
of barley
meal;
he must not
pour
olive oil
on
it or
put
frankincense
on
it, because
it is
a grain offering
of suspicion,
a grain offering
for remembering,
for bringing iniquity to remembrance.
\par }{\PP \VS{16}“‘Then
the priest
will bring her near and have her stand
before
the
{\ND{Lord}}.
\VS{17}The priest
will then take
holy
water
in a pottery
jar,
and take
some
of the dust
that
is on the floor
of the tabernacle,
and put
it into
the water.
\VS{18}Then
the priest
will have the
woman
stand
before
the {\ND{Lord}}, uncover
the woman’s
head,
and put
the
grain offering
for remembering
in her hands,
which is the grain offering
of suspicion.
The priest
will
hold in his hand
the bitter
water
that brings a curse.
\VS{19}Then the priest
will put the woman
under oath
and say
to
the her, “If
no
other man
has
had sexual relations
with
you, and if
you have not
gone astray
and become defiled
while under
your husband’s
authority, may you be free
from this bitter
water
that brings a curse.
\VS{20}But if
you
have gone astray
while under
your husband’s authority, and if
you have defiled
yourself and some man other than
your husband
has had sexual relations with you….”
\VS{21}Then the priest
will put
the woman
under
the oath
of the curse
and will say
to the her,
“The
{\ND{Lord}}
make
you an attested curse
among
your people,
if the
{\ND{Lord}}
makes
your thigh
fall away
and your abdomen
swell;
\VS{22}and this
water
that causes the curse
will go into
your stomach,
and make your abdomen
swell
and your thigh
rot.”
Then the woman
must say,
“Amen,
amen.”
\par }{\PP \VS{23}“‘Then the priest
will write
these
curses
on a scroll
and then scrape
them off into
the bitter
water.
\VS{24}He will make the
woman
drink
the bitter
water
that brings
a curse,
and the water
that brings a curse
will enter her to produce bitterness.
\VS{25}The priest
will take
the grain offering
of suspicion
from the woman’s
hand,
wave
the
grain offering
before
the {\ND{Lord}}, and bring
it to
the altar.
\VS{26}Then the priest
will take a handful
of the grain offering
as its memorial
portion, burn
it on the altar,
and afterward
make the
woman
drink
the water.
\VS{27}When he has made her drink
the water,
then, if
she has defiled
herself and behaved
unfaithfully
toward her husband,
the water
that brings
a curse
will enter her to produce bitterness
– her abdomen
will swell,
her thigh
will fall away,
and the woman
will become
a curse
among
her people.
\VS{28}But if
the woman
has not
defiled
herself, and is clean,
then she
will be free
of ill effects
and will be able to bear children.
\par }{\PP \VS{29}“‘This
is the law
for cases of jealousy,
when a wife,
while
under
her husband’s authority, goes astray
and defiles herself,
\VS{30}or
when jealous feelings come
over
a
man
and he becomes suspicious
of his wife;
then he must have the woman
stand
before
the {\ND{Lord}}, and the priest
will carry
out all
this
law
upon her.
\VS{31}Then the man
will be free
from iniquity,
but that woman
will bear
the consequences of her iniquity.’ ”


\par }\Chap{6}{\PP \VerseOne{1}Then the
{\ND{Lord}}
spoke
to
Moses:
\VS{2}“Speak
to
the Israelites,
and tell
them,
‘When either a man
or
a woman
takes a special
vow, to take
a vow
as a Nazirite,
to separate
himself to the
{\ND{Lord}},
\VS{3}he must separate
himself from wine
and strong drink,
he must drink
neither
vinegar
made from wine
nor vinegar
made from strong drink,
nor
may he drink
any
juice
of grapes,
nor
eat
fresh
grapes
or raisins.
\VS{4}All
the days
of his separation
he must not
eat
anything
that is produced
by the grapevine,
from seed
to skin.
\par }{\PP \VS{5}“‘All
the days
of the vow
of his separation
no
razor
may be used
on
his head
until
the time
is fulfilled
for which
he separated
himself to the
{\ND{Lord}}. He will be
holy,
and he must let the locks
of hair
on his head
grow long.
\par }{\PP \VS{6}“‘All
the days
that he separates
himself to
the {\ND{Lord}}
he must not
contact
a dead
body.
\VS{7}He must not
defile
himself even for his father
or his mother
or his brother
or his sister
if they die,
because
the separation
for his God
is on
his head.
\VS{8}All
the days
of his separation
he must be holy
to the
{\ND{Lord}}.
\par }{\SH Contingencies for Defilement
\par }{\PP \VS{9}“‘If
anyone
dies
very
suddenly
beside
him and he defiles
his consecrated
head,
then he must shave
his head
on the day
of his purification
– on
the seventh
day
he must shave it.
\VS{10}On the eighth
day
he is to bring
two
turtledoves
or
two
young
pigeons
to
the priest,
to
the entrance
to the tent
of meeting.
\VS{11}Then
the priest
will offer
one
for a purification
offering and the other
as a burnt offering,
and make atonement
for him,
because
of his transgression
in regard to the corpse.
So he must reconsecrate
his head
on that day.
\VS{12}He must rededicate
to the
{\ND{Lord}}
the
days
of his separation
and bring
a male lamb
in its first year
as a reparation
offering, but the former
days
will
not be counted
because
his separation
was defiled.
\par }{\SH Fulfilling the Vows
\par }{\PP \VS{13}“‘Now this
is the law
of the Nazirite: When
the days
of his separation
are fulfilled,
he must be brought
to
the entrance
of the tent
of meeting,
\VS{14}and he must present
his offering
to the
{\ND{Lord}}: one male lamb
in its first
year
without blemish
for a burnt offering,
one
ewe
lamb in its first
year
without blemish
for a purification
offering, one
ram
without blemish
for a peace offering,
\VS{15}and a basket
of bread made without yeast,
cakes
of fine flour
mixed
with olive oil,
wafers
made without yeast
and smeared
with olive oil,
and their grain offering
and their drink offerings.
\par }{\PP \VS{16}“‘Then the priest
must present
all these before
the {\ND{Lord}}
and offer
his purification
offering and his burnt offering.
\VS{17}Then he must offer
the ram
as a peace offering
to the
{\ND{Lord}}, with the basket
of bread
made
without yeast;
the priest
must also offer his grain offering
and his drink offering.
\par }{\PP \VS{18}“‘Then the Nazirite
must shave
his consecrated
head
at the entrance
to the tent
of meeting
and must take
the hair
from his consecrated
head
and put
it on
the fire
where
the peace offering
is burning.
\VS{19}And the priest
must
take
the boiled
shoulder of the ram,
one
cake
made without yeast
from
the basket,
and one
wafer
made without yeast,
and put
them on
the hands
of the Nazirite
after
he has shaved
his consecrated head;
\VS{20}then
the
priest
must wave
them as a wave offering
before
the {\ND{Lord}}; it is
a holy
portion for the priest,
together with the breast
of the wave offering
and the thigh
of the raised
offering. After
this the Nazirite
may drink
wine.’
\par }{\PP \VS{21}“This
is the law
of the Nazirite
who
vows
to the
{\ND{Lord}}
his offering
according
to his separation,
as well as whatever
else
he can provide. Thus he must fulfill
his vow
that
he makes,
according
to the law
of his separation.”
\par }{\SH The Priestly Benediction
\par }{\PP \VS{22}The
{\ND{Lord}}
spoke
to
Moses:
\VS{23}“Tell
Aaron
and his sons,
‘This
is the way you are to bless
the Israelites.
Say to them:
\par }{\Q \VS{24}“The
{\ND{Lord}}
bless
you and protect you;
\par }{\Q \VS{25}The
{\ND{Lord}}
make his face
to shine
upon you,
\par }{\Q and be gracious to you;
\par }{\Q \VS{26}The
{\ND{Lord}}
lift up
his countenance
upon
you

\par }{\Q and give
you peace.” ’
\par }{\PP \VS{27}So they will put
my name
on
the Israelites,
and I
will bless them.”


\par }\Chap{7}{\PP \VerseOne{1}When
Moses
had completed
setting up
the tabernacle,
he anointed
it and consecrated
it and all
its furnishings,
and he anointed
and consecrated
the altar
and all
its utensils.
\VS{2}Then the leaders
of Israel,
the heads
of their
clans,
made an offering.
They were
the leaders
of the tribes;
they were
the ones
who had been supervising
the numbering.
\VS{3}They brought
their offering
before
the {\ND{Lord}}, six
covered
carts
and twelve
oxen
– one cart
for every two
of the leaders,
and an ox
for each one;
and they presented
them in front of
the tabernacle.
\par }{\SH The Distribution of the Gifts
\par }{\PP \VS{4}Then the
{\ND{Lord}}
spoke
to Moses:
\VS{5}“Receive
these gifts from them,
that they may be
used
in doing the work
of the tent
of meeting;
and you must
give
them to
the Levites,
to every man
as his service requires.”
\par }{\PP \VS{6}So
Moses
accepted
the carts
and the oxen
and gave
them to
the Levites.
\VS{7}He gave two
carts
and four
oxen
to the Gershonites,
as their service
required;
\VS{8}and he gave four
carts
and eight
oxen
to the Merarites,
as their service
required,
under the authority
of Ithamar
son
of Aaron
the priest.
\VS{9}But to the Kohathites
he gave
none,
because
the service
of the holy things,
which they carried
on
their shoulders, was their responsibility.
\par }{\SH The Time of Presentation
\par }{\PP \VS{10}The leaders
offered
gifts for the dedication
of the altar
when
it was anointed.
And the leaders
presented
their offering
before
the altar.
\VS{11}For the
{\ND{Lord}}
said
to
Moses,
“They must present
their offering,
one
leader
for each
day,
for the dedication
of the altar.”
\par }{\SH The Tribal Offerings
\par }{\PP \VS{12}The one who presented
his offering on the first
day
was Nahshon
son
of Amminadab,
from the tribe
of Judah.
\VS{13}His offering
was one
silver
platter
weighing 130
shekels,
and one
silver
sprinkling bowl
weighing 70
shekels,
both according to the sanctuary
shekel,
each of them
full
of fine flour
mixed
with olive oil
as a grain offering;
\VS{14}one
gold
pan
weighing
10
shekels, full
of incense;
\VS{15}one
young
bull,
one ram,
and one
male lamb
in its first
year,
for a burnt offering;
\VS{16}one
male
goat
for a purification offering;
\VS{17}and for the sacrifice
of peace offerings: two
bulls,
five
rams,
five
male goats,
and five
male lambs
in their first year.
This
was the offering
of Nahshon
son
of Amminadab.
\par }{\PP \VS{18}On the second
day
Nethanel
son
of Zuar,
leader
of Issachar,
presented an offering.
\VS{19}He offered
for his offering
one
silver
platter
weighing 130
shekels
and one
silver
sprinkling bowl
weighing
70,
both according to the sanctuary
shekel,
each of them
full
of fine flour
mixed
with olive oil
as a grain offering;
\VS{20}one
gold
pan
weighing
10
shekels, full
of incense;
\VS{21}one
young
bull,
one ram,
and one
male lamb
in its first
year,
for a burnt offering;
\VS{22}one
male
goat
for a purification offering;
\VS{23}and for the sacrifice
of peace offerings: two
bulls,
five
rams,
five
male goats,
and five
male lambs
in their first year.
This
was the offering
of Nethanel
son
of Zuar.
\par }{\PP \VS{24}On
the third
day
Eliab
son
of Helon,
leader
of the Zebulunites, presented an offering.
\VS{25}His offering
was one
silver
platter
weighing 130
shekels
and one
silver
sprinkling bowl
weighing 70
shekels,
both according to the sanctuary
shekel,
each of them
full
of fine flour
mixed
with olive oil
as a grain offering;
\VS{26}one
gold
pan
weighing
10
shekels, full
of incense;
\VS{27}one
young
bull,
one ram,
and one
male lamb
in its first
year,
for a burnt offering;
\VS{28}one
male
goat
for a purification offering;
\VS{29}and for the sacrifice
of peace offerings: two
bulls,
five
rams,
five
male goats,
and five
male lambs
in their first year.
This
was the offering
of Eliab
son
of Helon.
\par }{\PP \VS{30}On the fourth
day
Elizur
son
of Shedeur,
leader
of the Reubenites, presented an offering.
\VS{31}His offering
was one
silver
platter
weighing 130
shekels
and one
silver
sprinkling bowl
weighing 70
shekels,
both according to the sanctuary
shekel,
each of them
full
of fine flour
mixed
with olive oil
as a grain offering;
\VS{32}one
gold
pan
weighing
10
shekels, full
of incense;
\VS{33}one
young
bull,
one ram,
and one
male lamb
in its first
year,
for a burnt offering;
\VS{34}one
male
goat
for a purification offering;
\VS{35}and for the sacrifice
of peace offerings: two
bulls,
five
rams,
five
male goats,
and five
lambs
in their first year.
This
was the offering
of Elizur
son
of Shedeur.
\par }{\PP \VS{36}On the fifth
day
Shelumiel
son
of Zurishaddai,
leader
of the Simeonites, presented an offering.
\VS{37}His offering
was one
silver
platter
weighing 130
shekels
and one
silver
sprinkling bowl
weighing 70
shekels,
both according to the sanctuary
shekel,
each of them
full
of fine flour
mixed
with olive oil
as a grain offering;
\VS{38}one
gold
pan
weighing
10
shekels;
\VS{39}one
young
bull,
one ram,
and one
male lamb
in its first
year,
for a burnt offering;
\VS{40}one
male
goat
for a purification offering;
\VS{41}and for the sacrifice
of peace offerings: two
bulls,
five
rams,
five
male goats,
and five
lambs
in their first year.
This
was the offering
of Sheloumiel
son
of Zurishaddai.
\par }{\PP \VS{42}On
the sixth
day
Eliasaph
son
of Deuel,
leader
of the Gadites, presented an offering.
\VS{43}His offering
was one
silver
platter
weighing 130
shekels
and one
silver
sprinkling bowl
weighing 70
shekels,
both according to the sanctuary
shekel,
each of them
full
of fine flour
mixed
with olive oil
as a grain offering;
\VS{44}one
gold
pan
weighing
10
shekels;
\VS{45}one
young
bull,
one ram,
and one
male lamb
in its first
year,
for a burnt offering;
\VS{46}one
male
goat
for a purification offering;
\VS{47}and for the sacrifice
of peace offerings: two
bulls,
five
rams,
five
male goats,
and five
lambs
in their first year.
This
was the offering
of Eliasaph
son
of Deuel.
\par }{\PP \VS{48}On the seventh
day
Elishama
son
of Ammihud,
leader
of the Ephraimites, presented an offering.
\VS{49}His offering
was one
silver
platter
weighing 130
shekels
and one
silver
sprinkling bowl
weighing 70
shekels,
both according to the sanctuary
shekel,
each of them
full
of fine flour
mixed
with olive oil
as a grain offering;
\VS{50}one
gold
pan
weighing
10
shekels, full
of incense;
\VS{51}one
young
bull,
one ram,
and one
male lamb
in its first
year,
for a burnt offering;
\VS{52}one
male
goat
for a purification offering;
\VS{53}and for the sacrifice
of peace offerings: two
bulls,
five
rams,
five
male goats,
and five
lambs
in their first year.
This
was the offering
of Elishama
son
of Ammihud.
\par }{\PP \VS{54}On the eighth
day
Gamaliel
son
of Pedahzur,
leader
of the Manassehites, presented an offering.
\VS{55}His offering
was one
silver
platter
weighing 130
shekels
and one
silver
sprinkling bowl
weighing 70
shekels,
both according to the sanctuary
shekel,
each of them
full
of fine flour
mixed
with olive oil
as a grain offering;
\VS{56}one
gold
pan
weighing
10
shekels, full
of incense;
\VS{57}one
young
bull,
one ram,
and one
male lamb
in its first
year,
for a burnt offering;
\VS{58}one
male
goat
for a purification offering;
\VS{59}and for the sacrifice
of peace offerings: two
bulls,
five
rams,
five
male goats,
and five
lambs
in their first year.
This
was the offering
of Gamaliel
son
of Pedahzur.
\par }{\PP \VS{60}On the ninth
day
Abidan
son
of Gideoni,
leader
of the Benjaminites, presented an offering.
\VS{61}His offering
was one
silver
platter
weighing 130
shekels
and one
silver
sprinkling bowl
weighing 70
shekels,
both according to the sanctuary
shekel,
each of them
full
of fine flour
mixed
with olive oil
as a grain offering;
\VS{62}one
gold
pan
weighing
10
shekels, full
of incense;
\VS{63}one
young
bull,
one ram,
and one
male lamb
in its first
year,
for a burnt offering;
\VS{64}one
male
goat
for a purification offering;
\VS{65}and for the sacrifice
of peace offerings: two
bulls,
five
rams,
five
male goats,
and five
lambs
in their first year.
This
was the offering
of Abidan
son
of Gideoni.
\par }{\PP \VS{66}On the tenth
day
Ahiezer
son
of Amishaddai,
leader
of the Danites, presented an offering.
\VS{67}His offering
was one
silver
platter
weighing 130
shekels
and one
silver
sprinkling bowl
weighing 70
shekels,
both according to the sanctuary
shekel,
each of them
full
of fine flour
mixed
with olive oil
as a grain offering;
\VS{68}one
gold
pan
weighing
10
shekels, full
of incense;
\VS{69}one
young
bull,
one ram,
and one
male lamb
in its first
year,
for a burnt offering;
\VS{70}one
male
goat
for a purification offering;
\VS{71}and for the sacrifice
of peace offerings: two
bulls,
five
rams,
five
male goats,
and five
lambs
in their first year.
This
was the offering
of Ahiezer
son
of Amishaddai.
\par }{\PP \VS{72}On
the eleventh
day
Pagiel
son
of Ocran,
leader
of the Asherites, presented an offering.
\VS{73}His offering
was one
silver
platter
weighing 130
shekels
and one
silver
sprinkling bowl
weighing 70
shekels,
both according to the sanctuary
shekel,
each of them
full
of fine flour
mixed
with olive oil
as a grain offering;
\VS{74}one
gold
pan
weighing
10
shekels, full
of incense;
\VS{75}one
young
bull,
one ram,
and one
male lamb
in its first
year,
for a burnt offering;
\VS{76}one
male
goat
for a purification offering;
\VS{77}and for the sacrifice
of peace offerings: two
bulls,
five
rams,
five
male goats,
and five
lambs
in their first year.
This
was the offering
of Pagiel
son
of Ocran.
\par }{\PP \VS{78}On
the twelfth
day
Ahira
son
of Enan,
leader
of the Naphtalites, presented an offering.
\VS{79}His offering
was one
silver
platter
weighing 130
shekels
and one
silver
sprinkling bowl
weighing 70
shekels,
both according to the sanctuary
shekel,
each of them
full
of fine flour
mixed
with olive oil
as a grain offering;
\VS{80}one
gold
pan
weighing
10
shekels;
\VS{81}one
young
bull,
one ram,
and one
male lamb
in its first
year,
for a burnt offering;
\VS{82}one
male
goat
for a purification offering;
\VS{83}and for the sacrifice
of peace offerings: two
bulls,
five
rams,
five
male goats,
and five
lambs
in their first year.
This
was the offering
of Ahira
son
of Enan.
\par }{\SH Summary
\par }{\PP \VS{84}This
was the dedication
for the altar
from the leaders
of Israel,
when
it was anointed: twelve
silver
platters,
twelve
silver
sprinkling bowls,
and twelve
gold
pans.
\VS{85}Each
silver
platter
weighed 130
shekels, and each
silver sprinkling bowl
weighed 70
shekels. All
the silver
of the vessels
weighed 2,400
shekels, according to the sanctuary
shekel.
\VS{86}The twelve
gold
pans
full
of incense
weighed 10
shekels each,
according to the sanctuary
shekel;
all
the gold
of the pans
weighed 120 shekels.
\VS{87}All
the animals
for the burnt offering
were 12
young bulls,
12 rams,
12
male lambs
in their first year,
with their grain offering,
and 12
male
goats
for a purification offering.
\VS{88}All
the animals
for the sacrifice
for the peace offering
were 24
young bulls,
60
rams,
60 male goats,
and 60
lambs
in their first year.
These
were the dedication
offerings for the altar
after
it was anointed.
\par }{\PP \VS{89}Now when Moses
went
into
the tent
of meeting
to speak
with
the
{\ND{Lord}}, he heard
the voice
speaking
to
him from above
the atonement lid
that
was on
the ark
of the testimony,
from between
the two
cherubim.
Thus he spoke
to him.


\par }\Chap{8}{\PP \VerseOne{1}The
{\ND{Lord}}
spoke
to
Moses:
\VS{2}“Speak
to
Aaron
and tell
him,
‘When you set up
the lamps,
the seven
lamps
are to
give light
in front
of the lampstand.’ ”
\par }{\PP \VS{3}And Aaron
did
so;
he set up
the lamps
to
face
toward
the front
of the lampstand,
as
the {\ND{Lord}}
commanded
Moses.
\VS{4}This
is how the lampstand
was made: It was beaten
work
in gold;
from its shaft
to its flowers
it was beaten
work. According to the pattern
which
the {\ND{Lord}}
had
shown
Moses,
so
he made
the lampstand.
\par }{\SH The Separation of the Levites
\par }{\PP \VS{5}Then the
{\ND{Lord}}
spoke
to
Moses:
\VS{6}“Take
the Levites
from among
the Israelites
and purify them.
\VS{7}And do
this
to them to purify
them: Sprinkle
water
of purification
on
them; then have them shave
all
their body
and wash
their clothes,
and so purify themselves.
\VS{8}Then they are to take
a young bull
with its grain offering
of fine flour
mixed
with olive oil;
and you are to take
a second
young
bull
for a purification offering.
\VS{9}You are to bring
the Levites
before
the tent
of meeting
and assemble
the entire
community
of the Israelites.
\VS{10}Then you are to bring
the Levites
before
the
{\ND{Lord}}, and the Israelites
are to lay
their hands
on
the Levites;
\VS{11}and Aaron
is to offer
the
Levites
before
the {\ND{Lord}}
as a wave offering
from the Israelites,
that they may
do the work
of the {\ND{Lord}}.
\VS{12}When the Levites
lay
their hands
on
the heads
of the bulls,
offer
the
one
for a purification
offering and the
other
for a whole burnt offering
to the
{\ND{Lord}}, to make atonement
for the Levites.
\VS{13}You are to have the Levites
stand
before
Aaron
and his sons,
and then offer them
as a wave offering
to the
{\ND{Lord}}.
\VS{14}And so you are to separate
the Levites
from among
the Israelites,
and the Levites
will be mine.
\par }{\PP \VS{15}“After
this,
the Levites
will go
in to do the work
of the tent
of meeting.
So you must cleanse
them
and offer them like
a wave offering.
\VS{16}For
they are
entirely
given
to me from among
the Israelites.
I have taken
them for myself instead
of all
who open
the womb,
the firstborn sons
of all
the Israelites.
\VS{17}For
all
the firstborn males
among the Israelites
are mine, both humans
and animals;
when
I destroyed
all
the firstborn
in the land
of Egypt
I set them apart for myself.
\VS{18}So I have taken
the Levites
instead
of all
the firstborn sons
among the Israelites.
\VS{19}I have given
the
Levites
as a gift to Aaron
and his sons
from among
the Israelites,
to do
the work
for the Israelites
in the tent
of meeting,
and to make atonement
for the Israelites,
so there will be
no
plague
among the Israelites
when the Israelites
come near
the sanctuary.”
\par }{\PP \VS{20}So
Moses
and Aaron
and the entire
community
of the Israelites
did this with the Levites.
According to all
that
the

{\ND{Lord}}
commanded
Moses
concerning the Levites,
this
is what the Israelites
did with them.
\VS{21}The Levites
purified
themselves and washed
their clothing;
then
Aaron
presented them like
a wave offering
before
the {\ND{Lord}}, and Aaron
made atonement
for them to purify them.
\VS{22}After
this,
the Levites
went
in to do their work
in the tent
of meeting
before
Aaron
and before
his sons.
As
the {\ND{Lord}}
had commanded
Moses
concerning
the Levites,
so
they did.
\par }{\SH The Work of the Levites
\par }{\PP \VS{23}Then the
{\ND{Lord}}
spoke
to
Moses:
\VS{24}“This
is what
pertains to the Levites: At the age of twenty-five
years
and upward
one may begin to join
the company
in the work
of the tent
of meeting,
\VS{25}and at the age of fifty
years
they must retire
from performing
the work
and may
no
longer
work.
\VS{26}They may assist
their colleagues
in the tent
of meeting,
to attend
to needs,
but they must do no
work.
This
is the way you must
establish the Levites
regarding
their duties.”


\par }\Chap{9}{\PP \VerseOne{1}The
{\ND{Lord}}
spoke
to
Moses
in the wilderness
of Sinai,
in the first
month
of the second
year
after they had come out
of the land
of Egypt:
\par }{\PP \VS{2}“The Israelites
are to observe
the Passover
at its appointed time.
\VS{3}In the fourteenth
day
of this
month,
at twilight,
you are to observe
it at its appointed
time; you must keep it in accordance with all
its statutes
and all
its customs.”
\VS{4}So Moses
instructed
the Israelites
to observe
the Passover.
\VS{5}And they observed
the Passover
on the fourteenth
day
of the first
month
at twilight
in the wilderness
of Sinai;
in accordance with all
that
the {\ND{Lord}}
had commanded
Moses,
so
the Israelites
did.
\par }{\PP \VS{6}It happened
that some men
who were ceremonially defiled
by the dead body of a man
could
not
keep
the Passover
on that day,
so they came
before
Moses
and before
Aaron
on that day.
\VS{7}And those
men
said
to him,
“We
are ceremonially defiled
by the dead body of a man;
why
are we kept
back from offering
the
{\ND{Lord}}’s
offering at its appointed
time among
the Israelites?”
\VS{8}So Moses
said
to
them, “Remain
here and I will hear
what
the {\ND{Lord}}
will command concerning you.”
\par }{\PP \VS{9}The
{\ND{Lord}}
spoke
to
Moses:
\VS{10}“Tell
the Israelites,
‘If any of
you or
of your posterity
become
ceremonially defiled
by touching a dead body,
or
are on a journey
far
away, then he may observe the Passover
to the
{\ND{Lord}}.
\VS{11}They may observe it on the fourteenth
day
of the second
month
at twilight;
they are to eat
it with bread
made
without yeast
and with bitter herbs.
\VS{12}They must not
leave
any
of it
until
morning,
nor
break
any
of its bones;
they must observe
it in accordance with every
statute
of the Passover.
\par }{\PP \VS{13}But the man
who
is ceremonially clean,
and was not
on a journey,
and fails
to keep
the Passover,
that person must be cut off
from his people.
Because
he did not
bring
the
{\ND{Lord}}’s
offering
at its appointed
time, that
man
must bear
his sin.
\VS{14}If
a resident foreigner
lives
among you
and wants to keep
the Passover
to the
{\ND{Lord}}, he must do
so
according to the statute
of the Passover,
and according
to its custom.
You must have the same
statute
for the resident foreigner
and for the one who was born
in the land.’ ”
\par }{\SH The Leading of the Lord
\par }{\PP \VS{15}On the day
that the tabernacle
was set
up, the
cloud
covered
the
tabernacle
– the tent
of the testimony –
and from evening
until
morning
there was a fiery
appearance
over
the tabernacle.
\VS{16}This is the way
it used to be
continually: The cloud
would cover
it by day, and there was a fiery
appearance
by night.
\VS{17}Whenever
the cloud
was taken up
from the tabernacle,
then after
that
the Israelites
would begin
their journey;
and in whatever
place
the cloud
settled,
there
the Israelites
would make camp.
\VS{18}At the commandment
of the {\ND{Lord}}
the Israelites
would begin their journey,
and at the commandment
of the {\ND{Lord}}
they would make camp;
as
long
as
the cloud
remained settled
over
the tabernacle
they would camp.
\VS{19}When the cloud
remained
over
the tabernacle
many
days,
then the Israelites
obeyed
the instructions
of the {\ND{Lord}}
and did not
journey.
\par }{\PP \VS{20}When
the cloud
remained over
the tabernacle
a number
of days,
they remained
camped
according
to the
{\ND{Lord}}’s
commandment, and according to the
{\ND{Lord}}’s
commandment
they would journey.
\VS{21}And when
the cloud
remained
only from evening
until
morning,
when the cloud
was taken up
the following morning,
then they traveled
on. Whether
by day
or by night,
when the cloud
was taken up
they traveled.
\VS{22}Whether
it was for two days,
or
a month,
or
a year,
that the cloud
prolonged
its stay over
the tabernacle,
the Israelites
remained
camped
without
traveling;
but when it was taken up,
they traveled on.
\VS{23}At the commandment
of the {\ND{Lord}}
they camped,
and at the commandment
of the {\ND{Lord}}
they traveled
on; they kept
the instructions
of the {\ND{Lord}}
according
to the commandment
of the {\ND{Lord}}, by the authority
of Moses.


\par }\Chap{10}{\PP \VerseOne{1}The
{\ND{Lord}}
spoke
to
Moses:
\VS{2}“Make
two
trumpets
of silver;
you are to make them from a single hammered
piece. You will use
them for assembling
the community
and for directing the traveling
of the
camps.
\VS{3}When they blow
them both, all
the community
must come
to
you to the entrance
of the tent
of meeting.
\par }{\PP \VS{4}“But if
they blow with one
trumpet,
then the leaders,
the heads
of the thousands
of Israel,
must come
to you.
\VS{5}When you blow
an alarm,
then the camps
that are located
on the east
side must begin to travel.
\VS{6}And when you blow
an alarm
the second
time, then the camps
that are located
on the south side
must begin to travel.
An alarm
must be sounded
for their journeys.
\VS{7}But when you assemble
the community,
you must blow,
but you must not
sound an alarm.
\VS{8}The sons
of Aaron,
the priests,
must blow
the trumpets;
and they will be
to you for an eternal
ordinance
throughout your generations.
\VS{9}If
you go
to war
in your land
against an adversary
who opposes
you, then you must sound an alarm
with the trumpets,
and you will be remembered
before
the {\ND{Lord}}
your God,
and you will be saved
from your enemies.
\par }{\PP \VS{10}“Also in the time
when you rejoice,
such as on your appointed festivals
or at the beginnings
of your months,
you must blow
with your trumpets
over
your burnt offerings
and over
the sacrifices
of your peace offerings,
so that they may become
a memorial
for you before
your God: I am
the {\ND{Lord}}
your God.”
\par }{\SH The Journey From Sinai to Kadesh
\par }{\PP \VS{11}On
the twentieth
day
of the second
month,
in the second
year,
the cloud
was taken up
from the tabernacle
of the testimony.
\VS{12}So
the Israelites
set
out on their journeys
from the wilderness
of Sinai;
and the cloud
settled
in the wilderness
of Paran.
\par }{\SH Judah Begins the Journey
\par }{\PP \VS{13}This was the first
time they set out on
their journey
according to the commandment
of the {\ND{Lord}}, by the authority
of Moses.
\par }{\PP \VS{14}The standard
of the camp
of the Judahites
set out first
according to their companies,
and over
his company
was Nahshon
son
of Amminadab.
\par }{\PP \VS{15}Over
the company
of the tribe
of Issacharites
was Nathanel
son
of Zuar,
\VS{16}and over
the company
of the tribe
of the Zebulunites
was Elion
son
of Helon.
\VS{17}Then
the tabernacle
was
dismantled,
and the sons
of Gershon
and the sons
of Merari
set out,
carrying
the tabernacle.
\par }{\SH Journey Arrangements for the Tribes
\par }{\PP \VS{18}The standard
of the camp
of Reuben
set out according to their companies;
over
his company
was Elizur
son
of Shedeur.
\VS{19}Over
the company
of the tribe
of the Simeonites
was Shelumiel
son
of Zurishaddai,
\VS{20}and over
the company
of the tribe
of the Gadites
was Eliasaph
son
of Deuel.
\VS{21}And the Kohathites
set out,
carrying
the articles
for the sanctuary;
the
tabernacle
was to be set up
before
they arrived.
\VS{22}And the standard
of the camp
of the Ephraimites
set out
according to their companies;
over
his company
was Elishama
son
of Ammihud.
\VS{23}Over
the company
of the tribe
of the Manassehites
was Gamaliel
son
of Pedahzur,
\VS{24}and over
the company
of the tribe
of Benjaminites
was Abidan
son
of Gideoni.
\par }{\PP \VS{25}The standard
of the camp
of the Danites
set out,
which was the rear guard
of all
the camps
by their companies;
over
his company
was Ahiezer
son
of Ammishaddai.
\VS{26}Over
the company
of the tribe
of the Asherites
was Pagiel
son
of Ocran,
\VS{27}and over
the company
of the tribe
of the Naphtalites
was Ahira
son
of Enan.
\VS{28}These
were the traveling arrangements
of the Israelites
according to their companies
when they traveled.
\par }{\SH The Appeal to Hobab
\par }{\PP \VS{29}Moses
said
to Hobab
son
of Reuel,
the Midianite,
Moses’
father-in-law, “We
are journeying
to
the place
about which
the {\ND{Lord}}
said,
‘I will give
it to you.’ Come
with
us and we will treat you well,
for
the {\ND{Lord}}
has promised
good
things for Israel.”
\VS{30}But Hobab said
to him,
“I will not
go,
but I will go
instead
to
my own land
and to
my kindred.”
\VS{31}Moses said,
“Do not
leave
us, because
you know
places
for us to camp
in the wilderness,
and you could be
our guide.
\VS{32}And if
you come
with
us, it is
certain that whatever
good
things the
{\ND{Lord}}
will favor us with, we will share with
you as well.”
\par }{\PP \VS{33}So they traveled
from the mountain
of the {\ND{Lord}}
three
days’
journey;
and the ark
of the covenant
of the {\ND{Lord}}
was traveling
before
them during the three
days’
journey,
to find
a resting place for them.
\VS{34}And the cloud
of the {\ND{Lord}}
was over
them by day,
when they traveled
from
the camp.
\VS{35}And when the ark
traveled,
Moses
would say,
“Rise
up, O
{\ND{Lord}}! May your enemies
be scattered,
and may those who hate
you flee
before you!”
\VS{36}And when it came to rest
he would say,
“Return,
O
{\ND{Lord}}, to the many thousands
of Israel!”


\par }\Chap{11}{\PP \VerseOne{1}When
the people
complained,
it displeased
the {\ND{Lord}}. When the
{\ND{Lord}}
heard
it, his anger
burned,
and so the fire
of the {\ND{Lord}}
burned
among them and consumed
some of the outer parts
of the camp.
\VS{2}When the people
cried
to
Moses,
he
prayed
to
the {\ND{Lord}}, and the fire
died out.
\VS{3}So he called
the name
of that place
Taberah
because
there the fire
of the {\ND{Lord}}
burned among them.
\par }{\SH Complaints about Food
\par }{\PP \VS{4}Now the mixed multitude
who
were among
them craved
more desirable
foods,
and so the Israelites
wept
again
and said,
“If only
we had meat
to eat!
\VS{5}We remember
the fish
we used to
eat
freely
in Egypt,
the cucumbers,
the melons,
the leeks,
the onions,
and the garlic.
\VS{6}But now
we
are dried up,
and there is nothing
at all
before
us
except
this manna!”
\VS{7}(Now the manna
was like coriander
seed,
and its color like the color of bdellium.
\VS{8}And the people
went about
and gathered
it, and ground
it with mills
or
pounded
it in mortars;
they baked
it in pans
and made
cakes
of it. It tasted
like fresh
olive oil.
\VS{9}And when the dew
came down
on
the camp
in the night,
the manna fell with it.)
\par }{\SH Moses’ Complaint to the Lord
\par }{\PP \VS{10}Moses
heard
the people
weeping
throughout their families,
everyone
at the door
of his tent;
and when the anger
of the
{\ND{Lord}}
was kindled
greatly,
Moses
was also displeased.
\VS{11}And Moses
said
to
the {\ND{Lord}}, “Why
have you afflicted
your servant? Why
have I not
found
favor
in your sight,
that you lay
the burden
of this
entire
people
on me?
\VS{12}Did I
conceive
this
entire
people? Did I
give birth
to them, that
you should say
to
me, ‘Carry
them in your arms,
as a foster father
bears
a nursing
child,’ to the land
which
you swore
to their fathers?
\VS{13}From where
shall I get meat
to give
to this
entire
people,
for
they cry
to me, ‘Give
us meat,
that we may eat!’
\VS{14}I am
not
able
to bear
this
entire
people
alone,
because
it is too heavy
for me!
\VS{15}But if
you
are going
to deal
with me like
this, then kill
me
immediately.
If
I have found
favor
in your sight
then do not
let me see
my trouble.”
\par }{\SH The Response of God
\par }{\PP \VS{16}The
{\ND{Lord}}
said
to
Moses,
“Gather
to me seventy
men
of the elders
of Israel,
whom
you know
are elders
of the people
and officials
over them, and bring
them to
the tent
of meeting;
let them take their position
there
with you.
\VS{17}Then I will come down
and speak
with
you there,
and I will take part
of the spirit
that
is on
you, and will put
it on
them, and they will bear
some of the burden
of the people
with
you, so that you
do not
bear
it all by yourself.
\par }{\PP \VS{18}“And say
to
the people,
‘Sanctify
yourselves for tomorrow,
and you will eat
meat,
for
you have wept
in the hearing
of the {\ND{Lord}}, saying,
“Who
will give us meat
to eat,
for
life was good
for us in Egypt?” Therefore the
{\ND{Lord}}
will give
you meat,
and you will eat.
\VS{19}You will eat,
not just one
day,
nor two days,
nor five
days,
nor
ten
days,
nor
twenty
days,
\VS{20}but a whole
month,
until
it comes
out
your nostrils
and makes
you sick,
because
you have despised
the

{\ND{Lord}}
who
is among
you and have wept
before
him, saying,
“Why
did we ever come out
of Egypt?” ’ ”
\par }{\PP \VS{21}Moses
said,
“The people
around me are 600,000
on foot;
but you
say,
‘I
will give
them meat,
that they may eat
for a whole
month.’
\VS{22}Would they have enough
if the flocks
and herds
were slaughtered
for them? If
all
the fish
of the sea
were caught
for them, would they have enough?”
\VS{23}And the
{\ND{Lord}}
said
to
Moses,
“Is the
{\ND{Lord}}’s
hand
shortened? Now
you will see
whether
my word
to you will come true or
not!”
\par }{\PP \VS{24}So Moses
went out
and told
the people
the words
of the {\ND{Lord}}. He then gathered
seventy
men
of the elders
of the people
and had them stand
around
the tabernacle.
\VS{25}And the
{\ND{Lord}}
came down
in the cloud
and spoke
to
them, and he took
some
of the Spirit
that
was on
Moses and put
it on
the seventy
elders.
When
the Spirit
rested
on
them, they prophesied,
but did not
do so again.
\par }{\SH Eldad and Medad
\par }{\PP \VS{26}But two
men
remained
in the camp;
one’s name
was Eldad,
and the other’s
name
was Medad.
And the spirit
rested
on
them. (Now they
were among those in
the registration,
but had not
gone
to the tabernacle.) So they prophesied
in the camp.
\VS{27}And a young man
ran
and told
Moses,
“Eldad
and Medad
are prophesying
in the camp!”
\VS{28}Joshua
son
of Nun,
the servant
of Moses,
one of his choice young men,
said, “My lord
Moses,
stop them!”
\VS{29}Moses
said
to him, “Are you
jealous
for me? I wish
that all
the
{\ND{Lord}}’s
people
were prophets,
that
the {\ND{Lord}}
would put
his Spirit
on them!”
\VS{30}Then Moses
returned
to
the camp
along with the elders
of Israel.
\par }{\SH Provision of Quail
\par }{\PP \VS{31}Now a wind
went
out from
the

{\ND{Lord}}
and brought
quail
from
the sea,
and let them fall
near the camp,
about
a day’s
journey
on
this side, and about
a day’s journey
on the other side, all around
the camp,
and about
three feet
high on
the surface of
the ground.
\VS{32}And the people
stayed up
all
that day,
all
that night,
and all
the next
day,
and gathered
the quail.
The one who gathered
the least gathered
ten
homers,
and they spread
them out
for themselves all around
the camp.
\VS{33}But while the meat
was still
between
their teeth,
before
they chewed
it, the anger
of the {\ND{Lord}}
burned
against the people,
and the
{\ND{Lord}}
struck
the people
with a very
great
plague.
\par }{\PP \VS{34}So
the name
of that place
was
called
Kibroth Hattaavah,
because
there
they buried
the people
that craved different food.
\VS{35}The people
traveled
from Kibroth Hattaavah
to Hazeroth,
and they stayed at Hazeroth.


\par }\Chap{12}{\PP \VerseOne{1}Then Miriam
and Aaron
spoke
against
Moses
because
of the Cushite
woman
he had
married
(for
he had married
an Ethiopian
woman).
\VS{2}They said,
“Has the
{\ND{Lord}}
only
spoken
through
Moses? Has he not
also
spoken
through us?” And the
{\ND{Lord}}
heard it.
\par }{\PP \VS{3}(Now the man
Moses
was very
humble,
more so than any
man
on
the face
of the earth.)
\par }{\SH The Response of the Lord
\par }{\PP \VS{4}The
{\ND{Lord}}
spoke
immediately
to
Moses,
Aaron,
and Miriam: “The three
of you come to
the tent
of meeting.”
So the three
of them went.
\VS{5}And the
{\ND{Lord}}
came down
in a pillar
of cloud
and stood
at the entrance
of the tent;
he then called
Aaron
and Miriam,
and they both
came forward.
\par }{\PP \VS{6}The
{\ND{Lord}} said,
“Hear
now
my words: If
there
is a prophet
among you, I the
{\ND{Lord}}
will make myself known
to him
in a vision;
I
will speak
with him in a dream.
\VS{7}My servant
Moses
is not
like this;
he is faithful
in all
my house.
\VS{8}With him
I will speak
face to face, openly,
and not
in riddles;
and he will see
the form
of the {\ND{Lord}}. Why
then were you not
afraid
to speak
against my servant
Moses?”
\VS{9}The anger
of the
{\ND{Lord}}
burned
against them, and he departed.
\VS{10}When the cloud
departed
from above
the tent,
Miriam
became leprous
as snow.
Then
Aaron
looked at Miriam,
and she was leprous!
\par }{\SH The Intercession of Moses
\par }{\PP \VS{11}So Aaron
said
to
Moses,
“O
my lord,
please
do not
hold
this sin
against
us, in which
we have acted foolishly
and have sinned!
\VS{12}Do
not
let
her be
like a baby born dead,
whose
flesh
is half-consumed
when it comes out
of its mother’s
womb!”
\par }{\PP \VS{13}Then Moses
cried
to
the {\ND{Lord}}, “Heal
her now,
O God.”
\VS{14}The
{\ND{Lord}}
said
to
Moses,
“If her father
had only spit
in her face,
would she not
have been disgraced
for seven
days? Shut
her out
from the camp
seven
days,
and afterward
she can be brought back in again.”
\par }{\PP \VS{15}So Miriam
was shut
outside
of the camp
for seven
days,
and the people
did not
journey
on until
Miriam
was brought back in.
\VS{16}After
that the people
moved
from Hazeroth
and camped
in the wilderness
of Paran.


\par }\Chap{13}{\PP \VerseOne{1}The
{\ND{Lord}}
spoke
to
Moses:
\VS{2}“Send
out men
to investigate
the land
of Canaan,
which
I
am giving
to the Israelites.
You are to send
one
man
from each
ancestral
tribe,
each one
a leader among them.”
\VS{3}So Moses
sent
them
from the wilderness
of Paran
at the command
of the {\ND{Lord}}. All
of them were
leaders
of the Israelites.
\par }{\PP \VS{4}Now these
were their names: from the tribe
of Reuben,
Shammua
son
of Zaccur;
\VS{5}from the tribe
of Simeon,
Shaphat
son
of Hori;
\VS{6}from the tribe
of Judah,
Caleb
son
of Jephunneh;
\VS{7}from the tribe
of Issachar,
Igal
son
of Joseph;
\VS{8}from the tribe
of Ephraim,
Hoshea
son
of Nun;
\VS{9}from the tribe
of Benjamin,
Palti
son
of Raphu;
\VS{10}from the tribe
of Zebulun,
Gaddiel
son
of Sodi;
\VS{11}from the tribe
of Joseph,
namely, the tribe
of Manasseh,
Gaddi
son
of Susi;
\VS{12}from the tribe
of Dan,
Ammiel
son
of Gemalli;
\VS{13}from the tribe
of Asher,
Sethur
son
of Michael;
\VS{14}from the tribe
of Naphtali,
Nahbi
son
of Vopshi;
\VS{15}from the tribe
of Gad,
Geuel
son
of Maki.
\VS{16}These
are the names
of the men
whom
Moses
sent
to investigate
the land.
And Moses
gave Hoshea
son
of Nun
the name
Joshua.
\par }{\SH The Spies’ Instructions
\par }{\PP \VS{17}When Moses
sent
them to investigate
the land
of Canaan,
he told
them,
“Go up
through the Negev,
and then go up
into the hill country
\VS{18}and see
what
the land
is
like, and whether the people
who live
in it are strong
or weak,
few
or
many,
\VS{19}and whether
the land
they live
in is good
or bad,
and whether
the cities
they inhabit
are like camps
or fortified cities,
\VS{20}and whether
the land
is rich
or
poor,
and whether
or not
there are
forests
in it. And be brave,
and bring back
some of the fruit
of the land.”
Now it was the time
of year
for the first ripe
grapes.
\par }{\SH The Spies’ Activities
\par }{\PP \VS{21}So they went up
and investigated
the land
from the wilderness
of Zin
to
Rehob,
at the entrance
of Hamath.
\VS{22}When they went up
through the Negev,
they came
to Hebron
where
Ahiman,
Sheshai,
and Talmai,
descendants
of Anak,
were living. (Now Hebron
had been built
seven
years
before
Zoan
in Egypt.)
\VS{23}When they came
to
the valley
of Eshcol,
they cut down
from there
a branch
with one
cluster
of grapes,
and they carried
it on a staff
between two
men, as well as some
of the pomegranates
and the figs.
\VS{24}That place
was
called
the Eshcol
Valley,
because
of the cluster
of grapes that
the Israelites
cut
from there.
\VS{25}They returned
from investigating
the land
after
forty
days.
\par }{\SH The Spies’ Reports
\par }{\PP \VS{26}They came
back to
Moses
and Aaron
and to
the whole
community
of the Israelites
in the wilderness
of Paran
at Kadesh.
They reported
to the
whole
community
and showed
the
fruit
of the land.
\VS{27}They told
Moses, “We went
to
the land
where
you sent
us. It is
indeed
flowing
with milk
and honey,
and this
is its fruit.
\VS{28}But
the inhabitants
are strong,
and the cities
are fortified
and very
large.
Moreover
we saw
the descendants
of Anak
there.
\VS{29}The Amalekites
live
in the land
of the Negev;
the Hittites,
Jebusites,
and Amorites
live
in the hill country;
and the Canaanites
live
by
the sea
and along the banks
of the Jordan.”
\par }{\PP \VS{30}Then Caleb
silenced
the people
before
Moses,
saying,
“Let us go
up
and occupy
it, for
we
are well able to conquer it.”
\VS{31}But the men
who had
gone up
with
him said,
“We are not
able
to go up
against these people,
because
they are stronger
than we are!”
\VS{32}Then they presented
the Israelites
with a discouraging report
of the land
they had
investigated,
saying,
“The land
that
we passed through
to investigate
is a land
that devours
its inhabitants.
All
the people
we saw
there are of great stature.
\VS{33}We even saw
the Nephilim
there
(the descendants
of Anak
came from
the Nephilim), and we seemed
liked
grasshoppers both to ourselves and to them.”


\par }\Chap{14}{\PP \VerseOne{1}Then
all
the community
raised
a loud
cry, and the people
wept
that night.
\VS{2}And all
the Israelites
murmured
against
Moses
and Aaron,
and the whole
congregation
said
to them,
“If only
we had died
in the land
of Egypt,
or
if only
we had perished
in this
wilderness!
\VS{3}Why
has the
{\ND{Lord}}
brought
us into
this
land
only
to be killed
by the sword,
that our wives
and our children
should become
plunder? Wouldn’t it be better
for us to return
to Egypt?”
\VS{4}So they said
to
one
another, “Let’s appoint
a leader
and return
to Egypt.”
\par }{\PP \VS{5}Then Moses
and Aaron
fell
down with their faces
to the ground before
the whole
assembled
community
of the Israelites.
\VS{6}And Joshua
son
of Nun
and Caleb
son
of Jephunneh,
two of those who had investigated
the land,
tore
their garments.
\VS{7}They said
to
the whole
community
of the Israelites,
“The land
we passed
through to investigate
is an exceedingly
good
land.
\VS{8}If
the {\ND{Lord}}
delights
in us, then he will bring
us into
this
land
and give it to us – a land that is flowing with milk and honey.
\VS{9}Only
do not
rebel
against the
{\ND{Lord}}, and do not
fear
the
people
of the land,
for
they are bread
for us. Their
protection
has turned aside
from them, but the
{\ND{Lord}}
is with
us. Do not
fear them!”
\par }{\PP \VS{10}However, the whole
community
threatened
to stone
them. But the glory
of the {\ND{Lord}}
appeared
to
all
the Israelites
at the tent
of meeting.
\par }{\SH The Punishment from God
\par }{\PP \VS{11}The
{\ND{Lord}}
said
to
Moses,
“How long
will this
people
despise
me, and how long
will they not
believe
in me, in spite of the signs
that
I have done
among them?
\VS{12}I will strike
them with the pestilence,
and I will disinherit them;
I will make
you into a nation
that is greater
and mightier
than they!”
\par }{\PP \VS{13}Moses
said
to
the {\ND{Lord}}, “When the Egyptians
hear it – for you brought up this people by your power from among them –
\VS{14}then they will tell
it to
the inhabitants
of this
land.
They have heard
that
you,

{\ND{Lord}}, are among
this
people,
that
you,

{\ND{Lord}},
are seen
face
to face,
that your
cloud
stands
over
them, and that
you
go
before
them by day
in a pillar
of cloud
and in a pillar
of fire
by night.
\VS{15}If you kill
this
entire people
at once,
then the nations
that
have heard
of your fame
will say,
\VS{16}‘Because the
{\ND{Lord}}
was not
able
to bring
this
people
into
the land
that
he swore
to them, he killed
them in the wilderness.’
\VS{17}So now,
let
the power
of my Lord
be great,
just
as you have said,
\VS{18}‘The
{\ND{Lord}}
is slow
to anger
and abounding
in loyal love,
forgiving
iniquity
and transgression,
but by no
means clearing
the guilty, visiting
the iniquity
of
the fathers
on
the children until
the third and fourth generations.’
\VS{19}Please
forgive
the iniquity
of this
people
according to your great
loyal love,
just as
you have forgiven
this
people
from Egypt
even until
now.”
\par }{\PP \VS{20}Then the
{\ND{Lord}}
said,
“I have forgiven
them as
you asked.
\VS{21}But truly,
as I
live,
all
the earth
will be filled
with the glory
of the {\ND{Lord}}.
\VS{22}For
all
the people
have seen
my glory
and my signs
that
I did
in Egypt
and in the wilderness,
and yet have tempted
me now these
ten
times,
and have not
obeyed
me,
\VS{23}they will by no means
see
the land
that
I swore
to their fathers,
nor
will any
of them who despised
me see it.
\VS{24}Only my servant
Caleb,
because
he had a different
spirit
and has followed
me fully
– I will bring
him into
the land
where
he had gone,
and his descendants
will possess it.
\VS{25}(Now the Amalekites
and the Canaanites
were living
in the valleys.) Tomorrow,
turn
and journey
into the wilderness
by the way
of the Red
Sea.”
\par }{\PP \VS{26}The
{\ND{Lord}}
spoke
to
Moses
and Aaron:
\VS{27}“How long
must I bear with this
evil
congregation
that
murmurs
against
me? I have heard
the complaints
of the Israelites
that
they
murmured
against me.
\VS{28}Say
to
them, ‘As I
live,
says
the {\ND{Lord}}, I will surely do to you just
what you have spoken
in my hearing.
\VS{29}Your dead bodies
will fall
in this
wilderness
– all
those of you who were numbered,
according to your full number,
from twenty
years
old and upward,
who have
murmured
against me.
\VS{30}You
will by no means
enter
into
the land
where
I swore
to settle
you. The only exceptions
are Caleb
son
of Jephunneh
and Joshua
son
of Nun.
\VS{31}But I will bring
in your little ones,
whom
you said
would become
victims of war,
and they will enjoy
the land
that
you have despised.
\VS{32}But as for you, your dead bodies
will fall
in this
wilderness,
\VS{33}and your children
will wander
in the wilderness
forty
years
and suffer
for your unfaithfulness,
until
your dead bodies
lie finished in the wilderness.
\VS{34}According to the number
of the days
you have investigated
this land,
forty
days
– one day
for a year
– you will suffer for
your iniquities,
forty
years,
and you will know
what it means to thwart me.
\VS{35}I,
the {\ND{Lord}}, have said,
“I will
surely
do
so to all
this
evil
congregation
that has gathered
together against
me. In this
wilderness
they will be finished, and there
they will die!” ’ ”
\par }{\PP \VS{36}The men
whom
Moses
sent
to investigate
the land,
who returned
and made the whole
community
murmur against
him by producing
an evil report
about the land,
\VS{37}those men
who produced
the evil
report
about the land,
died
by
the plague
before
the
{\ND{Lord}}.
\VS{38}But Joshua
son
of Nun
and Caleb
son
of Jephunneh,
who were
among the men
who went
to investigate
the land,
lived.
\VS{39}When Moses
told
these
things to
all
the Israelites,
the people
mourned
greatly.
\par }{\PP \VS{40}And early
in the morning
they went up
to
the crest
of the hill country,
saying,
“Here
we are, and we will go up
to
the place
that
the {\ND{Lord}}
commanded, for
we have sinned.”
\VS{41}But Moses
said,
“Why
are you
now transgressing
the commandment
of the {\ND{Lord}}? It
will not
succeed!
\VS{42}Do not
go up,
for
the {\ND{Lord}}
is not
among
you, and you will be defeated
before
your enemies.
\VS{43}For
the Amalekites
and the Canaanites
are there
before
you, and you will fall
by the sword.
Because
you have
turned
away
from the
{\ND{Lord}}, the
{\ND{Lord}}
will not
be
with you.”
\par }{\PP \VS{44}But they dared
to go up
to
the crest
of the hill,
although neither the ark
of the covenant
of the {\ND{Lord}}
nor
Moses
departed
from the camp.
\VS{45}So
the Amalekites
and the Canaanites
who lived
in that hill country
swooped
down and attacked
them as far
as Hormah.


\par }\Chap{15}{\PP \VerseOne{1}The
{\ND{Lord}}
spoke
to
Moses:
\VS{2}“Speak
to
the Israelites
and tell
them,
‘When
you enter
the land
where you are to live,
which
I am
giving you,
\VS{3}and you make
an offering by fire
to the
{\ND{Lord}}
from
the herd
or
from
the flock
(whether a burnt offering
or
a sacrifice
for discharging
a vow
or
as a freewill
offering or
in your solemn feasts) to create
a pleasing
aroma
to the
{\ND{Lord}},
\VS{4}then
the one who presents
his offering
to the
{\ND{Lord}}
must bring a grain offering
of one-tenth of an ephah
of finely ground flour
mixed
with one fourth
of a hin
of olive oil.
\VS{5}You must also prepare
one-fourth
of a hin
of wine
for a drink offering
with the burnt offering
or
the sacrifice
for each
lamb.
\VS{6}Or
for a ram,
you must prepare
as a grain offering
two-tenths of an ephah
of finely ground flour
mixed
with one-third
of a hin
of olive oil,
\VS{7}and for a drink offering
you must offer one-third
of a hin
of wine
as a pleasing
aroma
to the
{\ND{Lord}}.
\VS{8}And when
you prepare
a young
bull
as a burnt offering
or
a sacrifice
for discharging
a vow
or
as a peace offering
to the
{\ND{Lord}},
\VS{9}then a grain offering
of three-tenths of an ephah
of finely ground flour
mixed
with half
a hin
of olive oil
must be presented
with the young
bull,
\VS{10}and you must present
as the drink offering
half
a hin
of wine
with the fire offering
as a pleasing
aroma
to the
{\ND{Lord}}.
\VS{11}This
is what is to be done
for each
ox,
or
each ram,
or
each
of the male lambs
or
the goats.
\VS{12}You must do
so
for each one
according
to the number
that you prepare.
\par }{\PP \VS{13}“‘Every
native-born
person must do
these
things in this way
to present
an offering made by fire
as a pleasing
aroma
to the
{\ND{Lord}}.
\VS{14}If
a resident foreigner
is living
with you
– or
whoever
is among
you in future generations –
and prepares
an offering made by fire
as a pleasing
aroma
to the
{\ND{Lord}}, he must do
it the same way
you are to do it.
\VS{15}One
statute
must apply
to you who belong to the congregation
and to the resident foreigner
who is living
among you, as a permanent
statute
for your future
generations.
You and the resident foreigner
will be
alike before
the {\ND{Lord}}.
\VS{16}One
law
and one
custom
must apply
to you and to the resident foreigner
who lives
alongside you.’ ”
\par }{\SH Rules for First Fruits
\par }{\PP \VS{17}The
{\ND{Lord}}
spoke
to
Moses:
\VS{18}“Speak
to
the Israelites
and tell
them,
‘When you enter
the land
to which
I am
bringing you
\VS{19}and you eat
some of the food
of the land,
you must offer
up a raised
offering to the
{\ND{Lord}}.
\VS{20}You must offer up
a cake
of the first
of your finely ground flour
as a raised offering;
as you offer the raised offering
of the threshing floor,
so
you must offer
it up.
\VS{21}You must give
to the
{\ND{Lord}}
some of the first
of your finely ground flour
as a raised offering
in your future generations.
\par }{\SH Rules for Unintentional Offenses
\par }{\PP \VS{22}“ ‘If
you sin unintentionally
and do not
observe
all
these
commandments
that
the {\ND{Lord}}
has spoken
to
Moses –
\VS{23}all
that
the {\ND{Lord}}
has
commanded
you by the authority
of Moses,
from
the day
that
the {\ND{Lord}}
commanded
Moses and continuing
through your future generations –
\VS{24}then
if
anything is done
unintentionally
without the knowledge
of the community,
the whole
community
must prepare
one
young
bull
for a burnt offering
– for a pleasing
aroma
to the
{\ND{Lord}} –
along with its grain offering
and its customary
drink offering,
and one
male
goat
for a purification offering.
\VS{25}And the priest
is to make atonement
for the whole
community
of the Israelites,
and they will be forgiven,
because
it was unintentional
and they
have brought
their offering,
an offering made by fire
to the
{\ND{Lord}}, and their purification
offering before
the {\ND{Lord}}, for their unintentional offense.
\VS{26}And the whole
community
of the Israelites
and the resident foreigner
who lives
among
them will be forgiven,
since
all
the people
were involved in the unintentional offense.
\par }{\PP \VS{27}“‘If
any
person
sins
unintentionally,
then he must bring
a yearling
female
goat
for a purification offering.
\VS{28}And the priest
must make atonement
for
the person
who sins unintentionally
– when he sins
unintentionally
before
the {\ND{Lord}} –
to make atonement
for
him, and he will be forgiven.
\VS{29}You must have
one
law
for the person who sins
unintentionally,
both for the native-born
among
the Israelites
and for the resident foreigner
who lives
among them.
\par }{\SH Deliberate Sin
\par }{\PP \VS{30}“‘But the person
who
acts defiantly,
whether native-born
or a resident foreigner,
insults
the {\ND{Lord}}. That person must be cut off
from among
his people.
\VS{31}Because
he has despised
the word
of the {\ND{Lord}}
and has broken
his commandment,
that person must be completely cut off.
His iniquity will be on him.’ ”
\par }{\PP \VS{32}When
the Israelites
were in the wilderness
they found
a man
gathering
wood
on the Sabbath
day.
\VS{33}Those
who found
him gathering
wood
brought him to
Moses
and Aaron
and to
the whole
community.
\VS{34}They put
him in custody,
because
there was no
clear
instruction about what
should be done to him.
\VS{35}Then the
{\ND{Lord}}
said
to Moses,
“The man
must
surely be put to death;
the whole
community
must stone
him with stones
outside
the camp.”
\VS{36}So
the whole
community
took
him outside
the camp
and stoned
him to death,
just
as the
{\ND{Lord}}
commanded
Moses.
\par }{\SH Rules for Tassels
\par }{\PP \VS{37}The
{\ND{Lord}}
spoke to
Moses:
\VS{38}“Speak
to
the Israelites
and tell
them to
make
tassels
for themselves on
the corners
of their garments
throughout their generations,
and put
a blue
thread
on
the tassel
of the corners.
\VS{39}You must have
this tassel
so that you may look
at it and remember
all
the commandments
of the {\ND{Lord}}
and obey
them and so
that you do not
follow
after
your own heart
and your own
eyes
that
lead you
to unfaithfulness.
\VS{40}Thus
you will remember
and obey
all
my commandments
and be
holy
to your God.
\VS{41}I am
the {\ND{Lord}}
your God,
who
brought you out
of the land
of Egypt
to be
your God.
I am
the {\ND{Lord}}
your God.”


\par }\Chap{16}{\PP \VerseOne{1}Now Korah
son
of Izhar,
the son
of Kohath,
the son
of Levi,
and Dathan
and Abiram,
the sons
of Eliab,
and On
son
of Peleth,
who were Reubenites,
took men
\VS{2}and rebelled
against Moses,
along with some
of the Israelites,
250
leaders
of the community,
chosen
from the assembly,
famous
men.
\VS{3}And they assembled
against
Moses
and Aaron,
saying
to them,
“You take
too much
upon yourselves, seeing that
the whole
community
is holy,
every
one of them, and the
{\ND{Lord}}
is among
them. Why
then do you exalt
yourselves above
the community
of the {\ND{Lord}}?”
\par }{\PP \VS{4}When Moses
heard
it he fell
down with his face to the ground.
\VS{5}Then
he said
to
Korah
and to
all
his company,
“In the morning
the {\ND{Lord}}
will make known
who
are his, and who
is holy.
He will cause that
person to
approach
him; the person he has
chosen
he will cause to
approach
him.
\VS{6}Do
this,
Korah,
you and all
your company: Take
censers,
\VS{7}put
fire
in them,
and set
incense
on
them before
the {\ND{Lord}}
tomorrow,
and the man
whom
the {\ND{Lord}}
chooses
will be
holy.
You take
too
much upon yourselves,
you sons
of Levi!”
\VS{8}Moses
said
to
Korah,
“Listen
now,
you sons
of Levi!
\VS{9}Does it seem
too
small a thing to you that
the God
of Israel
has separated
you from the
community
of Israel
to bring you near
to
himself, to perform
the service
of the tabernacle
of the {\ND{Lord}},
and to stand
before
the community
to minister to them?
\VS{10}He has brought you near
and all
your brothers,
the sons
of Levi,
with
you. Do you now seek
the priesthood
also?
\VS{11}Therefore
you
and all
your company
have assembled
together against
the {\ND{Lord}}! And Aaron
– what
is he
that
you murmur against him?”
\VS{12}Then Moses
summoned
Dathan
and Abiram,
the sons
of Eliab,
but they said,
“We will not
come up.
\VS{13}Is it a small
thing that
you have brought us up
out of the land
that flows
with milk
and honey,
to kill
us
in the wilderness? Now do
you want
to make yourself
a prince over us?
\VS{14}Moreover,
you have not
brought
us into
a land
that flows
with milk
and honey,
nor given
us an inheritance
of fields
and vineyards.
Do you think you can blind
these
men? We will not
come up.”
\par }{\PP \VS{15}Moses
was very
angry,
and he said
to
the {\ND{Lord}}, “Have no
respect
for their offering! I have not
taken
so much as one
donkey
from them,
nor
have I harmed
any one
of them!”
\par }{\PP \VS{16}Then Moses
said to
Korah,
“You
and all
your company
present yourselves
before
the {\ND{Lord}} –
you
and they,
and Aaron
– tomorrow.
\VS{17}And each
of you take
his censer,
put
incense
in it, and then each
of you present
his censer
before
the {\ND{Lord}}: 250
censers,
along with you,
and Aaron
– each
of you with his censer.”
\VS{18}So everyone
took
his censer,
put
fire
in it, and set
incense
on
it, and stood
at the entrance
of the tent
of meeting,
with Moses
and Aaron.
\VS{19}When Korah
assembled
the whole
community
against
them
at the entrance
of the tent
of meeting,
then the glory
of the {\ND{Lord}}
appeared
to
the whole
community.
\par }{\SH The Judgment on the Rebels
\par }{\PP \VS{20}The
{\ND{Lord}}
spoke
to
Moses
and Aaron:
\VS{21}“Separate
yourselves from among
this
community,
that I may consume
them in an instant.”
\VS{22}Then they threw
themselves down with their faces
to the ground and said,
“O God,
the God
of the spirits
of all
people,
will you be angry
with the whole
community
when only one
man
sins?”
\par }{\PP \VS{23}So the
{\ND{Lord}}
spoke
to
Moses:
\VS{24}“Tell
the community: ‘Get away
from around
the homes
of Korah,
Dathan,
and Abiram.’ ”
\VS{25}Then Moses
got up
and went
to
Dathan
and Abiram;
and the elders
of Israel
went
after him.
\VS{26}And he said
to
the community,
“Move
away from the tents
of these
wicked
men,
and do not
touch
anything
they have,
lest
you be destroyed
because of all
their sins.”
\VS{27}So
they got away
from the homes
of Korah,
Dathan,
and Abiram
on every side,
and Dathan
and Abiram
came out
and stationed
themselves in the entrances
of their tents
with their wives,
their children,
and their toddlers.
\VS{28}Then Moses
said,
“This
is how you will know
that
the {\ND{Lord}}
has sent
me to do
all
these
works,
for
I have not done them of my own will.
\VS{29}If
these
men
die
a natural death,
or if they share
the fate
of all
men,
then the
{\ND{Lord}}
has not
sent me.
\VS{30}But if
the {\ND{Lord}}
does something entirely new,
and the earth
opens
its mouth
and swallows
them up along
with all
that
they have, and they go down
alive
to the grave,
then you will know
that
these
men
have despised
the

{\ND{Lord}}!”
\par }{\PP \VS{31}When
he had finished
speaking
all
these
words,
the ground
that
was under
them split open,
\VS{32}and the earth
opened
its
mouth
and swallowed
them, along with their households,
and all
Korah’s
men,
and all
their goods.
\VS{33}They
and all
that
they had
went down
alive
into the pit,
and the earth
closed
over
them. So they perished
from among
the community.
\VS{34}All
the Israelites
who
were around
them fled
at their cry,
for
they said,
“What if
the earth
swallows
us too?”
\VS{35}Then a fire
went out
from the
{\ND{Lord}}
and devoured
the 250
men
who offered
incense.
\par }{\SH The Atonement for the Rebellion
\par }{\PP \VS{36} The
{\ND{Lord}}
spoke
to
Moses:
\VS{37}“Tell
Eleazar
son
of Aaron
the priest
to pick up
the censers
out of the flame,
for
they are holy,
and then
scatter
the coals
of fire
at a distance.
\VS{38}As for the
censers
of these
men who sinned
at the cost of their lives,
they must be made
into hammered
sheets
for covering
the altar,
because
they presented
them before
the {\ND{Lord}}
and sanctified
them. They will become
a sign
to the Israelites.”
\VS{39}So
Eleazar
the priest
took
the bronze
censers
presented
by those who had been burned up,
and they were hammered
out as a covering
for the altar.
\VS{40}It was
a memorial
for the Israelites,
that
no outsider
who
is not
a descendant
of Aaron
should approach
to burn
incense
before
the {\ND{Lord}}, that
he might
not
become
like Korah
and his company
– just
as the
{\ND{Lord}}
had spoken
by the authority
of Moses.
\VS{41}But on
the next day
the whole
community
of Israelites
murmured
against
Moses
and Aaron,
saying,
“You
have killed
the
{\ND{Lord}}’s
people!”
\VS{42}When
the community
assembled
against
Moses
and Aaron,
they turned
toward
the tent
of meeting
– and the cloud
covered
it, and the glory
of the {\ND{Lord}}
appeared.
\VS{43}Then Moses
and Aaron
stood
before
the tent
of meeting.
\par }{\PP \VS{44}The
{\ND{Lord}}
spoke
to
Moses:
\VS{45}“Get away
from
this
community,
so that I can consume
them in an instant!” But they threw
themselves down
with their faces to the ground.
\VS{46}Then Moses
said to
Aaron,
“Take
the
censer,
put
burning
coals from the altar
in it, place
incense
on it, and go
quickly
into
the assembly
and make atonement
for them,
for
wrath
has gone out
from the
{\ND{Lord}} –
the plague
has begun!”
\VS{47}So
Aaron
did as
Moses
commanded
and ran
into
the middle
of the assembly,
where the plague
was just beginning
among the people.
So he placed
incense
on
the coals and made atonement
for the people.
\VS{48}He stood
between
the dead
and the living,
and the plague
was stopped.
\VS{49}Now
14,700
people died
in the plague,
in addition
to those who died
in the event
with Korah.
\VS{50}Then Aaron
returned
to
Moses
at the entrance
of the tent
of meeting,
and the plague
was stopped.


\par }\Chap{17}{\PP \VerseOne{1}The
{\ND{Lord}}
spoke
to
Moses:
\VS{2}“Speak
to
the Israelites,
and receive
from them a staff
from each tribe,
one from every
tribal leader,
twelve
staffs;
you must write
each man’s
name
on
his staff.
\VS{3}You must write
Aaron’s
name
on
the staff
of Levi;
for
one
staff
is for the head
of every tribe.
\VS{4}You must place
them in the tent
of meeting
before
the ark
of the covenant where
I meet with you.
\VS{5}And the staff
of the man
whom
I choose
will blossom;
so I will rid
myself of the complaints
of the Israelites,
which
they
murmur
against you.”
\par }{\PP \VS{6}So Moses
spoke
to
the Israelites,
and each of their leaders
gave
him a staff,
one
for each
leader,
according to their tribes –
twelve
staffs;
the staff
of Aaron
was among
their staffs.
\VS{7}Then Moses
placed
the staffs
before
the {\ND{Lord}}
in the tent
of the testimony.
\par }{\PP \VS{8}On the next day
Moses
went
into
the tent
of the testimony
– and
the staff
of Aaron
for the house
of Levi
had sprouted,
and brought forth
buds,
and produced
blossoms,
and yielded
almonds!
\VS{9}So Moses
brought out
all
the staffs
from before
the
{\ND{Lord}}
to
all
the Israelites.
They looked
at them, and each man
took
his staff.
\par }{\SH The Memorial
\par }{\PP \VS{10}The
{\ND{Lord}}
said
to
Moses,
“Bring
Aaron’s
staff
back before
the testimony
to be preserved for
a sign
to the rebels,
so that you may bring
their murmurings
to an end before me, that they will not
die.”
\VS{11}So Moses
did
as
the {\ND{Lord}}
commanded him – this is what he did.
\par }{\PP \VS{12}The Israelites
said
to
Moses,
“We are bound to die! We perish,
we all
perish!
\VS{13}Anyone who even comes
close
to
the tabernacle
of the {\ND{Lord}}
will die! Are we
all to die?”


\par }\Chap{18}{\PP \VerseOne{1}The
{\ND{Lord}}
said
to
Aaron,
“You
and your sons
and your tribe
with
you must bear
the
iniquity
of the sanctuary,
and you
and your sons
with
you must bear
the
iniquity
of your priesthood.
\par }{\PP \VS{2}“Bring with you your brothers,
the tribe
of Levi,
the tribe
of your father,
so
that they may join
with you
and minister
to you while you
and your sons
with
you are before
the tent
of the testimony.
\VS{3}They must be responsible
to care
for you and to care
for the entire
tabernacle.
However,
they must not
come near
the furnishings
of the sanctuary
and the altar,
or
both
they
and you
will die.
\VS{4}They must
join
with you, and they will be responsible
for the care
of the tent
of meeting,
for all
the service
of the tent,
but no
unauthorized
person may approach you.
\VS{5}You
will be responsible
for the care
of the sanctuary
and the
care
of the altar,
so that there will be no
more
wrath
on
the Israelites.
\VS{6}I myself
have
chosen
your brothers
the Levites
from among
the Israelites.
They are given
to you as a gift
from the
{\ND{Lord}}, to perform
the duties
of the tent
of meeting.
\VS{7}But you
and your sons
with
you are responsible
for your priestly duties,
for everything
at the altar
and within the curtain.
And you must serve.
I give
you the
priesthood
as a gift
for service;
but the unauthorized
person who approaches must
be put to death.”
\par }{\SH The Portion of the Priests
\par }{\PP \VS{8}The
{\ND{Lord}}
spoke
to
Aaron,
“See,
I
have given
you the
responsibility
for my raised offerings;
I have given
all
the holy things
of the Israelites
to you as your priestly portion
and to your sons
as a perpetual
ordinance.
\VS{9}Of all the most
holy offerings reserved
from
the fire
this
will be
yours: Every
offering
of theirs, whether from every
grain offering
or from every
purification
offering or from every
reparation
offering which
they bring
to me, will be most
holy
for you and for your sons.
\VS{10}You are to eat
it as a most
holy
offering; every
male
may eat
it. It will be
holy to you.
\par }{\PP \VS{11}“And this
is yours: the raised
offering of their gift,
along with all
the wave offerings
of the Israelites.
I have given
them to you and to your sons
and daughters
with
you as a perpetual
ordinance.
Everyone
who is ceremonially clean
in your household
may eat of it.
\par }{\PP \VS{12}“All
the best
of the olive oil
and all
the best
of the wine
and of the wheat,
the first fruits
of these things that
they give
to the
{\ND{Lord}}, I have given to you.
\VS{13}And whatever
first ripe fruit
in their land
they bring
to the
{\ND{Lord}}
will be
yours; everyone
who
is ceremonially clean
in your household
may eat of it.
\par }{\PP \VS{14}“Everything
devoted
in Israel
will be yours.
\VS{15}The firstborn
of every
womb
which
they present
to the
{\ND{Lord}}, whether human
or animal,
will be
yours. Nevertheless,
the firstborn sons
you must redeem,
and the firstborn
males
of unclean
animals
you must redeem.
\VS{16}And those that must be redeemed
you are to redeem
when they are a month
old,
according to your estimation,
for five
shekels
of silver
according to the sanctuary
shekel
(which
is twenty
gerahs).
\VS{17}But
you must not
redeem
the firstborn
of a cow
or
a sheep
or
a goat;
they are
holy.
You must
splash
their blood
on
the altar
and burn
their fat
for an offering made by fire
for a pleasing
aroma
to the
{\ND{Lord}}.
\VS{18}And their meat
will be yours, just as
the breast
and the right
hip
of the raised offering is yours.
\VS{19}All
the raised offerings
of the holy
things that
the Israelites
offer
to the
{\ND{Lord}}, I have given
to you, and to your sons
and daughters
with
you, as a perpetual
ordinance.
It is
a covenant
of salt
forever
before
the {\ND{Lord}}
for you and for your descendants
with you.”
\par }{\SH Duties of the Levites
\par }{\PP \VS{20}The
{\ND{Lord}}
spoke to
Aaron,
“You will have no
inheritance
in their land,
nor
will
you have any portion
of property among them – I am your portion and your inheritance among the Israelites.
\VS{21}See,
I have given
the Levites
all
the tithes
in Israel
for an inheritance,
for
their service
which
they
perform
– the service
of the tent
of meeting.
\VS{22}No
longer
may the Israelites
approach
the tent
of meeting,
or else they will bear
their sin
and die.
\VS{23}But
the Levites
must
perform
the service
of the tent
of meeting,
and they
must bear
their iniquity.
It will be a perpetual
ordinance
throughout your generations
that among
the Israelites
the Levites have no
inheritance.
\VS{24}But I have given
to the Levites
for
an inheritance
the
tithes
of the Israelites
that
are offered to the
{\ND{Lord}}
as a raised
offering.
That is why
I said
to them
that among
the Israelites
they are to have no
inheritance.”
\par }{\SH Instructions for the Levites
\par }{\PP \VS{25}The
{\ND{Lord}}
spoke
to
Moses:
\VS{26}“You are to speak to
the Levites,
and you must
tell
them,
‘When
you receive
from the
Israelites
the
tithe
that
I have given
you from them as your inheritance,
then you are to offer
up from
it as a raised
offering to the
{\ND{Lord}}
a tenth
of the tithe.
\VS{27}And your raised
offering will be credited
to you as though it were grain
from
the threshing floor
or as new wine
from the winepress.
\VS{28}Thus
you
are to offer
up a raised
offering to the
{\ND{Lord}}
of all
your tithes
which
you receive
from the
Israelites;
and you must give
the
{\ND{Lord}}’s
raised
offering from
it to Aaron
the priest.
\VS{29}From all
your gifts
you must offer
up every
raised
offering due the
{\ND{Lord}}, from all
the best
of it, and the holiest part of it.’
\par }{\PP \VS{30}“Therefore you will say
to them, ‘When you offer
up the
best
of it, then it
will be
credited
to the Levites
as the product
of the threshing floor
and as the product
of the winepress.
\VS{31}And you may eat
it in any
place,
you
and your household,
because
it is your wages
for
your service
in the tent
of meeting.
\VS{32}And you will bear
no
sin
concerning it when you offer
up the
best
of it.
And you must not profane
the
holy
things of the Israelites,
or else you will die.’ ”


\par }\Chap{19}{\PP \VerseOne{1}The
{\ND{Lord}}
spoke
to
Moses
and Aaron:
\VS{2}“This
is the ordinance
of the law
which
the {\ND{Lord}}
has commanded: ‘Instruct
the Israelites
to
bring
you a red
heifer
without blemish,
which
has no
defect
and has
never
carried
a yoke.
\VS{3}You must give
it to
Eleazar
the priest
so
that he can take it outside
the camp,
and it must be slaughtered
before him.
\VS{4}Eleazar
the priest
is to take
some of its blood
with his finger,
and sprinkle
some of the blood
seven
times
directly
in front
of the tent
of meeting.
\VS{5}Then the heifer
must be burned
in his sight
– its
skin,
its
flesh,
its blood,
and its offal
is to be burned.
\VS{6}And the priest
must take
cedar
wood,
hyssop,
and scarlet
wool and throw
them into
the midst
of the fire where the heifer
is burning.
\VS{7}Then the priest
must wash
his clothes
and bathe
himself in water,
and afterward
he may come
into the camp,
but the priest
will be ceremonially unclean
until
evening.
\VS{8}The one who burns
it must wash
his clothes
in water
and bathe
himself in water.
He will be ceremonially unclean
until
evening.
\par }{\PP \VS{9}“‘Then a man
who is ceremonially clean
must gather up
the ashes
of the red heifer
and put
them in a ceremonially clean
place
outside
the camp.
They must be
kept
for the community
of the Israelites
for use in the water
of purification –
it is a purification for sin.
\VS{10}The one who gathers
the ashes
of the heifer
must wash
his clothes
and be ceremonially unclean
until
evening.
This will be
a permanent
ordinance
both for the Israelites
and the resident foreigner
who lives
among them.
\par }{\SH Purification from Uncleanness
\par }{\PP \VS{11}“‘Whoever touches
the corpse
of any
person
will be ceremonially unclean
seven
days.
\VS{12}He must purify
himself with water on
the third
day
and on
the seventh
day,
and so will be clean.
But if
he does not
purify
himself on
the third
day
and the seventh
day,
then he will not
be clean.
\VS{13}Anyone
who touches
the corpse
of any
dead
person
and does not
purify
himself defiles
the tabernacle
of the
{\ND{Lord}}. And that person must be cut off
from Israel,
because
the water
of purification
was not
sprinkled
on
him. He will be
unclean;
his uncleanness remains on him.
\par }{\PP \VS{14}“‘This
is the law: When
a man
dies
in a tent,
anyone
who comes
into
the tent
and all
who
are in the tent
will be ceremonially unclean
seven
days.
\VS{15}And every
open
container
that
has no
covering
fastened
on
it is
unclean.
\VS{16}And whoever
touches
the body of someone killed
with a sword
in the open
fields,
or
the body of someone who died
of natural causes, or
a human
bone,
or
a grave,
will be unclean
seven
days.
\par }{\PP \VS{17}“‘For a ceremonially unclean
person you must take
some of the ashes
of the heifer burnt
for purification
from sin and pour fresh running
water
over
them
in a vessel.
\VS{18}Then a ceremonially clean
person
must take
hyssop,
dip
it in the water,
and sprinkle
it on
the tent,
on
all
its furnishings,
and on
the people
who
were there,
or on
the one who touched
a bone,
or
one killed,
or
one who died,
or
a grave.
\VS{19}And the clean
person must sprinkle
the unclean
on
the third
day
and on
the seventh
day;
and on
the seventh
day
he must purify
him, and then he must wash
his clothes,
and bathe
in water,
and he will be clean
in the evening.
\VS{20}But the man
who
is unclean
and does not
purify
himself, that person must be cut off
from among
the community,
because
he has polluted
the sanctuary
of the {\ND{Lord}}; the water
of purification
was not
sprinkled
on
him, so he is
unclean.
\par }{\PP \VS{21}“‘So this will be
a perpetual
ordinance
for them: The one who sprinkles
the water
of purification
must wash
his clothes,
and the one who touches
the water
of purification
will be unclean
until
evening.
\VS{22}And whatever
the unclean
person touches
will be unclean,
and the person
who touches
it will be unclean
until
evening.’ ”


\par }\Chap{20}{\PP \VerseOne{1}Then the entire
community
of Israel
entered
the wilderness
of Zin
in the first
month,
and the people
stayed
in Kadesh.
Miriam
died
and was buried
there.
\par }{\PP \VS{2}And there was
no
water
for the community,
and so they gathered
themselves together against
Moses
and Aaron.
\VS{3}The people
contended
with
Moses,
saying, “If only
we
had died
when our brothers
died
before
the {\ND{Lord}}!
\VS{4}Why
have you brought
up the
{\ND{Lord}}’s
community
into
this
wilderness? So
that we
and our cattle
should die here?
\VS{5}Why
have you brought us up
from Egypt
only to bring
us
to this
dreadful
place? It is no
place
for grain,
or figs,
or vines,
or pomegranates;
nor is there
any water
to drink!”
\par }{\SH Moses Responds
\par }{\PP \VS{6}So
Moses
and Aaron
went
from the presence
of the assembly
to
the entrance
to the tent
of meeting.
They then threw
themselves down with their faces
to the ground, and the glory
of the {\ND{Lord}}
appeared
to them.
\VS{7}Then the
{\ND{Lord}}
spoke
to
Moses:
\VS{8}“Take
the
staff
and assemble
the
community,
you
and Aaron
your brother,
and then speak
to
the rock
before their eyes.
It will
pour forth its water,
and you will bring
water
out
of the rock
for them, and so you will give the
community
and their beasts
water to drink.”
\par }{\PP \VS{9}So Moses
took
the staff
from before
the {\ND{Lord}}, just
as he commanded him.
\VS{10}Then Moses
and Aaron
gathered
the community
together in front
of the rock,
and he said
to them, “Listen,
you rebels,
must we
bring
water
out of this
rock for you?”
\VS{11}Then Moses
raised
his hand,
and struck
the rock
twice
with his staff.
And water
came out
abundantly.
So the community
drank,
and their beasts
drank too.
\par }{\SH The Lord’s Judgment
\par }{\PP \VS{12}Then the
{\ND{Lord}}
spoke to
Moses
and Aaron,
“Because
you did not
trust
me enough to show me as holy
before
the Israelites,
therefore
you will not
bring
this
community
into
the land
I have
given them.”
\par }{\PP \VS{13}These
are the waters
of Meribah,
because the Israelites
contended
with
the
{\ND{Lord}}, and his holiness was maintained among them.
\par }{\SH Rejection by the Edomites
\par }{\PP \VS{14}Moses
sent
messengers
from Kadesh
to
the king
of Edom: “Thus
says
your brother
Israel: ‘You
know
all
the hardships we have experienced,
\VS{15}how our ancestors
went down
into Egypt,
and we lived
in Egypt
a long time,
and the Egyptians
treated
us and our ancestors
badly.
\VS{16}So when we cried
to
the {\ND{Lord}}, he heard
our voice
and sent
a messenger,
and has brought us up out
of Egypt.
Now
we
are here in Kadesh,
a town
on the edge
of your country.
\VS{17}Please
let us pass
through your country.
We will not
pass
through the fields
or through the vineyards,
nor
will we drink
water
from any well.
We will go by
the King’s
Highway; we
will not
turn
to the right
or the left
until
we have
passed
through your region.’ ”
\par }{\PP \VS{18}But Edom
said
to him,
“You will not
pass
through me, or
I will come out
against
you with the sword.”
\VS{19}Then the Israelites
said
to
him, “We will go along
the highway,
and if
we
or our cattle
drink
any of your water,
we will pay for it.
We will only
pass through
on our feet,
without doing
anything else.”
\par }{\PP \VS{20}But he said,
“You may not
pass
through.” Then
Edom
came out against them
with a large
and powerful force.
\VS{21}So Edom
refused
to give
Israel
passage through
his border;
therefore Israel
turned away from him.
\par }{\SH Aaron’s Death
\par }{\PP \VS{22}So the entire
company
of Israelites
traveled
from Kadesh
and came
to Mount
Hor.
\VS{23}And the
{\ND{Lord}}
spoke
to
Moses
and Aaron
in Mount
Hor,
by the border
of the land
of Edom.
He said:
\VS{24}“Aaron
will be gathered
to
his ancestors,
for
he will not
enter
into
the land
I have
given
to the Israelites
because
both of you rebelled
against my word
at the waters
of Meribah.
\VS{25}Take
Aaron
and Eleazar
his son,
and bring them up
on Mount
Hor.
\VS{26}Remove
Aaron’s
priestly garments
and put
them on
Eleazar
his son,
and Aaron
will be gathered
to his ancestors and will die
there.”
\par }{\PP \VS{27}So Moses
did
as
the {\ND{Lord}}
commanded;
and they went up
Mount
Hor
in the sight
of the whole
community.
\VS{28}And Moses
removed
Aaron’s
garments
and put
them on
his son
Eleazar.
So Aaron
died
there
on the top
of the mountain.
And Moses
and Eleazar
came down
from
the mountain.
\VS{29}When all
the community
saw
that
Aaron
was dead,
the whole
house
of Israel
mourned
for Aaron
thirty
days.


\par }\Chap{21}{\PP \VerseOne{1}When
the Canaanite
king
of Arad
who lived in
the Negev
heard that
Israel
was approaching
along the road
to Atharim,
he fought
against Israel
and took
some of
them prisoner.
\par }{\PP \VS{2}So Israel
made
a vow
to the
{\ND{Lord}}
and said,
“If
you will
indeed deliver
this
people
into our hand,
then we will utterly
destroy their cities.”
\VS{3}The
{\ND{Lord}}
listened
to the voice
of Israel
and delivered
up the
Canaanites,
and they utterly
destroyed them and their cities.
So the name
of the place
was called
Hormah.
\par }{\SH Fiery Serpents
\par }{\PP \VS{4}Then they traveled
from Mount
Hor
by the road
to the Red
Sea,
to go around
the land
of Edom,
but the people
became impatient
along the way.
\VS{5}And the people
spoke
against God
and against Moses,
“Why
have you brought us up
out of Egypt
to die
in the wilderness,
for
there is no
bread
or
water,
and we detest
this worthless
food.”
\par }{\PP \VS{6}So the
{\ND{Lord}}
sent
poisonous
snakes
among the people,
and they bit
the people;
many
people
of Israel
died.
\VS{7}Then
the people
came
to
Moses
and said,
“We have sinned,
for
we have spoken
against the
{\ND{Lord}}
and against you. Pray
to
the {\ND{Lord}}
that he would take away
the
snakes
from us.” So
Moses
prayed
for
the people.
\par }{\PP \VS{8}The
{\ND{Lord}}
said
to
Moses,
“Make
a poisonous
snake and set
it on
a pole.
When
anyone
who is bitten
looks
at it, he will live.”
\VS{9}So Moses
made
a bronze
snake
and put
it on
a pole,
so that if
a snake
had bitten
someone, when
he looked
at the bronze
snake he lived.
\par }{\SH The Approach to Moab
\par }{\PP \VS{10}The Israelites
traveled
on and camped
in Oboth.
\VS{11}Then they traveled
on from Oboth
and camped
at Iye Abarim,
in the wilderness
that
is before
Moab,
on the eastern side.
\VS{12}From there
they moved
on and camped
in the valley
of Zered.
\VS{13}From there
they moved
on and camped
on the other side
of the Arnon,
in the wilderness
that
extends
from the regions
of the Amorites,
for
Arnon
is the border
of Moab,
between
Moab
and the Amorites.
\VS{14}This is why
it is said
in the Book
of the Wars
of the {\ND{Lord}},
\par }{\Q “Waheb
in Suphah
and the
wadis,
\par }{\Q the Arnon
\VS{15}and the slope
of the valleys
\par }{\Q that extends
to the dwelling
of Ar,
\par }{\Q and falls off
at the border
of Moab.”
\par }{\PP \VS{16}And from there
they traveled to Beer;
that
is the well
where
the {\ND{Lord}}
spoke
to Moses,
“Gather
the people
and I will give
them water.”
\VS{17}Then
Israel
sang
this
song:
\par }{\Q “Spring up,
O well,
sing to it!
\par }{\Q \VS{18}The well
which the princes
dug,
\par }{\Q which the leaders
of the people
opened
\par }{\Q with their scepters
and their staffs.”
\par }{\PP And from the wilderness
they traveled to Mattanah;
\VS{19}and from Mattanah
to Nahaliel;
and from Nahaliel
to Bamoth;
\VS{20}and from Bamoth
to the valley
that
is in the country
of Moab,
near the top
of Pisgah,
which overlooks
the wilderness.
\par }{\SH The Victory over Sihon and Og
\par }{\PP \VS{21}Then
Israel
sent
messengers
to
King
Sihon
of the Amorites,
saying,
\par }{\PP \VS{22}“Let us pass
through your land;
we will not
turn aside
into the fields
or into the vineyards,
nor
will we drink
water
from any well,
but we will go along
the King’s
Highway
until
we pass
your borders.”
\VS{23}But Sihon
did not
permit
Israel
to pass through
his border;
he gathered
all
his forces
together and went out
against
Israel
into the wilderness.
When he came to
Jahaz,
he fought against
Israel.
\VS{24}But the Israelites
defeated
him in battle
and took possession
of his land
from the Arnon
to
the Jabbok,
as far
as the Ammonites,
for
the border
of the Ammonites
was strongly defended.
\VS{25}So
Israel
took
all
these
cities;
and Israel
settled
in all
the cities
of the Amorites,
in Heshbon,
and in all
its villages.
\VS{26}For
Heshbon
was the city
of King
Sihon
of the Amorites.
Now he
had fought
against the former
king
of Moab
and had taken
all
of his land
from his control,
as far
as the Arnon.
\VS{27}That is why
those who speak in proverbs
say,
\par }{\Q “Come
to Heshbon,
let it be built.
\par }{\Q Let the city
of Sihon
be established!
\par }{\Q \VS{28}For
fire
went out
from Heshbon,
\par }{\Q a flame
from the city
of Sihon.
\par }{\Q It has consumed
Ar
of Moab
\par }{\Q and the lords
of the high places
of Arnon.
\par }{\Q \VS{29}Woe
to you, Moab.
\par }{\Q You are ruined,
O people
of Chemosh!

\par }{\Q He has made
his sons
fugitives,
\par }{\Q and his daughters
the prisoners
of King
Sihon
of the Amorites.
\par }{\Q \VS{30}We have overpowered
them;

\par }{\Q Heshbon
has perished
as far as
Dibon.
\par }{\Q We have shattered
them as far as
Nophah,
\par }{\Q which
reaches to Medeba.”
\par }{\PP \VS{31}So the Israelites
lived
in the land
of the Amorites.
\VS{32}Moses
sent
spies
to reconnoiter Jaazer,
and they captured
its villages
and dispossessed
the Amorites
who
were there.
\par }{\PP \VS{33}Then they turned
and went up
by the road
to Bashan.
And King
Og
of Bashan
and all
his forces
marched out against them to do battle
at Edrei.
\VS{34}And the
{\ND{Lord}}
said
to
Moses,
“Do not
fear
him, for
I have delivered
him
and all
his people
and his land
into your hand. You will
do
to him what
you
did
to King
Sihon
of the Amorites,
who
lived
in Heshbon.
\VS{35}So they defeated
Og, his sons,
and all
his people,
until
there were
no survivors,
and they possessed
his land.


\par }\Chap{22}{\PP \VerseOne{1}The Israelites
traveled
on and camped
in the plains
of Moab
on the side
of the Jordan River
across from Jericho.
\VS{2}Balak
son
of Zippor
saw
all
that
the Israelites
had
done
to the Amorites.
\VS{3}And the Moabites
were greatly
afraid
of the people,
because
they were so numerous.
The Moabites
were sick with fear because
of the Israelites.
\par }{\PP \VS{4}So the Moabites
said
to
the elders
of Midian,
“Now
this mass of people
will lick up
everything
around
us, as the bull
devours
the
grass
of the field.
Now Balak
son
of Zippor
was king
of the Moabites
at this
time.
\VS{5}And he sent
messengers
to
Balaam
son
of Beor
at Pethor,
which
is by
the Euphrates
River in the land
of Amaw, to summon
him, saying,
“Look,
a nation
has come out
of Egypt.
They cover
the face
of the
earth,
and they are settling
next to me.
\VS{6}So now,
please
come
and curse
this
nation
for me, for
they are too powerful
for me. Perhaps
I will prevail
so that we may conquer
them and drive
them out
of
the land.
For
I know
that whoever
you bless
is blessed,
and whoever
you curse
is cursed.”
\par }{\PP \VS{7}So
the elders
of Moab
and the elders
of Midian
departed with the fee for divination
in their hand.
They came
to
Balaam
and reported
to him
the words
of Balak.
\VS{8}He replied
to
them, “Stay
here
tonight,
and I will bring back
to you whatever
word
the {\ND{Lord}}
may speak
to me.”
So the princes
of Moab
stayed
with
Balaam.
\VS{9}And God
came
to
Balaam
and said,
“Who
are these
men
with you?”
\VS{10}Balaam
said
to
God,
“Balak
son
of Zippor,
king
of Moab,
has sent
a message
to me, saying,
\VS{11}“Look,
a nation
has come out
of Egypt,
and it covers
the face
of the
earth.
Come
now
and put a curse
on them for me; perhaps
I will be able
to defeat
them and drive them out.”
\VS{12}But God
said
to
Balaam,
“You must not
go
with
them; you must not
curse
the people,
for
they are blessed.”
\par }{\PP \VS{13}So Balaam
got
up in the morning,
and said
to
the princes
of Balak,
“Go
to
your land,
for
the {\ND{Lord}}
has refused
to permit
me to go
with you.”
\VS{14}So
the princes
of Moab
departed and went
back to
Balak
and said,
“Balaam
refused
to come
with us.”
\par }{\SH Balaam Accompanies the Moabite Princes
\par }{\PP \VS{15}Balak
again
sent
princes,
more
numerous
and more
distinguished than the first.
\VS{16}And they came
to
Balaam
and said
to him, “Thus
says
Balak
son
of Zippor: ‘Please do not
let
anything hinder
you from coming
to me.
\VS{17}For
I
will honor
you greatly,
and whatever
you tell
me
I will do.
So come,
put a curse
on this
nation for me.’ ”
\par }{\PP \VS{18}Balaam
replied
to
the servants
of Balak,
“Even if
Balak
would give
me his palace
full
of silver
and gold,
I could
not
transgress
the commandment
of the {\ND{Lord}}
my God
to do
less
or
more.
\VS{19}Now
therefore, please
stay
the night
here
also,
that I may
know
what
more
the {\ND{Lord}}
might say
to me.”
\VS{20}God
came
to
Balaam
that night,
and said
to him, “If
the men
have come
to call
you, get
up and go
with
them; but
the
word
that
I will say
to
you, that you must do.”
\VS{21}So Balaam
got
up in the morning,
saddled
his donkey,
and went
with
the princes
of Moab.
\par }{\SH God Opposes Balaam
\par }{\PP \VS{22}Then God’s
anger
was kindled
because
he went,
and the angel
of the {\ND{Lord}}
stood
in the road
to oppose
him. Now he
was riding
on
his donkey
and his two
servants
were with him.
\VS{23}And the donkey
saw
the
angel
of the {\ND{Lord}}
standing
in the road
with his sword
drawn
in his hand,
so the donkey
turned aside
from
the road
and went
into the field.
But
Balaam
beat
the donkey,
to make her turn
back to the road.
\par }{\PP \VS{24}Then the angel
of the {\ND{Lord}}
stood
in a path
among the vineyards,
where there was a wall
on either
side.
\VS{25}And when the donkey
saw
the angel
of the {\ND{Lord}}, she pressed
herself into
the wall,
and crushed
Balaam’s
foot
against the wall.
So he beat
her again.
\par }{\PP \VS{26}Then
the angel
of the {\ND{Lord}}
went farther,
and stood
in a narrow
place,
where
there was no
way
to turn
either to the right
or to the left.
\VS{27}When the donkey
saw
the angel
of the {\ND{Lord}}, she crouched down
under
Balaam.
Then Balaam
was angry,
and he
beat
his donkey
with a staff.
\par }{\PP \VS{28}Then the
{\ND{Lord}}
opened
the
mouth
of the donkey,
and she said
to Balaam,
“What
have I done
to you that
you have beaten
me these
three
times?”
\VS{29}And Balaam
said
to the donkey,
“You have made me look stupid;
I wish
there
were a sword
in my hand,
for
I would kill
you right now.”
\VS{30}The donkey
said
to
Balaam,
“Am not
I
your donkey
that
you have ridden
ever since I was
yours until
this
day? Have
I ever
attempted
to treat
you this
way?” And he said,
“No.”
\VS{31}Then the
{\ND{Lord}}
opened Balaam’s
eyes,
and he saw
the angel
of the {\ND{Lord}}
standing
in the way
with his sword
drawn
in his hand;
so he bowed
his head and threw himself down
with his face to the ground.
\VS{32}The angel
of the {\ND{Lord}}
said
to
him, “Why
have you beaten
your donkey
these
three
times? Look,
I
came out
to oppose
you because
what you are doing
is perverse
before me.
\VS{33}The donkey
saw
me and turned
from me these
three
times.
If
she had
not turned
from me, I would
have
killed
you but saved her alive.”
\VS{34}Balaam
said
to
the angel
of the {\ND{Lord}}, “I have sinned,
for
I did not
know
that
you
stood
against
me in the road.
So now,
if
it is evil
in your sight,
I will go back home.”
\VS{35}But the angel
of the {\ND{Lord}}
said to
Balaam,
“Go
with
the men,
but you may only
speak
the word
that
I will speak
to
you.” So
Balaam
went
with
the princes
of Balak.
\par }{\SH Balaam Meets Balak
\par }{\PP \VS{36}When Balak
heard
that
Balaam
was coming,
he went out
to meet
him at a city
of Moab
which
was on
the border
of the Arnon
at
the boundary
of his territory.
\VS{37}Balak
said
to
Balaam,
“Did I not
send again
and again
to
you to summon
you? Why
did you not
come
to me? Am
I not
able
to honor you?”
\VS{38}Balaam
said
to
Balak,
“Look,
I have come
to
you. Now,
am I able
to speak
just anything? I must speak
only the word
that
God
puts
in my mouth.”
\VS{39}So Balaam
went
with
Balak,
and they came
to Kiriath-huzoth.
\VS{40}And Balak
sacrificed
bulls
and sheep,
and sent
some to Balaam,
and to the princes
who
were with him.
\VS{41}Then
on the next morning
Balak
took
Balaam,
and brought him up
to Bamoth
Baal. From there
he saw
the extent
of the nation.


\par }\Chap{23}{\PP \VerseOne{1}Balaam
said
to
Balak,
“Build
me seven
altars
here, and prepare
for me here
seven
bulls
and seven
rams.”
\VS{2}So Balak
did
just
as Balaam
had said.
Balak
and Balaam
then offered
on each altar
a bull
and a ram.
\VS{3}Balaam
said
to Balak,
“Station yourself
by
your burnt offering,
and I will go
off;
perhaps
the {\ND{Lord}}
will come to meet
me, and whatever
he reveals
to me I will tell
you.” Then he went
to a deserted height.
\par }{\PP \VS{4}Then God
met
Balaam,
who said
to him,
“I have prepared
seven
altars,
and I have offered
on each altar
a bull
and a ram.”
\VS{5}Then
the {\ND{Lord}}
put
a message
in Balaam’s
mouth
and said,
“Return
to
Balak,
and speak what
I tell you.”
\par }{\PP \VS{6}So he returned
to
him, and he was still
standing
by his burnt offering,
he
and all
the princes
of Moab.
\VS{7}Then
Balaam uttered
his oracle,
saying,
\par }{\Q “Balak,
the king
of Moab,
brought me
from
Aram,
\par }{\Q out
of the mountains
of the east,
saying,
\par }{\Q ‘Come,
pronounce a curse
on
Jacob
for me;
\par }{\Q come,
denounce
Israel.’
\par }{\Q \VS{8}How
can I curse
one whom
God
has
not cursed,
\par }{\Q or how
can I denounce
one whom the
{\ND{Lord}}
has not
denounced?
\par }{\Q \VS{9}For
from the top
of the rocks
I see
them;

\par }{\Q from the hills
I watch them.

\par }{\Q Indeed,
a nation
that lives alone,
\par }{\Q and it will not
be reckoned
among the nations.
\par }{\Q \VS{10}Who can
count
the dust
of Jacob,
\par }{\Q Or number
the
fourth part
of Israel?
\par }{\Q Let me die
the death
of the upright,
\par }{\Q and let the end
of my life
be
like theirs.”
\par }{\SH Balaam Relocates
\par }{\PP \VS{11}Then Balak
said
to Balaam,
“What
have you done
to me? I brought you to curse
my enemies,
but on the contrary
you have
only blessed them!”
\VS{12}Balaam
replied, “Must I not
be careful
to speak
what
the {\ND{Lord}}
has put
in my mouth?”
\VS{13}Balak
said
to him,
“Please come
with
me to
another
place
from which
you can
observe
them. You will see
only
a part
of them, but you will not
see
all
of them. Curse
them for me from there.”
\par }{\PP \VS{14}So
Balak brought
Balaam to
the field
of Zophim,
to
the top
of Pisgah,
where he built
seven
altars
and offered
a bull
and a ram
on each altar.
\VS{15}And Balaam said
to Balak,
“Station yourself
here by
your burnt offering,
while
I
meet
the
{\ND{Lord}} there.
\VS{16}Then the
{\ND{Lord}}
met
Balaam
and put
a message
in his mouth
and said,
“Return
to
Balak,
and speak what
I tell you.”
\VS{17}When Balaam came
to
him, he was still
standing
by his burnt offering,
along with
the princes
of Moab.
And Balak
said
to him, “What
has the
{\ND{Lord}}
spoken?”
\par }{\SH Balaam Prophesies Again
\par }{\PP \VS{18}Balaam uttered
his oracle,
and said,
\par }{\Q “Rise
up, Balak,
and hear;
\par }{\Q Listen
to
me, son
of Zippor:
\par }{\Q \VS{19}God
is not
a man,
that he should lie,
\par }{\Q nor a human
being, that he should change
his mind.
\par }{\Q Has he said,
and will he not
do
it?
\par }{\Q Or has he
spoken,
and will he not
make
it happen?
\par }{\Q \VS{20}Indeed,
I have received
a command to bless;
\par }{\Q he has blessed,
and I cannot
reverse it.
\par }{\Q \VS{21}He has not
looked on
iniquity
in Jacob,
\par }{\Q nor
has he seen
trouble
in Israel.
\par }{\Q The
{\ND{Lord}}
their God
is with
them;
\par }{\Q his acclamation
as king is among them.
\par }{\Q \VS{22}God
brought them out
of Egypt.
\par }{\Q They have, as it were, the strength
of a wild bull.
\par }{\Q \VS{23}For
there is no
spell
against Jacob,
\par }{\Q nor
is there any divination
against Israel.
\par }{\Q At this time
it must be said
of Jacob
\par }{\Q and of Israel,
‘Look at what
God
has done!’
\par }{\Q \VS{24}Indeed,
the people
will rise
up like a lioness,
\par }{\Q and like a lion
raises
himself up;
\par }{\Q they
will not
lie
down until
they eat
their prey,
\par }{\Q and drink
the blood
of the slain.”
\par }{\SH Balaam Relocates Yet Again
\par }{\PP \VS{25}Balak
said
to Balaam,
“Neither
curse
them at all nor
bless them at all!”
\VS{26}But
Balaam
replied
to
Balak,
“Did I not
tell
you, ‘All
that
the {\ND{Lord}}
speaks,
I must do’?”
\par }{\PP \VS{27}Balak
said
to
Balaam,
“Come,
please;
I will take
you to
another
place.
Perhaps
it will please
God
to let you curse
them for me from there.”
\VS{28}So
Balak
took
Balaam
to the top
of Peor,
that looks
toward
the wilderness.
\VS{29}Then Balaam
said to
Balak,
“Build
seven
altars
here for me, and prepare
seven
bulls
and seven
rams.”
\VS{30}So Balak
did
as
Balaam
had
said,
and offered
a bull
and a ram
on each altar.


\par }\Chap{24}{\PP \VerseOne{1}When Balaam
saw
that
it pleased
the {\ND{Lord}}
to bless
Israel,
he did not
go
as at
the other times
to seek for omens,
but he set
his face
toward the wilderness.
\VS{2}When Balaam
lifted
up his eyes,
he saw
Israel
camped
tribe by tribe;
and the Spirit
of God came upon him.
\VS{3}Then
he uttered
this oracle:

\par }{\Q “The oracle
of Balaam
son
of Beor;
\par }{\Q the oracle
of the man
whose eyes are open;
\par }{\Q \VS{4}the oracle
of the one who hears
the words
of God,
\par }{\Q who
sees a vision
from the Almighty,
\par }{\Q although falling
flat
on
the ground with
eyes
open:
\par }{\Q \VS{5}‘How
beautiful
are your tents,
O Jacob,
\par }{\Q and your dwelling places,
O Israel!
\par }{\Q \VS{6}They are like valleys
stretched
forth,
\par }{\Q like gardens
by the river’s side,
\par }{\Q like aloes
that the
{\ND{Lord}}
has planted,
\par }{\Q and like cedar
trees beside
the waters.
\par }{\Q \VS{7}He will pour
the water
out of his buckets,
\par }{\Q and their descendants
will be like abundant
water;
\par }{\Q their king
will be greater
than
Agag,
\par }{\Q and their kingdom
will be exalted.
\par }{\Q \VS{8}God
brought them out
of Egypt.
\par }{\Q They have, as it were, the strength
of a young bull;
\par }{\Q they will devour
hostile
people

\par }{\Q and will break their
bones
\par }{\Q and will pierce
them through with arrows.
\par }{\Q \VS{9}They crouch
and lie
down like a lion,
\par }{\Q and as a lioness,
who can
stir
him?
\par }{\Q Blessed
is the one who blesses
you,
\par }{\Q and cursed
is the one who curses you!’ ”
\par }{\PP \VS{10}Then Balak
became very angry
at Balaam,
and he struck
his hands
together. Balak
said
to
Balaam,
“I called
you to curse
my enemies,
and look,
you have done nothing but bless
them these
three
times!
\VS{11}So now,
go back
where
you came
from! I said
that I would
greatly honor
you; but now
the {\ND{Lord}}
has stood
in the way
of your honor.”
\par }{\PP \VS{12}Balaam
said
to
Balak,
“Did I not
also
tell
your messengers
whom
you sent
to me,
\VS{13}‘If
Balak
would give
me his palace
full
of silver
and gold,
I cannot
go beyond
the
commandment
of the {\ND{Lord}}
to do
either good
or
evil
of my own will, but whatever
the {\ND{Lord}}
tells
me I must
speak’?
\VS{14}And now,
I am
about to go
back to my own people.
Come
now, and I will advise
you as
to what
this
people
will do
to your people
in the future.”
\par }{\SH Balaam Prophesies a Fourth Time
\par }{\PP \VS{15}Then
he uttered
this oracle:

\par }{\Q “The oracle
of Balaam
son
of Beor;
\par }{\Q the oracle
of the man
whose eyes are open;
\par }{\Q \VS{16}the oracle
of the one who hears
the words
of God,
\par }{\Q and who knows
the knowledge
of the Most
High,
\par }{\Q who sees a vision
from the Almighty,
\par }{\Q although falling
flat
on
the ground with
eyes open:
\par }{\Q \VS{17}‘I see
him, but not
now;
\par }{\Q I behold
him, but not
close
at hand.

\par }{\Q A star
will march forth out
of Jacob,
\par }{\Q and a scepter
will rise out
of Israel.
\par }{\Q He will crush
the skulls
of Moab,
\par }{\Q and the heads of all
the sons
of Sheth.
\par }{\Q \VS{18}Edom
will be
a possession,
\par }{\Q Seir,
his enemies,
will also be a possession;
\par }{\Q but Israel
will act
valiantly.
\par }{\Q \VS{19}A ruler
will be established from Jacob;
\par }{\Q he will destroy
the remains
of the city.’ ”
\par }{\SH Balaam’s Final Prophecies
\par }{\PP \VS{20}Then Balaam looked
on Amalek
and delivered
this oracle:

\par }{\Q “Amalek
was the first
of the nations,
\par }{\Q but his end
will be that he will perish.”
\par }{\PP \VS{21}Then
he looked
on the Kenites
and uttered
this oracle:
\par }{\Q “Your dwelling place
seems strong,
\par }{\Q and your nest
is set
on a rocky cliff.
\par }{\Q \VS{22}Nevertheless
the Kenite
will be
consumed.
\par }{\Q How
long will Asshur
take you away captive?”
\par }{\PP \VS{23}Then he uttered
this oracle:
\par }{\Q “O,
who
will survive
when God does this!
\par }{\Q \VS{24}Ships
will come from the coast
of Kittim,
\par }{\Q and will afflict
Asshur,
and will afflict
Eber,
\par }{\Q and he
will also
perish
forever.”
\par }{\PP \VS{25}Balaam
got
up and departed
and returned
to his home,
and Balak
also
went
his way.


\par }\Chap{25}{\PP \VerseOne{1}When Israel
lived
in Shittim,
the people
began
to commit sexual immorality
with the daughters
of Moab.
\VS{2}These women invited
the people
to the sacrifices
of their gods;
then the people
ate
and bowed
down to their gods.
\VS{3}When
Israel
joined
themselves to Baal-peor,
the anger
of the {\ND{Lord}}
flared up
against Israel.
\par }{\SH God’s Punishment
\par }{\PP \VS{4}The
{\ND{Lord}}
said
to
Moses,
“Arrest
all
the leaders
of the people,
and hang
them up
before the
{\ND{Lord}}
in broad
daylight,
so that the fierce
anger
of the {\ND{Lord}}
may be turned away
from Israel.”
\VS{5}So Moses
said
to
the judges
of Israel,
“Each
of you must execute those
of his men
who
were joined
to Baal-peor.”
\par }{\PP \VS{6}Just then
one
of the Israelites
came
and brought
to
his brothers
a Midianite
woman in the plain view
of Moses
and of the whole
community
of the Israelites,
while they
were weeping
at the entrance
of the tent
of meeting.
\VS{7}When
Phinehas
son
of Eleazar,
the son
of Aaron
the priest,
saw it, he got
up from among
the assembly,
took
a javelin
in his hand,
\VS{8}and went
after
the Israelite
man
into
the tent
and thrust through
the Israelite
man
and into
the woman’s
abdomen.
So the plague
was stopped
from
the Israelites.
\VS{9}Those that died
in the plague
were 24,000.
\par }{\SH The Aftermath
\par }{\PP \VS{10}The
{\ND{Lord}}
spoke
to
Moses:
\VS{11}“Phinehas
son
of Eleazar,
the son
of Aaron
the priest,
has turned
my anger
away from the
Israelites,
when he manifested
such zeal
for my sake among
them, so that I did not
consume
the Israelites
in my zeal.
\VS{12}Therefore,
announce: ‘I am going to give
to him my covenant
of peace.
\VS{13}So it will be
to him and his descendants
after
him a covenant
of a permanent
priesthood,
because he has
been zealous
for his God,
and has made atonement
for the Israelites.’ ”
\par }{\PP \VS{14}Now the name
of the Israelite
who
was stabbed
– the one who was stabbed
with
the Midianite woman – was Zimri son of Salu, a leader of a clan of the Simeonites.
\VS{15}The name
of the Midianite
woman
who was killed
was Cozbi
daughter
of Zur.
He was a leader
over the people
of a clan
of Midian.
\par }{\PP \VS{16}Then the
{\ND{Lord}}
spoke
to
Moses:
\VS{17}“Bring trouble
to the Midianites,
and destroy them,
\VS{18}because
they
bring trouble
to you by their treachery
with which
they have deceived
you in
the matter
of Peor,
and in
the matter
of Cozbi,
the daughter
of a prince
of Midian,
their sister,
who was killed
on the day
of the plague
that happened
as a result of Peor.”


\par }\Chap{26}{\PP \VerseOne{1}After
the plague
the {\ND{Lord}}
said
to
Moses
and to
Eleazar
son
of Aaron
the priest,
\VS{2}“Take
a census
of the whole
community
of Israelites,
from twenty
years
old and upward,
by their clans,
everyone
who can serve
in the army
of Israel.”
\VS{3}So Moses
and Eleazar
the priest
spoke
with
them in the plains
of Moab,
by
the Jordan River
across from Jericho.
They said,
\VS{4}“Number the people from twenty
years
old and upward,
just
as the
{\ND{Lord}}
commanded
Moses
and the Israelites
who went out
from the land
of Egypt.”
\par }{\SH Reuben
\par }{\PP \VS{5}Reuben
was the firstborn
of Israel.
The Reubenites: from Hanoch,
the family
of the Hanochites;
from Pallu,
the family
of the Palluites;
\VS{6}from Hezron,
the family
of the Hezronites;
from Carmi,
the family
of the Carmites.
\VS{7}These
were the families
of the Reubenites;
and those numbered
of them were 43,730.
\VS{8}Pallu’s
descendant was Eliab.
\VS{9}Eliab’s
descendants
were Nemuel,
Dathan,
and Abiram.
It was
Dathan
and Abiram
who as
leaders
of the community
rebelled
against
Moses
and Aaron
with the followers
of Korah
when they rebelled
against
the {\ND{Lord}}.
\VS{10}The earth
opened
its mouth
and swallowed
them and Korah
at the time that company
died,
when the fire
consumed
250
men.
So they became
a warning.
\VS{11}But the descendants
of Korah
did not
die.
\par }{\SH Simeon
\par }{\PP \VS{12}The Simeonites
by their families: from Nemuel,
the family
of the Nemuelites;
from Jamin,
the family
of the Jaminites;
from Jakin,
the family
of the Jakinites;
\VS{13}from Zerah,
the family
of the Zerahites;
and from Shaul,
the family
of the Shaulites.
\VS{14}These
were the families
of the Simeonites,
22,200.
\par }{\SH Gad
\par }{\PP \VS{15}The Gadites
by their families: from Zephon,
the family
of the Zephonites;
from Haggi,
the family
of the Haggites;
from Shuni,
the family
of the Shunites;
\VS{16}from Ozni,
the family
of the Oznites;
from Eri,
the family
of the Erites;
\VS{17}from Arod,
the family
of the Arodites,
and from Areli,
the family
of the Arelites.
\VS{18}These
were the families
of the Gadites
according to those numbered
of them, 40,500.
\par }{\SH Judah
\par }{\PP \VS{19}The descendants
of Judah
were Er
and Onan,
but Er
and Onan
died
in the land
of Canaan.
\VS{20}And the Judahites
by their families
were: from Shelah,
the family
of the Shelahites;
from Perez,
the family
of the Perezites;
and from Zerah,
the family
of the Zerahites.
\VS{21}And the Perezites
were: from Hezron,
the family
of the Hezronites;
from Hamul,
the family
of the Hamulites.
\VS{22}These
were the families
of Judah
according to those numbered
of them, 76,500.
\par }{\SH Issachar
\par }{\PP \VS{23}The Issacharites
by their families: from Tola,
the family
of the Tolaites;
from Puah,
the family
of the Puites;
\VS{24}from Jashub,
the family
of the Jashubites;
and from Shimron,
the family
of the Shimronites.
\VS{25}These
were the families
of Issachar,
according to those numbered
of them, 64,300.
\par }{\SH Zebulun
\par }{\PP \VS{26}The Zebulunites
by their families: from Sered,
the family
of the Sardites;
from Elon,
the family
of the Elonites;
from Jahleel,
the family
of the Jahleelites.
\VS{27}These
were the families
of the Zebulunites,
according to those numbered
of them, 60,500.
\par }{\SH Manasseh
\par }{\PP \VS{28}The descendants
of Joseph
by their families: Manasseh
and Ephraim.
\VS{29}The Manassehites: from Machir,
the family
of the Machirites
(now Machir
became the father
of Gilead); from Gilead,
the family
of the Gileadites.
\VS{30}These
were the Gileadites: from Iezer,
the family
of the Iezerites;
from Helek,
the family
of the Helekites;
\VS{31}from Asriel,
the family
of the Asrielites;
from Shechem,
the family
of the Shechemites;
\VS{32}from Shemida,
the family
of the Shemidaites;
from Hepher,
the family
of the Hepherites.
\VS{33}Now Zelophehad
son
of Hepher
had no
sons,
but only
daughters;
and the names
of the daughters
of Zelophehad
were Mahlah,
Noah,
Hoglah,
Milcah,
and Tirzah.
\VS{34}These
were the families
of Manasseh;
those numbered
of them were 52,700.
\par }{\SH Ephraim
\par }{\PP \VS{35}These
are the Ephraimites
by their families: from Shuthelah,
the family
of the Shuthelahites;
from Beker,
the family
of the Bekerites;
from Tahan,
the family
of the Tahanites.
\VS{36}Now these
were the Shuthelahites: from Eran,
the family
of the Eranites.
\VS{37}These
were the families
of the Ephraimites,
according to those numbered
of them, 32,500.
These
were the descendants
of Joseph
by their families.
\par }{\SH Benjamin
\par }{\PP \VS{38}The Benjaminites
by their families: from Bela,
the family
of the Belaites;
from Ashbel,
the family
of the Ashbelites;
from Ahiram,
the family
of the Ahiramites;
\VS{39}from Shupham,
the family
of the Shuphamites;
from Hupham,
the family
of the Huphamites.
\VS{40}The descendants
of Bela
were Ard
and Naaman.
From Ard, the family
of the Ardites;
from Naaman,
the family
of the Naamanites.
\VS{41}These
are the Benjaminites,
according to their families,
and according to those numbered
of them, 45,600.
\par }{\SH Dan
\par }{\PP \VS{42}These
are the Danites
by their families: from Shuham,
the family
of the Shuhamites.
These
were the families
of Dan,
according to their families.
\VS{43}All
the families
of the Shuhahites
according to those numbered
of them were 64,400.
\par }{\SH Asher
\par }{\PP \VS{44}The Asherites
by their families: from Imnah,
the family
of the Imnahites;
from Ishvi,
the family
of the Ishvites;
from Beriah,
the family
of the Beriahites.
\VS{45}From the Beriahites: from Heber,
the family
of the Heberites;
from Malkiel,
the family
of the Malkielites.
\VS{46}Now the name
of the daughter
of Asher
was Serah.
\VS{47}These
are the families
of the Asherites,
according to those numbered
of them, 53,400.
\par }{\SH Naphtali
\par }{\PP \VS{48}The Naphtalites
by their families: from Jahzeel,
the family
of the Jahzeelites;
from Guni,
the family
of the Gunites;
\VS{49}from Jezer,
the family
of the Jezerites;
from Shillem,
the family
of the Shillemites.
\VS{50}These
were the families
of Naphtali
according to their families;
and those numbered
of them were 45,400.
\par }{\SH Total Number and Division of the Land
\par }{\PP \VS{51}These
were those numbered
of the Israelites,
601,730.
\par }{\PP \VS{52}Then the
{\ND{Lord}}
spoke
to
Moses:
\VS{53}“To these
the land
must be divided
as an inheritance
according to the number
of the names.
\VS{54}To a
larger
group
you will give a
larger inheritance,
and to a smaller
group you will give a
smaller
inheritance.
To each
one its inheritance
must be given
according
to the number of people in it.
\VS{55}The land
must
be divided
by lot;
and they will inherit
in accordance with the names
of their ancestral
tribes.
\VS{56}Their inheritance
must be apportioned
by lot
among
the larger
and smaller groups.
\par }{\PP \VS{57}And these
are the Levites
who were numbered
according to their families: from Gershon,
the family
of the Gershonites;
of Kohath,
the family
of the Kohathites;
from Merari,
the family
of the Merarites.
\VS{58}These
are the families
of the Levites: the family
of the Libnites,
the family
of the Hebronites,
the family
of the Mahlites,
the family
of the Mushites,
the family
of the Korahites.
Kohath
became the father
of Amram.
\VS{59}Now the name
of Amram’s
wife
was Jochebed,
daughter
of Levi,
who
was born
to Levi
in Egypt.
And to Amram
she bore
Aaron,
Moses,
and Miriam
their sister.
\VS{60}And to Aaron
were born
Nadab
and Abihu,
Eleazar
and Ithamar.
\VS{61}But Nadab
and Abihu
died
when they offered
strange
fire
before
the {\ND{Lord}}.
\VS{62}Those of them who were
numbered
were 23,000,
all
males
from a month
old and upward,
for
they were not
numbered
among
the Israelites;
no
inheritance
was given
to them
among
the Israelites.
\par }{\PP \VS{63}These
are those who were numbered
by Moses
and Eleazar
the priest,
who
numbered
the Israelites
in the plains
of Moab
along the Jordan River
opposite Jericho.
\VS{64}But there was not
a man
among
these
who had
been among those
numbered by
Moses
and Aaron
the priest
when they numbered
the Israelites
in the wilderness
of Sinai.
\VS{65}For
the {\ND{Lord}}
had said
of them, “They will surely
die
in the wilderness.”
And there was not
left
a single
man
of them,
except
Caleb
son
of Jephunneh
and Joshua
son
of Nun.

\par }\Chap{27}{\PP \VerseOne{1}Then
the daughters
of Zelophehad
son
of Hepher,
the son
of Gilead,
the son
of Machir,
the son
of Manasseh
of the families
of Manasseh,
the son
Joseph
came forward. Now these
are the names
of his daughters: Mahlah,
Noah,
Hoglah,
Milcah,
and Tirzah.
\VS{2}And they stood
before
Moses
and Eleazar
the priest
and the leaders
of the whole
assembly
at the entrance
to the tent
of meeting
and said,
\VS{3}“Our father
died
in the wilderness,
although he was not
part
of the company
of those that gathered
themselves together against
the {\ND{Lord}}
in the company
of Korah;
but he died
for
his own sin,
and he had no
sons.
\VS{4}Why
should
the name
of our father
be lost
from among
his family
because
he had no
son? Give
us a possession
among
the relatives
of our father.”
\par }{\PP \VS{5}So
Moses
brought
their case
before
the {\ND{Lord}}.
\VS{6}The
{\ND{Lord}}
said
to
Moses:
\VS{7}“The daughters
of Zelophehad
have a valid
claim. You
must indeed give
them possession
of an inheritance
among
their father’s
relatives,
and you must transfer
the inheritance
of their father to them.
\VS{8}And you must
tell
the Israelites,
‘If
a man
dies
and has no
son,
then
you must transfer
his inheritance
to his daughter;
\VS{9}and if
he has no
daughter,
then you are to give
his inheritance
to his brothers;
\VS{10}and if
he has no
brothers,
then you are to give
his inheritance
to his father’s
brothers;
\VS{11}and if
his father
has no
brothers,
then you are to give
his inheritance
to his relative
nearest
to
him from his family,
and he will possess
it. This will be
for the Israelites
a legal
requirement,
as
the
{\ND{Lord}}
commanded
Moses.’ ”
\par }{\SH Leadership Change
\par }{\PP \VS{12}Then the
{\ND{Lord}}
said to
Moses,
“Go up
this
mountain
of the Abarim
range,
and see
the land
I have
given
to the Israelites.
\VS{13}When you have seen
it, you
will be gathered
to
your ancestors,
as Aaron
your brother
was gathered to his ancestors.
\VS{14}For
in the wilderness
of Zin
when the community
rebelled against
me, you rebelled
against my command
to show me as holy
before their eyes
over the water
– the water
of Meribah
in Kadesh
in the wilderness
of Zin.”
\par }{\PP \VS{15}Then Moses
spoke
to
the {\ND{Lord}}:
\VS{16}“Let
the {\ND{Lord}}, the God
of the spirits
of all
humankind,
appoint
a man
over
the community,
\VS{17}who
will go out
before
them, and who
will come
in before
them, and who
will lead them out,
and who
will bring
them in, so that
the community
of the {\ND{Lord}}
may not
be
like sheep
that
have no
shepherd.”
\par }{\PP \VS{18}The
{\ND{Lord}}
replied
to
Moses,
“Take
Joshua
son
of Nun,
a man
in whom
is such a spirit,
and lay
your hand
on him;
\VS{19}set
him before
Eleazar
the priest
and before
the whole
community,
and commission
him publicly.
\VS{20}Then
you must delegate
some of your authority to him, so that
the whole
community
of the Israelites
will be obedient.
\VS{21}And he will stand
before
Eleazar
the priest,
who will seek counsel
for him before
the {\ND{Lord}}
by the decision
of the Urim.
At his command they will go out,
and at his command
they will come
in, he
and all
the Israelites
with
him, the whole
community.”
\par }{\PP \VS{22}So Moses
did
as
the {\ND{Lord}}
commanded
him; he took
Joshua
and set
him before
Eleazar
the priest
and before
the whole
community.
\VS{23}He laid
his hands
on
him and commissioned
him, just
as the
{\ND{Lord}}
commanded,
by the authority
of Moses.


\par }\Chap{28}{\PP \VerseOne{1}The
{\ND{Lord}}
spoke
to
Moses:
\VS{2}“Command
the Israelites: ‘With regard
to
my offering,
be sure
to offer
my food
for my offering made by fire,
as a pleasing
aroma
to me at its appointed time.’
\VS{3}You will say
to them, ‘This
is the offering made by fire
which
you must offer
to the
{\ND{Lord}}: two
unblemished
lambs
one year
old each day
for a continual
burnt offering.
\VS{4}The first
lamb
you must offer
in the morning,
and the second
lamb
you must offer
in
the late afternoon,
\VS{5}with one-tenth
of an ephah
of finely ground flour
as a grain offering
mixed
with one quarter
of a hin
of pressed
olive oil.
\VS{6}It is a continual
burnt offering
that was instituted
on Mount
Sinai
as a pleasing
aroma,
an offering made by fire
to the
{\ND{Lord}}.
\par }{\PP \VS{7}“‘And its drink offering
must be one
quarter
of a hin
for each
lamb.
You must pour
out the strong drink
as a drink offering
to the
{\ND{Lord}}
in the holy place.
\VS{8}And the second
lamb
you must offer
in the late afternoon;
just as you offered the grain offering
and drink offering
in the morning,
you must offer
it as an offering made by fire,
as a pleasing
aroma
to the
{\ND{Lord}}.
\par }{\SH Weekly Offerings
\par }{\PP \VS{9}“‘On the Sabbath
day,
you must offer two
unblemished
lambs
a year
old, and two-tenths of an ephah
of finely ground flour
as a grain offering,
mixed
with olive oil,
along with its drink offering.
\VS{10}This is the burnt offering
for every Sabbath,
besides
the continual
burnt offering
and its drink offering.
\par }{\SH Monthly Offerings
\par }{\PP \VS{11}“‘On the first
day of each month
you must offer
as a burnt offering
to the
{\ND{Lord}}
two
young
bulls,
one
ram,
and seven
unblemished
lambs
a year old,
\VS{12}with three-tenths of an ephah
of finely ground flour
mixed
with olive oil
as a grain offering
for each
bull,
and two-tenths of an ephah
of finely ground flour
mixed
with olive oil
as a grain offering
for the ram,
\VS{13}and one-tenth of an ephah
of finely ground flour
mixed
with olive oil
as a grain offering
for each
lamb,
as a burnt offering
for a pleasing
aroma,
an offering made by fire
to the
{\ND{Lord}}.
\VS{14}For their drink offerings,
include half
a hin
of wine
with each bull,
one-third
of a
hin
for the ram,
and one-fourth
of a hin
for each lamb.
This
is the burnt offering
for each
month
throughout the months
of the year.
\VS{15}And one male
goat
must be offered to the
{\ND{Lord}}
as a purification
offering, in addition to the continual
burnt offering
and its drink offering.
\par }{\SH Passover and Unleavened Bread
\par }{\PP \VS{16}“‘On the fourteenth
day
of the first
month
is the
{\ND{Lord}}’s
Passover.
\VS{17}And on the fifteenth
day
of this
month
is the festival.
For seven
days
bread made without yeast
must be eaten.
\VS{18}And on
the first
day
there is to be a holy
assembly;
you must do
no
ordinary
work on it.
\par }{\PP \VS{19}“‘But you must offer
to the
{\ND{Lord}}
an offering made
by fire,
a burnt offering
of two
young
bulls,
one
ram,
and seven
lambs
one year
old; they must all be unblemished.
\VS{20}And their grain offering
is to be of finely ground flour
mixed
with olive oil.
For each bull
you must offer
three-tenths of an ephah,
and two-tenths
for the ram.
\VS{21}For each
of the seven
lambs
you are to offer
one-tenth of an ephah,
\VS{22}as well as one goat
for a purification
offering, to make atonement for you.
\VS{23}You must offer
these
in addition
to the burnt offering
in the morning
which
is for a continual
burnt offering.
\VS{24}In this
manner you must offer
daily
throughout the seven
days
the food
of the sacrifice made by fire
as a sweet
aroma
to the
{\ND{Lord}}. It is to be offered
in addition to the continual
burnt offering
and its drink offering.
\VS{25}On
the seventh
day
you are to have a holy
assembly,
you must do
no
regular
work.
\par }{\SH Firstfruits
\par }{\PP \VS{26}“‘Also, on the day
of the first fruits,
when you bring
a new
grain offering
to the
{\ND{Lord}}
during your Feast of Weeks,
you are to have a holy
assembly.
You must do
no
ordinary
work.
\VS{27}But you must offer
as the burnt offering,
as a sweet
aroma
to the
{\ND{Lord}},
two
young
bulls,
one
ram,
seven
lambs
one year old,
\VS{28}with their grain offering
of finely ground flour
mixed
with olive oil: three-tenths of an ephah
for each
bull,
two-tenths
for the one
ram,
\VS{29}with one-tenth
for each
of the seven
lambs,
\VS{30}as well as one male
goat
to make an
atonement for you.
\VS{31}You are to offer
them with their drink offerings
in addition
to the continual
burnt offering
and its grain offering
– they must be
unblemished.


\par }\Chap{29}{\PP \VerseOne{1}“‘On the first
day
of the seventh
month,
you are to hold a holy
assembly.
You must not
do
your ordinary
work,
for
it is a day
of blowing trumpets for you.
\VS{2}You must offer
a burnt offering
as a sweet
aroma
to the
{\ND{Lord}}: one young bull,
one
ram,
and seven
lambs
one
year
old without blemish.
\par }{\PP \VS{3}“‘Their grain offering
is to be of finely ground flour
mixed
with olive oil,
three-tenths of an ephah
for the bull,
two-tenths of an ephah
for the ram,
\VS{4}and one-tenth
for each
of the seven
lambs,
\VS{5}with one male
goat
for a purification
offering to make an atonement for you;
\VS{6}this is in addition
to the monthly
burnt offering
and its grain offering,
and the daily
burnt offering
with its grain offering
and their drink offerings
as prescribed,
as a sweet
aroma,
a sacrifice made by fire
to the
{\ND{Lord}}.
\par }{\SH The Day of Atonement
\par }{\PP \VS{7}“‘On the tenth
day of this
seventh
month
you are to have a holy
assembly.
You must humble
yourselves;
you must not
do
any
work on it.
\VS{8}But you must offer
a burnt offering
as a pleasing
aroma
to the
{\ND{Lord}}, one young
bull,
one
ram,
and seven
lambs
one
year
old, all of them without blemish.
\VS{9}Their grain offering
must be of finely ground flour
mixed
with olive oil,
three-tenths of an ephah
for the bull,
two-tenths
for the ram,
\VS{10}and one-tenth
for each
of the seven
lambs,
\VS{11}along with one male
goat
for a
purification offering,
in addition
to the purification
offering for atonement
and the continual
burnt offering
with its grain offering
and their drink offerings.
\par }{\SH The Feast of Temporary Shelters
\par }{\PP \VS{12}“‘On
the fifteenth
day
of the seventh
month
you are to have a holy
assembly;
you must do no ordinary
work,
and you must keep a
festival
to the
{\ND{Lord}}
for seven
days.
\VS{13}You must offer
a burnt offering,
an offering made by fire
as a pleasing
aroma
to the
{\ND{Lord}}: thirteen
young
bulls,
two
rams,
and fourteen
lambs
each one year
old, all of them without blemish.
\VS{14}Their grain offering
must be of finely ground flour
mixed
with olive oil,
three-tenths of an ephah
for each
of the thirteen
bulls,
two-tenths of an ephah
for each
of the two
rams,
\VS{15}and one-tenth
for each
of the fourteen
lambs,
\VS{16}along with one
male
goat
for a purification
offering, in addition
to the continual
burnt offering
with its grain offering
and its drink offering.
\par }{\PP \VS{17}“‘On the second
day
you must offer twelve
young
bulls,
two
rams,
fourteen
lambs
one year
old, all without blemish,
\VS{18}and their grain offering
and their drink offerings
for the bulls,
for the rams,
and for the lambs,
according to their number
as prescribed,
\VS{19}along with one
male
goat
for a purification
offering, in addition
to the continual
burnt offering
with its grain offering
and their drink offerings.
\par }{\PP \VS{20}“‘On
the third
day
you must offer eleven
bulls,
two
rams,
fourteen
lambs
one year
old, all without blemish,
\VS{21}and their grain offering
and their drink offerings
for the bulls,
for the rams,
and for the lambs,
according to their number
as prescribed,
\VS{22}along with one
male goat
for a purification
offering, in addition
to the continual
burnt offering
with its grain offering
and its drink offering.
\par }{\PP \VS{23}“‘On the fourth
day
you must offer ten
bulls,
two
rams,
and fourteen
lambs
one year
old, all without blemish,
\VS{24}and their grain offering
and their drink offerings
for the bulls,
for the rams,
and for the lambs,
according to their number
as prescribed,
\VS{25}along with one
male
goat
for a purification
offering, in addition
to the continual
burnt offering
with its grain offering
and its drink offering.
\par }{\PP \VS{26}“‘On the fifth
day
you must offer nine
bulls,
two
rams,
and fourteen
lambs
one year
old,
all without blemish,
\VS{27}and their grain offering
and their drink offerings
for the bulls,
for the rams,
and for the lambs,
according to their number
as prescribed,
\VS{28}along with one
male goat
for a purification
offering, in addition
to the continual
burnt offering
with its grain offering
and its drink offering.
\par }{\PP \VS{29}“‘On
the sixth
day
you must offer eight
bulls,
two
rams,
and fourteen
lambs
one year
old, all without blemish,
\VS{30}and their grain offering
and their drink offerings
for the bulls,
for the rams,
and for the lambs,
according to their number
as prescribed,
\VS{31}along with one
male goat
for a purification
offering, in addition
to the continual
burnt offering
with its grain offering
and its drink offering.
\par }{\PP \VS{32}“‘On
the seventh
day
you must offer seven
bulls,
two
rams,
and fourteen
lambs
one year
old, all without blemish,
\VS{33}and their grain offering
and their drink offerings
for the bulls,
for the rams,
and for the lambs,
according to their number
as prescribed,
\VS{34}along with one
male goat
for a purification
offering, in addition
to the continual
burnt offering
with its grain offering
and its drink offering.
\par }{\PP \VS{35}“‘On
the eighth
day
you are to have a holy assembly;
you must do
no
ordinary
work on it.
\VS{36}But you must offer
a burnt offering,
an offering made by fire,
as a pleasing
aroma
to the
{\ND{Lord}}, one bull,
one
ram,
seven
lambs
one
year
old, all of them without blemish,
\VS{37}and with their grain offering
and their drink offerings
for the bull,
for the ram,
and for the lambs,
according to their number
as prescribed,
\VS{38}along with one
male goat
for a purification
offering, in addition
to the continual
burnt offering
with its grain offering
and its drink offering.
\par }{\PP \VS{39}“‘These
things you must present
to the
{\ND{Lord}}
at your appointed
times, in addition
to your vows
and your freewill
offerings, as your burnt offerings,
your grain offerings,
your drink offerings,
and your peace offerings.’ ”
\VS{40}So Moses
told
the Israelites
everything,
just as
the {\ND{Lord}}
had commanded
him.

\par }\Chap{30}{\PP \VerseOne{1}Moses
told
the leaders
of the tribes
concerning the Israelites,
“This
is what
the {\ND{Lord}}
has
commanded:
\VS{2}If
a man
makes
a vow
to the
{\ND{Lord}}
or
takes
an oath
of binding
obligation
on
himself,
he must not
break
his word,
but must do
whatever
he has promised.
\par }{\SH Vows Made by Single Women
\par }{\PP \VS{3}“If
a young
woman
who is still living in her father’s
house
makes
a vow
to the
{\ND{Lord}}
or places
herself under an obligation,
\VS{4}and her father
hears
of her vow
or the obligation
to which
she has pledged
herself,
and her father
remains silent
about her, then all
her vows
will stand,
and every
obligation
to which
she has pledged
herself
will stand.
\VS{5}But if
her father
overrules
her when
he hears
about it, then none
of her vows
or her obligations
which
she has pledged
for
herself
will stand.
And the
{\ND{Lord}}
will release
her from it, because
her father
overruled her.
\par }{\SH Vows Made by Married Women
\par }{\PP \VS{6}“And if
she marries
a husband
while under a vow,
or
she uttered
anything impulsively
by which
she has pledged
herself,
\VS{7}and her husband
hears
about it, but remains silent
about her when
he hears
about it, then her vows
will stand
and her obligations
which
she has pledged
for
herself
will stand.
\VS{8}But if
when
her husband
hears
it he overrules
her, then he will nullify
the vow
she has taken, and whatever
she uttered
impulsively
which
she has pledged
for herself.
And the
{\ND{Lord}}
will release her from it.
\par }{\SH Vows Made by Widows
\par }{\PP \VS{9}“But every vow
of a widow
or of a divorced
woman which
she has pledged
for
herself
will remain intact.
\VS{10}If
she made the vow
in her husband’s
house
or
put
herself
under obligation
with an oath,
\VS{11}and her husband
heard
about it, but remained silent
about her, and did not
overrule
her, then all
her vows
will stand,
and every
obligation
which
she pledged
for herself
will stand.
\VS{12}But if
her husband clearly
nullifies
them when
he hears
them, then whatever
she says
by way of vows
or obligations
will not
stand.
Her husband
has made them void,
and the
{\ND{Lord}}
will release her from them.
\par }{\PP \VS{13}“Any
vow
or sworn
obligation
that would bring affliction
to her,
her husband
can confirm
or nullify.
\VS{14}But if
her husband
remains completely silent
about her from day
to
day,
he thus confirms
all
her vows
or
all
her obligations
which
she is under; he confirms
them because
he remained silent
about when
he heard them.
\VS{15}But if
he should nullify
them after
he has heard
them, then he will bear
her iniquity.”
\par }{\PP \VS{16}These
are the statutes
that
the {\ND{Lord}}
commanded
Moses,
relating to
a man
and his wife,
and a father
and his young
daughter
who is still living in her father’s
house.


\par }\Chap{31}{\PP \VerseOne{1}The
{\ND{Lord}}
spoke
to
Moses:
\VS{2}“Exact vengeance
for the Israelites
on the Midianites –
after
that you will be gathered
to
your people.”
\par }{\PP \VS{3}So Moses
spoke
to
the people: “Arm
men
from among you
for the war,
to attack
the Midianites
and to execute
the
{\ND{Lord}}’s
vengeance
on Midian.
\VS{4}You must send
to the battle
a thousand
men from every tribe
throughout all
the tribes
of Israel.”
\VS{5}So a thousand
from every tribe,
twelve
thousand
armed
for battle
in all, were provided
out of the thousands
of Israel.
\par }{\SH Campaign Against the Midianites
\par }{\PP \VS{6}So Moses
sent
them to the war,
one thousand
from every tribe,
with Phinehas
son
of Eleazar
the priest,
who was in charge
of the holy
articles
and the signal
trumpets.
\VS{7}They fought
against
the Midianites,
as
the {\ND{Lord}}
commanded
Moses,
and they killed
every
male.
\VS{8}They killed
the kings
of Midian
in addition
to those slain
– Evi,
Rekem,
Zur,
Hur,
and Reba
– five
Midianite
kings.
They also killed
Balaam
son
of Beor
with the sword.
\par }{\PP \VS{9}The Israelites
took the women
of Midian
captives
along
with their little ones,
and took
all
their herds,
all
their flocks,
and all
their goods
as plunder.
\VS{10}They burned
all
their towns
where
they lived
and all
their encampments.
\VS{11}They took
all
the plunder
and all
the spoils,
both people
and animals.
\VS{12}They brought
the
captives
and the
spoils
and the
plunder
to
Moses,
to
Eleazar
the priest,
and to
the Israelite
community,
to
the camp
on
the plains
of Moab,
along
the Jordan
River across from Jericho.
\VS{13}Moses,
Eleazar
the priest,
and all
the leaders
of the community
went out to meet
them
outside
the camp.
\par }{\SH The Death of the Midianite Women
\par }{\PP \VS{14}But Moses
was furious
with the officers
of the army,
the commanders
over thousands
and commanders
over hundreds,
who had come
from service
in the war.
\VS{15}Moses
said
to
them, “Have you allowed all
the women
to live?
\VS{16}Look,
these
people through the counsel
of Balaam
caused
the Israelites
to act treacherously
against
the {\ND{Lord}}
in the matter
of Peor
– which resulted
in the plague
among the community
of the {\ND{Lord}}!
\VS{17}Now
therefore kill
every
boy,
and kill
every
woman
who has had sexual intercourse
with a man.
\VS{18}But all
the young
women
who
have not
had sexual intercourse
with a man
will be yours.
\par }{\SH Purification After Battle
\par }{\PP \VS{19}“Any
of you
who has killed
anyone or touched
any
of the dead,
remain
outside
the camp
for seven
days;
purify
yourselves
and your captives
on
the third
day,
and on
the seventh
day.
\VS{20}You must purify
each garment
and everything
that is made
of skin,
everything
made
of goat’s
hair, and everything
made
of wood.”
\par }{\PP \VS{21}Then Eleazar
the priest
said
to
the men
of war
who had gone
into the battle,
“This
is the ordinance
of the law
that
the
{\ND{Lord}}
commanded
Moses:
\VS{22}‘Only
the gold,
the silver,
the bronze,
the iron,
the tin,
and the
lead,
\VS{23}everything
that
may stand
the fire,
you are to pass through
the fire,
and it will be ceremonially clean,
but
it must still be purified
with the water
of purification.
Anything
that
cannot
withstand
the fire
you must pass through
the water.
\VS{24}You must wash
your clothes
on
the seventh
day,
and you will be ceremonially clean,
and afterward
you may enter
the camp.’ ”
\par }{\SH The Distribution of Spoils
\par }{\PP \VS{25}Then the
{\ND{Lord}}
spoke
to Moses:
\VS{26}“You
and Eleazar
the priest,
and all the family
leaders
of the community,
take
the sum
of the plunder
that was captured,
both people
and animals.
\VS{27}Divide
the
plunder
into two parts,
one for those who took
part in the war
– who went out
to battle
– and the other for all
the community.
\par }{\PP \VS{28}“You must exact
a tribute
for the
{\ND{Lord}}
from the fighting
men
who went out
to battle: one
life
out of five
hundred,
from
the people,
the cattle,
and from
the donkeys
and the sheep.
\VS{29}You are to take
it from their half-share
and give
it to Eleazar
the priest
for a raised
offering to the
{\ND{Lord}}.
\VS{30}From the Israelites’
half-share
you are to take
one
portion
out of
fifty
of the people,
the cattle,
the donkeys,
and the sheep
– from
every
kind of animal
– and you are to give
them to the Levites,
who are responsible
for the care
of the
{\ND{Lord}}’s
tabernacle.”
\par }{\PP \VS{31}So
Moses
and Eleazar
the priest
did as
the {\ND{Lord}}
commanded
Moses.
\VS{32}The spoil
that remained
of the plunder
which
the fighting
men
had gathered
was 675,000
sheep,
\VS{33}72,000
cattle,
\VS{34}61,000
donkeys,
\VS{35}and 32,000
young women
who
had never
had sexual intercourse
with a man.
\par }{\PP \VS{36}The half-portion
of those who went
to war
numbered
337,500
sheep;
\VS{37}the
{\ND{Lord}}’s
tribute
from
the sheep
was 675.
\VS{38}The cattle
numbered 36,000;
the
{\ND{Lord}}’s
tribute
was 72.
\VS{39}The donkeys
were 30,500,
of which the
{\ND{Lord}}’s
tribute
was 61.
\VS{40}The people
were 16,000,
of which the
{\ND{Lord}}’s
tribute
was 32
people.
\par }{\PP \VS{41}So Moses
gave
the tribute,
which was the
{\ND{Lord}}’s
raised
offering, to Eleazar
the priest,
as
the {\ND{Lord}}
commanded
Moses.
\par }{\PP \VS{42}From the Israelites’
half-share
that
Moses
had
separated
from
the fighting
men,
\VS{43}there
were 337,500
sheep
from
the portion
belonging to the community,
\VS{44}36,000
cattle,
\VS{45}30,500
donkeys,
\VS{46}and 16,000
people.
\par }{\PP \VS{47}From the Israelites’
share
Moses
took
one
of
every fifty
people
and animals
and gave
them to the
Levites
who were responsible
for the care
of the Lord’s
tabernacle,
just
as the
{\ND{Lord}}
commanded
Moses.
\par }{\PP \VS{48}Then the officers who were over
the thousands
of the army,
the commanders
over thousands
and the commanders
over hundreds,
approached
Moses
\VS{49}and said
to
him, “Your servants
have taken
a count
of the men
who were in the battle,
who
were under our authority,
and not one
is missing.
\VS{50}So we have brought
as an offering
for the
{\ND{Lord}}
what
each man
found: gold
ornaments,
armlets,
bracelets,
signet rings,
earrings,
and necklaces,
to make atonement
for ourselves
before
the {\ND{Lord}}.”
\VS{51}Moses
and Eleazar
the priest
took
the gold
from them,
all
of it in the form
of ornaments.
\VS{52}All
the gold
of the offering
they offered up
to the
{\ND{Lord}}
from the commanders
of thousands
and the
commanders
of hundreds
weighed 16,750
shekels.
\VS{53}Each
soldier
had taken plunder for himself.
\VS{54}So
Moses
and Eleazar
the priest
received the gold
from the
commanders
of thousands
and commanders of hundreds
and brought
it into
the
tent
of meeting
as a memorial
for the Israelites
before
the {\ND{Lord}}.


\par }\Chap{32}{\PP \VerseOne{1}Now
the Reubenites
and the Gadites
possessed
a very
large
number of cattle.
When they saw
that the
lands
of Jazer
and Gilead
were ideal
for cattle,
\VS{2}the Gadites
and the Reubenites
came
and addressed
Moses,
Eleazar
the priest,
and the leaders
of the community.
They said,
\VS{3}“Ataroth,
Dibon,
Jazer,
Nimrah,
Heshbon,
Elealeh,
Sebam,
Nebo,
and Beon,
\VS{4}the land
that
the {\ND{Lord}}
subdued
before
the community
of Israel,
is ideal for cattle,
and your servants
have cattle.”
\VS{5}So they said,
“If
we have found
favor
in your sight,
let
this
land
be given to your servants
for our inheritance.
Do not
have us cross
the Jordan River.”
\par }{\SH Moses’ Response
\par }{\PP \VS{6}Moses
said
to the Gadites
and the Reubenites,
“Must your brothers
go
to war
while you
remain
here?
\VS{7}Why
do you frustrate
the intent
of the Israelites
to cross
over into
the land
which
the {\ND{Lord}}
has given them?
\VS{8}Your fathers
did
the same thing when I sent
them from Kadesh Barnea
to see
the
land.
\VS{9}When they went up
to
the Eshcol
Valley
and saw
the
land,
they frustrated
the intent
of the
Israelites
so that they did not
enter
the land
that
the {\ND{Lord}}
had given them.
\VS{10}So the anger
of the {\ND{Lord}}
was kindled
that day,
and he
swore,
\VS{11}‘Because
they have not
followed
me wholeheartedly,
not
one of the men
twenty
years
old
and upward who
came from
Egypt
will see
the land
that
I swore
to give to Abraham,
Isaac,
and Jacob,
\VS{12}except
Caleb
son
of Jephunneh
the Kenizzite,
and Joshua
son
of Nun,
for
they followed
the {\ND{Lord}}
wholeheartedly.’
\VS{13}So the
{\ND{Lord}}’s
anger
was kindled
against the Israelites,
and he made them wander
in the wilderness
for forty
years,
until
all
that generation
that had done
wickedly
before
the {\ND{Lord}} was finished.
\VS{14}Now look,
you are standing
in your fathers’
place,
a brood
of sinners,
to increase
still
further the fierce
wrath
of the
{\ND{Lord}}
against the Israelites.
\VS{15}For
if you turn
away from following
him, he will once again
abandon
them in the wilderness,
and you will be the reason for their destruction.”
\par }{\SH The Offer of the Reubenites and Gadites
\par }{\PP \VS{16}Then they came very close
to him
and said,
“We will build
sheep
folds
here
for our flocks
and cities
for our families,
\VS{17}but we
will maintain ourselves in armed
readiness
and go before
the Israelites
until
whenever
we have
brought
them to
their place.
Our descendants
will be living
in fortified
towns
as a protection against the inhabitants
of the land.
\VS{18}We will not
return
to
our homes
until
every
Israelite
has his inheritance.
\VS{19}For
we will not
accept any inheritance
on the other side
of the Jordan River
and beyond,
because
our inheritance
has come
to us
on this eastern
side
of the Jordan.”
\par }{\PP \VS{20}Then Moses
replied, “If
you will do
this
thing,
and if
you will arm
yourselves for battle
before
the {\ND{Lord}},
\VS{21}and if all
your armed
men cross
the Jordan
before
the {\ND{Lord}}
until
he drives out
his enemies
from his presence
\VS{22}and the land
is subdued
before
the {\ND{Lord}}, then afterward
you may return
and be
free of your obligation
to the
{\ND{Lord}}
and to Israel.
This
land
will then be
your possession
in the
{\ND{Lord}}’s
sight.
\par }{\PP \VS{23}“But if
you do not
do
this,
then look,
you will have sinned
against the
{\ND{Lord}}. And know
that your sin
will find you out.
\VS{24}So build
cities
for your descendants
and pens
for your sheep,
but do what you have said
you would do.”
\par }{\PP \VS{25}So the Gadites
and the Reubenites
replied
to
Moses,
“Your servants
will do
as
my lord
commands.
\VS{26}Our children,
our wives,
our flocks,
and all
our livestock
will be
there
in the cities
of Gilead,
\VS{27}but your servants
will cross
over, every
man armed
for war,
to do battle
in the
{\ND{Lord}}’s
presence,
just
as my lord
says.”
\par }{\PP \VS{28}So Moses
gave orders
about them
to Eleazar
the priest,
to Joshua
son
of Nun,
and to the
heads
of the families
of the Israelite
tribes.
\VS{29}Moses
said
to
them: “If
the Gadites
and the Reubenites
cross
the
Jordan
with you,
each one equipped
for battle
in the
{\ND{Lord}}’s
presence,
and you conquer
the land,
then you
must
allot
them the territory
of Gilead
as their possession.
\VS{30}But if
they do not
cross
over with
you armed,
they must receive possessions
among
you in Canaan.”
\VS{31}Then
the Gadites
and the Reubenites
answered,
“Your servants
will do
what the
{\ND{Lord}}
has
spoken.
\VS{32}We
will cross
armed
in the
{\ND{Lord}}’s
presence
into the land
of Canaan,
and then
the possession
of our inheritance
that we inherit will be ours on this side
of the Jordan River.”
\par }{\SH Land Assignment
\par }{\PP \VS{33}So Moses
gave
to the Gadites,
the Reubenites,
and to half
the tribe
of Manasseh
son
of Joseph
the
realm
of King
Sihon
of the Amorites,
and the realm
of King
Og
of Bashan,
the entire land
with its cities
and the territory
surrounding them.
\VS{34}The Gadites
rebuilt
Dibon,
Ataroth,
Aroer,
\VS{35}Atroth Shophan,
Jazer,
Jogbehah,
\VS{36}Beth Nimrah,
and Beth Haran
as fortified
cities,
and constructed pens
for their flocks.
\VS{37}The Reubenites
rebuilt
Heshbon,
Elealeh,
Kiriathaim,
\VS{38}Nebo,
Baal Meon
(with a change
of name), and Sibmah.
They renamed
the cities
they built.
\par }{\PP \VS{39}The descendants
of Machir
son
of Manasseh
went
to Gilead,
took it,
and dispossessed
the Amorites
who were in it.
\VS{40}So Moses
gave
Gilead
to Machir,
son
of Manasseh,
and he lived there.
\VS{41}Now Jair
son
of Manasseh
went
and captured
their small towns
and named
them Havvoth Jair.
\VS{42}Then Nobah
went
and captured
Kenath
and its
villages
and called
it Nobah
after his own name.


\par }\Chap{33}{\PP \VerseOne{1}These
are the journeys
of the Israelites,
who
went out
of the land
of Egypt
by their divisions
under the authority
of Moses
and Aaron.
\VS{2}Moses
recorded
their departures
according to their journeys,
by
the commandment
of the {\ND{Lord}}; now these
are their journeys
according to their departures.
\VS{3}They departed
from Rameses
in the first
month,
on the fifteenth
day
of the first
month;
on the day after
the Passover
the Israelites
went out
defiantly
in plain sight
of all
the Egyptians.
\VS{4}Now the Egyptians
were burying
all
their firstborn,
whom
the {\ND{Lord}}
had killed
among them; the
{\ND{Lord}}
also executed
judgments
on their gods.
\par }{\PP \VS{5}The Israelites
traveled
from Rameses
and camped
in Succoth.
\par }{\PP \VS{6}They traveled
from Succoth,
and camped
in Etham,
which
is on the edge
of the wilderness.
\VS{7}They traveled
from Etham,
and turned
again to Pi-hahiroth,
which
is before
Baal-Zephon;
and they camped
before
Migdal.
\VS{8}They traveled
from Pi-hahiroth,
and passed
through the middle
of the sea
into the wilderness,
and went
three
days’
journey
in the wilderness
of Etham,
and camped
in Marah.
\VS{9}They traveled
from Marah
and came
to Elim;
in Elim
there are twelve
fountains
of water
and seventy
palm
trees, so they camped
there.
\par }{\PP \VS{10}They traveled
from Elim,
and camped
by
the Red
Sea.
\VS{11}They traveled
from the Red
Sea
and camped
in the wilderness
of Zin.
\VS{12}They traveled
from the wilderness
of Zin
and camped
in Dophkah.
\VS{13}And they traveled
from Dophkah,
and camped
in Alush.
\par }{\PP \VS{14}They traveled
from Alush
and camped
at Rephidim,
where there
was no
water
for the people
to drink.
\VS{15}They traveled
from Rephidim
and camped
in the wilderness
of Sinai.
\par }{\SH Wanderings in the Wilderness
\par }{\PP \VS{16}They traveled
from the desert
of Sinai
and camped
at Kibroth Hattaavah.
\VS{17}They traveled
from Kibroth Hattaavah
and camped
at Hazeroth.
\VS{18}They traveled
from Hazeroth
and camped
in Rithmah.
\VS{19}They traveled
from Rithmah
and camped
at Rimmon-perez.
\VS{20}They traveled
from Rimmon-perez
and camped
in Libnah.
\VS{21}They traveled
from Libnah
and camped
at Rissah.
\VS{22}They traveled
from Rissah
and camped
in Kehelathah.
\VS{23}They traveled
from Kehelathah
and camped
at Mount
Shepher.
\VS{24}They traveled
from Mount
Shepher
and camped
in Haradah.
\VS{25}They traveled
from Haradah
and camped
in Makheloth.
\VS{26}They traveled
from Makheloth
and camped
at Tahath.
\VS{27}They traveled
from Tahath
and camped
at Terah.
\VS{28}They traveled
from Terah
and camped
in Mithcah.
\VS{29}They traveled
from Mithcah
and camped
in Hashmonah.
\VS{30}They traveled
from Hashmonah
and camped
in Moseroth.
\VS{31}They traveled
from Moseroth
and camped
in Bene-jaakan.
\VS{32}They traveled
from Bene-jaakan
and camped
at Hor-haggidgad.
\VS{33}They traveled
from Hor-haggidgad
and camped
in Jotbathah.
\VS{34}They traveled
from Jotbathah
and camped
in Abronah.
\VS{35}They traveled
from Abronah
and camped
at Ezion-geber.
\VS{36}They traveled
from Ezion-geber
and camped
in the wilderness
of Zin,
which is
Kadesh.
\par }{\SH Wanderings from Kadesh to Moab
\par }{\PP \VS{37}They traveled
from Kadesh
and camped
in Mount
Hor
at the edge
of the land
of Edom.
\VS{38}Aaron
the priest
ascended
Mount
Hor
at the command
of the {\ND{Lord}}, and he died
there
in the fortieth
year
after the Israelites
had come out
of the land
of Egypt
on the first
day of the fifth
month.
\VS{39}Now Aaron
was 123
years
old
when he died
in Mount
Hor.
\VS{40}The king
of Arad,
the Canaanite
king who lived in
the south
of the land
of Canaan,
heard
about the approach
of the Israelites.
\par }{\PP \VS{41}They traveled
from Mount
Hor
and camped
in Zalmonah.
\VS{42}They traveled
from Zalmonah
and camped
in Punon.
\VS{43}They traveled
from Punon
and camped
in Oboth.
\VS{44}They traveled
from Oboth
and camped
in Iye-abarim,
on the border
of Moab.
\VS{45}They traveled
from Iim
and camped
in Dibon-gad.
\VS{46}They traveled
from Dibon-gad
and camped
in Almon-diblathaim.
\VS{47}They traveled
from Almon-diblathaim
and camped
in the mountains
of Abarim
before
Nebo.
\VS{48}They traveled
from the mountains
of Abarim
and camped
in the plains
of Moab
by
the Jordan River
across from Jericho.
\VS{49}They camped
by
the Jordan,
from Beth-jeshimoth
as far
as Abel-shittim
in the plains
of Moab.
\par }{\SH At the Border of Canaan
\par }{\PP \VS{50}The
{\ND{Lord}}
spoke
to
Moses
in the plains
of Moab
by
the Jordan,
across from Jericho.
He said:
\VS{51}“Speak
to
the Israelites
and tell
them,
‘When
you
have crossed
the Jordan
into
the land
of Canaan,
\VS{52}you must drive out
all
the inhabitants
of the land
before
you. Destroy
all
their carved images,
all
their molten
images,
and demolish
their high places.
\VS{53}You must
dispossess
the inhabitants of the land
and live
in it, for
I have given
you the land
to possess it.
\VS{54}You must divide
the
land
by lot
for an inheritance among
your families.
To a larger
group
you must give a larger inheritance,
and to
a smaller
group you must give a smaller
inheritance.
Everyone’s
inheritance must be
in the place where
his lot
falls. You must inherit
according to
your ancestral
tribes.
\VS{55}But if
you do not
drive out
the inhabitants
of the land
before
you, then those whom you allow to remain
will be
irritants
in your eyes
and thorns
in your side,
and will cause you trouble
in
the land
where
you
will be living.
\VS{56}And what
I intended
to do
to them I will do to you.”


\par }\Chap{34}{\PP \VerseOne{1}Then the
{\ND{Lord}}
spoke
to
Moses:
\VS{2}“Give these instructions
to the Israelites,
and tell
them: ‘When
you
enter
Canaan,
the land
that
has been assigned
to you as an inheritance,
the land
of Canaan
with its borders,
\VS{3}your southern
border will
extend from
the wilderness
of Zin
along
the Edomite
border,
and your
southern
border
will
run eastward
to the extremity of the Salt
Sea,
\VS{4}and then the border
will turn
from the south
to the Scorpion Ascent,
continue
to Zin,
and then its direction
will be
from the south
to Kadesh Barnea.
Then it will go
to Hazar Addar
and pass over
to Azmon.
\VS{5}There the border
will turn
from Azmon
to the Brook
of Egypt,
and then its direction
is to the sea.
\par }{\SH The Western Border of the Land
\par }{\PP \VS{6}“‘And for a western
border
you will
have the Great
Sea.
This
will be
your western
border.
\par }{\SH The Northern Border of the Land
\par }{\PP \VS{7}“‘And this
will be
your northern
border: From
the Great
Sea
you will draw a line
to Mount
Hor;
\VS{8}from Mount
Hor
you will draw a line
to Lebo Hamath,
and the direction
of the border
will be
to Zedad.
\VS{9}The border
will continue
to Ziphron,
and its direction
will be
to Hazar Enan.
This
will be
your northern
border.
\par }{\SH The Eastern Border of the Land
\par }{\PP \VS{10}“‘For your eastern border
you will draw
a line
from Hazar Enan
to Shepham.
\VS{11}The border
will run down
from Shepham
to Riblah,
on the east
side of Ain,
and the border
will descend
and reach
the eastern
side
of the Sea
of Chinnereth.
\VS{12}Then the border
will continue down
the Jordan River
and its direction
will be
to the Salt
Sea.
This
will be
your land
by its borders
that surround it.’ ”
\par }{\PP \VS{13}Then Moses
commanded
the
Israelites: “This
is the land
which
you will inherit
by lot,
which
the {\ND{Lord}}
has commanded
to be given
to the nine
and a half
tribes,
\VS{14}because
the tribe
of the Reubenites
by their families,
the tribe
of the Gadites
by their families,
and half
of the tribe
of Manasseh
have received
their inheritance.
\VS{15}The two
and a half
tribes
have received
their inheritance
on this side
of the Jordan,
east
of Jericho,
toward the sunrise.”
\par }{\SH Appointed Officials
\par }{\PP \VS{16}The
{\ND{Lord}}
said
to
Moses:
\VS{17}“These
are the names
of the men
who
are to allocate
the land
to you as an inheritance: Eleazar
the priest
and Joshua
son
of Nun.
\VS{18}You must take
one
leader
from every tribe
to assist in allocating
the land as an inheritance.
\VS{19}These
are the names
of the men: from the tribe
of Judah,
Caleb
son
of Jephunneh;
\VS{20}from the tribe
of the Simeonites,
Shemuel
son
of Ammihud;
\VS{21}from the tribe
of Benjamin,
Elidad
son
of Kislon;
\VS{22}and from the tribe
of the Danites,
a leader,
Bukki
son
of Jogli.
\VS{23}From the Josephites,
Hanniel
son
of Ephod,
a leader
from the tribe
of Manasseh;
\VS{24}from the tribe
of the Ephraimites,
a leader,
Kemuel
son
of Shiphtan;
\VS{25}from the tribe
of the Zebulunites,
a leader,
Elizaphan
son
of Parnach;
\VS{26}from the tribe of the Issacharites,
a leader,
Paltiel
son
of Azzan;
\VS{27}from the tribe
of the Asherites,
a leader,
Ahihud
son
of Shelomi;
\VS{28}and from the tribe
of the Naphtalites,
a leader,
Pedahel
son
of Ammihud.”
\VS{29}These
are the ones whom
the {\ND{Lord}}
commanded
to divide up the inheritance
among the Israelites
in the land
of Canaan.


\par }\Chap{35}{\PP \VerseOne{1}Then the
{\ND{Lord}}
spoke
to
Moses
in the Moabite
plains
by the Jordan
near Jericho.
He said:
\VS{2}“Instruct
the
Israelites
to give
the Levites
towns
to live
in from the inheritance
the Israelites will possess.
You must also give
the Levites
grazing land
around
the towns.
\VS{3}Thus they will
have towns
in which to live,
and their grazing lands
will be
for their cattle,
for their possessions,
and for all
their animals.
\VS{4}The grazing lands
around
the towns
that
you will give
to the Levites
must
extend to a distance of 500
yards
from the town
wall.
\par }{\PP \VS{5}“You must measure
from outside
the wall
of the town
on the east
1,000 yards,
and on the south
side
1,000 yards,
and on the west
side
1,000 yards,
and on the north
side
1,000 yards,
with the town
in the middle.
This
territory must belong
to them as grazing land
for the towns.
\VS{6}Now from these towns
that
you will give
to the Levites
you must select six
towns
of refuge
to which
a person who has killed someone may
flee.
And you must give
them forty-two
other towns.
\par }{\PP \VS{7}“So the total
of the towns
you will give
the Levites
is forty-eight.
You must give these together with their grazing lands.
\VS{8}The towns
you will give
must be from the possession
of the Israelites.
From the larger
tribes you must give more;
and from the smaller
tribes fewer. Each
must contribute
some of its own
towns
to
the Levites
in proportion to the inheritance
allocated to each.
\par }{\SH The Cities of Refuge
\par }{\PP \VS{9}Then the
{\ND{Lord}}
spoke
to
Moses:
\VS{10}“Speak
to
the Israelites
and tell
them, ‘When
you
cross
over the Jordan River
into the land
of Canaan,
\VS{11}you must then designate
some towns
as towns
of refuge
for you, to which a person
who has killed
someone
unintentionally
may flee.
\VS{12}And they must stand as your towns
of refuge
from the avenger
in order that the killer
may not
die
until
he has
stood
trial
before
the community.
\VS{13}These towns
that
you must give
shall be your six
towns
for refuge.
\par }{\PP \VS{14}“You must
give
three
towns
on this side
of the Jordan,
and you must
give
three
towns
in the land
of Canaan;
they must be
towns
of refuge.
\VS{15}These
six
towns
will be
places of refuge
for the Israelites,
and for the foreigner,
and for the settler
among
them, so that anyone
who kills
any person
accidentally
may flee
there.
\par }{\PP \VS{16}“But if
he hits someone
with an iron
tool
so
that he dies,
he is a murderer.
The murderer
must surely be put
to death.
\VS{17}If
he strikes
him by throwing a stone
large enough
that
he could
die,
and he dies,
he is a murderer.
The murderer
must surely be put to death.
\VS{18}Or
if he strikes
him
with
a wooden
hand
weapon
so that he could die,
and he dies,
he is a murderer.
The murderer
must surely be put to death.
\VS{19}The avenger
of blood
himself
must kill
the murderer;
when he meets
him, he must kill him.
\par }{\PP \VS{20}“But if
he strikes
him out
of hatred
or
throws
something at him intentionally
so that he dies,
\VS{21}or
with enmity
he strikes
him with his hand
and he dies,
the one who struck
him must
surely be put to death,
for he is
a murderer.
The avenger
of blood
must kill
the
murderer
when he meets him.
\par }{\PP \VS{22}“But if
he strikes
him suddenly,
without
enmity,
or
throws
anything
at him unintentionally,
\VS{23}or
with any
stone
large enough that
a man could die,
without seeing
him, and throws
it at
him,
and he dies,
even though
he was not
his enemy
nor
sought
his harm,
\VS{24}then the community
must judge
between
the slayer
and the avenger
of blood
according
to these
decisions.
\VS{25}The community
must deliver
the slayer
out of the hand
of the avenger
of blood,
and the
community
must restore
him to
the town
of refuge
to which
he fled,
and he must live
there
until
the death
of the high
priest,
who
was anointed
with the consecrated
oil.
\VS{26}But if
the slayer
at any time
goes outside
the boundary
of the town
to which
he had fled,
\VS{27}and the avenger
of blood
finds
him outside
the borders
of the town
of refuge,
and the avenger
of blood
kills
the slayer,
he will not
be guilty of blood,
\VS{28}because
the slayer should have stayed
in his town
of refuge
until
the death
of the high
priest.
But after
the death
of the high
priest,
the slayer
may return
to
the land
of his possessions.
\VS{29}So these
things must be a statutory
ordinance
for you throughout your generations,
in all
the places where you live.
\par }{\PP \VS{30}“Whoever kills
any person,
the murderer
must be put to death by the testimony
of witnesses;
but one
witness
cannot
testify
against any person
to cause him to be put to death.
\VS{31}Moreover, you must not
accept
a ransom
for the life
of a murderer
who
is
guilty
of death;
he must
surely
be put to death.
\VS{32}And you must not
accept
a ransom
for anyone who has fled
to
a town
of refuge,
to allow him to
return
home and live
on his own land
before
the death
of the high priest.
\par }{\PP \VS{33}“You must not
pollute
the
land
where
you
live, for
blood
defiles
the
land,
and the land
cannot
be cleansed
of the blood
that is
shed
there, except
by the blood
of the person
who
shed it.
\VS{34}Therefore do not
defile
the land
that
you
will inhabit,
in which
I
live,
for
I
the {\ND{Lord}}
live
among
the Israelites.”


\par }\Chap{36}{\PP \VerseOne{1}Then the heads
of the family
groups
of the Gileadites,
the descendant
of Machir, the descendant
of Manasseh,
who were from the Josephite
families,
approached
and spoke
before
Moses
and the leaders
who were the heads
of the Israelite
families.
\VS{2}They said,
“The

{\ND{Lord}}
commanded
my lord
to give
the
land
as an inheritance
by lot
to the Israelites;
and my lord
was commanded
by the
{\ND{Lord}}
to give
the
inheritance
of our brother
Zelophehad
to his daughters.
\VS{3}Now if
they should be married
to one
of the men from another Israelite
tribe,
their inheritance
would be taken
from the inheritance
of our fathers
and added
to the
inheritance
of the tribe
into which
they marry. As a result, it will be
taken
from the lot
of our inheritance.
\VS{4}And when
the Jubilee
of the Israelites
is to take place, their inheritance
will be added
to the inheritance
of the tribe
into which
they marry. So their inheritance
will be
taken
away from the inheritance
of our ancestral
tribe.”
\par }{\SH Moses’ Decision
\par }{\PP \VS{5}Then Moses
gave a ruling
to the Israelites
by
the word
of the {\ND{Lord}}: “What the tribe
of the Josephites
is saying
is right.
\VS{6}This
is what
the {\ND{Lord}}
has
commanded
for Zelophehad’s
daughters: ‘Let them marry
whomever they think
best,
only
they must marry
within the family
of their father’s
tribe.
\VS{7}In this way
the inheritance
of the Israelites
will not
be transferred
from tribe
to
tribe.
But
every one
of the Israelites
must retain
the ancestral
heritage.
\VS{8}And every
daughter
who possesses
an inheritance
from any of the tribes
of the Israelites
must become
the wife
of a man from any
family
in her father’s
tribe,
so that every Israelite
may
retain
the inheritance
of his fathers.
\VS{9}No
inheritance
may pass from tribe
to tribe.
But
every one of the tribes
of the Israelites
must retain
its inheritance.”
\par }{\PP \VS{10}As
the
{\ND{Lord}}
had commanded
Moses,
so
the daughters
of Zelophehad
did.
\VS{11}For the daughters
of Zelophehad
– Mahlah,
Tirzah,
Hoglah,
Milcah,
and Noah
– were married
to the sons
of their uncles.
\VS{12}They were
married
into the families
of the Manassehites,
the descendants
of Joseph,
and their inheritance
remained
in
the tribe
of their father’s
family.
\par }{\PP \VS{13}These
are the commandments
and the decisions
that
the {\ND{Lord}}
commanded
the Israelites
through the authority
of Moses,
on
the plains
of Moab
by the Jordan River
opposite Jericho.
\par }