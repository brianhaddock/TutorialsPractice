\NormalFont\ShortTitle{Exodus}
{\MT Exodus

\par }\ChapOne{1}{\SH Blessing during Bondage in Egypt
\par }{\PP \VerseOne{1}These
are the names
of the sons
of Israel
who entered
Egypt
– each man
with his household
entered
with
Jacob:
\VS{2}Reuben,
Simeon,
Levi,
and Judah,
\VS{3}Issachar,
Zebulun,
and Benjamin,
\VS{4}Dan
and Naphtali,
Gad
and Asher.
\VS{5}All
the people who
were directly
descended
from Jacob
numbered seventy.
But Joseph
was already
in Egypt,
\VS{6}and in time Joseph
and his brothers
and all
that generation
died.
\VS{7}The Israelites,
however, were fruitful,
increased greatly,
multiplied,
and became
extremely
strong,
so that the land
was filled
with them.
\par }{\PP \VS{8}Then
a new
king,
who
did not
know
about Joseph,
came to power over
Egypt.
\VS{9}He said
to
his people,
“Look
at the Israelite
people,
more numerous
and stronger
than we are!
\VS{10}Come, let’s
deal wisely
with them. Otherwise
they will continue to multiply,
and if
a war
breaks
out, they will ally
themselves with our enemies
and fight
against us
and leave
the country.”
\par }{\PP \VS{11}So they put
foremen
over
the Israelites to
oppress
them with hard labor.
As a result they built
Pithom
and Rameses
as store
cities
for Pharaoh.
\VS{12}But
the more
the Egyptians oppressed
them, the more
they multiplied
and spread.
As a result
the Egyptians loathed
the Israelites,
\VS{13}and they made the Israelites
serve
rigorously.
\VS{14}They made their lives
bitter
by hard
service
with mortar
and bricks
and by all
kinds of service
in the fields.
Every
kind of service
the Israelites were required to give was rigorous.
\par }{\PP \VS{15}The king
of Egypt
said
to the Hebrew
midwives,
one
of whom
was named
Shiphrah
and the other
Puah,
\VS{16}“When you assist
the Hebrew
women in childbirth,
observe
at the delivery: If
it is a son,
kill
him, but if
it is a daughter,
she
may live.”
\VS{17}But the midwives
feared
God
and did not
do
what the king
of Egypt
had
told
them;
they let the boys
live.
\par }{\PP \VS{18}Then the king
of Egypt
summoned
the midwives
and said
to them, “Why
have you done
this
and let the boys
live?”
\VS{19}The midwives
said
to
Pharaoh,
“Because
the Hebrew women
are not
like the Egyptian
women
– for
the Hebrew women are vigorous;
they
give birth
before
the midwife
gets
to them!”
\VS{20}So God
treated
the midwives
well,
and the people
multiplied
and became very strong.
\VS{21}And because
the midwives
feared
God,
he made
households for them.
\par }{\PP \VS{22}Then Pharaoh
commanded
all
his people,
“All
sons
that are born
you must throw
into the river,
but all
daughters
you may let live.”

\par }\Chap{2}{\PP \VerseOne{1}A man
from the household
of Levi
married
a woman who was a descendant
of Levi.
\VS{2}The woman
became pregnant
and gave birth
to a son.
When she saw
that
he was a healthy
child, she hid
him for three
months.
\VS{3}But when she was no
longer
able
to hide
him, she took
a papyrus
basket
for him and sealed it with bitumen
and pitch.
She put
the
child
in it and set
it among the reeds
along the edge
of the Nile.
\VS{4}His sister
stationed
herself at a distance
to find
out what
would happen to him.
\par }{\PP \VS{5}Then the daughter
of Pharaoh
came down
to wash
herself by
the Nile,
while her attendants
were walking
alongside
the river,
and she saw
the
basket
among
the reeds.
She sent
one of her attendants,
took it,
\VS{6}opened
it, and saw
the child –
a boy,
crying! – and she felt compassion for him and said, “This is one of the Hebrews’ children.”
\par }{\PP \VS{7}Then his sister
said to
Pharaoh’s
daughter,
“Shall I go
and get a nursing
woman
for you from
the Hebrews,
so that she may nurse
the child for you?”
\VS{8}Pharaoh’s
daughter
said
to her, “Yes, do so.” So
the young girl went
and got
the child’s
mother.
\VS{9}Pharaoh’s
daughter
said
to her, “Take this
child
and nurse
him for me, and I
will pay
your wages.”
So the woman
took
the child
and nursed him.
\par }{\PP \VS{10}When the child
grew
older she brought
him to Pharaoh’s
daughter,
and he became
her son.
She named
him Moses,
saying,
“Because
I drew
him from
the water.”
\par }{\SH The Presumption of the Deliverer
\par }{\PP \VS{11}In those
days,
when Moses
had grown up,
he went out
to
his people
and observed
their hard labor,
and he saw
an Egyptian
man
attacking
a Hebrew
man,
one of his own people.
\VS{12}He looked
this
way
and that and saw
that
no
one
was there, and then he attacked
the
Egyptian
and concealed
the body in the sand.
\VS{13}When he went out
the next
day,
there
were two
Hebrew
men
fighting.
So he said
to the one
who was in the wrong, “Why
are you attacking
your fellow
Hebrew?”
\par }{\PP \VS{14}The man
replied,
“Who
made
you a ruler
and a judge
over
us? Are you
planning
to kill
me like
you killed
that Egyptian?” Then Moses
was afraid,
thinking, “Surely
what I did
has become known.”
\VS{15}When Pharaoh
heard
about this
event,
he sought
to kill
Moses.
So Moses
fled
from Pharaoh
and settled
in the land
of Midian,
and he settled
by a certain well.
\par }{\PP \VS{16}Now a priest
of Midian
had seven
daughters,
and they came
and began to draw
water and fill
the troughs
in order to water
their father’s
flock.
\VS{17}When some shepherds
came
and drove
them away, Moses
came up
and defended
them and then watered
their flock.
\VS{18}So when
they came
home to their father
Reuel,
he asked,
“Why
have you come
home so early
today?”
\VS{19}They said,
“An Egyptian
man
rescued
us from the shepherds,
and he actually
drew water
for us and watered
the flock!”
\VS{20}He said
to
his daughters,
“So where
is he? Why
in the world
did you leave
the
man? Call
him, so that he may eat
a meal with us.”
\par }{\PP \VS{21}Moses
agreed
to stay
with
the man,
and he gave
his daughter
Zipporah
to Moses in marriage.
\VS{22}When she bore
a son,
Moses named
him Gershom,
for
he said,
“I have become
a resident foreigner
in a foreign
land.”
\par }{\SH The Call of the Deliverer
\par }{\PP \VS{23}During
that long
period of time
the king
of Egypt
died,
and the Israelites
groaned
because of the slave labor.
They cried out,
and their desperate cry
because of their slave labor
went up
to
God.
\VS{24}God
heard
their groaning,
God
remembered
his covenant
with
Abraham,
with
Isaac,
and with
Jacob,
\VS{25}God
saw
the Israelites,
and God
understood….


\par }\Chap{3}{\PP \VerseOne{1}Now Moses
was shepherding
the
flock
of his father-in-law
Jethro,
the priest
of Midian,
and he led
the flock
to the far
side of the desert
and came
to
the mountain
of God,
to Horeb.
\VS{2}The angel
of the {\ND{Lord}}
appeared
to him
in a flame
of fire
from within
a bush.
He looked –
and the bush
was ablaze
with fire,
but it was not
being consumed!
\VS{3}So Moses
thought, “I will turn
aside
to see
this
amazing sight. Why
does the bush
not
burn up?”
\VS{4}When the
{\ND{Lord}}
saw
that
he had turned aside
to look,
God
called
to him
from within
the bush
and said,
“Moses, Moses!” And Moses
said,
“Here I am.”
\VS{5}God said,
“Do not
approach
any closer! Take your sandals
off your feet,
for
the place
where
you
are standing
is holy
ground.”
\VS{6}He added,
“I am
the God
of your father,
the God
of Abraham,
the God
of Isaac,
and the God
of Jacob.”
Then Moses
hid
his face,
because
he was afraid
to look
at God.
\par }{\PP \VS{7}The
{\ND{Lord}}
said,
“I have surely seen
the affliction
of my people
who
are in Egypt.
I have heard
their cry
because
of their taskmasters,
for
I know
their sorrows.
\VS{8}I have come down
to deliver
them from the hand
of the Egyptians
and to bring them up
from
that land
to
a land
that is
both good
and spacious,
to
a land
flowing
with milk
and honey,
to
the region
of the Canaanites,
Hittites,
Amorites,
Perizzites,
Hivites,
and Jebusites.
\VS{9}And now
indeed
the cry
of the Israelites
has come
to
me, and I have also
seen
how severely
the Egyptians
oppress them.
\VS{10}So now
go,
and I will send
you to
Pharaoh
to bring
my people,
the Israelites,
out of Egypt.”
\par }{\PP \VS{11}Moses
said
to
God, “Who
am
I, that
I should go
to
Pharaoh,
or that
I should
bring
the Israelites
out of Egypt?”
\VS{12}He replied, “Surely
I will be
with
you, and this
will be the sign
to you that
I
have sent
you: When you bring
the
people
out of Egypt,
you and they will serve
God
on
this
mountain.”
\par }{\PP \VS{13}Moses
said
to
God,
“If I
go
to
the Israelites
and tell
them, ‘The God
of your fathers
has sent
me to
you,’ and they ask
me, ‘What
is his name?’ – what should I say to them?”
\par }{\PP \VS{14}God
said
to
Moses,
“{\ND{I am}}
that
{\ND{I am}}.” And he said, “You must say this
to the Israelites,
‘{\ND{I am}}
has sent
me to you.’ ”
\VS{15}God
also
said
to
Moses,
“You must
say
this
to
the Israelites,
‘The
{\ND{Lord}} –
the God
of your fathers,
the God
of Abraham,
the God
of Isaac,
and the God
of Jacob
– has sent
me to
you. This
is my name
forever,
and this
is my memorial
from generation
to generation.’
\par }{\PP \VS{16}“Go
and bring together
the elders
of Israel
and tell
them, ‘The
{\ND{Lord}}, the God
of your fathers,
appeared
to me – the God of Abraham, Isaac, and Jacob – saying, “I have attended carefully to you and to what has been done to you in Egypt,
\VS{17}and I have promised that I will bring you up
out of the affliction
of Egypt
to
the land
of the Canaanites,
Hittites,
Amorites,
Perizzites,
Hivites,
and Jebusites,
to
a land
flowing
with milk
and honey.” ’
\par }{\PP \VS{18}“The elders will listen
to you, and then you
and the elders
of Israel
must go
to
the king
of Egypt
and tell
him, ‘The
{\ND{Lord}}, the God
of the Hebrews,
has met
with us. So now,
let
us go
three
days’
journey
into the wilderness,
so that we may sacrifice
to the
{\ND{Lord}}
our God.’
\VS{19}But I
know
that
the king
of Egypt
will not
let
you go,
not
even under force.
\VS{20}So I will extend
my hand
and strike
Egypt
with all
my wonders
that
I will do
among
them, and after
that
he will release
you.
\par }{\PP \VS{21}“I
will grant
this
people
favor
with the Egyptians,
so that when
you depart
you will not
leave
empty-handed.
\VS{22}Every woman
will ask
her neighbor
and the one who happens to be staying
in her house
for items
of silver
and gold
and for clothing.
You will put
these articles
on
your sons
and daughters
– thus you
will plunder Egypt!” ‘span class=”footnote” id=”footnote-65” ’‘span class=”key” ’65‘a href=”\#note-65” class=”backref” ’3:19‘/a’‘span class=”text” ’
{\IT{tn}}:
{\IT{Heb}} “and not with a mighty hand.” This expression ({\TL{וְלֹא בְּיָד חֲזָקָה}}, vÿlo’ vÿyad khazaqa) is unclear, since v.
20 says that God will stretch out his hand and do his wonders. Some have taken v.
19b to refer to God’s mighty hand also, meaning that the king would not let them go unless a mighty hand compels him (NIV). The expression “mighty hand” is used of God’s rescuing Israel elsewhere (
Exod 6:1, 13:9, 32:11; but note also
Num 20:20). This idea is a rather general interpretation of the words; it owes much to the LXX, which has “except by a mighty hand,” though “and not with” does not have the meaning of “except” or “unless” in other places. In view of these difficulties, others have suggested that v.
19b means “strong [threats]” from the Israelites (as in
4:24ff. and
5:3; see B. Jacob,
{\IT{Exodus}}, 81). This does not seem as convincing as the first view. Another possibility is that the phrase conveys Pharaoh’s point of view and intention; the Lord knows that Pharaoh plans to resist letting the Israelites go, regardless of the exercise of a strong hand against him (P. Addinall, “Exodus III 19B and the Interpretation of Biblical Narrative,”
{\IT{VT}} 49 [1999]: 289-300; see also the construction “and not with” in
Num 12:8;
1 Sam 20:15 and elsewhere). If that is the case, v.
20 provides an ironic and pointed contradiction to Pharaoh’s plans as the Lord announces the effect that his hand will have. At any rate, Pharaoh will have to be forced to let Israel go.

\par }\Chap{4}{\PP \VerseOne{1}Moses
answered
again, “And if
they do not
believe
me or
pay attention
to me,
but
say,
‘The
{\ND{Lord}}
has not
appeared
to you’?”
\VS{2}The
{\ND{Lord}}
said
to
him, “What is
that in your hand?” He said,
“A staff.”
\VS{3}The
{\ND{Lord}} said,
“Throw
it to the ground.”
So he threw
it to the ground,
and it became
a snake,
and Moses
ran from it.
\VS{4}But the
{\ND{Lord}}
said
to
Moses,
“Put out
your hand
and grab
it by the tail”
– so he put out
his hand
and caught
it, and it became
a staff
in his hand –
\VS{5}“that
they may believe
that
the {\ND{Lord}}, the God
of their fathers,
the God
of Abraham,
the God
of Isaac,
and the God
of Jacob,
has appeared
to you.”
\par }{\PP \VS{6}The
{\ND{Lord}}
also
said
to him, “Put
your hand
into your robe.”
So he put his hand
into his robe,
and when he brought
it out
– there was his hand,
leprous
like snow!
\VS{7}He said,
“Put your hand
back
into your robe.”
So he put his hand
back
into his robe,
and when he brought it out
from his robe
– there
it was, restored
like the rest of his skin!
\VS{8}“If
they do not
believe
you or
pay attention
to the former
sign,
then they may believe
the latter
sign.
\VS{9}And if
they do not
believe
even
these
two
signs
or
listen
to you,
then take
some water
from the Nile
and pour
it out
on the dry ground.
The water
you take out
of the Nile
will become
blood
on the dry ground.”
\par }{\PP \VS{10}Then Moses
said
to
the {\ND{Lord}}, “O
my Lord,
I am
not
an eloquent
man,
neither
in the past
nor
since
you have spoken
to
your servant,
for
I am
slow
of speech
and slow
of tongue.”
\par }{\PP \VS{11}The
{\ND{Lord}}
said
to
him, “Who
gave
a mouth
to man,
or
who
makes
a person mute
or
deaf
or
seeing
or
blind? Is it not
I,
the {\ND{Lord}}?
\VS{12}So now
go,
and I
will be
with
your mouth
and will teach
you what
you must say.”
\par }{\PP \VS{13}But Moses said, “O
my Lord,
please
send
anyone else whom you wish to send!”
\par }{\PP \VS{14}Then the
{\ND{Lord}}
became angry
with Moses,
and he said,
“What about your brother
Aaron
the Levite? I know
that
he can
speak
very well. Moreover,
he is
coming
to meet
you, and when he sees
you he will be glad
in his heart.
\par }{\PP \VS{15}“So you are to speak
to
him and put
the
words
in his mouth.
And as for me, I
will be
with
your mouth
and with
his mouth,
and I will teach
you both what
you must do.
\VS{16}He
will speak
for you to
the people,
and it will be
as if he were your mouth
and as if you
were
his God.
\VS{17}You will also take
in your hand
this
staff,
with which
you will do
the signs.”
\par }{\SH The Return of Moses
\par }{\PP \VS{18}So Moses
went
back
to
his father-in-law
Jethro
and said
to him, “Let me go,
so that I may return
to
my relatives
in Egypt
and see
if they are still
alive.”
Jethro
said
to Moses,
“Go
in peace.”
\VS{19}The
{\ND{Lord}}
said
to
Moses
in Midian,
“Go
back
to Egypt,
because
all
the men
who were seeking
your life
are dead.”
\VS{20}Then
Moses
took
his wife
and sons
and put
them on
a donkey
and headed
back
to the land
of Egypt,
and Moses
took
the
staff
of God
in his hand.
\VS{21}The
{\ND{Lord}}
said
to
Moses,
“When you go
back
to Egypt,
see
that
you do
before
Pharaoh
all
the wonders
I have put
under your control.
But I
will harden
his heart
and he will not
let
the
people go.
\VS{22}You must say
to
Pharaoh,
‘Thus
says
the {\ND{Lord}}, “Israel
is my son,
my firstborn,
\VS{23}and I said
to
you, ‘Let
my son
go that he may serve
me,’ but since you have refused
to let him
go, I
will surely kill
your son,
your firstborn!” ’ ”
\par }{\PP \VS{24}Now
on the way,
at a place where they stopped
for the night, the
{\ND{Lord}}
met
Moses and sought
to kill him.
\VS{25}But Zipporah
took
a flint
knife, cut off
the foreskin
of her son
and touched
it to Moses’ feet,
and said,
“Surely
you
are a bridegroom
of blood to me.”
\VS{26}So the
{\ND{Lord}} let
him alone.
(At that time
she said,
“A bridegroom
of blood,”
referring to the circumcision.)
\par }{\PP \VS{27}The
{\ND{Lord}}
said
to
Aaron,
“Go
to the wilderness
to meet
Moses.
So he went
and met
him at the mountain
of God
and greeted him with a kiss.
\VS{28}Moses
told
Aaron
all
the words
of the {\ND{Lord}}
who had
sent
him and all
the signs
that
he had
commanded him.
\VS{29}Then Moses
and Aaron
went
and brought together
all
the Israelite
elders.
\VS{30}Aaron
spoke
all
the words
that
the {\ND{Lord}}
had spoken
to
Moses
and did
the signs
in the sight
of the people,
\VS{31}and the people
believed.
When they heard
that
the {\ND{Lord}}
had attended
to the Israelites
and that
he had seen
their affliction,
they bowed down close to the ground.


\par }\Chap{5}{\PP \VerseOne{1}Afterward
Moses
and Aaron
went
to
Pharaoh
and said,
“Thus
says the
{\ND{Lord}}, the God
of Israel,
‘Release
my people
so that they may hold a pilgrim
feast to me in the desert.’ ”
\VS{2}But Pharaoh
said,
“Who
is the
{\ND{Lord}}
that
I should obey
him
by releasing
Israel? I do not
know
the {\ND{Lord}}, and I will not
release
Israel!”
\VS{3}And they said,
“The God
of the Hebrews
has met
with us. Let us go
a three-day
journey
into the desert
so that we may sacrifice
to the
{\ND{Lord}}
our God,
so
that he does not strike
us with plague
or
the sword.”
\VS{4}The king
of Egypt
said
to
them, “Moses
and Aaron,
why
do you cause the people
to refrain
from their work? Return
to your labor!”
\VS{5}Pharaoh
was thinking, “The people
of the land
are now
many,
and you are giving them rest
from their labor.”
\par }{\PP \VS{6}That same day
Pharaoh
commanded
the slave masters
and foremen
who were over the people:
\VS{7}“You must no
longer
give
straw
to the people
for making
bricks
as before.
Let
them
go
and collect
straw for themselves.
\VS{8}But you must require of them the same quota
of bricks
that
they
were making
before.
Do not
reduce
it, for
they are
slackers.
That is why
they
are crying,
‘Let us go
sacrifice
to our God.’
\VS{9}Make
the work harder
for the men
so they will keep
at it and pay no
attention
to lying
words!”
\par }{\PP \VS{10}So
the slave
masters
of the people
and their foremen
went to
the Israelites and said, “Thus
says
Pharaoh: ‘I am not
giving
you straw.
\VS{11}You
go
get
straw
for yourselves wherever
you can find
it, because
there will be no
reduction
at all in your workload.’ ”
\VS{12}So
the people
spread
out through all
the land
of Egypt
to collect
stubble
for straw.
\VS{13}The slave masters
were pressuring
them, saying,
“Complete
your work
for each
day,
just
like when
there was straw!”
\VS{14}The Israelite
foremen
whom
Pharaoh’s
slave masters
had set
over
them were beaten
and were asked, “Why
did you not
complete
your requirement
for brickmaking
as in the past
– both
yesterday
and today?”
\VS{15}\par }{\PP The Israelite
foremen
went
and cried out
to
Pharaoh,
“Why
are you treating
your servants
this way?
\VS{16}No
straw
is given
to your servants,
but we are told, ‘Make
bricks!’ Your servants
are even
being beaten,
but the fault
is with your people.”
\par }{\PP \VS{17}But Pharaoh
replied, “You are slackers! Slackers! That is why
you
are saying,
‘Let us go
sacrifice
to the
{\ND{Lord}}.’
\VS{18}So now,
get back
to work! You will not
be given
straw,
but you must still produce your quota
of bricks!”
\VS{19}The Israelite
foremen
saw
that
they were in trouble
when
they were told, “You must not
reduce
the daily
quota of your bricks.”
\par }{\PP \VS{20}When they went out
from Pharaoh,
they encountered
Moses
and Aaron
standing
there to meet them,
\VS{21}and they said
to
them, “May the
{\ND{Lord}}
look
on
you and judge,
because
you have made us stink
in the opinion
of
Pharaoh
and his servants,
so that you have given
them an excuse
to kill us!”
\par }{\SH The Assurance of Deliverance
\par }{\PP \VS{22}Moses
returned
to
the {\ND{Lord}}, and said,
“Lord,
why
have you caused
trouble for this
people? Why
did you ever
send me?
\VS{23}From the time
I went
to
speak
to
Pharaoh
in your name,
he has caused
trouble for this
people,
and you have certainly
not
rescued them!”


\par }\Chap{6}{\PP \VerseOne{1}Then the
{\ND{Lord}}
said to
Moses,
“Now
you will see
what
I will do
to Pharaoh,
for
compelled
by my strong hand
he will release
them, and by my strong
hand
he will drive
them out of his land.”
\par }{\PP \VS{2}God
spoke
to
Moses
and said
to him,
“I am
the {\ND{Lord}}.
\VS{3}I appeared
to
Abraham,
to
Isaac,
and to
Jacob
as God
Almighty,
but by my name
‘the
{\ND{Lord}}’
I was not
known to them.
\VS{4}I also
established
my covenant
with
them to give
them the land
of Canaan,
where they were living
as
resident foreigners.
\VS{5}I
have also
heard
the groaning
of the Israelites,
whom
the Egyptians
are enslaving,
and I have remembered
my covenant.
\VS{6}Therefore,
tell
the Israelites,
‘I am
the {\ND{Lord}}. I will bring you out
from your enslavement
to the Egyptians,
I will rescue
you from the hard labor
they impose, and I will redeem
you with an outstretched
arm
and with great
judgments.
\VS{7}I will take
you to myself for a people,
and I will be
your God.
Then you will know
that
I am
the {\ND{Lord}}
your God,
who brought you out
from your enslavement
to the Egyptians.
\VS{8}I will bring
you to
the land
I swore
to give
to Abraham,
to Isaac,
and to Jacob
– and I will give
it to you as a possession.
I am
the {\ND{Lord}}{\SC{!’}}”
\VS{9}\par }{\PP Moses
told
this
to
the Israelites,
but they did not
listen
to
him
because of their discouragement
and hard
labor.
\VS{10}Then
the {\ND{Lord}}
said
to
Moses,
\VS{11}“Go,
tell
Pharaoh
king
of Egypt
that he must release
the Israelites
from his land.”
\VS{12}But Moses
replied
to
the {\ND{Lord}}, “If
the Israelites
did not
listen
to
me, then how
will Pharaoh
listen
to me, since I speak with difficulty?”
\par }{\PP \VS{13}The
{\ND{Lord}}
spoke
to
Moses
and Aaron
and gave them a charge
for the Israelites
and Pharaoh
king
of Egypt
to bring
the Israelites
out of the land
of Egypt.
\par }{\SH The Ancestry of the Deliverer
\par }{\PP \VS{14}These
are the heads
of their fathers’
households:

\par }{\PP The sons
of Reuben,
the firstborn son
of Israel,
were Hanoch
and Pallu,
Hezron
and Carmi.
These
were the clans
of Reuben.
\par }{\PP \VS{15}The sons
of Simeon
were Jemuel,
Jamin,
Ohad,
Jakin,
Zohar,
and Shaul,
the son
of a Canaanite
woman. These
were the clans
of Simeon.
\par }{\PP \VS{16}Now these
are the names
of the sons
of Levi,
according to their records: Gershon,
Kohath,
and Merari.
(The length of Levi’s
life
was 137
years.)
\par }{\PP \VS{17}The sons
of Gershon,
by their families,
were Libni
and Shimei.
\par }{\PP \VS{18}The sons
of Kohath
were Amram,
Izhar,
Hebron,
and Uzziel.
(The length
of Kohath’s
life
was 133
years.)
\par }{\PP \VS{19}The sons
of Merari
were Mahli
and Mushi.
These
were the clans
of Levi,
according to their records.
\par }{\PP \VS{20}Amram
married
his father’s sister
Jochebed,
and she
bore
him Aaron
and Moses.
(The length
of Amram’s
life
was 137
years.)
\par }{\PP \VS{21}The sons
of Izhar
were Korah,
Nepheg,
and Zikri.
\par }{\PP \VS{22}The sons
of Uzziel
were Mishael,
Elzaphan,
and Sithri.
\par }{\PP \VS{23}Aaron
married
Elisheba,
the daughter
of Amminadab
and sister
of Nahshon,
and she bore
him Nadab
and Abihu,
Eleazar
and Ithamar.
\par }{\PP \VS{24}The sons
of Korah
were Assir,
Elkanah,
and Abiasaph.
These
were the Korahite
clans.
\par }{\PP \VS{25}Now Eleazar
son
of Aaron
married
one of the daughters
of Putiel
and she bore
him Phinehas.
\par }{\PP These
are the heads
of the fathers’
households of Levi
according to their clans.
\par }{\PP \VS{26}It was the same
Aaron
and Moses
to whom
the {\ND{Lord}}
said,
“Bring
the Israelites
out of the land
of Egypt
by
their regiments.”
\VS{27}They
were the men who were speaking
to
Pharaoh
king
of Egypt,
in order to bring
the Israelites
out of Egypt.
It was the same
Moses
and Aaron.
\par }{\SH The Authentication of the Word
\par }{\PP \VS{28}When
the {\ND{Lord}}
spoke
to
Moses
in the land
of Egypt,
\VS{29}he said
to
him, “I am
the {\ND{Lord}}. Tell
Pharaoh
king
of Egypt
all
that
I
am telling you.”
\VS{30}But Moses
said
before
the {\ND{Lord}}, “Since
I
speak
with difficulty,
why
should Pharaoh
listen
to me?”

\par }\Chap{7}{\PP \VerseOne{1}So the
{\ND{Lord}}
said
to
Moses,
“See,
I have made
you like God
to Pharaoh,
and your brother
Aaron
will be
your prophet.
\VS{2}You
are to speak
everything
I command
you, and your brother
Aaron
is to tell
Pharaoh
that he must release
the
Israelites
from his land.
\VS{3}But I
will harden
Pharaoh’s
heart,
and although I will multiply
my signs
and my wonders
in the land
of Egypt,
\VS{4}Pharaoh
will
not
listen
to
you. I will reach into
Egypt
and bring out
my regiments,
my people
the Israelites,
from the land
of Egypt
with great
acts of judgment.
\VS{5}Then the Egyptians
will know
that
I am
the {\ND{Lord}}, when I extend
my hand
over
Egypt
and bring
the Israelites
out from among them.
\par }{\PP \VS{6}And Moses
and Aaron
did so;
they did
just
as the
{\ND{Lord}}
commanded them.
\VS{7}Now Moses
was eighty
years
old and Aaron
was eighty-three
years
old when they spoke
to
Pharaoh.
\par }{\PP \VS{8}The
{\ND{Lord}}
said
to
Moses
and Aaron,
\VS{9}“When Pharaoh
says
to you, ‘Do a miracle,’
and you say
to
Aaron,
‘Take
your staff
and throw
it down
before
Pharaoh,’
it will become
a snake.”
\VS{10}When Moses
and Aaron
went
to
Pharaoh,
they did
so,
just
as the
{\ND{Lord}}
had commanded them – Aaron threw down his staff before Pharaoh and his servants and it became a snake.
\VS{11}Then Pharaoh
also
summoned
wise men
and sorcerers,
and the magicians
of Egypt
by their
secret arts
did
the same thing.
\VS{12}Each man
threw down
his staff,
and the staffs became
snakes.
But Aaron’s
staff
swallowed up
their staffs.
\VS{13}Yet Pharaoh’s
heart
became hard, and he did not
listen
to
them, just
as the
{\ND{Lord}}
had predicted.
\par }{\SH The First Blow: Water to Blood
\par }{\PP \VS{14}The
{\ND{Lord}}
said
to
Moses,
“Pharaoh’s
heart
is hard; he refuses
to release
the people.
\VS{15}Go
to
Pharaoh
in the morning
when
he goes
out
to the water.
Position
yourself to meet
him by
the edge
of the Nile,
and take
in your hand
the staff
that
was turned
into a snake.
\VS{16}Tell
him,
‘The
{\ND{Lord}}, the God
of the Hebrews,
has sent
me to
you to
say, “Release
my people,
that they may serve
me in the desert!” But until
now
you have not
listened.
\VS{17}Thus
says
the {\ND{Lord}}: “By this
you will know
that
I am
the {\ND{Lord}}: I
am
going to strike
the water
of the Nile
with the staff
that
is in my hand,
and it will be turned
into blood.
\VS{18}Fish
in the Nile
will die,
the Nile
will stink,
and the Egyptians
will be unable
to drink
water
from
the Nile.” ’ ”
\VS{19}Then the
{\ND{Lord}}
said to
Moses,
“Tell
Aaron,
‘Take
your staff
and stretch
out your hand
over
Egypt’s
waters
– over
their rivers,
over
their canals,
over
their ponds,
and over
all
their reservoirs –
so that it becomes
blood.’
There will be
blood
everywhere
in the land
of Egypt,
even in wooden
and stone containers.”
\VS{20}Moses
and Aaron
did
so,
just
as the
{\ND{Lord}}
had commanded.
Moses raised
the staff
and struck
the
water
that
was in the Nile
right before
the eyes
of Pharaoh
and his servants,
and all
the water
that
was in the Nile
was turned
to blood.
\VS{21}When the fish
that
were in the Nile
died,
the Nile
began to stink,
so that the Egyptians
could
not
drink
water
from
the Nile.
There was
blood
everywhere
in the land
of Egypt!
\VS{22}But the magicians
of Egypt
did
the same
by their secret arts,
and so Pharaoh’s
heart
remained hard, and he refused
to listen
to
Moses
and Aaron –
just
as the
{\ND{Lord}}
had predicted.
\VS{23}And Pharaoh
turned
and went
into
his house.
He did not
pay
any attention
to this.
\VS{24}All
the Egyptians
dug
around
the Nile
for water
to drink,
because
they could
not
drink
the water
of the Nile.
\par }{\SH The Second Blow: Frogs
\par }{\PP \VS{25}Seven
full
days
passed after
the {\ND{Lord}}
struck
the Nile.

\par }\Chap{8}{\PP \VerseOne{1}Then the
{\ND{Lord}}
said to
Moses,
“Go
to
Pharaoh
and tell
him,
‘Thus
says
the {\ND{Lord}}: “Release
my people
in order that they may serve me!
\VS{2}But if
you
refuse
to release
them, then I
am
going to plague
all
your territory
with frogs.
\VS{3}The Nile
will swarm
with frogs,
and they will come up
and go into
your house,
in your bedroom,
and on
your bed,
and into the houses
of your servants
and your people,
and into your ovens
and your kneading troughs.
\VS{4}Frogs
will come up
against you, your people,
and all
your servants.” ’ ”
\par }{\PP \VS{5}The
{\ND{Lord}}
spoke to
Moses,
“Tell
Aaron,
‘Extend
your hand
with your staff
over
the rivers,
over
the canals,
and over
the ponds,
and bring
the
frogs
up over
the land
of Egypt.’ ”
\VS{6}So Aaron
extended
his hand
over
the waters
of Egypt,
and frogs
came up
and covered
the land
of Egypt.
\par }{\PP \VS{7}The magicians
did
the same
with their secret arts
and brought up
frogs
on
the land
of Egypt too.
\par }{\PP \VS{8}Then Pharaoh
summoned
Moses
and Aaron
and said,
“Pray
to
the {\ND{Lord}}
that he may take the frogs
away
from me
and my people,
and I will release
the
people
that they may sacrifice
to the
{\ND{Lord}}.”
\VS{9}Moses
said
to Pharaoh,
“You may have the honor
over me – when shall I pray for you, your servants, and your people, for the frogs to be removed from you and your houses, so that they will be left only in the Nile?”
\VS{10}He said,
“Tomorrow.”
And Moses said, “It will be as you say,
so that you may
know
that
there is no
one like the
{\ND{Lord}}
our God.
\VS{11}The frogs
will depart
from
you, your houses,
your servants,
and your people;
they will be left
only
in the Nile.”
\par }{\PP \VS{12}Then
Moses
and Aaron
went out from Pharaoh,
and Moses
cried
to
the {\ND{Lord}}
because
of the frogs
that he had
brought
on Pharaoh.
\VS{13}The
{\ND{Lord}}
did
as Moses asked – the frogs died out of the houses, the villages, and the fields.
\VS{14}The Egyptians piled
them in countless
heaps,
and the land
stank.
\VS{15}But when Pharaoh
saw
that
there was
relief,
he hardened
his heart
and did not
listen
to
them, just
as the
{\ND{Lord}}
had predicted.
\par }{\SH The Third Blow: Gnats
\par }{\PP \VS{16}The
{\ND{Lord}}
said
to
Moses,
“Tell
Aaron,
‘Extend
your staff
and strike
the dust
of the ground,
and it will become
gnats
throughout all
the land
of Egypt.’ ”
\VS{17}They did
so;
Aaron
extended
his hand
with his staff,
he struck
the
dust
of the ground,
and it became
gnats
on people
and on animals.
All
the dust
of the ground
became
gnats
throughout all
the land
of Egypt.
\VS{18}When
the magicians
attempted
to bring forth
gnats
by their secret arts, they could
not.
So there
were gnats
on people
and on animals.
\VS{19}The magicians
said
to
Pharaoh,
“It is
the finger
of God!” But Pharaoh’s
heart
remained hard, and he did not
listen
to
them, just
as the
{\ND{Lord}}
had predicted.
\par }{\SH The Fourth Blow: Flies
\par }{\PP \VS{20}The
{\ND{Lord}}
said
to
Moses,
“Get up early
in the morning
and position
yourself before
Pharaoh
as he goes
out
to the water,
and tell
him,
‘Thus
says
the {\ND{Lord}}, “Release
my people
that they may serve me!
\VS{21}If
you do not
release
my people,
then I am going
to send
swarms of flies
on
you and on
your servants
and on
your people
and in your houses.
The
houses
of the Egyptians
will be full
of flies,
and even
the ground
they
stand on.
\VS{22}But on that day
I will mark
off the
land
of Goshen,
where
my people
are staying,
so that no swarms of flies
will be
there,
that you may
know
that
I am
the {\ND{Lord}}
in the midst
of this land.
\VS{23}I will put
a division
between
my people
and your people.
This
sign
will take place tomorrow.” ’ ”
\VS{24}The
{\ND{Lord}}
did
so;
a thick swarm of flies
came
into Pharaoh’s
house
and into the houses
of his servants,
and throughout the whole
land
of Egypt
the land
was ruined
because of the swarms of flies.
\par }{\PP \VS{25}Then Pharaoh
summoned
Moses
and Aaron
and said,
“Go,
sacrifice
to your God
within the land.”
\VS{26}But Moses
said,
“That would not
be the right
thing to do,
for
the sacrifices
we make to the
{\ND{Lord}}
our God
would be an abomination
to the Egyptians.
If
we make sacrifices
that are an abomination
to the Egyptians
right before their eyes,
will they not
stone us?
\VS{27}We must go
on a three-day
journey
into the desert
and sacrifice
to the
{\ND{Lord}}
our God,
just
as he is telling
us.”
\par }{\PP \VS{28}Pharaoh
said,
“I
will release
you so that you may sacrifice
to the
{\ND{Lord}}
your God
in the desert.
Only
you must
not
go
very far.
Do pray
for me.”
\par }{\PP \VS{29}Moses
said,
“I
am
going to go out
from you
and pray
to
the {\ND{Lord}}, and the swarms
of flies
will go away from
Pharaoh,
from his servants,
and from his people
tomorrow.
Only
do not
let Pharaoh
deal falsely
again
by not
releasing
the
people
to sacrifice
to the
{\ND{Lord}}.”
\VS{30}So Moses
went out
from Pharaoh
and prayed
to
the {\ND{Lord}},
\VS{31}and the
{\ND{Lord}}
did
as
Moses asked – he removed the swarms of flies from Pharaoh, from his servants, and from his people. Not one remained!
\VS{32}But
Pharaoh
hardened
his heart
this
time
also
and did not
release
the people.

\par }\Chap{9}{\PP \VerseOne{1}Then the
{\ND{Lord}}
said to
Moses,
“Go
to
Pharaoh
and tell
him,
‘Thus
says
the {\ND{Lord}},
the God
of the Hebrews,
“Release
my people
that they may serve me!
\VS{2}For
if
you
refuse
to release
them and continue
holding them,
\VS{3}then
the hand
of the {\ND{Lord}}
will
surely
bring a
very
terrible
plague
on your livestock
in the field,
on the horses,
the donkeys,
the camels,
the herds,
and the flocks.
\VS{4}But
the Lord
will distinguish
between
the livestock
of Israel
and the livestock
of Egypt,
and nothing
will die
of all
that the Israelites have.” ’ ”
\par }{\PP \VS{5}The
{\ND{Lord}}
set
an appointed
time, saying,
“Tomorrow
the {\ND{Lord}}
will do this
in the land.”
\VS{6}And the
{\ND{Lord}}
did
this
on the next
day; all
the livestock
of the Egyptians
died,
but of the Israelites’
livestock
not
one
died.
\VS{7}Pharaoh
sent
representatives to investigate,
and indeed,
not
even
one
of the livestock
of Israel
had died.
But
Pharaoh’s
heart
remained hard,
and he did not
release
the people.
\par }{\SH The Sixth Blow: Boils
\par }{\PP \VS{8}Then the
{\ND{Lord}}
said
to Moses
and Aaron,
“Take
handfuls
of soot
from a furnace,
and have Moses
throw
it into the air
while Pharaoh
is watching.
\VS{9}It will become
fine dust
over
the whole
land
of Egypt
and will cause boils
to break
out and fester
on
both people
and animals
in all
the land
of Egypt.”
\VS{10}So they took
soot
from a furnace
and stood
before
Pharaoh,
Moses
threw
it into the air,
and it caused
festering
boils
to break
out on both people
and animals.
\par }{\PP \VS{11}The magicians
could
not
stand
before
Moses
because
of the boils,
for
boils
were
on the magicians
and on all
the Egyptians.
\VS{12}But
the {\ND{Lord}}
hardened
Pharaoh’s
heart,
and he did not
listen
to
them, just
as the
{\ND{Lord}}
had predicted
to
Moses.
\par }{\SH The Seventh Blow: Hail
\par }{\PP \VS{13}The
{\ND{Lord}}
said
to
Moses,
“Get up early
in the morning,
stand
before
Pharaoh,
and tell
him,
‘Thus
says
the {\ND{Lord}}, the God
of the Hebrews: “Release
my people
so that they may serve me!
\VS{14}For
this
time
I
will send
all
my plagues
on your very self
and on your servants
and your people,
so that you may know
that
there is no
one like
me in all
the earth.
\VS{15}For
by now
I could have stretched out
my hand
and struck
you and your people
with plague,
and you would have been destroyed
from
the earth.
\VS{16}But
for
this
purpose
I have caused you to stand: to show
you my strength,
and so
that my name
may be declared
in all
the earth.
\VS{17}You are still
exalting
yourself against my people
by not
releasing them.
\VS{18}I am going
to cause very
severe
hail to rain
down about this time
tomorrow,
such hail
as has
never
occurred in Egypt
from
the day
it was founded
until
now.
\VS{19}So now,
send
instructions to gather
your livestock
and all
your possessions in the fields
to a safe place. Every
person
or animal
caught
in the field
and not
brought
into the house
– the hail
will come down
on
them, and they will die!” ’ ”
\par }{\PP \VS{20}Those of Pharaoh’s
servants
who feared
the word
of the {\ND{Lord}}
hurried
to bring their servants
and livestock
into
the houses,
\VS{21}but those who
did not
take
the word
of the {\ND{Lord}}
seriously
left
their servants
and their cattle
in the field.
\par }{\PP \VS{22}Then the
{\ND{Lord}}
said to
Moses,
“Extend
your hand
toward the sky
that there may be
hail
in all
the land
of Egypt,
on
people
and on
animals,
and on
everything
that grows
in the field
in the land
of Egypt.”
\VS{23}When Moses
extended
his staff
toward the
sky,
the {\ND{Lord}}
sent thunder
and hail,
and fire
fell to the earth;
so the
{\ND{Lord}}
caused hail
to rain
down on
the land
of Egypt.
\VS{24}Hail
fell and fire
mingled
with the hail;
the hail was so
severe
that
there had
not
been any like
it in all
the land
of Egypt
since
it had become
a nation.
\VS{25}The hail
struck
everything
in the open fields,
both people
and animals,
throughout all
the land
of Egypt.
The
hail
struck
everything
that grows
in the field,
and it broke
all
the trees
of the field to pieces.
\VS{26}Only
in the land
of Goshen,
where
the Israelites
lived, was there
no
hail.
\par }{\PP \VS{27}So Pharaoh
sent
and summoned
Moses
and Aaron
and said
to them,
“I have sinned
this time! The
{\ND{Lord}}
is righteous,
and I
and my people
are guilty.
\VS{28}Pray
to
the {\ND{Lord}}, for the mighty
thunderings
and hail
are too much! I will release
you and you will stay
no
longer.”
\par }{\PP \VS{29}Moses
said
to
him, “When I leave
the
city
I will spread
my hands
to
the {\ND{Lord}}, the thunder
will cease,
and there will be
no
more
hail,
so that you may
know
that
the earth
belongs to the
{\ND{Lord}}.
\VS{30}But as for you
and your servants,
I know
that
you do not yet
fear
the {\ND{Lord}}
God.”
\par }{\PP \VS{31}(Now the flax
and the barley
were struck by the hail,
for
the barley
had ripened
and the flax
was in bud.
\VS{32}But the wheat
and the spelt
were not
struck,
for
they are later crops.)
\par }{\PP \VS{33}So Moses
left
Pharaoh,
went out of the
city,
and spread
out his hands
to
the {\ND{Lord}}, and the thunder
and the hail
ceased,
and the rain
stopped
pouring
on the earth.
\VS{34}When Pharaoh
saw
that
the rain
and hail
and thunder
ceased,
he sinned
again: both he and his servants
hardened
their hearts.
\VS{35}So Pharaoh’s
heart
remained hard, and he did not
release
the Israelites,
as the
{\ND{Lord}}
had
predicted
through
Moses. ‘span class=”footnote” id=”footnote-38” ’‘span class=”key” ’38‘a href=”\#note-38” class=”backref” ’9:18‘/a’‘span class=”text” ’
{\IT{tn}}:
{\IT{Heb}} “which not was like it in Egypt.” The pronoun suffix serves as the resumptive pronoun for the relative particle: “which…like it” becomes “the like of which has not been.” The word “hail” is added in the translation to make clear the referent of the relative particle.

\par }\Chap{10}{\PP \VerseOne{1}The
{\ND{Lord}}
said
to
Moses,
“Go
to
Pharaoh,
for
I
have hardened
his heart
and the heart
of his servants,
in order
to display
these
signs
of mine before him,
\VS{2}and in order
that in the hearing
of
your son
and your grandson
you may tell how
I made fools
of the Egyptians
and about
my signs
that
I displayed
among them, so that you may know
that
I am
the {\ND{Lord}}.”
\par }{\PP \VS{3}So Moses
and Aaron
came
to
Pharaoh
and told
him, “Thus
says
the {\ND{Lord}}, the God
of the Hebrews: ‘How
long
do you refuse
to humble
yourself before
me? Release
my people
so that they may serve me!
\VS{4}But if
you
refuse
to release
my people,
I am going
to bring
locusts
into your territory
tomorrow.
\VS{5}They will cover
the surface
of the
earth,
so that you will be unable
to see
the
ground.
They will eat
the
remainder
of what escaped –
what is left over for you – from the hail, and they will eat every tree that grows for you from the field.
\VS{6}They will fill
your houses,
the houses
of your servants,
and all
the houses
of Egypt,
such as neither
your fathers
nor your grandfathers
have seen
since
they have been
in the land
until
this
day!’ ” Then Moses turned
and went out
from Pharaoh.
\par }{\PP \VS{7}Pharaoh’s
servants
said
to
him, “How long
will this
man be a menace
to us? Release
the people
so that they may serve
the {\ND{Lord}}
their God.
Do you not
know
that
Egypt
is destroyed?”
\par }{\PP \VS{8}So
Moses
and Aaron
were brought back to
Pharaoh,
and he said
to
them, “Go,
serve
the {\ND{Lord}}
your God.
Exactly
who
is going with you?”
\VS{9}Moses
said,
“We will go
with our young
and our old,
with our sons
and our daughters,
and with our sheep
and our cattle
we will go,
because
we are to hold a pilgrim feast
for the
{\ND{Lord}}.”
\par }{\PP \VS{10}He said
to
them, “The
{\ND{Lord}}
will
need
to be with
you if
I release
you and your dependents! Watch
out! Trouble
is right
in front of you!
\VS{11}No! Go,
you men
only, and serve
the {\ND{Lord}}, for
that is what
you
want.”
Then Moses and Aaron were driven
out of Pharaoh’s
presence.
\par }{\PP \VS{12}The
{\ND{Lord}}
said
to
Moses,
“Extend
your hand
over
the land
of Egypt
for the locusts,
that they may come up
over
the land
of Egypt
and eat
everything
that grows
in the ground,
everything
that
the hail
has
left.”
\VS{13}So Moses
extended
his staff
over
the land
of Egypt,
and then the
{\ND{Lord}}
brought
an east
wind
on the land
all
that day
and all
night.
The morning
came,
and the east
wind
had brought up
the
locusts!
\VS{14}The locusts
went up
over
all
the land
of Egypt
and settled
down in all
the territory
of Egypt.
It was very
severe;
there had been no
locusts
like
them
before,
nor
will there be
such ever again.
\VS{15}They covered
the surface
of all
the ground,
so that the ground
became dark
with them, and they ate
all
the vegetation
of the ground
and all
the fruit
of the trees
that
the hail
had left.
Nothing
green
remained
on the trees
or on anything that grew
in the fields
throughout the whole
land
of Egypt.
\VS{16}\par }{\PP Then Pharaoh
quickly
summoned
Moses
and Aaron
and said,
“I have sinned
against the
{\ND{Lord}}
your God and against you!
\VS{17}So now,
forgive
my sin
this time
only,
and pray
to the
{\ND{Lord}}
your God
that he would
only
take this
death away from me.”
\VS{18}Moses went out
from Pharaoh
and prayed
to
the {\ND{Lord}},
\VS{19}and the
{\ND{Lord}}
turned
a very
strong
west
wind,
and it picked
up the locusts
and blew
them into the Red
Sea.
Not
one
locust
remained
in all
the territory
of Egypt.
\VS{20}But
the {\ND{Lord}}
hardened
Pharaoh’s
heart,
and he did not
release
the Israelites.
\par }{\SH The Ninth Blow: Darkness
\par }{\PP \VS{21}The
{\ND{Lord}}
said
to
Moses,
“Extend
your hand
toward heaven
so that there may be
darkness
over
the land
of Egypt,
a darkness
so thick it can be felt.”
\par }{\PP \VS{22}So Moses
extended
his hand
toward heaven,
and there was
absolute
darkness
throughout
the land
of Egypt
for three
days.
\VS{23}No
one could see
another
person,
and no
one
could rise
from his place
for three
days.
But the Israelites
had light
in the places where they lived.
\par }{\PP \VS{24}Then Pharaoh
summoned
Moses
and said,
“Go,
serve
the {\ND{Lord}} –
only
your flocks
and herds
will be detained.
Even
your families
may go
with you.”
\par }{\PP \VS{25}But Moses
said,
“Will you
also
provide us
with sacrifices
and burnt offerings
that we may
present them to the
{\ND{Lord}}
our God?
\VS{26}Our livestock
must also
go
with
us! Not
a hoof
is to be left
behind! For we
must take
these animals to serve
the {\ND{Lord}}
our God.
Until
we arrive
there,
we
do not
know
what
we must use to serve
the {\ND{Lord}}.”
\par }{\PP \VS{27}But
the {\ND{Lord}}
hardened
Pharaoh’s
heart,
and he was not
willing
to release them.
\VS{28}Pharaoh
said
to him, “Go
from me! Watch
out for yourself! Do not
appear
before
me again,
for
when
you see
my face
you will die!”
\VS{29}Moses
said,
“As you wish! I will not
see
your face
again.”


\par }\Chap{11}{\PP \VerseOne{1}The
{\ND{Lord}}
said
to
Moses,
“I will bring
one
more
plague
on
Pharaoh
and on
Egypt;
after
that
he will release
you from this
place. When
he releases
you, he will drive
you out completely
from
this place.
\VS{2}Instruct
the people
that each man
and each
woman
is to request
from his or her neighbor
items
of silver
and gold.”
\par }{\PP \VS{3}(Now
the {\ND{Lord}}
granted
the people
favor
with the Egyptians.
Moreover,
the man
Moses
was very
great
in the land
of Egypt,
respected
by Pharaoh’s
servants
and by the Egyptian
people.)
\par }{\PP \VS{4}Moses
said,
“Thus
says
the {\ND{Lord}}: ‘About midnight
I
will go throughout Egypt,
\VS{5}and all
the firstborn
in the land
of Egypt
will die,
from the firstborn son
of Pharaoh
who sits
on
his throne,
to
the firstborn son
of the slave girl
who is at her hand mill,
and all
the firstborn
of the cattle.
\VS{6}There will be
a great
cry
throughout the whole
land
of Egypt,
such
as there has never been, nor
ever will be again.
\VS{7}But against any
of the Israelites
not
even
a dog
will bark
against either people or animals,
so that you may
know
that
the {\ND{Lord}}
distinguishes
between
Egypt
and Israel.’
\VS{8}All
these
your servants
will come down
to
me and bow down
to me, saying,
‘Go,
you
and all
the people
who
follow
you,’ and after
that
I will go out.”
Then Moses went out
from
Pharaoh
in great
anger.
\par }{\PP \VS{9}The
{\ND{Lord}}
said
to
Moses,
“Pharaoh
will not
listen
to
you, so
that my wonders
may be multiplied
in the land
of Egypt.”
\par }{\PP \VS{10}So Moses
and Aaron
did
all
these
wonders
before
Pharaoh,
but
the {\ND{Lord}}
hardened
Pharaoh’s
heart,
and he did not
release
the Israelites
from his land.


\par }\Chap{12}{\PP \VerseOne{1}The
{\ND{Lord}}
said
to
Moses
and Aaron
in the land
of Egypt,
\VS{2}“This
month
is to be your beginning
of months;
it will be your first
month
of the year.
\VS{3}Tell
the whole
community
of Israel,
‘In the tenth
day of this
month
they each
must take
a lamb
for themselves according to their families –
a lamb
for each household.
\VS{4}If
any household
is too small
for a lamb,
the man
and his next-door
neighbor
are to take
a lamb according to the number
of people
– you will make your count
for the lamb
according
to how much each one
can eat.
\VS{5}Your lamb
must be perfect,
a male,
one year
old;
you may take
it from
the sheep
or from
the goats.
\VS{6}You must care
for it until
the fourteenth
day
of this
month,
and then
the whole
community
of Israel
will kill it around sundown.
\VS{7}They will take
some
of the blood
and put
it on
the two
side posts and top of the doorframe
of the houses
where
they will eat it.
\VS{8}They will eat
the meat
the same night;
they will eat
it roasted
over the fire
with bread made without yeast
and with bitter herbs.
\VS{9}Do not
eat
it raw
or boiled
in water,
but
roast
it over the fire
with its head,
its legs,
and its entrails.
\VS{10}You must leave nothing
until
morning,
but you must burn
with fire
whatever remains
of it
until
morning.
\VS{11}This
is how you are to eat it – dressed to travel, your sandals on your feet, and your staff in your hand. You are to eat it in haste. It is the
{\ND{Lord’s}} Passover.
\par }{\PP \VS{12}I will pass through
the land
of Egypt
in the same
night,
and I will attack
all
the firstborn
in the land
of Egypt,
both of humans
and of animals,
and on all
the gods
of Egypt
I will execute
judgment.
I am
the {\ND{Lord}}.
\VS{13}The blood
will be
a sign
for you on
the houses
where
you
are, so that when I see
the
blood
I will pass
over
you, and this plague
will not
fall on
you to destroy
you when I attack
the land
of Egypt.
\par }{\PP \VS{14}This
day
will become
a memorial
for you, and you will celebrate
it as a festival
to the
{\ND{Lord}} –
you will celebrate
it perpetually
as a lasting
ordinance.
\VS{15}For seven
days
you must eat
bread made without yeast.
Surely
on
the first
day
you must put
away yeast
from your houses
because
anyone
who eats
bread made with yeast
from the first
day
to the seventh
day
will be cut off
from Israel.
\par }{\PP \VS{16}On the first
day
there will be a holy
convocation,
and on the seventh
day
there will be
a holy
convocation
for you. You must do
no
work
of any kind
on them, only
what
every
person
will eat
– that
alone
may be prepared for you.
\VS{17}So you will keep
the Feast of Unleavened
Bread, because
on this very
day
I brought your regiments
out
from the land
of Egypt,
and so you must keep
this
day
perpetually
as a lasting
ordinance.
\VS{18}In the first
month, from the fourteenth
day
of the month,
in the evening,
you will eat
bread made without yeast
until
the twenty-first
day
of the month
in the evening.
\VS{19}For seven
days
yeast
must not
be found
in your houses,
for
whoever
eats
what is made with yeast
– that person will be cut off
from the community
of Israel,
whether a foreigner
or one born
in the land.
\VS{20}You will not
eat
anything
made with yeast;
in all
the places
where you live you must eat
bread made without yeast.’ ”
\par }{\PP \VS{21}Then Moses
summoned
all
the elders
of Israel,
and told
them,
“Go and select
for yourselves a lamb or young goat for your
families,
and kill
the Passover animals.
\VS{22}Take
a branch
of hyssop,
dip
it in the blood
that
is in the basin,
and apply
to
the top of the doorframe
and the two
side posts some
of the blood
that
is in the basin.
Not
one of you
is to go out
the door
of his house
until
morning.
\VS{23}For the
{\ND{Lord}}
will pass
through to strike
Egypt,
and when he sees
the
blood
on
the top of the doorframe
and the two
side posts, then
the {\ND{Lord}}
will pass over
the door,
and he will
not
permit
the destroyer
to enter
your houses
to
strike you.
\VS{24}You must observe
this
event
as an ordinance
for you and for your children
forever.
\VS{25}When
you enter
the land
that
the {\ND{Lord}}
will give
to you, just
as he said,
you must observe
this
ceremony.
\VS{26}When
your children
ask you, ‘What
does this
ceremony mean to you?’ –
\VS{27}then you will say,
‘It is
the sacrifice
of the
{\ND{Lord}}’s
Passover,
when he passed
over
the houses
of the Israelites
in Egypt,
when he struck
Egypt
and delivered
our households.’ ”
The people
bowed down low to the ground,
\VS{28}and the Israelites
went
away and did
exactly
as the
{\ND{Lord}}
had commanded
Moses
and Aaron.
\par }{\SH The Deliverance from Egypt
\par }{\PP \VS{29}It happened
at midnight
– the
{\ND{Lord}}
attacked
all
the firstborn
in the land
of Egypt,
from the firstborn
of Pharaoh
who sat
on
his throne
to
the firstborn
of the captive
who
was in the prison,
and all
the firstborn
of the cattle.
\VS{30}Pharaoh
got
up in the night,
along with all
his servants
and all
Egypt,
and there was
a great
cry
in Egypt,
for
there was no
house
in which
there
was not
someone dead.
\VS{31}Pharaoh summoned
Moses
and Aaron
in the night
and said,
“Get
up, get out
from among
my people,
both
you
and the Israelites! Go,
serve
the {\ND{Lord}}
as you have requested!
\VS{32}Also,
take
your flocks
and your herds,
just
as you have requested,
and leave.
But bless
me also.”
\par }{\PP \VS{33}The Egyptians
were urging
the people
on,
in order to send
them out
of the land
quickly,
for
they were saying,
“We are all
dead!”
\VS{34}So the people
took
their dough
before
the yeast was added,
with their kneading troughs
bound up
in their clothing
on
their shoulders.
\VS{35}Now the Israelites
had done
as Moses
told them – they had requested from the Egyptians silver and gold items and clothing.
\VS{36}The
{\ND{Lord}}
gave
the
people
favor
in the sight
of the Egyptians,
and they gave them
whatever they wanted,
and so they plundered
Egypt.
\par }{\PP \VS{37}The Israelites
journeyed
from Rameses
to Sukkoth.
There were about 600,000
men
on foot,
plus their dependants.
\VS{38}A mixed multitude
also
went up
with
them, and flocks
and herds
– a very
large
number of cattle.
\VS{39}They baked
cakes
of bread without yeast
using the dough
they had
brought
from Egypt,
for
it was made without
yeast
– because
they were thrust out
of Egypt
and were not
able
to delay,
they could not
prepare
food for themselves either.
\par }{\PP \VS{40}Now the length
of time the Israelites
lived
in Egypt
was 430
years.
\VS{41}At the end
of the 430
years,
on the very
day,
all
the regiments
of the {\ND{Lord}}
went out
of the land
of Egypt.
\VS{42}It was
a night
of vigil
for the
{\ND{Lord}}
to bring them out
from the land
of Egypt,
and so on this
night
all
Israel
is to keep the vigil
to the
{\ND{Lord}}
for generations to come.
\par }{\SH Participation in the Passover
\par }{\PP \VS{43}The
{\ND{Lord}}
said
to
Moses
and Aaron,
“This
is the ordinance
of the Passover.
No
foreigner
may share in eating it.
\VS{44}But everyone’s
servant
who is bought
for money,
after you have circumcised
him,
may eat it.
\VS{45}A foreigner
and a hired worker
must not
eat it.
\VS{46}It must be eaten
in one
house;
you must not
bring
any
of the meat
outside
the house,
and you must not
break
a bone of it.
\VS{47}The whole
community
of Israel
must observe it.
\par }{\PP \VS{48}“When
a foreigner
lives
with
you and wants to observe
the Passover
to the
{\ND{Lord}}, all
his males
must be circumcised,
and then
he may approach
and observe
it, and he will be
like one who is born
in the land –
but no
uncircumcised person
may eat of it.
\VS{49}The same law
will
apply to the person who is native-born
and to the foreigner
who lives
among you.”
\par }{\PP \VS{50}So
all
the Israelites
did
exactly
as the
{\ND{Lord}}
commanded
Moses
and Aaron.
\VS{51}And on this
very
day
the {\ND{Lord}}
brought
the Israelites
out of the land
of Egypt
by
their regiments. ‘span class=”footnote” id=”footnote-49” ’‘span class=”key” ’49‘a href=”\#note-49” class=”backref” ’12:16‘/a’‘span class=”text” ’
{\IT{tn}}:
{\IT{Heb}} “all/every work will not be done.” The word refers primarily to the work of one’s occupation. B. Jacob ({\IT{Exodus}}, 322) explains that since this comes prior to the fuller description of laws for Sabbaths and festivals, the passage simply restricts all work except for the preparation of food. Once the laws are added, this qualification is no longer needed. Gesenius translates this as “no manner of work shall be done” (GKC 478-79 §152.{\IT{b}}).

\par }\Chap{13}{\PP \VerseOne{1}The
{\ND{Lord}}
spoke
to
Moses:
\VS{2}“Set apart
to me every
firstborn male
– the first offspring
of every
womb
among the Israelites,
whether human
or animal; it is mine.”
\par }{\PP \VS{3}Moses
said
to
the people,
“Remember
this
day
on which
you came out
from Egypt,
from the place where you were enslaved,
for
the {\ND{Lord}}
brought you out
of there with a mighty
hand
– and no
bread made with yeast
may be eaten.
\VS{4}On this day,
in the month
of Abib,
you
are going out.
\par }{\PP \VS{5}When
the {\ND{Lord}}
brings
you to
the land
of the Canaanites,
Hittites,
Amorites,
Hivites,
and Jebusites,
which
he swore
to your fathers
to give
you, a land
flowing
with milk
and honey,
then you will keep
this
ceremony
in this
month.
\VS{6}For seven
days
you must eat
bread made without yeast,
and on
the seventh
day
there is to be a festival
to the
{\ND{Lord}}.
\VS{7}Bread made without yeast
must be eaten
for seven
days;
no
bread made with yeast
shall be seen
among you, and you must have no
yeast
among you within any
of your borders.
\par }{\PP \VS{8}You are to tell
your son
on that day, ‘It is
because
of what
the {\ND{Lord}}
did
for me when I came out
of Egypt.’
\VS{9}It will be
a sign
for you on
your hand
and a memorial
on your forehead,
so
that the law
of the {\ND{Lord}}
may be
in your mouth,
for
with a mighty
hand
the {\ND{Lord}}
brought you out
of Egypt.
\VS{10}So you must keep
this
ordinance
at its appointed
time
from year to year.
\par }{\PP \VS{11}When
the {\ND{Lord}}
brings
you into
the land
of the Canaanites,
as
he swore
to you and to your fathers,
and gives it to you,
\VS{12}then you must give over
to the
{\ND{Lord}}
the first offspring
of every
womb.
Every
firstling
of
a beast
that
you have –
the males
will be
the
{\ND{Lord}}’s.
\VS{13}Every
firstling
of a donkey
you must redeem
with a lamb,
and if
you do not
redeem
it, then you must break its neck.
Every
firstborn
of your sons
you must redeem.
\VS{14}\par }{\PP In the future,
when
your son
asks
you ‘What
is this?’ you are to tell
him,
‘With a mighty
hand
the {\ND{Lord}}
brought us out
from Egypt,
from the land of slavery.
\VS{15}When
Pharaoh
stubbornly
refused to release
us, the
{\ND{Lord}}
killed
all
the firstborn
in the land
of Egypt,
from the firstborn
of people
to
the firstborn
of animals.
That is why
I am
sacrificing
to the
{\ND{Lord}}
the first male
offspring
of every
womb,
but all
my firstborn
sons
I redeem.’
\VS{16}It will be
for a sign
on
your hand
and for frontlets
on your forehead,
for
with a mighty
hand
the {\ND{Lord}}
brought us out
of Egypt.”
\par }{\SH The Leading of God
\par }{\PP \VS{17}When
Pharaoh
released
the
people,
God
did not
lead
them by the way
to the land
of the Philistines,
although
that was
nearby,
for
God
said, “Lest
the people
change
their minds and return
to Egypt
when
they experience
war.”
\VS{18}So God
brought
the people
around
by the way
of the desert
to the Red
Sea,
and the Israelites
went up
from the land
of Egypt
prepared for battle.
\par }{\PP \VS{19}Moses
took
the
bones
of Joseph
with
him, for
Joseph had made
the Israelites
solemnly
swear, “God
will surely attend
to you, and you will carry my bones
up
from this
place with you.”
\par }{\PP \VS{20}They journeyed
from Sukkoth
and camped
in Etham,
on the edge
of the desert.
\VS{21}Now the
{\ND{Lord}}
was going
before
them by day
in a pillar
of cloud
to lead
them in the way,
and by night
in a pillar
of fire
to give them light,
so
that they could travel
day
or night.
\VS{22}He did not
remove
the pillar
of cloud
by day
nor the pillar
of fire
by night
from before
the people.


\par }\Chap{14}{\PP \VerseOne{1}The
{\ND{Lord}}
spoke
to
Moses:
\VS{2}“Tell
the Israelites
that they must turn
and camp
before
Pi-hahiroth,
between
Migdol
and the sea;
you are to camp
by
the sea
before
Baal Zephon
opposite it.
\VS{3}Pharaoh
will think regarding the Israelites,
‘They are
wandering
around confused
in the land
– the desert
has closed
in on them.’
\VS{4}I will harden
Pharaoh’s
heart,
and he will chase
after
them. I will gain honor
because of Pharaoh
and because of all
his army,
and the Egyptians
will know
that
I am
the {\ND{Lord}}.” So this is what
they did.
\par }{\PP \VS{5}When it was reported
to the king
of Egypt
that
the people
had fled,
the heart
of Pharaoh
and his servants
was turned against
the people,
and the king and his servants said, “What
in the world have we done? For
we have released
the people of Israel
from serving us!”
\VS{6}Then he prepared
his chariots
and took
his army
with him.
\VS{7}He took
six
hundred
select
chariots,
and all
the rest of the chariots
of Egypt,
and officers
on
all of them.
\par }{\PP \VS{8}But
the {\ND{Lord}}
hardened
the heart
of Pharaoh
king
of Egypt,
and he chased
after
the Israelites.
Now the Israelites
were going out
defiantly.
\VS{9}The Egyptians
chased
after
them, and all
the horses
and chariots
of Pharaoh
and his horsemen
and his army
overtook
them camping
by the sea,
beside
Pi-hahiroth,
before
Baal-Zephon.
\VS{10}When Pharaoh
got closer, the Israelites
looked
up, and there
were the Egyptians
marching
after
them, and they were
terrified.
The Israelites
cried out
to
the {\ND{Lord}},
\VS{11}and they said
to
Moses,
“Is it because
there are no
graves
in Egypt
that you have taken
us away to die
in the desert? What
in the world
have you done
to us by bringing us out
of Egypt?
\VS{12}Isn’t this
what
we told
you in Egypt,
‘Leave
us alone
so that we can serve
the Egyptians,
because
it is better
for us to serve
the Egyptians
than to die
in the desert!’ ”
\par }{\PP \VS{13}Moses
said
to
the people,
“Do not
fear! Stand
firm and see
the salvation
of the {\ND{Lord}}
that
he will provide
for you today;
for
the Egyptians
that
you see
today
you will never,
ever
see
again.
\VS{14}The
{\ND{Lord}}
will fight
for you, and you
can be still.”
\par }{\PP \VS{15}The
{\ND{Lord}}
said
to
Moses,
“Why
do you cry out
to me? Tell
the Israelites
to move on.
\VS{16}And as for you,
lift up
your staff
and extend
your hand
toward the sea
and divide
it, so that the Israelites
may go through
the middle
of the sea
on dry ground.
\VS{17}And as for me, I am
going
to harden
the
hearts
of the Egyptians
so that they will come
after
them, that I may be honored
because of Pharaoh
and his army
and his chariots
and his horsemen.
\VS{18}And the Egyptians
will know
that
I am
the {\ND{Lord}}
when I have gained my honor
because of Pharaoh,
his chariots,
and his horsemen.”
\par }{\PP \VS{19}The angel
of God,
who was going
before
the camp
of Israel,
moved
and went behind
them, and the pillar
of cloud
moved
from before
them and stood
behind them.
\VS{20}It came
between
the Egyptian
camp
and the Israelite
camp;
it was
a dark
cloud
and it lit
up the
night
so that one
camp
did not
come near
the other
the whole
night.
\VS{21}Moses
stretched
out his hand
toward the sea,
and the
{\ND{Lord}}
drove
the sea
apart by a
strong
east
wind
all
that night,
and he made
the
sea
into dry land,
and the water
was divided.
\VS{22}So the Israelites
went
through the middle
of the sea
on dry ground,
the water
forming a wall
for them on their right
and on their left.
\par }{\PP \VS{23}The Egyptians
chased
them and followed
them into
the middle
of the sea
– all
the horses
of Pharaoh,
his chariots,
and his horsemen.
\VS{24}In the morning
watch
the {\ND{Lord}}
looked
down on the Egyptian
army
through the pillar
of fire
and cloud,
and he threw the
Egyptian
army
into a panic.
\VS{25}He jammed the wheels
of their chariots
so that they had difficulty
driving, and the Egyptians
said,
“Let’s flee
from Israel,
for
the {\ND{Lord}}
fights
for them against Egypt!”
\par }{\PP \VS{26}The
{\ND{Lord}}
said
to
Moses,
“Extend
your hand
toward the sea,
so that the waters
may flow back
on
the Egyptians,
on
their chariots,
and on
their horsemen!”
\VS{27}So Moses
extended
his hand
toward the sea,
and the sea
returned
to its normal
state
when
the sun began to rise. Now the Egyptians
were fleeing
before it, but the
{\ND{Lord}}
overthrew
the
Egyptians
in the middle
of the sea.
\VS{28}The water
returned
and covered
the
chariots
and the
horsemen
and all
the army
of Pharaoh
that was coming
after
the Israelites into the sea –
not
so much as
one
of them survived!
\VS{29}But the Israelites
walked
on dry ground
in the middle
of the sea,
the water
forming a wall
for them on their right
and on their left.
\VS{30}So the
{\ND{Lord}}
saved
Israel
on that day
from the power
of the Egyptians,
and Israel
saw
the Egyptians
dead
on
the shore
of the sea.
\VS{31}When Israel
saw
the great
power
that
the {\ND{Lord}}
had exercised
over the Egyptians,
they
feared
the {\ND{Lord}}, and they believed
in the
{\ND{Lord}}
and in his servant
Moses.


\par }\Chap{15}{\PP \VerseOne{1}Then
Moses
and the Israelites
sang
this
song
to the
{\ND{Lord}}. They said,
\par }{\Q “I will sing
to the
{\ND{Lord}},
for
he
has triumphed
gloriously,
\par }{\Q the horse
and its rider
he has thrown
into the sea.
\par }{\Q \VS{2}The
{\ND{Lord}}
is my strength
and my song,
\par }{\Q and he has become my salvation.
\par }{\Q This
is my God,
and I will praise him,

\par }{\Q my father’s
God,
and I will exalt him.
\par }{\Q \VS{3}The
{\ND{Lord}}
is a
warrior,
\par }{\Q the {\ND{Lord}}
is his name.
\par }{\Q \VS{4}The chariots
of Pharaoh
and his army
he has thrown into
the sea,
\par }{\Q and his chosen
officers
were drowned
in the Red
Sea.
\par }{\Q \VS{5}The depths
have covered
them,

\par }{\Q they went down
to the bottom
like
a stone.
\par }{\Q \VS{6}Your right hand,
O
{\ND{Lord}}, was majestic
in power,
\par }{\Q your right hand,
O
{\ND{Lord}}, shattered
the enemy.
\par }{\Q \VS{7}In the abundance
of your majesty
you have overthrown
\par }{\Q those who rise up
against you.

\par }{\Q You sent
forth your wrath;
\par }{\Q it consumed
them like stubble.
\par }{\Q \VS{8}By
the blast
of your nostrils
the waters were piled
up,
\par }{\Q the flowing water
stood
upright like
a heap,
\par }{\Q and the deep
waters
were solidified
in the heart
of the sea.
\par }{\Q \VS{9}The enemy
said,
‘I will chase,
I will overtake,
\par }{\Q I will divide
the spoil;
\par }{\Q my desire
will be satisfied
on them.
\par }{\Q I will draw
my sword,
my hand
will destroy them.’
\par }{\Q \VS{10}But
you blew
with your breath,
and the sea
covered
them.
\par }{\Q They sank
like lead
in the mighty
waters.
\par }{\Q \VS{11}Who
is like
you, O
{\ND{Lord}}, among the gods?

\par }{\Q Who
is like you? – majestic in holiness, fearful in praises, working wonders?
\par }{\Q \VS{12}You stretched
out your right hand,
\par }{\Q the earth
swallowed them.
\par }{\Q \VS{13}By your loyal love
you will lead
the people
whom
you have redeemed;
\par }{\Q you will guide
them by your strength
to
your holy
dwelling place.
\par }{\Q \VS{14}The nations
will hear
and tremble;
\par }{\Q anguish
will seize
the inhabitants
of Philistia.
\par }{\Q \VS{15}Then
the chiefs
of Edom
will be terrified,
\par }{\Q trembling
will seize
the leaders
of Moab,
\par }{\Q and the inhabitants
of Canaan
will shake.
\par }{\Q \VS{16}Fear
and dread
will fall
on
them;
\par }{\Q by the greatness
of your arm
they will be as still
as stone
\par }{\Q until
your people
pass
by, O
{\ND{Lord}},
\par }{\Q until
the people
whom
you have bought
pass by.
\par }{\Q \VS{17}You will bring
them in and plant
them in the mountain
of your inheritance,
\par }{\Q in the place
you made for your residence,
O Lord,
\par }{\Q the sanctuary,
O
{\ND{Lord}}, that your hands
have established.
\par }{\Q \VS{18}The
{\ND{Lord}}
will reign
forever
and ever!
\par }{\Q \VS{19}For
the horses
of Pharaoh
came
with his chariots
and his footmen
into the sea,
\par }{\Q and the
{\ND{Lord}}
brought back
the waters
of the sea
on
them,
\par }{\Q but the Israelites
walked
on dry land
in the middle
of the sea.”
\par }{\PP \VS{20}Miriam
the prophetess,
the sister
of Aaron,
took
a hand-drum
in her hand,
and all
the women
went out
after
her with hand-drums
and with dances.
\VS{21}Miriam
sang in response
to them, “Sing
to the
{\ND{Lord}}, for
he
has triumphed
gloriously;
the horse
and its rider
he has thrown
into the sea.”
\par }{\SH The Bitter Water
\par }{\PP \VS{22}Then Moses
led Israel
to journey
away from the Red
Sea.
They went out
to
the Desert
of Shur,
walked
for three
days
into the desert,
and found
no
water.
\VS{23}Then they came
to Marah,
but they were not
able
to drink
the waters
of Marah,
because
they were
bitter.
(That is why
its name
was Marah.)
\par }{\PP \VS{24}So the people
murmured
against
Moses,
saying,
“What
can we drink?”
\VS{25}He cried out
to
the {\ND{Lord}}, and the
{\ND{Lord}}
showed him
a tree.
When Moses threw
it into the water,
the water
became safe
to drink. There
the Lord
made
for them a binding
ordinance,
and there
he tested them.
\VS{26}He said,
“If
you will diligently
obey
the {\ND{Lord}}
your God,
and do
what is right
in his sight,
and pay attention
to his commandments,
and keep
all
his statutes,
then all
the diseases that
I
brought on the Egyptians
I
will not
bring
on
you, for
I,
the {\ND{Lord}},
am your healer.”
\par }{\PP \VS{27}Then they came
to Elim,
where there
were twelve
wells
of water
and seventy
palm
trees, and they camped
there
by
the water.


\par }\Chap{16}{\PP \VerseOne{1}When they journeyed
from Elim,
the entire
company
of Israelites
came
to
the Desert
of Sin,
which
is between
Elim
and Sinai,
on the fifteenth
day
of the second
month
after their exodus
from the land
of Egypt.
\VS{2}The entire
company
of Israelites
murmured
against
Moses
and Aaron
in the desert.
\VS{3}The Israelites
said
to
them, “If only
we had died
by the hand
of the {\ND{Lord}}
in the land
of Egypt,
when we sat
by
the pots
of meat,
when we ate
bread
to the full,
for
you have brought us out
into
this
desert
to kill
this whole
assembly
with hunger!”
\par }{\PP \VS{4}Then the
{\ND{Lord}}
said to
Moses,
“I am going
to rain
bread
from
heaven
for you, and the people
will go out
and gather
the amount
for each
day,
so that I may
test
them. Will they walk
in my law
or
not?
\VS{5}On
the sixth
day
they will prepare
what
they bring
in, and it will be
twice
as
much as they gather
every other day.”
\par }{\PP \VS{6}Moses
and Aaron
said
to
all
the Israelites,
“In the evening
you will know
that
the {\ND{Lord}}
has brought you out
of the land
of Egypt,
\VS{7}and in the morning
you will see
the glory
of the {\ND{Lord}}, because he has heard
your murmurings
against
the {\ND{Lord}}. As for us, what
are we, that
you should murmur
against us?”
\par }{\PP \VS{8}Moses
said,
“You will know this when the
{\ND{Lord}}
gives
you meat
to eat
in the evening
and bread
in the morning
to satisfy
you, because the
{\ND{Lord}}
has heard
your murmurings
that
you
are murmuring
against
him. As for us, what
are we? Your murmurings
are not
against
us, but against
the {\ND{Lord}}.”
\par }{\PP \VS{9}Then Moses
said
to
Aaron,
“Tell
the whole
community
of the Israelites,
‘Come
before
the {\ND{Lord}}, because
he has heard
your murmurings.’ ”
\par }{\PP \VS{10}As
Aaron
spoke
to
the whole
community
of the Israelites
and they looked
toward
the desert,
there
the glory
of the {\ND{Lord}}
appeared
in the cloud,
\VS{11}and the
{\ND{Lord}}
spoke
to
Moses:
\VS{12}“I have heard
the murmurings
of the Israelites.
Tell
them, ‘During
the evening
you will eat
meat,
and in the morning
you will be satisfied
with bread,
so that you may know
that
I
am the
{\ND{Lord}}
your God.’ ”
\par }{\PP \VS{13}In the evening
the quail
came up
and covered
the camp,
and in the morning
a layer
of dew
was all around
the camp.
\VS{14}When the layer
of dew
had evaporated,
there on
the surface
of the desert
was a thin
flaky substance,
thin
like frost
on
the earth.
\VS{15}When
the Israelites
saw it, they said
to one
another, “What is it?” because
they did not
know
what
it was.
Moses
said
to
them, “It is
the bread
that
the {\ND{Lord}}
has given
you for food.
\par }{\PP \VS{16}“This
is what
the {\ND{Lord}}
has
commanded:
 ‘Each person
is to gather
from
it what
he can eat,
an omer
per person
according to the number
of your people; each
one will pick
it up for whoever lives in his tent.’ ”
\VS{17}The Israelites
did
so,
and they gathered
– some more,
some less.
\VS{18}When they measured
with an omer,
the one who gathered
much
had
nothing
left over,
and the one who gathered little
lacked
nothing; each
one had gathered
what
he could eat.
\par }{\PP \VS{19}Moses
said
to them,
“No
one
is to keep
any
of it until
morning.”
\VS{20}But they did not
listen
to
Moses;
some
kept part
of
it until
morning,
and it was full
of worms
and began to stink,
and Moses
was angry
with them.
\VS{21}So they gathered
it each morning,
each person
according
to what he could eat,
and when the sun
got hot,
it would melt.
\VS{22}And on
the sixth
day
they gathered
twice
as much food,
two
omers
per
person; and all
the leaders
of the community
came
and told
Moses.
\VS{23}He said
to
them, “This
is what
the {\ND{Lord}}
has said: ‘Tomorrow
is a time of cessation from work,
a holy
Sabbath
to the
{\ND{Lord}}. Whatever
you want to bake,
bake
today; whatever
you want to boil,
boil
today; whatever
is left
put aside
for yourselves to be kept
until
morning.’ ”
\par }{\PP \VS{24}So they put
it aside
until
the morning,
just
as Moses
had commanded,
and it did not
stink,
nor
were there any worms in it.
\VS{25}Moses
said,
“Eat
it today,
for
today
is a Sabbath
to the
{\ND{Lord}}; today
you will not
find
it in the area.
\VS{26}Six
days
you will gather
it, but on the seventh
day,
the Sabbath,
there will not
be any.”
\par }{\PP \VS{27}On
the seventh
day
some
of the people
went out
to gather
it, but they found nothing.
\VS{28}So the
{\ND{Lord}}
said
to
Moses,
“How long
do you refuse
to obey
my commandments
and my instructions?
\VS{29}See,
because
the {\ND{Lord}}
has given
you the Sabbath,
that is why
he is
giving
you food
for two days
on the sixth
day.
Each
of you stay
where you are; let no
one
go out
of his place
on
the seventh
day.”
\VS{30}So
the people
rested
on the seventh
day.
\par }{\PP \VS{31}The house
of Israel
called
its name
“manna.”
It was like coriander
seed
and was white,
and it tasted
like wafers
with honey.
\par }{\PP \VS{32}Moses
said,
“This
is what
the {\ND{Lord}}
has
commanded: ‘Fill
an omer
with
it to be kept
for generations
to come, so that
they may see
the food
I fed
you in the desert
when I brought
you out
from the land
of Egypt.’ ”
\VS{33}Moses
said
to
Aaron,
“Take
a jar
and put
in it an omer
full
of manna,
and place
it before
the {\ND{Lord}}
to be kept
for generations to come.”
\VS{34}Just
as the
{\ND{Lord}}
commanded
Moses,
so Aaron
placed
it before
the Testimony
for safekeeping.
\par }{\PP \VS{35}Now the Israelites
ate
manna
forty
years,
until
they came
to
a land
that was inhabited;
they ate
manna
until
they came
to
the border
of the land
of Canaan.
\VS{36}(Now an omer
is one tenth
of an ephah.)


\par }\Chap{17}{\PP \VerseOne{1}The whole
community
of the Israelites
traveled
on their journey
from the Desert
of Sin
according to
the
{\ND{Lord}}’s
instruction,
and they pitched camp
in Rephidim.
Now there was no
water
for the people
to drink.
\VS{2}So the people
contended
with
Moses,
and they said,
“Give
us water
to drink!” Moses
said
to them, “Why
do you contend
with me? Why
do you test
the {\ND{Lord}}?”
\VS{3}But the people
were very thirsty
there
for water,
and they murmured
against
Moses
and said,
“Why
in the world
did you bring us up
out of Egypt
– to kill
us and our children
and our cattle
with thirst?”
\par }{\PP \VS{4}Then Moses
cried out
to
the {\ND{Lord}}, “What
will I do
with this
people? – a little more and they will stone me!”
\VS{5}The
{\ND{Lord}}
said
to
Moses,
“Go
over before
the people;
take
with
you some of the elders
of Israel
and take
in your hand
your staff
with which
you struck
the
Nile
and go.
\VS{6}I
will be standing
before
you there
on
the rock
in Horeb,
and you will strike
the rock,
and water
will come out
of it
so that the people
may drink.”
And Moses
did
so
in
plain view
of the elders
of Israel.
\par }{\PP \VS{7}He called
the name
of the place
Massah
and Meribah,
because of the contending
of the Israelites
and because
of their testing
the

{\ND{Lord}},
 saying,
“Is
the {\ND{Lord}}
among
us or
not?”
\par }{\SH Victory over the Amalekites
\par }{\PP \VS{8}Amalek
came
and attacked
Israel
in Rephidim.
\VS{9}So Moses
said to
Joshua,
“Choose
some of our men
and go out,
fight
against Amalek.
Tomorrow
I
will stand
on
top
of the hill
with the staff
of God
in my hand.”
\par }{\PP \VS{10}So
Joshua
fought
against Amalek
just as
Moses
had instructed him; and Moses and
Aaron and
Hur went
up to
the top of
the hill.
\VS{11}Whenever
Moses
would raise
his hands,
then Israel
prevailed,
but whenever
he would rest
his hands,
then
Amalek
prevailed.
\VS{12}When the hands
of Moses
became heavy,
they took
a stone
and put
it under
him, and Aaron
and Hur
held up
his hands,
one on one
side and one
on the other, and so
his hands
were steady
until
the sun
went down.
\VS{13}So Joshua
destroyed
Amalek
and his army
with the sword.
\par }{\PP \VS{14}The
{\ND{Lord}}
said
to
Moses,
“Write
this
as a memorial
in the book,
and rehearse it
in Joshua’s
hearing;
for
I will surely
wipe out
the remembrance
of Amalek
from under
heaven.
\VS{15}Moses
built
an altar,
and he called
it “The
{\ND{Lord}}
is my Banner,”
\VS{16}for he said,
“For
a hand
was lifted up to the throne
of the
{\ND{Lord}} –
that the
{\ND{Lord}}
will have war
with Amalek
from generation
to generation.”


\par }\Chap{18}{\PP \VerseOne{1}Jethro,
the priest
of Midian,
Moses’
father-in-law,
heard
about all
that
God
had
done
for Moses
and for his people
Israel,
that
the {\ND{Lord}}
had brought
Israel
out of Egypt.
\par }{\PP \VS{2}Jethro,
Moses’
father-in-law,
took
Moses’
wife
Zipporah
after
he had sent her back,
\VS{3}and her two
sons,
one
of whom
was named
Gershom
(for
Moses had
said,
“I have been
a foreigner
in a foreign
land”),
\VS{4}and the other
Eliezer
(for
Moses had said, “The God
of my father
has been my help
and delivered
me from the sword
of Pharaoh”).
\par }{\PP \VS{5}Jethro,
Moses’ father-in-law,
together with Moses’
sons
and his wife,
came
to
Moses
in the desert
where
he
was camping
by the mountain
of God.
\VS{6}He said
to
Moses,
“I,
your father-in-law
Jethro,
am coming
to
you, along with your wife
and her two
sons
with her.”
\VS{7}Moses
went out
to meet
his father-in-law
and bowed
down and kissed
him; they each
asked
about the other’s
welfare,
and then they went
into the tent.
\VS{8}Moses
told
his father-in-law
all
that
the {\ND{Lord}}
had
done
to Pharaoh
and to Egypt
for Israel’s
sake,
and all
the hardship
that had
come on them along the way,
and how the
{\ND{Lord}}
had delivered them.
\par }{\PP \VS{9}Jethro
rejoiced
because
of all
the good
that
the {\ND{Lord}}
had
done
for Israel,
whom
he had delivered
from the hand
of Egypt.
\VS{10}Jethro
said,
“Blessed
be the
{\ND{Lord}}
who
has delivered
you from the hand
of Egypt,
and from the hand
of Pharaoh,
who
has delivered
the people
from
the Egyptians’
control!
\VS{11}Now
I know
that
the {\ND{Lord}}
is greater
than all
the gods,
for
in the thing
in which
they dealt proudly against them he has destroyed them.”
\VS{12}Then Jethro,
Moses’
father-in-law,
brought
a burnt offering
and sacrifices
for God,
and Aaron
and all
the elders
of Israel
came
to eat
food
with
the father-in-law
of Moses
before
God.
\par }{\PP \VS{13}On the next
day Moses
sat
to judge
the people,
and the people
stood
around Moses
from
morning
until
evening.
\VS{14}When Moses’
father-in-law
saw
all
that
he was
doing
for the people,
he said,
“What
is this
that
you
are doing
for the people? Why
are you
sitting
by yourself,
and all
the people
stand
around you from
morning
until
evening?”
\par }{\PP \VS{15}Moses
said
to his father-in-law,
“Because
the people
come
to me
to inquire
of God.
\VS{16}When
they have a dispute,
it comes
to
me and I decide
between
a man
and his neighbor,
and I make known
the
decrees
of God
and his laws.”
\par }{\PP \VS{17}Moses’
father-in-law
said
to him,
“What
you
are doing
is not
good!
\VS{18}You will surely
wear
out, both
you
and these
people
who
are with
you, for
this is too heavy
a burden
for you; you are not
able
to do
it by yourself.
\VS{19}Now
listen
to me,
I will give you advice,
and may God
be with
you: You be a representative
for the people
to God,
and you
bring
their disputes
to
God;
\VS{20}warn
them
of the
statutes
and the
laws,
and make known
to them the way
in which
they must walk
and the
work
they must do.
\VS{21}But you
choose
from the people
capable
men,
God-fearing,
men
of truth,
those who hate
bribes,
and put
them over
the people as rulers
of thousands,
rulers
of hundreds,
rulers
of fifties,
and rulers
of tens.
\VS{22}They will judge
the
people
under normal
circumstances,
and every
difficult
case they will bring
to
you, but every
small
case they themselves
will judge,
so that you may make it easier
for yourself, and they
will bear
the burden with you.
\VS{23}If
you do
this
thing,
and God
so
commands
you, then you will be able
to endure,
and all
these
people
will be able to go
home
satisfied.”
\par }{\PP \VS{24}Moses
listened
to his father-in-law
and did
everything
he had
said.
\VS{25}Moses
chose
capable
men
from all
Israel,
and he made
them heads
over
the people,
rulers
of thousands,
rulers
of hundreds,
rulers
of fifties,
and rulers
of tens.
\VS{26}They judged
the people
under normal
circumstances;
the difficult
cases
they would bring
to
Moses,
but every
small
case they would judge
themselves.
\par }{\PP \VS{27}Then Moses
sent
his father-in-law
on his way, and so Jethro went
to
his own land.


\par }\Chap{19}{\PP \VerseOne{1}In the third
month
after the Israelites
went out from
the land
of Egypt,
on the very day,
they came
to the Desert
of Sinai.
\VS{2}After they journeyed
from Rephidim,
they came
to the Desert
of Sinai,
and they camped
in the desert;
Israel
camped
there
in front
of the mountain.
\par }{\PP \VS{3}Moses
went up
to
God,
and the
{\ND{Lord}}
called
to
him from
the mountain,
“Thus
you will tell
the house
of Jacob,
and declare
to the people of Israel:
\VS{4}‘You yourselves
have seen
what
I did
to Egypt
and how I lifted
you on
eagles’
wings
and brought
you to myself.
\VS{5}And now,
if
you will diligently
listen
to me
and keep
my covenant,
then you will be
my special possession
out of all
the nations,
for
all
the earth is mine,
\VS{6}and you
will be
to me a kingdom
of priests
and a holy
nation.’
These
are the words
that
you will speak
to
the Israelites.”
\par }{\PP \VS{7}So Moses
came
and summoned
the elders
of Israel. He set
before
them all
these
words
that
the {\ND{Lord}}
had commanded him,
\VS{8}and all
the people
answered
together,
“All
that
the {\ND{Lord}}
has commanded
we will do!” So
Moses
brought the words
of the people
back to
the {\ND{Lord}}.
\par }{\PP \VS{9}The
{\ND{Lord}}
said
to
Moses,
“I
am going to come
to
you in a dense
cloud,
so that
the people
may hear
when I speak
with
you and so that
they will always
believe
in you.” And Moses
told
the words
of the people
to
the {\ND{Lord}}.
\par }{\PP \VS{10}The
{\ND{Lord}}
said
to
Moses,
“Go
to
the people
and sanctify
them today
and tomorrow,
and make them wash
their clothes
\VS{11}and be
ready
for the third
day,
for
on
the third
day
the {\ND{Lord}}
will come down
on
Mount
Sinai
in the sight
of all
the people.
\VS{12}You must set boundaries
for the people
all around,
saying,
‘Take heed
to yourselves not to go up
on the mountain
nor touch
its edge.
Whoever
touches
the mountain
will surely
be put to death!
\VS{13}No
hand
will touch him – but he will surely be stoned or shot through, whether a beast or a human being; he must not live.’ When the ram’s horn sounds a long blast they may go up on the mountain.”
\par }{\PP \VS{14}Then Moses
went down
from
the mountain
to
the people
and sanctified
the people,
and they washed
their clothes.
\VS{15}He said
to
the people,
“Be
ready
for the third
day.
Do not
go near
your wives.”
\par }{\PP \VS{16}On
the third
day
in the morning
there was
thunder
and lightning
and a dense
cloud
on
the mountain,
and the sound
of a very
loud
horn;
all
the people
who
were in the camp
trembled.
\VS{17}Moses
brought
the
people
out of the camp
to meet
God,
and they took their place at the foot
of the mountain.
\VS{18}Now Mount
Sinai
was completely
covered with smoke
because
the {\ND{Lord}}
had
descended
on
it in fire,
and its smoke
went up
like the smoke
of a great furnace,
and the whole
mountain
shook violently.
\VS{19}When
the sound
of the horn
grew louder
and louder,
Moses
was speaking
and God
was answering him
with a voice.
\par }{\PP \VS{20}The
{\ND{Lord}}
came down
on
Mount
Sinai,
on the top
of the mountain,
and the
{\ND{Lord}}
summoned
Moses
to
the top
of the mountain,
and Moses
went up.
\VS{21}The
{\ND{Lord}}
said
to
Moses,
“Go down
and solemnly
warn the people,
lest
they force
their way through to
the {\ND{Lord}}
to look,
and many
of them perish.
\VS{22}Let the priests
also,
who approach
the {\ND{Lord}}, sanctify
themselves, lest
the {\ND{Lord}}
break through against them.”
\par }{\PP \VS{23}Moses
said
to
the {\ND{Lord}}, “The people
are not
able
to come up
to
Mount
Sinai,
because
you
solemnly
warned
us, ‘Set boundaries
for the mountain
and set it apart.’ ”
\VS{24}The
{\ND{Lord}}
said
to him,
“Go,
get down,
and come up,
and Aaron
with
you,
but do not
let the priests
and the people
force
their way through to come up
to
the {\ND{Lord}}, lest
he break through against them.”
\VS{25}So Moses
went down
to
the people
and spoke
to them.


\par }\Chap{20}{\PP \VerseOne{1}God
spoke
all
these
words:
\par }{\PP \VS{2}“I,
the {\ND{Lord}}, am your God,
who
brought
you from the land
of Egypt,
from the house
of slavery.
\par }{\PP \VS{3}“You shall have no
other
gods
before me.
\par }{\PP \VS{4}“You shall not
make
for yourself a carved image
or any
likeness
of anything that
is in heaven
above
or that
is on the earth
beneath
or that
is in the water
below.
\VS{5}You shall not
bow
down to them or
serve
them, for
I,
the {\ND{Lord}}, your God,
am a jealous
God,
responding
to the transgression
of fathers
by dealing
with children
to the third and fourth generations
of those who
reject
me,
\VS{6}and showing
covenant faithfulness
to a thousand
generations of those who love
me and keep
my commandments.
\par }{\PP \VS{7}“You shall not
take
the name
of the {\ND{Lord}}
your God
in vain,
for
the {\ND{Lord}}
will not
hold guiltless
anyone who
takes his name
in vain.
\par }{\PP \VS{8}“Remember
the Sabbath
day
to set it apart as holy.
\VS{9}For six
days
you may labor
and do
all
your work,
\VS{10}but the seventh
day
is a Sabbath
to the
{\ND{Lord}}
your God;
on it you shall not
do
any
work,
you,
or your son,
or your daughter,
or your male servant,
or your female servant,
or your cattle,
or the resident foreigner
who is in your gates.
\VS{11}For
in six
days
the {\ND{Lord}}
made
the
heavens
and the
earth
and the
sea
and all
that
is in them, and he rested
on
the seventh
day;
therefore
the {\ND{Lord}}
blessed
the
Sabbath
day
and set it apart as holy.
\par }{\PP \VS{12}“Honor
your father
and your mother,
that
you may live a long
time
in
the land
the {\ND{Lord}}
your God
is giving to you.
\par }{\PP \VS{13}“You shall not
murder.
\par }{\PP \VS{14}“You shall not
commit adultery.
\par }{\PP \VS{15}“You shall not
steal.
\par }{\PP \VS{16}“You shall not
give
false
testimony
against your neighbor.
\par }{\PP \VS{17}“You shall not
covet
your neighbor’s
house.
You shall not
covet
your neighbor’s
wife,
nor his male servant,
nor his female servant,
nor his ox,
nor his donkey,
nor anything
that belongs
to your neighbor.”
\par }{\PP \VS{18}All
the people
were seeing
the thundering
and the lightning,
and heard
the sound
of the horn,
and saw the mountain
smoking
– and when the people
saw
it they trembled
with fear and kept
their
distance.
\VS{19}They said
to
Moses,
“You
speak
to us
and we
will listen,
but do not
let God
speak
with
us, lest
we die.”
\VS{20}Moses
said
to
the people,
“Do not
fear,
for
God
has come
to test
you, that the fear
of him
may
be
before
you so that you do not
sin.”
\VS{21}The people
kept
their distance,
but Moses
drew near
the thick darkness
where
God was.
\par }{\SH The Altar
\par }{\PP \VS{22}The
{\ND{Lord}}
said
to
Moses: “Thus
you will tell
the Israelites: ‘You yourselves
have seen
that
I have spoken
with
you from
heaven.
\VS{23}You must not
make
gods
of silver
alongside
me, nor
make
gods
of gold for yourselves.
\par }{\PP \VS{24}‘You must make for me an altar
made
of earth,
and you will sacrifice
on
it your burnt offerings
and your peace offerings,
your sheep
and your cattle.
In every
place
where
I cause
my name
to
be honored
I will come
to
you and I will bless you.
\VS{25}If
you make
me an altar
of stone,
you must not
build
it of stones shaped with tools,
for if
you use
your tool
on
it you have defiled it.
\VS{26}And you must not
go up
by steps
to my altar,
so that
your nakedness
is not
exposed.’


\par }\Chap{21}{\PP \VerseOne{1}“These
are the decisions
that
you will set
before them:
\par }{\SH Hebrew Servants
\par }{\PP \VS{2}“If
you buy
a Hebrew
servant,
he is to serve
you for six
years,
but in the seventh
year he will go out
free
without paying anything.
\VS{3}If
he came
in by himself
he will go out
by himself;
if
he had a wife
when he came
in, then his wife
will go out with him.
\VS{4}If
his master
gave
him a wife,
and she bore
sons
or
daughters,
the wife
and the children
will belong
to her master,
and he will go out
by himself.
\VS{5}But if
the servant
should declare, ‘I love
my master,
my wife,
and my children;
I will not
go out
free,’
\VS{6}then
his master
must bring him to the judges, and he will bring
him to
the door
or
the doorposts,
and his master
will pierce his ear
with an awl,
and he shall serve
him forever.
\par }{\PP \VS{7}“If
a man
sells
his daughter
as a female servant,
she will not
go
out
as the male servants do.
\VS{8}If
she does not please
her master,
who has designated
her for himself, then he must let her be redeemed.
He has no
right
to sell
her to a foreign
nation,
because he has dealt deceitfully with her.
\VS{9}If
he designated
her for his son,
then he will deal with her according
to the customary
rights
of daughters.
\VS{10}If
he takes
another
wife, he must not diminish
the first one’s food,
her clothing,
or
her marital rights.
\VS{11}If
he does not
provide
her with these
three
things, then she will go out
free, without
paying
money.
\par }{\SH Personal Injuries
\par }{\PP \VS{12}“Whoever strikes
someone
so
that he dies
must surely be put to death.
\VS{13}But if he does not
do it with premeditation,
but it happens
by accident,
then I will appoint
for you a place
where
he may flee.
\VS{14}But if
a man
willfully
attacks his neighbor
to kill
him cunningly,
you will take
him
even from my altar
that he may die.
\par }{\PP \VS{15}“Whoever strikes
his father
or his mother
must surely be put to death.
\par }{\PP \VS{16}“Whoever kidnaps
someone
and sells
him,
or is caught
still holding
him,
must surely be put to death.
\par }{\PP \VS{17}“Whoever treats
his father
or his mother
disgracefully must
surely be put to death.
\par }{\PP \VS{18}“If
men
fight,
and one
strikes
his neighbor
with a stone
or
with his fist
and he does not
die,
but must remain
in bed,
\VS{19}and then if
he gets up
and walks
about outside
on
his staff,
then the one who struck
him is innocent,
except
he must pay for the injured person’s loss of time
and see to it that he is fully healed.
\par }{\PP \VS{20}“If
a man strikes
his male servant
or
his female servant
with a staff
so that he or she dies
as
a result
of the blow, he will surely
be punished.
\VS{21}However,
if
the injured servant survives
one or
two days,
the owner
will not
be punished,
for he has suffered the loss.
\par }{\PP \VS{22}“If
men
fight
and hit
a pregnant
woman
and her child
is born prematurely,
but there is no
serious injury,
he will
surely be
punished
in accordance with what the
woman’s
husband
demands
of him, and he will
pay what the court decides.
\VS{23}But if
there is serious injury,
then you will give
a life
for
a life,
\VS{24}eye
for
eye,
tooth
for
tooth,
hand
for
hand,
foot
for
foot,
\VS{25}burn
for
burn,
wound
for
wound,
bruise
for
bruise.
\par }{\PP \VS{26}“If
a man
strikes
the
eye
of his male servant
or
his female servant
so that he destroys
it, he will let the servant go free
as compensation
for the eye.
\VS{27}If
he knocks out
the tooth
of his male servant
or
his female servant,
he will let the servant go free
as
compensation
for the tooth.
\par }{\SH Laws about Animals
\par }{\PP \VS{28}“If
an ox
gores
a man
or
a woman
so
that either dies,
then the ox
must surely be stoned
and its flesh
must not
be eaten,
but the owner
of the ox
will be acquitted.
\VS{29}But if
the ox
had the habit
of goring,
and its owner
was warned,
and he did not
take the necessary precautions,
and then it killed
a man
or
a woman,
the ox
must be stoned
and the man
must
be put to death.
\VS{30}If
a ransom
is set
for him,
then he must pay
the redemption
for his life
according to whatever
amount
was set
for him.
\VS{31}If
the ox gores
a son
or
a daughter,
the owner will be dealt with according
to this rule.
\VS{32}If
the ox
gores
a male servant
or
a female servant,
the owner
must pay
thirty
shekels
of silver,
and the ox
must be stoned.
\par }{\PP \VS{33}“If
a man
opens
a pit
or
if
a man
digs
a pit
and does not
cover
it, and an ox
or
a donkey
falls into it,
\VS{34}the owner
of the pit
must repay
the loss. He must give money
to its owner,
and the dead
animal will become his.
\VS{35}If
the ox
of one man
injures
the ox
of his neighbor
so that it dies,
then they will sell
the
live
ox
and divide
its proceeds, and they will also
divide
the dead ox.
\VS{36}Or
if it is known
that
the ox
had the habit
of goring,
and its owner did not
take the necessary
precautions, he must
surely pay
ox
for
ox,
and the dead
animal will become his.


\par }\Chap{22}{\PP \VerseOne{1} “If
a man
steals
an ox
or
a sheep
and kills
it or
sells
it, he must pay back
five
head of cattle
for
the ox,
and four
sheep
for
the one sheep.
\par }{\PP \VS{2}“If
a thief
is caught
breaking
in and is struck
so that he dies,
there will be no
blood guilt for him.
\VS{3}If
the sun
has risen
on
him, then there is blood
guilt for him. A thief must surely
make full restitution;
if
he has nothing,
then he will be sold
for his theft.
\VS{4}If
the stolen item
should in fact
be found
alive
in his possession,
whether it be an ox
or a donkey
or a sheep,
he must pay back
double.
\par }{\PP \VS{5}“If
a man
grazes
his livestock in a field
or
a vineyard,
and he lets the livestock
loose
and they graze
in the field
of another
man, he must make restitution
from the best
of his own field
and the best
of his own vineyard.
\par }{\PP \VS{6}“If
a fire
breaks out
and spreads
to thorn bushes,
so that stacked grain
or
standing grain
or
the whole field
is consumed,
the one who started
the
fire
must surely make
restitution.
\par }{\PP \VS{7}“If
a man
gives
his neighbor
money
or
articles
for safekeeping,
and it is stolen
from the man’s
house,
if
the thief
is caught,
he must repay
double.
\VS{8}If
the thief
is not
caught,
then
the owner
of the house
will be brought before the judges to
see whether
he has laid
his hand
on
his neighbor’s goods.
\VS{9}In
all
cases
of illegal possessions,
whether for
an ox,
a donkey,
a sheep,
a garment,
or any
kind of lost
item, about
which
someone says
‘This
belongs to me,’ the matter
of the two
of them will come
before
the judges,
and the one whom
the judges declare guilty
must repay
double
to his neighbor.
\VS{10}If
a man
gives
his neighbor
a donkey
or
an ox
or
a sheep
or any
beast
to keep,
and it dies
or
is hurt
or
is carried away
without
anyone seeing it,
\VS{11}then there will be an oath
to the
{\ND{Lord}}
between
the two
of them, that he has not
laid
his hand
on
his neighbor’s
goods, and its owner
will accept
this, and he will not
have to pay.
\VS{12}But if
it was stolen
from
him,
he will pay
its owner.
\VS{13}If
it is torn
in pieces,
then he will bring
it for evidence,
and he will not
have to pay
for what was torn.
\par }{\PP \VS{14}“If
a man
borrows
an animal
from his neighbor,
and it is hurt
or
dies
when its owner
was not
with
it,
the man who borrowed
it will surely pay.
\VS{15}If
its owner
was with
it, he will not
have to pay;
if
it was hired,
what was paid
for the hire covers it.
\par }{\SH Moral and Ceremonial Laws
\par }{\PP \VS{16}“If
a man
seduces
a virgin
who
is not
engaged
and has sexual relations
with
her, he must surely endow
her to be his wife.
\VS{17}If
her
father
refuses
to give
her to him, he must pay
money
for the bride price
of virgins.
\par }{\PP \VS{18}“You must not
allow a sorceress
to live.
\par }{\PP \VS{19}“Whoever
has sexual
relations with
a beast
must surely be put to death.
\par }{\PP \VS{20}“Whoever sacrifices
to a god
other than
the {\ND{Lord}}
alone
must be utterly destroyed.
\par }{\PP \VS{21}“You must not wrong
a foreigner
nor
oppress
him, for
you were foreigners
in the land
of Egypt.
\par }{\PP \VS{22}“You must not
afflict
any
widow
or orphan.
\VS{23}If
you afflict
them in any way and they cry
to
me, I will surely hear
their cry,
\VS{24}and my anger
will burn
and I will kill
you with the sword,
and your wives
will be widows
and your children
will be fatherless.
\par }{\PP \VS{25}“If
you lend
money
to any of my people
who are needy
among you, do not
be like
a moneylender
to him; do not
charge
him interest.
\VS{26}If
you do take
the garment
of your neighbor
in pledge,
you must return
it to him by the time
the sun
goes down,
\VS{27}for
it is
his only
covering
– it is
his garment
for his body.
What
else can he sleep
in? And when
he cries out
to me,
I will hear,
for
I am
gracious.
\par }{\PP \VS{28}“You must not
blaspheme
God
or
curse
the ruler
of your people.
\par }{\PP \VS{29}“Do not
hold back
offerings from your granaries
or your vats.
You must give
me the firstborn
of your sons.
\VS{30}You must also do
this
for your oxen
and for your sheep;
seven
days
they may
remain with
their mothers,
but give
them to me on
the eighth
day.
\par }{\PP \VS{31}“You will be
holy
people
to me; you must not
eat
any meat
torn
by animals in the field.
You must throw
it to the dogs.


\par }\Chap{23}{\PP \VerseOne{1}“You must not
give
a false
report.
Do not
make
common cause
with
the wicked
to be
a malicious
witness.
\par }{\PP \VS{2}“You must not
follow
a crowd
in doing evil
things; in a lawsuit
you must not
offer testimony
that agrees
with a crowd
so as to pervert justice,
\VS{3}and you must not
show partiality
to a poor
man in his lawsuit.
\par }{\PP \VS{4}“If
you encounter
your enemy’s
ox
or
donkey
wandering
off, you must by all means return it to him.
\VS{5}If
you see
the donkey
of someone who hates
you fallen
under
its load,
you must not
ignore
him, but
be sure
to help
him with it.
\par }{\PP \VS{6}“You must not
turn
away justice
for your poor
people in their lawsuits.
\VS{7}Keep your distance
from a false charge – do not kill the innocent and the righteous, for I will not justify the wicked.
\par }{\PP \VS{8}“You must not
accept
a bribe,
for
a bribe
blinds
those who see
and subverts
the words
of the righteous.
\par }{\PP \VS{9}“You must not
oppress
a foreigner,
since you
know
the life
of a foreigner,
for
you were foreigners
in the land
of Egypt.
\par }{\SH Sabbaths and Feasts
\par }{\PP \VS{10}“For six
years
you are to sow
your land
and gather
in its produce.
\VS{11}But in the seventh
year you must let it lie fallow
and leave
it alone so that the poor
of your people
may eat,
and what they leave
any animal
in the field
may eat; you must do
likewise
with your vineyard
and your olive grove.
\VS{12}For six
days
you are to do
your work,
but on the seventh
day
you must cease,
in order that
your ox
and your donkey
may rest
and that your female servant’s
son
and any hired help
may refresh themselves.
\par }{\PP \VS{13}“Pay attention
to do everything
I have
told
you, and do not
even mention
the names
of other
gods
– do not let them be heard
on
your lips.
\par }{\PP \VS{14}“Three
times
in the year
you must make a pilgrim feast to me.
\VS{15}You are to observe
the Feast
of Unleavened
Bread; seven
days
you must eat
bread made without yeast,
as
I commanded
you, at the appointed
time of the month
of Abib,
for
at that time you came out
of Egypt.
No
one may appear
before
me empty-handed.
\par }{\PP \VS{16}“You are also to observe the Feast
of Harvest,
the firstfruits
of your labors
that you
have sown
in the field,
and the Feast
of Ingathering
at the end
of the year
when you have gathered
in your harvest
out
of the field.
\VS{17}At three
times
in the year
all
your males
will appear
before
the Lord

{\ND{God}}.
\par }{\PP \VS{18}“You must not
offer
the blood
of my sacrifice with bread containing yeast;
the fat
of my festal
sacrifice
must
not
remain
until
morning.
\VS{19}The first
of the firstfruits
of your soil
you must bring
to the house
of the {\ND{Lord}}
your God.
\par }{\PP “You must not
cook
a young goat
in its mother’s
milk.
\par }{\SH The Angel of the Presence
\par }{\PP \VS{20}“I
am
going to send
an angel
before
you to protect
you as you journey
and to bring
you into
the place
that
I have prepared.
\VS{21}Take heed
because
of him, and obey
his voice;
do not
rebel
against him, for
he will not
pardon
your transgressions,
for
my name
is in him.
\VS{22}But if
you diligently
obey
him
and do
all
that
I command,
then I will be an enemy
to your enemies,
and I will be an adversary
to your adversaries.
\VS{23}For
my angel
will go
before
you and bring
you to
the Amorites,
the Hittites,
the Perizzites,
the Canaanites,
the Hivites,
and the Jebusites,
and I will destroy them completely.
\par }{\PP \VS{24}“You must not
bow
down to their gods;
you must not
serve
them or
do
according
to their practices.
Instead
you must
completely overthrow
them and smash their standing stones to pieces.
\VS{25}You must serve
the {\ND{Lord}}
your God,
and he will bless
your bread
and your water,
and I will remove
sickness
from your midst.
\VS{26}No
woman will miscarry
her young or be
barren
in your
land.
I will fulfill
the number
of your days.
\par }{\PP \VS{27}“I will send
my terror
before
you, and I will destroy
all
the people
whom
you encounter;
I will make
all
your enemies
turn their backs
to you.
\VS{28}I will send
hornets
before
you that will drive
out the Hivite,
the Canaanite,
and the
Hittite
before you.
\VS{29}I will not
drive
them out before
you in one
year,
lest
the land
become desolate
and the wild
animals
multiply
against you.
\VS{30}Little
by little
I will drive
them out before
you, until
you become fruitful
and inherit
the land.
\VS{31}I will set
your boundaries
from the Red
Sea
to
the sea
of the Philistines,
and from the desert
to
the River,
for
I will deliver
the
inhabitants
of the land
into your hand, and you will drive
them out before you.
\par }{\PP \VS{32}“You must make
no
covenant
with them or with their gods.
\VS{33}They must not
live
in your land,
lest
they make you sin
against me, for if
you serve
their gods,
it will surely
be
a snare to you.”


\par }\Chap{24}{\PP \VerseOne{1}But to
Moses
the
{\ND{Lord}} said,
“Come up
to
the {\ND{Lord}}, you
and Aaron,
Nadab
and Abihu,
and seventy
of the elders
of Israel,
and worship
from a distance.
\VS{2}Moses
alone
may come near
the {\ND{Lord}}, but
the others
must not
come near,
nor
may the people
go up
with him.”
\par }{\PP \VS{3}Moses
came
and told
the people
all
the
{\ND{Lord}}’s
words
and all
the decisions.
All
the people
answered
together, “We are
willing
to do
all
the words
that
the {\ND{Lord}}
has said,”
\VS{4}and Moses
wrote
down all
the words
of the {\ND{Lord}}. Early
in the morning
he built
an altar
at the foot
of the mountain
and arranged twelve
standing stones –
according to the twelve
tribes
of Israel.
\VS{5}He sent
young
Israelite
men,
and they offered
burnt offerings
and sacrificed
young bulls
for peace
offerings
to the
{\ND{Lord}}.
\VS{6}Moses
took
half
of the blood
and put
it in bowls,
and half
of the blood
he splashed
on
the altar.
\VS{7}He took
the Book
of the Covenant
and read
it aloud
to the people,
and they said,
“We are willing to do
and obey
all
that
the {\ND{Lord}}
has spoken.”
\VS{8}So
Moses
took
the blood
and splashed
it on
the people
and said,
“This is
the blood
of the covenant
that
the {\ND{Lord}}
has
made with
you in accordance with all
these
words.”
\par }{\PP \VS{9}Moses
and Aaron,
Nadab
and Abihu,
and the seventy
elders
of Israel
went up,
\VS{10}and they saw
the God
of Israel.
Under
his feet
there was something like a pavement
made of sapphire,
clear
like the sky itself.
\VS{11}But he did not
lay a hand
on the leaders
of the Israelites,
so
they saw
God,
and they ate
and they drank.
\VS{12}\par }{\PP The
{\ND{Lord}}
said
to
Moses,
“Come up
to me to
the mountain
and remain there,
and I will give
you the
stone
tablets
with the law
and the commandments
that
I have written,
so that you may teach them.”
\VS{13}So
Moses
set out with Joshua
his attendant,
and Moses
went up
the mountain
of God.
\VS{14}He told
the elders,
“Wait
for us in this
place until
we return
to
you. Here
are Aaron
and Hur
with
you. Whoever
has any matters
of dispute
can
approach
them.”
\par }{\PP \VS{15}Moses
went up
the mountain,
and the cloud
covered
the mountain.
\VS{16}The glory
of the {\ND{Lord}}
resided
on
Mount
Sinai,
and the cloud
covered
it for six
days. On
the seventh
day
he called
to
Moses
from within
the cloud.
\VS{17}Now the appearance
of the glory
of the {\ND{Lord}}
was like a devouring
fire
on the top
of the mountain
in plain view
of the people.
\VS{18}Moses
went
into
the cloud
when he went up
the mountain,
and Moses
was on the mountain
forty
days
and forty
nights.


\par }\Chap{25}{\PP \VerseOne{1}The
{\ND{Lord}}
spoke
to
Moses:
\VS{2}“Tell
the Israelites
to take
an offering
for me; from every
person
motivated
by a willing
heart
you are to receive
my offering.
\VS{3}This
is the offering
you are to accept
from them: gold,
silver,
bronze,
\VS{4}blue,
purple,
scarlet,
fine linen,
goat’s hair,
\VS{5}ram
skins
dyed red,
fine
leather,
acacia
wood,
\VS{6}oil
for the light,
spices
for the anointing
oil
and for fragrant
incense,
\VS{7}onyx
stones,
and other gems
to be set
in the ephod
and in the breastpiece.
\VS{8}Let them make
for me a sanctuary,
so that I may live
among them.
\VS{9}According to all
that
I am showing you – the pattern of the tabernacle and the pattern of all its furnishings – you must make it exactly so.
\par }{\SH The Ark of the Covenant
\par }{\PP \VS{10}“They are to make
an ark
of acacia
wood
– its length
is to be three feet nine inches,
its width
two feet three inches,
and its height
two feet three inches.
\VS{11}You are to overlay
it with pure
gold
– both inside and outside
you must
overlay
it, and you are to make
a surrounding
border
of gold
over
it.
\VS{12}You are to cast
four
gold
rings
for it and put
them on
its four
feet,
with two
rings
on
one
side
and two
rings
on
the other side.
\VS{13}You are to make
poles
of acacia
wood,
overlay
them with gold,
\VS{14}and put
the poles
into the rings
at
the sides
of the ark
in order
to carry
the ark with them.
\VS{15}The poles
must remain in the rings
of the ark;
they must not
be removed
from it.
\VS{16}You are to put
into
the ark
the testimony
that
I will give
to you.
\par }{\PP \VS{17}“You are to make
an atonement lid
of pure
gold;
its length
is to be three feet nine inches,
and its width
is to be two feet three inches.
\VS{18}You are to make
two
cherubim
of gold;
you are to make them of hammered
metal
on the
two
ends
of the atonement lid.
\VS{19}Make
one
cherub
on one end
and one
cherub on
the other end; from
the atonement lid
you are to make
the cherubim
on
the two
ends.
\VS{20}The cherubim
are to be
spreading
their wings
upward,
overshadowing
the atonement lid
with their wings,
and the cherubim
are to face
each
other,
looking
toward
the atonement lid.
\VS{21}You are to put
the
atonement lid
on
top
of the ark,
and in the ark
you are to put
the
testimony
I am giving you.
\VS{22}I will meet
with you there,
and from above
the atonement lid,
from between
the two
cherubim
that
are over
the ark
of the testimony,
I will speak
with you about all
that
I will command
you for the Israelites.
\par }{\SH The Table for the Bread of the Presence
\par }{\PP \VS{23}“You are to make
a table
of acacia
wood;
its length
is to be three feet,
its width
one foot six inches,
and its height
two feet three inches.
\VS{24}You are to overlay
it with pure
gold,
and you are
to make
a surrounding
border
of gold
for it.
\VS{25}You are to make
a surrounding
frame
for it about three inches broad,
and you are
to make a surrounding
border
of gold
for its frame.
\VS{26}You are to make
four
rings
of gold
for it and attach
the rings
at the four
corners
where
its four
legs are.
\VS{27}The
rings
are to be
close
to the frame
to provide
places for the poles
to carry
the table.
\VS{28}You are to make
the poles
of acacia
wood
and overlay
them with gold,
so that the table
may be carried with them.
\VS{29}You are to make
its plates,
its ladles,
its pitchers,
and its bowls,
to be used in pouring out
offerings; you are to make
them
of pure
gold.
\VS{30}You are to set
the Bread
of the Presence
on
the table
before
me continually.
\par }{\SH The Lampstand
\par }{\PP \VS{31}“You are to make
a lampstand
of pure
gold.
The lampstand
is to be made of hammered
metal;
its base and its shaft,
its cups,
its buds,
and its blossoms
are to be
from
the same piece.
\VS{32}Six
branches
are to extend from
the sides
of the lampstand, three
branches
of the lampstand
from one
side
of it and three
branches
of the lampstand
from the other side of it.
\VS{33}Three
cups
shaped like almond
flowers
with buds
and blossoms
are to be on one
branch, and three
cups
shaped like almond
flowers
with buds
and blossoms
are to be on the next branch, and the same
for the six
branches
extending
from
the lampstand.
\VS{34}On the lampstand
there are to be four
cups
shaped like almond
flowers with buds
and blossoms,
\VS{35}with a bud
under
the first two
branches
from
it, and a bud
under
the next two
branches
from
it, and a bud
under
the third two
branches
from
it, according
to the six
branches
that extend
from
the lampstand.
\VS{36}Their buds
and their branches
will be
one piece,
all
of it one
hammered
piece of pure
gold.
\par }{\PP \VS{37}“You are to make
its seven
lamps,
and then set
its lamps
up on
it, so that it will give light
to the area
in front of it.
\VS{38}Its trimmers
and its trays
are to be of pure
gold.
\VS{39}About seventy-five pounds
of pure
gold
is to be used
for it
and for all
these
utensils.
\VS{40}Now be sure
to make
them according to the pattern
you
were shown
on the mountain.


\par }\Chap{26}{\PP \VerseOne{1}“The tabernacle
itself you are to make
with ten
curtains
of fine
twisted
linen
and blue
and purple
and scarlet;
you are to make
them with cherubim
that are the work
of an artistic designer.
\VS{2}The length
of each
curtain
is to be forty-two feet,
and the width
of each curtain is
to be six feet –
the same
size
for each
of the curtains.
\VS{3}Five
curtains
are to be
joined,
one to another,
and the other five
curtains
are to
be joined,
one to another.
\VS{4}You are to make
loops
of blue
material along
the edge
of the end curtain
in one
set,
and in the same way
you are to make
loops in the outer
edge
of the end
curtain
in the second
set.
\VS{5}You are to make fifty
loops
on the one
curtain,
and you are to make fifty
loops
on the end
curtain
which
is on the second
set,
so that the loops
are opposite one to another.
\VS{6}You are to make
fifty
gold
clasps
and join
the curtains
together
with the clasps,
so that the tabernacle
is a unit.
\par }{\PP \VS{7}“You are to make
curtains
of goats’
hair for a tent
over
the tabernacle;
you are to make
eleven
curtains.
\VS{8}The length
of each
curtain
is to be forty-five feet,
and the width
of each curtain is
to be six feet
– the same
size
for the eleven
curtains.
\VS{9}You are to join
five
curtains
by themselves
and six
curtains
by themselves.
You are to double
over the
sixth
curtain
at
the front
of the tent.
\VS{10}You are to make
fifty
loops
along
the edge
of the end
curtain
in one
set
and fifty
loops
along
the edge
of the curtain
that joins
the second set.
\VS{11}You are to make
fifty
bronze
clasps
and put
the clasps
into the loops
and join
the tent
together so that it is a unit.
\VS{12}Now the part
that remains
of the curtains
of the tent
– the half
curtain
that remains
will hang
over
at the back
of the tabernacle.
\VS{13}The foot and a half
on the one side and the foot and a half
on the other
side of what remains
in the length
of the curtains
of the tent
will
hang
over
the sides
of the tabernacle,
on one side and the other side, to cover it.
\par }{\PP \VS{14}“You are to make
a covering
for the tent
out of ram
skins
dyed red
and over
that a covering
of fine
leather.
\par }{\PP \VS{15}“You are to make
the frames
for the tabernacle
out of acacia
wood
as uprights.
\VS{16}Each frame
is to be fifteen feet
long,
and each
frame
is to be two feet three inches
wide,
\VS{17}with two
projections
per frame
parallel
one
to
another.
You are to make
all
the frames
of the tabernacle
in this way.
\VS{18}So you are to make
the frames
for the tabernacle: twenty
frames
for the south
side,
\VS{19}and you are to make
forty
silver
bases
to go under
the twenty
frames
– two
bases
under
the first
frame
for its two
projections,
and likewise two
bases
under
the next
frame
for its two
projections;
\VS{20}and for the second
side
of the tabernacle,
the north
side,
twenty
frames,
\VS{21}and their forty
silver
bases,
two
bases
under
the first frame,
and two
bases
under
the next frame.
\VS{22}And for the back
of the tabernacle
on the west
you will make
six
frames.
\VS{23}You are
to make two
frames
for the corners
of the tabernacle
on the back.
\VS{24}At the two
corners
they must be
doubled
at the lower
end and finished together
at the top
in one
ring.
So
it will be
for both.
\VS{25}So there are to be
eight
frames
and their silver
bases,
sixteen
bases,
two
bases
under
the first
frame,
and two
bases
under
the next
frame.
\par }{\PP \VS{26}“You are to make
bars
of acacia
wood,
five
for the frames
on one
side
of the tabernacle,
\VS{27}and five
bars
for the frames
on the second
side
of the tabernacle,
and five
bars
for the frames
on the back
of the tabernacle
on the west.
\VS{28}The middle
bar
in the center
of the frames
will reach
from
end
to
end.
\VS{29}You are to overlay
the
frames
with gold
and make
their rings
of gold
to provide
places for the bars,
and you are to overlay
the
bars
with gold.
\VS{30}You are to set
up the
tabernacle
according
to the plan that
you were shown
on the mountain.
\par }{\PP \VS{31}“You are to make
a special curtain
of blue,
purple,
and scarlet
yarn
and fine
twisted
linen;
it is to be made
with cherubim,
the work
of an artistic designer.
\VS{32}You are to hang
it with gold
hooks
on
four
posts
of acacia
wood overlaid
with gold,
set in
four
silver
bases.
\VS{33}You are to hang
this curtain
under
the clasps
and bring
the ark
of the testimony
in there
behind the curtain.
The curtain
will make a division
for you between
the Holy Place
and the Most Holy Place.
\VS{34}You are to put
the atonement lid
on
the ark
of the testimony
in the Most
Holy Place.
\VS{35}You are to put
the
table
outside
the curtain
and the
lampstand
on
the south
side
of the tabernacle,
opposite
the table,
and you are to place
the table
on
the north
side.
\par }{\PP \VS{36}“You are to make
a hanging
for the entrance
of the tent
of blue,
purple,
and scarlet
yarn
and fine
twined
linen,
the work
of an embroiderer.
\VS{37}You are to make
for the hanging
five
posts
of acacia wood
and overlay
them with gold,
and their hooks
will be gold,
and you are to cast
five
bronze
bases for them.


\par }\Chap{27}{\PP \VerseOne{1}“You are to make
the altar
of acacia
wood,
seven feet six inches
long,
and seven feet six inches
wide;
the altar
is to be
square,
and its height
is to be four
feet
six inches.
\VS{2}You are to make
its four horns
on
its four
corners;
its horns
will be
part of it, and you are to overlay
it with bronze.
\VS{3}You are to make
its pots
for the ashes,
its shovels,
its tossing bowls,
its meat hooks, and its fire pans
– you are to make
all
its utensils
of bronze.
\VS{4}You are to make
a grating
for it, a network
of bronze,
and you are to make
on
the network
four
bronze
rings
on
its four
corners.
\VS{5}You are to put
it under
the ledge
of the altar
below,
so that the network
will come
halfway
up the altar.
\VS{6}You are to make
poles
for the altar,
poles
of acacia
wood,
and you are to overlay
them with bronze.
\VS{7}The poles
are to be put
into the rings
so that the poles
will be
on
two
sides
of the altar
when carrying it.
\VS{8}You are to make the altar hollow,
out of boards.
Just as
it was shown
you on the mountain,
so
they must make it.
\par }{\SH The Courtyard
\par }{\PP \VS{9}“You are to make
the courtyard
of the tabernacle.
For the south
side
there are to be hangings
for the courtyard
of fine
twisted
linen,
one hundred fifty feet
long
for one
side,
\VS{10}with twenty
posts
and their twenty
bronze
bases,
with the hooks
of the posts
and their bands
of silver.
\VS{11}Likewise
for its length
on the north
side,
there are to be hangings
for one hundred
fifty feet,
with twenty
posts
and their twenty
bronze
bases,
with silver
hooks
and bands
on the posts.
\VS{12}The width
of the court
on the west
side
is to be seventy-five feet
with hangings,
with their ten
posts
and their ten
bases.
\VS{13}The width
of the court
on
the east
side,
toward the sunrise,
is to be seventy-five feet.
\VS{14}The hangings
on one side
of the gate are to be twenty-two and a half feet
long, with their three
posts
and their three
bases.
\VS{15}On the second
side
there are to be hangings
twenty-two and a half
feet long, with their three
posts
and their three
bases.
\VS{16}For the gate
of the courtyard
there is to be a curtain
of thirty feet,
of blue,
purple,
and scarlet
yarn
and fine
twined
linen,
the work
of an embroiderer,
with four
posts
and their four
bases.
\VS{17}All
the posts
around
the courtyard
are to have
silver
bands;
their hooks
are to be silver,
and their bases
bronze.
\VS{18}The length
of the courtyard
is to be one hundred fifty feet
and the width
seventy-five
feet, and the height
of the fine
twisted
linen
hangings is to be seven and a half feet,
with their bronze
bases.
\VS{19}All
the utensils
of the tabernacle
used in all
its service,
all
its tent pegs,
and all
the tent pegs
of the courtyard
are to be made of bronze.
\par }{\SH Offering the Oil
\par }{\PP \VS{20}“You
are to command
the
Israelites
that they bring
to
you pure
oil
of pressed
olives
for the light,
so that the lamps
will burn
regularly.
\VS{21}In the tent
of meeting
outside
the curtain
that
is before
the testimony,
Aaron
and his sons
are to arrange
it from evening
to
morning
before
the {\ND{Lord}}. This is to be a lasting
ordinance
among the Israelites
for generations to come.

\par }\Chap{28}{\PP \VerseOne{1}“And you,
bring near
to
you your brother
Aaron
and his sons
with
him from among
the Israelites,
so that they may minister as my priests – Aaron, Nadab and Abihu, Eleazar and Ithamar, Aaron’s sons.
\VS{2}You must make
holy
garments
for your brother
Aaron,
for glory
and for beauty.
\VS{3}You
are to speak
to
all
who are specially
skilled,
whom
I have filled
with the spirit
of wisdom,
so that they may make
Aaron’s
garments
to set him apart
to minister as my priest.
\VS{4}Now these
are the garments
that
they are to make: a breastpiece,
an ephod,
a robe,
a fitted
tunic,
a turban,
and a sash.
They are to make
holy
garments
for your brother
Aaron
and for his sons,
that they may minister as my priests.
\VS{5}The artisans
are to use
the gold,
blue,
purple,
scarlet,
and fine linen.
\par }{\PP \VS{6}“They are to make
the ephod
of gold,
blue,
purple,
scarlet,
and fine
twisted
linen,
the work
of an artistic designer.
\VS{7}It is to have two
shoulder pieces
attached
to
two
of its corners,
so it can be joined together.
\VS{8}The artistically woven waistband
of the ephod
that
is on
it is to be
like it, of one piece with the ephod, of gold,
blue,
purple,
scarlet,
and fine
twisted
linen.
\par }{\PP \VS{9}“You are to take
two
onyx
stones
and engrave
on
them the names
of the sons
of Israel,
\VS{10}six
of their names
on
one
stone,
and the six
remaining
names
on
the second
stone,
according to the order of their birth.
\VS{11}You are to engrave
the
two
stones
with the names
of the sons
of Israel
with the work
of an engraver
in stone,
like the engravings
of a seal;
you are to have them
set
in gold
filigree
settings.
\VS{12}You are to put
the
two
stones
on
the shoulders
of the ephod,
stones
of memorial
for the sons
of Israel,
and Aaron
will bear
their names
before
the {\ND{Lord}}
on
his two
shoulders
for a memorial.
\VS{13}You are to make
filigree settings
of gold
\VS{14}and two
braided
chains
of pure
gold,
like a cord,
and attach
the chains
to the settings.
\par }{\PP \VS{15}“You are to make
a breastpiece
for use in making decisions,
the work
of an artistic
designer; you
are to make
it in the same fashion as the ephod;
you are to make
it of gold,
blue,
purple,
scarlet,
and fine
twisted
linen.
\VS{16}It is to be square
when doubled,
nine
inches long
and nine
inches wide.
\VS{17}You are to set
in it a setting
for stones,
four
rows
of stones,
a row
with a ruby,
a topaz,
and a beryl
– the first
row;
\VS{18}and the second
row,
a turquoise,
a sapphire,
and an emerald;
\VS{19}and the third
row,
a jacinth,
an agate,
and an amethyst;
\VS{20}and the fourth
row,
a chrysolite,
an onyx,
and a jasper.
They are to be
enclosed
in gold
in their filigree settings.
\VS{21}The stones
are to be
for
the names
of the sons
of Israel,
twelve,
according
to the number of their names.
Each
name
according
to the twelve
tribes
is to be
like the engravings
of a seal.
\par }{\PP \VS{22}“You are to make
for
the breastpiece
braided
chains
like cords
of pure
gold,
\VS{23}and you are to make
for the breastpiece
two
gold
rings
and attach
the
two
rings
to the upper two
ends
of the breastpiece.
\VS{24}You
are to attach
the two
gold
chains
to the two
rings
at the ends
of the breastpiece;
\VS{25}the
other two
ends
of the two
chains
you will attach
to the two
settings
and then attach them
to the shoulder pieces
of the ephod
at the front of it.
\VS{26}You are to make
two
rings
of gold
and put
them
on
the other two
ends
of the breastpiece,
on
its edge
that
is on the inner side
of the ephod.
\VS{27}You are to make
two
more gold
rings
and attach them
to the
bottom
of the two
shoulder pieces
on the front
of the
ephod,
close
to the juncture
above
the waistband
of the ephod.
\VS{28}They are to tie
the breastpiece
by its rings
to
the rings
of the ephod
by blue
cord,
so that it may be
above
the waistband
of the ephod,
and so that the breastpiece
will not
be loose
from
the ephod.
\VS{29}Aaron
will bear
the names
of the sons
of Israel
in the breastpiece
of decision
over
his heart
when he goes
into
the holy
place, for a memorial
before
the {\ND{Lord}}
continually.
\par }{\PP \VS{30}“You are to
put
the Urim
and the Thummim
into the breastpiece
of decision;
and they are to be
over
Aaron’s
heart
when he goes
in before
the {\ND{Lord}}. Aaron
is to bear
the
decisions
of the Israelites
over
his heart
before
the {\ND{Lord}}
continually.
\par }{\PP \VS{31}“You are to make
the robe
of the ephod
completely
blue.
\VS{32}There is to be
an opening
in its top
in the center
of it, with an edge
all around
the opening,
the work
of a weaver,
like the opening
of a collar,
so that it cannot
be torn.
\VS{33}You are to make
pomegranates
of blue,
purple,
and scarlet
all around
its hem
and bells
of gold
between
them all around.
\VS{34}The pattern is to be a gold
bell
and a pomegranate,
a gold
bell
and a pomegranate,
all around
the hem
of the robe.
\VS{35}The robe is to be
on
Aaron
as he ministers,
and his sound
will be heard
when
he enters
the Holy
Place before
the {\ND{Lord}}
and when
he leaves, so
that he does not
die.
\par }{\PP \VS{36}“You are to make
a plate
of pure
gold
and engrave
on it the way a seal
is engraved: “Holiness
to the
{\ND{Lord}}.”
\VS{37}You are to attach
to it a blue
cord
so that it will be
on
the turban;
it is to
be
on
the front of
the turban,
\VS{38}It will be
on
Aaron’s
forehead,
and Aaron
will bear
the iniquity
of the holy
things, which
the Israelites
are to sanctify
by all
their holy
gifts;
it will always be
on
his forehead,
for their acceptance
before
the {\ND{Lord}}.
\VS{39}You are to weave
the tunic
of fine linen
and make
the turban
of fine linen,
and make
the sash
the work
of an embroiderer.
\par }{\PP \VS{40}“For Aaron’s
sons
you are to make
tunics,
sashes,
and headbands
for glory
and for beauty.
\par }{\PP \VS{41}“You are to clothe
them – your brother Aaron and his sons with him – and anoint them and ordain them and set them apart as holy, so that they may minister as my priests.
\VS{42}Make
for them linen
undergarments
to cover
their naked
bodies;
they must cover
from the waist
to
the thighs.
\VS{43}These must be
on
Aaron
and his sons
when they enter
to
the tent
of meeting,
or
when they approach
the altar
to minister
in the Holy
Place, so that they bear
no
iniquity
and die.
It is to be a perpetual
ordinance
for him and for his descendants
after him.

\par }\Chap{29}{\PP \VerseOne{1}“Now this
is what
you are to do
for them to consecrate
them so that they may minister
as my priests. Take
a
young
bull
and two
rams
without blemish;
\VS{2}and bread
made without yeast,
and perforated
cakes without yeast
mixed
with oil,
and wafers
without yeast
spread
with oil
– you are to make
them using
fine
wheat
flour.
\VS{3}You are to put
them in one
basket
and present
them in the basket,
along with
the bull
and the two
rams.
\par }{\PP \VS{4}“You are to
present
Aaron
and his sons
at the entrance
of the tent
of meeting.
You are to wash
them
with water
\VS{5}and take
the garments
and clothe
Aaron
with the tunic,
the robe
of the ephod,
the ephod,
and the breastpiece;
you are to fasten
the ephod
on him by using the skillfully woven waistband.
\VS{6}You are to put
the turban
on
his head
and put
the holy
diadem on
the turban.
\VS{7}You are to take
the anointing
oil
and pour
it on
his head
and anoint him.
\VS{8}You are to present
his sons
and clothe
them with tunics
\VS{9}and wrap
the sashes
around Aaron
and his sons
and put headbands
on them,
and so the ministry
of priesthood
will
belong to them
by
a perpetual
ordinance.
Thus you are to consecrate
Aaron
and his
sons.
\par }{\PP \VS{10}“You are to present
the
bull
at the front
of the tent
of meeting,
and Aaron
and his sons
are to put their hands
on
the head
of the bull.
\VS{11}You are to kill
the bull
before
the {\ND{Lord}}
at the entrance
to the tent
of meeting
\VS{12}and take
some of the blood
of the bull
and put
it on
the horns
of the altar
with your finger;
all
the rest of the blood
you are to
pour
out at the base
of the altar.
\VS{13}You are to take
all
the fat
that covers
the
entrails,
and the
lobe
that is above
the liver,
and the
two
kidneys
and the
fat
that
is on
them, and burn
them on the altar.
\VS{14}But the
meat
of the bull,
its skin,
and its
dung
you are to burn up
outside
the camp.
It is
the purification offering.
\par }{\PP \VS{15}“You are to take
one
ram,
and Aaron
and his sons
are to lay
their hands
on
the ram’s
head,
\VS{16}and you are to kill
the ram
and take
its blood
and splash
it all around
on
the altar.
\VS{17}Then
you are to cut
the ram
into pieces
and wash
the entrails
and its legs
and put
them on
its pieces
and on
its head
\VS{18}and burn
the whole
ram
on the altar.
It is
a burnt offering
to the
{\ND{Lord}}, a soothing
aroma;
it is
an offering made by fire
to the
{\ND{Lord}}.
\par }{\PP \VS{19}“You are to take
the second
ram,
and Aaron
and his sons
are to lay
their hands
on
the ram’s
head,
\VS{20}and you are to kill
the ram
and take
some of its blood
and put
it on
the tip
of the right ear
of Aaron,
on
the tip
of the right
ear
of his sons,
on
the thumb
of their right
hand,
and on
the big toe
of their right
foot,
and then splash
the blood
all around
on
the altar.
\VS{21}You are to take
some
of the blood
that
is on
the altar
and some of the anointing
oil
and sprinkle
it on
Aaron,
on
his garments,
on
his sons,
and on
his sons’
garments
with
him, so that
he may be holy,
he and his garments
along with
his sons
and his sons’
garments.
\par }{\PP \VS{22}“You are to take
from
the ram
the fat,
the fat tail, the fat that covers the entrails, the lobe of the liver, the two kidneys and the fat that is on them, and the right thigh – for it is the ram for consecration –
\VS{23}and one
round
flat cake of
bread,
one
perforated
cake of oiled
bread,
and one
wafer
from the basket
of bread made without yeast
that
is before
the {\ND{Lord}}.
\VS{24}You are to put
all
these in Aaron’s
hands
and in his sons’
hands,
and you are to wave
them as a wave offering
before
the {\ND{Lord}}.
\VS{25}Then you are to take
them
from their hands
and burn
them on the altar
for a burnt offering,
for a soothing
aroma
before
the {\ND{Lord}}. It is
an offering made by fire
to the
{\ND{Lord}}.
\VS{26}You are to take
the
breast
of the ram
of Aaron’s
consecration;
you are to wave
it as a wave offering
before
the {\ND{Lord}}, and it is to be
your share.
\VS{27}You are to sanctify
the
breast
of the wave offering
and the thigh
of the contribution,
which
were waved
and lifted up
as a contribution from the ram
of consecration,
from what belongs
to Aaron
and to his sons.
\VS{28}It is to belong to Aaron
and to his sons
from the Israelites,
by a perpetual
ordinance,
for
it is a contribution.
It is to be
a contribution
from the
Israelites
from their peace offerings,
their contribution
to the
{\ND{Lord}}.
\par }{\PP \VS{29}“The holy
garments
that belong
to Aaron
are to belong to his sons
after
him, so that they may be anointed
in them and consecrated in them.
\VS{30}The priest
who succeeds
him from his sons,
when he first comes
to
the tent
of meeting
to minister
in the Holy
Place, is to wear
them for seven
days.
\par }{\PP \VS{31}“You are to take
the
ram
of the consecration
and cook
its meat
in a holy
place.
\VS{32}Aaron
and his sons
are to eat
the
meat
of the ram
and the
bread
that
was in the basket
at the entrance
of the tent
of meeting.
\VS{33}They are to eat
those things by which
atonement
was made to consecrate
and to set them apart,
but no
one else
may eat
them, for
they
are holy.
\VS{34}If
any
of the meat
from the consecration
offerings or any
of the bread
is left over until
morning,
then
you are to burn up
what is left
over. It must not
be eaten,
because
it is
holy.
\par }{\PP \VS{35}“Thus you are to do
for Aaron
and for his sons,
according
to all
that
I have
commanded
you; you are to consecrate
them
for seven
days.
\VS{36}Every day
you are to prepare
a bull
for a purification
offering for atonement.
You are
to purge
the altar
by
making atonement
for it, and you are to anoint
it to set it apart as holy.
\VS{37}For seven
days
you are to make atonement
for the altar
and set it apart
as holy. Then
the altar
will be
most
holy.
Anything
that touches
the altar
will be holy.
\par }{\PP \VS{38}“Now this
is what
you are to prepare
on
the altar
every day
continually: two
lambs
a year
old.
\VS{39}The first
lamb
you are
to prepare
in the morning,
and the second
lamb
you are
to prepare
around sundown.
\VS{40}With the first
lamb
offer a tenth of an ephah
of fine flour
mixed
with a fourth
of a hin
of oil
from pressed olives,
and a fourth
of a hin
of wine
as a drink offering.
\VS{41}The
second
lamb
you are
to offer
around sundown;
you are to prepare for it the same meal offering
as for the morning
and the same drink offering,
for a soothing
aroma,
an offering made by fire
to the
{\ND{Lord}}.
\par }{\PP \VS{42}“This will be a regular
burnt offering
throughout your generations
at the entrance
of the tent
of meeting
before
the {\ND{Lord}}, where
I will meet
with you to speak
to
you there.
\VS{43}There
I will meet
with the Israelites,
and it will be set apart
as holy by my glory.
\par }{\PP \VS{44}“So I will set apart
as holy the
tent
of meeting
and the
altar,
and I will set apart
as holy Aaron
and his sons,
that they may minister as priests to me.
\VS{45}I will reside
among
the Israelites,
and I will be
their God,
\VS{46}and they will know
that
I am
the {\ND{Lord}}
their God,
who
brought them out
from the
land
of Egypt,
so that I may reside
among
them. I am
the {\ND{Lord}}
their God.


\par }\Chap{30}{\PP \VerseOne{1}“You are to make
an altar
for burning
incense;
you are to make
it of acacia
wood.
\VS{2}Its length
is to be a foot and a half and its width
a foot and a half;
it will be square.
Its height
is to be three feet,
with its horns
of one piece with it.
\VS{3}You are to overlay
it with pure
gold
– its
top,
its four walls,
and its horns
– and make
a surrounding
border
of gold
for it.
\VS{4}You are to make two
gold
rings
for it under
its border,
on
its two
flanks; you are to make
them on
its two
sides.
The rings will be
places
for poles
to carry it with.
\VS{5}You are to make
the poles
of acacia
wood
and overlay
them with gold.
\par }{\PP \VS{6}“You are to put
it in front
of the curtain
that
is before
the ark
of the testimony
(before
the atonement lid
that
is over
the testimony), where
I will meet you.
\VS{7}Aaron
is to burn
sweet
incense
on
it morning
by morning;
when he attends
to the lamps
he is to burn incense.
\VS{8}When Aaron
sets up
the lamps
around
sundown
he is to burn incense
on it; it is to be a regular
incense
offering before
the {\ND{Lord}}
throughout your generations.
\VS{9}You must not
offer
strange
incense
on it, nor
burnt offering,
nor
meal offering, and you must not
pour
out a drink offering
on it.
\VS{10}Aaron
is to make atonement
on
its horns
once
in the year
with some of the blood
of the sin offering
for atonement;
once
in the year
he is to make atonement
on
it throughout your generations.
It is
most
holy
to the
{\ND{Lord}}.”
\par }{\SH The Ransom Money
\par }{\PP \VS{11}The
{\ND{Lord}}
spoke
to
Moses:
\VS{12}“When
you take
a census
of the
Israelites
according to their number,
then
each man
is to pay a ransom
for his life
to the
{\ND{Lord}}
when
you number them,
so that there will be no
plague
among them when you number them.
\VS{13}Everyone
who crosses
over
to those who are numbered
is to pay
this: a half
shekel
according to the shekel
of the sanctuary
(a shekel
weighs twenty
gerahs). The half
shekel
is to be an offering
to the
{\ND{Lord}}.
\VS{14}Everyone
who crosses
over
to those numbered,
from twenty
years
old and up, is to pay
an offering
to the
{\ND{Lord}}.
\VS{15}The rich
are not
to increase
it, and the poor
are not
to pay less
than the half
shekel
when giving
the offering
of the {\ND{Lord}}, to make atonement
for
your lives.
\VS{16}You are to receive
the
atonement
money
from the
Israelites
and give
it for the service
of the tent
of meeting.
It will be
a memorial
for the Israelites
before
the {\ND{Lord}}, to make atonement
for your lives.”
\par }{\SH The Bronze Laver
\par }{\PP \VS{17}The
{\ND{Lord}}
spoke
to
Moses:
\VS{18}“You are also to make
a large
bronze
basin
with a bronze
stand
for washing.
You are to put
it between
the tent
of meeting
and the altar
and put
water
in it,
\VS{19}and Aaron
and his sons
must wash
their hands
and their feet
from it.
\VS{20}When they enter
the tent
of
meeting,
they must wash
with water
so that they do not
die.
Also,
when they approach
the altar
to
minister
by burning incense
as an offering made by fire
to the
{\ND{Lord}},
\VS{21}they must wash
their hands
and their feet
so that they do not
die.
And this will be
a perpetual
ordinance
for them and for their descendants
throughout their generations.”
\par }{\SH Oil and Incense
\par }{\PP \VS{22}The
{\ND{Lord}}
spoke
to
Moses:
\VS{23}“Take
choice
spices: twelve and a half pounds
of free-flowing
myrrh,
half that – about six and a quarter pounds – of sweet-smelling cinnamon, six and a quarter pounds of sweet-smelling cane,
\VS{24}and twelve and a half pounds
of cassia,
all weighed according to the sanctuary
shekel,
and four quarts
of olive
oil.
\VS{25}You are to make
this into a sacred
anointing
oil,
a perfumed
compound,
the work
of a perfumer.
It will be
sacred
anointing
oil.
\par }{\PP \VS{26}“With it you are to anoint
the tent
of meeting,
the ark
of the testimony,
\VS{27}the table
and all
its utensils,
the lampstand
and its
utensils,
the altar
of incense,
\VS{28}the altar
for the burnt offering
and all
its utensils,
and the laver
and its
base.
\VS{29}So you are to sanctify
them, and they will be
most
holy;
anything
that touches
them will be holy.
\par }{\PP \VS{30}“You are to anoint
Aaron
and his sons
and sanctify
them, so that they may minister as my priests.
\VS{31}And you are to
tell
the Israelites: ‘This
is to be my sacred
anointing
oil
throughout your generations.
\VS{32}It must not
be applied to people’s
bodies,
and you must not
make
any like
it with the same recipe.
It is
holy,
and it must be
holy to you.
\VS{33}Whoever
makes perfume
like
it and whoever
puts
any
of it on
someone not
a priest will be cut off
from his people.’ ”
\par }{\PP \VS{34}The
{\ND{Lord}}
said
to
Moses: “Take
spices,
gum
resin, onycha,
galbanum,
and pure
frankincense
of equal amounts
\VS{35}and make
it into an incense,
a perfume,
the work
of a perfumer.
It is to be finely
ground, and pure
and sacred.
\VS{36}You are to beat
some
of it very fine and put
some
of it before
the ark of the testimony
in the tent
of meeting
where
I will meet
with you; it is to be
most
holy to you.
\VS{37}And the incense
that
you are
to make,
you must not
make
for yourselves using the same recipe;
it is to be
most holy
to you, belonging to the
{\ND{Lord}}.
\VS{38}Whoever
makes
anything like
it, to use as perfume,
will be cut off
from his people.”


\par }\Chap{31}{\PP \VerseOne{1}The
{\ND{Lord}}
spoke
to
Moses:
\VS{2}“See,
I have chosen
Bezalel
son
of Uri,
the son
of Hur,
of the tribe
of Judah,
\VS{3}and I have filled
him with the Spirit
of God
in skill,
in understanding,
in knowledge,
and in all
kinds of craftsmanship,
\VS{4}to make
artistic
designs
for work
with gold,
with silver,
and with bronze,
\VS{5}and with cutting
and setting
stone,
and with
cutting
wood,
to work
in all
kinds of craftsmanship.
\VS{6}Moreover,
I
have also given
him
Oholiab
son
of Ahisamach,
of the tribe
of Dan,
and I have given ability to all
the specially
skilled,
that they may
make
everything
I
have
commanded you:
\VS{7}the tent
of meeting,
the ark
of the testimony,
the atonement lid
that
is on
it, all
the furnishings
of the tent,
\VS{8}the table
with its
utensils,
the pure
lampstand
with all
its utensils,
the altar
of incense,
\VS{9}the altar
for the burnt offering
with all
its utensils,
the large basin
with its
base,
\VS{10}the woven
garments,
the holy
garments
for Aaron
the priest
and the garments
for his sons,
to minister as priests,
\VS{11}the anointing
oil,
and sweet
incense
for the Holy
Place. They will make
all
these things just
as I have commanded you.”
\par }{\SH Sabbath Observance
\par }{\PP \VS{12}The
{\ND{Lord}}
said
to
Moses,
\VS{13}“Tell
the Israelites,
‘Surely
you must keep
my Sabbaths,
for
it is
a sign
between
me and you throughout your generations,
that you may know
that
I am
the {\ND{Lord}}
who sanctifies
you.
\VS{14}So you must keep
the Sabbath,
for
it is
holy
for you. Everyone
who defiles
it must surely be put
to death;
indeed,
if anyone
does
any work
on it, then that person will be cut off
from among
his people.
\VS{15}Six
days
work
may be done,
but on the seventh
day
is a Sabbath
of complete rest,
holy
to the
{\ND{Lord}}; anyone
who does
work
on the Sabbath
day
must surely be put
to death.
\VS{16}The Israelites
must keep
the Sabbath
by observing
the Sabbath
throughout their generations
as a perpetual
covenant.
\VS{17}It is
a sign
between
me and the Israelites
forever;
for
in six
days
the {\ND{Lord}}
made
the
heavens
and the
earth,
and on
the seventh
day
he rested
and was refreshed.’ ”
\par }{\PP \VS{18}He gave
Moses
two
tablets
of testimony
when he had finished
speaking
with
him on Mount
Sinai,
tablets
of stone
written
by the finger
of God.


\par }\Chap{32}{\PP \VerseOne{1}When the people
saw
that
Moses
delayed
in coming down
from
the mountain,
they gathered around
Aaron
and said
to
him, “Get up,
make
us gods
that
will go
before
us. As for
this
fellow Moses,
the man
who
brought
us up
from the land
of Egypt,
we do not
know
what
has become of him!”
\par }{\PP \VS{2}So Aaron
said
to them,
“Break off
the gold
earrings
that
are on the ears
of your wives,
your sons,
and your daughters,
and bring
them to me.”
\VS{3}So all
the people
broke off
the gold
earrings
that
were on their ears
and brought
them to
Aaron.
\VS{4}He accepted
the gold from them,
fashioned
it with an engraving tool,
and made
a molten
calf.
Then they said,
“These
are your gods,
O Israel,
who
brought
you up
out of Egypt.”
\par }{\PP \VS{5}When Aaron
saw
this, he built
an altar
before
it, and Aaron
made a proclamation
and said,
“Tomorrow
will be a feast
to the
{\ND{Lord}}.”
\VS{6}So they got up early
on the next
day and offered
up burnt offerings
and brought
peace offerings,
and the people
sat
down to eat
and drink,
and they rose
up to play.
\par }{\PP \VS{7}The
{\ND{Lord}}
spoke
to
Moses: “Go
quickly, descend,
because
your people,
whom
you brought up
from the land
of Egypt,
have acted corruptly.
\VS{8}They
have quickly
turned aside
from
the way
that
I commanded them – they have made for themselves a molten calf and have bowed down to it and sacrificed to it and said, ‘These are your gods, O Israel, which brought you up from the land of Egypt.’ ”
\par }{\PP \VS{9}Then the
{\ND{Lord}}
said to
Moses: “I have seen
this
people.
Look
what a stiff-necked
people they are!
\VS{10}So now,
leave
me alone so that my anger
can burn
against them and I can destroy
them, and I will make
from you a great
nation.”
\par }{\PP \VS{11}But Moses
sought
the
favor of the
{\ND{Lord}}
his God
and said,
“O
{\ND{Lord}}, why
does your anger
burn
against your people,
whom
you have brought out
from the land
of Egypt
with great
power
and with a mighty
hand?
\VS{12}Why
should the Egyptians
say, ‘For evil
he led them out
to kill
them in the
mountains
and to destroy
them from the face
of the earth’? Turn
from your burning
anger,
and relent
of this evil
against
your people.
\VS{13}Remember
Abraham,
Isaac,
and Israel
your servants,
to whom
you swore
by yourself and told
them,
‘I will multiply
your descendants
like the stars
of heaven,
and all
this
land
that
I have
spoken about I will give
to your descendants,
and they will inherit
it forever.’ ”
\VS{14}Then the
{\ND{Lord}}
relented
over
the evil
that he had
said
he would do
to his people.
\par }{\PP \VS{15}Moses
turned
and went down
from
the mountain
with the two
tablets
of the testimony
in his hands.
The tablets
were written
on
both
sides
– they
were written on the front and on the back.
\VS{16}Now the tablets
were the work
of God,
and the writing
was the writing
of God,
engraved
on
the tablets.
\VS{17}When Joshua
heard
the
noise
of the people
as they shouted,
he said
to
Moses,
“It is the sound
of war
in the camp!”
\VS{18}Moses said,
“It is not
the sound
of those who shout
for victory,
nor
is it the sound
of those who cry
because they are overcome,
but the sound
of singing
I
hear.”
\par }{\PP \VS{19}When
he approached
the camp
and saw
the
calf
and the dancing,
Moses
became extremely angry.
He threw
the
tablets
from his hands
and broke
them
to pieces at the bottom
of the mountain.
\VS{20}He took
the
calf
they had
made
and burned
it in the fire,
ground it to powder,
poured it out
on
the water,
and made the
Israelites
drink it.
\par }{\PP \VS{21}Moses
said
to
Aaron,
“What
did
this
people
do
to you, that
you have brought
on
them so great
a sin?”
\VS{22}Aaron
said,
“Do not
let your anger
burn hot,
my lord;
you
know
these people,
that
they tend
to evil.
\VS{23}They said
to me, ‘Make
us gods
that
will go
before
us, for
as for this
fellow Moses,
the man
who
brought us up
out of the land
of Egypt,
we do not
know
what
has happened to him.’
\VS{24}So I said
to them, ‘Whoever
has gold,
break it off.’
So they gave
it to me, and I threw
it into the fire,
and this
calf
came out.”
\par }{\PP \VS{25}Moses
saw
that the people
were running
wild, for
Aaron
had let them get
completely out of control, causing derision from their enemies.
\VS{26}So Moses
stood
at the entrance
of the camp
and said,
“Whoever
is for the
{\ND{Lord}}, come to
me.” All
the Levites
gathered around him,
\VS{27}and he said
to them, “Thus
says
the {\ND{Lord}},
the God
of Israel,
‘Each man
fasten
his sword
on
his side,
and go
back
and forth from entrance
to entrance
throughout the camp,
and each
one kill
his brother,
his friend,
and his neighbor.’ ”
\par }{\PP \VS{28}The Levites
did what
Moses
ordered, and that day
about three
thousand
men
of the people
died.
\VS{29}Moses
said,
“You have been consecrated
today
for the
{\ND{Lord}}, for
each
of you
was against
his son
or against
his brother,
so he has given
a blessing
to you today.”
\par }{\PP \VS{30}The next
day Moses
said
to
the people, “You
have committed
a very serious
sin, but now
I will go up
to
the {\ND{Lord}} –
perhaps
I can make atonement
on behalf
of your sin.”
\par }{\PP \VS{31}So Moses
returned
to
the {\ND{Lord}}
and said,
“Alas,
this
people
has committed
a very serious
sin, and they have made
for themselves gods
of gold.
\VS{32}But now,
if
you will forgive
their sin…, but if
not,
wipe
me
out
from your book
that
you have written.”
\VS{33}The
{\ND{Lord}}
said
to
Moses,
“Whoever
has sinned against me – that person I will wipe out of my book.
\VS{34}So now
go,
lead
the
people
to
the place I have
spoken
to you about. See,
my angel
will go
before
you. But on the day
that I punish,
I will indeed punish
them for their sin.”
\par }{\PP \VS{35}And the
{\ND{Lord}}
sent a plague
on the people
because
they had
made
the calf –
the one Aaron
made.


\par }\Chap{33}{\PP \VerseOne{1}The
{\ND{Lord}}
said
to
Moses,
“Go
up
from here,
you
and the people
whom
you brought up
out of the land
of Egypt,
to
the land
I promised
on oath
to Abraham,
to Isaac,
and to Jacob,
saying,
‘I will give
it to your descendants.’
\VS{2}I will send
an angel
before
you, and I will drive
out the
Canaanite,
the Amorite,
the Hittite,
the Perizzite,
the Hivite,
and the Jebusite.
\VS{3}Go up to
a land
flowing
with milk
and honey.
But
I will not
go up
among
you, for
you are a stiff-necked
people,
and I might destroy you
on the way.”
\par }{\PP \VS{4}When the people
heard
this
troubling
word
they mourned;
no
one
put on
his ornaments.
\VS{5}For the
{\ND{Lord}}
had said
to
Moses,
“Tell
the Israelites,
‘You
are a stiff-necked
people.
If I
went up
among
you for a moment,
I might destroy
you. Now
take off
your ornaments,
that I may know
what
I should do to you.’ ”
\VS{6}So the Israelites
stripped
off their ornaments
by Mount
Horeb.
\par }{\SH The Presence of the Lord
\par }{\PP \VS{7}Moses
took
the
tent
and pitched
it outside
the camp,
at a good distance
from
the camp,
and he called
it the tent
of meeting.
Anyone
seeking
the {\ND{Lord}}
would go out
to
the tent
of meeting
that
was outside
the camp.
\par }{\PP \VS{8}And when
Moses
went out
to
the tent,
all
the people
would get
up and stand
at the entrance
to their tents
and watch
Moses
until
he entered
the tent.
\VS{9}And whenever
Moses
entered
the tent,
the pillar
of cloud
would descend
and stand
at the entrance
of the tent,
and the
{\ND{Lord}} would speak
with
Moses.
\VS{10}When
all
the people
would see the
pillar
of cloud
standing
at the entrance
of the tent,
all
the people,
each one
at the entrance
of his own tent,
would rise
and worship.
\VS{11}The
{\ND{Lord}}
would speak
to
Moses
face
to
face,
the way a person
speaks
to
a friend.
Then Moses would return
to
the camp,
but his servant,
Joshua
son
of Nun,
a young man,
did not
leave
the tent.
\par }{\PP \VS{12}Moses
said
to
the {\ND{Lord}}, “See,
you
have been saying
to me,
‘Bring
this
people
up,’ but you
have not
let me know
whom
you will send
with
me. But you
said,
‘I know
you by name,
and also
you have found
favor
in my sight.’
\VS{13}Now
if
I have found
favor
in your sight,
show
me
your way,
that I may know
you, that I may
continue to find
favor
in your sight.
And see
that
this
nation
is your people.”
\par }{\PP \VS{14}And the
{\ND{Lord}} said,
“My presence
will go
with you, and I will give you rest.”
\par }{\PP \VS{15}And Moses said
to him,
“If
your presence
does not
go
with us, do not
take us up
from here.
\VS{16}For
how
will it be known
then
that
I have found
favor
in your sight,
I
and your people? Is it not
by your going
with
us, so that we will be distinguished,
I
and your people,
from all
the people
who
are on
the face
of the earth?”
\par }{\PP \VS{17}The
{\ND{Lord}}
said
to
Moses,
“I will do
this
thing
also
that
you have requested,
for
you have found
favor
in my sight,
and I know
you by name.”
\par }{\PP \VS{18}And Moses said,
“Show
me
your glory.”
\par }{\PP \VS{19}And the
{\ND{Lord}} said,
“I
will make all
my goodness
pass
before your face,
and I will proclaim
the {\ND{Lord}}
by name
before
you; I will be gracious
to whom
I will be gracious,
I will show mercy
to whom
I will show mercy.”
\VS{20}But he added,
“You cannot
see
my face,
for
no
one
can see
me and live.”
\VS{21}The
{\ND{Lord}}
said,
“Here
is a place
by me; you will station
yourself on
a rock.
\VS{22}When
my glory
passes
by, I will put
you in a cleft
in the rock
and will cover you with
my hand
while
I pass by.
\VS{23}Then I will take away
my hand,
and you will see
my back,
but my face
must not
be seen.”

\par }\Chap{34}{\PP \VerseOne{1}The
{\ND{Lord}}
said
to
Moses,
“Cut
out two
tablets
of stone
like the first,
and I will write
on
the tablets
the
words
that
were on
the first
tablets,
which
you smashed.
\VS{2}Be
prepared
in the morning,
and go up
in the morning
to
Mount
Sinai,
and station
yourself for me there
on
the top
of the mountain.
\VS{3}No
one is to come up
with
you; do not
let anyone
be seen
anywhere
on the mountain;
not even
the flocks
or the herds
may
graze
in front
of that mountain.”
\VS{4}So Moses
cut
out two
tablets
of stone
like the first;
early
in the morning
he went up
to
Mount
Sinai,
just
as the
{\ND{Lord}}
had commanded
him,
and he took
in his hand
the two
tablets
of stone.
\par }{\PP \VS{5}The
{\ND{Lord}}
descended
in the cloud
and stood
with
him there
and proclaimed
the {\ND{Lord}}
by name.
\VS{6}The
{\ND{Lord}}
passed
by before
him and proclaimed: “The
{\ND{Lord}}, the
{\ND{Lord}}, the compassionate
and gracious
God,
slow
to anger,
and abounding
in loyal love
and faithfulness,
\VS{7}keeping
loyal love
for thousands,
forgiving
iniquity
and transgression
and sin.
But he by no
means leaves the guilty unpunished,
responding
to the transgression
of fathers
by dealing with children
and children’s
children,
to the third and fourth
generation.”
\par }{\PP \VS{8}Moses
quickly
bowed
to the ground
and worshiped
\VS{9}and said,
“If
now
I have found
favor
in your sight,
O Lord,
let
my Lord
go
among
us, for
we are a stiff-necked
people;
pardon
our iniquity
and our sin,
and take us for your inheritance.”
\par }{\PP \VS{10}He said,
“See,
I am
going to make
a covenant
before
all
your people.
I will do
wonders
such as have
not
been
done in all
the earth,
nor in any
nation.
All
the people
among
whom
you
live
will see
the
work
of the {\ND{Lord}}, for
it is
a fearful
thing that
I am
doing
with you.
\par }{\PP \VS{11}“Obey
what
I am
commanding
you this day.
I am going
to drive
out before
you the
Amorite,
the Canaanite,
the Hittite,
the Perizzite,
the Hivite,
and the Jebusite.
\VS{12}Be careful
not
to make
a covenant
with the inhabitants
of the land
where
you
are going,
lest
it become
a snare
among you.
\VS{13}Rather
you must destroy
their altars,
smash their images,
and cut
down
their Asherah poles.
\VS{14}For
you must not
worship
any other
god,
for
the {\ND{Lord}}, whose
name
is Jealous,
is a jealous
God.
\VS{15}Be careful
not to make
a covenant
with the inhabitants
of the land,
for when they prostitute
themselves
to their gods
and sacrifice
to their gods,
and someone invites
you, you will eat
from his sacrifice;
\VS{16}and you then take
his daughters
for your sons,
and when his daughters
prostitute
themselves
to their gods,
they
will make your sons
prostitute
themselves
to their gods as well.
\VS{17}You must not
make
yourselves molten
gods.
\par }{\PP \VS{18}“You must keep
the Feast
of Unleavened
Bread. For seven
days
you must eat
bread made without yeast,
as
I commanded
you; do this at the appointed
time of the month
Abib,
for
in the month
Abib
you came out
of Egypt.
\par }{\PP \VS{19}“Every
firstborn
of the womb
belongs to me, even every
firstborn
of your cattle
that is a male, whether ox
or sheep.
\VS{20}Now the firstling
of a donkey
you may redeem
with a lamb,
but if
you do not
redeem
it, then break
its neck.
You must redeem
all
the firstborn
of your sons.
\par }{\PP “No
one will appear
before
me empty-handed.
\par }{\PP \VS{21}“On
six
days
you may labor,
but on the seventh
day
you must rest;
even at the time of plowing
and of harvest
you are to rest.
\par }{\PP \VS{22}“You must observe
the Feast
of Weeks
– the firstfruits
of the harvest
of wheat
– and the Feast
of Ingathering
at the end
of the year.
\VS{23}At three
times
in the year
all
your men
must appear before
the Lord

{\ND{God}}, the God
of Israel.
\VS{24}For
I will drive
out the nations
before
you and enlarge
your borders;
no
one will covet
your land
when you go up
to appear
before
the {\ND{Lord}}
your God
three
times
in the year.
\par }{\PP \VS{25}“You must not
offer
the blood
of my sacrifice
with yeast;
the sacrifice
from the feast
of Passover
must not
remain
until the following morning.
\par }{\PP \VS{26}“The first
of the firstfruits
of your soil
you must bring
to the house
of the {\ND{Lord}}
your God.
\par }{\PP You must not
cook
a young goat
in its mother’s
milk.”
\par }{\PP \VS{27}The
{\ND{Lord}}
said
to
Moses,
“Write
down these
words,
for
in accordance
with these
words
I
have made
a covenant
with
you and with
Israel.”
\VS{28}So he was there
with
the {\ND{Lord}}
forty
days
and forty
nights;
he did not
eat
bread,
and he did not
drink
water.
He wrote
on
the tablets
the
words
of the covenant,
the ten
commandments.
\par }{\SH The Radiant Face of Moses
\par }{\PP \VS{29}Now when
Moses
came down
from Mount
Sinai
with the two
tablets
of the testimony
in his
hand –
when he came down
from
the mountain,
Moses
did not
know
that
the skin
of his face
shone
while he talked
with him.
\VS{30}When
Aaron
and all
the Israelites
saw
Moses,
the skin
of his face
shone;
and they were afraid
to approach
him.
\VS{31}But Moses
called
to
them, so
Aaron
and all
the leaders
of the community
came
back
to him,
and Moses
spoke
to them.
\VS{32}After
this
all
the Israelites
approached,
and he commanded
them all
that
the {\ND{Lord}}
had
spoken
to him
on Mount
Sinai.
\VS{33}When Moses
finished
speaking
with
them, he would put
a veil
on his face.
\VS{34}But when
Moses
went
in before
the {\ND{Lord}}
to speak
with
him, he would remove
the veil
until
he came out.
Then he would come out
and tell
the Israelites
what
he had
been commanded.
\VS{35}When
the Israelites
would see the
face
of Moses,
that
the skin
of Moses’
face
shone,
Moses
would put the veil
on
his face
again,
until
he went
in to speak
with the
{\ND{Lord}}.


\par }\Chap{35}{\PP \VerseOne{1}Moses
assembled
the whole
community
of the Israelites
and said
to
them, “These
are the things
that
the {\ND{Lord}}
has commanded
you to do.
\VS{2}In six
days
work
may be done,
but on the seventh
day
there must be a holy day
for you, a Sabbath
of complete rest
to the
{\ND{Lord}}. Anyone
who does
work
on it will be put to death.
\VS{3}You must not
kindle
a fire
in any
of your homes
on the Sabbath
day.”
\par }{\SH Willing Workers
\par }{\PP \VS{4}Moses
spoke to
the whole
community
of the Israelites,
“This
is the word
that
the {\ND{Lord}}
has commanded:
\VS{5}‘Take
an offering
for the
{\ND{Lord}}. Let everyone
who has a willing
heart
bring
an offering
to the
{\ND{Lord}}: gold,
silver,
bronze,
\VS{6}blue,
purple,
and scarlet
yarn,
fine linen,
goat’s hair,
\VS{7}ram
skins
dyed red,
fine
leather,
acacia
wood,
\VS{8}olive oil
for the light,
spices
for the anointing
oil
and for the fragrant
incense,
\VS{9}onyx
stones,
and other gems
for mounting
on the ephod
and the breastpiece.
\VS{10}Every
skilled
person
among you is to come
and make
all
that
the {\ND{Lord}}
has commanded:
\VS{11}the tabernacle
with its
tent,
its covering,
its
clasps,
its frames,
its crossbars,
its
posts,
and its
bases;
\VS{12}the ark,
with its
poles,
the atonement lid,
and the special curtain
that conceals it;
\VS{13}the table
with its
poles
and all
its vessels,
and the
Bread
of the Presence;
\VS{14}the lampstand
for the light
and its accessories,
its lamps,
and oil
for the light;
\VS{15}and the altar
of incense
with its
poles,
the
anointing
oil,
and the
fragrant
incense;
the
hanging
for the door
at the entrance
of the tabernacle;
\VS{16}the altar
for the burnt offering
with
its bronze
grating
that
is on it, its
poles,
and all
its utensils;
the large basin
and its pedestal;
\VS{17}the hangings
of the courtyard,
its posts
and its
bases,
and the curtain
for the gateway
to the courtyard;
\VS{18}tent pegs
for the tabernacle
and tent pegs
for the courtyard
and their ropes;
\VS{19}the woven
garments
for serving
in the holy
place, the holy
garments
for Aaron
the priest,
and the garments
for his sons
to minister as priests.”
\par }{\PP \VS{20}So the whole
community
of the Israelites
went out from
the presence
of Moses.
\VS{21}Everyone
whose
heart
stirred
him to action and everyone
whose
spirit
was willing
came
and brought
the offering
for the
{\ND{Lord}}
for the work
of the tent
of meeting,
for all
its service,
and for the holy
garments.
\VS{22}They came,
men
and women
alike, all
who had
willing
hearts.
They brought
brooches,
earrings,
rings
and ornaments,
all
kinds of gold
jewelry,
and everyone
came who
waved
a wave offering
of gold
to the
{\ND{Lord}}.
\par }{\PP \VS{23}Everyone
who had
blue,
purple,
or scarlet
yarn,
fine linen,
goats’
hair, ram
skins
dyed red,
or fine
leather
brought them.
\VS{24}Everyone
making
an offering
of silver
or bronze
brought
it as an offering
to the
{\ND{Lord}}, and everyone
who
had acacia
wood
for any
work
of the service
brought it.
\VS{25}Every
woman
who was skilled
spun
with her hands
and brought
what she had spun,
blue,
purple,
or scarlet
yarn,
or fine linen,
\VS{26}and all
the women
whose
heart stirred
them to action and who were skilled
spun
goats’ hair.
\par }{\PP \VS{27}The leaders
brought
onyx
stones
and other gems
to be mounted
for the ephod
and the breastpiece,
\VS{28}and spices
and olive oil
for the light,
for the anointing
oil,
and for the fragrant
incense.
\par }{\PP \VS{29}The Israelites
brought
a freewill
offering to the
{\ND{Lord}}, every
man
and woman
whose
heart
was willing
to bring
materials for all
the work
that
the {\ND{Lord}}
through
Moses
had commanded
them to do.
\par }{\PP \VS{30}Moses
said
to
the Israelites,
“See,
the {\ND{Lord}}
has chosen
Bezalel
son
of Uri,
the son
of Hur,
of the tribe
of Judah.
\VS{31}He has filled
him with the Spirit
of God
– with skill,
with understanding,
with knowledge,
and in all
kinds of work,
\VS{32}to design artistic
designs,
to work
in gold,
in silver,
and in bronze,
\VS{33}and in cutting
stones
for their setting,
and in cutting
wood,
to do
work
in every
artistic craft.
\VS{34}And he has put
it
in his heart
to teach,
he and Oholiab
son
of Ahisamach,
of the tribe
of Dan.
\VS{35}He has filled
them with skill
to do
all
kinds of work
as craftsmen,
as designers,
as embroiderers
in blue,
purple,
and scarlet
yarn
and in fine linen,
and as weavers.
They are craftsmen in all
the work
and artistic
designers.

\par }\Chap{36}{\PP \VerseOne{1}So Bezalel
and Oholiab
and every
skilled
person
in whom
the {\ND{Lord}}
has put
skill
and ability
to know
how to do
all
the work
for the service
of the sanctuary
are to do the work according to all
that
the {\ND{Lord}}
has commanded.”
\par }{\PP \VS{2}Moses
summoned
Bezalel
and Oholiab
and every
skilled
person
in whom
the {\ND{Lord}}
had
put
skill
– everyone
whose
heart
stirred him
to volunteer
to
do
the work,
\VS{3}and they received
from Moses
all
the offerings
the Israelites
had
brought
to do the work
for the service
of the sanctuary,
and they
still
continued to
bring
him
a freewill
offering each
morning.
\VS{4}So all
the skilled people
who were doing
all
the work
on the sanctuary
came
from the work
they
were doing
\VS{5}and told
Moses,
“The people
are bringing
much
more than is needed
for
the completion
of the work
which
the {\ND{Lord}}
commanded
us to do!”
\par }{\PP \VS{6}Moses
instructed
them to take
his message
throughout the camp,
saying,
“Let no
man
or
woman
do
any more
work
for the offering
for the sanctuary.”
So the people
were restrained
from bringing any more.
\VS{7}Now
the materials
were more than enough
for them to do
all
the work.
\par }{\SH The Building of the Tabernacle
\par }{\PP \VS{8}All
the skilled
among those who were doing
the work
made
the tabernacle
with ten
curtains
of fine
twisted
linen
and blue
and purple
and scarlet;
they were made
with cherubim
that were the work
of an artistic designer.
\VS{9}The length
of one
curtain
was forty-two feet,
and the width
of one
curtain
was six feet
– the same size
for each
of the curtains.
\VS{10}He joined
five
of the curtains
to one
another,
and the other
five
curtains
he joined
to
one
another.
\VS{11}He made
loops
of blue
material along the edge
of the end
curtain
in the first
set;
he did
the same
along the edge
of the end
curtain
in the second
set.
\VS{12}He made fifty
loops
on the first
curtain,
and he made
fifty
loops
on the end
curtain
that
was in the second
set,
with the loops
opposite one
another.
\VS{13}He made
fifty
gold
clasps
and joined
the curtains
together
to
one
another with the clasps,
so that the tabernacle
was a unit.
\par }{\PP \VS{14}He made
curtains
of goats’
hair for a tent
over
the tabernacle;
he made
eleven
curtains.
\VS{15}The length
of one
curtain
was forty-five feet,
and the width
of one curtain was six feet – one size for all eleven curtains.
\VS{16}He joined
five
curtains
by themselves
and six
curtains
by themselves.
\VS{17}He made
fifty
loops
along
the edge
of the end
curtain
in the first set
and fifty
loops
along
the edge
of the curtain
that joined
the second set.
\VS{18}He made
fifty
bronze
clasps
to join
the tent
together so that it might be
a unit.
\VS{19}He made
a covering
for the tent
out of ram
skins
dyed red
and over
that a covering
of fine
leather.
\par }{\PP \VS{20}He made
the frames
for the tabernacle
of acacia
wood
as uprights.
\VS{21}The length
of each frame
was fifteen feet,
the width
of each
frame
was two and a quarter feet,
\VS{22}with two
projections
per frame
parallel
one
to
another. He made
all
the frames
of the tabernacle
in this way.
\VS{23}So he made
frames
for the tabernacle: twenty
frames
for the south
side.
\VS{24}He made
forty
silver
bases
under
the twenty
frames
– two
bases
under
the first
frame
for its two
projections,
and likewise two
bases
under
the next
frame
for its two
projections,
\VS{25}and for the second
side
of the tabernacle,
the north
side,
he made
twenty
frames
\VS{26}and their forty
silver
bases,
two
bases
under
the first frame
and two
bases
under
the next frame.
\VS{27}And for the back
of the tabernacle
on the west
he made
six
frames.
\VS{28}He made
two
frames
for the corners
of the tabernacle
on the back.
\VS{29}At the two
corners
they were
doubled
at the lower
end and finished together
at
the top
in
one
ring.
So
he did
for both.
\VS{30}So there were
eight
frames
and their silver
bases,
sixteen
bases,
two
bases
under
each
frame.
\par }{\PP \VS{31}He made
bars
of acacia
wood,
five
for the frames
on one
side
of the tabernacle
\VS{32}and five
bars
for the frames
on the second
side
of the tabernacle,
and five
bars
for the frames
of the tabernacle
for the back
side on the west.
\VS{33}He made
the middle
bar
to reach
from
end
to
end
in
the center
of the frames.
\VS{34}He overlaid
the frames
with gold
and made
their rings
of gold
to provide
places for the bars,
and he overlaid
the bars
with gold.
\par }{\PP \VS{35}He made
the special curtain
of blue,
purple,
and scarlet
yarn
and fine
twisted
linen;
he made
it with cherubim,
the work
of an artistic designer.
\VS{36}He made
for it four
posts
of acacia wood
and overlaid
them with gold,
with gold
hooks,
and he cast
for them four
silver
bases.
\par }{\PP \VS{37}He made
a hanging
for the entrance
of the tent
of blue,
purple,
and scarlet
yarn
and fine
twisted
linen,
the work
of an embroiderer,
\VS{38}and its five
posts
and their hooks.
He overlaid
their tops
and their bands
with gold,
but their five
bases
were bronze.

\par }\Chap{37}{\PP \VerseOne{1}Bezalel
made
the ark
of acacia
wood;
its length
was three feet nine inches,
its width
two feet three inches,
and its height
two feet three inches.
\VS{2}He overlaid
it with pure
gold,
inside
and out,
and he made
a surrounding
border
of gold
for it.
\VS{3}He cast
four
gold
rings
for it that he put on
its four
feet,
with two
rings
on
one
side
and two
rings
on
the other side.
\VS{4}He made
poles
of acacia
wood,
overlaid
them with gold,
\VS{5}and put
the poles
into the rings
on
the sides
of the ark
in order to carry
the ark.
\par }{\PP \VS{6}He made
an atonement lid
of pure
gold;
its length
was three feet nine inches,
and its width
was two feet three inches.
\VS{7}He made
two
cherubim
of gold;
he made them of hammered
metal
on the two
ends
of the atonement lid,
\VS{8}one
cherub
on one end
and one
cherub on
the other end.
He made
the cherubim
from
the atonement lid
on its two
ends.
\VS{9}The cherubim
were
spreading
their wings
upward,
overshadowing
the atonement lid
with their wings.
The cherubim
faced
each
other,
looking
toward
the atonement lid.
\par }{\SH The Making of the Table
\par }{\PP \VS{10}He made
the table
of acacia
wood;
its length
was three feet,
its width
one foot six inches,
and its height
two feet three inches.
\VS{11}He overlaid
it with pure
gold,
and he made
a surrounding
border
of gold
for it.
\VS{12}He made
a surrounding
frame
for it about three inches wide,
and he made
a surrounding
border
of gold
for its frame.
\VS{13}He cast
four
gold
rings
for it and attached
the rings
at the four
corners
where
its four
legs were.
\VS{14}The
rings
were close
to the frame
to provide
places for the poles
to carry
the table.
\VS{15}He made
the poles
of acacia
wood
and overlaid
them with gold,
to carry
the table.
\VS{16}He made
the vessels
which
were on
the table
out of pure
gold,
its plates,
its
ladles,
its
pitchers,
and its bowls,
to be used in pouring out offerings.
\par }{\SH The Making of the Lampstand
\par }{\PP \VS{17}He made
the lampstand
of pure
gold.
He made
the lampstand
of hammered
metal; its base and its shaft,
its cups,
its buds,
and its blossoms
were from
the same piece.
\VS{18}Six
branches
were extending
from its sides,
three
branches
of the lampstand
from one
side
of it, and three
branches
of the lampstand
from the other side of it.
\VS{19}Three
cups
shaped like almond
flowers
with buds
and blossoms
were on the first branch, and three
cups
shaped like almond
flowers
with buds
and blossoms
were on the next branch, and the same
for the six
branches
that were extending
from
the lampstand.
\VS{20}On the lampstand
there were four
cups
shaped like almond
flowers with buds
and blossoms,
\VS{21}with a bud
under
the first two
branches
from
it, and a bud
under
the next two
branches
from
it, and a bud
under
the third two
branches
from
it; according
to the six
branches
that extended
from it.
\VS{22}Their buds
and their branches
were of one piece;
all
of it was one
hammered
piece of pure
gold.
\VS{23}He made
its seven
lamps,
its trimmers,
and its trays
of pure
gold.
\VS{24}He made
the lampstand and all
its accessories
with seventy-five pounds
of pure
gold.
\par }{\SH The Making of the Altar of Incense
\par }{\PP \VS{25}He made
the incense
altar
of acacia
wood.
Its length
was a foot and a half
and its width
a foot and a half
– a square
– and its height
was three feet.
Its horns were of one piece with it.
\VS{26}He overlaid
it with pure
gold
– its
top,
its four walls,
and its horns
– and he made
a surrounding
border
of gold
for it.
\VS{27}He also made
two
gold
rings
for it under
its border,
on
its two
sides,
on
opposite sides,
as places
for poles
to carry it with.
\VS{28}He made
the poles
of acacia
wood
and overlaid
them with gold.
\par }{\PP \VS{29}He made
the sacred
anointing
oil
and the pure
fragrant
incense,
the work
of a perfumer.

\par }\Chap{38}{\PP \VerseOne{1}He made
the altar
for the burnt offering
of acacia
wood
seven feet six inches
long
and seven feet six inches
wide
– it was square
– and its height
was four
feet
six inches.
\VS{2}He made
its horns
on
its four
corners;
its horns
were part of it, and he overlaid
it with bronze.
\VS{3}He made
all
the utensils
of the altar
– the pots,
the shovels,
the tossing bowls,
the
meat hooks, and the fire pans
– he made
all
its utensils
of bronze.
\VS{4}He made
a grating
for the altar,
a network
of bronze
under
its ledge,
halfway
up from the bottom.
\VS{5}He cast
four
rings
for the four
corners
of the bronze
grating,
to provide
places for the poles.
\VS{6}He made
the poles
of acacia
wood
and overlaid
them with bronze.
\VS{7}He put
the poles
into the rings
on
the sides
of the altar,
with which to carry
it. He made
the altar hollow,
out of boards.
\par }{\PP \VS{8}He made
the large basin
of bronze
and its pedestal
of bronze
from the mirrors
of the women
who
served
at the entrance
of the tent
of meeting.
\par }{\SH The Construction of the Courtyard
\par }{\PP \VS{9}He made
the courtyard.
For the south
side
the hangings
of the courtyard
were of fine
twisted
linen,
one hundred fifty feet long,
\VS{10}with their twenty
posts
and their twenty
bronze
bases,
with the hooks
of the posts
and their bands
of silver.
\VS{11}For the north
side
the hangings were one hundred fifty feet,
with their twenty
posts
and their twenty
bronze
bases,
with the hooks
of the posts
and their bands
of silver.
\VS{12}For the west
side
there were hangings
seventy-five feet
long, with their ten
posts
and their ten
bases,
with the hooks
of the posts
and their bands
of silver.
\VS{13}For the east
side,
toward the sunrise,
it was seventy-five feet wide,
\VS{14}with hangings
on one side
of the gate that were twenty-two and a half feet
long, with their three
posts
and their three
bases,
\VS{15}and for the second
side
of the gate
of the courtyard,
just like the other, the hangings
were twenty-two and a half feet
long, with their three
posts
and their three
bases.
\VS{16}All
the hangings
around
the courtyard
were of fine
twisted
linen.
\VS{17}The bases
for the posts
were bronze.
The hooks
of the posts
and their bands
were silver,
their tops
were overlaid
with silver,
and all
the posts
of the courtyard
had silver
bands.
\VS{18}The curtain
for the gate
of the courtyard
was of blue,
purple,
and scarlet
yarn
and fine
twisted
linen,
the work
of an embroiderer.
It was thirty feet
long,
and like
the hangings
in the courtyard,
it was seven and a half feet
high,
\VS{19}with four
posts
and their four
bronze
bases.
Their hooks
and their bands
were silver,
and their tops
were overlaid
with silver.
\VS{20}All
the tent pegs
of the tabernacle
and of the courtyard
all around
were bronze.
\par }{\SH The Materials of the Construction
\par }{\PP \VS{21}This
is the inventory
of the tabernacle,
the tabernacle
of the testimony,
which
was counted
by
the order
of Moses,
being the work
of the Levites
under the direction
of Ithamar,
son
of Aaron
the priest.
\VS{22}Now Bezalel
son
of Uri,
the son
of Hur,
of the tribe
of Judah,
made
everything
that
the {\ND{Lord}}
had commanded
Moses;
\VS{23}and with
him was Oholiab
son
of Ahisamach,
of the tribe
of Dan,
an artisan,
a designer,
and an embroiderer
in blue,
purple,
and scarlet
yarn
and fine linen.
\par }{\PP \VS{24}All
the gold
that was used
for the work,
in all
the work
of the sanctuary
(namely,
the gold
of the wave offering) was twenty-nine
talents
and 730
shekels,
according to the sanctuary
shekel.
\par }{\PP \VS{25}The silver
of those who were numbered
of the community
was one hundred
talents
and 1,775
shekels,
according to the sanctuary
shekel,
\VS{26}one beka
per person,
that is, a half
shekel,
according to the sanctuary
shekel,
for everyone
who crossed
over
to those numbered,
from twenty
years
old or older,
603,550 in all.
\VS{27}The one hundred
talents
of silver
were used for casting
the bases
of the sanctuary
and the bases
of the special curtain
– one hundred
bases
for one hundred
talents,
one talent
per base.
\VS{28}From the remaining 1,775 shekels he
made
hooks
for the posts,
overlaid
their tops,
and made bands for them.
\par }{\PP \VS{29}The bronze
of the wave offering
was seventy
talents
and 2,400
shekels.
\VS{30}With it he made
the
bases
for the door
of the tent
of meeting,
the
bronze
altar,
the bronze
grating
for it, and all
the utensils
of the altar,
\VS{31}the
bases
for the courtyard
all around,
the
bases
for the gate
of the courtyard,
all
the tent pegs
of the tabernacle,
and all
the tent pegs
of the courtyard
all around.

\par }\Chap{39}{\PP \VerseOne{1}From
the blue,
purple,
and scarlet
yarn
they made
woven
garments
for serving
in the sanctuary;
they made
holy
garments
that
were for Aaron,
just
as the
{\ND{Lord}}
had commanded
Moses.
\par }{\SH The Ephod
\par }{\PP \VS{2}He made
the ephod
of gold,
blue,
purple,
scarlet,
and fine
twisted
linen.
\VS{3}They hammered
the gold
into thin sheets
and cut
it into narrow strips
to weave
them into
the blue,
purple,
and scarlet
yarn,
and into
the fine linen,
the work
of an artistic designer.
\VS{4}They made
shoulder pieces
for it, attached
to two
of its corners,
so it could be joined together.
\VS{5}The artistically woven waistband
of the ephod
that
was on
it was like it, of one piece with it, of gold,
blue,
purple,
and scarlet
yarn
and fine
twisted
linen,
just
as the
{\ND{Lord}}
had commanded
Moses.
\par }{\PP \VS{6}They set
the onyx
stones
in gold
filigree
settings,
engraved
as with the engravings
of a seal
with the names
of the sons
of Israel.
\VS{7}He put
them on
the shoulder pieces
of the ephod
as stones
of memorial
for the Israelites,
just
as the
{\ND{Lord}}
had commanded
Moses.
\par }{\SH The Breastpiece of Decision
\par }{\PP \VS{8}He made
the breastpiece,
the work
of an artistic designer,
in the same fashion
as the ephod,
of gold,
blue,
purple,
and
scarlet,
and fine
twisted
linen.
\VS{9}It was square
– they made
the breastpiece
doubled,
nine
inches long
and nine
inches wide
when doubled.
\VS{10}They set
on it four
rows
of stones: a row
with a ruby,
a topaz,
and a beryl
– the first
row;
\VS{11}and the second
row,
a turquoise,
a sapphire,
and an emerald;
\VS{12}and the third
row,
a jacinth,
an agate,
and an amethyst;
\VS{13}and the fourth
row,
a chrysolite,
an onyx,
and a jasper.
They were enclosed
in gold
filigree
settings.
\VS{14}The stones
were for
the names
of the sons
of Israel,
twelve,
corresponding
to the number of their names.
Each
name
corresponding
to one of the twelve
tribes
was like the engravings
of a seal.
\par }{\PP \VS{15}They made
for
the breastpiece
braided
chains
like cords
of pure
gold,
\VS{16}and they made
two
gold
filigree settings
and two
gold
rings,
and they attached
the two
rings
to the upper two
ends
of the breastpiece.
\VS{17}They attached
the two
gold
chains
to the two
rings
at the ends
of the breastpiece;
\VS{18}the
other two
ends
of the two
chains
they attached
to the two
settings,
and they attached them
to the shoulder pieces
of the ephod
at the front of it.
\VS{19}They made
two
rings
of gold
and put
them on
the other two
ends
of the breastpiece
on
its edge,
which
is on the inner side
of the ephod.
\VS{20}They made
two
more gold
rings
and attached them
to the bottom
of the two
shoulder pieces
on the front
of the
ephod,
close
to the juncture
above
the waistband
of the ephod.
\VS{21}They tied
the breastpiece
by its rings
to
the rings
of the ephod
by blue
cord,
so that it was
above
the waistband
of the ephod,
so that the breastpiece
would not
be loose
from
the ephod,
just
as the
{\ND{Lord}}
had commanded
Moses.
\par }{\SH The Other Garments
\par }{\PP \VS{22}He made
the robe
of the ephod
completely
blue,
the work
of a weaver.
\VS{23}There was an opening
in the center
of the robe,
like the opening
of a collar,
with an edge
all around
the opening
so that it could not
be torn.
\VS{24}They made
pomegranates
of blue,
purple,
and scarlet
yarn
and twisted linen
around the hem
of the robe.
\VS{25}They made
bells
of pure
gold
and attached
the bells
between the pomegranates
around
the hem
of the robe
between the pomegranates.
\VS{26}There was a bell
and a pomegranate,
a bell
and a pomegranate,
all around
the hem
of the robe,
to be used in ministering,
just
as the
{\ND{Lord}}
had commanded
Moses.
\par }{\PP \VS{27}They made
tunics
of fine linen
– the work
of a weaver,
for Aaron
and for his sons –
\VS{28}and the turban
of fine linen,
the headbands
of fine linen,
and the undergarments
of fine
twisted
linen.
\VS{29}The sash
was of fine
twisted
linen
and blue,
purple,
and scarlet
yarn,
the work
of an embroiderer,
just
as the
{\ND{Lord}}
had commanded
Moses.
\VS{30}They made
a plate,
the holy
diadem,
of pure
gold
and wrote
on
it an inscription,
as on the engravings
of a seal,
“Holiness
to the
{\ND{Lord}}.”
\VS{31}They attached
to
it a blue
cord,
to attach
it to
the turban
above,
just
as the
{\ND{Lord}}
had commanded
Moses.
\par }{\SH Moses Inspects the Sanctuary
\par }{\PP \VS{32}So all
the work
of the tabernacle,
the tent
of meeting,
was completed,
and the Israelites
did
according to all
that
the {\ND{Lord}}
had commanded
Moses
– they did it
exactly so.
\VS{33}They brought
the tabernacle
to
Moses,
the tent
and all
its furnishings,
clasps,
frames,
bars,
posts,
and bases;
\VS{34}and the coverings
of ram
skins
dyed red,
the covering
of fine
leather,
and the protecting
curtain;
\VS{35}the ark
of the testimony
and its
poles,
and the atonement lid;
\VS{36}the table,
all
its utensils,
and the Bread
of the Presence;
\VS{37}the pure
lampstand,
its
lamps,
with the lamps
set in order,
and all
its accessories,
and oil
for the light;
\VS{38}and the
gold
altar,
and the
anointing
oil,
and the
fragrant
incense;
and the curtain
for the entrance
to the tent;
\VS{39}the bronze
altar
and its
bronze
grating,
its
poles,
and all
its utensils;
the large basin
with its
pedestal;
\VS{40}the
hangings
of the courtyard,
its posts
and its bases,
and the
curtain
for the gateway
of the courtyard,
its ropes
and its tent pegs,
and all
the furnishings
for the service
of the tabernacle,
for the tent
of meeting;
\VS{41}the woven
garments
for serving
in the sanctuary,
the holy
garments
for Aaron
the priest,
and the garments
for his sons
to minister as priests.
\par }{\PP \VS{42}The Israelites
did
all
the work
according to all
that
the {\ND{Lord}}
had commanded
Moses.
\VS{43}Moses
inspected
all
the work
– and they had done
it just
as the
{\ND{Lord}}
had commanded
– they had done
it exactly
– and Moses
blessed them.

\par }\Chap{40}{\PP \VerseOne{1}Then the
{\ND{Lord}}
spoke
to
Moses:
\VS{2}“On
the first day
of the first
month
you are to set
up the tabernacle,
the tent
of meeting.
\VS{3}You are to place
the ark
of the testimony
in it and shield
the ark
with
the special curtain.
\VS{4}You are to bring
in the
table
and set out
the
things that belong on
it; then you are to bring
in the
lampstand
and set up
its
lamps.
\VS{5}You are to put
the
gold
altar
for incense
in front
of the ark
of the testimony
and put
the
curtain
at the entrance
to the tabernacle.
\VS{6}You are to put
the
altar
for the burnt offering
in front
of the entrance
to the tabernacle,
the tent
of meeting.
\VS{7}You are to put
the
large basin
between
the tent
of meeting
and the altar
and put
water in it.
\VS{8}You are to set
up the courtyard
around
it and put
the curtain
at the gate
of the courtyard.
\VS{9}And take
the
anointing
oil,
and anoint
the tabernacle
and all
that
is in it, and sanctify
it and all
its furnishings,
and it will be
holy.
\VS{10}Then you are to anoint
the
altar
for the burnt offering
with all
its utensils;
you are to sanctify
the altar,
and it will be
the most
holy
altar.
\VS{11}You must also anoint
the large basin
and its
pedestal,
and you are to sanctify it.
\par }{\PP \VS{12}“You are to bring
Aaron
and his sons
to
the entrance
of the tent
of meeting
and wash
them
with water.
\VS{13}Then you are to clothe
Aaron
with
the holy
garments
and anoint
him and sanctify
him so that he may minister as my priest.
\VS{14}You are to bring
his sons
and clothe
them with tunics
\VS{15}and anoint
them just
as you anointed
their father,
so that they may minister
as my priests; their anointing
will make them a priesthood
that will continue
throughout their generations.”
\VS{16}This is what Moses
did,
according to all
the {\ND{Lord}}
had
commanded him – so he did.
\par }{\PP \VS{17}So the tabernacle
was set
up on the first
day
of the first
month,
in the second
year.
\VS{18}When Moses
set
up the tabernacle
and put
its bases
in place, he set
up its frames,
attached
its bars,
and set
up its
posts.
\VS{19}Then he spread
the tent
over
the tabernacle
and put
the covering
of the tent
over
it, as
the {\ND{Lord}}
had
commanded
Moses.
\VS{20}He took
the testimony
and put
it in
the
ark,
attached
the poles
to
the ark,
and then put
the
atonement lid
on
the ark.
\VS{21}And he brought
the
ark
into
the tabernacle,
hung the protecting
curtain,
and shielded
the ark
of the testimony
from view, just
as the
{\ND{Lord}}
had commanded
Moses.
\par }{\PP \VS{22}And he put
the table
in the tent
of meeting,
on
the north
side
of the tabernacle,
outside
the curtain.
\VS{23}And he set
the bread
in order
on
it before
the {\ND{Lord}}, just
as the
{\ND{Lord}}
had commanded
Moses.
\par }{\PP \VS{24}And he put
the lampstand
in the tent
of meeting
opposite
the table,
on
the south
side
of the tabernacle.
\VS{25}Then he set up
the lamps
before
the {\ND{Lord}}, just
as the
{\ND{Lord}}
had commanded
Moses.
\par }{\PP \VS{26}And he put
the
gold
altar
in the tent
of meeting
in front
of the curtain,
\VS{27}and he burned
fragrant
incense
on
it, just
as the
{\ND{Lord}}
had commanded
Moses.
\par }{\PP \VS{28}Then he put
the curtain
at the entrance
to the tabernacle.
\VS{29}He also put
the altar
for the burnt offering
by the entrance
to the tabernacle,
the tent
of meeting,
and offered
on
it the
burnt offering
and the
meal offering,
just
as the
{\ND{Lord}}
had commanded
Moses.
\par }{\PP \VS{30}Then he put
the large basin
between
the tent
of meeting
and the altar
and put
water
in it for washing.
\VS{31}Moses
and Aaron
and his sons
would wash
their hands
and their feet
from it.
\VS{32}Whenever they entered
the tent
of meeting,
and whenever they approached
the altar,
they would wash,
just
as the
{\ND{Lord}}
had commanded
Moses.
\par }{\PP \VS{33}And he set
up the courtyard
around
the tabernacle
and the altar,
and put
the
curtain
at the gate
of the courtyard.
So Moses
finished
the
work.
\par }{\PP \VS{34}Then the cloud
covered
the tent
of meeting,
and the glory
of the {\ND{Lord}}
filled
the tabernacle.
\VS{35}Moses
was not
able
to enter
the tent
of meeting
because
the cloud
settled
on
it and the glory
of the {\ND{Lord}}
filled
the
tabernacle.
\VS{36}But when the cloud
was lifted
up
from the tabernacle,
the Israelites
would set out on all
their journeys;
\VS{37}but if
the cloud
was not
lifted up,
then they would not
journey
further until
the day
it was lifted up.
\VS{38}For
the cloud
of the {\ND{Lord}}
was on
the tabernacle
by day,
but fire
would be
on it at night,
in plain view
of all
the house
of Israel,
throughout all
their journeys.

\par }