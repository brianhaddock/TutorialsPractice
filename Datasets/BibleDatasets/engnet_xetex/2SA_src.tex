\NormalFont\ShortTitle{2 Samuel}
{\MT 2 Samuel

\par }\ChapOne{1}{\SH David Learns of the Deaths of Saul and Jonathan
\par }{\PP \VerseOne{1}After
the death
of Saul,
when David
had returned
from defeating
the Amalekites,
he
stayed
at Ziklag
for two
days.
\VS{2}On
the third
day
a man
arrived
from
the camp
of Saul
with
his clothes
torn
and dirt
on
his head.
When
he approached
David,
the man threw
himself
to the ground.
\par }{\PP \VS{3}David
asked
him, “Where
are you coming
from?” He replied,
“I have escaped
from the camp
of Israel.”
\VS{4}David
inquired, “How
were things
going? Tell
me!” He replied,
“The people
fled
from
the battle
and many
of them fell
dead.
Even
Saul
and his son
Jonathan
are dead!”
\VS{5}David
said
to
the young man
who was telling
him this, “How
do you know
that
Saul
and his son
Jonathan
are dead?”
\VS{6}The young man
who was telling
him this said,
“I just
happened
to be on Mount
Gilboa
and came across Saul
leaning
on
his spear
for support. The chariots
and leaders
of the horsemen
were in hot pursuit
of him.
\VS{7}When he turned
around
and saw
me, he called
out to me.
I answered,
‘Here I am!’
\VS{8}He asked
me, ‘Who
are you?’ I told
him,
‘I’m
an Amalekite.’
\VS{9}He said
to
me, ‘Stand
over
me and finish
me off! I’m very dizzy,
even
though I’m still
alive.’
\VS{10}So I stood
over
him and put him to death,
since
I knew
that
he couldn’t
live
in such a condition. Then
I took
the crown
which
was on
his head
and the bracelet
which
was on
his arm.
I have brought
them
here
to
my lord.”
\par }{\PP \VS{11}David
then grabbed
his own clothes
and tore
them, as did
all
the men
who
were with him.
\VS{12}They lamented
and wept
and fasted
until
evening
because
Saul,
his son
Jonathan,
the
{\ND{Lord}}’s
people,
and the house
of Israel
had
fallen
by the sword.
\par }{\PP \VS{13}David
said
to
the young man
who told
this
to him, “Where
are you
from?” He replied,
“I am
an Amalekite,
the son
of a resident foreigner.”
\VS{14}David
replied
to him,
“How
is it that you were not
afraid
to reach out
your hand
to destroy
the
{\ND{Lord}}’s
anointed?”
\VS{15}Then David
called
one
of the soldiers
and said,
“Come
here and strike
him down!” So he struck
him down, and he died.
\VS{16}David
said
to
him, “Your blood
be on
your own head! Your own mouth
has testified
against you, saying
‘I
have put the
{\ND{Lord}}’s
anointed
to death.’ ”
\par }{\SH David’s Tribute to Saul and Jonathan
\par }{\PP \VS{17}Then David
chanted
this
lament
over
Saul
and his son
Jonathan.
\VS{18}(He gave instructions
that the people
of Judah
should be taught
“The Bow.”
Indeed,
it is written down
in
the Book
of Yashar.)
\par }{\Q \VS{19}The beauty
of Israel
lies slain
on
your high places!
\par }{\Q How
the mighty
have fallen!
\par }{\Q \VS{20}Don’t
report
it in Gath,
\par }{\Q don’t
spread the news
in the streets
of Ashkelon,
\par }{\Q or
the daughters
of the Philistines
will rejoice,
\par }{\Q the daughters
of the uncircumcised
will celebrate!
\par }{\Q \VS{21}O mountains
of Gilboa,
\par }{\Q may there be no
dew
or
rain
on
you, nor fields
of grain offerings!

\par }{\Q For
it was there
that the shield
of warriors
was defiled;
\par }{\Q the shield
of Saul
lies neglected
without
oil.
\par }{\Q \VS{22}From the blood
of the slain,
from the fat
of warriors,
\par }{\Q the bow
of Jonathan
was
not
turned away.
\par }{\Q The sword
of Saul
never
returned
empty.
\par }{\Q \VS{23}Saul
and Jonathan
were greatly loved
during
their lives,
\par }{\Q and not
even in their deaths
were they separated.
\par }{\Q They were swifter
than eagles,
stronger
than lions.
\par }{\Q \VS{24}O daughters
of Israel,
weep
over Saul,
\par }{\Q who clothed
you in scarlet
as well as
jewelry,
\par }{\Q who put gold
jewelry
on
your clothes.
\par }{\Q \VS{25}How
the warriors
have fallen
\par }{\Q in the midst
of battle!
\par }{\Q Jonathan
lies slain
on
your high places!
\par }{\Q \VS{26}I grieve
over
you, my brother
Jonathan!
\par }{\Q You were very
dear
to me.
\par }{\Q Your love
was more special
to me than the love
of women.
\par }{\Q \VS{27}How
the warriors
have fallen!
\par }{\Q The weapons
of war
are destroyed!

\par }\Chap{2}{\PP \VerseOne{1}Afterward
David
inquired
of the {\ND{Lord}}, “Should I go up
to one
of the cities
of Judah?” The
{\ND{Lord}}
told
him,
“Go up.”
David
asked,
“Where
should I go?” The
{\ND{Lord}} replied, “To Hebron.”
\VS{2}So
David
went up, along
with his two
wives,
Ahinoam
the Jezreelite
and Abigail,
formerly the wife
of Nabal
the Carmelite.
\VS{3}David
also brought
along the men
who
were with
him, each
with his family.
They settled
in the cities
of Hebron.
\VS{4}The men
of Judah
came
and there
they anointed
David
as king
over
the people
of Judah.
\par }{\PP David
was told, “The people
of Jabesh
Gilead
are the ones who
buried
Saul.”
\VS{5}So
David
sent
messengers
to
the people
of Jabesh
Gilead
and told
them,
“May you be blessed
by the
{\ND{Lord}}
because you
have shown
this
kindness
to your lord
Saul
by burying him.
\VS{6}Now
may
the {\ND{Lord}}
show you true
kindness! I
also
will reward
you, because you have
done
this deed.
\VS{7}Now
be courageous
and prove to be valiant
warriors,
for
your lord
Saul
is dead.
The people
of Judah
have
anointed
me as king
over them.”
\par }{\SH David’s Army Clashes with the Army of Saul
\par }{\PP \VS{8}Now Abner
son
of Ner,
the general in command
of Saul’s
army,
had taken
Saul’s
son
Ish-bosheth
and had brought
him to Mahanaim.
\VS{9}He appointed him king
over Gilead,
the Geshurites,
Jezreel,
Ephraim,
Benjamin,
and all
Israel.
\VS{10}Ish-bosheth
son
of Saul
was forty
years
old when he began to rule
over
Israel.
He ruled
two
years.
However,
the people
of Judah
followed
David.
\VS{11}David
was king
in Hebron
over
the people
of Judah
for seven
and a half
years.
\par }{\PP \VS{12}Then
Abner
son
of Ner
and the servants
of Ish-bosheth
son
of Saul
went out
from Mahanaim
to Gibeon.
\VS{13}Joab
son
of Zeruiah
and the servants
of David
also went out
and confronted
them at the pool
of Gibeon.
One group
stationed
themselves
on one side
of the pool,
and the other
group on the other side
of the pool.
\VS{14}Abner
said
to
Joab,
“Let
the soldiers
get
up and fight
before
us.” Joab
said,
“So be it!”
\par }{\PP \VS{15}So they got
up and crossed
over by number: twelve
belonging to Benjamin
and to Ish-bosheth
son
of Saul,
and twelve
from the servants
of David.
\VS{16}As they grappled
with one another,
each one
stabbed his opponent
with his sword
and they fell dead
together.
So that
place
is called
the Field of Flints;
it is in Gibeon.
\par }{\PP \VS{17}Now
the battle
was very
severe
that day;
Abner
and the men
of Israel
were overcome
by David’s
soldiers.
\VS{18}The three
sons
of Zeruiah
were
there
– Joab,
Abishai,
and Asahel.
(Now Asahel
was as quick
on his feet
as one
of the gazelles
in the field.)
\VS{19}Asahel
chased
Abner,
without
turning
to the right
or to the left
as he followed
Abner.
\par }{\PP \VS{20}Then Abner
turned
and asked,
“Is that you,
Asahel?” He replied, “Yes it is!”
\VS{21}Abner
said
to him, “Turn aside
to
your right
or
to your left.
Capture
one
of the soldiers
and take
his equipment
for yourself!” But Asahel
was not
willing
to turn aside
from following him.
\VS{22}So
Abner
spoke
again
to
Asahel,
“Turn aside
from following
me! I do not want
to strike
you to the ground.
How
then could I show
my face in the presence
of Joab
your brother?”
\VS{23}But Asahel refused
to turn aside.
So Abner
struck
him in the abdomen
with the back
end of his spear.
The spear
came out
his back; Asahel
collapsed
on the spot
and died
there
right before
Abner. Everyone
who now comes
to
the place
where
Asahel
fell
dead
pauses in respect.
\par }{\PP \VS{24}So Joab
and Abishai
chased
Abner.
At sunset
they
came
to
the hill
of Ammah
near Giah
on
the way
to the wilderness
of Gibeon.
\VS{25}The Benjaminites
formed their ranks
behind
Abner
and were
like a single
army,
standing
at
the top
of a certain
hill.
\par }{\PP \VS{26}Then Abner
called out
to
Joab,
“Must the sword
devour
forever? Don’t
you realize
that
this will turn
bitter
in the end? When
will you tell
the people
to turn
aside from pursuing
their brothers?”
\VS{27}Joab
replied,
“As surely
as God
lives, if
you had not
said
this,
it would have been morning
before
the people
would have abandoned pursuit of their brothers!”
\VS{28}Then Joab
blew
the ram’s horn
and all
the people
stopped in their tracks. They stopped chasing
Israel
and ceased
fighting.
\VS{29}Abner
and his men
went through
the Arabah
all
that night.
They crossed
the
Jordan River
and went through
the whole
region of Bitron
and came
to Mahanaim.
\par }{\PP \VS{30}Now Joab
returned
from chasing
Abner
and assembled
all
the people.
Nineteen
of David’s
soldiers
were missing,
in addition to Asahel.
\VS{31}But David’s
soldiers
had slaughtered
the Benjaminites
and Abner’s
men
– in all, 360
men
had died!
\VS{32}They took
Asahel’s
body and buried
him in his father’s
tomb
at Bethlehem.
Joab
and his men
then traveled
all
that night
and reached Hebron
by dawn.

\par }\Chap{3}{\PP \VerseOne{1}However, the war
was
prolonged
between
the house
of Saul
and the house
of David.
David
was becoming steadily
stronger,
while the house
of Saul
was becoming increasingly
weaker.
\par }{\PP \VS{2}Now sons
were born
to David
in Hebron.
His firstborn
was Amnon,
born to Ahinoam
the Jezreelite.
\VS{3}His second
son was Kileab,
born to Abigail
the widow
of Nabal
the Carmelite.
His third
son was Absalom,
the son
of Maacah
daughter
of King
Talmai
of Geshur.
\VS{4}His fourth
son was Adonijah,
the son
of Haggith.
His fifth
son was Shephatiah,
the son
of Abitail.
\VS{5}His sixth
son was Ithream,
born to David’s
wife
Eglah.
These
sons were all born
to David
in Hebron.
\par }{\SH Abner Defects to David’s Camp
\par }{\PP \VS{6}As
the war
continued between
the house
of Saul
and the house
of David,
Abner
was becoming
more influential
in the house
of Saul.
\VS{7}Now Saul
had a concubine
named
Rizpah
daughter
of Aiah.
Ish-bosheth said
to
Abner,
“Why
did you have sexual relations
with
my father’s
concubine?”
\par }{\PP \VS{8}These words
of Ish-bosheth
really
angered
Abner
and he said,
“Am I the head
of a dog
that belongs
to Judah? This very day
I
am demonstrating
loyalty
to the house
of Saul
your father
and to
his relatives
and his friends! I have not
betrayed
you into the hand
of David.
Yet you have accused
me of sinning
with this woman
today!
\VS{9}God
will severely
judge Abner
if
I do not do
for
David
exactly
what the
{\ND{Lord}}
has
promised him,
\VS{10}namely, to transfer
the kingdom
from the house
of Saul
and to establish
the throne
of David
over
Israel
and over
Judah
all the way
from Dan
to Beer Sheba!”
\VS{11}Ish-bosheth
was unable
to answer
Abner
with even a single word
because he was afraid of him.
\par }{\PP \VS{12}Then
Abner
sent
messengers
to
David
saying,
“To whom
does the land
belong? Make an agreement
with
me, and I
will do whatever I can to cause all
Israel
to turn to you.”
\VS{13}So David said,
“Good! I
will make
an agreement
with
you. I
ask
only
one
thing
from
you. You will not
see
my face
unless
you bring
Saul’s
daughter
Michal
when you come
to visit me.”
\par }{\PP \VS{14}David
sent
messengers
to
Ish-bosheth
son
of Saul
with this demand: “Give
me my wife
Michal
whom
I acquired
for a hundred
Philistine
foreskins.”
\VS{15}So
Ish-bosheth
took
her from her husband
Paltiel
son
of Laish.
\VS{16}Her husband
went
along behind
her, weeping
all the way
to Bahurim.
Finally Abner
said
to him,
“Go
back!” So he returned home.
\par }{\PP \VS{17}Abner
advised
the elders
of Israel,
“Previously you were wanting
David
to be your king.
\VS{18}Act
now! For
the {\ND{Lord}}
has said
to
David,
‘By the hand
of my servant
David
I will save
my people
Israel
from
the Philistines
and from
all
their enemies.’ ”
\par }{\PP \VS{19}Then
Abner
spoke
privately
with the Benjaminites.
Abner
also
went
to Hebron
to inform David
privately
of all
that
Israel
and the entire
house
of Benjamin had agreed to.
\VS{20}When Abner,
accompanied
by twenty
men,
came
to David
in Hebron,
David
prepared
a banquet
for Abner
and the men
who
were with him.
\VS{21}Abner
said
to
David,
“Let me leave
so that I may go
and gather
all
Israel
to
my lord
the king
so that they may make
an agreement
with
you. Then you will rule
over all
that
you desire.”
So David
sent
Abner
away, and he left in peace.
\par }{\SH Abner Is Killed
\par }{\PP \VS{22}Now
David’s
soldiers
and Joab
were coming
back from a raid,
bringing
a great
deal of plunder
with
them. Abner
was no
longer with
David
in Hebron,
for
David had sent
him away and he had left
in peace.
\VS{23}When Joab
and all
the army
that
was with
him arrived,
Joab
was told: “Abner
the son
of Ner
came
to
the king;
he sent
him away, and he left
in peace!”
\par }{\PP \VS{24}So Joab
went
to
the king
and said,
“What
have you done? Abner
has
come
to
you! Why
would you send
him
away? Now he’s gone
on his way!
\VS{25}You know
Abner
the son
of Ner! Surely
he came
here to spy on
you and to determine
when you leave
and when you return
and to discover
everything
that
you
are doing!”
\par }{\PP \VS{26}Then Joab
left
David
and sent
messengers
after
Abner.
They brought him back
from the well
of Sirah.
(But David
was not
aware of it.)
\VS{27}When Abner
returned
to Hebron,
Joab
took him aside
at the gate
as if to speak
privately
with
him. Joab then stabbed
him in the abdomen
and killed
him, avenging the shed blood
of his brother
Asahel.
\par }{\PP \VS{28}When David
later
heard
about this,
he said,
“I
and my kingdom
are forever
innocent
before the
{\ND{Lord}}
of the shed blood
of Abner
son
of Ner!
\VS{29}May his blood whirl
over
the head
of Joab
and the entire
house
of his father! May the males of Joab’s
house
never cease to have someone with
a running sore
or a skin disease
or one who works at the spindle
or one who falls
by the sword
or one who lacks
food!”
\par }{\PP \VS{30}So Joab
and his brother
Abishai
killed
Abner,
because
he had
killed
their brother
Asahel
in Gibeon
during the battle.
\par }{\PP \VS{31}David
instructed
Joab
and all
the people
who
were with
him, “Tear
your clothes! Put on
sackcloth! Lament
before
Abner!” Now King
David
followed
behind
the funeral bier.
\VS{32}So they buried
Abner
in Hebron.
The king
cried
loudly
over Abner’s
grave
and all
the people
wept too.
\VS{33}The king
chanted
the following lament for Abner:
\par }{\Q “Should Abner
have died
like a fool?
\par }{\Q \VS{34}Your hands
were not
bound,
\par }{\Q and your feet
were not
put into irons.
\par }{\Q You fell
the way one falls
before
criminals.”
\par }{\PP All
the people
wept
over him again.
\VS{35}Then all
the people
came
and encouraged
David
to eat food
while it was still
day.
But David
took an oath
saying,
“God
will punish
me severely
if
I taste
bread
or
anything
whatsoever
before
the sun
sets!”
\par }{\PP \VS{36}All
the people
noticed
this and it pleased
them. In fact, everything
the king
did
pleased
all
the people.
\VS{37}All
the people
and all
Israel
realized
on that day
that
the killing of Abner
son
of Ner
was not
done at the king’s
instigation.
\par }{\PP \VS{38}Then the king
said
to his servants,
“Do you not
realize
that
a great
leader
has fallen
this
day
in Israel?
\VS{39}Today
I am
weak,
even though I am anointed
as king.
These
men,
the sons
of Zeruiah,
are
too
much for me to bear! May the
{\ND{Lord}}
punish appropriately
the one who has done
this evil
thing!”

\par }\Chap{4}{\PP \VerseOne{1}When Ish-bosheth the son
of Saul
heard
that
Abner
had died
in Hebron,
he was very disheartened,
and all
Israel
was afraid.
\VS{2}Now Saul’s
son
had two
men
who were in charge of raiding units;
one
was named
Baanah
and the other
Recab.
They were sons
of Rimmon
the Beerothite,
who was a Benjaminite.
(Beeroth
is regarded as
belonging to Benjamin,
\VS{3}for the Beerothites
fled
to Gittaim
and have remained
there
as resident foreigners
until
the present time.)
\par }{\PP \VS{4}Now Saul’s
son
Jonathan
had a son
who was crippled
in both feet.
He
was five
years
old when
the news
about Saul
and Jonathan
arrived
from Jezreel.
His nurse
picked
him up
and fled,
but in her haste
to get away,
he fell
and was injured.
Mephibosheth
was his name.
\par }{\PP \VS{5}Now the sons
of Rimmon
the Beerothite
– Recab
and Baanah
– went
at the hottest
part of the day
to
the home
of Ish-bosheth,
as he was
enjoying his midday
rest.
\VS{6}They
entered
the house
under the pretense
of getting
wheat
and mortally wounded
him in
the stomach.
Then Recab
and his brother
Baanah
escaped.
\par }{\PP \VS{7}They had entered
the house
while Ish-bosheth
was resting on
his bed
in his bedroom.
They mortally
wounded him
and then
cut off
his head.
Taking
his head,
they traveled
on the way
of the Arabah
all
that night.
\VS{8}They brought
the
head
of Ish-bosheth
to
David
in Hebron,
saying
to
the king,
“Look! The head
of Ish-bosheth
son
of Saul,
your enemy
who
sought
your life! The
{\ND{Lord}}
has granted
vengeance
to my lord
the king
this
day
against Saul
and his descendants!”
\par }{\PP \VS{9}David
replied
to Recab
and his brother
Baanah,
the sons
of Rimmon
the Beerothite,
“As surely
as the
{\ND{Lord}}
lives, who
has delivered
my life
from all
adversity,
\VS{10}when
someone told
me that Saul
was dead
– even though he thought he was
bringing good news –
I seized
him and killed
him in Ziklag.
That was the good news
I gave to him!
\VS{11}Surely when
wicked
men have killed
an innocent
man
as he slept
in his own house,
should I not
now
require his blood
from your hands
and remove
you from
the earth?”
\par }{\PP \VS{12}So David
issued orders
to the
soldiers
and they put them
to death.
Then they cut
off their hands
and feet
and hung
them
near the pool
in Hebron.
But they took
the
head
of Ish-bosheth
and buried
it in the tomb
of Abner
in Hebron.

\par }\Chap{5}{\PP \VerseOne{1}All
the tribes
of Israel
came
to
David
at Hebron
saying,
“Look,
we
are your very flesh
and blood!
\VS{2}In the past,
when
Saul
was our king,
you
were the real
leader in Israel.
The
{\ND{Lord}}
said
to you,
‘You will shepherd
my people
Israel;
you
will
rule
over
Israel.’ ”
\par }{\PP \VS{3}When all
the leaders
of Israel
came
to
the king
at Hebron,
King
David
made
an agreement
with them in Hebron
before
the {\ND{Lord}}. They designated
David
as king
over
Israel.
\VS{4}David
was thirty
years
old when he began to reign and he reigned
for forty
years.
\VS{5}In Hebron
he reigned
over
Judah
for seven
years
and six
months,
and in Jerusalem
he reigned
for thirty-three
years
over
all
Israel
and Judah.
\par }{\SH David Occupies Jerusalem
\par }{\PP \VS{6}Then the king
and his men
advanced
to
Jerusalem
against the Jebusites
who lived
in the land.
The Jebusites said
to David,
“You cannot
invade
this place! Even
the blind and
the lame
will turn
you back,
saying,
‘David
cannot
invade
this place!’ ”
\par }{\PP \VS{7}But David
captured
the fortress
of Zion
(that is,
the city
of David).
\VS{8}David
said
on that
day,
“Whoever
attacks
the Jebusites
must approach
the
‘lame’
and the
‘blind’
who
are David’s
enemies by going through the water tunnel.”
For
this reason
it is said,
“The blind
and the lame
cannot
enter
the
palace.”
\par }{\PP \VS{9}So David
lived
in the fortress
and called
it the City
of David.
David
built
all around
it, from
the terrace
inwards.
\VS{10}David’s
power grew steadily,
for the
{\ND{Lord}}
God
who commands armies
was with him.
\par }{\PP \VS{11}King
Hiram
of Tyre
sent
messengers
to
David,
along with cedar
logs,
carpenters,
and stonemasons.
They built
a palace
for David.
\VS{12}David
realized
that
the {\ND{Lord}}
had
established
him as king
over
Israel
and that
he had elevated
his kingdom
for the sake of
his people
Israel.
\VS{13}David
married
more
concubines
and wives
from Jerusalem
after
he arrived
from Hebron.
Even more
sons
and daughters
were born
to David.
\VS{14}These
are the names
of children born
to him in Jerusalem: Shammua,
Shobab,
Nathan,
Solomon,
\VS{15}Ibhar,
Elishua,
Nepheg,
Japhia,
\VS{16}Elishama,
Eliada,
and Eliphelet.
\par }{\SH Conflict with the Philistines
\par }{\PP \VS{17}When the Philistines
heard
that
David
had been designated
king
over
Israel,
they all
went up
to search
for David.
When David
heard
about it, he went down
to
the fortress.
\VS{18}Now the Philistines
had arrived
and spread
out in the valley
of Rephaim.
\VS{19}So David
asked
the {\ND{Lord}}, “Should I march up
against the Philistines? Will you
hand
them over to me?” The
{\ND{Lord}}
said
to
David,
“March up,
for I
will
indeed
hand
the Philistines
over to you.”
\par }{\PP \VS{20}So David
marched
against Baal Perazim
and defeated
them there.
Then he
said,
“The
{\ND{Lord}}
has burst out
against my enemies
like water
bursts out.”
So
he called
the name
of that place
Baal Perazim.
\VS{21}The Philistines abandoned
their idols
there,
and David
and his men
picked them up.
\par }{\PP \VS{22}The Philistines
again
came up
and spread
out in the valley
of Rephaim.
\VS{23}So David
asked
the {\ND{Lord}}
what he should do. This time the
{\ND{Lord}} said
to him, “Don’t
march straight up.
Instead, circle
around behind
them and come
against them opposite
the trees.
\VS{24}When
you hear
the sound
of marching
in the tops
of the trees,
act decisively.
For
at that moment the
{\ND{Lord}}
is going before
you to strike
down the army
of the Philistines.”
\VS{25}David
did
just
as the
{\ND{Lord}}
commanded
him, and he struck
down the Philistines
from Gibeon
all
the way to Gezer.

\par }\Chap{6}{\PP \VerseOne{1}David
again
assembled all
the best men
in Israel,
thirty
thousand in number.
\VS{2}David
and all
the men
who
were with
him traveled
to Baalah in Judah
to bring up
from there
the
ark
of God
which
is called
by the name
of the {\ND{Lord}}
of hosts,
who sits enthroned
between the cherubim
that are on it.
\VS{3}They loaded
the
ark
of God
on a new
cart
and carried
it from the house
of Abinadab,
which
was on the hill.
Uzzah
and Ahio,
the sons
of Abinadab,
were guiding
the
new
cart.
\VS{4}They brought
it with
the ark
of God
up
from the house
of Abinadab
on the hill.
Ahio
was walking
in front
of the ark,
\VS{5}while David
and all
Israel
were energetically celebrating
before
the {\ND{Lord}}, singing and playing various
stringed instruments,
tambourines,
rattles,
and cymbals.
\par }{\PP \VS{6}When
they arrived
at the threshing floor
of Nacon,
Uzzah
reached out
and grabbed hold
of the ark
of God,
because
the oxen
stumbled.
\VS{7}The
{\ND{Lord}}
was so furious
with Uzzah,
he killed
him on
the spot
for his negligence.
He died
right there
beside
the ark
of God.
\par }{\PP \VS{8}David
was angry
because
the {\ND{Lord}}
attacked
Uzzah;
so he called
that place
Perez Uzzah,
which remains
its name to this
very day.
\VS{9}David
was afraid
of the {\ND{Lord}}
that day
and said,
“How
will the ark
of the {\ND{Lord}}
ever come
to me?”
\VS{10}So
David
was no
longer
willing
to
bring the ark
of the {\ND{Lord}}
to be with him
in the City
of David.
David
left
it in the house
of Obed-Edom
the Gittite.
\VS{11}The ark
of the {\ND{Lord}}
remained
in the house
of Obed-Edom
the Gittite
for three
months.
The
{\ND{Lord}}
blessed
Obed-Edom
and all
his family.
\VS{12}David
was told, “The
{\ND{Lord}}
has blessed
the
family
of Obed-Edom
and everything
he owns because of the ark
of God.”
So
David
went
and joyfully
brought
the
ark
of God
from the house
of Obed-Edom
to the City
of David.
\VS{13}Those
who carried
the ark
of the {\ND{Lord}}
took six
steps
and then David sacrificed
an ox
and a fatling calf.
\VS{14}Now David,
wearing
a linen
ephod,
was dancing
with all
his
strength
before
the {\ND{Lord}}.
\VS{15}David
and all
Israel
were bringing up
the ark
of the {\ND{Lord}}, shouting
and blowing trumpets.
\par }{\PP \VS{16}As the ark
of the {\ND{Lord}}
entered
the City
of David,
Saul’s
daughter
Michal
looked
out the window.
When she saw
King
David
leaping
and dancing
before
the {\ND{Lord}}, she despised him.
\VS{17}They brought
the ark
of the {\ND{Lord}}
and put
it in its place
in the middle
of the tent
that
David
had pitched
for it. Then David
offered
burnt sacrifices
and peace offerings
before
the {\ND{Lord}}.
\VS{18}When David
finished
offering
the burnt sacrifices
and peace offerings,
he pronounced a blessing
over the
people
in the name
of the {\ND{Lord}}
of hosts.
\VS{19}He then handed out
to each
member
of the entire
assembly
of Israel,
both men
and women,
a portion
of bread,
a date cake,
and a
raisin cake.
Then
all
the people
went
home.
\VS{20}When David
went home
to pronounce a blessing
on his own house,
Michal,
Saul’s
daughter,
came out to meet
him.
She said,
“How
the king
of Israel
has
distinguished
himself this day! He has exposed
himself today
before
his servants’
slave girls
the way a vulgar fool might do!”
\par }{\PP \VS{21}David
replied
to
Michal,
“It was before
the {\ND{Lord}}! I was celebrating
before
the {\ND{Lord}},
who
chose
me over
your father
and his entire
family
and appointed
me as leader
over
the
{\ND{Lord}}’s
people
Israel.
\VS{22}I am
willing to shame
and humiliate
myself even more
than this! But with
the slave girls
whom
you mentioned
let me
be distinguished!”
\VS{23}Now Michal,
Saul’s
daughter,
had no
children
to
the day
of her death.

\par }\Chap{7}{\PP \VerseOne{1}The king
settled
into his palace,
for the
{\ND{Lord}}
gave him relief
from all
his enemies
on all sides.
\VS{2}The king
said
to
Nathan
the prophet,
“Look! I am
living
in a palace
made from cedar,
while the ark
of God
sits
in the middle
of a tent.”
\VS{3}Nathan
replied
to
the king,
“You should go
and do
whatever
you have in mind,
for
the {\ND{Lord}}
is with you.”
\VS{4}That night
the {\ND{Lord}}
told
Nathan,
\VS{5}“Go,
tell
my servant
David: ‘This is what
the {\ND{Lord}}
says: Do you really intend to build
a house
for me to live in?
\VS{6}I have not
lived
in a house
from the time
I brought
the
Israelites
up from Egypt
to the present
day.
Instead, I was
traveling
with them and living in
a tent.
\VS{7}Wherever
I moved among
all
the Israelites,
I did not say
to any
of the leaders
whom
I appointed
to care for
my people
Israel,
“Why
have you not
built
me a house
made from cedar?” ’
\par }{\PP \VS{8}“So now,
say
this
to my servant
David: ‘This is what
the {\ND{Lord}}
of hosts
says: I
took
you from
the pasture
and from your
work as a shepherd
to make you leader
of my people
Israel.
\VS{9}I was
with
you wherever
you went,
and I defeated
all
your enemies
before
you. Now I will make
you as famous
as the great men
of the earth.
\VS{10}I will establish
a place
for my people
Israel
and settle
them there; they will live there and not
be disturbed
any more.
Violent men
will
not
oppress
them again,
as
they did in the beginning
\VS{11}and during
the time
when I appointed
judges
to lead my people
Israel.
Instead, I will give you relief
from all
your enemies.
The
{\ND{Lord}}
declares
to you that
he himself will build a dynastic
house
for you.
\VS{12}When
the time
comes
for you to die,
I will raise up
your descendant,
one of your own sons,
to succeed you, and I will establish
his kingdom.
\VS{13}He
will build
a house
for my name,
and I will make
his dynasty
permanent.
\VS{14}I
will become
his father
and he
will become
my son.
When
he sins,
I will correct
him with the rod
of men
and with wounds inflicted
by human beings.
\VS{15}But my loyal love
will not
be removed
from
him as
I removed
it from Saul,
whom
I removed
from before you.
\VS{16}Your house
and your kingdom
will stand
before
me permanently;
your dynasty
will be
permanent.’ ”
\VS{17}Nathan
told
David
all
these
words
that were revealed to him.
\par }{\SH David Offers a Prayer to God
\par }{\PP \VS{18}King
David
went
in, sat
before
the {\ND{Lord}}, and said,
“Who
am
I, O
{\ND{Lord}}
God,
and what
is my family,
that
you should have brought
me to this point?
\VS{19}And you didn’t stop
there, O
{\ND{Lord}}
God! You have also
spoken
about the future
of your servant’s
family.
Is this
your usual way
of dealing with men,
O
{\ND{Lord}}
God?
\VS{20}What
more
can David
say
to you? You
have given your servant
special recognition,
O
{\ND{Lord}}
God!
\VS{21}For the sake of
your promise
and according to your purpose
you have done
this great thing
in order to reveal
it to your servant.
\VS{22}Therefore
you are great,
O
{\ND{Lord}}
God,
for
there is none
like
you! There is no
God
besides
you! What we have heard
is true!
\VS{23}Who
is like your people,
Israel,
a unique
nation
on
the earth? Their God
went
to claim
a nation
for himself and to make
a name
for himself! You did
great
and awesome
acts for your land,
before
your people
whom
you delivered
for yourself from the Egyptian
empire
and its gods.
\VS{24}You made
Israel
your very
own
people
for all time. You,
O
{\ND{Lord}}, became
their God.
\VS{25}So now,
O
{\ND{Lord}}
God,
make this promise
you have
made about
your servant
and his family
a permanent
reality.
Do
as
you promised,
\VS{26}so you may gain
lasting
fame,
as people say, ‘The
{\ND{Lord}}
of hosts
is God
over
Israel!’ The dynasty
of your servant
David
will be
established
before you,
\VS{27}for
you,
O
{\ND{Lord}}
of hosts,
the God
of Israel,
have told your servant,
‘I will build
you a dynastic
house.’
That is why
your servant
has
had the
courage
to
pray
this
prayer
to you.
\VS{28}Now,
O sovereign

{\ND{Lord}}, you
are the true God! May your words
prove to be
true! You have
made this
good
promise
to
your servant!
\VS{29}Now
be willing
to bless
your servant’s
dynasty
so that it may
stand permanently
before
you, for
you,
O sovereign

{\ND{Lord}}, have spoken.
By your blessing
may your servant’s
dynasty
be blessed
on into the future!”

\par }\Chap{8}{\PP \VerseOne{1}Later
David
defeated
the Philistines
and subdued
them. David
took
Metheg
Ammah from the Philistines.
\VS{2}He defeated
the
Moabites.
He
made them lie
on the
ground
and then
used a rope
to measure
them off. He put two-thirds of them to death
and spared
the other third. The Moabites
became David’s
subjects
and brought
tribute.
\VS{3}David
defeated
King
Hadadezer
son
of Rehob
of Zobah
when he came
to reestablish
his authority over
the Euphrates River.
\VS{4}David
seized
from
him 1,700
charioteers
and 20,000
infantrymen.
David
cut the hamstrings
of all
but
a hundred
of the chariot horses.
\VS{5}The Arameans
of Damascus
came to help
King
Hadadezer
of Zobah,
but David
killed
22,000
of the Arameans.
\VS{6}David
placed
garrisons
in the territory of the Arameans
of Damascus;
the Arameans
became David’s
subjects
and brought
tribute.
The
{\ND{Lord}}
protected
David
wherever
he campaigned.
\VS{7}David
took
the golden
shields
that belonged
to
Hadadezer’s
servants
and brought
them to Jerusalem.
\VS{8}From Tebah
and Berothai,
Hadadezer’s
cities,
King
David
took
a great deal
of bronze.
\par }{\PP \VS{9}When
King
Toi
of Hamath
heard that
David
had defeated
the entire
army
of Hadadezer,
\VS{10}he
sent
his son
Joram
to
King
David
to extend
his best wishes
and to pronounce a blessing
on
him for his victory
over Hadadezer,
for
Toi
had been
at war
with Hadadezer.
He brought
with him various items
made of silver,
gold,
and bronze.
\VS{11}King
David
dedicated these things
to the
{\ND{Lord}}, along with
the dedicated
silver
and gold
that
he had
taken from all
the nations
that
he had
subdued,
\VS{12}including Aram,
Moab,
the Ammonites,
the Philistines,
and Amelek.
This also included some of the plunder
taken from King
Hadadezer
son
of Rehob
of Zobah.
\par }{\PP \VS{13}David
became famous
when he returned
from defeating
the Arameans
in the Valley
of Salt,
he defeated 18,000 in all.
\VS{14}He placed
garrisons
throughout
Edom,
and all
the Edomites
became
David’s
subjects.
The
{\ND{Lord}}
protected
David
wherever
he campaigned.
\VS{15}David
reigned
over
all
Israel;
he
guaranteed
justice
for all
his people.
\par }{\SH David’s Cabinet
\par }{\PP \VS{16}Joab
son
of Zeruiah
was general
in command of the army; Jehoshaphat
son
of Ahilud
was secretary;
\VS{17}Zadok
son
of Ahitub
and Ahimelech
son
of Abiathar
were priests;
Seraiah
was scribe;
\VS{18}Benaiah
son
of Jehoida
supervised the Kerithites
and Pelethites;
and David’s
sons
were priests.

\par }\Chap{9}{\PP \VerseOne{1}Then David
asked,
“Is anyone
still
left
from the family
of Saul,
so that I may extend
kindness
to him for the sake of
Jonathan?”
\par }{\PP \VS{2}Now there was a servant
from Saul’s
house
named
Ziba,
so he was summoned
to
David.
The king
asked
him,
“Are you
Ziba?” He replied,
“At your service.”
\VS{3}The king
asked,
“Is there not
someone
left from Saul’s
family,
that I may
extend
God’s
kindness
to him?” Ziba
said
to
the king,
“One of Jonathan’s
sons
is left; both of his feet
are crippled.”
\VS{4}The king
asked
him, “Where
is he?” Ziba
told
the king,
“He is
at the house
of Makir
son
of Ammiel
in Lo Debar.
\par }{\PP \VS{5}So
King
David
had him brought from
the house
of Makir
son
of Ammiel
in Lo Debar.
\VS{6}When Mephibosheth
son
of Jonathan,
the son
of Saul,
came
to
David,
he bowed
low with his face
toward the ground.
David
said,
“Mephibosheth?” He replied,
“Yes,
at your service.”
\par }{\PP \VS{7}David
said
to him, “Don’t
be afraid,
because
I will certainly
extend
kindness
to
you for the sake of
Jonathan
your father.
You
will be a regular
guest
at
my table.”
\VS{8}Then Mephibosheth bowed
and said,
“Of what importance
am I, your servant,
that
you show regard
for a dead
dog
like me?”
\par }{\PP \VS{9}Then the king
summoned
Ziba,
Saul’s
attendant,
and said
to him,
“Everything
that belonged
to Saul
and to his entire
house
I hereby give
to your master’s
grandson.
\VS{10}You will cultivate
the land for him – you and your sons and your servants. You will bring its produce and it will be food for your master’s grandson to eat. But Mephibosheth, your master’s grandson, will be a regular guest at my table.” (Now Ziba had fifteen sons and twenty servants.)
\par }{\PP \VS{11}Ziba
said
to
the king,
“Your servant will do everything
that
my lord
the king
has instructed
his servant
to do.”
So
Mephibosheth
was a regular guest
at
David’s table,
just as though he were one
of the king’s
sons.
\par }{\PP \VS{12}Now Mephibosheth
had a young
son
whose name
was Mica.
All
the members
of Ziba’s
household
were Mephibosheth’s
servants.
\VS{13}Mephibosheth
was living
in Jerusalem,
for
he was a regular
guest
at the king’s
table.
But both
his feet
were crippled.


\par }\Chap{10}{\PP \VerseOne{1}Later
the king
of the Ammonites
died
and his son
Hanun
succeeded him.
\VS{2}David
said,
“I will express
my loyalty
to Hanun
son
of Nahash
just
as his father
was loyal
to me.”
So David
sent
his servants
with
a message expressing
sympathy
over his father’s
death. When David’s
servants
entered
the land
of the Ammonites,
\VS{3}the Ammonite
officials
said
to
their lord
Hanun,
“Do you really think
David
is trying to honor
your father
by sending
these messengers to express
his sympathy? No,
David
has sent
his servants
to
you to
get information
about the
city
and spy
on it so they can
overthrow it!”
\par }{\PP \VS{4}So Hanun
seized
David’s
servants
and shaved off
half
of each one’s beard.
He cut the
lower part
of their robes
off so that
their buttocks
were exposed, and then sent
them away.
\VS{5}Messengers told
David
what had happened, so
he summoned
them, for
the men
were thoroughly
humiliated.
The king
said,
“Stay
in Jericho
until
your beards
have grown
again; then you may come back.”
\par }{\PP \VS{6}When
the Ammonites
realized that
David
was disgusted
with them, they
sent
and hired
20,000
foot soldiers
from Aram
Beth Rehob
and Aram
Zobah,
in addition to 1,000
men
from the king
of Maacah
and 12,000
men
from Ish-tob.
\par }{\PP \VS{7}When David
heard
the news, he sent
Joab
and the entire
army
to meet them.
\VS{8}The Ammonites
marched out
and were deployed
for battle
at the entrance
of the city gate,
while the men
from Aram
Zobah,
Rehob,
Ish-tob,
and Maacah
were by themselves
in the field.
\par }{\PP \VS{9}When Joab
saw
that
the battle
would be
fought on two fronts,
he chose
some of Israel’s
best men
and deployed
them against
the Arameans.
\VS{10}He put
his
brother
Abishai
in charge of the rest
of the army
and they were deployed
against
the Ammonites.
\VS{11}Joab said,
“If
the Arameans
start to overpower
me, you come to my rescue.
If
the Ammonites
start
to overpower
you, I will come
to your rescue.
\VS{12}Be strong! Let’s
fight bravely
for the sake of our people
and the cities
of our God! The
{\ND{Lord}}
will do
what he decides
is best!”
\par }{\PP \VS{13}So
Joab
and his men
marched
out to do battle
with the Arameans,
and they fled
before him.
\VS{14}When the Ammonites
saw
the Arameans
flee,
they fled
before
his brother Abishai
and went into
the city.
Joab
withdrew
from
fighting the Ammonites
and returned
to Jerusalem.
\par }{\PP \VS{15}When
the Arameans
realized that
they had been defeated
by
Israel,
they consolidated
their forces.
\VS{16}Then Hadadezer
sent
for Arameans
from beyond
the Euphrates River,
and they came
to Helam.
Shobach,
the general in command
of Hadadezer’s
army,
led them.
\par }{\PP \VS{17}When David
was informed,
he gathered
all
Israel,
crossed
the Jordan River,
and came
to Helam.
The Arameans
deployed
their forces against
David
and fought
with him.
\VS{18}The Arameans
fled
before
Israel.
David
killed
700
Aramean
charioteers
and 40,000
foot soldiers.
He also struck down
Shobach,
the general
in command of the army,
who died
there.
\VS{19}When
all
the kings
who were subject
to Hadadezer
saw they were defeated
by Israel,
they made peace
with
Israel
and became subjects
of Israel. The Arameans
were no longer
willing to help
the
Ammonites.

\par }\Chap{11}{\PP \VerseOne{1}In the spring
of the year,
at the time
when kings
normally conduct
wars, David
sent out
Joab
with
his officers
and the entire
Israelite
army. They defeated
the Ammonites
and besieged
Rabbah.
But David
stayed
behind in Jerusalem.
\VS{2}One
evening
David
got
up from his bed
and walked
around on
the roof
of his palace.
From the roof
he saw
a woman
bathing. Now this woman
was very
attractive.
\VS{3}So David
sent
someone to inquire about
the woman.
The messenger said,
“Isn’t
this
Bathsheba,
the daughter
of Eliam,
the wife
of Uriah
the Hittite?”
\par }{\PP \VS{4}David
sent
some messengers
to get
her. She came
to him
and he had sexual
relations with
her. (Now at that time she
was in the process of purifying
herself from her menstrual uncleanness.) Then she returned
to
her home.
\VS{5}The woman
conceived
and then sent
word
to David
saying,
“I’m
pregnant.”
\par }{\PP \VS{6}So David
sent
a message
to Joab
that said,
“Send
me
Uriah
the Hittite.”
So Joab
sent
Uriah
to
David.
\VS{7}When Uriah
came
to
him, David
asked
about how Joab
and the army
were doing
and how the campaign
was going.
\VS{8}Then David
said
to Uriah,
“Go down
to your home
and relax.”
When Uriah
left
the palace,
the king
sent
a gift
to him.
\VS{9}But Uriah
stayed
at the door
of the palace
with
all
the servants
of his lord.
He did not
go down
to
his house.
\par }{\PP \VS{10}So they informed
David,
“Uriah
has not
gone down
to
his house.”
So David
said
to
Uriah,
“Haven’t you
just arrived
from a journey? Why
haven’t
you gone down
to
your house?”
\VS{11}Uriah
replied
to
David,
“The ark
and Israel
and Judah
reside
in temporary shelters,
and my lord
Joab
and my lord’s
soldiers
are camping
in
the open field.
Should I
go
to
my house
to eat
and drink
and have marital relations
with
my wife? As surely
as you are alive,
I will not
do
this
thing!”
\VS{12}So David
said
to
Uriah,
“Stay
here
another
day.
Tomorrow
I will send
you back.” So Uriah
stayed
in Jerusalem
both that day
and the following one.
\VS{13}Then David
summoned
him. He ate
and drank
with him, and got him drunk.
But in the evening
he went out
to sleep
on his bed
with
the servants
of his lord;
he did not
go down
to
his own house.
\par }{\PP \VS{14}In the morning
David
wrote
a letter
to
Joab
and sent
it with Uriah.
\VS{15}In the letter
he wrote: “Station
Uriah
in the thick
of the battle
and then withdraw
from him
so he will be
cut down and killed.”
\par }{\PP \VS{16}So
as Joab
kept watch
on the city,
he stationed
Uriah
at the place
where
he knew
the best enemy
soldiers were.
\VS{17}When
the men
of the city
came out
and fought
with
Joab,
some
of David’s
soldiers
fell
in battle. Uriah
the Hittite
also
died.
\par }{\PP \VS{18}Then Joab
sent
a full battle
report
to David.
\VS{19}He instructed
the messenger
as follows: “When you finish
giving
the battle
report
to
the king,
\VS{20}if
the king
becomes
angry
and asks
you, ‘Why
did you go so close
to
the city
to fight? Didn’t
you realize
they would shoot
from the wall?
\VS{21}Who
struck down
Abimelech
the son
of Jerub-Besheth? Didn’t
a woman
throw
an upper
millstone
down on
him from the wall
so that he died
in Thebez? Why
did you go so close
to
the wall?’ just say
to him, ‘Your servant
Uriah
the Hittite
is also
dead.’ ”
\par }{\PP \VS{22}So
the messenger
departed.
When he arrived,
he informed
David
of all
the news that
Joab
had
sent with him.
\VS{23}The messenger
said
to
David,
“The men
overpowered
us and attacked
us in the field.
But we forced them to retreat all the way
to the door
of the city gate.
\VS{24}Then
the archers
shot
at your servants
from the wall
and some of the king’s
soldiers
died.
Your servant
Uriah
the Hittite
is also
dead.”
\VS{25}David
said
to
the messenger,
“Tell
Joab,
‘Don’t
let this
thing
upset
you. There is no way to anticipate whom the sword
will cut down.
Press
the battle
against the city
and conquer
it.’ Encourage him with these words.”
\par }{\PP \VS{26}When Uriah’s
wife
heard
that
her husband
Uriah
was dead,
she mourned for him.
\VS{27}When the time of mourning
passed,
David
had
her brought
to
his palace.
She became
his wife
and she bore
him a son.
But what
David
had
done
upset
the {\ND{Lord}}.

\par }\Chap{12}{\PP \VerseOne{1}So the
{\ND{Lord}}
sent
Nathan
to
David.
When he came
to
David, Nathan said, “There were two
men
in a certain city,
one
rich
and the other
poor.
\VS{2}The rich
man had a great
many
flocks
and herds.
\VS{3}But the poor man
had nothing
except for
a little
lamb
he had acquired.
He raised
it, and it grew up
alongside
him and his children.
It used to eat
his food,
drink
from his cup,
and sleep
in his arms.
It was
just like a daughter to him.
\par }{\PP \VS{4}“When
a traveler
arrived at the rich
man’s
home, he did not want
to use
one of his own sheep
or cattle
to feed
the traveler
who had come
to visit him. Instead, he took
the poor
man’s
lamb
and cooked
it for the man
who had come
to visit him.”
\par }{\PP \VS{5}Then David
became very
angry
at this man.
He said
to
Nathan,
“As surely
as the
{\ND{Lord}}
lives, the man
who did
this deserves to die!
\VS{6}Because
he committed
this
cold-hearted
crime,
he must pay
for the lamb
four times over!”
\par }{\PP \VS{7}Nathan
said
to
David,
“You
are that man! This is what
the {\ND{Lord}}
God
of Israel
says: ‘I
chose
you to be king
over
Israel
and I
rescued
you from the hand
of Saul.
\VS{8}I gave
you your master’s
house,
and put your master’s
wives
into your arms.
I also gave
you the house
of Israel
and Judah.
And if
all that somehow
seems insignificant,
I would have given you so much more as well!
\VS{9}Why
have you shown contempt
for the
word
of the {\ND{Lord}}
by doing
evil
in my sight? You have struck
down Uriah
the Hittite
with the sword
and you have taken
his wife
as your own! You have killed
him with the sword
of the Ammonites.
\VS{10}So now
the sword
will never
depart
from
your house.
For
you have despised
me by taking
the
wife
of Uriah
the Hittite as your own!’
\VS{11}This is what
the {\ND{Lord}}
says: ‘I am about to bring
disaster
on
you from inside your own household! Right before your eyes
I will take
your wives
and hand
them over
to your companion.
He will have sexual relations
with
your wives
in broad daylight!
\VS{12}Although
you
have acted
in secret,
I
will do
this
thing
before
all
Israel, and in broad daylight.’ ”
\par }{\PP \VS{13}Then David
exclaimed to
Nathan,
“I have sinned
against the
{\ND{Lord}}!” Nathan
replied
to
David,
“Yes,
and the
{\ND{Lord}}
has forgiven
your sin.
You are not
going to die.
\VS{14}Nonetheless,
because
you have treated
the {\ND{Lord}}
with such contempt in this
matter,
the son
who has been born
to you will certainly
die.”
\par }{\PP \VS{15}Then Nathan
went
to
his home.
The
{\ND{Lord}}
struck
the child
that
Uriah’s
wife
had borne
to David,
and the child became very ill.
\VS{16}Then David
prayed
to God
for
the child
and fasted.
He
would even go
and spend
the night lying
on the ground.
\VS{17}The elders
of his house
stood
over
him and tried to lift
him from
the ground,
but he was unwilling,
and refused to eat food
with them.
\par }{\PP \VS{18}On
the seventh
day
the child
died.
But the servants
of David
were afraid
to inform
him that
the child
had
died,
for
they said,
“While
the child
was still alive
he would
not
listen
to us when
we spoke to him. How
can we tell
him
that the child
is dead? He will do
himself harm!”
\par }{\PP \VS{19}When David
saw
that
his servants
were whispering
to one another, he
realized
that the child
was dead.
So David
asked
his servants,
“Is the child
dead?” They replied,
“Yes, he’s dead.”
\VS{20}So David
got
up from the ground,
bathed,
put on oil,
and changed
his clothes.
He went
to the house
of the {\ND{Lord}}
and worshiped.
Then, when he entered
his palace,
he requested
that food
be brought to him,
and he ate.
\par }{\PP \VS{21}His servants
said
to him,
“What
is this
that
you have done? While
the child
was still alive,
you fasted
and wept.
Once the child
was dead
you got
up and ate
food!”
\VS{22}He replied,
“While
the child
was still alive,
I fasted
and wept
because
I
thought, ‘Perhaps
the {\ND{Lord}}
will show
pity
and the child
will live.
\VS{23}But now
he is dead.
Why
should I
fast? Am I
able
to bring him back? I
will go
to him,
but he cannot
return
to me!’ ”
\par }{\PP \VS{24}So David
comforted
his wife
Bathsheba.
He went
to her
and had marital
relations with
her. She gave birth
to a son,
and David named
him Solomon.
Now the
{\ND{Lord}}
loved the child
\VS{25}and sent
word through
Nathan
the prophet
that he should be named
Jedidiah
for
the
{\ND{Lord}}’s
sake.
\par }{\SH David’s Forces Defeat the Ammonites
\par }{\PP \VS{26}So Joab
fought
against Rabbah
of the Ammonites
and captured
the royal
city.
\VS{27}Joab
then sent
messengers
to
David,
saying,
“I have fought
against Rabbah
and have
captured
the water
supply of the city.
\VS{28}So now
assemble
the
rest
of the army
and besiege
the city
and capture
it. Otherwise
I
will capture
the city
and it will be named
for me.”
\par }{\PP \VS{29}So David
assembled
all
the army
and went
to Rabbah
and fought
against it and captured it.
\VS{30}He took
the
crown
of their king
from his head
– it was gold,
weighed
about seventy-five pounds,
and held a precious
stone
– and it was placed on
David’s
head.
He also took from the city
a great
deal
of plunder.
\VS{31}He removed
the people
who
were in it and made
them do hard labor
with saws,
iron picks,
and iron
axes,
putting
them to work at the brick kiln.
This
was his policy
with all
the Ammonite
cities.
Then David
and all
the army
returned
to Jerusalem.

\par }\Chap{13}{\PP \VerseOne{1}Now David’s
son
Absalom
had a beautiful
sister
named
Tamar.
In the course of time David’s
son
Amnon
fell madly in love with her.
\VS{2}But Amnon
became frustrated
because he was so lovesick
over his sister
Tamar.
For
she was a virgin,
and to Amnon
it seemed
out of the question
to do
anything to her.
\par }{\PP \VS{3}Now Amnon
had a friend
named
Jonadab,
the son
of David’s
brother
Shimeah.
Jonadab
was a very
crafty
man.
\VS{4}He asked
Amnon, “Why
are you,
the king’s
son,
so
depressed
every morning? Can’t
you tell
me?” So Amnon
said
to him, “I’m
in love
with Tamar
the sister
of my brother
Absalom.”
\VS{5}Jonadab
replied
to him, “Lie
down on
your bed
and pretend to be sick.
When your father
comes
in to see
you, say
to
him, ‘Please
let my sister
Tamar
come in so she can fix
some food
for me. Let her prepare
the food
in my sight
so
I can watch.
Then I will eat
from her hand.’ ”
\par }{\PP \VS{6}So Amnon
lay
down and pretended to be sick.
When the king
came
in to see
him,
Amnon
said
to
the king,
“Please
let my
sister
Tamar
come
in so
she can make a couple
of cakes
in my sight. Then I
will eat from her
hand.”
\par }{\PP \VS{7}So David
sent
Tamar
to the house
saying,
“Please
go
to the house
of Amnon
your brother
and prepare
some food for him.”
\VS{8}So Tamar
went
to the house
of Amnon
her brother,
who was
lying
down. She took
the dough,
kneaded
it, made some cakes while he watched, and baked them.
\VS{9}But when she took
the pan
and set it
before
him, he refused
to eat.
Instead Amnon
said,
“Get everyone out
of here!” So everyone
left.
\par }{\PP \VS{10}Then Amnon
said
to Tamar,
“Bring
the cakes
into the bedroom;
then I will eat
from your hand.”
So Tamar
took
the cakes
that
she had prepared
and brought
them to her brother
Amnon
in the bedroom.
\VS{11}As she brought
them to
him to eat,
he grabbed
her and said
to her, “Come
on! Get in bed
with
me, my sister!”
\par }{\PP \VS{12}But she said
to him, “No,
my brother! Don’t
humiliate
me! This
just isn’t
done
in Israel! Don’t
do
this foolish thing!
\VS{13}How
could I
ever be rid
of my humiliation? And you
would be considered one
of the fools
in Israel! Just
speak
to
the king,
for
he will not
withhold
me from you.”
\VS{14}But he refused
to listen
to her.
He overpowered
her and humiliated
her by raping her.
\VS{15}Then Amnon
greatly despised
her.
His disdain
toward her surpassed the love
he had
previously felt toward
her. Amnon
said
to her, “Get up
and leave!”
\par }{\PP \VS{16}But she said
to him, “No
I won’t, for sending
me away
now would be worse
than what you did
to me earlier!” But he refused
to listen to her.
\VS{17}He called
his personal
attendant
and said
to him, “Take
this
woman out
of my sight and lock
the door
behind her!”
\VS{18}(Now she was wearing a long
robe,
for
this
is what the king’s
virgin
daughters
used to wear.) So Amnon’s attendant
removed
her and bolted
the door
behind her.
\VS{19}Then
Tamar
put ashes
on
her head
and tore
the long
robe
she was wearing. She put
her hands
on
her head
and went
on her way,
wailing as she went.
\par }{\PP \VS{20}Her brother
Absalom
said
to
her, “Was
Amnon
your brother
with
you? Now
be quiet,
my sister.
He is
your brother.
Don’t
take it
so seriously!” Tamar,
devastated,
lived
in the house
of her brother
Absalom.
\par }{\PP \VS{21}Now King
David
heard
about all
these
things
and was very
angry.
\VS{22}But Absalom
said
nothing to Amnon,
either
bad
or good,
yet Absalom
hated
Amnon
because
he had
humiliated
his sister
Tamar.
\par }{\SH Absalom Has Amnon Put to Death
\par }{\PP \VS{23}Two years
later
Absalom’s
sheepshearers
were in Baal Hazor,
near
Ephraim.
Absalom
invited
all
the king’s
sons.
\VS{24}Then Absalom
went
to
the king
and said,
“My shearers
have begun their work. Let
the king
and his servants
go
with me.”
\par }{\PP \VS{25}But the king
said
to
Absalom,
“No,
my son.
We shouldn’t
all
go.
We shouldn’t
burden
you in
that way.” Though Absalom pressed
him, the king was not
willing
to go.
Instead, David blessed him.
\par }{\PP \VS{26}Then Absalom
said,
“If you will not
go,
then let
my brother
Amnon
go with
us.” The king
replied
to him, “Why
should he go
with you?”
\VS{27}But when Absalom
pressed
him, he sent
Amnon
and all
the king’s
sons
along with him.
\par }{\PP \VS{28}Absalom
instructed
his servants,
“Look! When Amnon
is drunk
and I say
to
you, ‘Strike
Amnon
down,’ kill
him then and there. Don’t
fear! Is it not
I
who have given you these instructions? Be strong
and courageous!”
\VS{29}So Absalom’s
servants
did
to Amnon
exactly what
Absalom
had instructed.
Then all
the king’s
sons
got up;
each
one rode
away on
his mule
and fled.
\par }{\PP \VS{30}While
they
were still on
their way,
the following report
reached
David: “Absalom
has killed
all
the king’s
sons;
not
one
of them
is left!”
\VS{31}Then
the king
stood up and tore
his garments
and lay
down on the ground.
All
his servants
were standing
there with torn
garments as well.
\par }{\PP \VS{32}Jonadab,
the son
of David’s
brother
Shimeah,
said,
“My lord
should not
say,
‘They have killed
all
the young men
who are the king’s
sons.’
For
only
Amnon
is dead.
This
is what Absalom
has talked
about from the day
that Amnon humiliated
his sister
Tamar.
\VS{33}Now
don’t
let
my lord
the king
be concerned
about the report
that has
come saying,
‘All
the king’s
sons
are dead.’
It is
only
Amnon
who is dead.”
\par }{\PP \VS{34}In the meantime Absalom
fled.
When the servant
who was the watchman
looked up,
he saw
many
people
coming
from the west on a road
beside
the hill.
\VS{35}Jonadab
said
to
the king,
“Look! The king’s
sons
have come! It’s just
as I
said!”
\par }{\PP \VS{36}Just as he finished
speaking,
the king’s
sons
arrived,
wailing
and weeping.
The king
and all
his servants
wept
loudly
as well.
\VS{37}But Absalom
fled
and went
to
King
Talmai
son
of Ammihud
of Geshur.
And David grieved
over
his son
every
day.
\par }{\PP \VS{38}After Absalom
fled
and went
to Geshur,
he remained
there
for three
years.
\VS{39}The king
longed
to go
to
Absalom,
for
he
had since been consoled
over
the death
of Amnon.

\par }\Chap{14}{\PP \VerseOne{1}Now Joab
son
of Zeruiah
realized
that
the king
longed
to see
Absalom.
\VS{2}So Joab
sent
to Tekoa
and brought from
there
a wise
woman.
He told
her, “Pretend to be in mourning
and put on
garments
for mourning.
Don’t
anoint
yourself with oil.
Instead, act like
a woman
who has been mourning
for the dead
for
some
time.
\VS{3}Go
to
the king
and speak
to him
in the following
fashion.” Then
Joab
told
her
what to say.
\par }{\PP \VS{4}So the Tekoan
woman
went to
the king.
She bowed down
with her face
to the ground
in deference
to him
and said,
“Please help
me, O king!”
\VS{5}The king
replied
to her, “What
do you want?” She answered,
“I am
a widow;
my husband
is dead.
\VS{6}Your servant
has two
sons.
When the two
of them got into a fight
in the field,
there was no
one present who could intervene.
One
of them struck
the other
and killed him.
\VS{7}Now
the entire
family
has risen up
against
your servant,
saying,
‘Turn over
the
one who struck down
his brother,
so that we can execute
him and avenge the death of his brother
whom
he killed.
In so doing we will also
destroy
the
heir.’
They want to extinguish
my remaining
coal,
leaving
no one
on
the face
of the earth
to carry
on the name of my husband.”
\par }{\PP \VS{8}Then the king
told
the woman,
“Go
to your home.
I
will give instructions
concerning your situation.”
\VS{9}The Tekoan
woman
said
to
the king,
“My lord
the king,
let any blame
fall on me and on
the house
of my father.
But let the king
and his throne
be innocent!”
\par }{\PP \VS{10}The king
said,
“Bring
to
me whoever speaks
to
you, and he won’t
bother
you again!”
\VS{11}She replied,
“In that case,
let
the king
invoke
the
name of the
{\ND{Lord}}
your God
so that the avenger
of blood
may not kill! Then
they will not
destroy
my son!” He replied,
“As surely
as the
{\ND{Lord}}
lives, not
a single
hair
of
your son’s
head will fall
to the ground.”
\par }{\PP \VS{12}Then the woman
said,
“Please
permit your servant
to speak to
my lord
the king
about another matter.”
He replied,
“Tell me.”
\VS{13}The woman
said,
“Why
have you devised
something like this
against
God’s
people? When the king
speaks
in this
fashion,
he makes himself guilty,
for the king
has not
brought back
the one he has banished.
\VS{14}Certainly
we must die,
and are like water
spilled
on the ground
that
cannot
be gathered up
again. But God
does not
take away
life;
instead
he devises
ways
for the banished
to be restored.
\VS{15}I have now
come
to speak
with my lord
the king
about this
matter,
because
the people
have made me fearful.
But your servant
said, ‘I will speak
to
the king! Perhaps
the king
will do
what
his female servant asks.
\VS{16}Yes! The king
may listen
and deliver
his female servant
from the hand
of the man
who seeks
to remove both
me and my son
from the inheritance
God has given us!’
\VS{17}So your servant
said,
‘May
the word
of my lord
the king
be my security,
for
my lord
the king
is like
the angel
of God
when it comes to deciding
between right
and wrong! May the
{\ND{Lord}}
your God
be
with you!’ ”
\par }{\PP \VS{18}Then
the king
replied
to
the woman,
“Don’t
hide
any
information
from
me when
I
question
you.” The woman
said,
“Let
my lord
the king
speak!”
\VS{19}The king
said,
“Did Joab
put you up to
all
of this?” The woman
answered,
“As surely
as you live,
my lord
the king,
there is no
deviation to the right
or to the left
from all
that
my lord
the king
has said.
For
your servant
Joab
gave
me instructions.
He
has put
all
these
words
in your servant’s
mouth.
\VS{20}Your servant
Joab
did
this
so as to
change
this situation.
But my lord
has wisdom
like that of the angel
of God,
and knows
everything
that
is happening in the land.”
\par }{\PP \VS{21}Then the king
said
to
Joab,
“All right! I will do
this
thing! Go
and bring back
the young man
Absalom!
\VS{22}Then
Joab
bowed
down with his face
toward the
ground
and thanked
the king.
Joab
said,
“Today
your servant
knows
that
I have found
favor
in your sight,
my lord
the king,
because
the king
has granted the request
of your servant!”
\par }{\PP \VS{23}So Joab
got up
and went
to Geshur
and brought
Absalom
back to Jerusalem.
\VS{24}But the king
said,
“Let him go over
to
his own house.
He may not
see
my face.”
So Absalom
went over
to
his own house;
he did not
see
the king’s
face.
\par }{\PP \VS{25}Now in all
Israel
everyone acknowledged
that there was
no
man
as handsome
as Absalom.
From the sole
of his feet
to
the top of
his head
he was
perfect in appearance.
\VS{26}When he would shave
his head
– at the end
of every year
he used to shave
his head, for
it grew too long
and he would shave it – he used to weigh the hair of his head at three pounds according to the king’s weight.
\VS{27}Absalom
had
three
sons
and one
daughter,
whose name
was Tamar.
She
was
a very attractive
woman.
\par }{\PP \VS{28}Absalom
lived
in Jerusalem
for two years
without
seeing
the king’s
face.
\VS{29}Then Absalom
sent
a message to
Joab
asking him to
send
him to
the king,
but Joab was not
willing
to come
to
him. So he sent
a second
message to him, but he still
was not
willing
to come.
\VS{30}So he said
to
his servants,
“Look,
Joab
has a portion
of field adjacent
to mine and he has some
barley
there.
Go
and set
it on fire.”
So Absalom’s
servants
set
Joab’s portion
of the field on fire.
\par }{\PP \VS{31}Then Joab
got
up and came
to
Absalom’s
house.
He said
to him,
“Why
did your servants
set
my portion
of field on fire?”
\VS{32}Absalom
said
to
Joab,
“Look,
I sent
a message to
you saying,
‘Come
here
so that I can send
you to
the king
with this message: “Why have
I
come
from Geshur? It would be better
for me if I
were still
there.” ’
Let me now
see
the face
of the king.
If
I
am
at fault,
let him put me to death!”
\par }{\PP \VS{33}So Joab
went
to
the king
and informed
him. The king summoned
Absalom,
and he came
to
the king.
Absalom
bowed down
before
the king
with his face
toward the ground
and the king
kissed him.

\par }\Chap{15}{\PP \VerseOne{1}Some time
later
Absalom
managed to acquire a chariot
and horses,
as well as fifty
men
to serve as his royal guard.
\VS{2}Now Absalom
used to get up early
and stand
beside
the road
that led to the city gate.
Whenever
anyone
came
by who had
a complaint
to bring
to
the king
for arbitration,
Absalom
would call
out to
him, “What
city
are you
from?” The person would answer,
“I, your servant,
am from one
of the tribes
of Israel.”
\VS{3}Absalom
would then say
to
him, “Look,
your claims are legitimate
and appropriate.
But there is no
representative
of the king
who will listen to you.”
\VS{4}Absalom
would then say,
“If only
they would make
me a judge
in the land! Then everyone
who
had
a judicial
complaint
could come
to
me and I would make sure he receives a just settlement.”
\par }{\PP \VS{5}When
someone
approached
to bow
before him, Absalom would extend
his hand
and embrace
him and kiss him.
\VS{6}Absalom
acted
this
way
toward everyone
in Israel
who
came
to
the king
for justice.
In this way
Absalom
won
the loyalty
of the citizens
of Israel.
\par }{\PP \VS{7}After
four
years
Absalom
said
to
the king,
“Let
me go
and repay
my vow
that
I made
to the
{\ND{Lord}}
while I was in Hebron.
\VS{8}For
I
made
this vow
when I was living
in Geshur
in Aram: ‘If
the {\ND{Lord}}
really
does allow me
to return
to Jerusalem,
I will serve
the {\ND{Lord}}.’ ”
\VS{9}The king
replied
to him, “Go
in peace.”
So Absalom got up
and went
to Hebron.
\par }{\PP \VS{10}Then Absalom
sent
spies
through all
the tribes
of Israel
who said,
“When you hear
the sound
of the horn,
you may assume that Absalom
rules
in Hebron.”
\VS{11}Now two hundred
men
had gone with
Absalom
from Jerusalem.
Since they were invited,
they went
naively
and were unaware
of what Absalom was planning.
\VS{12}While he was offering
sacrifices,
Absalom
sent
for Ahithophel
the Gilonite,
David’s
adviser,
to come
from his city,
Giloh.
The conspiracy
was gaining momentum,
and the people
were starting to side
with
Absalom.
\par }{\SH David Flees from Jerusalem
\par }{\PP \VS{13}Then a messenger came
to
David
and reported,
“The men
of Israel
are loyal
to Absalom!”
\VS{14}So
David
said
to all
his servants
who
were with
him in Jerusalem, “Come on! Let’s
escape! Otherwise
no
one will be
delivered
from Absalom! Go
immediately,
or else
he will quickly
overtake
us
and bring disaster
on us and kill
the city’s residents
with the sword.”
\VS{15}The king’s
servants
replied
to
the king,
“We
will do whatever
our lord
the king decides.”
\par }{\PP \VS{16}So
the king
and all
the members
of his royal
court set out
on foot,
though the king
left
behind ten
concubines
to attend
to the palace.
\VS{17}The king
and all
the people
set out
on foot,
pausing at a spot some distance away.
\VS{18}All
his servants
were leaving
with him, along
with all
the Kerethites,
all
the Pelethites,
and all
the Gittites
– some six
hundred
men
who had
come
on foot
from Gath.
They were leaving
with the king.
\par }{\PP \VS{19}Then the king
said
to Ittai
the Gittite,
“Why
should
you
come
with
us? Go back
and stay
with
the new king,
for
you
are a foreigner
and an exile
from your
own country.
\VS{20}It seems like you arrived
just yesterday.
Today
should I make you wander around
by going
with
us? I
go
where
I
must go.
But as for you, go back
and take
your men
with
you. May genuine loyal love protect you!”
\par }{\PP \VS{21}But Ittai
replied
to the
king,
“As surely
as the
{\ND{Lord}}
lives
and as my lord
the king
lives, wherever
my lord
the king
is, whether
dead
or
alive,
there
I will be
as well!”
\VS{22}So David
said
to
Ittai,
“Come along
then.” So Ittai
the Gittite
went along,
accompanied
by all
his men
and all
the dependents
who
were with him.
\par }{\PP \VS{23}All
the land
was weeping
loudly
as all
these people
were leaving.
As the king
was crossing
over the Kidron
Valley,
all
the people
were leaving
on
the road
that leads to the desert.
\VS{24}Zadok
and all
the Levites
who were with
him were carrying
the ark
of the covenant
of God.
When they positioned
the
ark
of God,
Abiathar
offered
sacrifices until
all
the people
had finished leaving
the city.
\par }{\PP \VS{25}Then the king
said
to Zadok,
“Take the
ark
of God
back to the city.
If
I find
favor
in the
{\ND{Lord}}’s
sight
he will bring me back
and enable me to see
both it and his dwelling place again.
\VS{26}However, if
he should
say,
‘I
do not
take pleasure
in you,’ then he will deal with me in a way that
he considers
appropriate.”
\par }{\PP \VS{27}The king
said
to
Zadok
the priest,
“Are you
a seer? Go back
to the city
in peace! Your son
Ahimaaz
and Abiathar’s son
Jonathan
may go with
you and Abiathar.
\VS{28}Look,
I
will be waiting
at the fords
of the desert
until
word
from you reaches me.”
\VS{29}So
Zadok
and Abiathar
took the ark
of God
back to Jerusalem
and remained
there.
\par }{\PP \VS{30}As David
was going up
the Mount
of Olives,
he was weeping
as he went;
his head
was covered
and his feet were bare.
All
the people
who
were with
him also had their heads
covered
and were weeping
as they went
up.
\VS{31}Now David
had been told,
“Ahithophel
has sided with the conspirators
who are with
Absalom.
So David
prayed, “Make the advice
of Ahithophel
foolish,
O
{\ND{Lord}}!”
\par }{\PP \VS{32}When
David
reached
the summit,
where
he used to worship
God,
Hushai
the Arkite
met
him with his clothes
torn
and dirt
on
his head.
\VS{33}David
said
to him, “If
you leave
with
me you will be
a burden to me.
\VS{34}But you will be able to counter
the advice
of Ahithophel
if
you go back
to the city
and say
to Absalom,
‘I
will be your servant,
O king! Previously
I
was your father’s
servant,
and now
I
will be your servant.’
\VS{35}Zadok
and Abiathar
the priests
will be
there
with
you. Everything
you hear
in the king’s
palace
you must
tell
Zadok
and Abiathar
the priests.
\VS{36}Furthermore,
their two
sons
are there
with
them, Zadok’s
son Ahimaaz
and Abiathar’s
son Jonathan.
You must send
them to me
with any
information
you hear.”
\par }{\PP \VS{37}So David’s
friend
Hushai
arrived
in the city,
just as Absalom
was entering
Jerusalem.

\par }\Chap{16}{\PP \VerseOne{1}When
David
had gone
a short
way beyond the summit,
Ziba
the servant
of Mephibosheth
was there to meet
him. He had a couple
of donkeys
that were saddled,
and on
them were two hundred
loaves of bread,
a hundred
raisin cakes,
a hundred
baskets of summer fruit,
and a container
of wine.
\par }{\PP \VS{2}The king
asked
Ziba,
“Why
did you bring these
things?” Ziba
replied,
“The donkeys
are for the king’s
family
to ride
on, the loaves of bread
and the summer fruit
are for the attendants
to eat,
and the wine
is for those who get exhausted
in the desert.”
\VS{3}The king
asked,
“Where
is your master’s
grandson?” Ziba
replied
to
the king,
“He remains in
Jerusalem,
for
he said,
‘Today
the house
of Israel
will give back
to me my grandfather’s
kingdom.’ ”
\VS{4}The king
said
to Ziba,
“Everything
that
was Mephibosheth’s
now belongs to you.” Ziba
replied,
“I bow
before you. May I find
favor
in your sight,
my lord
the king.”
\par }{\SH Shimei Curses David and His Men
\par }{\PP \VS{5}Then King
David
reached
Bahurim.
There
a man
from Saul’s
extended
family
named
Shimei
son
of Gera
came
out,
yelling curses as he approached.
\VS{6}He threw
stones
at David
and all
of King
David’s
servants,
as well as all
the people
and the soldiers
who were on his right
and on his left.
\VS{7}As he yelled
curses,
Shimei
said,
“Leave! Leave! You man
of bloodshed,
you wicked
man!
\VS{8}The

{\ND{Lord}}
has punished
you for all
the spilled blood
of the house
of Saul,
in whose
place
you
rule.
Now
the {\ND{Lord}}
has given
the kingdom
into the hand
of your son
Absalom.
Disaster
has overtaken
you,
for
you are
a man
of bloodshed!”
\par }{\PP \VS{9}Then Abishai
son
of Zeruiah
said
to
the king,
“Why
should this
dead
dog
curse
my lord
the king? Let
me go over
and cut off
his head!”
\VS{10}But the king
said,
“What
do we have in common, you sons
of Zeruiah? If
he curses
because
the {\ND{Lord}}
has said
to him, ‘Curse
David!’, who can
say
to him, ‘Why
have you done
this?’ ”
\VS{11}Then David
said
to
Abishai
and to
all
his servants,
“My own son,
my very own flesh and blood,
is trying
to take my life.
So also
now
this
Benjaminite! Leave
him alone so that he can curse,
for
the {\ND{Lord}}
has spoken to him.
\VS{12}Perhaps
the {\ND{Lord}}
will notice
my affliction
and this
day
grant me good
in place
of his curse.”
\par }{\PP \VS{13}So
David
and his men
went on
their way.
But Shimei
kept going
along
the side
of the hill
opposite him, yelling
curses
as
he threw stones
and dirt at them.
\VS{14}The king
and all
the people
who
were with
him arrived exhausted
at their destination, where
David refreshed himself.
\par }{\SH The Advice of Ahithophel
\par }{\PP \VS{15}Now when Absalom
and all
the men
of Israel
arrived
in Jerusalem,
Ahithophel
was with him.
\VS{16}When
David’s
friend
Hushai
the Arkite
came
to
Absalom,
Hushai
said
to
him, “Long live
the king! Long live
the king!”
\par }{\PP \VS{17}Absalom
said
to
Hushai,
“Do you call this
loyalty
to your friend? Why
didn’t
you go
with
your friend?”
\VS{18}Hushai
replied
to
Absalom,
“No,
I will be
loyal to the one whom
the {\ND{Lord}}, these
people,
and all
the men
of Israel
have chosen.
\VS{19}Moreover,
whom
should I
serve? Should it not
be his son? Just
as I served
your father,
so
I will serve you.”
\par }{\PP \VS{20}Then Absalom
said
to Ahithophel,
“Give
us your advice.
What
should we do?”
\VS{21}Ahithophel
replied
to
Absalom,
“Have sex
with
your father’s
concubines
whom
he left
to care
for the palace.
All
Israel
will hear
that
you have made yourself repulsive
to your father.
Then your followers will be motivated
to support you.”
\VS{22}So they pitched
a tent
for Absalom
on
the roof,
and Absalom
had sex with his father’s
concubines
in the sight
of all
Israel.
\par }{\PP \VS{23}In those
days
Ahithophel’s
advice
was considered as
valuable
as
a prophetic
revelation. Both
David
and Absalom
highly
regarded
the advice
of Ahithophel.

\par }\Chap{17}{\PP \VerseOne{1}Ahithophel
said
to
Absalom,
“Let me pick
out twelve
thousand
men.
Then
I will go
and pursue
David
this very night.
\VS{2}When I catch up
with him he
will be exhausted
and worn
out. I will rout
him, and the entire
army
that
is with
him will flee.
I will kill
only
the king
\VS{3}and will bring
the entire
army
back
to
you. In exchange for the life of the man
you
are seeking,
you will get back everyone.
The entire
army
will
return unharmed.”
\par }{\PP \VS{4}This seemed
like a good idea
to Absalom
and to all
the leaders
of Israel.
\VS{5}But Absalom
said,
“Call
for Hushai
the Arkite,
and let’s hear
what
he has to say.”
\VS{6}So Hushai
came
to
Absalom.
Absalom
said
to
him, “Here
is what
Ahithophel
has advised.
Should we follow
his advice? If
not,
what would you
recommend?”
\par }{\PP \VS{7}Hushai
replied
to
Absalom,
“Ahithophel’s
advice
is not
sound this
time.”
\VS{8}Hushai
went on to say,
“You
know
your father
and his men
– they are soldiers
and are as dangerous
as a bear
out in the wild
that has been robbed of her cubs.
Your father
is an experienced
soldier;
he will not
stay overnight
with
the army.
\VS{9}At this
very moment
he is
hiding
out in one
of the caves
or
in some other
similar
place.
If it should turn out that he attacks our troops
first,
whoever hears
about
it will say,
‘Absalom’s
army
has
been
slaughtered!’
\VS{10}If that
happens even
the bravest soldier
– one who
is lion-hearted
– will virtually
melt away.
For
all
Israel
knows
that
your father
is a warrior
and that those who
are with
him are brave.
\VS{11}My advice
therefore
is this: Let all
Israel
from
Dan
to
Beer Sheba
– in number
like
the sand
by
the sea! – be mustered to you, and you lead them personally into battle.
\VS{12}We will
come
against
him
wherever
he happens to be found.
We will descend
on
him like the dew
falls
on
the ground.
Neither he nor
any
of the men
who are with him will be spared alive – not one of them!
\VS{13}If
he regroups
in
a city,
all
Israel
will take up
ropes
to
that city
and drag
it down
to
the valley,
so that
not
a single pebble
will be left
there!”
\par }{\PP \VS{14}Then Absalom
and all
the men
of Israel
said, “The advice
of Hushai
the Arkite
sounds better
than the advice
of Ahithophel.”
Now the
{\ND{Lord}}
had
decided to frustrate
the sound advice
of Ahithophel,
so that
the {\ND{Lord}}
could bring
disaster
on Absalom.
\par }{\PP \VS{15}Then Hushai
reported to
Zadok
and Abiathar
the priests,
“Here is what
Ahithophel
has advised Absalom
and the leaders
of Israel
to do, and here is what
I have
advised.
\VS{16}Now
send
word quickly
to David
and warn
him, “Don’t
spend
the night at the fords
of the desert
tonight.
Instead, be sure you cross
over, or else
the king
and everyone
who
is with him may
be overwhelmed.”
\par }{\PP \VS{17}Now Jonathan
and Ahimaaz
were staying
in En Rogel.
A female servant
would go
and inform
them, and they
would then go
and inform
King
David.
It was not
advisable
for them to be seen
going
into the city.
\VS{18}But a young man
saw
them on one occasion and informed
Absalom.
So the two
of them quickly
departed
and went
to
the house
of a man
in Bahurim.
There was a well
in his courtyard,
and they got down
in it.
\VS{19}His wife
then took
the covering
and spread
it over
the top of the well
and scattered
some grain
over
it. No
one was aware
of what
she had done.
\par }{\PP \VS{20}When the servants
of Absalom
approached
the woman
at her home,
they asked,
“Where
are Ahimaaz
and Jonathan?” The woman
replied
to them, “They crossed over
the stream.”
Absalom’s men searched
but did not
find
them, so they returned
to Jerusalem.
\par }{\PP \VS{21}After
the men had left,
Ahimaaz and Jonathan climbed
out of the well.
Then they left
and informed
King
David.
They advised
David,
“Get
up and cross
the
stream
quickly,
for
Ahithophel
has devised
a plan to catch you.”
\VS{22}So David
and all
the people
who
were with
him got up
and crossed
the Jordan River.
By
dawn
there was not
one
person left
who had
not
crossed
the Jordan.
\par }{\PP \VS{23}When
Ahithophel
realized
that
his advice
had
not
been followed, he
saddled
his donkey
and
returned
to
his house
in
his hometown.
After setting
his household
in order, he hanged
himself. So he died
and was buried
in the grave
of his father.
\par }{\PP \VS{24}Meanwhile David
had gone
to Mahanaim,
while Absalom
and all
the men
of Israel
had crossed
the Jordan River.
\VS{25}Absalom
had made
Amasa
general in command of the army
in place of
Joab.
(Now Amasa
was the son
of an Israelite
man
named
Jether,
who had
married
Abigail
the daughter
of Nahash
and sister
of Zeruiah,
Joab’s
mother.)
\VS{26}The army of Israel
and Absalom
camped
in the land
of Gilead.
\par }{\PP \VS{27}When
David
came
to Mahanaim,
Shobi
the son
of Nahash
from Rabbah
of the Ammonites,
Makir
the son
of Ammiel
from Lo Debar,
and Barzillai
the Gileadite
from Rogelim
\VS{28}brought bedding,
basins,
and pottery
utensils.
They also brought food for David and all who were with him, including wheat,
barley,
flour,
roasted grain,
beans,
lentils,
\VS{29}honey,
curds,
flocks,
and cheese.
For
they said,
“The people
are no doubt hungry,
tired,
and thirsty
there in the desert.”

\par }\Chap{18}{\PP \VerseOne{1}David
assembled
the army
that
was with
him. He appointed
leaders
of thousands
and leaders
of hundreds.
\VS{2}David
then sent
out the army
– a third
under the leadership
of Joab,
a third
under the leadership
of Joab’s
brother
Abishai
son
of Zeruiah,
and a third
under the leadership
of Ittai
the Gittite.
The king
said
to
the troops,
“I
too
will indeed march
out
with you.”
\par }{\PP \VS{3}But the soldiers
replied, “You should not
do this! For
if
we should have to
make
a rapid
retreat,
they won’t be
too concerned
about us. Even if
half
of us
should
die,
they won’t
be too concerned
about us. But you are like
ten
thousand
of us! So it is better
if
you remain in the city
for support.”
\VS{4}Then the king
said
to them,
“I will do whatever
seems
best
to you.”
\par }{\PP So
the king
stayed
beside
the city gate,
while all
the army
marched
out by hundreds
and by thousands.
\VS{5}The king
gave this order
to Joab,
Abishai,
and Ittai: “For my sake deal gently
with the young man
Absalom.”
Now the entire
army
was listening
when the king
gave
all
the leaders
this order
concerning
Absalom.
\par }{\PP \VS{6}Then the army
marched
out to the field
to fight
against Israel.
The battle
took place in the forest
of Ephraim.
\VS{7}The army
of Israel
was defeated
there
by David’s
men. The slaughter there
was great
that day
– 20,000
soldiers were killed.
\VS{8}The battle
there
was spread
out over
the whole
area,
and the forest
consumed
more
soldiers
than the sword
devoured
that day.
\par }{\PP \VS{9}Then Absalom
happened
to come across
David’s
men.
Now as Absalom
was riding
on
his mule,
it
went
under
the branches
of a large
oak tree.
His head
got caught in the oak
and he was suspended
in
midair,
while the mule
he had
been riding kept going.
\par }{\PP \VS{10}When one
of the men
saw
this, he reported
it to Joab
saying,
“I
saw
Absalom
hanging
in an oak tree.
\VS{11}Joab
replied
to the man
who was telling
him this, “What! You saw
this? Why
didn’t
you strike
him down
right on
the spot? I would have given
you ten
pieces of silver
and a commemorative belt!”
\par }{\PP \VS{12}The man
replied
to
Joab,
“Even if
I
were receiving
a thousand
pieces of silver,
I would not
strike
the king’s
son! In our very presence
the king
gave this order
to you and Abishai
and Ittai,
‘Protect
the young man
Absalom for my sake.’
\VS{13}If
I had acted
at risk
of my own life –
and nothing
is hidden
from
the king! – you would have abandoned me.”
\par }{\PP \VS{14}Joab
replied,
“I will not
wait
around like this
for you!” He took
three
spears
in his hand
and thrust
them into the middle
of Absalom
while he was still
alive
in the middle
of the oak tree.
\VS{15}Then
ten
soldiers
who were Joab’s
armor
bearers
struck
Absalom
and finished him off.
\par }{\PP \VS{16}Then
Joab
blew
the trumpet
and the army
turned back
from chasing
Israel,
for
Joab
had called for the army
to halt.
\VS{17}They took
Absalom,
threw
him into a large
pit
in the forest,
and stacked a huge pile
of stones
over
him. In the meantime
all
the Israelite
soldiers
fled
to their homes.
\par }{\PP \VS{18}Prior to this Absalom
had set up
a monument
and dedicated it to himself in the
King’s
Valley,
reasoning
“I have no
son
who will carry on
my name.”
He
named
the monument
after himself, and to this
day
it is known
as Absalom’s
Memorial.
\par }{\SH David Learns of Absalom’s Death
\par }{\PP \VS{19}Then Ahimaaz
the son
of Zadok
said,
“Let
me run
and give
the king
the good news
that
the {\ND{Lord}}
has vindicated
him before his enemies.”
\VS{20}But Joab
said
to him, “You will not
be a bearer
of good news
today.
You
will bear good news
some other
day,
but not
today,
for
the king’s
son
is dead.”
\par }{\PP \VS{21}Then Joab
said
to the Cushite,
“Go
and tell
the king
what
you have seen.”
After bowing
to Joab,
the Cushite
ran off.
\VS{22}Ahimaaz
the son
of Zadok
again
spoke
to
Joab,
“Whatever
happens,
let
me
go after
the Cushite.”
But Joab
said,
“Why
is
it that you
want
to go, my son? You have no
good news
that will bring you a reward.”
\VS{23}But
he said, “Whatever
happens,
I want to go!” So Joab said
to him, “Then
go!” So Ahimaaz
ran
by the way
of
the Jordan plain,
and he passed
the
Cushite.
\par }{\PP \VS{24}Now David
was sitting
between
the inner and outer gates,
and the watchman
went
up to
the roof
over the gate
at the wall.
When he looked,
he saw
a man
running
by himself.
\VS{25}So the watchman
called
out and informed
the king.
The king
said,
“If
he is by himself,
he brings good news.”
The runner
came ever closer.
\par }{\PP \VS{26}Then the watchman
saw
another
man
running.
The watchman
called
out to
the gatekeeper,
“There
is another man
running
by himself.”
The king
said,
“This
one also
is bringing good news.”
\VS{27}The watchman
said,
“It appears
to me that the first
runner
is Ahimaaz
son
of Zadok.”
The king
said,
“He is a good
man,
and he comes
with good
news.”
\par }{\PP \VS{28}Then Ahimaaz
called
out and said
to
the king,
“Greetings!” He bowed
down before the king
with his face
toward the ground
and said,
“May the
{\ND{Lord}}
your God
be praised
because
he has defeated
the
men
who
opposed
my lord
the king!”
\par }{\PP \VS{29}The king
replied,
“How
is the young man
Absalom?” Ahimaaz
replied,
“I saw
a great
deal of confusion
when
Joab
was sending
the king’s
servant
and me, your servant,
but I don’t know
what it was all about.”
\VS{30}The king
said,
“Turn
aside
and take
your place here.”
So he turned
aside and waited.
\par }{\PP \VS{31}Then
the Cushite
arrived
and said, “May
my lord
the king
now receive the good news! The
{\ND{Lord}}
has vindicated
you today
and delivered you from the hand
of all
who have
rebelled
against you!”
\VS{32}The king
asked
the Cushite,
“How
is the young man
Absalom?” The Cushite
replied,
“May the enemies
of my lord
the king
and all
who have
plotted
against you be
like that young man!”
\par }{\PP \VS{33}The king
then became very upset.
He went up
to the upper
room over the gate
and wept.
As he went
he said,
“My son,
Absalom! My son,
my son,
Absalom! If only
I
could have died
in your place! Absalom,
my son,
my son!”

\par }\Chap{19}{\PP \VerseOne{1}Joab
was told,
“The king
is weeping
and mourning
over
Absalom.”
\VS{2}So the victory
of that day
was
turned to mourning
as far as all
the people
were concerned. For
the people
heard
on that day,
“The king
is
grieved
over
his son.”
\VS{3}That day
the people
stole
away to go
to the city
the way people
who are embarrassed
steal
away
in fleeing
from battle.
\VS{4}The king
covered
his face
and cried
out loudly, “My son,
Absalom! Absalom,
my son,
my son!”
\par }{\PP \VS{5}So Joab
visited
the king
at
his home.
He said,
“Today
you have embarrassed
all
your servants
who have saved
your life
this day,
as well as the lives
of your sons,
your daughters,
your wives,
and your concubines.
\VS{6}You seem to love
your enemies
and hate
your friends! For
you have as much as declared
today
that
leaders
and servants
don’t
matter to you. I realize
now
that
if
Absalom
were alive
and all
of us were dead
today,
it would be all right with you.
\VS{7}So
get up
now
and go
out
and give
some encouragement
to
your servants.
For
I swear
by the
{\ND{Lord}}
that if
you don’t
go out
there, not
a single man
will stay
here with
you tonight! This disaster
will be worse for you than any
disaster
that
has overtaken
you from your youth
right to
the present time!”
\par }{\PP \VS{8}So the king
got
up and sat
at the city gate.
When all
the people
were informed
that the king
was sitting
at the city gate,
they all
came
before him.
\par }{\SH David Goes Back to Jerusalem
\par }{\PP But the Israelite soldiers had all fled to their own homes.
\VS{9}All
the people
throughout all
the tribes
of Israel
were arguing
among themselves saying,
“The king
delivered
us from the hand
of our enemies.
He
rescued
us from
the hand
of the Philistines,
but now
he has fled
from
the land
because
of Absalom.
\VS{10}But Absalom,
whom
we anointed
as our king, has died
in battle.
So now
why
do you
hesitate
to bring
the king back?”
\par }{\PP \VS{11}Then King
David
sent
a message to
Zadok
and Abiathar
the priests
saying,
“Tell
the elders
of Judah,
‘Why
should you delay any further
in bringing
the
king
back
to
his palace,
when
everything
Israel
is saying has come
to
the king’s attention.
\VS{12}You
are my brothers
– my very
own flesh
and blood! Why
should you delay any further
in bringing
the king
back?’
\VS{13}Say
to Amasa,
‘Are you not
my
flesh
and blood? God
will punish me severely,
if
from this time on
you are not
the commander
of my army
in place
of Joab!’ ”
\par }{\PP \VS{14}He won
over the
hearts
of all
the men
of Judah
as though they were one
man.
Then they sent
word to
the king
saying, “Return,
you
and all
your servants as well.”
\VS{15}So the king
returned
and came
to
the Jordan River.
\par }{\PP Now the people of Judah
had come
to Gilgal
to meet
the king
and to help
him cross
the
Jordan.
\VS{16}Shimei
son
of Gera
the Benjaminite
from Bahurim
came down
quickly
with
the men
of Judah
to meet
King
David.
\VS{17}There were a thousand
men
from Benjamin
with him, along with
Ziba
the servant
of Saul’s
household,
and with him
his fifteen
sons
and twenty
servants.
They hurriedly
crossed the Jordan
within sight
of the king.
\VS{18}They crossed
at the ford
in order to help the
king’s
household
cross
and to do
whatever
he thought appropriate.
\par }{\PP Now after he had crossed
the Jordan,
Shimei
son
of Gera
threw
himself down before
the king.
\VS{19}He said
to
the king,
“Don’t
think badly
of me, my lord,
and don’t
recall
the sin
of your servant
on the day
when
you, my lord
the king,
left
Jerusalem! Please don’t
call it
to
mind!
\VS{20}For
I, your servant,
know
that
I
sinned,
and I have come
today
as the first
of all
the house
of Joseph
to come down
to meet
my lord
the king.”
\par }{\PP \VS{21}Abishai
son
of Zeruiah
replied,
“For
this
should not
Shimei
be put to death? After all, he cursed
the
{\ND{Lord}}’s
anointed!”
\VS{22}But David
said,
“What
do we have in common, you sons
of Zeruiah? You are like my enemy
today! Should anyone
be put to death
in Israel
today? Don’t
you realize
that
today
I
am king
over
Israel?”
\VS{23}The king
said
to
Shimei,
“You won’t
die.”
The king
vowed an oath concerning this.
\par }{\PP \VS{24}Now Mephibosheth,
Saul’s
grandson,
came down
to meet
the king.
From
the day
the king
had left
until
the day
he safely
returned,
Mephibosheth had not
cared for his feet
nor trimmed
his mustache
nor
washed
his clothes.
\par }{\PP \VS{25}When
he came
from Jerusalem
to meet
the king,
the king
asked
him, “Why
didn’t
you go
with
me, Mephibosheth?”
\VS{26}He replied,
“My lord
the king,
my servant
deceived
me! I said,
‘Let me get my donkey
saddled
so that I can ride
on
it and go
with
the king,’
for
I
am lame.
\VS{27}But my servant
has slandered
me to
my lord
the king.
But my lord
the king
is like an angel
of God.
Do
whatever
seems appropriate to you.
\VS{28}After
all, there was
no
one in the entire
house
of my grandfather
who did not deserve
death
from my lord
the king.
But instead you allowed me
to eat
at your own table! What
further
claim do I have
to
ask the king for anything?”
\par }{\PP \VS{29}Then the king
replied
to him, “Why
should you continue
speaking
like this? You
and Ziba
will inherit
the field together.”
\VS{30}Mephibosheth
said
to
the king,
“Let him have
the whole
thing! My lord
the king
has returned safely
to
his house!”
\par }{\PP \VS{31}Now when Barzillai
the Gileadite
had come down
from Rogelim,
he crossed
the Jordan
with
the king
so he could send him on his way from there.
\VS{32}But Barzillai
was very
old
– eighty
years old, in fact – and he had taken care of the king when he stayed in Mahanaim, for he was a very rich man.
\VS{33}So the king
said
to
Barzillai,
“Cross
over with
me, and I
will take care
of you while you are
with me
in Jerusalem.”
\par }{\PP \VS{34}Barzillai
replied
to
the king,
“How
many days
do I have left to my life,
that
I should go up
with
the king
to Jerusalem?
\VS{35}I am presently
eighty
years
old. Am
I able to discern
good
and bad? Can
I
taste
what
I eat
and drink? Am I still
able to hear
the voices
of male and female singers? Why
should I
continue
to be a burden
to
my lord
the king?
\VS{36}I
will cross
the Jordan
with
the king
and go a short
distance.
Why
should the king
reward
me in this way?
\VS{37}Let me
return
so that I may
die
in my own city
near
the grave
of my father
and my mother.
But look,
here is your servant
Kimham.
Let him cross over
with
my lord
the king.
Do
for him whatever
seems appropriate to you.”
\par }{\PP \VS{38}The king
replied,
“Kimham
will cross over
with
me, and I
will do
for him whatever I deem
appropriate.
And whatever
you choose,
I will do for you.”
\par }{\PP \VS{39}So all
the people
crossed
the Jordan,
as did the king.
After the king
had kissed
him and blessed
him, Barzillai
returned
to his home.
\VS{40}When the king
crossed
over to Gilgal,
Kimham
crossed
over with
him. Now all
the soldiers
of Judah
along with half
of the soldiers
of Israel
had helped the king
cross over.
\par }{\PP \VS{41}Then
all
the men
of Israel
began coming
to
the king.
They asked
the king,
“Why
did our brothers,
the men
of Judah,
sneak
the king away
and help the
king
and his household
cross
the
Jordan
– and not only him but all
of David’s
men
as well?”
\par }{\PP \VS{42}All
the men
of Judah
replied
to the men
of Israel,
“Because
the king
is our close relative! Why
are you so upset
about this? Have we eaten
at the king’s
expense? Or
have we misappropriated anything for our own use?”
\VS{43}The men
of Israel
replied
to the
men
of Judah,
“We have ten
shares
in the king,
and we have a greater
claim
on David
than
you do! Why
do you want to curse
us? Weren’t
we the
first
to suggest bringing back
our king?” But the comments
of the men
of Judah
were
more severe
than those of the
men
of Israel.

\par }\Chap{20}{\PP \VerseOne{1}Now a wicked
man
named
Sheba
son
of Bicri,
a Benjaminite,
happened
to be there.
He blew
the trumpet
and said,
\par }{\Q “We have no
share
in David;
\par }{\Q we have no
inheritance
in this son
of Jesse!
\par }{\Q Every man
go home,
O Israel!”
\par }{\PP \VS{2}So
all
the men
of Israel
deserted
David
and followed
Sheba
son
of Bicri.
But the men
of Judah
stuck
by their king
all the way from
the Jordan River
to Jerusalem.
\par }{\PP \VS{3}Then David
went
to
his palace
in Jerusalem.
The king
took
the
ten
concubines
he had
left
to care
for the palace
and placed
them under confinement.
Though he provided
for their needs,
he did not
have sexual relations
with
them. They remained
in confinement
until
the day
they died,
living out the rest of their lives
as widows.
\par }{\PP \VS{4}Then the king
said to
Amasa,
“Call
the men
of Judah
together for
me in three
days,
and you
be present
here with them too.”
\VS{5}So Amasa
went out
to call
Judah
together. But in doing so he took longer
than
the time
that
the king had allotted him.
\par }{\PP \VS{6}Then David
said
to
Abishai,
“Now
Sheba
son
of Bicri
will cause greater disaster
for us than
Absalom
did! Take
your
lord’s
servants
and pursue
him. Otherwise
he will secure
fortified
cities
for himself and get away from us.”
\VS{7}So
Joab’s
men,
accompanied
by the Kerethites,
the Pelethites,
and all
the warriors,
left
Jerusalem
to pursue
Sheba
son
of Bicri.
\par }{\PP \VS{8}When they
were near
the big
rock
that
is in Gibeon,
Amasa
came
to them.
Now Joab
was dressed
in military
attire
and had a dagger
in
its sheath
belted
to his waist.
When he
advanced, it fell
out.
\par }{\PP \VS{9}Joab
said
to Amasa,
“How
are you,
my brother?” With his right hand
Joab
took hold
of Amasa’s
beard
as if to greet him with a kiss.
\VS{10}Amasa
did not
protect
himself from the knife
in Joab’s other hand,
and Joab
stabbed
him in the abdomen,
causing Amasa’s intestines
to
spill out
on the ground.
There was no
need to stab him again;
the first blow was fatal.
Then Joab
and his brother
Abishai
pursued
Sheba
son
of Bicri.
\par }{\PP \VS{11}One
of Joab’s
soldiers
who stood
over
Amasa said,
“Whoever
is for Joab
and whoever is
for David,
follow
Joab!”
\VS{12}Amasa
was squirming
in his own blood
in the middle
of the path,
and this man
had noticed
that
all
the
soldiers
stopped.
Having noticed
that
everyone
who came
across
Amasa stopped,
the man pulled
him
away from
the path
and into the field
and threw
a garment
over him.
\VS{13}Once he had removed
Amasa
from
the path,
everyone
followed
Joab
to pursue
Sheba
son
of Bicri.
\par }{\PP \VS{14}Sheba traveled
through all
the tribes
of Israel
to Abel
of Beth Maacah
and all
the Berite
region. When they had assembled,
they too
joined him.
\VS{15}So Joab’s men came
and laid siege against
him in
Abel
of Beth Maacah.
They prepared
a siege ramp
outside
the city
which stood
against its outer rampart.
As all
of Joab’s
soldiers
were trying to break through
the wall
so that it would collapse,
\VS{16}a wise
woman
called
out from
the city,
“Listen
up! Listen
up! Tell
Joab,
‘Come
near
so that I may speak
to you.’ ”
\par }{\PP \VS{17}When he approached
her,
the woman
asked,
“Are you
Joab?” He replied,
“I am.”
She said
to him, “Listen
to the words
of your servant.”
He said,
“Go ahead. I’m
listening.”
\VS{18}She said,
“In the past
they would
always say,
‘Let them inquire
in Abel,’
and that is how they settled things.
\VS{19}I
represent the peaceful
and the faithful
in Israel.
You
are attempting
to destroy
an important
city
in Israel.
Why
should you swallow up
the
{\ND{Lord}}’s
inheritance?”
\par }{\PP \VS{20}Joab
answered,
“Get serious! I don’t
want to swallow up
or
destroy anything!
\VS{21}That’s
not
the way
things
are. There is a man
from the hill country
of Ephraim
named
Sheba
son
of Bicri.
He has rebelled against King
David.
Give
me just this one man, and I will leave
the city.”
The woman
said
to
Joab,
“This
very minute
his head
will be thrown
over the wall
to you!”
\par }{\PP \VS{22}Then the woman
went
to
all
the people
with her wise advice
and they cut off
Sheba’s
head
and threw
it out
to
Joab.
Joab blew the trumpet, and his men dispersed from the city, each going to his own home. Joab returned to the king in Jerusalem.
\par }{\PP \VS{23}Now Joab
was the general
in command
of all
the army
of Israel.
Benaiah
the son
of Jehoida
was over
the Kerethites
and the Perethites.
\VS{24}Adoniram
was supervisor
of the work crews.
Jehoshaphat
son
of Ahilud
was the secretary.
\VS{25}Sheva
was the scribe,
and Zadok
and Abiathar
were the priests.
\VS{26}Ira
the Jairite
was
David’s
personal priest.

\par }\Chap{21}{\PP \VerseOne{1}During
David’s
reign there was
a famine
for three
consecutive years.
So David
inquired
of the {\ND{Lord}}. The
{\ND{Lord}}
said,
“It is because of Saul
and his bloodstained
family,
because he murdered
the Gibeonites.”
\par }{\PP \VS{2}So the king
summoned
the Gibeonites
and spoke
with them.
(Now the Gibeonites
were not
descendants
of Israel;
they
were a remnant of
the Amorites.
The Israelites
had
made a promise
to them,
but
Saul
tried
to kill
them because of his zeal
for the people of Israel
and Judah.)
\VS{3}David
said
to
the Gibeonites,
“What
can I do
for you, and how
can I make amends
so that you will bless
the
{\ND{Lord}}’s
inheritance?”
\par }{\PP \VS{4}The Gibeonites
said
to him, “We have no
claim to silver
or gold
from Saul
or from his family,
nor would we be justified
in putting to death
anyone
in Israel.”
David asked, “What
then are you
asking me to do for you?”
\VS{5}They replied
to
the king,
“As
for this man
who
exterminated
us and who
schemed
against
us so that we were destroyed
and left without status
throughout all
the borders
of Israel –
\VS{6}let seven
of his male
descendants
be turned over
to us, and we will execute
them before the
{\ND{Lord}}
in Gibeah
of Saul,
who was the
{\ND{Lord}}’s
chosen one.”
The king
replied,
“I
will turn
them over.”
\par }{\PP \VS{7}The king
had mercy
on
Mephibosheth
son
of Jonathan,
the son
of Saul,
in light of the
{\ND{Lord}}’s
oath
that had
been taken between
David
and Jonathan
son
of Saul.
\VS{8}So the king
took
Armoni
and Mephibosheth,
the two
sons
of Aiah’s
daughter
Rizpah
whom
she had born
to Saul,
and the five
sons
of Saul’s
daughter
Merab
whom
she had born
to Adriel
the son
of Barzillai
the Meholathite.
\VS{9}He
turned them over
to the Gibeonites,
and they executed
them on a hill
before
the {\ND{Lord}}. The seven
of them died
together;
they
were put to death
during
harvest time – during the first days of the beginning of the barley harvest.
\par }{\PP \VS{10}Rizpah
the daughter
of Aiah
took
sackcloth
and spread
it out
for herself on
a rock.
From the beginning
of the harvest
until
the rain
fell
on
them, she did not
allow
the birds
of the air
to feed
on
them by day,
nor the wild
animals
by night.
\VS{11}When David
was told
what Rizpah
daughter
of Aiah,
Saul’s
concubine,
had
done,
\VS{12}he
went
and took
the
bones
of Saul
and of his son
Jonathan
from the
leaders
of Jabesh
Gilead.
(They had secretly taken
them from the
plaza
at Beth Shan.
It was there
that Philistines
publicly exposed
their corpses
after
they
had killed
Saul
at Gilboa.)
\VS{13}David brought
the bones
of Saul
and of Jonathan
his son
from there;
they also gathered up
the bones
of those who had been executed.
\par }{\PP \VS{14}They buried
the
bones
of Saul
and his son
Jonathan
in the land
of Benjamin
at Zela
in the grave
of his father
Kish.
After they had done
everything
that
the king
had commanded,
God
responded to their prayers
for the land.
\par }{\SH Israel Engages in Various Battles with the Philistines
\par }{\PP \VS{15}Another
battle
was fought between the Philistines
and Israel.
So David
went down
with
his soldiers
and fought
the Philistines.
David
became exhausted.
\VS{16}Now Ishbi-Benob,
one of the descendants
of Rapha,
had a spear
that weighed
three
hundred
bronze
shekels,
and he
was armed
with a new
weapon. He had said
that he would kill
David.
\VS{17}But Abishai
the son
of Zeruiah
came to David’s aid, striking
the Philistine
down and killing him.
Then
David’s
men
took an oath
saying,
“You will not
go out
to battle
with
us again! You must not
extinguish
the lamp
of Israel!”
\par }{\PP \VS{18}Later
there was
another
battle
with
the Philistines,
this time
in Gob.
On that occasion
Sibbekai
the Hushathite
killed Saph,
who was one of the descendants
of Rapha.
\VS{19}Yet another
battle
occurred
with
the Philistines
in Gob.
On that occasion Elhanan
the son
of Jair
the Bethlehemite
killed
the brother of Goliath
the Gittite,
the shaft
of whose spear
was like a weaver’s
beam.
\VS{20}Yet another
battle
occurred in Gath.
On that occasion
there was
a large man
who had six
fingers
on each hand
and six
toes
on each foot,
twenty-four
in all! He too
was a descendant
of Rapha.
\VS{21}When he taunted
Israel,
Jonathan,
the son
of David’s
brother
Shimeah,
killed him.
\VS{22}These
four
were the descendants of Rapha
who lived in Gath;
they were killed
by
David
and his soldiers.

\par }\Chap{22}{\PP \VerseOne{1}David
sang
to the
{\ND{Lord}}
the words
of this
song
when
the {\ND{Lord}}
rescued
him
from the power
of all
his enemies,
including
Saul.
\VS{2}He said:
\par }{\Q “The
{\ND{Lord}}
is my high ridge,
my stronghold,
my deliverer.
\par }{\Q \VS{3}My God
is my rocky
summit where I take shelter,
\par }{\Q my shield,
the horn
that saves
me, my stronghold,
\par }{\Q my refuge,
my savior.
You save
me from violence!
\par }{\Q \VS{4}I called
to the
{\ND{Lord}}, who is worthy of praise,
\par }{\Q and I was delivered
from my enemies.
\par }{\Q \VS{5}The waves
of death
engulfed
me;
\par }{\Q the currents
of chaos
overwhelmed me.
\par }{\Q \VS{6}The ropes
of Sheol
tightened
around
me;

\par }{\Q the snares
of death trapped me.
\par }{\Q \VS{7}In my distress
I called
to the
{\ND{Lord}};
\par }{\Q I called
to
my God.
\par }{\Q From his heavenly temple
he heard
my voice;
\par }{\Q he listened to my cry for help.
\par }{\Q \VS{8}The earth
heaved
and shook;
\par }{\Q the foundations
of the sky
trembled.
\par }{\Q They heaved
because
he was angry.
\par }{\Q \VS{9}Smoke
ascended
from his nose;
\par }{\Q fire
devoured
as it came from
his mouth;
\par }{\Q he hurled down fiery coals.
\par }{\Q \VS{10}He made
the sky
sink
as he descended;
\par }{\Q a thick cloud
was under
his feet.
\par }{\Q \VS{11}He mounted
a winged
angel and flew;
\par }{\Q he glided
on
the wings
of the wind.
\par }{\Q \VS{12}He shrouded
himself in darkness,
\par }{\Q in thick
rain
clouds.
\par }{\Q \VS{13}From the brightness
in front
of him
\par }{\Q came coals
of fire.
\par }{\Q \VS{14}The
{\ND{Lord}}
thundered
from
the sky;
\par }{\Q the sovereign One
shouted
loudly.
\par }{\Q \VS{15}He shot
arrows
and scattered
them,

\par }{\Q lightning
and routed them.
\par }{\Q \VS{16}The depths
of the sea
were exposed;
\par }{\Q the inner regions
of the world
were uncovered
\par }{\Q by
the
{\ND{Lord}}’s
battle cry,

\par }{\Q by
the powerful breath
from his nose.
\par }{\Q \VS{17}He reached
down from above
and grabbed
me;

\par }{\Q he pulled
me from the surging
water.
\par }{\Q \VS{18}He rescued me from my strong
enemy,
\par }{\Q from those who hate
me,
\par }{\Q for
they were too
strong for me.
\par }{\Q \VS{19}They confronted
me in my day
of calamity,
\par }{\Q but the
{\ND{Lord}}
helped me.
\par }{\Q \VS{20}He brought
me out
into a wide open place;
\par }{\Q he delivered
me because
he was pleased with me.
\par }{\Q \VS{21}The
{\ND{Lord}}
repaid
me for my godly deeds;
\par }{\Q he rewarded
my blameless
behavior.
\par }{\Q \VS{22}For
I have obeyed
the
{\ND{Lord}}’s
commands;
\par }{\Q I have not
rebelled
against my God.
\par }{\Q \VS{23}For
I am aware
of all
his regulations,
\par }{\Q and I do not
reject
his rules.
\par }{\Q \VS{24}I was
blameless
before him;
\par }{\Q I kept
myself from sinning.
\par }{\Q \VS{25}The
{\ND{Lord}}
rewarded
me for my godly deeds;
\par }{\Q he took notice of my blameless
behavior.
\par }{\Q \VS{26}You
prove to be loyal
to one who is faithful;
\par }{\Q you prove to be trustworthy
to one
who is innocent.
\par }{\Q \VS{27}You prove
to be
reliable to
one who is blameless,
\par }{\Q but
you prove to be deceptive
to one who is perverse.
\par }{\Q \VS{28}You deliver
oppressed
people,
\par }{\Q but you watch
the proud and bring them down.
\par }{\Q \VS{29}Indeed,
you
are my lamp,

{\ND{Lord}}.
\par }{\Q The
{\ND{Lord}}
illumines
the darkness around me.
\par }{\Q \VS{30}Indeed, with
your help I can charge against an army;

\par }{\Q by
my God’s power
I can jump over a
wall.
\par }{\Q \VS{31}The one true God
acts in a faithful manner;
\par }{\Q the
{\ND{Lord}}’s
promise
is reliable;
\par }{\Q he is
a shield
to all
who take shelter in him.
\par }{\Q \VS{32}Indeed,
who
is God
besides
the {\ND{Lord}}?
\par }{\Q Who
is a protector
besides
our God?
\par }{\Q \VS{33}The one true God
is my mighty refuge;
\par }{\Q he removes
the obstacles
in my way.
\par }{\Q \VS{34}He gives me the agility
of a deer;
\par }{\Q he enables me to negotiate
the rugged terrain.
\par }{\Q \VS{35}He trains
my hands
for battle;
\par }{\Q my arms can bend
even the strongest
bow.
\par }{\Q \VS{36}You give
me your protective
shield;
\par }{\Q your willingness
to help enables me
to prevail.
\par }{\Q \VS{37}You widen
my path;
\par }{\Q my feet
do not
slip.
\par }{\Q \VS{38}I chase
my enemies
and destroy
them;
\par }{\Q I do not
turn back
until
I wipe them out.
\par }{\Q \VS{39}I wipe
them out
and beat
them to death;
\par }{\Q they cannot
get up;
\par }{\Q they fall
at my feet.
\par }{\Q \VS{40}You give
me strength
for battle;
\par }{\Q you make my foes kneel
before me.
\par }{\Q \VS{41}You make
my enemies
retreat;
\par }{\Q I destroy
those who hate
me.
\par }{\Q \VS{42}They cry
out, but there is no
one to help
them;

\par }{\Q they cry out to
the {\ND{Lord}}, but he does not
answer them.
\par }{\Q \VS{43}I grind
them as fine as the dust
of the ground;
\par }{\Q I crush
them and stomp on
them like clay
in the streets.
\par }{\Q \VS{44}You rescue
me from a hostile
army;
\par }{\Q you preserve
me as a
leader
of nations;
\par }{\Q people
over whom I had no
authority are now my subjects.
\par }{\Q \VS{45}Foreigners
are powerless
before me;

\par }{\Q when they hear
of my exploits, they submit
to me.
\par }{\Q \VS{46}Foreigners
lose their courage;
\par }{\Q they shake with fear as they leave their strongholds.
\par }{\Q \VS{47}The
{\ND{Lord}}
is alive!

\par }{\Q My protector
is praiseworthy!

\par }{\Q The God
who delivers
me is exalted as king!
\par }{\Q \VS{48}The one true God
completely vindicates
me;

\par }{\Q he makes nations
submit
to me.
\par }{\Q \VS{49}He delivers
me from my enemies;
\par }{\Q you snatch
me away from those who attack me;

\par }{\Q you rescue
me from violent
men.
\par }{\Q \VS{50}So
I will give you thanks,
O
{\ND{Lord}}, before the nations!

\par }{\Q I will sing praises
to you.
\par }{\Q \VS{51}He gives
his chosen king
magnificent
victories;
\par }{\Q he
is faithful
to his chosen ruler,
\par }{\Q to David
and to his descendants
forever!”


\par }\Chap{23}{\PP \VerseOne{1}These
are the final
words
of David:
\par }{\Q “The oracle
of David
son
of Jesse,
\par }{\Q the oracle
of the man
raised up
as
\par }{\Q the ruler
chosen
by the God
of Jacob,
\par }{\Q Israel’s
beloved
singer of songs:
\par }{\Q \VS{2}The
{\ND{Lord}}’s
spirit
spoke
through me;
\par }{\Q his word
was on
my tongue.
\par }{\Q \VS{3}The God
of Israel
spoke,
\par }{\Q the protector
of Israel
spoke
to me.
\par }{\Q The one who rules
fairly
among men,
\par }{\Q the one who rules
in the fear
of God,
\par }{\Q \VS{4}is like the light
of morning
when the sun
comes up,
\par }{\Q a morning
in which there are no
clouds.
\par }{\Q He is like the brightness
after rain
\par }{\Q that produces grass
from the earth.
\par }{\Q \VS{5}My dynasty
is
approved
by God,
\par }{\Q for
he has made
a perpetual
covenant
with me,
\par }{\Q arranged
in all
its particulars
and secured.
\par }{\Q He always
delivers
me,
\par }{\Q and brings all
I desire
to fruition.
\par }{\Q \VS{6}But evil
people are like thorns
–
\par }{\Q all
of them are tossed away,
\par }{\Q for
they cannot
be held
in the hand.
\par }{\Q \VS{7}The one
who touches
them
\par }{\Q must use an iron
instrument
\par }{\Q or the wooden shaft
of a spear.
\par }{\Q They are completely burned up
right where
they lie!”
\par }{\SH David’s Warriors
\par }{\PP \VS{8}These
are the names
of David’s
warriors:
\par }{\PP Josheb-Basshebeth, a Tahkemonite,
was head
of the officers.
He killed
eight
hundred
men with his spear
in one battle.
\VS{9}Next in command
was Eleazar
son
of Dodo,
the son
of Ahohi.
He was one of the three
warriors
who were with
David
when they defied
the Philistines
who were assembled
there
for battle.
When the men
of Israel
retreated,
\VS{10}he
stood
his ground and fought
the Philistines
until
his hand
grew so tired
that
it seemed stuck
to
his sword.
The
{\ND{Lord}}
gave a great
victory
on that day.
When the army
returned
to him,
the only
thing left to do was to plunder
the corpses.
\par }{\PP \VS{11}Next in command
was Shammah
son
of Agee
the Hararite.
When the Philistines
assembled
at Lehi, where there
happened
to be an area
of a field
that was full
of lentils,
the army
retreated
before
the Philistines.
\VS{12}But he made a stand
in the middle
of that area.
He defended
it and defeated
the
Philistines;
the {\ND{Lord}}
gave them
a great
victory.
\par }{\PP \VS{13}At the time of the harvest
three
of the thirty
leaders
went down
to
David
at
the cave
of Adullam.
A band
of Philistines
was camped
in the valley
of Rephaim.
\VS{14}David
was in the stronghold
at the time,
while a Philistine
garrison
was in Bethlehem.
\VS{15}David
was thirsty
and said,
“How I wish
someone would
give me some water
to drink
from the cistern
in Bethlehem
near the gate!”
\VS{16}So the three
elite warriors
broke through
the Philistine
forces and drew
some water
from the cistern
in Bethlehem
near the gate.
They carried
it back
to
David,
but he refused
to drink
it. He poured
it out
as a drink offering to the
{\ND{Lord}}
\VS{17}and said,
“O
{\ND{Lord}}, I will not
do
this! It is equivalent to the blood
of the men
who risked their lives
by going.”
So he refused
to drink
it. Such
were the exploits
of the three
elite warriors.
\par }{\PP \VS{18}Abishai
son
of Zeruiah,
the brother
of Joab,
was
head
of the three.
He
killed
three
hundred
men with his spear
and gained
fame
among the three.
\VS{19}From
the three
he was given honor
and he became
their officer,
even though he was not
one of the three.
\par }{\PP \VS{20}Benaiah
son
of Jehoida
was a brave warrior
from Kabzeel
who performed
great
exploits.
He struck down
the two
sons of Ariel
of Moab.
He also went down
and killed
a lion
in
a cistern
on a snowy
day.
\VS{21}He
also
killed
an
impressive-looking
Egyptian.
The Egyptian
wielded
a spear,
while Benaiah attacked him with a club.
He grabbed
the spear
out of the Egyptian’s
hand
and killed
him with his own spear.
\VS{22}Such
were the exploits
of Benaiah
son
of Jehoida,
who gained fame among
the three
elite warriors.
\VS{23}He received honor
from
the thirty
warriors, though he was not
one of the three
elite warriors. David
put
him in charge of
his bodyguard.
\par }{\PP \VS{24}Included with the thirty
were the following: Asahel
the brother
of Joab,
Elhanan
son
of Dodo
from Bethlehem,
\VS{25}Shammah
the Harodite,
Elika
the Harodite,
\VS{26}Helez
the Paltite,
Ira
son
of Ikkesh
from Tekoa,
\VS{27}Abiezer
the Anathothite,
Mebunnai
the Hushathite,
\VS{28}Zalmon
the Ahohite,
Maharai
the Netophathite,
\VS{29}Heled
son
of Baanah
the Netophathite,
Ittai
son
of Ribai
from Gibeah
in Benjamin,
\VS{30}Benaiah
the Pirathonite,
Hiddai
from the wadis
of Gaash,
\VS{31}Abi-Albon
the Arbathite,
Azmaveth
the Barhumite,
\VS{32}Eliahba
the Shaalbonite,
the sons
of Jashen,
Jonathan
\VS{33}son
of Shammah
the Hararite,
Ahiam
son
of Sharar
the Hararite,
\VS{34}Eliphelet
son
of Ahasbai
the Maacathite,
Eliam
son
of Ahithophel
the Gilonite,
\VS{35}Hezrai
the Carmelite,
Paarai
the Arbite,
\VS{36}Igal
son
of Nathan
from Zobah,
Bani
the Gadite,
\VS{37}Zelek
the Ammonite,
Naharai
the Beerothite
(the armor-bearer
of Joab
son
of Zeruiah),
\VS{38}Ira
the Ithrite,
Gareb
the Ithrite
\VS{39}and Uriah
the Hittite.
Altogether
there were thirty-seven.

\par }\Chap{24}{\PP \VerseOne{1}The
{\ND{Lord}}’s
anger
again
raged
against Israel,
and he incited
David
against them, saying,
“Go
count
Israel
and Judah.”
\VS{2}The king
told
Joab,
the general in command
of his army,
“Go through
all
the tribes
of Israel
from Dan
to Beer Sheba
and muster
the army,
so I may know
the
size
of the army.”
\par }{\PP \VS{3}Joab
replied
to
the king,
“May
the {\ND{Lord}}
your God
make the army
a hundred
times
larger right before the eyes
of my lord
the king! But
why
does
my master
the king
want
to do this?”
\par }{\PP \VS{4}But
the king’s
edict
stood, despite the objections
of Joab
and the leaders
of the army.
So
Joab
and the leaders
of the army
left the king’s
presence
in order to muster
the
Israelite
army.
\par }{\PP \VS{5}They crossed
the Jordan
and camped
at Aroer,
on the south
side of the city,
at the wadi
of Gad,
near Jazer.
\VS{6}Then they went
on to Gilead
and to
the region
of Tahtim Hodshi,
coming
to Dan Jaan
and on around
to
Sidon.
\VS{7}Then they went
to the fortress
of Tyre
and all
the cities
of the Hivites
and the Canaanites.
Then they went
on to
the Negev
of Judah,
to Beer Sheba.
\VS{8}They went through
all
the land
and after
nine
months
and twenty
days
came back
to Jerusalem.
\par }{\PP \VS{9}Joab
reported
the number
of warriors
to
the king.
In Israel
there were
800,000
sword-wielding
warriors,
and in Judah
there were 500,000
soldiers.
\par }{\PP \VS{10}David
felt
guilty after
he had numbered
the army.
David
said
to
the {\ND{Lord}}, “I have sinned
greatly
by doing
this! Now,
O
{\ND{Lord}}, please
remove
the guilt
of your servant,
for
I have acted very
foolishly.”
\par }{\PP \VS{11}When David
got
up the next morning,
the {\ND{Lord}}
had already
spoken to
Gad
the prophet,
David’s
seer:
\VS{12}“Go,
tell
David,
‘This is what
the {\ND{Lord}}
says: I am
offering you three
forms
of judgment. Pick
one
of them
and I
will carry it out against you.’ ”
\par }{\PP \VS{13}Gad
went
to
David
and told
him, “Shall seven
years
of famine
come upon your land? Or shall
you flee
for three
months
from your enemy
with him
in hot pursuit? Or shall
there be
three
days
of plague
in your land? Now
decide
what
I should tell
the one who sent me.”
\VS{14}David
said
to
Gad,
“I am
very
upset! I prefer
that we be attacked
by the
{\ND{Lord}}, for
his mercy
is great;
I do not
want to be attacked
by men!”
\par }{\PP \VS{15}So the
{\ND{Lord}}
sent
a plague
through Israel
from the morning
until
the completion
of the appointed
time.
Seventy
thousand
men
died
from
Dan
to
Beer Sheba.
\VS{16}When
the angel
extended
his hand
to destroy
Jerusalem,
the {\ND{Lord}}
relented
from his judgment.
He told
the angel
who was killing
the people,
“That’s
enough! Stop now!” (Now the
{\ND{Lord}}’s
angel
was
near
the threshing floor
of Araunah
the Jebusite.)
\par }{\PP \VS{17}When he saw
the angel
who was destroying
the people,
David
said
to
the {\ND{Lord}}, “Look,
it is I
who have sinned
and done this evil thing! As for these
sheep
– what
have they done? Attack me
and my family.”
\par }{\SH David Acquires a Threshing Floor and Constructs an Altar There
\par }{\PP \VS{18}So Gad
went
to
David
that day
and told
him, “Go up
and build
an altar
for the
{\ND{Lord}}
on the threshing floor
of Araunah
the Jebusite.”
\VS{19}So David
went up
as
Gad
instructed
him to do, according to the
{\ND{Lord}}’s
instructions.
\par }{\PP \VS{20}When Araunah
looked
out and saw
the king
and his servants
approaching
him, he
went out
and bowed
to the king
with his face
to the ground.
\VS{21}Araunah
said,
“Why
has my lord
the king
come
to
his servant?” David
replied,
“To buy
from you
the
threshing floor
so I can build
an altar
for the
{\ND{Lord}}, so
that the plague
may be removed from the people.”
\VS{22}Araunah
told
David,
“My lord
the king
may take
whatever
he wishes
and offer it. Look! Here are oxen
for burnt offerings,
and threshing sledges
and harnesses
for wood.
\VS{23}I,
the servant of my lord the king,
give
it all
to the king!” Araunah
also told
the king,
“May the
{\ND{Lord}}
your God
show you favor!”
\VS{24}But the king
said
to
Araunah,
“No,
I insist on
buying
it from you! I will not
offer
to the
{\ND{Lord}}
my God
burnt sacrifices
that cost me nothing.”
\par }{\PP So David
bought
the threshing floor
and the oxen
for fifty
pieces
of silver.
\VS{25}Then David
built
an altar
for the
{\ND{Lord}}
there
and offered
burnt sacrifices
and peace offerings.
And the
{\ND{Lord}}
accepted prayers
for the land,
and the plague
was removed from Israel.
\par }