\NormalFont\ShortTitle{2 Corinthians}
{\MT 2 Corinthians

\par }\ChapOne{1}{\SH Salutation
\par }{\PP \VerseOne{1}From Paul,
an apostle
of Christ
Jesus
by
the will
of God,
and
Timothy
our brother,
to the church
of God
that is
in
Corinth,
with
all
the saints
who are
in
all
Achaia.
\VS{2}Grace
and
peace
to you
from
God
our
Father
and
the Lord
Jesus
Christ!
\par }{\SH Thanksgiving for God’s Comfort
\par }{\PP \VS{3}Blessed
is the
God
and Father
of our
Lord
Jesus
Christ,
the Father
of mercies
and
God
of all
comfort,
\VS{4}who comforts
us
in
all
our
troubles
so that
we
may be able
to comfort
those experiencing any
trouble
with
the comfort
with which
we
ourselves
are comforted
by
God.
\VS{5}For
just as
the sufferings
of Christ
overflow
toward
us,
so
also
our
comfort
through
Christ
overflows to you.
\VS{6}But
if
we are afflicted,
it is for
your
comfort
and
salvation;
if
we are comforted,
it is for
your
comfort
that you experience
in
your patient
endurance of the same
sufferings
that
we
also suffer.
\VS{7}And our hope for you is steadfast because we know
that
as
you
share
in our sufferings,
so
also
you will share in our comfort.
\VS{8}For
we do
not
want
you
to be unaware,
brothers and sisters,
regarding
the affliction
that happened
to us
in
the province of Asia,
that
we were burdened
excessively,
beyond
our strength,
so
that we
despaired
even
of living.
\VS{9}Indeed
we felt as if the sentence
of death
had been passed against
us, so that
we
would
not
trust
in
ourselves
but
in
God
who raises
the dead.
\VS{10}He delivered
us
from
so great a risk
of death,
and
he will deliver
us. We have set
our hope
on
him
that
he will deliver
us yet again,
\VS{11}as
you
also
join in helping
us
by prayer,
so that
many
people
may give thanks to
God
on
our
behalf for the gracious gift
given to us
through
the help of many.
\par }{\SH Paul Defends His Changed Plans
\par }{\PP \VS{12}For
our
reason for confidence
is
this: the testimony
of our
conscience,
that
with
pure motives
and
sincerity
which are from God –
not
by
human
wisdom
but
by
the grace
of God
– we conducted
ourselves in
the world,
and
all the more
toward
you.
\VS{13}For
we do
not
write
you
anything
other
than
what
you can read
and
also understand.
But
I hope
that
you will understand
completely
\VS{14}just as
also
you have partly understood
us,
that
we
are your
source of pride
just as
you
also
are ours
in
the day
of the Lord
Jesus.
\VS{15}And
with this
confidence
I intended
to come
to
you
first so that
you would get
a second
opportunity to see us,
\VS{16}and
through
your
help to go on
into
Macedonia
and
then
from
Macedonia
to come back
to
you
and
be helped on
our way
into
Judea
by
you.
\VS{17}Therefore
when
I was planning to do
this,
I did not do so without thinking
about what I
was doing,
did I? Or
do I make my plans
according to
mere human
standards
so that
I would be saying both “Yes,
yes”
and
“No,
no” at the same time?
\VS{18}But
as God
is faithful,
our
message
to
you
is
not
“Yes”
and
“No.”
\VS{19}For
the Son
of God,
Jesus
Christ,
the one who was proclaimed
among
you
by
us
– by
me
and
Silvanus
and
Timothy
– was
not
“Yes”
and
“No,”
but
it has always been
“Yes”
in
him.
\VS{20}For
every one
of God’s
promises
are “Yes”
in
him;
therefore
also
through
him
the “Amen”
is spoken, to
the glory
we
give to God.
\VS{21}But
it is God
who establishes
us
together with
you
in
Christ
and
who anointed
us,
\VS{22}who also
sealed
us
and
gave
us the Spirit
in
our
hearts
as a down payment.
\par }{\SH Why Paul Postponed His Visit
\par }{\PP \VS{23}Now I
appeal
to God
as
my witness,
that
to spare
you
I did
not
come again
to
Corinth.
\VS{24}I do not
mean that
we rule over
your
faith,
but
we are
workers
with you for your
joy,
because
by faith
you stand firm.

\par }\Chap{2}{\PP \VerseOne{1}So
I made up
my own mind
not
to pay you
another
painful
visit.
\VS{2}For
if
I
make
you
sad,
who
would be left to make
me
glad
but the one I
caused to be sad?
\VS{3}And
I wrote
this
very thing
to you, so that
when
I came
I would
not
have
sadness
from
those who
ought
to make
me
rejoice,
since I am confident
in
you
all
that
my
joy
would be
yours.
\VS{4}For
out of
great
distress
and
anguish
of heart
I wrote
to you
with
many
tears,
not
to
make you sad,
but
to
let you know
the love
that
I have
especially
for
you.
\VS{5}But
if
anyone
has caused sadness,
he has
not
saddened
me alone,
but
to
some extent
(not
to exaggerate) he has saddened all
of you as well.
\VS{6}This
punishment
on such an individual
by
the majority
is enough for him,
\VS{7}so that
now instead
you
should rather forgive
and
comfort
him. This will keep him from being overwhelmed
by excessive
grief to the point of despair.
\VS{8}Therefore
I urge
you
to reaffirm
your love
for
him.
\VS{9}For
this reason
also
I wrote
you: to
test
you
to see if
you are
obedient
in
everything.
\VS{10}If you forgive
anyone for anything, I also forgive him – for indeed what I have forgiven (if I have forgiven anything) I did so for you in the presence of Christ,
\VS{11}so that
we may
not
be exploited
by
Satan
(for
we are
not
ignorant
of his
schemes).
\VS{12}Now when I arrived
in
Troas
to
proclaim the gospel
of Christ,
even though
the Lord
had opened
a door
of opportunity for me,
\VS{13}I had
no
relief
in my
spirit,
because I did
not
find
my
brother
Titus
there. So I said good-bye
to them
and
set out
for
Macedonia.
\par }{\SH Apostolic Ministry
\par }{\PP \VS{14}But
thanks
be to God
who always
leads
us
in triumphal procession
in
Christ
and
who makes known
through
us
the fragrance
that consists of the knowledge
of him
in
every
place.
\VS{15}For
we are
a sweet aroma
of Christ
to God
among
those who are being saved
and
among
those who are perishing –
\VS{16}to the latter an odor
from
death
to
death,
but
to the former a fragrance
from
life
to
life.
And
who is adequate
for
these things?
\VS{17}For
we are
not
like
so many
others, hucksters who peddle
the word
of God
for profit,
but
we are speaking
in
Christ
before
God
as
persons of
sincerity,
as
persons sent from
God.

\par }\Chap{3}{\PP \VerseOne{1}Are we beginning
to commend
ourselves
again? We
don’t
need
letters
of recommendation
to
you
or
from
you
as
some other people do, do we?
\VS{2}You
yourselves
are
our
letter,
written
on
our
hearts,
known
and
read
by
everyone,
\VS{3}revealing
that
you are
a letter
of Christ,
delivered
by
us,
written
not
with ink
but
by the Spirit
of the living
God,
not
{\IT{
on
stone
tablets}}
but
on
tablets
of human
hearts.
\par }{\PP \VS{4}Now
we have
such
confidence
in
God
through
Christ.
\VS{5}Not
that
we are
adequate
in
ourselves
to consider
anything
as
if it were coming from
ourselves,
but
our
adequacy
is from
God,
\VS{6}who
made
us
adequate
to be servants
of a new
covenant
not
based on the letter
but
on the Spirit,
for
the letter
kills,
but
the Spirit
gives life.
\par }{\SH The Greater Glory of the Spirit’s Ministry
\par }{\PP \VS{7}But
if
the ministry
that produced death
– carved
in
letters
on stone tablets –
came
with
glory,
so
that the Israelites
could
not
keep
their eyes fixed
on
the face
of Moses
because
of the glory
of his
face
(a glory which was made ineffective),
\VS{8}how
much more
glorious
will
the ministry
of the Spirit
be?
\VS{9}For
if
there was glory
in the ministry
that produced condemnation,
how much
more
does the ministry
that produces righteousness
excel
in glory!
\VS{10}For
indeed, what had been glorious
now has no
glory
because of the tremendously
greater glory of what replaced it.
\VS{11}For
if
what was made ineffective came
with
glory,
how much
more
has what remains
come in
glory!
\VS{12}Therefore,
since
we have
such
a hope,
we behave
with great
boldness,
\VS{13}and
not
like
Moses
who used to put
a veil
over
his
face
to
keep
the Israelites
from staring
at
the result
of the glory that was made ineffective.
\VS{14}But
their
minds
were closed.
For
to
this very day,
the same
veil
remains when they hear the old
covenant
read.
It has
not
been removed
because
only in
Christ
is it taken away.
\VS{15}But
until
this very day
whenever
Moses
is read,
a veil
lies
over
their
minds,
\VS{16}but
when
one turns
to
the Lord,
{\IT{
the veil
is removed}}.
\VS{17}Now
the Lord
is
the Spirit,
and
where
the Spirit
of the Lord
is present, there is freedom.
\VS{18}And
we
all,
with unveiled
faces
reflecting
the glory
of the Lord,
are being transformed
into the same
image
from
one degree of glory
to
another,
which is from
the Lord,
who is the Spirit.

\par }\Chap{4}{\PP \VerseOne{1}Therefore,
since we have
this
ministry,
just as
God has shown
us mercy,
we do
not
become discouraged.
\VS{2}But
we have rejected
shameful
hidden deeds,
not
behaving
with
deceptiveness
or
distorting
the word
of God,
but
by open proclamation
of the truth
we commend
ourselves
to
everyone’s
conscience
before
God.
\VS{3}But
even
if
our
gospel
is
veiled,
it is
veiled
only to
those who are perishing,
\VS{4}among
whom
the god
of this
age
has blinded
the minds
of those who do not believe
so
they would not
see
the light
of the glorious
gospel
of Christ,
who
is
the image
of God.
\VS{5}For
we do
not
proclaim
ourselves,
but
Jesus
Christ
as Lord,
and
ourselves
as your
slaves
for
Jesus’
sake.
\VS{6}For
God,
who said “{\IT{
Let
light
shine
out of
darkness}},” is the one who
shined
in
our
hearts
to
give us the light
of the glorious
knowledge
of God
in
the face
of Christ.
\par }{\SH An Eternal Weight of Glory
\par }{\PP \VS{7}But
we have
this
treasure
in
clay
jars,
so that
the extraordinary
power
belongs to God
and
does not
come from
us.
\VS{8}We are experiencing trouble
on
every
side, but
are
not
crushed;
we are perplexed,
but
not
driven to despair;
\VS{9}we are persecuted,
but
not
abandoned;
we are knocked down,
but
not
destroyed,
\VS{10}always
carrying around
in
our body
the death
of Jesus,
so that
the life
of Jesus
may
also
be made visible
in
our
body.
\VS{11}For
we
who are alive
are constantly
being handed over
to
death
for
Jesus’
sake,
so that
the life
of Jesus
may
also be made visible
in
our
mortal
body.
\VS{12}As a result,
death
is at work
in
us,
but
life
is at work in
you.
\VS{13}But
since
we have
the same
spirit
of faith
as that shown in what has been written, “{\QT{
I believed;
therefore
I spoke}},” we
also
believe,
therefore
we
also
speak.
\VS{14}We do so because we know
that
the one who raised up
Jesus
will
also
raise
us
up
with
Jesus
and
will bring
us with
you into his presence.
\VS{15}For
all these things
are for
your
sake,
so that
the grace
that is including
more and more
people may cause thanksgiving
to increase
to
the glory
of God.
\VS{16}Therefore
we do
not
despair,
but
even if
our
physical body
is wearing away,
our
inner
person
is being renewed
day
by
day.
\VS{17}For
our momentary,
light
suffering
is producing
for us
an eternal
weight
of glory
far beyond
all comparison
\VS{18}because we
are
not
looking at
what can be seen
but
at what cannot
be seen.
For
what can be seen
is temporary,
but
what cannot
be seen
is eternal.

\par }\Chap{5}{\PP \VerseOne{1}For
we know
that
if
our
earthly
house,
the tent
we live in, is dismantled,
we have
a building
from
God,
a house
not built by human hands,
that is eternal
in
the heavens.
\VS{2}For
in
this
earthly house
we groan,
because we desire
to put on
our
heavenly
dwelling,
\VS{3}if
indeed,
after we have put on
our heavenly house,
we will
not
be found
naked.
\VS{4}For
we groan
while
we are
in
this tent,
since we are weighed down,
because we do
not
want
to be unclothed,
but
clothed,
so that
what is mortal
may be swallowed up
by
life.
\VS{5}Now the one who prepared
us
for
this
very purpose is God,
who gave
us
the Spirit
as a down payment.
\VS{6}Therefore
we are
always
full of courage,
and
we know
that
as long as we are alive here on earth
we are absent
from
the Lord –
\VS{7}for
we live
by
faith,
not
by
sight.
\VS{8}Thus
we are full of courage
and
would prefer
to be away
from
the body
and
at home
with
the Lord.
\VS{9}So then
whether
we are alive
or
away,
we make
it our ambition
to please
him.
\VS{10}For
we must
all
appear
before
the judgment seat
of Christ,
so that
each one
may be paid back
according to
what
he has done
while in the body,
whether
good
or
evil.
\par }{\SH The Message of Reconciliation
\par }{\PP \VS{11}Therefore,
because we know
the fear
of the Lord,
we try to persuade
people,
but
we are well known
to God,
and
I hope
we are well known
to your
consciences too.
\VS{12}We are
not
trying to commend
ourselves
to you
again,
but
are giving
you
an opportunity
to be proud
of
us,
so that
you may be able
to answer those who take pride
in
outward appearance
and
not
in what is in
the heart.
\VS{13}For
if
we are out of
our minds,
it is for God;
if
we are of sound mind,
it is for you.
\VS{14}For
the love
of Christ
controls
us,
since we have concluded
this,
that
Christ
died
for
all;
therefore
all
have died.
\VS{15}And
he died
for
all
so that
those who live
should
no longer
live
for themselves
but
for him who died
for
them
and
was raised.
\VS{16}So then
from
now on
we
acknowledge
no one
from
an outward human point of view.
Even though we have known
Christ
from such
a human point of view,
now
we do
not
know
him in that way any longer.
\VS{17}So then,
if
anyone
is in
Christ,
he is a new
creation;
what is old
has passed away –
look,
what is new
has come!
\VS{18}And
all these things
are from
God
who reconciled
us
to himself
through
Christ,
and
who has given
us
the ministry
of reconciliation.
\VS{19}In other words,
in
Christ
God
was
reconciling
the world
to himself,
not
counting
people’s
trespasses
against them,
and
he has given
us
the message
of reconciliation.
\VS{20}Therefore
we are ambassadors
for
Christ,
as though
God
were making
His plea
through
us.
We plead
with you on
Christ’s
behalf,
“Be reconciled
to God!”
\VS{21}God made
the one who did
not
know
sin
to be sin
for
us,
so that
in
him
we
would become
the righteousness
of God.

\par }\Chap{6}{\PP \VerseOne{1}Now because we are fellow workers,
we also
urge
you
not
to receive
the grace
of God
in vain.
\VS{2}For
he says, “{\QT{
I heard
you
at the acceptable
time, and
in
the day
of salvation
I helped
you}}.” Look,
now
is
{\QT{
the acceptable
time}}; look,
now
is
{\QT{
the day
of salvation}}!
\VS{3}We do
not
give
anyone
an occasion for taking an offense in anything,
so that
no
fault
may be found with our ministry.
\VS{4}But
as
God’s
servants,
we have commended
ourselves
in
every
way, with
great
endurance,
in
persecutions,
in
difficulties,
in
distresses,
\VS{5}in
beatings,
in
imprisonments,
in
riots,
in
troubles,
in
sleepless nights,
in
hunger,
\VS{6}by
purity,
by
knowledge,
by
patience,
by
benevolence,
by
the Holy
Spirit,
by
genuine
love,
\VS{7}by
truthful
teaching,
by
the power
of God,
with
weapons
of righteousness
both for the right hand
and
for the left,
\VS{8}through
glory
and
dishonor,
through
slander
and
praise; regarded
as
impostors,
and
yet true;
\VS{9}as
unknown,
and
yet well-known;
as
dying
and yet – see! – we continue to live; as those who are scourged and yet not executed;
\VS{10}as
sorrowful,
but
always
rejoicing,
as
poor,
but making
many
rich,
as
having
nothing,
and
yet possessing
everything.
\par }{\PP \VS{11}We have spoken
freely to
you,
Corinthians;
our
heart
has been opened wide to you.
\VS{12}Our affection for you is not
restricted,
but
you are restricted
in
your
affections
for
us.
\VS{13}Now as a fair exchange
– I speak
as
to my children
– open wide
your hearts to us
also.
\par }{\SH Unequal Partners
\par }{\PP \VS{14}Do
not
become
partners
with those who do not believe,
for
what
partnership
is there between righteousness
and
lawlessness,
or
what
fellowship
does light
have with
darkness?
\VS{15}And
what
agreement
does Christ
have with
Beliar? Or
what
does a believer
share
in common with
an unbeliever?
\VS{16}And
what
mutual agreement
does the temple
of God
have with
idols? For
we
are the temple
of the living
God,
just as
God
said, “{\QT{
I will live
in
them
and
will walk
among them, and
I will be
their
God, and
they
will be
my
people}}: *.”
\VS{17}Therefore “{\QT{
come out
from
their
midst, and
be separate}},” says
the Lord, “{\IT{
{\BD{
and
touch
no
unclean
thing}}, and I
will welcome
you}},
\VS{18}{\IT{
and
I will be
a father
to you,
and
you
will be
my
sons
and
daughters}},” says
the All-Powerful
Lord.

\par }\Chap{7}{\PP \VerseOne{1}Therefore,
since we have
these
promises,
dear friends,
let us cleanse
ourselves
from
everything
that could defile
the body
and
the spirit,
and thus accomplish
holiness
out of reverence
for God.
\VS{2}Make
room for us
in your hearts; we have wronged
no one,
we have ruined
no one,
we have exploited
no one.
\VS{3}I do
not
say
this to condemn
you, for
I told
you before
that
you are
in
our
hearts
so that
we die together
and
live together with you.
\par }{\SH A Letter That Caused Sadness
\par }{\PP \VS{4}I
have great
confidence
in
you;
I take great
pride
on
your
behalf.
I am filled
with encouragement;
I am overflowing
with joy
in the midst of all
our
suffering.
\VS{5}For
even when
we
came
into
Macedonia,
our
body
had
no
rest
at all, but
we were troubled
in
every way
– struggles
from the outside,
fears
from within.
\VS{6}But
God,
who encourages
the downhearted,
encouraged
us
by
the arrival
of Titus.
\VS{7}We were encouraged not
only
by
his
arrival,
but
also
by
the encouragement
you
gave him, as he reported
to us
your
longing,
your
mourning,
your
deep concern
for
me,
so that
I
rejoiced
more than ever.
\VS{8}For
even if
I made
you
sad
by
my letter,
I do
not
regret
having written it (even though
I did regret
it, for I see
that
my letter
made
you
sad,
though only for
a short time).
\VS{9}Now
I rejoice,
not
because
you were made sad,
but
because
you were made sad
to the point of repentance.
For
you were made sad
as
God
intended, so that
you were
not
harmed
in
any way by
us.
\VS{10}For
sadness as intended by God
produces
a repentance
that leads
to
salvation,
leaving no regret,
but
worldly
sadness
brings
about death.
\VS{11}For
see
what this very thing, this
sadness
as
God
intended, has produced
in you: what eagerness,
what defense
of yourselves, what indignation,
what alarm,
what longing,
what deep concern,
what punishment! In
everything
you have proved
yourselves
to be
innocent
in this matter.
\VS{12}So then,
even
though
I wrote
to you,
it was not
on account
of the one who did wrong,
or
on account
of the one who was wronged,
but
to reveal
to
you
your
eagerness
on
our
behalf
before
God.
\VS{13}Therefore
we have been encouraged.
And
in addition
to our own
encouragement,
we rejoiced
even more
at
the joy
of Titus,
because
all
of you
have refreshed
his
spirit.
\VS{14}For
if
I have boasted
to him
about
anything concerning you,
I have
not
been embarrassed
by you, but
just as
everything
we said
to you
was true,
so
our
boasting
to
Titus
about you has proved
true as well.
\VS{15}And
his
affection
for
you
is
much greater
when
he remembers
the obedience
of you
all,
how
you welcomed
him
with
fear
and
trembling.
\VS{16}I rejoice
because
in
everything
I am fully confident
in
you.

\par }\Chap{8}{\PP \VerseOne{1}Now we make known
to you,
brothers and sisters,
the grace
of God
given
to the churches
of Macedonia,
\VS{2}that
during
a severe
ordeal
of suffering,
their
abundant
joy
and
their
extreme
poverty
have overflowed
in
the wealth
of their
generosity.
\VS{3}For
I testify,
they gave according to
their means
and
beyond
their means.
They did so voluntarily,
\VS{4}begging
us
with
great
earnestness
for the blessing
and
fellowship
of helping
the saints.
\VS{5}And
they did this not
just as
we had hoped,
but
they gave
themselves
first
to the Lord
and
to us
by
the will
of God.
\VS{6}Thus
we
urged
Titus
that,
just as
he had previously begun
this work, so
also
he should complete
this
act of kindness
for
you.
\VS{7}But
as
you excel
in
everything
– in faith,
in speech,
in knowledge,
and
in all
eagerness
and
in
the love
from
us
that is in
you –
make sure that you excel
in
this
act of kindness too.
\VS{8}I am
not
saying
this as a command,
but
I am testing the genuineness
of your
love
by comparison
with
the eagerness
of others.
\VS{9}For
you know
the grace
of our
Lord
Jesus
Christ,
that
although he was
rich,
he became poor
for
your
sakes, so that
you
by his
poverty
could become rich.
\VS{10}So
here is my opinion
on
this matter: It is to your advantage,
since you made
a good start last year
both in your giving and your desire to give,
\VS{11}to finish
what you started,
so
that just as
you wanted
to do
it eagerly,
you can also complete
it according to
your means.
\VS{12}For
if
the eagerness
is present,
the gift itself is acceptable
according to
whatever
one has,
not
according to
what he does
not
have.
\VS{13}For
I do not
say this so
there would be relief
for others
and suffering
for you,
but
as a matter of
equality.
\VS{14}At the present time, your abundance will meet their need, so that
one day their abundance
may also meet
your
need,
and thus there may be
equality,
\VS{15}as
it is written: “{\QT{
The one who gathered much
did
not
have too much, and
the one who gathered little
did
not
have too little}}.”
\par }{\SH The Mission of Titus
\par }{\PP \VS{16}But
thanks
be to God
who put
in
the heart
of Titus
the same
devotion
I have for
you,
\VS{17}because
he
not only
accepted
our request, but since he was
very eager,
he is coming
to
you
of his own accord.
\VS{18}And
we are sending
along with
him
the brother
who
is praised
by
all
the churches
for his work in
spreading the gospel.
\VS{19}In addition, this brother has also
been chosen
by
the churches
as our
traveling companion
as we administer
this
generous gift
to
the glory
of the Lord
himself and
to show our
readiness to help.
\VS{20}We did this
as a precaution
so that no
one
should blame
us
in regard
to this
generous gift
we
are administering.
\VS{21}For
we are
{\IT{
concerned
about what is right
not
only
before
the Lord
but
also
before
men}}.
\VS{22}And
we are sending
with them
our
brother
whom
we have tested
many
times
and found eager
in many matters, but who now
is much more
eager
than ever because of the great
confidence
he has in
you.
\VS{23}If
there is any question about
Titus,
he is my
partner
and
fellow worker
among
you;
if
there is any question about our
brothers,
they are messengers
of the churches,
a glory
to Christ.
\VS{24}Therefore
show
them
openly before
the churches
the proof
of your
love
and
of our
pride
in
you.

\par }\Chap{9}{\PP \VerseOne{1}For
it is
not necessary
for me
to write
you
about
this service
to
the saints,
\VS{2}because
I know
your
eagerness to help.
I keep boasting
to the Macedonians
about this eagerness of yours,
that
Achaia
has been ready
to give since last year,
and
your
zeal to participate
has stirred up
most of them.
\VS{3}But
I am sending
these brothers
so that
our
boasting
about
you
may
not
be empty
in
this
case,
so that
you may be ready
just as
I kept telling
them.
\VS{4}For if
any of the Macedonians
should come
with
me
and
find
that you
are not ready
to give, we
would be humiliated
(not
to mention
you) by this
confidence we had in you.
\VS{5}Therefore
I thought
it necessary
to urge
these brothers
to
go
to
you
in advance
and
to arrange ahead of time
the generous contribution
you
had promised,
so this
may be
ready
as
a generous gift
and
not
as
something you feel forced to do.
\VS{6}My point is this: The person who sows
sparingly
will
also
reap
sparingly,
and
the person who sows
generously
will
also
reap
generously.
\VS{7}Each one
of you should give just as
he has decided
in his heart,
not
reluctantly
or
under
compulsion,
because
God
loves
a cheerful
giver.
\VS{8}And
God
is able
to make all
grace
overflow
to
you
so that
because you have
enough
of
everything
in every way
at all times, you will overflow
in
every
good
work.
\VS{9}Just as
it is written, “{\QT{
He has scattered
widely, he has given
to the poor;
his
righteousness
remains
forever}}.”
\VS{10}Now
God who provides
seed
for the sower
and
bread
for
food
will provide
and
multiply
your
supply of seed
and
will cause
the harvest
of your
righteousness
to grow.
\VS{11}You will be enriched
in
every way
so that you may be generous
on
every
occasion, which
is producing
through
us
thanksgiving
to God,
\VS{12}because
the service
of this
ministry
is
not
only
providing
for the needs
of the saints
but
is
also
overflowing
with
many
thanks
to God.
\VS{13}Through
the evidence
of this
service
they will glorify
God
because of
your obedience
to your
confession
in
the
gospel
of Christ
and
the generosity
of your sharing
with
them
and
with
everyone.
\VS{14}And
in their
prayers
on
your
behalf
they long for
you
because of
the extraordinary
grace
God
has shown to
you.
\VS{15}Thanks
be to God
for
his
indescribable
gift!

\par }\Chap{10}{\PP \VerseOne{1}Now
I,
Paul,
appeal
to you
personally by
the meekness
and
gentleness
of Christ
(I who am meek
when
present
among
you,
but
am full of courage
toward
you
when away!) –
\VS{2}now I ask
that when I am present
I may not
have to be bold
with the confidence
that
(I expect) I will dare
to use against
some
who consider
us
to be behaving
according to
human standards.
\VS{3}For
though we live
as
human beings,
we do
not
wage war
according to
human standards,
\VS{4}for
the weapons
of our
warfare
are not
human weapons,
but
are made powerful
by God
for
tearing down
strongholds. We tear down arguments
\VS{5}and
every
arrogant obstacle
that is raised up
against
the knowledge
of God,
and
we take
every
thought
captive
to make it obey
Christ.
\VS{6}We are
also
ready
to punish
every
act of disobedience,
whenever
your
obedience
is complete.
\VS{7}You are looking at
outward
appearances.
If
anyone
is confident
that he
belongs
to Christ,
he should reflect
on
this
again: Just as
he
himself belongs to Christ,
so
too
do we.
\VS{8}For
if
I boast
somewhat
more about
our
authority
that
the Lord
gave
us for
building
you up
and
not
for
tearing
you
down,
I will
not
be ashamed of doing so.
\VS{9}I do not
want to seem as though
I am trying to terrify
you
with
my letters,
\VS{10}because
some say,
“His letters
are weighty
and
forceful,
but
his physical
presence
is weak
and
his speech
is of no account.”
\VS{11}Let
such
a person consider
this: What
we say
by
letters
when
we are absent,
we also
are in actions
when
we are present.
\par }{\SH Paul’s Mission
\par }{\PP \VS{12}For
we would
not
dare
to classify
or
compare
ourselves
with some
of those who recommend
themselves.
But
when
they
measure
themselves
by themselves
and
compare
themselves
with themselves,
they are
without
understanding.
\VS{13}But
we
will
not
boast
beyond
certain limits,
but
will confine our boasting according to
the limits
of the work to which
God
has appointed us,
that reaches
even
as far as
you.
\VS{14}For
we were not overextending ourselves, as though we did not
reach as far as you,
because
we were the first to reach
as far as you
with
the gospel
about Christ.
\VS{15}Nor
do
we boast
beyond certain limits
in the work
done by
others,
but we hope
that as your
faith
continues to grow,
our work may be greatly
expanded
among
you
according to
our limits,
\VS{16}so that
we may preach the gospel
in the regions that lie beyond
you,
and not
boast
of
work already done
in
another person’s
area.
\VS{17}But
{\BD{
{\QT{
{\IT{the one who boasts}}}}
must boast
in
the Lord
must boast
in
the Lord}}.
\VS{18}For
it is not
the person who commends
himself
who
is
approved,
but
the person
the Lord
commends.

\par }\Chap{11}{\PP \VerseOne{1}I wish
that you would be patient with
me
in a little
foolishness,
but
indeed
you are being patient with
me!
\VS{2}For
I am jealous
for you
with godly
jealousy,
because
I promised
you
in marriage to one
husband,
to present
you as a
pure
virgin
to Christ.
\VS{3}But
I am afraid
that just as
the serpent
deceived
Eve
by
his
treachery,
your
minds
may
be led astray
from
a sincere
and
pure devotion
to
Christ.
\VS{4}For
if
someone comes
and proclaims
another
Jesus
different from the one we proclaimed,
or
if you receive
a
different
spirit
than the one you received,
or
a different
gospel
than the one you accepted,
you put up
with it well enough!
\VS{5}For
I consider
myself not
at all inferior
to
those “super-apostles.”
\VS{6}And
even
if
I am unskilled
in speaking,
yet
I am certainly not
so in knowledge.
Indeed,
we have made
this plain
to
you
in
everything
in
every way.
\VS{7}Or
did I commit
a sin
by humbling
myself
so that
you
could be exalted,
because
I proclaimed
the gospel
of God
to you
free of charge?
\VS{8}I robbed
other
churches
by receiving
support
from
them so that I could serve
you!
\VS{9}When
I was
with
you
and
was in need,
I was
not
a burden
to anyone,
for
the brothers
who came
from
Macedonia
fully supplied
my
needs.
I kept
myself
from being a burden
to you
in
any way,
and
will continue to do so.
\VS{10}As the truth
of Christ
is
in
me,
this
boasting
of
mine
will
not
be stopped
in
the regions
of Achaia.
\VS{11}Why? Because
I do
not
love
you? God
knows I do!
\VS{12}And
what
I am doing
I will continue to do,
so that
I may eliminate
any opportunity
for those who want
a chance
to be regarded as our equals in
the things
they boast about.
\VS{13}For
such people
are false apostles,
deceitful
workers,
disguising
themselves as
apostles
of Christ.
\VS{14}And
no
wonder,
for
even Satan
disguises
himself
as
an angel
of light.
\VS{15}Therefore
it is not
surprising
his
servants
also
disguise
themselves as
servants
of righteousness,
whose
end
will
correspond to
their
actions.
\par }{\SH Paul’s Sufferings for Christ
\par }{\PP \VS{16}I say
again,
let no
one
think
that I am
a fool.
But
if
you do,
then
at least
accept
me
as
a fool,
so that
I too
may boast
a little.
\VS{17}What
I am saying
with
this
boastful
confidence
I do
not
say
the way the Lord
would. Instead it is, as it were, foolishness.
\VS{18}Since
many
are boasting
according
to human standards,
I too
will boast.
\VS{19}For
since you are
so wise,
you put up
with fools
gladly.
\VS{20}For
you put up
with it if
someone
makes slaves
of you,
if
someone
exploits
you, if
someone
takes advantage
of you, if
someone
behaves arrogantly
toward you, if
someone
strikes
you
in
the face.
\VS{21}(To
my disgrace
I must say
that
we
were
too weak
for
that!) But
whatever
anyone
else dares
to boast about (I am speaking
foolishly), I also
dare to boast about the same thing.
\VS{22}Are they
Hebrews? So
am I.
Are they
Israelites? So
am I.
Are they
descendants
of Abraham? So
am I.
\VS{23}Are they
servants
of Christ? (I am talking
like I
am out of
my mind!) I am even more so: with much greater
labors,
with
far more
imprisonments,
with
more severe
beatings,
facing
death
many times.
\VS{24}Five times
I received
from
the Jews
forty
lashes less
one.
\VS{25}Three times
I was beaten with a rod.
Once
I received a stoning.
Three times
I suffered shipwreck.
A night and a day
I spent
adrift in
the open sea.
\VS{26}I have been on journeys
many times,
in dangers
from rivers,
in dangers
from robbers,
in dangers
from
my own countrymen,
in dangers
from
Gentiles,
in dangers
in
the city,
in dangers
in
the wilderness,
in dangers
at
sea,
in dangers
from
false brothers,
\VS{27}in hard work
and
toil,
through
many
sleepless nights,
in
hunger
and
thirst,
many times
without food,
in
cold
and
without enough clothing.
\VS{28}Apart from
other things,
there is the daily
pressure
on me
of my anxious concern
for all
the churches.
\VS{29}Who
is weak,
and
I am
not
weak? Who
is led into sin,
and
I
do
not
burn with indignation?
\VS{30}If
I
must
boast,
I will boast
about the things that show my
weakness.
\VS{31}The God
and
Father
of the Lord
Jesus,
who is
blessed
forever,
knows
I am
not
lying.
\VS{32}In
Damascus,
the governor
under King
Aretas
was guarding
the city
of Damascus
in order to arrest
me,
\VS{33}but
I was let down
in
a rope-basket
through
a window
in the city wall,
and
escaped
his
hands.

\par }\Chap{12}{\PP \VerseOne{1}It is necessary
to go on boasting.
Though
it is
not
profitable,
I will go on
to
visions
and
revelations
from
the Lord.
\VS{2}I know
a man
in
Christ
who fourteen
years
ago
(whether
in
the body
or
out
of the body
I do
not
know,
God
knows) was caught up
to
the third
heaven.
\VS{3}And
I know
that this
man
(whether
in
the body
or
apart from
the body
I do
not
know,
God
knows)
\VS{4}was caught up
into
paradise
and
heard
things too sacred
to be put into words,
things that
a person
is
not
permitted
to speak.
\VS{5}On behalf
of such an individual
I will boast,
but
on
my own
behalf
I will
not
boast,
except
about
my weaknesses.
\VS{6}For
even if
I wish
to boast,
I will
not
be
a fool,
for
I would be telling
the truth,
but
I refrain
from this so that no
one
may regard
me
beyond
what
he sees
in me
or
what he hears
from
me,
\VS{7}even
because of the extraordinary character
of the revelations.
Therefore,
so that
I would
not
become arrogant,
a thorn
in the flesh
was given
to me,
a messenger
of Satan
to
trouble
me
– so that
I would
not
become arrogant.
\VS{8}I asked
the Lord
three times
about
this,
that
it would depart
from
me.
\VS{9}But
he said
to me,
“My
grace
is enough
for you,
for
my power
is made perfect
in
weakness.”
So then,
I will boast
most
gladly
about
my weaknesses,
so that
the power
of Christ
may reside
in
me.
\VS{10}Therefore
I am content
with
weaknesses,
with
insults,
with
troubles,
with
persecutions
and
difficulties
for the sake of
Christ,
for
whenever
I am weak,
then
I am
strong.
\par }{\SH The Signs of an Apostle
\par }{\PP \VS{11}I have become
a fool.
You
yourselves
forced
me
to do it, for
I
should
have been commended
by
you.
For
I lack
nothing
in comparison
to those “super-apostles,”
even though
I am
nothing.
\VS{12}Indeed,
the signs
of an apostle
were performed
among
you
with
great
perseverance
by signs
and
wonders
and
powerful deeds.
\VS{13}For
how
were
you treated worse
than
the other
churches,
except
that
I
myself
was
not
a burden
to you? Forgive
me
this
injustice!
\VS{14}Look, for
the third time
I am
ready
to come
to
you,
and
I will
not
be a burden
to you, because
I do
not
want
your
possessions, but
you.
For
children
should
not
have to save up
for their parents,
but
parents
for their children.
\VS{15}Now I
will
most gladly
spend
and
be spent
for
your
lives! If
I love
you
more,
am I to be loved
less?
\VS{16}But
be
that as it may, I
have
not
burdened
you.
Yet
because I was
a crafty person,
I took
you
in
by deceit!
\VS{17}I have
not
taken advantage
of you
through
anyone
I have sent
to
you, have I?
\VS{18}I urged
Titus
to visit you and
I sent
our brother
along with
him. Titus
did
not
take advantage
of you,
did he? Did we
not
conduct
ourselves in the same
spirit? Did we not behave in the same way?
\VS{19}Have you been thinking
all this time
that
we have been defending
ourselves to you? We are speaking
in
Christ
before
God,
and
everything
we do, dear friends,
is to build
you
up.
\VS{20}For
I am afraid
that somehow
when
I come
I will
not
find
you
what I wish,
and
you
will find
me not
what you wish.
I am afraid that somehow
there may be quarreling,
jealousy,
intense anger,
selfish ambition,
slander,
gossip,
arrogance,
and disorder.
\VS{21}I am afraid that when I come
again,
my
God
may humiliate
me
before
you,
and
I will grieve
for many
of those who previously sinned
and
have
not
repented
of
the impurity,
sexual immorality,
and
licentiousness
that
they have practiced.

\par }\Chap{13}{\PP \VerseOne{1}This
is the third time
I am coming
to visit you.
{\BD{
{\QT{
{\IT{By}}}}
the testimony
the testimony}}
{\QT{of two}}
or
three
witnesses
every
matter
will be established.
\VS{2}I said before
when
I was present
the second time
and
now,
though absent,
I say again
to those who sinned previously
and
to all
the rest,
that
if
I come
again,
I will
not
spare anyone,
\VS{3}since
you are demanding
proof
that Christ
is speaking
through
me.
He is
not
weak
toward
you
but
is powerful
among
you.
\VS{4}For
indeed
he was crucified
by
reason of weakness,
but
he lives
because of
God’s
power.
For
we
also
are weak
in
him,
but
we will live
together with
him,
because of
God’s
power
toward
you.
\VS{5}Put
yourselves
to the test
to see if
you are
in
the faith;
examine
yourselves! Or
do you
not
recognize
regarding yourselves
that
Jesus
Christ
is in
you
– unless,
indeed, you
fail the test!
\VS{6}And
I hope
that
you will realize
that
we
have
not failed the test!
\VS{7}Now we pray
to
God
that you may not
do
anything
wrong,
not
so
that we
may appear
to have passed
the test, but
so that
you
may do
what is right
even
if we may appear to have failed the test.
\VS{8}For
we
cannot
do
anything
against
the truth,
but
only for the sake of
the truth.
\VS{9}For
we rejoice
whenever
we
are weak,
but
you
are
strong.
And
we pray
for this: that you
may become fully qualified.
\VS{10}Because of
this
I am writing
these things
while absent,
so that
when
I arrive
I may
not
have to deal
harshly
with you by
using my authority
– the Lord
gave
it to me
for
building up,
not
for
tearing down!
\par }{\SH Final Exhortations and Greetings
\par }{\PP \VS{11}Finally,
brothers and sisters,
rejoice,
set things right,
be encouraged,
agree
with one another, live in peace,
and
the God
of love
and
peace
will be
with
you.
\VS{12}Greet
one another
with
a holy
kiss. All the saints greet you.
\VS{13}The grace of the Lord Jesus Christ and the love of God and the fellowship of the Holy Spirit
be with you
all.
\VS{14}[[EMPTY]]
\par }