\NormalFont\ShortTitle{1 Kings}
{\MT 1 Kings

\par }\ChapOne{1}{\SH Adonijah Tries to Seize the Throne
\par }{\PP \VerseOne{1}King
David
was very old;
even when they covered
him with blankets,
he could not
get warm.
\VS{2}His servants
advised
him, “A young
virgin
must be found
for our master,
the king,
to take care
of the king’s
needs and serve
as his nurse.
She can also sleep
with you and keep our master,
the king,
warm.”
\VS{3}So they looked
through all
Israel
for a beautiful
young woman
and found
Abishag,
a Shunammite,
and brought
her to the king.
\VS{4}The young
woman was very
beautiful;
she became
the king’s
nurse
and served
him, but the king
did not
have sexual relations with her.
\par }{\PP \VS{5}Now Adonijah,
son
of David and Haggith,
was promoting
himself, boasting, “I
will be king!” He managed
to acquire chariots
and horsemen,
as well as fifty
men
to serve as his royal guard.
\VS{6}(Now his father
had never
corrected
him by saying,
“Why
do you do
such things?” He was also
very
handsome
and had been born
right after
Absalom. )
\VS{7}He collaborated
with
Joab
son
of Zeruiah
and with
Abiathar
the priest,
and they supported
him.
\VS{8}But Zadok
the priest,
Benaiah
son
of Jehoiada,
Nathan
the prophet,
Shimei,
Rei,
and David’s
elite warriors
did not
ally themselves with
Adonijah.
\VS{9}Adonijah
sacrificed
sheep,
cattle,
and fattened steers
at
the Stone
of Zoheleth
near
En Rogel.
He invited
all
his brothers,
the king’s
sons,
as well as all
the men
of Judah,
the king’s
servants.
\VS{10}But he did not
invite
Nathan
the prophet,
Benaiah,
the elite warriors,
or his brother
Solomon.
\par }{\PP \VS{11}Nathan
said
to
Bathsheba,
Solomon’s
mother,
“Has it been reported
to you that
Haggith’s
son
Adonijah
has become king
behind our master
David’s
back?
\VS{12}Now
let
me give you some advice
as to how you can save
your life
and your son
Solomon’s
life.
\VS{13}Visit
King
David
and say
to him,
‘My master,
O king,
did you
not
solemnly
promise your servant,
“Surely
your son
Solomon
will be king
after
me; he will sit
on
my throne”? So why
has Adonijah
become king?’
\VS{14}While
you are still
there
speaking
to the king,
I
will arrive
and verify
your report.”
\par }{\PP \VS{15}So Bathsheba
visited
the king
in
his private quarters.
(The king
was very
old,
and Abishag
the Shunammite
was serving
the king.)
\VS{16}Bathsheba
bowed down on
the floor before
the king.
The king
said,
“What do you want?”
\VS{17}She replied
to him, “My master,
you
swore
an oath to your servant
by the
{\ND{Lord}}
your God,
‘Solomon
your son
will be king
after
me and he
will sit
on
my throne.’
\VS{18}But now,
look,
Adonijah
has become king! But
you, my master
the king,
are not
even aware of it!
\VS{19}He has sacrificed many cattle,
steers,
and sheep
and has invited
all
the king’s
sons,
Abiathar
the priest,
and Joab,
the commander
of the army,
but he has not
invited
your servant
Solomon.
\VS{20}Now,
my master,
O
king,
all
Israel
is
watching
anxiously to see
who
is named to succeed
my master
the king
on
the throne.
\VS{21}If
a decision
is not made, when
my master
the king
is buried with
his ancestors,
my son
Solomon
and I
will be
considered state
criminals.”
\par }{\PP \VS{22}Just then, while
she was still
speaking
to the king,
Nathan
the prophet
arrived.
\VS{23}The king
was told,
“Nathan
the prophet
is here.”
Nathan entered
and bowed
before
the king
with his face
to the floor.
\VS{24}Nathan
said,
“My master,
O
king,
did you
announce,
‘Adonijah
will be king
after
me; he will sit
on
my throne’?
\VS{25}For
today
he has gone down
and sacrificed
many cattle,
steers,
and sheep
and has invited
all
the king’s
sons,
the army
commanders,
and Abiathar
the priest.
At this moment
they are having
a feast
in his presence,
and they have declared,
‘Long live
King
Adonijah!’
\VS{26}But
he did not
invite
me
– your servant
– or Zadok
the priest,
or Benaiah
son
of Jehoiada,
or your servant
Solomon.
\VS{27}Has my
master
the king
authorized
this
without
informing
your servants
who
should succeed
my master
the king
on
his throne?”
\par }{\SH David Picks Solomon as His Successor
\par }{\PP \VS{28}King
David
responded, “Summon
Bathsheba!” She came
and stood
before
the king.
\VS{29}The king
swore
an oath: “As certainly as the
{\ND{Lord}}
lives
(he who
has rescued
me
from every
danger),
\VS{30}I will keep
today
the oath I swore
to you by the
{\ND{Lord}}
God
of Israel: ‘Surely
Solomon
your son
will be king
after
me; he
will sit
in my place
on
my throne.’ ”
\VS{31}Bathsheba
bowed
down to the king
with her face
to the floor
and said,
“May my master,
King
David,
live
forever!”
\par }{\PP \VS{32}King
David
said,
“Summon
Zadok
the priest,
Nathan
the prophet,
and Benaiah
son
of Jehoiada.”
They came
before
the king,
\VS{33}and he told
them, “Take
your master’s
servants
with
you, put
my son
Solomon
on
my mule,
and lead him down
to
Gihon.
\VS{34}There
Zadok
the priest
and Nathan
the prophet
will anoint
him king
over
Israel;
then blow
the trumpet
and declare,
‘Long live
King
Solomon!’
\VS{35}Then follow
him up
as he comes
and sits
on
my throne.
He
will be king
in my place;
I have decreed
that he will be
ruler
over
Israel
and Judah.”
\VS{36}Benaiah
son
of Jehoiada
responded
to the king: “So be it! May
the {\ND{Lord}}
God
of my master
the king
confirm it!
\VS{37}As
the {\ND{Lord}}
is with
my master
the king,
so
may he be
with
Solomon,
and may he make him an even greater
king
than
my master
King
David!”
\par }{\PP \VS{38}So Zadok
the priest,
Nathan
the prophet,
Benaiah
son
of Jehoiada,
the Kerethites,
and the Pelethites
went down,
put
Solomon
on
King
David’s
mule,
and led
him to Gihon.
\VS{39}Zadok
the priest
took
a horn
filled with olive oil
from
the tent
and poured
it on Solomon;
the trumpet
was blown and all
the people
declared, “Long live
King
Solomon!”
\VS{40}All
the people
followed
him up,
playing
flutes
and celebrating
so loudly
they made the ground
shake.
\par }{\PP \VS{41}Now Adonijah
and all
his guests
heard
the commotion
just as
they
had finished
eating.
When
Joab
heard
the
sound
of the trumpet,
he asked,
“Why
is there such a noisy commotion
in the city?”
\VS{42}As he was still
speaking,
Jonathan
son
of Abiathar
the priest
arrived.
Adonijah
said,
“Come
in, for
an important
man
like you
must be bringing good news.”
\VS{43}Jonathan
replied
to Adonijah: “No! Our master
King
David
has made
Solomon king.
\VS{44}The king
sent
with
him Zadok
the priest,
Nathan
the prophet,
Benaiah
son
of Jehoiada,
the Kerethites,
and the Pelethites
and they put
him on
the king’s
mule.
\VS{45}Then Zadok
the priest
and Nathan
the prophet
anointed
him king
in Gihon.
They went up
from there
rejoicing,
and the city
is in an uproar. That is
the sound
you hear.
\VS{46}Furthermore,
Solomon
has assumed the royal
throne.
\VS{47}The king’s
servants
have even
come
to congratulate
our master
King
David,
saying,
‘May your God
make Solomon
more famous
than you and make him an even greater
king
than you!’ Then the king
leaned
on
the bed
\VS{48}and said
this: ‘The
{\ND{Lord}}
God
of Israel
is worthy of praise
because today
he has
placed
a successor
on
my throne
and allowed
me to see it.’ ”
\par }{\PP \VS{49}All
of Adonijah’s
guests
panicked;
they jumped up
and rushed off their separate ways.
\VS{50}Adonijah
feared
Solomon,
so he got
up and went
and grabbed
hold of the horns
of the altar.
\VS{51}Solomon
was told,
“Look,
Adonijah
fears
you; see,
he has taken hold
of the horns
of the altar,
saying,
‘May King
Solomon
solemnly promise
me today
that he will not
kill
his servant
with the sword.’ ”
\VS{52}Solomon
said,
“If
he is a loyal subject, not
a hair of his head
will be harmed, but if
he is found
to be a traitor, he will die.”
\VS{53}King
Solomon
sent
men to bring him down
from the altar.
He came
and bowed
down to King
Solomon,
and Solomon
told
him, “Go
home.”

\par }\Chap{2}{\PP \VerseOne{1}When
David
was close
to death,
he told
Solomon
his son:
\VS{2}“I am
about
to die. Be strong
and become
a man!
\VS{3}Do
the
job
the {\ND{Lord}}
your God
has assigned you by
following his
instructions
and obeying
his rules,
commandments,
regulations,
and laws
as written
in the law
of Moses.
Then you will
succeed
in all
you do
and seek to accomplish,
\VS{4}and the
{\ND{Lord}}
will
fulfill
his promise
to me, ‘If
your descendants
watch
their step
and live
faithfully
in my presence
with all
their heart
and being,
then,’ he promised, ‘you will not
fail
to have a successor
on
the throne
of Israel.’
\par }{\PP \VS{5}“You
know
what Joab
son
of Zeruiah
did to me – how he murdered two commanders of the Israelite armies, Abner son of Ner and Amasa son of Jether. During peacetime he struck them down like he would in battle; when he shed their blood as if in battle, he stained his own belt and the sandals on his feet.
\VS{6}Do
to him what you think is appropriate,
but don’t
let him live long
and die a peaceful
death.
\par }{\PP \VS{7}“Treat fairly
the sons
of Barzillai
of Gilead
and provide
for their needs,
because
they helped
me
when I had to flee
from
your brother
Absalom.
\par }{\PP \VS{8}“Note
well, you still have
to contend with
Shimei
son
of Gera,
the Benjaminite
from Bahurim,
who tried to call down upon
me a horrible
judgment when
I went
to Mahanaim.
He
came down
and met
me at the Jordan,
and I solemnly
promised him by the
{\ND{Lord}}, ‘I will not
strike
you down with the sword.’
\VS{9}But now
don’t
treat
him as if
he were innocent. You
are a wise
man
and you know
how
to handle
him; make
sure he has a bloody
death.”
\par }{\PP \VS{10}Then David
passed away
and was buried
in the city
of David.
\VS{11}David
reigned
over
Israel
forty
years;
he reigned
in Hebron
seven
years,
and in Jerusalem
thirty-three
years.
\par }{\SH Solomon Secures the Throne
\par }{\PP \VS{12}Solomon
sat
on
his father
David’s
throne,
and his royal
authority was firmly
solidified.
\par }{\PP \VS{13}Haggith’s
son
Adonijah
visited
Bathsheba,
Solomon’s
mother.
She asked,
“Do you come
in peace?” He answered,
“Yes.”
\VS{14}He added, “I have something
to
say to
you.” She replied,
“Speak.”
\VS{15}He said,
“You
know
that
the kingdom
was mine and all
Israel
considered
me king.
But then
the kingdom
was
given to my brother,
for
the {\ND{Lord}}
decided it should be his.
\VS{16}Now
I’d like to ask
you for just one
thing.
Please
don’t
refuse
me.” She said,
“Go
ahead
and ask.”
\VS{17}He said,
“Please
ask
King
Solomon
if
he would give
me Abishag
the Shunammite
as a wife,
for
he won’t
refuse you.”
\VS{18}Bathsheba
replied,
“That’s
fine, I’ll
speak
to
the king on your behalf.”
\par }{\PP \VS{19}So
Bathsheba
visited
King
Solomon
to
speak
to him on
Adonijah’s
behalf. The king
got up
to greet
her, bowed
to her, and then sat
on
his throne.
He ordered
a throne
to be brought
for the king’s
mother,
and she sat
at his right hand.
\VS{20}She said,
“I
would like to ask
you for just one
small
favor.
Please
don’t
refuse
me.” He said, “Go
ahead
and ask,
my mother,
for
I would not
refuse you.”
\VS{21}She said,
“Allow
Abishag
the Shunammite
to be given to your brother
Adonijah
as a wife.”
\VS{22}King
Solomon
answered
his mother,
“Why
just request
Abishag
the Shunammite
for him? Since
he is my older
brother,
you should also request
the kingdom
for
him,
for Abiathar
the priest,
and for Joab
son
of Zeruiah!”
\par }{\PP \VS{23}King
Solomon
then swore
an oath by the
{\ND{Lord}}, “May
God
judge me severely,
if
Adonijah
does not pay for this
request
with his life!
\VS{24}Now,
as certainly as the
{\ND{Lord}}
lives
(he who
made me secure,
allowed me to sit
on
my father
David’s
throne,
and established a dynasty
for
me as
he promised), Adonijah
will be executed
today!”
\VS{25}King
Solomon
then sent
Benaiah
son
of Jehoiada,
and he killed Adonijah.
\par }{\PP \VS{26}The king
then told
Abiathar
the priest,
“Go
back to your property
in Anathoth.
You
deserve
to die,
but
today
I will not
kill
you
because
you did carry
the
ark
of the sovereign

{\ND{Lord}}
before
my father
David
and you suffered
with my father
through
all
his difficult times.”
\VS{27}Solomon
dismissed
Abiathar
from his position as priest
of the {\ND{Lord}}, fulfilling
the decree
of judgment the
{\ND{Lord}}
made
in Shiloh
against
the family
of Eli.
\par }{\PP \VS{28}When the news
reached
Joab
(for
Joab
had supported
Adonijah,
although he had not
supported
Absalom), he
ran
to
the tent
of the {\ND{Lord}}
and grabbed hold
of the horns
of the altar.
\VS{29}When King
Solomon
heard
that
Joab
had run
to
the tent
of the {\ND{Lord}}
and was right there beside
the altar,
he
ordered
Benaiah
son
of Jehoiada, “Go,
strike
him down.”
\VS{30}When Benaiah
arrived
at
the tent
of the {\ND{Lord}}, he said
to
him, “The king
says,
‘Come out!’ ” But he replied,
“No,
I will die
here!” So Benaiah
sent
word
to the king
and reported
Joab’s
reply.
\VS{31}The king
told
him, “Do
as
he said! Strike
him down
and bury
him. Take away
from me and from my father’s
family
the guilt
of Joab’s
murderous,
bloody deeds.
\VS{32}May the
{\ND{Lord}}
punish him for the blood
he shed; behind my father
David’s
back
he struck down and murdered
with the sword
two
men
who
were more innocent
and morally
upright than he – Abner son of Ner, commander of Israel’s army, and Amasa son of Jether, commander of Judah’s army.
\VS{33}May Joab
and his descendants
be perpetually
guilty of their shed blood,
but may
the {\ND{Lord}}
give perpetual
peace
to
David,
his descendants,
his family,
and his dynasty.”
\VS{34}So
Benaiah
son
of Jehoiada
went up
and executed
Joab; he was buried
at his home
in the wilderness.
\VS{35}The
king
appointed Benaiah
son
of Jehoiada
to take his place
at the head of the army,
and the king
appointed
Zadok
the priest
to take Abiathar’s
place.
\par }{\PP \VS{36}Next
the king
summoned
Shimei
and told
him, “Build
yourself a house
in Jerusalem
and live
there
– but you may not
leave
there
to go anywhere!
\VS{37}If
you ever
do leave
and cross
the Kidron
Valley,
know
for
sure
that you will certainly
die! You will be
responsible
for your own death.”
\VS{38}Shimei
said
to the king,
“My master
the king’s
proposal
is
acceptable.
Your servant
will do
as
you say.”
So Shimei
lived
in Jerusalem
for a long time.
\par }{\PP \VS{39}Three
years
later
two
of Shimei’s
servants
ran away
to
King
Achish
son
of Maacah
of Gath.
Shimei
was told,
“Look,
your servants
are in Gath.”
\VS{40}So Shimei
got
up, saddled
his donkey,
and went
to
Achish
at Gath
to
find
his servants;
Shimei
went
and brought
back his servants
from Gath.
\VS{41}When Solomon
was told
that
Shimei
had gone
from Jerusalem
to Gath
and had then returned,
\VS{42}the king
summoned
Shimei
and said
to him,
“You will recall that I
made you take an oath
by the
{\ND{Lord}},
and I solemnly
warned
you, ‘If you ever
leave
and go
anywhere,
know
for sure
that
you will certainly
die.’
You said
to me,
‘The proposal is
acceptable; I agree
to it.’
\VS{43}Why
then have
you broken
the oath
you made before the
{\ND{Lord}}
and disobeyed
the order I gave you?”
\VS{44}Then the king
said
to
Shimei,
“You
are well aware
of the way
you mistreated
my father
David.
The
{\ND{Lord}}
will punish you for what you did.
\VS{45}But King
Solomon
will be empowered
and David’s
dynasty
will
endure
permanently
before
the {\ND{Lord}}.”
\VS{46}The king
then gave the order
to Benaiah
son
of Jehoiada
who went
and executed
Shimei.

\par }{\PP So Solomon
took firm
control
of the kingdom.

\par }\Chap{3}{\PP \VerseOne{1}Solomon
made an alliance by marriage
with
Pharaoh,
king
of Egypt;
he married
Pharaoh’s
daughter.
He brought
her to
the City
of David
until
he could finish
building
his residence
and the
temple
of the {\ND{Lord}}
and the
wall
around
Jerusalem.
\VS{2}Now
the people
were offering sacrifices
at the high places,
because
in those
days
a temple
had not
yet been built
to honor
the {\ND{Lord}}.
\VS{3}Solomon
demonstrated his loyalty
to the
{\ND{Lord}}
by following
the practices
of his father
David,
except
that he
offered sacrifices
and burned incense
on the high places.
\par }{\PP \VS{4}The king
went
to Gibeon
to offer sacrifices,
for
it
had the most prominent
of the high places.
Solomon
would offer up a thousand
burnt sacrifices
on
the altar
there.
\VS{5}One night
in Gibeon
the {\ND{Lord}}
appeared
to
Solomon
in a dream.
God
said,
“Tell me
what
I should give you.”
\VS{6}Solomon
replied,
“You
demonstrated
great
loyalty
to your servant,
my father
David,
as
he served you faithfully,
properly,
and sincerely. You have maintained
this
great
loyalty
to this
day
by allowing
his son
to sit
on
his throne.
\VS{7}Now,
O
{\ND{Lord}}
my God,
you
have made your servant
king
in
my father
David’s
place,
even though I
am only a young
man
and am inexperienced.
\VS{8}Your servant
stands among
your chosen
people;
they are a great
nation
that is too numerous
to count
or number.
\VS{9}So give
your servant
a discerning
mind
so he can make judicial decisions
for your people
and distinguish
right
from wrong.
Otherwise
no one
is able
to make judicial decisions
for this
great
nation of yours.”
\VS{10}The Lord
was pleased
that
Solomon
made this
request.
\VS{11}God
said
to him,
“Because
you asked
for the ability to make wise judicial
decisions,
and not
for
long
life, or
riches,
or
vengeance
on your
enemies,
\VS{12}I grant
your request,
and give
you a wise
and discerning
mind
superior to that
of anyone who has preceded
or
will succeed you.
\VS{13}Furthermore,
I am giving
you what
you did not
request
– riches
and honor
so that
you will be
the greatest
king
of your generation.
\VS{14}If
you follow
my instructions
by obeying
my rules
and regulations,
just
as your father
David
did,
then I will grant you long
life.”
\VS{15}Solomon
then woke
up and realized
it was a dream.
He went
to Jerusalem,
stood
before
the ark
of the Lord’s
covenant,
offered up
burnt sacrifices,
presented peace offerings,
and held
a feast
for all
his servants.
\par }{\SH Solomon Demonstrates His Wisdom
\par }{\PP \VS{16}Then
two
prostitutes
came
to
the king
and stood
before him.
\VS{17}One
of the women
said,
“My master,
this
woman
and I
live
in the same
house.
I had a baby
while she was with
me in the house.
\VS{18}Then
three
days
after
I had my baby,
this
woman
also
had
a baby.
We
were alone;
there was no
one else
in the house
except
the two
of us.
\VS{19}This woman’s
child
suffocated
during the night
when she rolled
on top of him.
\VS{20}She got up
in the middle
of the night
and took
my son
from my side,
while your servant
was sleeping.
She put
him in her arms,
and put
her dead
son
in my arms.
\VS{21}I got
up in the morning
to nurse
my son,
and there
he was, dead! But when I examined him carefully
in the morning,
I realized
it
was not
my baby.”
\VS{22}The other
woman
said,
“No! My son
is alive;
your son
is dead!” But the first woman replied,
“No,
your son
is dead;
my son
is alive.”
Each presented her case
before
the king.
\par }{\PP \VS{23}The king
said, “One
says, ‘My son
is alive;
your son
is dead,’
while the other
says,
‘No,
your son
is dead;
my son
is alive.’ ”
\VS{24}The king
ordered,
“Get
me a sword!” So they placed
a sword
before
the king.
\VS{25}The king
then said,
“Cut
the living
child
in two,
and give
half
to one
and half
to the other!”
\VS{26}The real mother
spoke
up to
the king,
for
her motherly instincts
were aroused.
She said,
“My master,
give
her the living
child! Whatever you do, don’t
kill
him!” But the other
woman said,
“Neither one
of us will have him! Let them cut
him in two!”
\VS{27}The king
responded,
“Give
the first woman the living
child;
don’t
kill
him.
She
is the mother.”
\VS{28}When all
Israel
heard
about the judicial
decision
which
the king
had rendered, they respected
the king,
for
they realized
that
he possessed supernatural
wisdom
to make
judicial decisions.

\par }\Chap{4}{\PP \VerseOne{1}King
Solomon
ruled
over
all
Israel.
\VS{2}These
were his officials:
\par }{\PP Azariah
son
of Zadok
was the priest.
\par }{\PP \VS{3}Elihoreph
and Ahijah,
the sons
of Shisha,
wrote down what happened.
\par }{\PP Jehoshaphat
son
of Ahilud was in charge of the records.
\par }{\PP \VS{4}Benaiah
son
of Jehoiada
was commander of the army.
\par }{\PP Zadok
and Abiathar
were priests.
\par }{\PP \VS{5}Azariah
son
of Nathan
was supervisor
of the district
governors.
\par }{\PP Zabud
son
of Nathan
was a priest
and adviser
to the king.
\par }{\PP \VS{6}Ahishar
was supervisor
of the palace.
\par }{\PP Adoniram
son
of Abda
was supervisor
of the work crews.
\par }{\PP \VS{7}Solomon
had twelve
district
governors appointed throughout
Israel
who acquired supplies
for the king
and his palace.
Each
was responsible
for one month
in the year.
\VS{8}These
were their names:
\par }{\PP Ben-Hur
was in charge of the hill country
of Ephraim.
\par }{\PP \VS{9}Ben-Deker
was in charge of Makaz,
Shaalbim,
Beth Shemesh,
and Elon Beth Hanan.
\par }{\PP \VS{10}Ben-Hesed
was in charge of Arubboth;
he controlled Socoh
and all
the territory
of Hepher.
\par }{\PP \VS{11}Ben-Abinadab
was in charge of Naphath Dor.
(He was married
to Solomon’s
daughter
Taphath.)
\par }{\PP \VS{12}Baana
son
of Ahilud
was in charge of Taanach
and Megiddo,
as well as all
of Beth Shan
next to
Zarethan
below
Jezreel,
from Beth Shan
to
Abel Meholah
and on past Jokmeam.
\par }{\PP \VS{13}Ben-Geber
was in charge of Ramoth
Gilead;
he controlled the tent villages
of Jair
son
of Manasseh
in Gilead,
as well as the region
of Argob
in Bashan,
including sixty
large
walled
cities
with bronze
bars locking their gates.
\par }{\PP \VS{14}Ahinadab
son
of Iddo
was in charge of Mahanaim.
\par }{\PP \VS{15}Ahimaaz
was in charge of Naphtali.
(He married
Solomon’s
daughter
Basemath.)
\par }{\PP \VS{16}Baana
son
of Hushai
was in charge of Asher
and Aloth.
\par }{\PP \VS{17}Jehoshaphat
son
of Paruah
was in charge of Issachar.
\par }{\PP \VS{18}Shimei
son
of Ela
was in charge of Benjamin.
\par }{\PP \VS{19}Geber
son
of Uri
was in charge of the land
of Gilead
(the territory
which had
once belonged to King
Sihon
of the Amorites
and to King
Og
of Bashan). He was sole governor
of the area.
\par }{\SH Solomon’s Wealth and Fame
\par }{\PP \VS{20}The people of Judah
and Israel
were as innumerable
as the sand
on
the seashore;
they had
plenty
to eat
and drink
and were happy.
\VS{21}Solomon
ruled
all
the kingdoms
from
the Euphrates
River to the land
of the Philistines,
as far
as the border
of Egypt.
These kingdoms paid
tribute
as Solomon’s
subjects
throughout
his lifetime.
\VS{22}Each
day
Solomon’s
royal court consumed
thirty
cors
of finely milled flour,
sixty
cors
of cereal,
\VS{23}ten
calves
fattened
in the stall, twenty
calves
from the pasture,
and a hundred
sheep,
not to mention
rams,
gazelles,
deer,
and well-fed
birds.
\VS{24}His royal court was so large because
he ruled over
all
the kingdoms west
of the Euphrates River
from Tiphsah
to
Gaza;
he was
at peace
with all
his neighbors.
\VS{25}All the people
of Judah
and Israel
had security;
everyone
from Dan
to Beer Sheba
enjoyed
the produce of their vines
and fig trees
throughout
Solomon’s
lifetime.
\VS{26}Solomon
had 4,000
stalls
for his chariot
horses
and 12,000
horses.
\VS{27}The district governors
acquired supplies
for King
Solomon
and all
who ate
in
his royal
palace.
Each
was responsible for one month
in the year; they made sure nothing
was lacking.
\VS{28}Each
one also brought
to
the assigned location
his quota
of barley
and straw
for the various horses.
\par }{\PP \VS{29}God
gave
Solomon
wisdom
and very
great discernment;
the breadth
of his understanding
was as infinite as the sand
on
the seashore.
\VS{30}Solomon
was wiser
than
all
the men
of the east
and all
the sages
of Egypt.
\VS{31}He was wiser
than any
man,
including Ethan
the Ezrahite
or Heman,
Calcol,
and Darda,
the sons
of Mahol.
He was famous
in all
the neighboring
nations.
\VS{32}He composed
3,000
proverbs
and 1,005
songs.
\VS{33}He produced manuals
on
botany,
describing every kind of plant, from
the cedars
of Lebanon
to the hyssop
that
grows
on
walls.
He also produced manuals
on
biology,
describing animals,
birds,
insects,
and fish.
\VS{34}People from all
nations
came
to hear
Solomon’s
display of wisdom;
they came from all
the kings
of the earth
who
heard
about his wisdom.

\par }\Chap{5}{\PP \VerseOne{1} King
Hiram
of Tyre
sent
messengers
to
Solomon
when
he heard
that
he had been anointed
king
in his father’s
place.
(Hiram
had always
been an ally
of David.)
\VS{2}Solomon
then sent
this message
to Hiram:
\VS{3}“You
know
that my father
David
was unable
to build
a temple
to honor
the {\ND{Lord}}
his God,
for he was busy fighting battles
on all fronts
while
the {\ND{Lord}}
subdued his enemies.
\VS{4}But now
the {\ND{Lord}}
my God
has made me secure
on all fronts;
there is no
adversary
or
dangerous
threat.
\VS{5}So I
have decided to build
a temple
to
honor
the {\ND{Lord}}
my God,
as
the {\ND{Lord}}
instructed
my father
David,
‘Your son,
whom
I will put
on
your throne
in your place,
is
the one
who will build
a temple
to honor me.’
\VS{6}So now
order
some cedars
of Lebanon
to be cut
for me. My servants
will
work with
your servants.
I will pay
your servants
whatever
you say
is appropriate, for
you
know
that
we have no
one
among us who knows
how to cut down
trees
like the Sidonians.”
\par }{\PP \VS{7}When
Hiram
heard
Solomon’s
message,
he was very
happy.
He said,
“The
{\ND{Lord}}
is worthy
of praise today
because he has
given
David
a wise
son
to rule over
this
great
nation.”
\VS{8}Hiram
then sent
this message to
Solomon: “I received
the message
you sent
to
me. I
will give
you all
the cedars
and evergreens you need.
\VS{9}My servants
will bring the timber down
from
Lebanon
to the sea.
I
will send
it
by sea
in raft-like
bundles to
the place
you designate.
There
I will separate the logs and you
can carry
them away. In exchange
you
will supply
the food
I need for my royal court.”
\par }{\PP \VS{10}So Hiram
supplied
the cedars
and evergreens
Solomon
needed,
\VS{11}and Solomon
supplied
Hiram
annually
with 20,000
cors
of wheat
as provision
for his royal court,
as well as
20,000 baths
of pure
olive oil.
\VS{12}So the
{\ND{Lord}}
gave
Solomon
wisdom,
as he had
promised
him. And Hiram
and Solomon
were at peace
and made
a treaty.
\par }{\PP \VS{13}King
Solomon
conscripted
work crews
from throughout
Israel,
30,000
men in all.
\VS{14}He sent
them to Lebanon
in shifts
of 10,000
men per month.
They worked
in Lebanon
for one month,
and then spent two
months
at home.
Adoniram
was supervisor
of the work crews.
\VS{15}Solomon
also had
70,000
common laborers
and 80,000
stonecutters
in the hills,
\VS{16}besides
3,300
officials
who
supervised
the workers.
\VS{17}By royal
order
they supplied
large
valuable
stones
in order to build the temple’s
foundation
with chiseled
stone.
\VS{18}Solomon’s
and Hiram’s
construction
workers,
along with men from Byblos,
did the chiseling and prepared
the wood
and stones
for the building
of the temple.

\par }\Chap{6}{\PP \VerseOne{1}In the four
hundred
and eightieth
year
after the Israelites
left
Egypt,
in the fourth
year
of Solomon’s
reign
over
Israel,
during the month
Ziv
(the second
month), he began building
the
{\ND{Lord}}’s
temple.
\VS{2}The temple
King
Solomon
built
for the
{\ND{Lord}}
was 90 feet
long,
30 feet
wide,
and 45 feet
high.
\VS{3}The porch
in
front
of the main hall
of the temple
was 30 feet
long,
corresponding
to the width
of the temple.
It was 15 feet
wide,
extending out
from the front
of the temple.
\VS{4}He made
framed
windows
for the temple.
\VS{5}He built
an extension
all around
the walls
of the temple’s main hall
and holy place
and constructed
side
rooms in it.
\VS{6}The bottom
floor of the extension
was seven and a half feet
wide,
the middle
floor nine feet
wide,
and the third
floor ten and a half feet
wide.
He made
ledges
on the temple’s
outer
walls so the beams would not
have to be inserted
into the walls.
\VS{7}As the temple
was being built,
only stones
shaped
at the quarry
were used;
the sound of hammers,
pickaxes,
or any other
iron
tool
was not
heard
at the temple
while it was being built.
\VS{8}The entrance
to
the bottom
level of side rooms was on the south
side
of the temple;
stairs
went up
to the middle floor
and then on
up to
the third
floor.
\VS{9}He finished
building
the temple
and covered
it
with rafters
and boards
made of cedar.
\VS{10}He built
an extension
all
around
the temple;
it was seven and a half feet
high
and it was attached
to the temple
by cedar
beams.
\VS{11}\par }{\PP The
{\ND{Lord}}
said
to
Solomon:
\VS{12}“As for this
temple
you
are building,
if
you follow
my rules,
observe my regulations,
and obey
all
my commandments,
I will fulfill
through
you the promise
I made
to
your father
David.
\VS{13}I will live
among
the Israelites
and will not
abandon
my people
Israel.”
\par }{\PP \VS{14}So Solomon
finished
building
the temple.
\VS{15}He constructed
the walls
inside
the temple
with
cedar
planks; he paneled
the inside with
wood
from
the floor
of the temple
to the rafters
of the ceiling.
He covered the temple
floor
with
boards
made from the wood of evergreens.
\VS{16}He built
a wall 30 feet
in from the rear
of the temple
as a partition for an inner sanctuary
that would be the most holy place.
He paneled
the wall with cedar
planks
from
the floor
to
the rafters.
\VS{17}The main hall
in front
of the inner sanctuary
was 60 feet long.
\VS{18}The inside
of the temple
was all cedar
and was adorned
with carvings
of round ornaments
and of flowers
in bloom. Everything
was cedar;
no
stones
were visible.
\par }{\PP \VS{19}He prepared
the inner sanctuary
inside
the temple
so that the ark
of the covenant
of the {\ND{Lord}}
could be
placed
there.
\VS{20}The inner sanctuary
was 30 feet
long,
30 feet
wide,
and 30 feet
high.
He plated
it with gold,
as well
as the cedar
altar.
\VS{21}Solomon
plated
the inside
of the temple
with gold.
He hung
golden
chains
in front of
the inner sanctuary
and plated
the inner sanctuary with gold.
\VS{22}He plated
the entire
inside of the temple
with gold,
as well as
the altar
inside the inner sanctuary.
\par }{\PP \VS{23}In the inner sanctuary
he made
two
cherubs
of olive
wood;
each stood
15 feet
high.
\VS{24}Each of the first
cherub’s
wings
was seven and a half feet
long; its entire wingspan
was 15 feet.
\VS{25}The second
cherub
also had a wingspan of 15 feet;
it was identical
to the first
in measurements
and shape.
\VS{26}Each
cherub
stood
15 feet
high.
\VS{27}He put
the
cherubs
in
the inner
sanctuary of the temple.
Their wings
were spread out.
One of the first
cherub’s
wings
touched
one wall
and one of the other
cherub’s
wings
touched
the opposite
wall.
The first cherub’s other wing
touched
the second cherub’s other wing
in the middle
of the room.
\VS{28}He plated
the cherubs
with gold.
\par }{\PP \VS{29}On all
the walls
around
the temple,
inside
and out,
he carved
cherubs,
palm trees,
and flowers
in bloom.
\VS{30}He plated
the floor
of the temple
with gold,
inside
and out.
\VS{31}He made
doors
of olive
wood
at the entrance
to the inner sanctuary;
the pillar
on each doorpost
was five-sided.
\VS{32}On the two
doors
made of olive
wood
he carved
cherubs,
palm trees,
and flowers
in bloom,
and he plated
them with gold.
He plated
the cherubs
and the palm trees
with hammered gold.
\VS{33}In the same
way he made
doorposts
of olive
wood
for the entrance
to the main hall,
only with four-sided pillars.
\VS{34}He also made
two
doors
out of wood
from evergreens;
each
door
had two
folding leaves.
\VS{35}He carved
cherubs,
palm trees,
and flowers
in bloom
and plated
them with gold,
leveled out
over
the carvings.
\VS{36}He built
the inner
courtyard
with three
rows
of chiseled
stones and a row
of cedar
beams.
\par }{\PP \VS{37}In the month
Ziv
of the fourth
year
of Solomon’s reign the foundation
was laid for the
{\ND{Lord}}’s
temple.
\VS{38}In the eleventh
year,
in the month
Bul
(the eighth
month) the temple
was completed
in accordance
with all
its specifications
and blueprints. It took seven
years
to build.

\par }\Chap{7}{\PP \VerseOne{1}Solomon
took
thirteen
years
to build
his palace.
\VS{2}He named it
“The Palace
of the Lebanon
Forest”; it was 150 feet
long,
75 feet
wide,
and 45 feet
high.
It
had four
rows
of cedar
pillars
and cedar
beams
above
the pillars.
\VS{3}The roof
above
the beams supported by the pillars
was also made of cedar;
there were forty-five
beams, fifteen
per row.
\VS{4}There were three
rows
of windows
arranged in sets
of three.
\VS{5}All
of the entrances
were rectangular
in shape
and they were arranged
in sets
of three.
\VS{6}He made
a colonnade
75 feet
long
and 45 feet
wide.
There was a porch
in
front
of this and pillars
and a roof
in
front of the porch.
\VS{7}He also made
a throne
room,
called
“The Hall
of Judgment,”
where
he made judicial decisions.
It was paneled
with cedar
from the floor
to the rafters.
\VS{8}The palace
where
he lived
was constructed in a similar way. He also constructed a palace
like this hall
for Pharaoh’s
daughter,
whom
he
had married.
\VS{9}All
of these
were built with the best
stones,
chiseled
to the right size
and cut
with a saw
on all sides,
from the foundation
to
the edge of the roof
and from the outside
to
the great
courtyard.
\VS{10}The foundation
was made of large
valuable
stones,
measuring either 15 feet
or 12 feet.
\VS{11}Above
the foundation the best
stones,
chiseled
to the right size,
were used along with cedar.
\VS{12}Around
the great
courtyard
were three
rows
of chiseled
stones and one row
of cedar
beams,
like the inner
courtyard
of the
{\ND{Lord}}’s
temple
and the hall
of the palace.
\par }{\SH Solomon Commissions Hiram to Supply the Temple
\par }{\PP \VS{13}King
Solomon
sent
for Hiram
of Tyre.
\VS{14}He was the son
of a widow
from the tribe
of Naphtali,
and his father
was a craftsman
in bronze
from Tyre.
He had the skill
and knowledge
to make
all
kinds of works
of bronze.
He reported
to
King
Solomon
and did
all
the work he was assigned.
\par }{\PP \VS{15}He fashioned
two
bronze
pillars;
each pillar
was 27 feet
high
and 18 feet
in circumference.
\VS{16}He made
two
bronze
tops
for the pillars;
each
was seven-and-a-half feet
high.
\VS{17}The latticework
on the tops of the pillars
was adorned
with ornamental
wreaths
and chains;
the top
of each pillar
had seven
groupings
of ornaments.
\VS{18}When he made
the pillars,
there were two
rows
of pomegranate-shaped
ornaments
around
the latticework
covering
the top of each pillar.
\VS{19}The tops
of the two pillars
in the porch
were shaped
like lilies
and were six feet high.
\VS{20}On
the top
of each
pillar,
right
above
the bulge
beside
the latticework,
there were two hundred
pomegranate-shaped
ornaments
arranged in rows
all the way around.
\VS{21}He set up
the pillars
on the porch
in front of the main hall.
He erected
one pillar
on the right
side and called
it Jakin;
he erected
the other pillar
on the left
side and called
it Boaz.
\VS{22}The tops
of the pillars
were shaped
like lilies.
So the construction
of the pillars
was completed.
\par }{\PP \VS{23}He also made
the large bronze basin
called “The Sea.”
It measured 15 feet
from rim
to
rim,
was circular
in shape,
and stood seven-and-a-half feet
high.
Its circumference
was 45 feet.
\VS{24}Under
the rim
all the way around
it were round
ornaments
arranged in settings
15 feet
long. The ornaments
were in two
rows
and had been cast
with “The Sea.”
\VS{25}“The Sea” stood
on
top
of twelve
bulls.
Three
faced
northward,
three
westward,
three
southward,
and three
eastward.
“The Sea”
was placed
on
top
of them, and they all
faced outward.
\VS{26}It was four fingers
thick
and its rim
was like that
of a cup
shaped
like a lily
blossom.
It could hold
about 12,000 gallons.
\par }{\PP \VS{27}He also made
ten
bronze
movable stands.
Each stand
was six feet
long,
six feet
wide,
and four-and-a-half feet
high.
\VS{28}The stands
were constructed
with frames
between
the joints.
\VS{29}On
these frames
and joints
were ornamental lions,
bulls,
and cherubs.
Under
the lions
and bulls
were decorative
wreaths.
\VS{30}Each
stand
had four
bronze
wheels
with bronze
axles
and four
supports.
Under
the basin
the supports
were fashioned
on each
side
with wreaths.
\VS{31}Inside
the stand
was a round opening
that was a foot-and-a-half deep;
it had a support
that was two and one-quarter feet long.
On the edge
of the opening
were carvings
in square
frames.
\VS{32}The four
wheels
were under
the frames
and the crossbars
of the axles
were connected to the stand.
Each wheel
was two and one-quarter feet
high.
\VS{33}The wheels
were constructed
like chariot
wheels;
their crossbars,
rims,
spokes,
and hubs
were made of cast metal.
\VS{34}Each
stand
had four
supports,
one
per side
projecting out
from the stand.
\VS{35}On top
of each stand
was a round
opening three-quarters of a foot
deep;
there were also supports
and frames
on
top
of the stands.
\VS{36}He engraved
ornamental cherubs,
lions,
and palm trees
on
the plates
of the supports
and frames
wherever there was room,
with wreaths
all around.
\VS{37}He made
the ten
stands
in this
way. All
of them were cast
in one
mold and were identical
in measurements
and shape.
\par }{\PP \VS{38}He also made
ten
bronze
basins,
each of which could hold
about 240 gallons.
Each
basin
was six feet
in diameter; there was one
basin
for each
stand.
\VS{39}He put
five
basins
on
the south
side
of the temple
and five
on
the north
side.
He put
“The
Sea”
on the south
side,
in the southeast corner.
\par }{\PP \VS{40}Hiram
also made
basins,
shovels,
and bowls.
He
finished
all
the work
on the
{\ND{Lord}}’s
temple
he had
been assigned
by King
Solomon.
\VS{41}He made the two
pillars,
the two
bowl-shaped
tops
of the pillars,
the latticework
for the bowl-shaped
tops
of the two pillars,
\VS{42}the four
hundred
pomegranate-shaped
ornaments for the latticework of the two
pillars (each latticework
had two
rows
of these ornaments
at the bowl-shaped
top
of the pillar),
\VS{43}the ten
movable stands with their ten basins,
\VS{44}the big bronze basin
called “The Sea”
with its twelve
bulls
underneath,
\VS{45}and the pots,
shovels,
and bowls.
All
these
items
King
Solomon
assigned Hiram
to make
for the
{\ND{Lord}}’s
temple
were made from polished
bronze.
\VS{46}The king
had them cast
in earth
foundries
in the region
of the Jordan
between
Succoth
and Zarethan.
\VS{47}Solomon
left
all
these items
unweighed;
there were so
many of them they did not
weigh
the bronze.
\par }{\PP \VS{48}Solomon
also
made
all
these items
for the
{\ND{Lord}}’s
temple: the gold
altar,
the gold
table
on
which
was kept the Bread
of the Presence,
\VS{49}the pure
gold
lampstands
at the entrance
to the inner sanctuary
(five
on the right
and five
on the left), the gold
flower-shaped
ornaments, lamps,
and tongs,
\VS{50}the
pure gold bowls,
trimming
shears, basins,
pans,
and censers,
and the gold
door
sockets
for the inner
sanctuary
(the most
holy
place) and for the doors
of the main hall
of the temple.
\VS{51}When King
Solomon
finished
constructing
the
{\ND{Lord}}’s
temple,
he
put
the holy
items that belonged to his father
David
(the
silver,
gold,
and other articles) in the treasuries
of the
{\ND{Lord}}’s
temple.

\par }\Chap{8}{\PP \VerseOne{1}Then
Solomon
convened
in Jerusalem
Israel’s
elders,
all
the leaders
of the Israelite
tribes
and families,
so they could witness the transferal
of the ark
of the
{\ND{Lord}}’s
covenant
from the city
of David
(that
is, Zion).
\VS{2}All
the men
of Israel
assembled
before King
Solomon
during the festival
in the month
Ethanim
(the seventh
month).
\VS{3}When all
Israel’s
elders
had arrived,
the priests
lifted
the ark.
\VS{4}The priests
and Levites
carried
the ark
of the {\ND{Lord}}, the tent
of meeting,
and all
the holy
items
in the tent.
\VS{5}Now King
Solomon
and all
the Israelites
who had assembled
with
him went on
ahead
of the ark
and sacrificed
more sheep
and cattle
than could
be counted
or numbered.
\par }{\PP \VS{6}The priests
brought
the ark
of the
{\ND{Lord}}’s
covenant
to
its assigned place
in
the inner sanctuary
of the temple,
in
the most
holy
place, under
the wings
of the cherubs.
\VS{7}The cherubs’
wings
extended over
the place
where the ark
sat;
the cherubs
overshadowed
the ark
and its poles.
\VS{8}The poles
were so long
their ends
were visible
from
the holy place
in front
of the inner sanctuary,
but they could not
be seen
from beyond that point.
They have remained
there
to this
very day.
\VS{9}There was nothing
in the ark
except
the two
stone
tablets
Moses
had
placed
there
in Horeb.
It was there that
the {\ND{Lord}}
made an agreement
with
the Israelites
after he brought them out
of the land
of Egypt.
\VS{10}Once the priests
left
the holy
place, a cloud
filled
the
{\ND{Lord}}’s
temple.
\VS{11}The priests
could
not
carry out their duties
because
of the cloud;
the
{\ND{Lord}}’s
glory
filled
his temple.
\par }{\PP \VS{12}Then
Solomon
said,
“The
{\ND{Lord}}
has said
that he lives
in thick darkness.
\VS{13}O
{\ND{Lord}}, truly
I have built
a lofty
temple
for you, a place
where you can live
permanently.”
\VS{14}Then the king
turned
around and pronounced a blessing
over the whole
Israelite
assembly
as they stood there.
\VS{15}He said,
“The
{\ND{Lord}}
God
of Israel
is worthy
of praise because he has fulfilled
what
he promised
my father
David.
\VS{16}He told David, ‘Since
the day
I brought
my people
Israel
out of Egypt,
I have not
chosen
a city
from all
the tribes
of Israel
to build
a temple
in which
to live.
But I have chosen
David
to lead
my people
Israel.’
\VS{17}Now
my father
David
had a strong desire
to build
a temple
to honor
the {\ND{Lord}}
God
of Israel.
\VS{18}The
{\ND{Lord}}
told
my father
David,
‘It is right
for
you to have
a strong desire
to build
a temple
to honor me.
\VS{19}But
you
will not
build
the temple;
your very own
son
will build
the temple
for my honor.’
\VS{20}The
{\ND{Lord}}
has kept
the promise
he made. I have
taken my father
David’s
place
and have occupied
the throne
of Israel,
as
the {\ND{Lord}}
promised.
I have built
this temple
for the honor
of the {\ND{Lord}}
God
of Israel
\VS{21}and set
up in it a place
for the ark
containing
the covenant
the {\ND{Lord}}
made
with
our ancestors
when he brought them out
of the
land
of Egypt.”
\par }{\SH Solomon Prays for Israel
\par }{\PP \VS{22}Solomon
stood
before
the altar
of the {\ND{Lord}}
in front
of the entire
assembly
of Israel
and spread
out his hands
toward the sky.
\VS{23}He prayed: “O
{\ND{Lord}},
God
of Israel,
there is no
god
like
you in heaven
above
or on
earth
below! You maintain
covenantal
loyalty
to your servants
who obey
you with sincerity.
\VS{24}You have
kept
your word to your servant,
my father
David;
this
very day
you have
fulfilled
what
you promised.
\VS{25}Now,
O
{\ND{Lord}},
God
of Israel,
keep
the promise you made to your servant,
my father
David,
when
you said,
‘You will never
fail
to have a successor
ruling
before
me on
the throne
of Israel,
provided
that your descendants
watch
their step
and serve me
as
you have
done.’
\VS{26}Now,
O God
of Israel,
may the promise you made
to your servant,
my father
David,
be realized.
\par }{\PP \VS{27}“God
does not really
live
on
the earth! Look,
if the sky
and the highest
heaven
cannot
contain
you, how much
less
this
temple
I have
built!
\VS{28}But
respond favorably
to your servant’s
prayer
and his request for help,
O
{\ND{Lord}}
my God.
Answer the desperate
prayer
your servant
is presenting
to you today.
\VS{29}Night
and day
may
you watch
over
this
temple,
the place
where
you promised
you would live.
May you answer
your servant’s
prayer
for
this
place.
\VS{30}Respond
to
the request
of your servant
and your people
Israel
for this
place.
Hear
from
inside your heavenly
dwelling
place
and respond
favorably.
\par }{\PP \VS{31}“When
someone
is accused of sinning
against his neighbor
and the latter pronounces
a curse
on the alleged
offender before
your altar
in this temple, be willing to forgive the accused if the accusation is false.
\VS{32}Listen
from heaven
and make
a just decision
about your servants’
claims. Condemn
the guilty
party,
declare
the other innocent,
and give
both
of them what they deserve.
\par }{\PP \VS{33}“The time will come when your people
Israel
are defeated
by an enemy
because
they sinned
against you. If they come back
to
you, renew
their allegiance
to you, and pray
for your help
in this
temple,
\VS{34}then
listen
from heaven,
forgive
the sin
of your people
Israel,
and bring them back
to
the land
you gave
to their ancestors.
\par }{\PP \VS{35}“The time will come when the skies
are shut up
tightly and no
rain
falls because
your people sinned against
you. When they direct their prayers
toward
this
place,
renew
their allegiance
to you, and turn
away from their sin
because
you punish them,
\VS{36}then
listen
from heaven
and forgive
the sin
of your servants,
your people
Israel.
Certainly
you will then teach
them the right
way
to live and send
rain
on
your land
that
you have given
your people
to possess.
\par }{\PP \VS{37}“The time will come
when
the land
suffers from a famine,
a plague,
blight
and disease,
or a locust
invasion,
or when
their enemy
lays siege
to the cities
of the land,
or when some other type of plague
or epidemic occurs.
\VS{38}When
all
your people
Israel
pray
and ask for help,
as they acknowledge
their pain
and spread out
their hands
toward
this
temple,
\VS{39}then listen
from your heavenly
dwelling place,
forgive
their sin, and act
favorably toward each
one based on
your evaluation
of his motives.
(Indeed
you
are the only one who can correctly
evaluate
the motives
of all
people.)
\VS{40}Then
they will
obey
you throughout
their
lifetimes
as they
live
on
the land
you gave
to our ancestors.
\par }{\PP \VS{41}“Foreigners,
who
do not
belong to your people
Israel,
will come
from a distant
land
because
of your reputation.
\VS{42}When
they hear
about your great
reputation
and your ability
to accomplish mighty
deeds, they will come
and direct
their prayers
toward this
temple.
\VS{43}Then listen
from your heavenly
dwelling place
and answer
all
the prayers
of the foreigners.
Then
all
the nations
of the earth
will acknowledge
your reputation,
obey
you like your people
Israel
do, and recognize
that
this
temple
I built belongs to you.
\par }{\PP \VS{44}“When
you direct
your people
to march out and fight
their enemies,
and they direct
their prayers to
the {\ND{Lord}}
toward
his chosen
city
and this temple
I built
for your honor,
\VS{45}then listen
from heaven
to their prayers
for help
and vindicate them.
\par }{\PP \VS{46}“The time will come when
your people will sin
against you (for
there is no
one
who
is sinless!) and you will be angry
with them and deliver
them over
to their enemies,
who will take
them as prisoners
to
their own land,
whether far away
or
close by.
\VS{47}When your people
come to
their senses
in the land
where
they are held prisoner,
they will repent
and beg for your mercy
in the land
of their imprisonment,
admitting,
‘We have sinned
and gone astray; we have
done evil.’
\VS{48}When they return
to
you with all
their heart
and being
in the land
where
they are held
prisoner,
and direct
their prayers to
you toward
the land
you gave
to their ancestors,
your chosen
city,
and the temple
I built
for your honor,
\VS{49}then listen
from your heavenly
dwelling
place
to their prayers
for help
and vindicate them.
\VS{50}Forgive
all
the rebellious
acts of your sinful
people
and cause
their captors
to have mercy on them.
\VS{51}After
all, they are
your people
and your special
possession
whom
you brought out
of Egypt,
from the middle
of the iron-smelting
furnace.
\par }{\PP \VS{52}“May you be
attentive
to
your servant’s
and
your people
Israel’s
requests for help
and may you respond
to
all
their prayers
to you.
\VS{53}After
all, you
picked
them out
of all
the nations
of the earth
to be your special possession,
just as
you, O sovereign

{\ND{Lord}}, announced
through
your servant
Moses
when you brought
our ancestors
out of Egypt.”
\par }{\PP \VS{54}When
Solomon
finished
presenting
all
these
prayers
and requests
to
the {\ND{Lord}}, he got
up from before
the altar
of the {\ND{Lord}}
where
he had kneeled
and spread
out his hands
toward the sky.
\VS{55}When he stood
up, he pronounced a blessing
over the entire
assembly
of Israel,
saying
in a loud
voice:
\VS{56}“The
{\ND{Lord}}
is worthy
of praise because he has
made
Israel
his people
secure
just
as he promised! Not
one
of all
the faithful promises
he made through
his servant
Moses
is left unfulfilled!
\VS{57}May
the {\ND{Lord}}
our God
be with
us, as
he was with
our ancestors.
May
he not
abandon
us or
leave us.
\VS{58}May he make us
submissive,
so we can follow
all
his instructions
and obey
the commandments,
rules,
and regulations
he commanded
our ancestors.
\VS{59}May
the {\ND{Lord}}
our God be constantly
aware
of these
requests
of mine I have presented
to
him, so that he might vindicate
his servant
and his people
Israel
as the need
arises.
\VS{60}Then
all
the nations
of the earth
will recognize
that
the {\ND{Lord}}
is
the only genuine
God.
\VS{61}May
you demonstrate
wholehearted
devotion
to the
{\ND{Lord}}
our God
by following
his rules
and obeying
his commandments,
as you are presently doing.”
\par }{\SH Solomon Dedicates the Temple
\par }{\PP \VS{62}The king
and all
Israel
with
him were presenting sacrifices
to the
{\ND{Lord}}.
\VS{63}Solomon
offered
as peace offerings
to the
{\ND{Lord}}
22,000
cattle
and 120,000
sheep.
Then the king
and all
the Israelites
dedicated
the
{\ND{Lord}}’s
temple.
\VS{64}That day
the king
consecrated
the
middle
of the courtyard
that
is in front
of the
{\ND{Lord}}’s
temple.
He
offered there
burnt sacrifices,
grain offerings,
and the fat
from the peace offerings,
because
the bronze
altar
that
stood before
the {\ND{Lord}}
was too small
to hold
all these
offerings.
\VS{65}At that time
Solomon
and
all
Israel
with
him celebrated
a festival
before
the {\ND{Lord}}
our God
for two
entire weeks.
This great
assembly
included people from all over the land, from Lebo Hamath
in the north to the Brook
of Egypt in the south.
\VS{66}On
the fifteenth day
after the festival
started, he dismissed
the
people.
They asked God to empower
the
king
and then went
to their homes,
happy
and content
because of all
the good
the {\ND{Lord}}
had
done
for his servant
David
and his people
Israel.


\par }\Chap{9}{\PP \VerseOne{1}After
Solomon
finished
building
the
{\ND{Lord}}’s
temple,
the royal
palace,
and all
the other construction projects he
had planned,
\VS{2}the {\ND{Lord}}
appeared
to
Solomon
a second
time, in the same way he had
appeared
to him
at Gibeon.
\VS{3}The
{\ND{Lord}}
said
to
him, “I have answered
your prayer
and your request
for help
that
you made to me.
I have consecrated
this
temple
you built
by making
it my permanent
home;
I will be
constantly
present there.
\VS{4}You
must serve
me with integrity
and sincerity,
just as your father
David
did. Do
everything
I commanded
and obey
my rules
and regulations.
\VS{5}Then
I will allow your dynasty
to rule
over
Israel
permanently,
just
as I promised
your father
David,
‘You will not
fail
to have a successor
on
the throne
of Israel.’
\par }{\PP \VS{6}“But if
you
or your sons
ever turn
away
from me,
fail
to obey
the regulations and rules
I instructed you
to keep, and decide
to serve
and worship
other
gods,
\VS{7}then I will remove
Israel
from the land
I have given
them, I will abandon this temple
I have consecrated
with my presence,
and Israel
will be
mocked
and ridiculed
among all
the nations.
\VS{8}This
temple
will become
a heap of ruins; everyone
who passes
by it will be shocked
and will hiss
out their scorn, saying,
‘Why
did
the {\ND{Lord}}
do this
to this
land
and this
temple?’
\VS{9}Others will then
answer, ‘Because they abandoned
the {\ND{Lord}}
their God,
who
led
their ancestors
out of Egypt.
They embraced
other
gods
whom they worshiped
and served.
That is why
the {\ND{Lord}}
has brought
all
this
disaster
down on them.’ ”
\par }{\SH Foreign Affairs and Building Projects
\par }{\PP \VS{10}After
twenty
years,
during which
Solomon
built
the
{\ND{Lord}}’s
temple
and the royal
palace,
\VS{11}King
Solomon
gave
King
Hiram
of Tyre
twenty
cities
in the region
of Galilee,
because Hiram
had supplied
Solomon
with cedars,
evergreens,
and all
the gold
he wanted.
\VS{12}When Hiram
went out
from Tyre
to inspect
the cities
Solomon
had
given
him, he was not
pleased with them.
\VS{13}Hiram asked, “Why
did you
give me these
cities,
my friend
?” He called
that area the region
of Cabul,
a name which it has retained
to this
day.
\VS{14}Hiram
had sent
to the king
one hundred
twenty
talents
of gold.
\par }{\PP \VS{15}Here
are the
details
concerning the work crews
King
Solomon
conscripted
to build
the
{\ND{Lord}}’s
temple,
his palace,
the
terrace,
the
wall
of Jerusalem,
and Gezer.
\VS{16}(Pharaoh,
king
of Egypt,
had attacked
and captured
Gezer.
He burned
it and killed
the Canaanites
who lived
in the city.
He gave
it as a wedding present
to his daughter,
who had married
Solomon.)
\VS{17}Solomon
built
up Gezer,
lower
Beth Horon,
\VS{18}Baalath,
Tadmor
in the wilderness,
\VS{19}all
the storage
cities
that belonged
to him,
and the cities
where chariots
and horses
were kept. He built
whatever
he
wanted
in Jerusalem,
Lebanon,
and throughout
his entire
kingdom.
\VS{20}Now several
non-Israelite
peoples
were left
in the land after the conquest of Joshua, including
the Amorites,
Hittites,
Perizzites,
Hivites,
and Jebusites.
\VS{21}Their descendants
remained
in the land
(the Israelites
were unable
to wipe
them out completely). Solomon
conscripted
them for his work crews,
and they
continue in that role
to
this
very
day.
\VS{22}Solomon
did not
assign
Israelites
to these work crews;
the Israelites served
as his soldiers,
attendants,
officers,
charioteers,
and commanders
of his chariot
forces.
\VS{23}These
men
were also in charge
of Solomon’s
work
projects; there were a total of 550
men
who supervised
the workers.
\VS{24}Solomon built
the terrace
as soon as Pharaoh’s
daughter
moved up
from the city
of David
to
the palace
Solomon built for her.
\par }{\PP \VS{25}Three
times
a year
Solomon
offered
burnt offerings
and peace offerings
on
the altar
he had
built
for the
{\ND{Lord}}, burning incense
along with
them before
the {\ND{Lord}}. He made the
temple
his official worship place.
\par }{\PP \VS{26}King
Solomon
also built
ships
in Ezion Geber,
which
is located near
Elat
in the land
of Edom,
on
the shore
of the Red
Sea.
\VS{27}Hiram
sent
his fleet
and some
of his sailors,
who were
well acquainted
with
the sea,
to serve
with
Solomon’s
men.
\VS{28}They sailed
to Ophir,
took
from there
four
hundred
twenty
talents
of gold,
and then brought
them to
King
Solomon.

\par }\Chap{10}{\PP \VerseOne{1}When the queen
of Sheba
heard
about Solomon,
she came
to challenge
him with difficult questions.
\VS{2}She arrived
in Jerusalem
with a great
display
of pomp,
bringing with her camels
carrying
spices,
a very
large quantity
of gold,
and precious
gems.
She visited
Solomon
and discussed
with
him everything
that
was
on
her mind.
\VS{3}Solomon
answered all
her questions;
there was no
question
too
complex
for the king.
\VS{4}When
the queen
of Sheba
saw
for herself Solomon’s
extensive
wisdom,
the palace
he had
built,
\VS{5}the food
in
his
banquet hall,
his servants
and attendants, their robes,
his cupbearers,
and his burnt offerings
which
he presented
in the
{\ND{Lord}}’s
temple,
she was amazed.
\VS{6}She said
to
the king,
“The report
I heard
in my own country
about your wise sayings
and insight
was true!
\VS{7}I did not
believe
these things
until
I came
and saw
them with my own eyes.
Indeed,
I didn’t
hear
even half
the story! Your wisdom
and wealth
surpass what was reported to me.
\VS{8}Your attendants,
who stand
before
you at all times
and hear
your wise sayings,
are truly
happy!
\VS{9}May
the {\ND{Lord}}
your God
be praised
because he favored
you by placing
you on
the throne
of Israel! Because of the
{\ND{Lord}}’s
eternal
love
for Israel,
he made
you king
so
you could
make
just
and right decisions.”
\VS{10}She gave
the king
120
talents
of gold,
a very
large quantity
of spices,
and precious
gems.
The quantity
of spices
the queen
of Sheba
gave
King
Solomon
has never
been matched.
\VS{11}(Hiram’s
fleet,
which
carried
gold
from Ophir,
also
brought
from Ophir
a very
large quantity
of fine timber
and precious
gems.
\VS{12}With the timber
the king
made
supports
for the
{\ND{Lord}}’s
temple
and for the royal
palace
and stringed instruments
for the musicians.
No
one has seen
so
much of this fine timber
to
this
very
day. )
\VS{13}King
Solomon
gave
the queen
of Sheba
everything
she requested,
besides
what
he had
freely offered
her. Then she left
and returned
to her homeland
with her
attendants.
\par }{\SH Solomon’s Wealth
\par }{\PP \VS{14}Solomon
received
666
talents
of gold
per
year,
\VS{15}besides
what he collected
from the merchants,
traders,
Arabian
kings,
and governors
of the land.
\VS{16}King
Solomon
made two hundred
large shields
of hammered
gold;
600
measures of gold
were used
for each
shield.
\VS{17}He also made three
hundred
small shields
of hammered
gold;
three
minas
of gold
were used
for each
of these shields.
The king
placed
them in the Palace
of the Lebanon
Forest.
\par }{\PP \VS{18}The king
made
a large
throne
decorated with ivory
and overlaid
it with pure gold.
\VS{19}There were six
steps
leading up to the throne,
and the back of it was rounded
on top.
The throne
had two
armrests
with a statue of a lion
standing
on each side.
\VS{20}There
were twelve
statues
of lions
on
the six
steps,
one lion at each end of each step. There was nothing
like
it in any
other kingdom.
\par }{\PP \VS{21}All
of King
Solomon’s
cups
were made of gold,
and all
the household items
in the Palace
of the Lebanon
Forest
were made of pure
gold.
There were no
silver
items, for silver was not
considered
very valuable
in Solomon’s
time.
\VS{22}Along
with
Hiram’s
fleet,
the king
had a fleet
of large merchant ships
that sailed the sea.
Once every
three
years
the fleet
came into
port
with cargoes
of gold,
silver,
ivory,
apes,
and peacocks.
\par }{\PP \VS{23}King
Solomon
was wealthier
and wiser
than any
of the kings
of the earth.
\VS{24}Everyone
in the world
wanted
to visit Solomon
to see him
display
his God-given
wisdom.
\VS{25}Year
after
year
visitors
brought
their gifts,
which included items
of silver,
items
of gold,
clothes,
perfume,
spices,
horses,
and mules.
\par }{\PP \VS{26}Solomon
accumulated
chariots
and horses.
He had
1,400
chariots
and 12,000
horses.
He kept
them in assigned cities
and in
Jerusalem.
\VS{27}The king
made
silver
as plentiful in Jerusalem
as stones;
cedar
was
as plentiful
as sycamore fig trees
are in the lowlands.
\VS{28}Solomon
acquired
his horses
from Egypt
and from Que;
the king’s
traders
purchased them from Que.
\VS{29}They paid
600
silver pieces
for each chariot
from Egypt
and 150
silver pieces for each horse.
They also
sold
chariots and horses to all
the kings
of the Hittites
and to the kings
of Syria.

\par }\Chap{11}{\PP \VerseOne{1}King
Solomon
fell in love
with many
foreign
women
(besides Pharaoh’s
daughter), including Moabites,
Ammonites,
Edomites,
Sidonians,
and Hittites.
\VS{2}They came
from
nations
about
which
the
{\ND{Lord}}
had warned
the Israelites,
“You must not
establish friendly relations
with them! If you do, they will surely
shift
your allegiance
to their gods.”
But Solomon
was irresistibly
attracted to them.
\par }{\PP \VS{3}He had 700
royal
wives
and 300
concubines;
his wives
had a powerful
influence over him.
\VS{4}When
Solomon
became old, his wives
shifted
his allegiance
to other
gods;
he was not
wholeheartedly
devoted
to the
{\ND{Lord}}
his God,
as
his father
David had been.
\VS{5}Solomon
worshiped
the Sidonian
goddess
Astarte
and
the detestable
Ammonite
god Milcom.
\VS{6}Solomon
did
evil
in the
{\ND{Lord}}’s
sight;
he did not
remain
loyal
to the
{\ND{Lord}}, like his father
David had.
\VS{7}Furthermore,
on
the hill
east
of Jerusalem
Solomon
built
a high place
for the detestable
Moabite
god Chemosh
and for the detestable
Ammonite
god Milcom.
\VS{8}He built high places
for all
his foreign
wives
so they could burn incense
and make sacrifices
to their gods.
\par }{\PP \VS{9}The
{\ND{Lord}}
was angry
with Solomon
because
he had shifted
his allegiance
away from the
{\ND{Lord}}, the God
of Israel,
who had appeared
to
him on two occasions
\VS{10}and had warned
him
about
this
very thing, so that he would not
follow
other
gods.
But he did not
obey
the
{\ND{Lord}}’s
command.
\VS{11}So the
{\ND{Lord}}
said
to Solomon,
“Because
you insist
on
doing these things
and have not
kept
the covenantal
rules
I gave
you, I will surely tear
the kingdom
away from you and give
it to your servant.
\VS{12}However,
for your father
David’s
sake
I will not
do
this
while you are alive. I will tear
it away from your son’s
hand instead.
\VS{13}But
I will not
tear
away the entire
kingdom;
I will
leave
your son
one
tribe
for my servant
David’s
sake
and for
the sake
of my chosen
city Jerusalem.”
\par }{\PP \VS{14}The
{\ND{Lord}}
brought
against Solomon
an enemy,
Hadad
the Edomite,
a descendant
of the Edomite
king.
\VS{15}During
David’s
campaign
against Edom,
Joab,
the commander
of the army,
while on a mission to bury
the dead,
killed
every
male
in Edom.
\VS{16}For
six
months
Joab
and the entire
Israelite
army stayed
there
until
they had exterminated
every
male
in Edom.
\VS{17}Hadad,
who was only a small
boy
at the time, escaped
with
some of his father’s
Edomite
servants
and headed
for Egypt.
\VS{18}They went
from Midian
to Paran;
they took
some men
from Paran
and went
to Egypt.
Pharaoh,
king
of Egypt,
supplied
him with
a house
and food
and even assigned
him some land.
\VS{19}Pharaoh
liked
Hadad
so well he gave
him his sister-in-law
(Queen
Tahpenes’
sister) as a wife.
\VS{20}Tahpenes’
sister
gave birth
to his son,
named Genubath.
Tahpenes
raised him in
Pharaoh’s
palace;
Genubath
grew up in Pharaoh’s
palace
among
Pharaoh’s
sons.
\VS{21}While in Egypt
Hadad
heard
that
David
had passed away
and that
Joab,
the commander
of the army,
was dead.
So Hadad
asked
Pharaoh,
“Give
me permission to leave
so I can return
to
my homeland.”
\VS{22}Pharaoh
said
to him, “What
do you
lack
here
that makes you want
to go
to your homeland?” Hadad replied, “Nothing,
but
please give me
permission to leave.”
\par }{\PP \VS{23}God
also brought
against Solomon another enemy,
Rezon
son
of Eliada
who
had run away
from his master,
King
Hadadezer
of Zobah.
\VS{24}He gathered
some men
and organized
a raiding band.
When David
tried to kill
them, they went
to Damascus,
where they settled down
and gained control
of the city.
\VS{25}He was
Israel’s
enemy
throughout
Solomon’s
reign
and, like Hadad,
caused
trouble.
He loathed
Israel
and ruled
over
Syria.
\par }{\PP \VS{26}Jeroboam
son
of Nebat,
one of Solomon’s
servants,
rebelled against the king.
He was an Ephraimite
from
Zeredah
whose mother
was a widow
named
Zeruah.
\VS{27}This
is what
prompted
him
to rebel
against the king: Solomon
built
a terrace
and he closed
up a gap
in the wall of the city
of his father
David.
\VS{28}Jeroboam
was a talented
man;
when Solomon
saw
that the young man
was an accomplished
worker,
he made him the leader
of the work crew
from the tribe
of Joseph.
\VS{29}At that time,
when Jeroboam
had left
Jerusalem,
the prophet
Ahijah
the Shilonite
met
him on the road;
the two
of them were alone
in the open country.
Ahijah was wearing a brand
new
robe,
\VS{30}and he
grabbed
the robe
and tore
it into twelve
pieces.
\VS{31}Then he told
Jeroboam,
“Take
ten
pieces,
for
this is what
the {\ND{Lord}}
God
of Israel
says: ‘Look,
I am about to tear
the kingdom
from Solomon’s
hand
and I will give
ten
tribes to you.
\VS{32}He will retain
one
tribe,
for
my servant
David’s
sake
and for the sake
of Jerusalem,
the city
I have chosen out
of all
the tribes
of Israel.
\VS{33}I am taking the kingdom from him because
they have abandoned
me and worshiped
the Sidonian
goddess
Astarte,
the Moabite
god
Chemosh,
and the Ammonite
god
Milcom.
They have not
followed
my instructions
by doing
what I approve
and obeying
my rules
and regulations,
like Solomon’s father
David did.
\VS{34}I will not
take
the
whole
kingdom
from his hand.
I will allow
him to be ruler
for
the rest
of his life
for the sake
of my chosen
servant
David
who
kept
my commandments
and rules.
\VS{35}I will take
the kingdom
from the hand
of his son
and give
ten
tribes to you.
\VS{36}I will leave his son
one
tribe
so
my servant
David’s
dynasty
may continue
to serve
me in Jerusalem,
the city
I have chosen
as my home.
\VS{37}I will select
you; you will rule
over all
you desire
to have and you will be
king
over
Israel.
\VS{38}You must
obey
all
I command
you to do, follow
my instructions,
do
what I approve,
and keep
my rules
and commandments,
like my servant
David
did.
Then I will be
with
you and establish
for you a lasting
dynasty,
as I did
for David;
I will give
you Israel.
\VS{39}I will humiliate
David’s
descendants
because
of this,
but not
forever.”
\VS{40}Solomon
tried
to kill
Jeroboam,
but Jeroboam
escaped
to Egypt
and found refuge with King
Shishak
of Egypt.
He stayed in Egypt
until
Solomon
died.
\par }{\SH Solomon’s Reign Ends
\par }{\PP \VS{41}The rest
of the events
of Solomon’s
reign, including all
his accomplishments
and his wise
decisions, are
recorded
in the scroll
called the Annals
of Solomon.
\VS{42}Solomon
ruled
over
all
Israel
from Jerusalem
for forty
years.
\VS{43}Then Solomon
passed away
and was buried
in the city
of his father
David.
His son
Rehoboam
replaced
him as king.

\par }\Chap{12}{\PP \VerseOne{1}Rehoboam
traveled
to Shechem,
for
all
Israel
had gathered
in Shechem
to make Rehoboam king.
\VS{2}When
Jeroboam
son
of Nebat
heard
the news, he was still
in Egypt,
where
he had fled
from
King
Solomon
and had been living ever since.
\VS{3}They sent
for him, and Jeroboam
and the whole
Israelite
assembly
came
and spoke
to
Rehoboam,
saying,
\VS{4}“Your
father
made
us work
too hard.
Now
if you
lighten
the demands he made
and don’t make
us work as
hard, we will serve you.”
\VS{5}He said
to
them, “Go
away
for three
days,
then return
to me.”
So the people
went away.
\par }{\PP \VS{6}King
Rehoboam
consulted
with
the older
advisers who had
served
his father
Solomon
when
he had been alive.
He asked
them, “How
do you
advise
me to answer
these
people?”
\VS{7}They said
to
him, “Today
if
you show a willingness to help
these
people
and grant
their request,
they will be
your servants
from this time forward.”
\VS{8}But Rehoboam rejected
their advice
and consulted
the young advisers
who served him, with
whom
he had
grown up.
\VS{9}He asked
them, “How
do you
advise
me to respond
to these
people
who
said
to me,
‘Lessen
the demands
your father
placed
on us’?”
\VS{10}The young advisers
with whom
Rehoboam had grown up
said
to him, “Say
this
to these
people
who
have said
to
you, ‘Your father
made us work hard, but now lighten
our burden.’
Say
this
to
them: ‘I am a lot harsher
than my father!
\VS{11}My father
imposed
heavy
demands
on
you; I
will make
them even heavier.
My father
punished
you with ordinary whips;
I
will punish
you with whips that really sting your flesh.’ ”
\par }{\PP \VS{12}Jeroboam
and all
the people
reported to Rehoboam
on
the third
day,
just as
the king
had ordered when he said,
“Return
to
me on
the third
day.”
\VS{13}The king
responded
to the
people
harshly.
He rejected
the advice
of the older men
\VS{14}and followed
the advice
of the
younger ones.
He said,
“My father
imposed heavy
demands
on
you; I
will make
them even heavier.
My father
punished
you with ordinary whips;
I
will punish you
with whips
that really sting your flesh.”
\VS{15}The king
refused
to listen
to
the people,
because
the {\ND{Lord}}
was
instigating this turn of events
so that
he might bring to pass the prophetic
announcement
he had made through
Ahijah
the Shilonite
to
Jeroboam
son
of Nebat.
\par }{\PP \VS{16}When
all
Israel
saw that
the king
refused
to listen
to them,
the people
answered
the king,
“We
have no portion
in David,
no
share
in the son
of Jesse! Return
to your homes,
O Israel! Now,
look
after your own dynasty,
O David!” So
Israel
returned to their homes.
\VS{17}(Rehoboam
continued to rule
over
the Israelites
who lived
in the cities
of Judah.)
\VS{18}King
Rehoboam
sent
Adoniram,
the supervisor
of the work crews,
out after them, but all
Israel
stoned
him to death.
King
Rehoboam
managed
to jump into
his chariot
and escape
to Jerusalem.
\VS{19}So Israel
has been in rebellion against
the Davidic
dynasty
to this
very day.
\VS{20}When
all
Israel
heard
that
Jeroboam
had returned,
they summoned
him to
the assembly
and made him king
over
all
Israel.
No
one except
the tribe
of Judah
remained
loyal to the Davidic
dynasty.
\par }{\PP \VS{21}When Rehoboam
arrived
in Jerusalem,
he summoned
180,000
skilled
warriors
from all
of Judah
and the tribe
of Benjamin
to attack
Israel
and restore
the kingdom
to Rehoboam
son
of Solomon.
\VS{22}But God
told
Shemaiah
the prophet,
\VS{23}“Say
this to
King
Rehoboam
son
of Solomon
of Judah,
and to
all
Judah
and Benjamin,
as well as the rest
of the people,
\VS{24}‘The
{\ND{Lord}}
says
this: “Do not
attack
and make war
with
your brothers,
the Israelites.
Each
of you go
home,
for
I have caused
this
to happen.” ’” They obeyed
the

{\ND{Lord}}
and went
home
as
the
{\ND{Lord}}
had ordered them to do.
\par }{\SH Jeroboam Makes Golden Calves
\par }{\PP \VS{25}Jeroboam
built
up Shechem
in the Ephraimite
hill country
and lived
there. From there
he went out
and built
up Penuel.
\VS{26}Jeroboam
then thought to himself: “Now
the Davidic
dynasty
could regain
the kingdom.
\VS{27}If
these
people
go up
to offer
sacrifices
in the
{\ND{Lord}}’s
temple
in Jerusalem,
their loyalty could shift
to
their former master,
King
Rehoboam
of Judah.
They might kill
me and return
to
King
Rehoboam
of Judah.”
\VS{28}After the king
had consulted
with his advisers, he made
two
golden
calves.
Then he said
to
the people, “It is too much
trouble for you to go up
to Jerusalem.
Look,
Israel,
here are your gods
who
brought you up
from the land
of Egypt.”
\VS{29}He put
one
in Bethel
and the
other
in Dan.
\VS{30}This
caused
Israel to sin;
the people
went
to Bethel and Dan to worship the calves.
\par }{\PP \VS{31}He built
temples
on the high places
and appointed
as priests
people
who
were not
Levites.
\VS{32}Jeroboam
inaugurated
a festival
on the fifteenth
day
of the eighth
month,
like the festival
celebrated in Judah.
On
the altar
in Bethel
he offered sacrifices
to the calves
he had
made.
In Bethel
he also
appointed
priests
for the high places
he had
made.
\par }{\SH A Prophet from Judah Visits Bethel
\par }{\PP \VS{33}On the fifteenth
day
of the eighth
month
(a date he had
arbitrarily chosen) Jeroboam offered sacrifices
on
the altar
he had made
in Bethel.
He inaugurated
a festival
for the Israelites
and went up
to the altar
to offer sacrifices.

\par }\Chap{13}{\PP \VerseOne{1}Just then
a prophet
from Judah,
sent by the
{\ND{Lord}}, arrived
in Bethel,
as Jeroboam
was standing
near the altar
ready to offer a sacrifice.
\VS{2}With the authority of the
{\ND{Lord}}
he cried out
against
the altar,
“O altar,
altar! This
is what
the {\ND{Lord}}
says,
‘Look,
a son
named
Josiah
will be born
to the Davidic
dynasty.
He will sacrifice
on
you the priests
of the high places
who offer sacrifices
on
you. Human
bones
will be burned
on you.’ ”
\VS{3}That day
he
also announced
a sign,
“This
is the sign
the {\ND{Lord}}
has predetermined: The altar
will be split
open and the ashes
on it will fall to the ground.”
\VS{4}When
the king
heard
what
the prophet
cried
out against
the altar
in Bethel,
Jeroboam,
standing at the altar,
extended
his hand
and ordered, “Seize
him!” The hand
he had
extended
shriveled
up and he could
not
pull
it back.
\VS{5}The altar
split
open and the ashes
fell from
the altar
to the ground, in fulfillment of the sign
the prophet
had announced
with the
{\ND{Lord}}’s authority.
\VS{6}The king
pled
with the prophet, “Seek
the favor
of the
{\ND{Lord}}
your God
and pray
for me,
so that my hand
may be restored.”
So the prophet
sought
the
{\ND{Lord}}’s
favor and the king’s
hand
was restored
to
its former condition.
\VS{7}The king
then said to
the prophet,
“Come
home
with
me
and have something to eat. I’d like to give
a present.”
\VS{8}But the prophet
said
to
the king,
“Even if
you were to give
me half
your possessions,
I could not
go
with
you and eat
and drink
in
this
place.
\VS{9}For
the
{\ND{Lord}}
gave
me strict orders, ‘Do not
eat
or
drink
there and do not
go home
the way
you came.’ ”
\VS{10}So he started back
on another
road;
he did not
travel back
on the same road
he had taken
to
Bethel.
\par }{\PP \VS{11}Now there was an
old
prophet
living
in Bethel.
When his sons
came
home,
they told
their father
everything
the prophet
had
done
in Bethel
that day
and all the words
he had
spoken
to
the king.
\VS{12}Their father
asked
them, “Which
road
did he take?” His sons
showed
him the road
the prophet
from Judah
had taken.
\VS{13}He then told
his sons,
“Saddle
the donkey
for me.” When they had saddled
the donkey
for him, he mounted it
\VS{14}and took off
after
the prophet,
whom he found
sitting
under
an oak tree.
He asked
him,
“Are you
the prophet
from Judah?” He answered,
“Yes, I am.”
\VS{15}He then said
to him,
“Come
home
with
me and eat
something.”
\VS{16}But he replied,
“I can’t
go back
with
you or
eat
and drink
with you
in this
place.
\VS{17}For
the
{\ND{Lord}}
gave me
strict orders, ‘Do not
eat
or
drink
there;
do not
go
back
the way
you came.’ ”
\VS{18}The old prophet then said, “I
too
am a prophet
like
you. An angel
told
me with the
{\ND{Lord}}’s
authority, ‘Bring
him back
with
you to
your house
so he can eat
and drink.’ ”
But he was lying to him.
\VS{19}So the prophet went back
with
him and ate
and drank
in his house.
\par }{\PP \VS{20}While
they
were sitting
at the table,
the {\ND{Lord}}
spoke
through the old prophet
\VS{21}and he cried
out to
the prophet
from Judah,
“This is what
the {\ND{Lord}}
says,
‘You have
rebelled
against the
{\ND{Lord}}
and have not
obeyed
the
command
the {\ND{Lord}}
your God
gave you.
\VS{22}You went back
and ate
and drank
in this place,
even though he said
to
you, “Do not
eat
or
drink
there.” Therefore your corpse
will not
be buried
in your ancestral
tomb.’ ”
\par }{\PP \VS{23}When
the prophet from Judah finished
his meal,
the old prophet
saddled
his visitor’s donkey for him.
\VS{24}As the prophet from Judah was traveling,
a lion
attacked him
on the road
and killed
him. His corpse
was lying
on the road,
and the donkey
and the lion
just stood
there beside
it.
\VS{25}Some
men
came by
and saw
the corpse
lying
in the road
with the
lion
standing
beside
it.
They went
and reported what
they had seen in the city
where
the old
prophet
lived.
\VS{26}When the old prophet
who had
invited him to his house heard
the news, he said,
“It is
the prophet
who
rebelled
against the

{\ND{Lord}}. The
{\ND{Lord}}
delivered
him over to the lion
and it ripped
him up and killed
him, just
as the
{\ND{Lord}}
warned him.”
\VS{27}He told
his sons,
“Saddle
my donkey,”
and they did so.
\VS{28}He went
and found
the
corpse
lying
in the road
with the donkey
and the lion
standing
beside
it;
the lion
had
neither
eaten
the corpse
nor
attacked the
donkey.
\VS{29}The old prophet
picked
up the corpse
of the prophet,
put
it on the donkey,
and brought it back.
The old
prophet
then entered
the city
to
mourn
him and to bury him.
\VS{30}He put
the corpse
into his own tomb,
and they mourned
over
him, saying, “Ah,
my brother!”
\VS{31}After
he buried
him, he said
to
his sons,
“When I die,
bury
me in the
tomb
where
the prophet
is buried;
put
my bones
right beside
his bones,
\VS{32}for
the prophecy
he announced
with the
{\ND{Lord}}’s
authority against
the altar
in Bethel
and against
all
the temples
on the high places
in the cities
of the north
will certainly be
fulfilled.”
\par }{\SH A Prophet Announces the End of Jeroboam’s Dynasty
\par }{\PP \VS{33}After
this
happened,
Jeroboam
still did not
change
his evil
ways;
he continued
to appoint common
people
as priests
at the high places.
Anyone who wanted
the job he consecrated
as a priest.
\VS{34}This
sin
caused
Jeroboam’s
dynasty
to come to an end and to be destroyed
from the face
of the earth.

\par }\Chap{14}{\PP \VerseOne{1}\par }{\PP At that
time
Jeroboam’s
son
Abijah
became sick.
\VS{2}Jeroboam
told
his wife,
“Disguise
yourself so
that people cannot
recognize
you
are Jeroboam’s
wife.
Then go
to Shiloh;
Ahijah
the prophet,
who told
me I would
rule
over
this
nation,
lives there.
\VS{3}Take
ten
loaves
of bread,
some small cakes,
and a container
of honey
and visit
him. He
will tell
you what
will happen
to the boy.”
\par }{\PP \VS{4}Jeroboam’s
wife
did
as
she was told.
She went
to Shiloh
and visited
Ahijah.
Now Ahijah
could
not
see;
he had lost
his eyesight
in his old age.
\VS{5}But the
{\ND{Lord}}
had told
Ahijah,
“Look,
Jeroboam’s
wife
is coming
to find out
from you what
will happen to
her son,
for
he
is sick.
Tell
her so-and-so.
When she comes,
she
will be
in a disguise.”
\VS{6}When
Ahijah
heard
the
sound
of her footsteps
as she came
through the door,
he said,
“Come
on in, wife
of Jeroboam! Why
are you
pretending
to be someone else? I have been commissioned
to
give
you bad news.
\VS{7}Go,
tell
Jeroboam,
‘This is what
the {\ND{Lord}}
God
of Israel
says: “I raised
you up from among
the people
and made
you ruler
over
my people
Israel.
\VS{8}I tore
the kingdom
away from the Davidic
dynasty
and gave
it to you. But you are not
like
my servant
David,
who
kept
my commandments
and followed
me wholeheartedly
by doing
only
what I approve.
\VS{9}You have sinned
more than
all
who came
before
you. You went
and angered
me by making
other
gods,
formed out of metal;
you have completely disregarded me.
\VS{10}So
I am
ready to bring
disaster
on
the dynasty
of Jeroboam.
I will cut off
every last male
belonging to Jeroboam
in Israel,
including even the weak
and incapacitated.
I will burn up
the dynasty
of Jeroboam,
just as
one burns
manure
until
it is completely
consumed.
\VS{11}Dogs
will eat
the members of your family
who die
in the city,
and the birds
of the sky
will eat
the ones who die
in the country.” ’
Indeed,
the {\ND{Lord}}
has announced it!
\par }{\PP \VS{12}“As for you,
get up
and go
home.
When you set
foot
in the city,
the boy
will die.
\VS{13}All
Israel
will mourn
him and bury
him. He is the only
one in Jeroboam’s
family
who will receive
a decent burial,
for
he is the only one in whom the
{\ND{Lord}}
God
of Israel
found
anything
good.
\VS{14}The
{\ND{Lord}}
will raise
up a king
over
Israel
who
will cut off
Jeroboam’s
dynasty.
It is
ready to happen!
\VS{15}The

{\ND{Lord}}
will attack
Israel,
making it like
a reed
that sways
in the water.
He will remove
Israel
from this
good
land
he gave
to their ancestors
and scatter
them beyond
the Euphrates
River, because
they angered
the

{\ND{Lord}}
by making
Asherah poles.
\VS{16}He will hand
Israel
over to their enemies
because of the sins
which
Jeroboam
committed
and which
he made Israel
commit.”
\par }{\PP \VS{17}So Jeroboam’s
wife
got up
and went back
to Tirzah.
As she crossed
the threshold
of the house,
the boy
died.
\VS{18}All
Israel
buried
him and mourned
for him, just
as the
{\ND{Lord}}
had
predicted
through
his servant
the prophet
Ahijah.
\par }{\SH Jeroboam’s Reign Ends
\par }{\PP \VS{19}The rest
of the events
of Jeroboam’s
reign, including
the details of his battles
and rule,
are recorded
in the scroll
called the Annals
of the Kings
of Israel.
\VS{20}Jeroboam
ruled
for twenty-two
years;
then he passed away.
His son
Nadab
replaced
him as king.
\par }{\SH Rehoboam’s Reign over Judah
\par }{\PP \VS{21}Now Rehoboam
son
of Solomon
ruled
in Judah.
He was forty-one
years
old
when he
became king
and he ruled
for seventeen
years
in Jerusalem,
the city
the {\ND{Lord}}
chose
from
all
the tribes
of Israel
to be
his home.
His mother
was an Ammonite
woman named
Naamah.
\par }{\PP \VS{22}Judah
did
evil
in the sight
of the
{\ND{Lord}}. They made him more jealous
by their sins
than
their ancestors
had
done.
\VS{23}They even
built
for themselves
high places,
sacred pillars,
and Asherah poles
on
every
high
hill
and under
every
green
tree.
\VS{24}There were also
male cultic prostitutes
in the land.
They committed
the same horrible
sins as the nations
that
the {\ND{Lord}}
had driven
out from before
the Israelites.
\par }{\PP \VS{25}In King
Rehoboam’s
fifth
year,
King
Shishak
of Egypt
attacked Jerusalem.
\VS{26}He took
away the treasures
of the
{\ND{Lord}}’s
temple
and of the royal
palace;
he took
everything,
including all
the golden
shields
that
Solomon
had made.
\VS{27}King
Rehoboam
made
bronze
shields
to replace
them and assigned
them to the officers
of the royal guard
who protected
the entrance
to the royal
palace.
\VS{28}Whenever
the king
visited
the
{\ND{Lord}}’s
temple,
the royal guard
carried
them and then brought
them back
to
the guardroom.
\par }{\PP \VS{29}The rest
of the events
of Rehoboam’s
reign, including
his accomplishments,
are recorded
in the scroll
called the Annals
of the Kings
of Judah.
\VS{30}Rehoboam
and Jeroboam
were continually
at
war
with each other.
\VS{31}Rehoboam
passed away
and was buried
with
his ancestors
in the city
of David.
His mother
was an Ammonite
named
Naamah.
His son
Abijah
replaced
him as king.

\par }\Chap{15}{\PP \VerseOne{1}In the eighteenth
year
of the reign
of Jeroboam
son
of Nebat,
Abijah
became king
over
Judah.
\VS{2}He ruled
for three
years
in Jerusalem.
His mother
was Maacah,
the daughter
of Abishalom.
\VS{3}He followed
all
the sinful
practices of his father
before
him. He was not
wholeheartedly
devoted
to the
{\ND{Lord}}
his God,
as
his ancestor
David had been.
\VS{4}Nevertheless for
David’s
sake
the {\ND{Lord}}
his God
maintained his dynasty
in Jerusalem
by giving
him
a son
to succeed him
and by protecting
Jerusalem.
\VS{5}He did
this because
David
had
done what he approved
and had
not
disregarded
any
of his commandments
his entire
lifetime,
except
for the incident involving
Uriah
the Hittite.
\VS{6}Rehoboam
and Jeroboam
were continually at
war
with each other
throughout Abijah’s lifetime.
\VS{7}The rest
of the events
of Abijah’s
reign, including all
his
accomplishments,
are recorded
in
the scroll
called the Annals
of the Kings
of Judah.
Abijah
and Jeroboam
had been
at
war
with each other.
\VS{8}Abijah
passed away
and was buried
in the city
of David.
His son
Asa
replaced
him as king.
\par }{\SH Asa’s Reign over Judah
\par }{\PP \VS{9}In the twentieth
year
of Jeroboam’s
reign over
Israel,
Asa
became
the king
of Judah.
\VS{10}He ruled
for forty-one
years
in Jerusalem.
His grandmother
was Maacah
daughter
of Abishalom.
\VS{11}Asa
did what the
{\ND{Lord}}
approved
like his ancestor
David
had done.
\VS{12}He removed
the male cultic prostitutes
from
the land
and got rid
of all
the disgusting idols
his ancestors
had
made.
\VS{13}He also
removed
Maacah
his grandmother
from her position as queen
because she had
made
a loathsome
Asherah pole.
Asa
cut down
her Asherah pole
and burned
it in the Kidron
Valley.
\VS{14}The high places
were not
eliminated,
yet
Asa
was wholeheartedly
devoted
to the
{\ND{Lord}}
throughout
his lifetime.
\VS{15}He brought
the holy
items
that he and his father
had made into
the
{\ND{Lord}}’s
temple,
including the silver,
gold,
and other articles.
\par }{\PP \VS{16}Now
Asa
and King
Baasha
of Israel
were continually at
war
with each other.
\VS{17}King
Baasha
of Israel
attacked
Judah
and established
Ramah
as a military outpost to prevent
anyone from leaving
or entering
the land of King
Asa
of Judah.
\VS{18}Asa
took
all
the silver
and gold
that was left
in the treasuries
of the
{\ND{Lord}}’s
temple
and of the royal
palace
and handed
it to his servants.
He
then told them to
deliver
it to Ben Hadad
son
of Tabrimmon,
the son
of Hezion,
king
of Syria,
ruler
in Damascus,
along with this message:
\VS{19}“I want to make a treaty
with
you, like the one our fathers
made. See,
I have sent
you silver
and gold
as a present.
Break
your treaty
with
King
Baasha
of Israel,
so he will retreat from my land.”
\VS{20}Ben Hadad
accepted
King
Asa’s
offer and ordered
his army
commanders
to
attack the cities
of Israel.
They conquered
Ijon,
Dan,
Abel Beth Maacah,
and all
the territory
of Naphtali,
including the region
of Kinnereth.
\VS{21}When
Baasha
heard
the news, he stopped
fortifying
Ramah
and settled
down in Tirzah.
\VS{22}King
Asa
ordered
all
the men of Judah
(no
exemptions
were granted) to carry
away the
stones
and wood
that
Baasha
had used to build
Ramah.
King
Asa
used the materials to build up Geba
(in Benjamin) and Mizpah.
\par }{\PP \VS{23}The rest
of the events
of Asa’s
reign, including all
his successes
and accomplishments,
as well as
a
record of the cities
he built,
are
recorded
in the scroll
called the Annals
of the Kings
of Judah.
Yet
when he was
very old he developed
a foot disease.
\VS{24}Asa
passed away
and was buried
with
his ancestors
in the city
of his ancestor
David.
His son
Jehoshaphat
replaced
him as king.
\par }{\SH Nadab’s Reign over Israel
\par }{\PP \VS{25}In the second year
of Asa’s
reign over
Judah,
Jeroboam’s
son
Nadab
became the king
of Israel;
he ruled
Israel
for two
years.
\VS{26}He did
evil
in the sight
of the
{\ND{Lord}}. He followed
in his father’s
footsteps
and encouraged
Israel
to sin.
\par }{\PP \VS{27}Baasha
son
of Ahijah,
from the tribe
of Issachar,
conspired
against
Nadab
and assassinated
him in Gibbethon,
which
was in Philistine
territory. This happened while Nadab
and all
the Israelite
army were besieging
Gibbethon.
\VS{28}Baasha
killed
him in the third
year
of Asa’s
reign over
Judah
and replaced him as king.
\VS{29}When
he became king,
he executed
Jeroboam’s
entire
family.
He wiped out
everyone
who breathed,
just as
the {\ND{Lord}}
had
predicted
through
his servant
Ahijah
the Shilonite.
\VS{30}This happened because
of the sins
which
Jeroboam
committed
and which
he made Israel
commit. These sins angered
the {\ND{Lord}}
God
of Israel.
\par }{\PP \VS{31}The rest
of the events
of Nadab’s
reign, including all
his accomplishments,
are recorded
in the scroll
called the Annals
of the Kings
of Israel.
\VS{32}Asa
and King
Nadab
of Israel
were continually
at
war
with each other.
\par }{\SH Baasha’s Reign over Israel
\par }{\PP \VS{33}In the third
year
of Asa’s
reign over
Judah,
Baasha
son
of Ahijah
became king
over
all
Israel
in Tirzah;
he ruled for twenty-four
years.
\VS{34}He did
evil
in the sight
of the
{\ND{Lord}}; he followed
in Jeroboam’s
footsteps
and encouraged
Israel
to sin.

\par }\Chap{16}{\PP \VerseOne{1}Jehu
son
of Hanani
received from the
{\ND{Lord}}
this message
predicting Baasha’s
downfall:
\VS{2}“I raised
you up from
the dust
and made
you ruler
over
my people
Israel.
Yet you followed
in Jeroboam’s
footsteps
and encouraged
my people
Israel
to sin;
their sins have made me angry.
\VS{3}So I am
ready to burn
up Baasha
and his family,
and make
your family
like the family
of Jeroboam
son
of Nebat.
\VS{4}Dogs
will eat
the members of Baasha’s
family who die
in the city,
and the birds
of the sky
will eat
the ones who die
in the country.”
\par }{\PP \VS{5}The rest
of the events
of Baasha’s
reign, including
his accomplishments
and successes,
are
recorded
in the scroll
called the Annals
of the Kings
of Israel.
\VS{6}Baasha
passed away
and was buried
in Tirzah.
His son
Elah
replaced
him as king.
\VS{7}The prophet
Jehu
son
of Hanani
received from the

{\ND{Lord}}
the message
predicting
the downfall of Baasha
and his family
because
of all
the evil
Baasha had
done
in the sight
of the {\ND{Lord}}. His actions
angered
the
{\ND{Lord}} (including the way he had destroyed Jeroboam’s dynasty), so that
his family ended up like Jeroboam’s.
\par }{\SH Elah’s Reign over Israel
\par }{\PP \VS{8}In the twenty-sixth
year
of King
Asa’s
reign
over Judah,
Baasha’s
son
Elah
became king over
Israel;
he ruled in Tirzah
for two years.
\VS{9}His servant
Zimri,
a commander
of half
of his chariot force,
conspired
against
him.
While Elah
was drinking
heavily
at the house
of Arza,
who
supervised
the palace
in Tirzah,
\VS{10}Zimri
came in
and struck
him dead.
(This happened in the twenty-seventh
year
of Asa’s
reign over
Judah.) Zimri replaced
Elah as king.
\VS{11}When he became
king
and occupied
the throne,
he killed
Baasha’s
entire
family.
He did not
spare any male
belonging to him; he killed his relatives
and his friends.
\VS{12}Zimri
destroyed
Baasha’s
entire
family,
just
as the
{\ND{Lord}}
had
predicted
to Baasha
through
Jehu
the prophet.
\VS{13}This happened because of all
the sins
which
Baasha
and his son
Elah
committed
and which
they made Israel
commit.
They angered
the

{\ND{Lord}}
God
of Israel
with their worthless idols.
\par }{\PP \VS{14}The rest
of the events
of Elah’s
reign, including all
his accomplishments,
are recorded
in the scroll
called the Annals
of the Kings
of Israel.
\par }{\SH Zimri’s Reign over Israel
\par }{\PP \VS{15}In the twenty-seventh
year
of Asa’s
reign over
Judah,
Zimri
became king over Israel; he ruled
for seven
days
in Tirzah.
Zimri’s revolt took place while the army
was deployed
in
Gibbethon,
which
was in Philistine territory.
\VS{16}While deployed
there, the army
received
this report: “Zimri
has conspired
against the king
and assassinated
him.” So all
Israel
made Omri,
the commander
of the army,
king over
Israel
that
very day
in the camp.
\VS{17}Omri
and all
Israel
went up
from Gibbethon
and besieged
Tirzah.
\VS{18}When
Zimri
saw
that
the city
was captured,
he went
into the fortified area
of the royal
palace.
He set
the palace
on
fire
and died in the flames.
\VS{19}This happened
because of the sins
he committed.
He did
evil
in the sight
of the
{\ND{Lord}}
and followed
in Jeroboam’s
footsteps
and encouraged
Israel
to continue sinning.
\par }{\PP \VS{20}The rest
of the events
of Zimri’s
reign, including
the details of his revolt,
are recorded
in
the scroll
called the Annals
of the Kings
of Israel.
\par }{\SH Omri’s Reign over Israel
\par }{\PP \VS{21}At that time
the people
of Israel
were divided
in their loyalties. Half
the people
supported
Tibni
son
of Ginath
and wanted to make
him king;
the other half
supported
Omri.
\VS{22}Omri’s
supporters
were stronger
than those
who supported
Tibni
son
of Ginath.
Tibni
died;
Omri
became king.
\par }{\PP \VS{23}In the thirty-first
year
of Asa’s
reign over
Judah,
Omri
became king
over
Israel.
He ruled
for twelve
years,
six
of them in Tirzah.
\VS{24}He purchased
the hill
of Samaria
from Shemer
for two talents
of silver.
He launched a construction project
there
and named
the city
he built
after
Shemer,
the former owner
of the hill
of Samaria.
\VS{25}Omri
did more evil
in the sight
of the
{\ND{Lord}}
than all
who
were before him.
\VS{26}He followed
in the footsteps
of Jeroboam
son
of Nebat
and encouraged
Israel
to sin;
they angered
the

{\ND{Lord}}
God
of Israel
with their worthless idols.
\par }{\PP \VS{27}The rest
of the events
of Omri’s
reign, including
his accomplishments
and successes,
are recorded
in the scroll
called the Annals
of the Kings
of Israel.
\VS{28}Omri
passed away
and was buried
in Samaria.
His son
Ahab
replaced
him as king.
\par }{\SH Ahab Promotes Idolatry
\par }{\PP \VS{29}In the thirty-eighth
year
of Asa’s
reign over
Judah,
Omri’s
son
Ahab
became king
over
Israel.
Ahab
son
of Omri
ruled
over
Israel
for twenty-two
years
in Samaria.
\VS{30}Ahab
son
of Omri
did more evil
in the sight
of the
{\ND{Lord}}
than all
who
were before him.
\VS{31}As
if
following
in the sinful
footsteps of Jeroboam
son
of Nebat
were not bad enough, he married
Jezebel
the daughter
of King
Ethbaal
of the Sidonians.
Then
he worshiped
and bowed
to Baal.
\VS{32}He set
up an altar
for Baal
in the temple
of Baal
he had
built
in Samaria.
\VS{33}Ahab
also made
an Asherah pole;
he
did
more
to anger
the {\ND{Lord}}
God
of Israel
than all
the kings
of Israel
who
were
before him.
\par }{\PP \VS{34}During
Ahab’s reign, Hiel
the Bethelite
rebuilt
Jericho.
Abiram,
his firstborn son,
died when he laid
the foundation;
Segub,
his youngest
son, died when he erected its
gates,
just
as the
{\ND{Lord}}
had warned
through
Joshua
son
of Nun.

\par }\Chap{17}{\PP \VerseOne{1}Elijah
the Tishbite,
from Tishbe
in Gilead,
said
to
Ahab,
“As certainly as the
{\ND{Lord}}
God
of Israel
lives
(whom
I serve), there will be
no
dew
or rain
in the years
ahead unless
I give the command.”
\VS{2}The
{\ND{Lord}}
told
him:
\VS{3}“Leave
here
and travel eastward.
Hide
out in the Kerith
Valley
near
the Jordan.
\VS{4}Drink
from the
stream;
I have already told the ravens
to bring you food
there.”
\VS{5}So
he did
as
the {\ND{Lord}}
told
him; he went
and lived
in the Kerith
Valley
near
the Jordan.
\VS{6}The ravens
would bring
him bread
and meat
each morning
and evening,
and he would drink
from
the stream.
\par }{\PP \VS{7}After
a while,
the stream
dried
up because
there had
been no
rain
in the land.
\VS{8}The
{\ND{Lord}}
told
him,
\VS{9}“Get
up, go
to Zarephath
in Sidonian
territory, and live
there.
I
have already
told
a widow
who lives there
to provide for you.”
\VS{10}So he got up
and went
to Zarephath.
When he went
through the city
gate,
there
was a widow
gathering
wood.
He called out
to
her, “Please
give
me a cup
of water,
so I can take a drink.”
\VS{11}As she went
to get
it, he called
out to
her,
“Please
bring me a piece
of bread.”
\VS{12}She said,
“As certainly as the
{\ND{Lord}}
your God
lives,
I have
no food,
except for a handful
of flour
in a jar
and a little
olive oil
in a jug.
Right now I am
gathering
a couple
of sticks
for
a fire. Then
I’m going
home to make
one final meal for
my son
and myself. After we
have eaten
that, we
will die of starvation.”
\VS{13}Elijah
said
to
her, “Don’t
be afraid.
Go
and do
as
you planned. But first
make a small
cake
for me and bring
it to me; then make
something
for yourself and your son.
\VS{14}For
this is what
the {\ND{Lord}}
God
of Israel
says, ‘The jar
of flour
will not
be empty
and the jug
of oil
will not
run
out until
the day
the
{\ND{Lord}} makes
it
rain
on
the surface
of the ground.’ ”
\VS{15}She went
and did
as Elijah told
her; there was always enough food
for Elijah
and for her and her family.
\VS{16}The jar
of flour
was never
empty
and the jug
of oil
never
ran out,
just
as the
{\ND{Lord}}
had
promised
through
Elijah.
\par }{\PP \VS{17}After
this
the son
of the woman
who owned
the house
got sick.
His illness
was so
severe
he could no
longer
breathe.
\VS{18}She asked
Elijah,
“Why,
prophet,
have you come
to me to
confront
me with my sin
and kill
my son?”
\VS{19}He said
to her,
“Hand
me your son.”
He took
him from her arms,
carried
him to
the upper
room where
he was
staying,
and laid
him down
on
his bed.
\VS{20}Then he called
out to
the
{\ND{Lord}}, “O
{\ND{Lord}}, my God,
are you also
bringing disaster on
this widow
I am
staying
with
by killing
her son?”
\VS{21}He stretched
out over
the boy
three
times
and called
out to
the {\ND{Lord}}, “O
{\ND{Lord}}, my God,
please
let this
boy’s
breath return to him.”
\VS{22}The
{\ND{Lord}}
answered
Elijah’s
prayer; the boy’s
breath returned
to him
and he lived.
\VS{23}Elijah
took
the boy,
brought him down
from
the upper
room to the house,
and handed
him to his mother.
Elijah
then said,
“See,
your son
is alive!”
\VS{24}The woman
said
to
Elijah,
“Now
I know
that
you are a prophet
and that the
{\ND{Lord}}
really
does speak through you.”

\par }\Chap{18}{\PP \VerseOne{1}Some
time
later,
in the third
year
of the famine,
the {\ND{Lord}}
told
Elijah, “Go,
make an appearance
before Ahab,
so I
may send rain
on
the surface
of the ground.”
\VS{2}So Elijah
went
to make an appearance
before Ahab.
\par }{\PP Now the famine
was severe
in Samaria.
\VS{3}So Ahab
summoned
Obadiah,
who supervised
the palace.
(Now Obadiah
was a very
loyal follower
of the {\ND{Lord}}.
\VS{4}When
Jezebel
was killing
the
{\ND{Lord}}’s
prophets,
Obadiah
took
one hundred
prophets
and hid
them in two caves
in two groups of fifty.
He also brought
them food
and water.)
\VS{5}Ahab
told
Obadiah,
“Go
through the land
to
all
the springs
and valleys.
Maybe
we can find
some grazing areas
so we can keep
the horses
and mules
alive
and not
have to kill
some of the animals.”
\VS{6}They divided
up the land
between
them; Ahab
went
one
way
and Obadiah
went
the other.
\par }{\PP \VS{7}As Obadiah
was traveling along,
Elijah
met
him. When he recognized
him, he fell
facedown
to the ground and said,
“Is it really
you, my master,
Elijah?”
\VS{8}He replied,
“Yes,
go
and say
to your master,
‘Elijah is back.’ ”
\VS{9}Obadiah said,
“What
sin have I committed
that
you
are ready to hand
your servant
over
to Ahab
for execution?
\VS{10}As certainly as
the {\ND{Lord}}
your God
lives,
my master
has sent
to every
nation
and kingdom
in an effort to find
you. When they say,
‘He’s not
here,’ he
makes them
swear an oath
that
they could not
find you.
\VS{11}Now
you
say,
‘Go
and say
to your master,
“Elijah is back.” ’
\VS{12}But when
I
leave
you, the
{\ND{Lord}}’s
spirit
will carry
you away so I can’t find
you. If I
go
tell
Ahab
I’ve seen you, he won’t
be able to find
you and he will kill
me. That
would not be
fair, because
your servant
has been a loyal follower
of the
{\ND{Lord}}
from my youth.
\VS{13}Certainly my master
is aware
of what
I did
when
Jezebel
was killing
the
{\ND{Lord}}’s
prophets.
I hid
one hundred
of the
{\ND{Lord}}’s
prophets
in two caves
in two groups of fifty
and I brought
them food
and water.
\VS{14}Now
you
say,
‘Go
and say
to your master,
“Elijah
is back,” ’ but he will kill me.”
\VS{15}But Elijah
said,
“As certainly as the
{\ND{Lord}}
who rules over all
lives
(whom
I serve), I will make an appearance
before
him today.”
\par }{\SH Elijah Confronts Baal’s Prophets
\par }{\PP \VS{16}When Obadiah
went
and informed
Ahab,
the king
went
to meet
Elijah.
\VS{17}When
Ahab
saw
Elijah,
he
said
to him, “Is it really you,
the one who brings disaster
on Israel?”
\VS{18}Elijah replied,
“I have not
brought disaster
on Israel.
But
you
and your father’s
dynasty
have,
by abandoning
the
{\ND{Lord}}’s
commandments
and following
the Baals.
\VS{19}Now
send out
messengers and assemble
all
Israel
before
me at
Mount
Carmel,
as well as the 450
prophets
of Baal
and 400
prophets
of Asherah
whom Jezebel
supports.
\par }{\PP \VS{20}Ahab
sent
messengers to all
the Israelites
and had the prophets
assemble
at Mount
Carmel.
\VS{21}Elijah
approached
all
the people
and said,
“How long
are you
going to be paralyzed
by indecision? If
the {\ND{Lord}}
is the true God,
then
follow
him, but if
Baal
is, follow
him!” But
the people
did not
say a word.
\VS{22}Elijah
said
to them: “I am
the only
prophet
of the {\ND{Lord}}
who is left,
but there are 450
prophets
of Baal.
\VS{23}Let
them bring us two
bulls.
Let them
choose
one
of the bulls
for themselves, cut
it up into pieces, and place
it on
the wood.
But they must not
set
it on fire.
I
will do
the same
to the
other bull
and place
it on
the wood.
But I will not
set
it on fire.
\VS{24}Then
you will invoke
the name
of your god,
and I
will invoke
the name
of the {\ND{Lord}}. The god
who
responds
with fire
will demonstrate
that he is the true God.”
All
the people
responded,
“This
will be
a fair test.”
\par }{\PP \VS{25}Elijah
told
the prophets
of Baal,
“Choose
one
of the bulls
for yourselves and go first,
for
you
are the majority.
Invoke
the name
of your god,
but do not
light
a fire.”
\VS{26}So they took
a bull,
as
he had
suggested,
and prepared
it. They invoked
the name
of Baal
from morning
until
noon,
saying,
“Baal,
answer
us.” But there was no
sound
and no
answer.
They jumped
around on
the altar
they had
made.
\VS{27}At noon
Elijah
mocked
them, “Yell
louder! After all, he is a god;
he may be deep in thought,
or perhaps he stepped out
for
a moment or has taken a trip.
Perhaps
he is sleeping
and needs to be awakened.”
\VS{28}So
they yelled
louder
and, in accordance with their prescribed
ritual, mutilated
themselves with swords
and spears
until
their bodies
were covered with blood.
\VS{29}Throughout the afternoon
they were in an ecstatic
frenzy,
but there was no
sound,
no
answer,
and no
response.
\par }{\PP \VS{30}Elijah
then told all
the people,
“Approach
me.”
So
all
the people
approached
him.
He repaired
the altar
of the {\ND{Lord}}
that had been torn down.
\VS{31}Then
Elijah
took
twelve
stones,
corresponding to the number
of tribes
that descended
from Jacob,
to whom
the {\ND{Lord}}
had said,
“Israel
will be
your new name.”
\VS{32}With
the stones
he constructed
an altar
for
the {\ND{Lord}}. Around
the altar
he made
a trench
large enough to contain
two seahs
of seed.
\VS{33}He arranged
the wood,
cut
up the bull,
and placed
it on
the wood.
\VS{34}Then he said, “Fill four water jars and pour the water on the offering and the wood.” When they had done so, he said,
“Do it again.”
So they did it again.
Then he said,
“Do it a third
time.” So they did it a third time.
\VS{35}The water
flowed down
all sides
of the altar
and filled
the trench.
\VS{36}When
it was time for the evening
offering,
Elijah
the prophet
approached
the altar and prayed: “O
{\ND{Lord}}
God
of Abraham,
Isaac,
and Israel,
prove
today
that
you
are God
in Israel
and that I am
your servant
and have done
all
these
things
at your command.
\VS{37}Answer
me, O
{\ND{Lord}}, answer
me, so these
people
will know
that
you,
O
{\ND{Lord}}, are the true God
and that you
are winning back their allegiance.”
\VS{38}Then fire
from the
{\ND{Lord}}
fell
from the sky. It consumed
the offering,
the wood,
the stones,
and the dirt,
and licked up
the water
in the trench.
\VS{39}When
all
the people
saw this, they threw
themselves down with their faces
to the ground and said,
“The
{\ND{Lord}}
is the true God! The
{\ND{Lord}}
is the true God!”
\VS{40}Elijah
told
them, “Seize
the prophets
of Baal! Don’t
let even one
of them escape!” So they
seized
them, and Elijah
led them down
to
the Kishon
Valley
and executed
them there.
\par }{\PP \VS{41}Then Elijah
told Ahab,
“Go on up
and eat
and drink,
for
the sound
of a heavy
rainstorm can be heard.”
\VS{42}So Ahab
went on up
to eat
and drink,
while Elijah
climbed
to
the top
of Carmel.
He bent
down toward the ground
and put
his face
between
his knees.
\VS{43}He told
his servant,
“Go on up
and look
in the direction
of the sea.”
So he went on up,
looked,
and reported,
“There is
nothing.”
Seven
times Elijah sent him to look.
\VS{44}The seventh
time the servant said,
“Look,
a small
cloud,
the size of the palm of a man’s
hand,
is rising up
from the sea.”
Elijah then said,
“Go
and tell
Ahab,
‘Hitch up
the chariots and go down,
so that the rain
won’t
overtake you.’ ”
\VS{45}Meanwhile
the sky
was covered with dark
clouds,
the wind
blew, and there was
a heavy rainstorm.
Ahab
rode
toward Jezreel.
\VS{46}Now
the {\ND{Lord}}
energized
Elijah
with power; he tucked
his robe into his belt
and ran
ahead
of Ahab
all the way
to Jezreel.

\par }\Chap{19}{\PP \VerseOne{1}Ahab
told
Jezebel
all
that
Elijah
had done,
including a detailed account
of how he killed
all
the prophets
with the sword.
\VS{2}Jezebel
sent
a messenger
to
Elijah
with this
warning, “May
the
gods
judge me severely
if
by this time
tomorrow
I
do
not take
your life
as you did
theirs!”
\par }{\PP \VS{3}Elijah was afraid, so he got
up and fled
for his life
to Beer Sheba
in Judah.
He left
his servant
there,
\VS{4}while
he went
a day’s
journey
into the desert.
He went
and sat
down under
a shrub
and asked
the
{\ND{Lord}} to take
his life: “I’ve
had enough! Now,
O
{\ND{Lord}}, take
my life.
After all, I’m
no
better
than my ancestors.”
\VS{5}He stretched
out and fell asleep
under
the shrub.
All of a sudden
an angelic
messenger
touched
him and said,
“Get
up and eat.”
\VS{6}He looked
and right
there by his head
was a cake
baking on hot coals
and a jug
of water.
He ate
and drank
and then slept some more.
\VS{7}The
{\ND{Lord}}’s
angelic messenger
came back again,
touched
him, and said,
“Get
up and eat,
for
otherwise you won’t
be able
to make the journey.”
\VS{8}So he got
up and ate
and drank.
That meal
gave him
the strength
to travel
forty
days
and forty
nights
until
he reached Horeb,
the mountain
of God.
\par }{\PP \VS{9}He went
into
a cave
there
and spent
the night. All of a sudden
the {\ND{Lord}}
spoke to
him, “Why
are you here,
Elijah?”
\VS{10}He answered, “I have been absolutely
loyal
to the
{\ND{Lord}}, the sovereign
God,
even though
the Israelites
have abandoned
the agreement
they made with you, torn
down your altars,
and killed
your prophets
with the sword.
I alone
am
left
and now they want
to take
my life.”
\VS{11}The
{\ND{Lord}} said,
“Go out
and stand
on the mountain
before
the {\ND{Lord}}. Look,
the {\ND{Lord}}
is ready to pass
by.”
\par }{\PP A very
powerful
wind
went before
the
{\ND{Lord}}, digging
into the mountain
and causing landslides,
but the
{\ND{Lord}}
was not
in the wind.
After
the windstorm
there was an earthquake,
but the
{\ND{Lord}}
was not
in the earthquake.
\VS{12}After
the earthquake,
there was a fire,
but the
{\ND{Lord}}
was not
in the fire.
After
the fire,
there was
a soft
whisper.
\VS{13}When
Elijah
heard
it, he covered
his face
with his robe
and went out
and stood
at the entrance
to the cave.
All of a sudden
a voice
asked
him,
“Why
are you here,
Elijah?”
\VS{14}He answered, “I have been
absolutely
loyal to the
{\ND{Lord}}, the sovereign
God,
even though
the Israelites
have abandoned
the agreement
they made with you, torn
down your altars,
and killed
your prophets
with the sword.
I alone
am
left
and now they want
to take
my life.”
\VS{15}The
{\ND{Lord}}
said
to
him, “Go
back
the way
you came
and then head for the Desert
of Damascus.
Go and anoint
Hazael
king
over
Syria.
\VS{16}You must anoint
Jehu
son
of Nimshi
king
over
Israel,
and Elisha
son
of Shaphat
from Abel Meholah
to take your place
as prophet.
\VS{17}Jehu
will kill
anyone who escapes
Hazael’s
sword,
and Elisha
will kill
anyone who escapes
Jehu’s
sword.
\VS{18}I still have left
in Israel
seven
thousand
followers who
have not
bowed their knees
to Baal
or
kissed
the images of him.”
\par }{\PP \VS{19}Elijah went
from there
and found
Elisha
son
of Shaphat.
He was
plowing
with twelve
pairs
of oxen; he
was near
the twelfth
pair. Elijah
passed
by
him and threw
his robe
over him.
\VS{20}He left
the
oxen,
ran
after
Elijah,
and said,
“Please
let me kiss
my father
and mother
goodbye, then
I will follow
you.” Elijah said
to him, “Go
back! Indeed,
what
have I done to you?”
\VS{21}Elisha went back
and took
his pair
of oxen
and slaughtered
them. He cooked
the meat
over a fire that he made by burning the harness and yoke. He gave
the people
meat and they ate.
Then he got
up and followed
Elijah
and became his assistant.

\par }\Chap{20}{\PP \VerseOne{1}Now King
Ben Hadad
of Syria
assembled
all
his army,
along with thirty-two
other kings
with
their horses
and chariots.
He marched
against
Samaria
and besieged
and attacked it.
\VS{2}He sent
messengers
to
King
Ahab
of Israel,
who was in the city.
\VS{3}He said to him, “This is what Ben Hadad says, ‘Your silver
and your gold
are
mine, as well as the best
of your wives
and sons.’ ”
\VS{4}The king
of Israel
replied,
“It is just as
you say,
my master,
O king.
I
and all
I own belong to you.”
\par }{\PP \VS{5}The messengers
came again
and said,
“This is what
Ben Hadad
says,
‘I sent
this message
to you, “You must give
me
your silver,
gold,
wives,
and sons.”
\VS{6}But
now
at this time
tomorrow
I will send
my servants
to
you and they will search
through your palace
and your servants’
houses.
They will
carry away all
your valuables.”
\VS{7}The king
of Israel
summoned
all
the leaders
of the land
and said,
“Notice
how this man is
looking
for
trouble.
Indeed,
he demanded
my wives,
sons,
silver,
and gold,
and I did not
resist
him.”
\VS{8}All
the leaders
and people
said
to
him, “Do not
give in
or
agree
to his demands.”
\VS{9}So he
said
to the messengers
of Ben Hadad,
“Say
this to my master,
the king,
‘I will give you everything
you demanded
at first
from your servant,
but
I am unable
to agree
to this
latest
demand.’ ” So
the messengers
went
back
and gave their report.
\par }{\PP \VS{10}Ben Hadad
sent
another message
to
him, “May the
gods
judge
me severely
if
there is enough
dirt
left in Samaria
for my soldiers
to scoop up
in their hands.”
\VS{11}The king
of Israel
replied,
“Tell
him the one who puts on his battle gear
should not
boast
like one who is taking it off.”
\VS{12}When
Ben Hadad received
this
reply,
he and the other kings
were drinking
in their quarters.
He ordered
his servants,
“Get ready
to
attack!” So
they got ready
to attack
the city.
\par }{\SH The Lord Delivers Israel
\par }{\PP \VS{13}Now
a prophet
visited
King
Ahab
of Israel
and said,
“This is what
the {\ND{Lord}}
says,
‘Do you see
this
huge
army? Look,
I am going to hand
it over to you this very day.
Then you will know
that
I am
the {\ND{Lord}}.’ ”
\VS{14}Ahab
asked,
“By whom
will this be accomplished?” He answered,
“This is what
the {\ND{Lord}}
says,
‘By the servants
of the district
governors.’ ”
Ahab asked,
“Who
will launch
the attack?” He answered,
“You will.”
\par }{\PP \VS{15}So Ahab assembled
the 232
servants
of the district
governors.
After
that he assembled
all
the Israelite
army,
numbering 7,000.
\VS{16}They marched
out at noon,
while Ben Hadad
and the thirty-two
kings
allied with him
were drinking
heavily
in their quarters.
\VS{17}The servants
of the district
governors
led
the march.
When Ben Hadad
sent
messengers, they reported back
to him, “Men
are marching out
of Samaria.”
\VS{18}He ordered,
“Whether
they come
in peace
or
to do battle,
take
them alive.”
\VS{19}They marched out
of the city
with the servants
of the district
governors
in the lead and the army
behind them.
\VS{20}Each one
struck
down an
enemy soldier;
the Syrians
fled
and Israel
chased
them. King
Ben Hadad
of Syria
escaped
on
horseback
with some horsemen.
\VS{21}Then
the king
of Israel
marched out and struck
down the horses
and chariots;
he thoroughly
defeated
Syria.
\par }{\SH The Lord Gives Israel Another Victory
\par }{\PP \VS{22}The prophet
visited
the king
of Israel
and instructed
him, “Go,
fortify
your defenses. Determine
what
you must do,
for
in the spring
the king
of Syria
will attack you.”
\VS{23}Now the advisers
of the king
of Syria
said
to him: “Their God
is a god
of the mountains.
That’s why
they overpowered
us. But if
we fight
them in
the plains,
we will certainly
overpower
them.
\VS{24}So do
this: Dismiss the kings
from their command, and replace
them with military commanders.
\VS{25}Muster
an army
like the one
you
lost,
with the same number
of horses
and chariots.
Then we will fight
them in the plains;
we will certainly
overpower
them.”
He approved
their plan
and did
as they advised.
\par }{\PP \VS{26}In the spring
Ben Hadad
mustered
the Syrian
army and marched
to Aphek
to fight
Israel.
\VS{27}When
the Israelites
had mustered
and had received their supplies,
they marched out
to face them
in
battle. When the Israelites
deployed
opposite
them, they were like two
small flocks
of goats,
but the Syrians
filled
the
land.
\VS{28}The prophet
visited
the king
of Israel
and said,
“This is what
the {\ND{Lord}}
says: ‘Because
the Syrians
said,
“The
{\ND{Lord}}
is a god
of the mountains
and not
a god
of the valleys,”
I will
hand
over to you this
entire
huge
army. Then you will know
that
I am
the {\ND{Lord}}.’ ”
\par }{\PP \VS{29}The armies were
deployed
opposite
each other
for seven
days.
On
the seventh
day
the battle
began,
and the Israelites
killed
100,000
Syrian
foot
soldiers in one
day.
\VS{30}The remaining
27,000
ran
to Aphek
and went into
the city,
but the wall
fell
on
them.
Now Ben Hadad
ran
into
the city
and hid
in an inner room.
\VS{31}His advisers
said
to
him, “Look,
we
have heard
that
the kings
of the Israelite
dynasty
are kind.
Allow us
to put
sackcloth
around our waists
and ropes
on our heads
and surrender
to
the king
of Israel.
Maybe
he will spare
our lives.”
\VS{32}So
they put sackcloth
around their waists
and ropes
on their heads
and went
to
the king
of Israel.
They said,
“Your servant
Ben Hadad
says,
‘Please
let me
live!’ ” Ahab replied,
“Is he still
alive? He is my brother.”
\VS{33}The men
took this as a good omen
and quickly
accepted
his offer, saying,
“Ben Hadad
is your brother.”
Ahab then said,
“Go,
get
him.” So Ben Hadad
came out
to
him, and Ahab pulled
him up
into
his chariot.
\VS{34}Ben Hadad said,
“I will return
the cities
my father
took
from your father.
You may set up markets
in Damascus,
just
as my father
did in Samaria.”
Ahab then said, “I
want to make a treaty
with you before I dismiss
you.” So
he made
a treaty
with him and then dismissed him.
\par }{\SH A Prophet Denounces Ahab’s Actions
\par }{\PP \VS{35}One
of the members of the prophetic guild,
speaking with divine authority, ordered
his companion,
“Wound
me!” But the man
refused
to wound him.
\VS{36}So the prophet said
to him, “Because
you have
disobeyed
the {\ND{Lord}}, as soon
as you leave
me
a lion
will kill
you.” When
he left
him,
a lion
attacked and killed him.
\VS{37}He found
another
man
and said,
“Wound
me!” So the man
wounded
him severely.
\VS{38}The prophet
then went
and stood
by the road,
waiting for the king.
He also disguised
himself by putting a bandage
down over
his eyes.
\VS{39}When
the king
passed
by, he called out
to
the king,
“Your servant
went out
into the heat
of the battle,
and then
a man
turned aside
and brought
me
a prisoner.
He told
me,
‘Guard
this
prisoner.
If
he ends up missing
for
any reason,
you will
pay
with your life
or
with a talent
of silver.’
\VS{40}Well, it just
so happened
that while your servant
was doing this
and that, he
disappeared.”
The king
of Israel
said
to him,
“Your punishment
is already determined
by your
own testimony.”
\VS{41}The prophet quickly
removed
the bandage
from his eyes
and the king
of Israel
recognized
he
was one of the prophets.
\VS{42}The
prophet then said
to him,
“This is what
the {\ND{Lord}}
says,
‘Because
you released
a man
I
had determined should die, you will pay
with your
life
and your people
will suffer instead
of his people.’ ”
\VS{43}The king
of Israel
went
home
to Samaria
bitter
and angry.

\par }\Chap{21}{\PP \VerseOne{1}After
this
the following episode
took place. Naboth
the Jezreelite
owned a vineyard
in Jezreel
adjacent
to the palace
of King
Ahab
of Samaria.
\VS{2}Ahab
said
to
Naboth,
“Give
me your vineyard
so I can make
a vegetable
garden
out of it, for
it
is adjacent
to
my palace.
I will give
you an even better
vineyard
in its place,
or if
you prefer,
I will pay
you silver
for it.”
\VS{3}But Naboth
replied
to
Ahab,
“The
{\ND{Lord}}
forbid
that I should sell
you my ancestral
inheritance.”
\par }{\PP \VS{4}So Ahab
went
into
his palace,
bitter
and angry
that
Naboth
the Jezreelite
had said, “I will not
sell
to you my ancestral
inheritance.”
He lay down
on
his bed,
pouted,
and would not
eat.
\VS{5}Then his wife
Jezebel
came
in and said
to
him, “Why
do you have a bitter
attitude
and refuse
to eat?”
\VS{6}He answered
her,
“While
I was talking
to
Naboth
the Jezreelite,
I said
to him, ‘Sell
me your vineyard
for silver,
or
if
you
prefer,
I will give
you another vineyard
in its place.’
But he said,
‘I will not
sell
you my vineyard.’ ”
\VS{7}His wife
Jezebel
said
to
him, “You
are the king
of Israel! Get
up, eat
some food,
and have a good
time. I
will
get the vineyard
of Naboth
the Jezreelite for you.”
\par }{\PP \VS{8}She wrote
out orders, signed
Ahab’s
name
to them,
and sealed
them with his seal.
She then sent
the orders
to the leaders
and to
the nobles
who
lived
in Naboth’s
city.
\VS{9}This is what she
wrote: “Observe
a time of fasting
and seat
Naboth
in front
of the people.
\VS{10}Also seat
two
villains
opposite
him and have them testify,
‘You cursed
God
and the king.’
Then take him out
and stone
him to death.”
\par }{\PP \VS{11}The men
of the city,
the leaders
and the nobles
who lived
there, followed the written
orders
Jezebel
had
sent
them.
\VS{12}They observed
a time of fasting
and put Naboth
in front
of the people.
\VS{13}The two
villains
arrived
and sat
opposite
him. Then the villains
testified
against Naboth
right before
the people,
saying,
“Naboth
cursed
God
and the king.”
So they dragged
him outside
the city
and stoned
him to death.
\VS{14}Then
they reported
to
Jezebel,
“Naboth
has been stoned
to death.”
\par }{\PP \VS{15}When
Jezebel
heard
that
Naboth
had been stoned
to death,
she
said
to Ahab,
“Get
up, take possession
of the vineyard
Naboth
the Jezreelite
refused
to sell
you for silver,
for
Naboth
is no
longer alive;
he’s
dead.”
\VS{16}When
Ahab
heard
that
Naboth
was dead,
he
got up
and went down
to
take possession
of the vineyard
of Naboth
the Jezreelite.
\par }{\PP \VS{17}The
{\ND{Lord}}
told
Elijah
the Tishbite:
\VS{18}“Get
up, go down
and meet
King
Ahab
of Israel
who
lives in Samaria.
He is at
the vineyard
of Naboth;
he has gone down
there
to take possession of it.
\VS{19}Say
to
him, ‘This is what
the {\ND{Lord}}
says: “Haven’t you committed murder
and taken possession
of the property of the deceased?” ’ Then say
to
him, ‘This is what
the {\ND{Lord}}
says: “In the spot
where
dogs
licked up
Naboth’s
blood
they
will also
lick
up your
blood – yes, yours!” ’ ”
\par }{\PP \VS{20}When Elijah
arrived, Ahab
said
to him, “So, you have found
me, my enemy!” Elijah replied,
“I have found
you, because
you are committed
to doing
evil
in the sight
of the
{\ND{Lord}}.
\VS{21}The
{\ND{Lord}} says, ‘Look,
I am ready to bring
disaster
on
you. I will destroy
you and cut off
every last male
belonging to Ahab
in Israel,
including even the weak
and incapacitated.
\VS{22}I will make
your dynasty
like those of Jeroboam
son
of Nebat
and Baasha
son
of Ahijah
because
you angered
me and made Israel
sin.’
\VS{23}The
{\ND{Lord}}
says
this about
Jezebel,
‘Dogs
will devour
Jezebel
by the outer
wall of Jezreel.’
\VS{24}As for Ahab’s
family, dogs
will eat
the ones
who die
in the city,
and the birds
of the sky
will eat
the ones who die in the country.”
\VS{25}(There
had
never
been
anyone like Ahab,
who
was firmly committed
to doing
evil
in the sight
of the
{\ND{Lord}}, urged
on by his wife
Jezebel.
\VS{26}He was
so
wicked
he worshiped
the disgusting
idols,
just
like the Amorites
whom
the {\ND{Lord}}
had driven
out from before
the Israelites.)
\par }{\PP \VS{27}When
Ahab
heard
these
words,
he tore
his clothes,
put
on sackcloth,
and fasted.
He slept
in sackcloth
and walked
around dejected.
\VS{28}The
{\ND{Lord}}
said
to
Elijah
the Tishbite,
\VS{29}“Have
you noticed
how Ahab
shows remorse
before
me? Because
he shows remorse
before
me, I will not
bring
disaster
on
his dynasty
during
his lifetime,
but during the reign of his son.”

\par }\Chap{22}{\PP \VerseOne{1}There
was no
war
between
Syria
and Israel
for three
years.
\VS{2}In the third
year
King
Jehoshaphat
of Judah
came down
to visit
the king
of Israel.
\VS{3}The king
of Israel
said
to
his servants,
“Surely
you recognize
that
Ramoth
Gilead
belongs to us, though we
are hesitant
to reclaim
it from
the king
of Syria.”
\VS{4}Then he said
to
Jehoshaphat,
“Will you go
with
me to attack
Ramoth
Gilead?” Jehoshaphat
replied
to
the king
of Israel,
“I will support
you; my army
and horses are at your disposal.”
\VS{5}Then Jehoshaphat
added, “First seek
an oracle
from the
{\ND{Lord}}.”
\VS{6}So the king
of Israel
assembled
about four
hundred
prophets
and asked
them,
“Should
I attack
Ramoth
Gilead
or
not?” They said,
“Attack! The sovereign
one will hand
it over to the king.”
\VS{7}But Jehoshaphat
asked,
“Is there not
a prophet
of the {\ND{Lord}}
still
here,
that we may ask him?”
\VS{8}The king
of Israel
answered
Jehoshaphat,
“There is still
one
man
through whom we can seek
the
{\ND{Lord}}’s
will. But I
despise
him because
he does not
prophesy
prosperity
for
me, but
disaster.
His name is Micaiah
son
of Imlah.
Jehoshaphat
said,
“The king
should not
say
such things.”
\VS{9}The king
of Israel
summoned
an official
and said,
“Quickly
bring Micaiah
son
of Imlah.”
\par }{\PP \VS{10}Now the king
of Israel
and King
Jehoshaphat
of Judah
were sitting
on
their respective
thrones,
dressed
in their robes,
at the threshing floor
at the entrance
of the gate
of Samaria.
All
the prophets
were prophesying
before them.
\VS{11}Zedekiah
son
of Kenaanah
made iron
horns
and said,
“This is what
the {\ND{Lord}}
says,
‘With these
you will gore
Syria
until
they are destroyed.’ ”
\VS{12}All
the prophets
were prophesying
the same,
saying,
“Attack
Ramoth
Gilead! You will succeed;
the {\ND{Lord}}
will hand
it over to the king.”
\VS{13}Now the messenger
who
went
to summon
Micaiah
said
to him,
“Look,
the prophets
are in complete agreement that
the king
will
succeed.
Your words
must agree
with theirs;
you must predict
success.”
\VS{14}But Micaiah
said,
“As certainly as the
{\ND{Lord}}
lives,
I will say
what
the {\ND{Lord}}
tells me to
say.”
\par }{\PP \VS{15}When he came
before
the king,
the king
asked
him,
“Micaiah,
should
we attack Ramoth
Gilead
or
not?” He answered
him,
“Attack! You will succeed;
the {\ND{Lord}}
will hand
it over to the king.”
\VS{16}The king
said
to him,
“How
many times
must I
make you solemnly
promise in the name
of the {\ND{Lord}}
to tell
me
only
the truth?”
\VS{17}Micaiah said,
“I saw
all
Israel
scattered
on
the mountains
like sheep
that
have no
shepherd.
Then the
{\ND{Lord}}
said,
‘They have no
master.
They
should go
home
in peace.’ ”
\VS{18}The king
of Israel
said
to
Jehoshaphat,
“Didn’t
I tell
you he does not
prophesy
prosperity
for
me, but
disaster?”
\VS{19}Micaiah said,
“That being
the case, hear
the word
of the {\ND{Lord}}. I saw
the {\ND{Lord}}
sitting
on
his throne,
with all
the heavenly
assembly
standing
on
his right
and on his left.
\VS{20}The
{\ND{Lord}}
said,
‘Who
will deceive
Ahab,
so
he will attack
Ramoth
Gilead
and die
there?’ One
said
this and another that.
\VS{21}Then
a spirit
stepped
forward and stood
before
the {\ND{Lord}}. He said,
‘I
will deceive him.’ The
{\ND{Lord}} asked him, ‘How?’
\VS{22}He
replied,
‘I will go out
and be
a lying
spirit
in the mouths
of all
his prophets.’
The
{\ND{Lord}} said,
‘Deceive
and overpower him. Go out
and do
as you have proposed.’
\VS{23}So now,
look,
the {\ND{Lord}}
has placed
a lying
spirit
in the mouths
of all
these
prophets
of yours;
but the
{\ND{Lord}}
has decreed
disaster for you.”
\VS{24}Zedekiah
son
of Kenaanah
approached,
hit
Micaiah
on
the jaw,
and said,
“Which
way did the
{\ND{Lord}}’s
spirit
go when
he went from me to speak
to you?”
\VS{25}Micaiah
replied,
“Look,
you will see
in the day
when
you go
into an inner room
to hide.”
\VS{26}Then the king
of Israel
said,
“Take
Micaiah
and return
him to
Amon
the city
official
and Joash
the king’s
son.
\VS{27}Say,
‘This is what
the king
says,
“Put
this
man in prison.
Give
him only a little
bread
and water
until
I safely
return.” ’ ”
\VS{28}Micaiah
said,
“If
you really
do safely
return,
then the
{\ND{Lord}}
has not
spoken
through me.” Then he added, “Take note,
all
you people.”
\par }{\PP \VS{29}The king
of Israel
and King
Jehoshaphat
of Judah
attacked Ramoth
Gilead.
\VS{30}The king
of Israel
said
to
Jehoshaphat,
“I will disguise
myself and then enter
into the battle;
but you
wear
your royal robes.”
So the king
of Israel
disguised
himself and then entered
into the battle.
\VS{31}Now the king
of Syria
had ordered
his thirty-two
chariot
commanders,
“Do not
fight
common
soldiers or high-ranking
officers; fight only
the king
of Israel.”
\VS{32}When
the chariot
commanders
saw
Jehoshaphat,
they
said,
“He must
be the king
of Israel.”
So they turned
and attacked
him, but Jehoshaphat
cried out.
\VS{33}When
the chariot
commanders
realized
he was
not
the king
of Israel,
they turned
away
from him.
\VS{34}Now an archer
shot
an arrow
at random,
and it struck
the king
of Israel
between
the plates
of his armor.
The king ordered
his charioteer,
“Turn
around
and take
me from
the battle line,
because
I’m wounded.”
\VS{35}While the battle
raged
throughout the day,
the king
stood
propped
up in his chariot
opposite
the Syrians.
He died
in the evening;
the blood
from the wound
ran down into
the bottom
of the chariot.
\VS{36}As the sun
was setting,
a cry
went through
the camp,
“Each
one
should return to
his city
and to
his homeland.”
\VS{37}So the king
died
and was taken
to Samaria,
where they buried him.
\VS{38}They washed
off the chariot
at the pool
of Samaria
(this was where the prostitutes
bathed); dogs
licked
his blood,
just as
the {\ND{Lord}}
had
said would happen.
\par }{\PP \VS{39}The rest
of the events
of Ahab’s
reign, including a record
of his accomplishments
and how
he built
a luxurious palace
and various
cities,
are recorded
in the scroll
called the Annals
of the Kings
of Israel.
\VS{40}Ahab
passed away.
His son
Ahaziah
replaced
him as king.
\par }{\SH Jehoshaphat’s Reign over Judah
\par }{\PP \VS{41}In the fourth
year
of King
Ahab’s
reign over Israel,
Asa’s
son
Jehoshaphat
became king
over
Judah.
\VS{42}Jehoshaphat
was thirty-five
years
old
when he became king and he reigned
for twenty-five
years
in Jerusalem.
His mother
was Azubah,
the daughter
of Shilhi.
\VS{43}He followed
in his father
Asa’s
footsteps
and was careful
to do what
the {\ND{Lord}}
approved.

However,
the high places
were not
eliminated;
the people
continued
to offer sacrifices and burn incense
on the high places.
\VS{44}Jehoshaphat
was also at peace
with
the king
of Israel.
\par }{\PP \VS{45}The rest
of the events
of Jehoshaphat’s
reign,
including his successes
and military exploits,
are recorded
in
the scroll
called the Annals
of the Kings
of Judah.
\VS{46}He removed
from
the land
any male cultic prostitutes
who had
managed to survive
the reign
of his father
Asa.
\VS{47}There was no
king
in Edom
at this time; a governor
ruled.
\VS{48}Jehoshaphat
built a fleet of large merchant ships
to travel
to Ophir
for gold,
but they never
made
the voyage
because
they were shipwrecked
in Ezion Geber.
\VS{49}Then
Ahaziah
son
of Ahab
said
to
Jehoshaphat,
“Let my sailors
join
yours
in the fleet,”
but Jehoshaphat
refused.
\par }{\PP \VS{50}Jehoshaphat
passed away
and was buried
with
his ancestors
in the city
of his ancestor
David.
His son
Jehoram
replaced
him as king.
\par }{\SH Ahaziah’s Reign over Israel
\par }{\PP \VS{51}In the seventeenth
year
of King
Jehoshaphat’s
reign over Judah,
Ahab’s
son
Ahaziah
became king
over
Israel
in Samaria.
He ruled
for two years
over
Israel.
\VS{52}He did
evil
in the sight
of the
{\ND{Lord}}
and followed
in the footsteps
of his father
and mother;
like
Jeroboam
son
of Nebat,
he encouraged
Israel
to sin.
\VS{53}He worshiped
and bowed
down to Baal,
angering
the {\ND{Lord}}
God
of Israel
just
as his father
had done.
\par }