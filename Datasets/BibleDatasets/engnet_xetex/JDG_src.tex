\NormalFont\ShortTitle{Judges}
{\MT Judges

\par }\ChapOne{1}{\SH Judah Takes the Lead
\par }{\PP \VerseOne{1}After
Joshua
died,
the Israelites
asked
the {\ND{Lord}}, “Who
should lead
the invasion against
the Canaanites
and launch
the attack?”
\VS{2}The
{\ND{Lord}}
said,
“The men of Judah
should take the lead.
Be sure
of this! I am handing
the land
over to them.”
\VS{3}The men of Judah
said
to their relatives,
the men of Simeon, “Invade
our
allotted
land with us
and help us
attack
the Canaanites.
Then
we
will go with
you into your allotted
land.” So the men of Simeon
went
with them.
\par }{\PP \VS{4}The men of Judah
attacked,
and the
{\ND{Lord}}
handed
the Canaanites
and Perizzites
over
to them. They killed
ten
thousand
men
at Bezek.
\VS{5}They met
Adoni-Bezek
at Bezek
and fought
him. They defeated
the Canaanites
and Perizzites.
\VS{6}When Adoni-Bezek
ran away,
they chased
him and captured
him. Then they cut off
his thumbs
and big toes.
\VS{7}Adoni-Bezek
said,
“Seventy
kings,
with thumbs
and big toes
cut off,
used to
lick up
food scraps under
my table.
God
has repaid
me for what
I did
to them.” They brought
him to Jerusalem,
where
he died.
\VS{8}The men
of Judah
attacked
Jerusalem
and captured
it. They put
the sword
to it and set
the city
on fire.
\par }{\PP \VS{9}Later
the men of Judah
went down
to attack
the Canaanites
living
in the hill country,
the Negev,
and the lowlands.
\VS{10}The men of Judah
attacked the Canaanites
living
in Hebron.
(Hebron
used to be called
Kiriath Arba.) They killed
Sheshai,
Ahiman,
and Talmai.
\VS{11}From there
they attacked
the people
of Debir.
(Debir
used to be
called
Kiriath Sepher.)
\VS{12}Caleb
said,
“To the man who attacks
and captures
Kiriath Sepher
I will give
my daughter
Acsah
as a wife.”
\VS{13}When Othniel
son
of Kenaz,
Caleb’s
younger
brother,
captured
it, Caleb gave
him his daughter
Acsah
as a wife.
\par }{\PP \VS{14}One time Acsah came
and charmed
her father
so she could ask him
for some land.
When she got down from her donkey,
Caleb
said
to her, “What would you like?”
\VS{15}She answered,
“Please give
me a special present.
Since
you have given
me land
in the Negev,
now give
me springs
of water.”
So Caleb
gave
her both the upper
and lower
springs.
\par }{\PP \VS{16}Now the descendants
of the Kenite,
Moses’
father-in-law,
went up
with
the people of Judah from the City
of Date
Palm Trees to Arad
in the desert
of Judah,
located
in the Negev.
They went
and lived
with
the people
of Judah.
\par }{\PP \VS{17}The men of Judah
went
with
their brothers
the men of Simeon
and defeated
the Canaanites
living
in Zephath.
They wiped out
Zephath. So people now call
the city
Hormah.
\VS{18}The men of Judah
captured
Gaza,
Ashkelon,
Ekron,
and
the territory surrounding each of these cities.
\par }{\PP \VS{19}The
{\ND{Lord}}
was with
the men of Judah.
They conquered
the hill country,
but
they could not
conquer
the
people living
in the coastal
plain, because
they had chariots
with iron-rimmed wheels.
\VS{20}Caleb
received Hebron,
just
as Moses
had promised.
He drove
out the three
Anakites.
\VS{21}The men
of Benjamin,
however,
did not
conquer
the Jebusites
living
in Jerusalem.
The Jebusites
live
with
the people of Benjamin
in Jerusalem
to
this
very
day.
\par }{\SH Partial Success
\par }{\PP \VS{22}When
the men
of Joseph
attacked
Bethel,
the {\ND{Lord}}
was with them.
\VS{23}When
the men
of Joseph
spied
out Bethel
(it used to be called
Luz),
\VS{24}the spies
spotted
a man
leaving
the city.
They said
to him, “If you show
us
a secret
entrance
into the city,
we will reward you.”
\VS{25}He showed
them
a secret entrance
into the city,
and they put
the city
to the sword.
But they let the
man
and his extended
family
leave safely.
\VS{26}He moved
to Hittite
country
and built
a city.
He named
it Luz,
and it has kept that
name
to
this
very
day.
\par }{\PP \VS{27}The men of Manasseh
did not
conquer
Beth Shan,
Taanach,
or their surrounding towns.
Nor did they conquer
the people living
in Dor,
Ibleam,
Megiddo
or their surrounding towns.
The Canaanites
managed
to remain
in those
areas.
\VS{28}Whenever
Israel
was strong
militarily, they forced
the Canaanites
to do hard labor,
but they never
totally
conquered them.
\par }{\PP \VS{29}The men of Ephraim
did not
conquer
the Canaanites
living
in Gezer.
The Canaanites
lived
among
them in Gezer.
\par }{\PP \VS{30}The men of Zebulun
did not
conquer
the people living
in Kitron
and Nahalol.
The Canaanites
lived
among
them and were
forced to do hard labor.
\par }{\PP \VS{31}The men of Asher
did not
conquer
the people living
in
Acco
or Sidon,
nor did they conquer Ahlab,
Aczib,
Helbah,
Aphek,
or Rehob.
\VS{32}The people of Asher
live
among
the Canaanites
residing in
the land
because
they did not
conquer them.
\par }{\PP \VS{33}The men of Naphtali
did not
conquer
the
people living
in Beth Shemesh
or Beth Anath.
They live
among
the Canaanites
residing
in the land.
The Canaanites living
in Beth Shemesh
and Beth Anath
were
forced to do hard labor for them.
\par }{\PP \VS{34}The Amorites
forced
the people of Dan
to live in the hill country.
They did not
allow
them to live in the coastal plain.
\VS{35}The Amorites
managed
to remain
in Har
Heres,
Aijalon,
and Shaalbim.
Whenever the tribe
of Joseph
was strong
militarily,
the Amorites were
forced to do hard labor.
\VS{36}The border
of Amorite
territory ran from the Scorpion Ascent
to Sela
and on up.

\par }\Chap{2}{\PP \VerseOne{1}The
{\ND{Lord}}’s
angelic messenger
went up
from
Gilgal
to
Bokim.
He said,
“I brought you up
from Egypt
and led
you into
the
land
I had
solemnly
promised to give to your ancestors.
I said,
‘I will never
break
my agreement
with you,
\VS{2}but you
must not
make an
agreement
with the people who live
in this
land.
You should
tear
down the altars
where they worship.’ But
you have
disobeyed
me. Why
would you do such a thing?
\VS{3}At that time I also
warned you, ‘If you disobey, I will not
drive
out the Canaanites before
you. They will
ensnare you and their gods
will
lure you away.’ ”
\par }{\PP \VS{4}When
the
{\ND{Lord}}’s
messenger
finished speaking
these
words
to
all
the Israelites,
the people
wept
loudly.
\VS{5}They named
that place
Bokim
and offered sacrifices
to the
{\ND{Lord}}
there.
\par }{\SH The End of an Era
\par }{\PP \VS{6}When
Joshua
dismissed
the people,
the Israelites
went
to their allotted
portions of territory, intending
to take possession
of the land.
\VS{7}The people
worshiped
the {\ND{Lord}}
throughout
Joshua’s
lifetime
and as long as
the elderly men
who
outlived
him
remained alive. These men had
witnessed
all
the great
things
the {\ND{Lord}}
had
done
for Israel.
\VS{8}Joshua
son
of Nun,
the
{\ND{Lord}}’s
servant,
died
at the age
of one hundred
ten.
\VS{9}The people buried
him in his allotted
land in Timnath
Heres in the hill country
of Ephraim,
north
of Mount
Gaash.
\VS{10}That entire
generation
passed away;
a new
generation
grew up
that had
not
personally experienced
the
{\ND{Lord}}’s
presence or seen what
he had
done
for Israel.
\par }{\SH A Monotonous Cycle
\par }{\PP \VS{11}The Israelites
did
evil
before
the {\ND{Lord}}
by worshiping
the Baals.
\VS{12}They abandoned
the

{\ND{Lord}}
God
of their ancestors
who brought them out
of the
land
of Egypt.
They followed
other
gods
– the gods
of the nations
who
lived around
them. They worshiped
them and made the

{\ND{Lord}}
angry.
\VS{13}They abandoned
the {\ND{Lord}}
and worshiped
Baal
and the Ashtars.
\par }{\PP \VS{14}The
{\ND{Lord}}
was furious
with Israel
and handed
them over
to robbers
who plundered
them. He turned
them over
to their enemies
who lived around
them. They could
not
withstand
their enemies’ attacks.
\VS{15}Whenever
they went out
to fight,
the {\ND{Lord}}
did
them harm,
just
as he had warned
and solemnly
vowed he would do. They suffered
greatly.
\par }{\PP \VS{16}The
{\ND{Lord}}
raised
up leaders
who delivered
them from these robbers.
\VS{17}But
they did not
obey
their leaders.
Instead
they prostituted
themselves
to other
gods
and worshiped
them. They quickly
turned aside
from
the path
their ancestors had
walked.
Their ancestors
had obeyed
the
{\ND{Lord}}’s
commands, but they did not.
\VS{18}When
the {\ND{Lord}}
raised up
leaders
for them, the
{\ND{Lord}}
was
with
each leader
and delivered
the people from
their enemies
while the leader
remained alive.
The
{\ND{Lord}}
felt sorry for them when they cried out in agony
because of
what their harsh
oppressors did to them.
\VS{19}When
a leader
died,
the next generation would again
act more wickedly
than the previous one.
They would follow
after
other
gods,
worshiping
them and bowing
down
to them. They did not
give up their practices
or their stubborn
ways.
\par }{\SH A Divine Decision
\par }{\PP \VS{20}The
{\ND{Lord}}
was furious
with Israel.
He said,
“This
nation
has violated
the terms
of the agreement
I
made with their ancestors
by disobeying me.
\VS{21}So I
will no
longer
remove before
them
any
of the nations
that
Joshua
left
unconquered
when he died.
\VS{22}Joshua left those nations to test
Israel.
I wanted
to see whether
or not
the people
would carefully
walk
in the path
marked
out by the
{\ND{Lord}}, as
their ancestors
were careful to do.”
\VS{23}This is why the
{\ND{Lord}}
permitted
these
nations
to remain and did not
conquer
them immediately;
he did not
hand
them over to Joshua.

\par }\Chap{3}{\PP \VerseOne{1}These
were the nations
the {\ND{Lord}}
permitted
to remain so he could use them to test
Israel
– he wanted to test all
those who had
not
experienced
battle
against the Canaanites.
\VS{2}He left
those nations
simply
because he wanted
to teach
the subsequent
generations
of Israelites,
who had
not
experienced
the earlier
battles,
how
to conduct
holy war.
\VS{3}These were the nations: the five
lords
of the Philistines,
all
the Canaanites,
the Sidonians,
and the Hivites
living
in Mount
Lebanon,
from Mount
Baal Hermon
to Lebo-Hamath.
\VS{4}They were
left to test
Israel,
so the
{\ND{Lord}}
would know
if his people would obey
the commands
he gave
their ancestors
through
Moses.
\par }{\PP \VS{5}The Israelites
lived
among
the Canaanites,
Hittites,
Amorites,
Perizzites,
Hivites,
and Jebusites.
\VS{6}They took
the Canaanites’ daughters
as wives
and gave
their daughters
to the Canaanites; they worshiped
their gods as well.
\par }{\SH Othniel: A Model Leader
\par }{\PP \VS{7}The Israelites
did
evil
in the
{\ND{Lord}}’s
sight.
They forgot
the {\ND{Lord}}
their God
and worshiped
the Baals
and the Asherahs.
\VS{8}The
{\ND{Lord}}
was furious
with Israel
and turned
them over
to King
Cushan-Rishathaim
of Aram-Naharaim.
They
were Cushan-Rishathaim’s
subjects for eight
years.
\VS{9}When the Israelites
cried
out for help to
the {\ND{Lord}}, he raised
up a deliverer
for the Israelites
who rescued
them. His name was Othniel
son
of Kenaz,
Caleb’s
younger
brother.
\VS{10}The
{\ND{Lord}}’s
spirit
empowered him and he led
Israel.
When he went
to do battle,
the {\ND{Lord}}
handed
over to him King
Cushan-Rishathaim
of Aram
and he
overpowered him.
\VS{11}The land
had rest
for forty
years;
then Othniel
son
of Kenaz
died.
\par }{\SH Deceit, Assassination, and Deliverance
\par }{\PP \VS{12}The Israelites
again
did
evil
in the
{\ND{Lord}}’s
sight.
The
{\ND{Lord}}
gave King
Eglon
of Moab
control
over
Israel
because
they had done
evil
in the
{\ND{Lord}}’s
sight.
\VS{13}Eglon formed alliances
with
the Ammonites
and Amalekites.
He came
and defeated
Israel,
and they seized
the City
of Date Palm Trees.
\VS{14}The Israelites
were subject
to King
Eglon
of Moab
for eighteen
years.
\par }{\PP \VS{15}When the Israelites
cried out
for help to
the {\ND{Lord}}, he
raised up
a deliverer
for them. His name was Ehud
son
of Gera
the Benjaminite,
a left-handed
man.
The Israelites
sent
him to King
Eglon
of Moab
with their tribute payment.
\VS{16}Ehud
made
himself a sword
– it had two
edges
and was eighteen inches
long.
He strapped
it under
his coat
on
his right
thigh.
\VS{17}He brought
the tribute
payment to King
Eglon
of Moab.
(Now Eglon
was a very
fat
man.)
\par }{\PP \VS{18}After
Ehud
brought
the tribute
payment, he dismissed
the people
who had
carried it.
\VS{19}But he
went back
once he reached the carved images
at
Gilgal.
He said
to Eglon, “I have a secret
message
for
you, O king.”
Eglon said,
“Be quiet!” All
his attendants
left.
\VS{20}When Ehud
approached
him, he
was sitting
in his well-ventilated
upper room
all by himself.
Ehud
said,
“I have a message
from God
for
you.” When Eglon rose up
from
his seat,
\VS{21}Ehud
reached
with his left hand,
pulled
the sword
from
his right
thigh,
and drove
it into Eglon’s belly.
\VS{22}The handle
went
in after
the blade,
and the fat
closed
around
the blade,
for
Ehud did not
pull
the sword
out
of his belly.
\VS{23}As Ehud
went out
into the vestibule,
he closed
the doors
of the upper
room behind him and locked them.
\par }{\PP \VS{24}When Ehud had left, Eglon’s servants
came
and saw
the locked
doors
of the upper
room. They said,
“He must be relieving
himself in the well-ventilated
inner room.”
\VS{25}They waited
so long
they were embarrassed,
but he still
did not
open
the doors
of the upper
room. Finally they took
the
key
and opened
the doors. Right
before their eyes was their master,
sprawled
out dead
on
the floor!
\VS{26}Now Ehud
had escaped
while
they were delaying.
When he passed
the carved images,
he escaped
to Seirah.
\par }{\PP \VS{27}When
he reached
Seirah, he blew
a trumpet
in the Ephraimite
hill country.
The Israelites
went down
with
him from
the hill country,
with Ehud in the lead.
\VS{28}He said
to
them,
“Follow
me, for
the {\ND{Lord}}
is about to defeat
your enemies,
the
Moabites!” They followed
him, captured
the fords
of the Jordan River
opposite Moab,
and did not
let
anyone
cross.
\VS{29}That day they killed
about ten
thousand
Moabites –
all
strong, capable
warriors;
not
one escaped.
\VS{30}Israel
humiliated
Moab
that day,
and the land
had rest
for eighty
years.
\par }{\PP \VS{31}After
Ehud came Shamgar
son
of Anath;
he killed
six
hundred
Philistines
with an oxgoad
and, like Ehud, delivered
Israel.

\par }\Chap{4}{\PP \VerseOne{1}The Israelites
again
did
evil
in the
{\ND{Lord}}’s
sight
after Ehud’s
death.
\VS{2}The
{\ND{Lord}}
turned
them over
to
King
Jabin
of Canaan,
who
ruled
in Hazor.
The general
of his army
was Sisera,
who lived
in Harosheth Haggoyim.
\VS{3}The Israelites
cried out
for help to
the {\ND{Lord}}, because
Sisera had nine
hundred
chariots
with iron-rimmed
wheels, and he cruelly
oppressed
the Israelites
for twenty
years.
\par }{\PP \VS{4}Now Deborah,
a prophetess,
wife
of Lappidoth,
was leading Israel
at that time.
\VS{5}She
would sit
under
the Date Palm Tree
of Deborah
between
Ramah
and Bethel
in the Ephraimite
hill country.
The Israelites
would come up
to her
to have their disputes settled.
\par }{\PP \VS{6}She summoned
Barak
son
of Abinoam
from Kedesh
in Naphtali.
She said
to
him, “Is it not
true that the
{\ND{Lord}}
God
of Israel
is commanding
you? Go,
march
to Mount
Tabor! Take
with
you ten
thousand
men
from Naphtali
and Zebulun!
\VS{7}I
will bring
Sisera,
the general
of Jabin’s
army,
to you at the Kishon
River,
along with his chariots
and huge
army. I will hand him over to you.”
\VS{8}Barak
said
to
her, “If
you go
with
me, I will go.
But if
you do not
go
with
me, I will not
go.”
\VS{9}She said,
“I will indeed go
with
you. But you will not
gain fame
on
the expedition
you
are undertaking,
for
the {\ND{Lord}}
will turn
Sisera
over
to a woman.”
Deborah
got up
and went
with
Barak
to Kedesh.
\VS{10}Barak
summoned
men from Zebulun
and Naphtali
to Kedesh.
Ten
thousand
men
followed
him; Deborah
went up
with him as well.
\VS{11}Now Heber
the Kenite
had moved away
from the Kenites,
the descendants
of Hobab,
Moses’
father-in-law.
He lived
near
the great tree
in Zaanannim
near
Kedesh.
\par }{\PP \VS{12}When Sisera
heard that
Barak
son
of Abinoam
had gone up
to Mount
Tabor,
\VS{13}he
ordered
all
his chariotry
– nine
hundred
chariots
with iron-rimmed wheels – and all the troops he had with him to go from Harosheth-Haggoyim to the River Kishon.
\VS{14}Deborah
said
to
Barak,
“Spring into action,
for
this
is the day
the {\ND{Lord}}
is handing
Sisera
over
to you! Has the
{\ND{Lord}}
not
taken the lead?” Barak
quickly went down
from Mount
Tabor
with ten
thousand
men
following him.
\VS{15}The
{\ND{Lord}}
routed
Sisera,
all
his chariotry,
and all
his army
with the edge
of the sword.
Sisera
jumped out
of his chariot
and ran away
on foot.
\VS{16}Now Barak
chased
the chariots
and the army
all
the way to Harosheth Haggoyim.
Sisera’s
whole
army
died by the edge
of the sword;
not
even
one
survived!
\par }{\PP \VS{17}Now Sisera
ran
away on foot
to
the tent
of Jael,
wife
of Heber
the Kenite,
for
King
Jabin
of Hazor
and the family
of Heber
the Kenite
had made a peace
treaty.
\VS{18}Jael
came out
to welcome
Sisera.
She said
to him,
“Stop
and rest, my lord.
Stop
and rest with me.
Don’t
be afraid.”
So Sisera stopped
to
rest in her tent,
and she put a blanket
over him.
\VS{19}He said
to her, “Give me
a little
water
to drink,
because
I’m thirsty.”
She opened
a goatskin
container of milk and gave him some milk
to drink.
Then she covered him up again.
\VS{20}He said
to her,
“Stand
watch at the entrance
to the tent.
If
anyone
comes
along and asks
you, ‘Is there
a man
here?’ say
‘No.’ ”
\VS{21}Then Jael
wife
of Heber
took
a tent peg
in one hand
and a hammer
in the other. She crept up
on him, drove
the tent peg
through his temple
into the ground
while
he was asleep
from exhaustion,
and he died.
\VS{22}Now
Barak
was chasing
Sisera.
Jael
went out
to welcome him.
She said
to him, “Come
here and I will show
you the
man
you
are searching
for.” He went
with her into
the tent, and there he saw
Sisera
sprawled
out dead
with the tent peg
in his temple.
\par }{\PP \VS{23}That day
God
humiliated
King
Jabin
of Canaan
before
the Israelites.
\VS{24}Israel’s
power
continued
to overwhelm
King
Jabin
of Canaan
until
they did away
with him.

\par }\Chap{5}{\PP \VerseOne{1}On that day
Deborah
and Barak
son
of Abinoam
sang
this victory song:
\par }{\Q \VS{2}“When the leaders
took the lead in Israel,
\par }{\Q When the people
answered the call to war –
\par }{\Q Praise the
{\ND{Lord}}!
\par }{\Q \VS{3}Hear,
O kings!
\par }{\Q Pay attention,
O rulers!
\par }{\Q I
will sing to the
{\ND{Lord}}!

\par }{\Q I
will sing
to the
{\ND{Lord}}
God
of Israel!
\par }{\Q \VS{4}O
{\ND{Lord}}, when
you departed
from Seir,
\par }{\Q when you marched
from Edom’s
plains,
\par }{\Q the earth
shook,
the heavens
poured
down,
\par }{\Q the clouds
poured
down rain.
\par }{\Q \VS{5}The mountains
trembled
before
the {\ND{Lord}}, the God of Sinai;
\par }{\Q before
the {\ND{Lord}}
God
of Israel.
\par }{\Q \VS{6}In the days
of Shamgar
son
of Anath,
\par }{\Q in the days
of Jael
caravans
disappeared;

\par }{\Q travelers had to go
on winding
side roads.
\par }{\Q \VS{7}Warriors
were scarce,
\par }{\Q they were scarce
in Israel,
\par }{\Q until
you arose,
Deborah,
\par }{\Q until you arose
as a motherly
protector in Israel.
\par }{\Q \VS{8}God
chose
new
leaders,

\par }{\Q then
fighters
appeared
in the city gates;
\par }{\Q but, I swear, not a shield
or
spear
could be found,

\par }{\Q among forty
military units in Israel.
\par }{\Q \VS{9}My heart
went out
to Israel’s
leaders,
\par }{\Q to the people
who answered the call to war.
\par }{\Q Praise
the {\ND{Lord}}!
\par }{\Q \VS{10}You who ride
on light-colored
female donkeys,
\par }{\Q who sit
on
saddle blankets,
\par }{\Q you who walk
on
the road,
pay attention!
\par }{\Q \VS{11}Hear the sound
of those who divide the sheep
among
the watering places;
\par }{\Q there
they tell
of the Lord’s
victorious deeds,
\par }{\Q the victorious deeds
of his warriors
in Israel.
\par }{\Q Then
the
{\ND{Lord}}’s
people
went down
to the city gates –
\par }{\Q \VS{12}Wake up,
wake up,
Deborah!
\par }{\Q Wake up,
wake up,
sing
a song!
\par }{\Q Get up,
Barak!
\par }{\Q Capture
your prisoners
of war, son
of Abinoam!
\par }{\Q \VS{13}Then
the survivors
came down
to the mighty ones;
\par }{\Q the
{\ND{Lord}}’s
people
came down
to me as warriors.
\par }{\Q \VS{14}They came from
Ephraim,
who uprooted
Amalek,
\par }{\Q they follow
after you, Benjamin,
with your soldiers.
\par }{\Q From
Makir
leaders
came down,
\par }{\Q from Zebulun
came the ones who march carrying
an officer’s
staff.
\par }{\Q \VS{15}Issachar’s
leaders
were with
Deborah,
\par }{\Q the men of Issachar
supported
Barak;
\par }{\Q into the valley
they were sent
under Barak’s command.
\par }{\Q Among the clans
of Reuben
there was intense
heart
searching.
\par }{\Q \VS{16}Why
do you remain
among
the sheepfolds,
\par }{\Q listening
to the shepherds playing their pipes
for their flocks?

\par }{\Q As for the clans
of Reuben
– there was intense
searching
of heart.
\par }{\Q \VS{17}Gilead
stayed put
beyond
the Jordan River.
\par }{\Q As for Dan
– why
did he seek temporary employment
in the shipyards?

\par }{\Q Asher
remained
on
the seacoast,
\par }{\Q he stayed
by his harbors.
\par }{\Q \VS{18}The men of Zebulun
were not concerned
about their lives;
\par }{\Q Naphtali
charged on
to the battlefields.
\par }{\Q \VS{19}Kings
came,
they fought;
\par }{\Q the kings
of Canaan
fought,
\par }{\Q at
Taanach
by the waters
of Megiddo,
\par }{\Q but they took
no
silver as plunder.
\par }{\Q \VS{20}From
the sky
the stars
fought,
\par }{\Q from their paths
in the heavens they fought
against
Sisera.
\par }{\Q \VS{21}The Kishon
River
carried
them off;
\par }{\Q the river
confronted them – the Kishon River.
\par }{\Q Step on the necks of the strong!
\par }{\Q \VS{22}The horses’
hooves
pounded
the ground;

\par }{\Q the stallions
galloped madly.
\par }{\Q \VS{23}‘Call judgment down on
Meroz,’
says
the
{\ND{Lord}}’s
angelic messenger;
\par }{\Q ‘Be sure
to call judgment down on
those who live
there,
\par }{\Q because
they did not
come
to help
in the
{\ND{Lord}}’s
battle,

\par }{\Q to help
in the
{\ND{Lord}}’s
battle against the warriors.’
\par }{\Q \VS{24}The most rewarded
of women
should be Jael,
\par }{\Q the wife
of Heber
the Kenite!
\par }{\Q She should be
the most rewarded
of women
who live in tents.
\par }{\Q \VS{25}He asked
for water,
\par }{\Q and she gave
him milk;
\par }{\Q in a bowl
fit for a king,
\par }{\Q she served
him curds.
\par }{\Q \VS{26}Her left hand
reached
for the tent peg,
\par }{\Q her right hand
for the workmen’s hammer.
\par }{\Q She “hammered” Sisera,
\par }{\Q she shattered his skull,
\par }{\Q she smashed his head,
\par }{\Q she drove the tent peg through his temple.
\par }{\Q \VS{27}Between
her feet
he collapsed,
\par }{\Q he fell limp
and was lifeless;
\par }{\Q between
her feet
he collapsed
and fell limp,
\par }{\Q in the spot where
he collapsed,
\par }{\Q there
he fell limp
– violently murdered!
\par }{\Q \VS{28}Through
the window
she looked;
\par }{\Q Sisera’s
mother
cried
out through
the lattice:
\par }{\Q ‘Why
is his chariot
so slow
to return?
\par }{\Q Why
are the hoofbeats
of his chariot-horses
delayed?’
\par }{\Q \VS{29}The wisest
of her ladies
answer;
\par }{\Q indeed
she
even thinks to herself,
\par }{\Q \VS{30}‘No
doubt they are gathering
and dividing
the plunder –
\par }{\Q a girl
or two
for each
man
to rape!

\par }{\Q Sisera
is grabbing up
colorful cloth,
\par }{\Q he is grabbing up
colorful
embroidered
cloth,
\par }{\Q two pieces of colorful
embroidered
cloth,
\par }{\Q for the neck
of the plunderer!’
\par }{\Q \VS{31}May all
your enemies
perish
like this,
O
{\ND{Lord}}!
\par }{\Q But may those who love
you shine
\par }{\Q like the rising sun
at its brightest!”

\par }{\PP And the land
had rest
for forty
years.

\par }\Chap{6}{\PP \VerseOne{1}The Israelites
did
evil
in the
{\ND{Lord}}’s
sight,
so
the {\ND{Lord}}
turned them over
to Midian
for seven
years.
\VS{2}The Midianites
overwhelmed
Israel.
Because
of Midian
the Israelites
made
shelters
for themselves in the hills,
as well as caves
and strongholds.
\VS{3}Whenever
the Israelites
planted
their crops,
the Midianites,
Amalekites,
and the people from the east
would attack them.
\VS{4}They invaded
the land
and devoured
its crops
all the way
to Gaza.
They left
nothing
for the Israelites
to eat,
and they took away the sheep,
oxen,
and donkeys.
\VS{5}When
they
invaded
with their cattle
and tents,
they were as thick
as locusts.
Neither they nor their camels
could be counted.
They came
to devour
the land.
\VS{6}Israel
was
so severely weakened
by Midian
that the Israelites
cried
out to
the {\ND{Lord}} for help.
\par }{\PP \VS{7}When
the Israelites
cried
out to
the {\ND{Lord}}
for help because
of Midian,
\VS{8}he sent
a prophet
to
the Israelites.
He said
to them, “This is what
the {\ND{Lord}}
God
of Israel
says: ‘I
brought you up
from Egypt
and took you out
of that place
of slavery.
\VS{9}I rescued
you from Egypt’s
power
and from the power
of all
who oppressed
you. I drove
them out before
you and gave
their land to you.
\VS{10}I said
to you, “I am
the {\ND{Lord}}
your God! Do not
worship
the gods
of the Amorites,
in whose
land
you
are now living!” But
you have
disobeyed me.’ ”
\par }{\SH Gideon Meets Some Visitors
\par }{\PP \VS{11}The
{\ND{Lord}}’s
angelic messenger
came
and sat
down under
the oak tree
in Ophrah
owned by Joash
the Abiezrite.
He arrived while
Joash’s son
Gideon
was threshing
wheat
in a winepress
so he could hide
it from
the Midianites.
\VS{12}The
{\ND{Lord}}’s
messenger
appeared
and said
to him,
“The
{\ND{Lord}}
is with
you, courageous
warrior!”
\VS{13}Gideon
said
to
him, “Pardon
me,
but if the
{\ND{Lord}}
is
with
us, why
has such disaster
overtaken
us? Where
are all
his miraculous deeds
our ancestors
told
us about? They said, ‘Did the
{\ND{Lord}}
not
bring
us up
from Egypt?’ But now
the {\ND{Lord}}
has abandoned
us and handed
us over
to Midian.”
\VS{14}Then the
{\ND{Lord}}
himself turned
to
him and said,
“You have the strength.
Deliver
Israel
from the power
of the Midianites! Have I not
sent you?”
\VS{15}Gideon said
to
him, “But Lord,
how
can I deliver
Israel? Just look! My clan
is the weakest
in Manasseh,
and I
am the youngest
in my family.”
\VS{16}The
{\ND{Lord}}
said
to him,
“Ah, but
I will be
with
you! You will strike
down the whole Midianite army.”
\VS{17}Gideon said
to him,
“If
you really
are pleased
with me, then
give
me a sign
as proof that it is really you speaking
with me.
\VS{18}Do not
leave
this
place until
I come
back with a gift
and present it
to you.” The
{\ND{Lord}} said,
“I
will stay
here until
you come back.”
\par }{\PP \VS{19}Gideon
went
and prepared
a young
goat,
along with unleavened
bread made from an ephah
of flour.
He put
the meat
in a basket
and the broth
in a pot.
He brought
the food to him
under
the oak tree
and presented
it to him.
\VS{20}God’s
messenger
said
to him,
“Put
the meat
and unleavened
bread on this
rock,
and pour out
the broth.”
Gideon
did
as instructed.
\VS{21}The
{\ND{Lord}}’s
messenger
touched
the meat
and the unleavened
bread with the
tip
of his staff.
Fire
flared up
from
the rock
and consumed
the
meat
and unleavened
bread. The
{\ND{Lord}}’s
messenger
then
disappeared.
\par }{\PP \VS{22}When Gideon
realized
that
it was the
{\ND{Lord}}’s
messenger,
he
said,
“Oh
no! Master,

{\ND{Lord}}! I have
seen
the
{\ND{Lord}}’s
messenger
face
to face!”
\VS{23}The
{\ND{Lord}}
said
to him, “You are safe! Do not
be afraid! You are not
going to die!”
\VS{24}Gideon
built
an altar
for the
{\ND{Lord}}
there,
and named
it “The
{\ND{Lord}}
is on friendly terms
with me.” To
this
day
it is still
there in Ophrah
of the Abiezrites.
\par }{\SH Gideon Destroys the Altar
\par }{\PP \VS{25}That night
the {\ND{Lord}}
said
to him, “Take
the
bull
from your father’s
herd, as well as
a second
bull,
one that is seven
years
old. Pull down
your father’s
Baal
altar
and cut down
the nearby Asherah pole.
\VS{26}Then build
an altar
for the
{\ND{Lord}}
your God
on
the top
of this
stronghold
according to the proper pattern.
Take
the
second
bull
and offer
it as a burnt sacrifice
on the wood
from the Asherah pole
that
you cut down.”
\VS{27}So
Gideon
took
ten
of his servants
and did
just
as the
{\ND{Lord}}
had told
him.
He was
too afraid
of his father’s
family
and the
men
of the city
to do
it in broad daylight,
so
he waited
until nighttime.
\par }{\PP \VS{28}When the men
of the city
got up
the next morning,
they saw
the Baal
altar
pulled down,
the nearby
Asherah pole
cut down,
and the
second
bull
sacrificed
on
the newly built
altar.
\VS{29}They said
to
one
another, “Who
did
this?” They investigated
the matter
thoroughly and concluded
that Gideon
son
of Joash
had done it.
\VS{30}The men
of the city
said
to
Joash,
“Bring out
your son,
so we
can execute
him! He pulled down
the Baal
altar
and cut down
the nearby
Asherah pole.”
\VS{31}But Joash
said
to all
those who
confronted
him, “Must you
fight Baal’s
battles? Must
you
rescue
him? Whoever
takes up his cause will die
by morning! If
he really is a god,
let him fight his own battles! After all, it was his altar
that
was pulled down.”
\VS{32}That very day
Gideon’s father named
him Jerub-Baal,
because he had said,
“Let Baal
fight
with him, for
it was
his altar
that
was pulled down.”
\par }{\SH Gideon Summons an Army and Seeks Confirmation
\par }{\PP \VS{33}All
the Midianites,
Amalekites,
and the people
from the east
assembled.
They crossed
the Jordan River and camped
in the Jezreel
Valley.
\VS{34}The
{\ND{Lord}}’s
spirit
took control
of Gideon.
He blew
a trumpet,
summoning
the Abiezrites
to follow him.
\VS{35}He sent
messengers
throughout
Manasseh
and summoned
them to follow
him as well.
He also
sent
messengers
throughout Asher,
Zebulun,
and Naphtali,
and they came up
to meet him.
\par }{\PP \VS{36}Gideon
said
to
God,
“If
you really
intend to use me to deliver
Israel,
as
you promised, then give me a sign as proof.
\VS{37}Look,
I am
putting
a wool
fleece
on the threshing floor.
If
there is dew
only on
the fleece,
and the ground
around it is dry,
then I will be sure
that
you will use me to deliver
Israel,
as
you promised.”
\VS{38}The
{\ND{Lord}} did as he asked. When
he got up
the next morning,
he squeezed
the fleece,
and enough dew
dripped
from
it
to fill
a bowl.
\VS{39}Gideon
said
to
God,
“Please
do not
get angry
at me,
when
I ask
for just
one more
sign. Please allow me one more test
with the fleece.
This time make
only
the fleece
dry,
while the ground
around it is covered with dew.”
\VS{40}That
night
God
did
as he
asked. Only
the fleece
was dry
and the ground
around it was
covered with dew.

\par }\Chap{7}{\PP \VerseOne{1}Jerub-Baal
(that
is, Gideon) and his men
got up
the next morning and camped
near
the spring of Harod.
The Midianites
were camped
north
of them near the hill
of Moreh
in the valley.
\VS{2}The
{\ND{Lord}}
said
to
Gideon,
“You have too many
men
for me
to hand
Midian
over
to you. Israel
might
brag, ‘Our own strength has delivered us.’
\VS{3}Now,
announce
to the men, ‘Whoever
is shaking
with fear
may turn around
and leave
Mount
Gilead.’ ”
Twenty-two
thousand
men
went home;
ten
thousand
remained.
\VS{4}The
{\ND{Lord}}
spoke
to
Gideon
again, “There are still
too many
men.
Bring them down
to
the water
and I will thin the ranks
some
more. When
I say,
‘This
one should go
with
you,’ pick him
to
go;
when I say, ‘This
one should not
go
with
you,’ do not
take
him.”
\VS{5}So he brought the men
down
to
the water.
Then the
{\ND{Lord}}
said to
Gideon,
“Separate those who lap
the water
as
a dog
laps
from those who
kneel
to drink.”
\VS{6}Three
hundred
men
lapped;
the rest
of the men
kneeled
to drink
water.
\VS{7}The
{\ND{Lord}}
said
to
Gideon,
“With the three
hundred
men
who lapped
I will deliver
the
whole
army
and I will
hand
Midian
over to you. The rest of the men
should go
home.”
\VS{8}The
men
who were chosen took
supplies
and their trumpets.
Gideon sent
all
the men
of Israel
back to their homes;
he kept only three
hundred
men.
Now the Midianites
were camped down below
in the valley.
\par }{\SH Gideon Reassured of Victory
\par }{\PP \VS{9}That night
the {\ND{Lord}}
said
to
Gideon, “Get
up! Attack
the camp,
for
I am handing
it over to you.
\VS{10}But if
you are
afraid
to attack,
go down
to the camp
with Purah
your servant
\VS{11}and listen
to what
they are saying.
Then
you will be brave
and attack
the camp.”
So he went down
with Purah
his servant
to
where the sentries were guarding
the camp.
\VS{12}Now the Midianites,
Amalekites,
and the people
from the east
covered
the valley
like a swarm of locusts.
Their camels
could not
be counted;
they were as innumerable
as the sand
on
the seashore.
\VS{13}When Gideon
arrived,
he heard
a man
telling
another man
about a dream
he had. The man said,
“Look! I had
a dream.
I saw
a stale cake
of barley
bread
rolling into
the Midianite
camp.
It hit
a tent
so hard it knocked
it over
and turned
it upside down.
The tent
just collapsed.”
\VS{14}The other
man said, “Without
a doubt
this
symbolizes the sword
of Gideon
son
of Joash,
the Israelite.
God
is handing
Midian
and all
the army over to him.”
\par }{\SH Gideon Routs the Enemy
\par }{\PP \VS{15}When
Gideon
heard
the report
of the dream
and its interpretation,
he praised
God. Then he went back
to
the Israelite
camp
and said,
“Get up,
for
the {\ND{Lord}}
is handing
the Midianite
army over to you!”
\VS{16}He divided
the three
hundred
men
into three
units.
He gave
them all
trumpets
and empty
jars
with torches
inside them.
\VS{17}He said
to
them, “Watch
me
and do as I do.
Watch
closely! I
am
going
to the edge
of the camp.
Do
as I
do!
\VS{18}When
I
and all
who
are with
me blow our
trumpets,
you
also
blow
your
trumpets all
around
the camp.
Then say,
‘For the
{\ND{Lord}}
and for Gideon!’ ”
\par }{\PP \VS{19}Gideon
took
a hundred
men
to
the edge
of the camp
at the beginning
of the middle
watch,
just after
they had changed
the guards.
They blew
their trumpets
and broke
the jars
they were carrying.
\VS{20}All three
units
blew
their trumpets
and broke
their jars.
They held
the torches
in their left hand
and the trumpets
in their right.
Then they yelled,
“A sword
for the
{\ND{Lord}}
and for Gideon!”
\VS{21}They stood
in order all around
the camp.
The whole
army
ran
away; they shouted
as they scrambled away.
\VS{22}When the three
hundred
men blew
their trumpets,
the {\ND{Lord}}
caused the Midianites to attack one
another
with their swords
throughout
the camp.
The army
fled
to
Beth Shittah
on the way to Zererah.
They went to
the border
of Abel Meholah
near
Tabbath.
\VS{23}Israelites
from
Naphtali,
Asher,
and Manasseh
answered the call
and chased
the Midianites.
\par }{\SH Gideon Appeases the Ephraimites
\par }{\PP \VS{24}Now Gideon
sent
messengers
throughout
the Ephraimite
hill country
who announced,
“Go down
and head off
the Midianites.
Take control
of the fords of the streams
all the way to
Beth Barah
and the Jordan River.”
When all
the Ephraimites
had assembled,
they took control
of the fords
all the way to
Beth Barah
and the Jordan River.
\VS{25}They captured
the two
Midianite
generals,
Oreb
and Zeeb.
They executed
Oreb
on the rock
of Oreb
and Zeeb
in the winepress
of Zeeb.
They chased
the Midianites
and brought
the heads
of Oreb
and Zeeb
to
Gideon,
who was now on the other side
of the Jordan River.

\par }\Chap{8}{\PP \VerseOne{1}The Ephraimites
said
to
him, “Why
have you done
such
a thing
to us? You did not
summon
us when
you went
to fight
the Midianites!” They argued
vehemently
with him.
\VS{2}He said
to
them, “Now
what
have I accomplished
compared to you? Even Ephraim’s
leftover grapes
are better
quality than Abiezer’s
harvest!
\VS{3}It was to you
that God
handed
over the
Midianite
generals,
Oreb
and Zeeb! What
did
I accomplish
to rival
that?” When
he said
this,
they calmed down.
\par }{\SH Gideon Tracks Down the Midianite Kings
\par }{\PP \VS{4}Now Gideon
and his three
hundred
men
had crossed over
the Jordan River,
and even though they were exhausted,
they were still chasing the Midianites.
\VS{5}He said
to the men
of Succoth,
“Give
some loaves
of bread
to the men
who are following me,
because
they are
exhausted.
I am
chasing
Zebah
and Zalmunna,
the kings
of Midian.”
\VS{6}The officials
of Succoth
said,
“You have not yet overpowered
Zebah
and Zalmunna.
So
why should we give
bread
to your army?”
\VS{7}Gideon
said,
“Since
you will not help, after the
{\ND{Lord}}
hands Zebah
and Zalmunna
over
to me, I will thresh
your skin
with
desert
thorns
and briers.”
\VS{8}He went up
from there
to Penuel
and made the same request.
The men
of Penuel
responded
the same way the men
of Succoth
had.
\VS{9}He also
threatened
the men
of Penuel,
warning, “When I return
victoriously,
I will tear
down this
tower.”
\par }{\PP \VS{10}Now Zebah
and Zalmunna
were in Karkor
with
their armies. There were about fifteen
thousand
survivors
from the army
of the eastern
peoples;
a hundred
and twenty
thousand
sword-wielding
soldiers had been killed.
\VS{11}Gideon
went up
the road
of the nomads
east
of Nobah
and Jogbehah
and ambushed
the surprised
army.
\VS{12}When Zebah
and Zalmunna
ran
away, Gideon chased
them
and captured
the two
Midianite
kings,
Zebah
and Zalmunna.
He had surprised
their entire
army.
\par }{\PP \VS{13}Gideon
son
of Joash
returned
from
the battle
by the pass
of Heres.
\VS{14}He captured
a young
man
from Succoth
and interrogated
him. The young man wrote down
for
him the names of Succoth’s
officials
and city leaders
– seventy-seven
men in all.
\VS{15}He approached
the men
of Succoth
and said,
“Look
what I have! Zebah
and Zalmunna! You insulted
me, saying,
‘You have not yet overpowered
Zebah
and Zalmunna.
So
why should we give
bread
to your exhausted
men?’ ”
\VS{16}He seized
the leaders
of the city,
along
with some desert
thorns
and briers;
he then “threshed”
the men
of Succoth with them.
\VS{17}He also
tore
down the tower
of Penuel
and executed
the city’s
men.
\par }{\PP \VS{18}He said
to
Zebah
and Zalmunna,
“Describe for me
the men
you killed
at Tabor.”
They said,
“They were like
you. Each one
looked like
a king’s
son.”
\VS{19}He said,
“They were
my brothers,
the sons
of my mother.
I swear,
as surely
as the
{\ND{Lord}}
is alive, if
you had
let them live,
I would
not
kill you.”
\VS{20}He ordered
Jether
his firstborn son,
“Come on! Kill
them!” But
Jether was too afraid
to draw
his sword,
because
he was still
young.
\VS{21}Zebah
and Zalmunna
said
to Gideon, “Come on, you
strike
us, for
a man
is judged by his strength.”
So
Gideon
killed
Zebah
and Zalmunna,
and he took
the
crescent-shaped ornaments
which
were on the necks
of their camels.
\par }{\SH Gideon Rejects a Crown but Makes an Ephod
\par }{\PP \VS{22}The men
of Israel
said
to
Gideon,
“Rule over us – you, your son, and your grandson. For you have delivered us from Midian’s power.”
\VS{23}Gideon
said
to
them, “I
will not
rule
over you, nor
will my son
rule
over you. The
{\ND{Lord}}
will rule over you.”
\VS{24}Gideon
continued, “I would like to make one request.
Each
of you give
me an
earring
from the plunder
you have taken.” (The Midianites had gold
earrings
because
they
were Ishmaelites.)
\VS{25}They said,
“We are happy to give
you
earrings.” So they spread
out a garment,
and each one
threw
an earring
from his plunder onto it.
\VS{26}The total weight
of the gold
earrings
he requested
came to seventeen hundred
gold
shekels. This was in addition to
the crescent-shaped ornaments,
jewelry,
purple
clothing
worn by
the Midianite
kings,
and the necklaces
on the camels.
\VS{27}Gideon
used all this to make
an ephod,
which he put
in his hometown
of Ophrah.
All
the Israelites
prostituted
themselves
to it by worshiping it there.
It became
a snare
to Gideon
and his family.
\par }{\SH Gideon’s Story Ends
\par }{\PP \VS{28}The Israelites
humiliated
Midian;
the Midianites’ fighting spirit
was broken.
The land
had rest
for forty
years
during Gideon’s
time.
\VS{29}Then
Jerub-Baal
son
of Joash
went home
and settled down.
\VS{30}Gideon
fathered
seventy
sons
through
his many
wives.
\VS{31}His concubine,
who
lived in Shechem,
also
gave
him
a son,
whom he named
Abimelech.
\VS{32}Gideon
son
of Joash
died
at a very old age
and was buried
in the tomb
of his father
Joash
located in Ophrah
of the Abiezrites.
\par }{\SH Israel Returns to Baal-Worship
\par }{\PP \VS{33}After
Gideon
died,
the Israelites
again
prostituted
themselves to the Baals.
They made
Baal-Berith
their god.
\VS{34}The Israelites
did not
remain
true to the
{\ND{Lord}}
their God,
who had delivered
them from all
the enemies
who lived around them.
\VS{35}They did not
treat
the family
of Jerub-Baal
(that is, Gideon) fairly
in return for all
the good
he had
done
for Israel.

\par }\Chap{9}{\PP \VerseOne{1}Now Abimelech
son
of Jerub-Baal
went
to Shechem
to
see his mother’s
relatives.
He said
to
them and to
his mother’s
entire
extended family,
\VS{2}“Tell
all
the leaders
of Shechem
this: ‘Why
would you want
to have seventy
men,
all
Jerub-Baal’s
sons,
ruling
over you, when you can have just one
ruler? Recall
that
I
am your own flesh
and blood.’ ”
\VS{3}His mother’s
relatives
spoke
on
his behalf
to all
the leaders
of Shechem
and reported his proposal.
The leaders were drawn
to Abimelech;
they said,
“He
is our close relative.”
\VS{4}They paid
him seventy
silver
shekels out of the temple
of Baal-Berith.
Abimelech
then used the silver to hire
some lawless,
dangerous
men
as his followers.
\VS{5}He went
to his father’s
home
in Ophrah
and murdered
his half-brothers,
the seventy
legitimate
sons
of Jerub-Baal,
on
one
stone.
Only Jotham,
Jerub-Baal’s
youngest
son,
escaped,
because
he hid.
\VS{6}All
the leaders
of Shechem
and Beth
Millo
assembled
and then went
and made
Abimelech
king
by the oak
near
the pillar
in Shechem.
\par }{\SH Jotham’s Parable
\par }{\PP \VS{7}When Jotham
heard
the news, he went
and stood
on the top
of Mount
Gerizim.
He spoke loudly
to the people below, “Listen
to me,
leaders
of Shechem,
so
that God
may listen
to you!
\par }{\PP \VS{8}“The trees
were determined
to go out
and choose
a king
for themselves. They said
to the olive
tree, ‘Be our king!’
\VS{9}But the olive
tree said
to them, ‘I am not going to stop
producing my oil,
which
is used to honor
gods
and men,
just to sway
above
the other trees!’
\par }{\PP \VS{10}“So the trees
said
to the fig tree,
‘You
come
and be our king!’
\VS{11}But the fig tree
said
to them, ‘I am not
going to stop producing
my sweet
figs, my excellent fruit,
just to sway
above
the other trees!’
\par }{\PP \VS{12}“So the trees
said
to the grapevine,
‘You
come
and be our king!’
\VS{13}But the grapevine
said
to them, ‘I am not going to stop
producing my wine,
which makes
gods
and men
so happy, just to sway
above
the other trees!’
\par }{\PP \VS{14}“So all
the trees
said to
the thornbush,
‘You
come
and be our king!’
\VS{15}The thornbush
said
to
the trees,
‘If
you
really
want to choose
me as your king,
then come along,
find safety
under my branches! Otherwise
may fire
blaze
from
the thornbush
and consume
the cedars
of Lebanon!’
\par }{\PP \VS{16}“Now,
if
you have shown loyalty
and integrity
when you made
Abimelech
king,
if
you have done
right
to
Jerub-Baal
and his family,
if
you have properly repaid him –
\VS{17}my father
fought
for you; he risked
his life
and delivered
you from
Midian’s
power.
\VS{18}But you
have attacked
my father’s
family
today.
You murdered
his seventy
legitimate sons
on
one
stone
and made
Abimelech,
the son
of his female slave,
king over
the leaders
of Shechem,
just because
he is
your close relative.
\VS{19}So if
you have shown
loyalty
and integrity
to Jerub-Baal
and his family
today,
then may Abimelech
bring you happiness
and
may you bring him
happiness!
\VS{20}But if
not,
may fire
blaze from
Abimelech
and consume
the leaders
of Shechem
and Beth
Millo! May fire
also blaze from
the leaders
of Shechem
and Beth
Millo
and consume
Abimelech!”
\VS{21}Then Jotham
ran away
to Beer
and lived
there
to escape from
Abimelech
his half-brother.
\par }{\SH God Fulfills Jotham’s Curse
\par }{\PP \VS{22}Abimelech
commanded
Israel
for
three
years.
\VS{23}God
sent
a spirit
to stir up hostility
between
Abimelech
and the leaders
of Shechem.
He made the leaders
of Shechem
disloyal
to Abimelech.
\VS{24}He did this so the violent
deaths of Jerub-Baal’s
seventy
sons
might be avenged and Abimelech,
their half-brother
who
murdered
them, might have to pay
for their spilled blood,
along
with the leaders
of Shechem
who
helped
him murder
them.
\VS{25}The leaders
of Shechem
rebelled against
Abimelech by putting
bandits
in the hills,
who robbed
everyone
who
traveled
by on
the road.
But Abimelech
found out about it.
\par }{\PP \VS{26}Gaal
son
of Ebed
came
through
Shechem
with his brothers.
The leaders
of Shechem
transferred their loyalty to him.
\VS{27}They went out
to the field,
harvested
their grapes,
squeezed out
the juice, and celebrated.
They came
to the temple
of their god
and ate,
drank,
and cursed
Abimelech.
\VS{28}Gaal
son
of Ebed
said,
“Who
is Abimelech
and who
is Shechem,
that
we should serve
him? Is he not
the son
of Jerub-Baal,
and is not Zebul
the deputy
he appointed? Serve
the sons
of Hamor,
the father
of Shechem! But why
should we
serve Abimelech?
\VS{29}If only
these
men
were under my command,
I would get rid
of Abimelech!” He challenged
Abimelech, “Muster
your army
and come out
for battle!”
\par }{\PP \VS{30}When Zebul,
the city
commissioner,
heard
the words
of Gaal
son
of Ebed,
he was furious.
\VS{31}He sent
messengers
to
Abimelech,
who was in Arumah,
reporting,
“Beware!
 Gaal
son
of Ebed
and his brothers
are coming
to Shechem
and inciting
the city
to rebel against you.
\VS{32}Now,
come up
at night
with
your
men
and set an ambush
in the field outside the city.
\VS{33}In the morning
at sunrise
quickly
attack
the city.
When he and his men
come out
to
fight you, do
what
you can to him.”
\par }{\PP \VS{34}So Abimelech
and all
his men
came up
at night
and set an ambush
outside Shechem
– they divided into four
units.
\VS{35}When Gaal
son
of Ebed
came out
and stood
at the entrance
to the city’s
gate,
Abimelech
and his men
got up
from
their hiding places.
\VS{36}Gaal
saw
the men
and said
to
Zebul,
“Look,
men
are coming down
from the tops
of the hills.”
But Zebul
said
to
him, “You
are seeing
the shadows
on the hills
– it just looks like men.”
\VS{37}Gaal
again
said,
“Look,
men
are coming down
from the very center
of the land.
A unit
is coming
by way
of the Oak
Tree of the Diviners.”
\VS{38}Zebul
said
to him,
“Where
now
are your bragging words, ‘Who
is Abimelech
that
we should serve
him?’ Are these
not
the men
you insulted? Go out
now
and fight them!”
\VS{39}So Gaal
led
the leaders
of
Shechem
out
and fought
Abimelech.
\VS{40}Abimelech
chased
him, and Gaal ran
from
him. Many
Shechemites fell
wounded
at the entrance
of the gate.
\VS{41}Abimelech
went back
to Arumah;
Zebul
drove
Gaal
and his brothers
out
of Shechem.
\par }{\PP \VS{42}The next day
the Shechemites
came
out
to the field.
When Abimelech
heard about it,
\VS{43}he took
his men
and divided
them into three
units
and set an ambush
in the field.
When he saw
the people
coming
out
of the city,
he attacked
and struck them down.
\VS{44}Abimelech
and his units
attacked and blocked
the entrance
to the city’s
gate.
Two
units
then attacked
all
the people in
the field
and struck them down.
\VS{45}Abimelech
fought
against the city
all
that day.
He
captured
the city
and killed
all the people
in it. Then he leveled
the city
and spread
salt over it.
\par }{\PP \VS{46}When all
the leaders
of the Tower
of Shechem
heard
the news, they went
to
the stronghold
of the temple
of El-Berith.
\VS{47}Abimelech
heard that
all
the leaders
of the Tower
of Shechem
were in one place.
\VS{48}He
and all
his men
went up
on Mount
Zalmon.
He
took
an ax
in
his hand
and cut off
a tree
branch.
He put
it on
his shoulder
and said
to
his men,
“Quickly,
do
what
you have just
seen
me do!”
\VS{49}So each
of his men
also
cut off
a branch
and followed
Abimelech.
They put
the branches against
the stronghold
and set
fire
to it.
All
the people
of the Tower
of Shechem
died
– about a thousand
men
and women.
\par }{\PP \VS{50}Abimelech
moved
on to
Thebez;
he besieged
and captured it.
\VS{51}There was
a fortified
tower
in the center
of the city,
so all
the men
and women,
as well as the city’s
leaders,
ran
into it and locked
the entrance. Then they went up
to
the roof
of the tower.
\VS{52}Abimelech
came
and attacked
the tower.
When he approached
the entrance
of the tower
to set it on fire,
\VS{53}a woman
threw
an
upper millstone
down on
his
head
and shattered
his skull.
\VS{54}He quickly
called
to
the young man
who carried
his weapons, “Draw
your sword
and kill
me, so
they will not say, ‘A woman
killed
him.’ ” So the young man
stabbed
him and he died.
\VS{55}When the Israelites
saw
that
Abimelech
was dead,
they went
home.
\par }{\PP \VS{56}God
repaid
Abimelech
for the evil
he did
to his father
by murdering
his seventy
half-brothers.
\VS{57}God
also repaid
the men
of Shechem
for their evil
deeds. The curse
spoken
by
Jotham
son
of Jerub-Baal fell on them.

\par }\Chap{10}{\PP \VerseOne{1}After
Abimelech’s
death, Tola
son
of Puah,
grandson
of Dodo,
from the tribe of Issachar,
rose
up to deliver
Israel.
He
lived
in Shamir
in the Ephraimite
hill country.
\VS{2}He led
Israel
for twenty-three
years,
then died
and was buried
in Shamir.
\par }{\PP \VS{3}Jair
the Gileadite
rose up
after
him; he led
Israel
for twenty-two
years.
\VS{4}He had thirty
sons
who rode
on
thirty
donkeys
and possessed thirty
cities.
To this
day
these
towns are called
Havvoth Jair –
they
are in the land
of Gilead.
\VS{5}Jair
died
and was buried
in Kamon.
\par }{\SH The Lord’s Patience Runs Short
\par }{\PP \VS{6}The Israelites
again
did
evil
in the
{\ND{Lord}}’s
sight.
They worshiped
the
Baals
and the
Ashtars,
as well as the
gods
of Syria,
Sidon,
Moab,
the
Ammonites,
and the
Philistines.
They abandoned
the

{\ND{Lord}}
and did not
worship him.
\VS{7}The
{\ND{Lord}}
was furious
with Israel
and turned
them over
to the Philistines
and Ammonites.
\VS{8}They ruthlessly
oppressed
the Israelites
that eighteenth
year –
that is, all
the Israelites
living east
of the Jordan
in Amorite
country
in Gilead.
\VS{9}The Ammonites
crossed
the Jordan
to fight
with Judah,
Benjamin,
and Ephraim.
Israel
suffered
greatly.
\par }{\PP \VS{10}The Israelites
cried
out for help to
the {\ND{Lord}}: “We have sinned
against you. We
abandoned
our God
and worshiped
the Baals.”
\VS{11}The
{\ND{Lord}}
said
to
the Israelites,
“Did I not
deliver you from
Egypt,
the Amorites,
the Ammonites,
the Philistines,
\VS{12}the Sidonians,
Amalek,
and Midian
when they oppressed
you? You cried out
for help to
me, and I delivered
you from their power.
\VS{13}But
since you
abandoned
me and worshiped
other
gods,
I
will not
deliver
you again.
\VS{14}Go
and cry
for help to
the gods
you have
chosen! Let them
deliver
you from trouble!”
\VS{15}But the Israelites
said to
the {\ND{Lord}}, “We have sinned.
You do
to us as you
see
fit,
but deliver
us
today!”
\VS{16}They threw away
the foreign
gods
they owned
and worshiped
the {\ND{Lord}}. Finally
the
{\ND{Lord}} grew tired
of seeing
Israel
suffer
so much.
\par }{\SH An Outcast Becomes a General
\par }{\PP \VS{17}The Ammonites
assembled
and camped
in Gilead;
the Israelites
gathered together
and camped
in Mizpah.
\VS{18}The leaders
of Gilead
said
to
one
another,
“Who
is willing
to lead the charge against
the Ammonites? He will become
the leader
of all
who live
in Gilead!”

\par }\Chap{11}{\PP \VerseOne{1}Now Jephthah
the Gileadite
was
a brave
warrior.
His mother
was a prostitute,
but Gilead
was his father.
\VS{2}Gilead’s
wife
also gave
him sons.
When his wife’s
sons
grew
up, they made Jephthah
leave and said
to him, “You are not
going to inherit
any of our father’s
wealth, because
you
are another
woman’s
son.”
\VS{3}So Jephthah
left
his half-brothers
and lived
in the land
of Tob.
Lawless
men
joined
Jephthah’s
gang and traveled
with him.
\par }{\PP \VS{4}It was some
time
after this when the Ammonites
fought
with
Israel.
\VS{5}When
the Ammonites
attacked,
the leaders
of Gilead
asked Jephthah
to come back from the land
of Tob.
\VS{6}They said, “Come,
be
our commander,
so we can fight
with the Ammonites.”
\VS{7}Jephthah
said
to the leaders
of Gilead,
“But you
hated
me and made me leave
my father’s
house.
Why
do you come
to me
now,
when you are in trouble?”
\VS{8}The leaders
of Gilead
said
to
Jephthah,
“That may be true,
but now
we pledge
to
you our loyalty. Come
with
us and fight
with the Ammonites.
Then you will become
the leader
of all
who live
in Gilead.”
\VS{9}Jephthah
said
to
the leaders
of Gilead,
“All right! If
you
take me back
to fight
with the Ammonites
and the
{\ND{Lord}}
gives
them to me, I
will be
your leader.”
\VS{10}The leaders
of Gilead
said
to
Jephthah,
“The
{\ND{Lord}}
will judge any grievance
you have
against us, if
we do not
do
as you say.”
\VS{11}So Jephthah
went
with
the leaders
of Gilead.
The people
made
him
their leader
and commander.
Jephthah
repeated
the terms
of the agreement before
the {\ND{Lord}}
in Mizpah.
\par }{\SH Jephthah Gives a History Lesson
\par }{\PP \VS{12}Jephthah
sent
messengers
to
the Ammonite
king,
saying,
“Why
have you come against
me to
attack
my land?”
\VS{13}The Ammonite
king
said
to
Jephthah’s
messengers,
“Because
Israel
stole
my land
when they came up
from Egypt
– from the Arnon
River
in the south to the Jabbok
River in the north, and as far
west as the Jordan.
Now
return
it peaceably!”
\par }{\PP \VS{14}Jephthah
sent
messengers
back to
the Ammonite
king
\VS{15}and said
to him, “This is what
Jephthah
says,
‘Israel
did not
steal
the land
of Moab
and the land
of the Ammonites.
\VS{16}When
they left
Egypt,
Israel
traveled
through the desert
as far
as the Red
Sea
and then came
to Kadesh.
\VS{17}Israel
sent
messengers
to
the king
of Edom,
saying,
“Please
allow us to pass through
your land.”
But
the king
of Edom
rejected the request.
Israel sent
the same
request to
the king
of Moab,
but he was unwilling
to cooperate. So Israel
stayed
at Kadesh.
\VS{18}Then Israel went through
the desert
and bypassed
the land
of Edom
and the land
of Moab.
They traveled
east
of the land
of Moab
and camped
on the other side
of the Arnon
River; they did not
go
through Moabite
territory
(the Arnon
was Moab’s
border).
\VS{19}Israel
sent
messengers
to
King
Sihon,
the Amorite
king
who ruled in Heshbon,
and said
to him, “Please
allow us to pass
through your land
to our land.”
\VS{20}But Sihon
did not
trust
Israel
to pass through
his territory.
He
assembled
his whole
army,
camped
in Jahaz,
and fought
with
Israel.
\VS{21}The
{\ND{Lord}}
God
of Israel
handed Sihon
and his whole
army
over
to Israel
and they defeated
them.
Israel
took
all
the land
of the Amorites
who lived
in that land.
\VS{22}They took
all
the Amorite
territory
from the Arnon
River
on the south to the Jabbok
River on the north, from
the desert
in the east to
the Jordan in the west.
\VS{23}Since
the {\ND{Lord}}
God
of Israel
has driven
out the Amorites
before
his people
Israel,
do you
think you can just take it from them?
\VS{24}You have
the right to take
what Chemosh
your god
gives you, but we will take
the land of all
whom
the {\ND{Lord}}
our God
has driven
out before us.
\VS{25}Are you really
better
than
Balak
son
of Zippor,
king
of Moab? Did he dare to quarrel
with
Israel? Did he dare to fight with them?
\VS{26}Israel
has been
living
in Heshbon
and its nearby towns,
in Aroer
and its nearby towns,
and in all
the cities
along
the Arnon
for three
hundred
years! Why
did you not
reclaim them
during that time?
\VS{27}I
have not
done you wrong,
but you
are doing
wrong
by attacking
me. May
the {\ND{Lord}}, the Judge,
judge
this day
between
the Israelites
and the Ammonites!’ ”
\VS{28}But the Ammonite
king
disregarded
the message
sent
by Jephthah.
\par }{\SH A Foolish Vow Spells Death for a Daughter
\par }{\PP \VS{29}The
{\ND{Lord}}’s
spirit
empowered
Jephthah.
He passed through
Gilead
and Manasseh
and went
to Mizpah
in Gilead.
From there he approached
the Ammonites.
\VS{30}Jephthah
made
a vow
to the
{\ND{Lord}}, saying,
“If
you really do hand
the Ammonites
over to me,
\VS{31}then
whoever is the first to come through
the doors
of my house
to meet
me when I return
safely
from fighting the Ammonites
– he will belong
to the
{\ND{Lord}}
and I will offer him up
as a burnt sacrifice.”
\VS{32}Jephthah
approached
the Ammonites
to fight
with them,
and the
{\ND{Lord}}
handed them over to him.
\VS{33}He defeated
them from Aroer
all the way to
Minnith
– twenty
cities
in all,
even as far
as Abel
Keramim! He wiped them out! The Israelites
humiliated
the Ammonites.
\par }{\PP \VS{34}When Jephthah
came
home
to Mizpah,
there was
his daughter
hurrying out
to meet
him, dancing
to the rhythm
of tambourines.
She was his only child;
except
for her he had no
son
or
daughter.
\VS{35}When
he saw
her,
he ripped
his clothes
and said,
“Oh
no! My daughter! You have completely ruined
me! You
have brought me disaster! I
made an oath
to
the {\ND{Lord}},
and I cannot
break it.”
\VS{36}She said
to him,
“My father,
since you made an oath
to
the {\ND{Lord}}, do
to me as
you promised.
After
all, the
{\ND{Lord}}
vindicated
you before your enemies,
the Ammonites.”
\VS{37}She then said
to
her father,
“Please grant
me this
one
wish.
For two
months
allow
me to walk
through
the hills
with my friends
and mourn
my virginity.”
\VS{38}He said,
“You may
go.”
He permitted
her to leave for two
months.
She went
with her friends
and mourned
her virginity
as she walked through the hills.
\VS{39}After
two
months
she returned
to
her father,
and he did
to her as he had
vowed.
She
died a virgin.
Her tragic death gave rise to a custom
in Israel.
\VS{40}Every year
Israelite
women commemorate
the daughter
of Jephthah
the Gileadite
for four
days.

\par }\Chap{12}{\PP \VerseOne{1}The Ephraimites
assembled
and crossed over
to Zaphon.
They said
to Jephthah,
“Why
did you go
and fight
with the Ammonites
without
asking
us to go
with
you? We will burn
your house
down
right over you!”
\par }{\PP \VS{2}Jephthah
said
to them,
“My people
and I
were entangled
in controversy
with the Ammonites.
I asked for
your help,
but you did not
deliver
me from their power.
\VS{3}When I saw
that
you were not
going to help,
I risked
my life and advanced
against the Ammonites,
and the
{\ND{Lord}}
handed
them over to me. Why
have you come up
to
fight
with me today?”
\VS{4}Jephthah
assembled
all
the men
of Gilead
and they fought
with
Ephraim.
The men
of Gilead
defeated
Ephraim,
because
the Ephraimites insulted them, saying, “You
Gileadites
are refugees
in Ephraim,
living within
Ephraim’s
and Manasseh’s
territory.”
\VS{5}The Gileadites
captured
the fords
of the Jordan River
opposite Ephraim.
Whenever an Ephraimite
fugitive
said,
“Let me cross over,”
the men
of Gilead
asked him, “Are you
an Ephraimite?” If he said,
“No,”
\VS{6}then they said
to him, “Say
‘Shibboleth!’ ” If he said,
“Sibboleth”
(and could not
pronounce
the word correctly), they grabbed
him and executed
him
right there at the fords
of the Jordan.
On that
day
forty-two
thousand
Ephraimites
fell dead.
\VS{7}Jephthah
led
Israel
for six
years;
then he died
and was
buried
in his city
in Gilead.
\par }{\SH Order Restored
\par }{\PP \VS{8}After
him Ibzan
of Bethlehem
led
Israel.
\VS{9}He had thirty
sons.
He arranged for thirty
of his daughters
to be married outside
his extended
family, and he arranged for thirty
young women
to be brought
from
outside
as wives for his sons.
Ibzan led
Israel
for seven
years;
\VS{10}then he
died
and was buried
in Bethlehem.
\par }{\PP \VS{11}After
him Elon
the Zebulunite
led
Israel
for ten
years.
\VS{12}Then Elon the Zebulunite
died
and was buried
in Aijalon
in the land
of Zebulun.
\par }{\PP \VS{13}After
him Abdon
son
of Hillel
the Pirathonite
led
Israel.
\VS{14}He had forty
sons
and thirty
grandsons
who rode
on
seventy
donkeys.
He led
Israel
for eight
years.
\VS{15}Then
Abdon
son
of Hillel
the Pirathonite
died
and was buried
in Pirathon
in the land
of Ephraim,
in the hill country
of the Amalekites.

\par }\Chap{13}{\PP \VerseOne{1}The Israelites
again
did
evil
in the
{\ND{Lord}}’s
sight,
so
the {\ND{Lord}}
handed
them over to the Philistines
for forty
years.
\par }{\PP \VS{2}There was
a man
named
Manoah
from Zorah,
from the Danite
tribe.
His wife
was infertile
and childless.
\VS{3}The
{\ND{Lord}}’s
angelic messenger
appeared
to
the woman
and said
to her,
“You
are infertile
and childless,
but you will conceive
and have a son.
\VS{4}Now
be careful! Do not
drink
wine
or beer,
and do not
eat
any
food that will make you ritually unclean.
\VS{5}Look,
you will conceive
and have a son.
You must never
cut his hair, for
the child
will be
dedicated
to God
from
birth.
He
will begin
to deliver
Israel
from the power
of the Philistines.”
\par }{\PP \VS{6}The woman
went
and said
to her husband,
“A man
sent from God
came
to
me! He looked like
God’s
angelic
messenger
– he was very
awesome.
I did not
ask
him where
he came from, and he
did not
tell
me his name.
\VS{7}He said
to me, ‘Look,
you will conceive
and have a son.
So now,
do not
drink
wine
or beer
and do not
eat
any
food that will make you ritually unclean.
For
the child
will be
dedicated
to God
from
birth
till
the day
he dies.’ ”
\par }{\PP \VS{8}Manoah
prayed
to
the {\ND{Lord}}, “Please,
Lord,
allow
the man
sent
from God
to visit
us again,
so he can teach
us how
we should raise
the child
who will be born.”
\VS{9}God
answered
Manoah’s
prayer. God’s
angelic messenger
visited
the woman
again
while she
was sitting
in the field.
But her husband
Manoah
was not
with her.
\VS{10}The woman
ran
at once and told
her
husband, “Come quickly,
the man
who visited
me
the other
day
has appeared
to me!”
\VS{11}So Manoah
got up
and followed
his wife.
When he met
the man,
he said
to him, “Are you
the man
who
spoke
to
my wife?” He said,
“Yes.”
\VS{12}Manoah
said,
“Now,
when your announcement
comes
true, how
should the child be raised and what should he do?”
\VS{13}The
{\ND{Lord}}’s
messenger
told
Manoah,
“Your wife
should pay attention
to everything
I told
her.
\VS{14}She should not drink
anything
that
the grapevine
produces.
She must not
drink
wine
or beer,
and she must not
eat
any
food that will make her ritually unclean.
She should
obey
everything
I commanded her to do.”
\VS{15}Manoah
said
to
the
{\ND{Lord}}’s
messenger,
“Please
stay here awhile,
so we can prepare
a young
goat for you to eat.”
\VS{16}The
{\ND{Lord}}’s
messenger
said
to
Manoah,
“If
I stay,
I will not
eat
your food.
But if
you want to make
a burnt sacrifice
to the
{\ND{Lord}}, you should offer
it.” (He said this because
Manoah
did not
know
that
he was the
{\ND{Lord}}’s
messenger.)
\VS{17}Manoah
said
to
the
{\ND{Lord}}’s
messenger,
“Tell us your name,
so we can honor
you when your announcement
comes true.”
\VS{18}The
{\ND{Lord}}’s
messenger
said
to him, “You should
not ask
me my name,
because you cannot comprehend it.”
\VS{19}Manoah
took
a young
goat
and a grain offering
and offered
them on
a rock
to the
{\ND{Lord}}. The
{\ND{Lord}}’s messenger did
an amazing
thing as Manoah
and his wife
watched.
\VS{20}As the flame
went up
from the altar
toward the sky,
the
{\ND{Lord}}’s
messenger
went up
in it
while Manoah
and his wife
watched.
They fell
facedown to the ground.
\par }{\PP \VS{21}The
{\ND{Lord}}’s
messenger
did not
appear
again
to
Manoah
and his
wife.
After all this happened Manoah
realized
that
the visitor had been the
{\ND{Lord}}’s
messenger.
\VS{22}Manoah
said
to
his wife,
“We
will certainly die,
because
we have seen
a supernatural being!”
\VS{23}But his wife
said
to him, “If
the {\ND{Lord}}
wanted
to kill
us,
he would
not
have accepted
the burnt offering
and the grain offering
from us. He would
not
have shown
us
all
these
things,
or
have
spoken
to us like
this just now.”
\par }{\PP \VS{24}Manoah’s
wife
gave birth to a son
and named
him Samson.
The child
grew
and the
{\ND{Lord}}
empowered him.
\VS{25}The
{\ND{Lord}}’s
spirit
began
to control him in Mahaneh
Dan
between
Zorah
and Eshtaol.

\par }\Chap{14}{\PP \VerseOne{1}Samson
went down
to Timnah,
where a Philistine
girl
caught his eye.
\VS{2}When
he got
home, he told
his father
and mother,
“A Philistine
girl
in Timnah
has caught my eye. Now
get
her for my wife.”
\VS{3}But his father
and mother
said
to him, “Certainly
you can find a wife
among your relatives
or among all
our people! You
should not have to go
and get
a wife
from the uncircumcised
Philistines.”
But Samson
said
to
his father,
“Get
her for me, because
she
is the right one for me.”
\VS{4}Now his father
and mother
did not
realize
this
was
the
{\ND{Lord}}’s
doing, because
he was
looking
for an opportunity
to stir up trouble with the Philistines
(for at that time
the Philistines
were ruling
Israel).
\par }{\PP \VS{5}Samson
went down
to Timnah.
When he approached
the vineyards
of Timnah,
he saw
a roaring
young
lion
attacking him.
\VS{6}The
{\ND{Lord}}’s
spirit
empowered him
and he tore the lion in two with his bare hands
as easily
as one
would tear a young goat.
But he did not
tell
his father
or mother
what he had
done.
\par }{\PP \VS{7}Samson
continued on down
to Timnah
and spoke
to the girl.
In his opinion, she was just the right one.
\VS{8}Some time
later, when he went back
to marry
her, he turned aside
to see
the lion’s
remains.
He saw
a swarm
of bees
in the lion’s
carcass,
as well as some honey.
\VS{9}He scooped
it up with his hands
and ate
it as he walked
along. When he returned
to
his father
and mother,
he offered
them some and they ate
it. But he did not
tell
them he had
scooped
the honey
out of the lion’s
carcass.
\par }{\PP \VS{10}Then Samson’s
father
accompanied him
to Timnah for the marriage.
Samson
hosted
a party
there,
for
this
was customary
for bridegrooms
to do.
\VS{11}When
the Philistines saw
he had no attendants, they gave him thirty
groomsmen who kept him company.
\VS{12}Samson
said
to them, “I will give
you a riddle.
If
you really can solve
it during the seven
days
the party
lasts,
I will give
you thirty
linen
robes and thirty
sets
of clothes.
\VS{13}But if
you cannot
solve
it, you
will give
me thirty
linen
robes and thirty
sets
of clothes.”
They said
to him, “Let
us hear
your riddle.”
\VS{14}He said
to them,
\par }{\Q “Out
of the one who eats
came something
to eat;
\par }{\Q out
of the strong
one came something sweet.”
\par }{\PI They could
not
solve
the riddle
for three
days.
\par }{\PP \VS{15}On
the fourth
day
they said
to Samson’s
bride, “Trick
your husband
into giving the solution to the riddle.
If you refuse, we
will burn up
you and your father’s
family.
Did you invite
us here to make us poor?”
\VS{16}So Samson’s
bride cried
on
his shoulder
and said,
“You must
hate
me; you do not
love
me! You told the young men
a riddle,
but you have not
told me the solution.”
He said
to her, “Look,
I have not
even told my father
or mother. Do you really expect me to tell you?”
\VS{17}She cried
on
his shoulder until the party
was
almost
over. Finally, on
the seventh
day, he told
her because
she had nagged
him so much. Then she told
the young
men
the solution to the riddle.
\VS{18}On the seventh
day,
before
the sun
set,
the men
of the city
said
to him,
\par }{\Q “What
is sweeter
than honey?
\par }{\Q What
is stronger
than a lion?”
\par }{\PI He said
to them,
\par }{\Q “If
you had not plowed
with my heifer,
\par }{\Q you would not
have
solved my riddle!”
\par }{\PP \VS{19}The
{\ND{Lord}}’s
spirit
empowered
him. He went down
to Ashkelon
and murdered
thirty
men.
He took
their clothes
and gave
them to the men who had solved
the riddle.
He was furious
as he went back
home.
\VS{20}Samson’s
bride was then given to his best man.

\par }\Chap{15}{\PP \VerseOne{1}Sometime later,
during the wheat
harvest,
Samson
took a young
goat
as a gift and went to visit
his bride.
He said
to her father, “I want to have sex
with
my bride
in her bedroom!” But her father
would not
let
him enter.
\VS{2}Her father
said,
“I really thought
you absolutely
despised
her, so I gave
her to your best man.
Her younger
sister
is more attractive
than
she is. Take
her instead!”
\VS{3}Samson
said
to them, “This time
I am
justified
in doing
the Philistines
harm!”
\VS{4}Samson
went
and captured
three
hundred
jackals
and got some
torches.
He tied the jackals in pairs
by their tails
and then tied
a torch
to each
pair.
\VS{5}He lit
the torches
and set
the jackals loose in the Philistines’
standing grain.
He burned
up the grain heaps
and the standing grain,
as well as
the vineyards
and olive
groves.
\VS{6}The Philistines
asked, “Who
did
this?” They were told, “Samson,
the Timnite’s
son-in-law,
because
the Timnite took
Samson’s
bride and gave
her to his best man.”
So
the Philistines
went up
and burned
her and her father.
\VS{7}Samson
said
to them, “Because
you did
this,
I will get revenge
against you before I quit fighting.”
\VS{8}He struck
them down
and defeated
them. Then
he went down
and lived
for a time in the cave
in the cliff
of Etam.
\par }{\PP \VS{9}The Philistines
went up
and invaded
Judah.
They arrayed
themselves for battle in Lehi.
\VS{10}The men
of Judah
said,
“Why
are you attacking
us?” The Philistines said,
“We have come up
to take Samson
prisoner
so we can do
to him what
he has done to us.”
\VS{11}Three
thousand
men
of Judah
went down
to
the cave
in the cliff
of Etam
and said
to Samson,
“Do you not
know
that
the Philistines
rule over
us? Why
have you done
this
to us?” He said
to them, “I have only done
to them what
they have done to me.”
\VS{12}They said
to him, “We have come down
to take
you prisoner
so we can hand
you over to the Philistines.”
Samson
said
to them, “Promise
me you
will not
kill me.”
\VS{13}They said
to him, “We promise! We will only take you prisoner
and hand
you over
to them. We promise
not
to kill
you.” They tied
him up
with two
brand new
ropes
and led
him up
from
the cliff.
\VS{14}When he
arrived
in Lehi,
the Philistines
shouted
as they approached
him. But the
{\ND{Lord}}’s
spirit
empowered
him. The ropes
around
his arms
were like flax
dissolving
in fire,
and they melted away
from his hands.
\VS{15}He happened to see
a solid
jawbone
of a donkey.
He grabbed
it and struck down
a thousand
men.
\VS{16}Samson
then said,
\par }{\Q “With the jawbone
of a donkey
\par }{\Q I have left them in heaps;
\par }{\Q with the jawbone
of a donkey
\par }{\Q I have struck
down a thousand
men!”
\par }{\PP \VS{17}When
he finished
speaking,
he threw
the jawbone
down and named
that place
Ramath Lehi.
\par }{\PP \VS{18}He was very
thirsty,
so he cried
out to
the {\ND{Lord}}
and said,
“You
have given
your servant
this
great
victory.
But now
must I die
of thirst
and fall
into hands
of the Philistines?”
\VS{19}So God
split open
the
basin
at Lehi
and water
flowed out
from
it. When he took a drink,
his strength
was restored
and he revived.
For this reason
he named
the spring En Hakkore.
It remains in Lehi
to this
very
day.
\VS{20}Samson led
Israel
for twenty
years
during the days
of Philistine prominence.

\par }\Chap{16}{\PP \VerseOne{1}Samson
went
to Gaza.
There
he saw
a prostitute
and went
in to
have sex with her.
\VS{2}The Gazites
were told, “Samson
has come
here!” So they surrounded
the town and hid
all
night
at the city
gate,
waiting for him to leave. They relaxed
all
night,
thinking, “He will not leave until
morning
comes;
then we will kill him!”
\VS{3}Samson
spent
half
the night
with the prostitute; then he got up
in the middle
of the night
and left. He grabbed
the doors
of the city
gate,
as well as the two
posts,
and pulled
them right
off,
bar
and all. He put
them on
his shoulders
and carried them up
to
the top
of a hill
east of Hebron.
\par }{\PP \VS{4}After
this Samson
fell in love
with a woman
named
Delilah,
who lived in the Sorek
Valley.
\VS{5}The rulers
of the Philistines
went up
to
visit her and said
to her, “Trick
him! Find out
what
makes him so
strong
and how
we can
subdue
him and humiliate
him. Each one
of us
will give
you eleven hundred
silver pieces.”
\par }{\PP \VS{6}So Delilah
said
to
Samson,
“Tell
me
what
makes you so
strong
and how
you can be subdued
and humiliated.”
\VS{7}Samson
said
to
her, “If
they tie
me up
with seven
fresh
bowstrings
that
have not
been dried,
I will become weak
and be
just like any other
man.”
\VS{8}So
the rulers
of the Philistines
brought her seven
fresh
bowstrings
which
had not
been dried
and they tied
him up with them.
\VS{9}They hid
in
the bedroom
and then she said
to
him, “The Philistines
are here, Samson!” He snapped
the
bowstrings
as easily
as
a thread
of yarn
snaps when it is put close
to fire.
The secret of his strength
was not
discovered.
\par }{\PP \VS{10}Delilah
said
to
Samson,
“Look,
you deceived
me and told
me
lies! Now
tell
me
how
you can be subdued.”
\VS{11}He said
to
her, “If
they tie
me tightly with brand new
ropes
that
have never
been used,
I will become weak
and be
just like any other
man.”
\VS{12}So Delilah
took
new
ropes
and tied
him with them and said
to
him, “The Philistines
are here, Samson!” (The Philistines were hiding
in the bedroom.) But he tore
the ropes from his arms
as if they were a piece of thread.
\par }{\PP \VS{13}Delilah
said
to
Samson,
“Up to
now
you have deceived
me and told me lies.
Tell
me how
you can be subdued.”
He said
to
her, “If
you weave
the seven
braids
of my hair
into the fabric
on the loom and secure it with the pin, I will become weak and be like any other man.”
\VS{14}So she made him
go to
sleep,
wove the seven braids of his hair into the fabric on the
loom, fastened
it with
the pin,
and said
to him,
“The Philistines
are here, Samson!” He woke
up and tore away the pin
of the
loom
and the
fabric.
\par }{\PP \VS{15}She said
to him,
“How
can you say,
‘I love
you,’ when you will not
share your secret
with
me? Three
times
you have deceived
me and have not
told
me what
makes
you so strong.”
\VS{16}She nagged
him every
day
and pressured
him until he was sick to death of it.
\VS{17}Finally he told
her his secret.
He said
to her, “My hair
has never
been cut,
for
I have been dedicated
to God
from the time I
was conceived.
If
my head were shaved,
my strength
would leave
me; I would become weak,
and be
just like all
other men.”
\VS{18}When Delilah
saw
that
he had told
her his secret,
she sent
for the rulers
of the Philistines,
saying,
“Come up
here again,
for
he has told
me
his secret.”
So
the rulers
of the Philistines
went up
to
visit her, bringing the silver
in their hands.
\VS{19}She made him go to sleep
on
her lap
and then called
a man
in to shave
off the seven
braids
of his hair.
She made him vulnerable
and his strength
left him.
\VS{20}She said,
“The Philistines
are here, Samson!” He woke up
and thought, “I will do
as I did before
and shake
myself free.”
But he did not
realize
that
the {\ND{Lord}}
had left him.
\VS{21}The Philistines
captured
him and gouged out
his eyes.
They brought
him down
to Gaza
and bound
him in bronze
chains. He became
a grinder
in the prison.
\VS{22}His hair
began
to grow
back after it had
been shaved off.
\par }{\SH Samson’s Death and Burial
\par }{\PP \VS{23}The rulers
of the Philistines
gathered
to offer
a great
sacrifice
to Dagon
their god
and to celebrate.
They said,
“Our god
has handed
Samson,
our enemy, over to us.”
\VS{24}When
the people
saw him, they praised
their god,
saying,
“Our god
has handed
our enemy
over to us, the one who ruined
our land
and killed
so many of us!”
\par }{\PP \VS{25}When
they really
started celebrating,
they said,
“Call
for Samson
so he can entertain
us!” So they summoned
Samson
from the prison
and he entertained them. They made him stand
between
two pillars.
\VS{26}Samson
said
to
the young man
who held
his hand,
“Position
me so I can touch
the pillars
that
support
the temple.
Then I can lean
on them.”
\VS{27}Now the temple
was filled
with men
and women,
and all
the rulers
of the Philistines
were there. There
were three
thousand
men
and women
on the roof
watching
Samson
entertain.
\VS{28}Samson
called
to
the
{\ND{Lord}}, “O Master,

{\ND{Lord}}, remember
me! Strengthen
me
just
one more time,
O God,
so I can get swift
revenge
against the Philistines
for my two
eyes!”
\VS{29}Samson
took hold
of the two
middle
pillars
that
supported
the temple
and he leaned
against
them, with his right hand
on one
and his left hand
on the other.
\VS{30}Samson
said,
“Let me die
with
the Philistines!” He pushed
hard and the temple
collapsed
on
the rulers
and all
the people
in it.
He killed
many
more people in his death
than he had
killed
during his life.
\VS{31}His brothers
and all
his family
went down
and brought
him back.
They buried
him between
Zorah
and Eshtaol
in the tomb
of Manoah
his father.
He
had led
Israel
for twenty
years.

\par }\Chap{17}{\PP \VerseOne{1}There was
a man
named
Micah
from the Ephraimite
hill country.
\VS{2}He said
to his mother,
“You know the eleven hundred
pieces of silver
which
were stolen
from you,
about which I heard
you pronounce
a curse? Look
here, I have the silver.
I
stole
it, but now I am giving it back to you.” His mother
said,
“May the
{\ND{Lord}}
reward
you, my son!”
\VS{3}When he gave back
to his mother
the
eleven hundred
pieces of silver,
his mother
said,
“I solemnly dedicate
this silver
to the
{\ND{Lord}}. It will be for my son’s
benefit.
We will use it to make
a carved image
and a metal image.”
\VS{4}When he gave the silver
back
to his mother,
she
took
two hundred
pieces of silver
to a silversmith,
who made
them into a carved
image
and a metal image.
She then put them in Micah’s
house.
\VS{5}Now this man
Micah
owned a shrine.
He made
an ephod
and some personal idols
and hired
one
of his sons
to serve as
a priest.
\VS{6}In those
days
Israel
had no
king.
Each man
did
what he considered to be right.
\par }{\SH Micah Hires a Professional
\par }{\PP \VS{7}There was
a young man
from Bethlehem
in Judah.
He was
a Levite
who had been temporarily
residing among the tribe
of Judah.
\VS{8}This man
left
the town
of Bethlehem
in Judah
to find
another place to live.
He came
to the Ephraimite
hill country
and made
his way
to
Micah’s
house.
\VS{9}Micah
said
to him, “Where
do you come
from?” He replied,
“I am
a Levite
from Bethlehem
in Judah.
I am
looking
for a new place to live.”
\VS{10}Micah
said
to him, “Stay
with me.
Become
my adviser
and priest.
I
will give
you ten
pieces of silver
per year,
plus
clothes and food.”
\VS{11}So the Levite
agreed
to stay
with
the man;
the young man
was like
a son to Micah.
\VS{12}Micah
paid
the Levite;
the young man
became his priest
and lived in Micah’s
house.
\VS{13}Micah
said,
“Now
I know
God
will make
me rich,
because
I have this
Levite
as my priest.”

\par }\Chap{18}{\PP \VerseOne{1}In those
days
Israel
had no
king.
And in those
days
the Danite
tribe
was looking for
a place to settle,
because
at that time
they
did not
yet
have a place to call their own
among
the tribes
of Israel.
\VS{2}The Danites
sent
out from their whole tribe
five
representatives,
capable
men
from Zorah
and Eshtaol,
to spy
out the
land
and explore
it. They said
to them,
“Go,
explore
the
land.”
They came
to the Ephraimite
hill country
and spent
the night at Micah’s
house.
\VS{3}As they
approached
Micah’s
house,
they
recognized
the accent
of the
young
Levite.
So they stopped
there
and said
to him, “Who
brought
you here? What
are you
doing
in this
place? What
is your business
here?”
\VS{4}He told
them
what
Micah
had done
for him, saying, “He hired
me and I became
his priest.”
\VS{5}They said
to him, “Seek
a divine
oracle
for us, so we can know
if we will be successful
on our mission.”
\VS{6}The priest
said
to them, “Go
with confidence. The
{\ND{Lord}}
will be
with you on
your mission.”
\par }{\PP \VS{7}So
the five
men
journeyed on and arrived
in Laish.
They noticed
that the
people
there were living
securely,
like the Sidonians
do, undisturbed
and unsuspecting.
No
conqueror
was troubling
them in any way. They lived far
from the Sidonians
and had
no
dealings
with
anyone.
\VS{8}When the Danites returned
to
their tribe
in Zorah
and Eshtaol,
their kinsmen
asked
them, “How did it go?”
\VS{9}They said,
“Come on,
let’s attack
them, for
we saw
their land
and it is very
good.
You
seem lethargic,
but don’t
hesitate
to invade
and conquer
the land.
\VS{10}When you invade,
you will encounter
unsuspecting
people.
The land
is wide! God
is handing
it over to you – a place that lacks nothing on earth!”
\par }{\PP \VS{11}So six
hundred
Danites,
fully armed,
set out
from Zorah
and Eshtaol.
\VS{12}They went up
and camped
in Kiriath Jearim
in Judah.
(To
this
day
that
place
is called
Camp of Dan.
It is west
of Kiriath Jearim.)
\VS{13}From there
they traveled
through the Ephraimite
hill country
and arrived
at Micah’s
house.
\VS{14}The five
men
who had gone
to spy out
the land
of Laish
said
to
their kinsmen, “Do you realize
that
inside these
houses
are
an ephod,
some personal idols,
a carved image,
and a metal image? Decide
now
what
you want to do.”
\VS{15}They stopped
there,
went
inside
the young
Levite’s
house
(which belonged to Micah), and asked
him how he was doing.
\VS{16}Meanwhile the six
hundred
Danites,
fully
armed,
stood
at the entrance
to the gate.
\VS{17}The five
men
who had gone
to spy out
the land
broke in
and stole
the carved image,
the ephod,
the personal
idols,
and the metal image,
while the priest
was standing
at the entrance
to the gate
with the six
hundred
fully
armed
men.
\VS{18}When these
men broke into
Micah’s
house
and stole
the carved image,
the ephod,
the personal idols,
and the metal image,
the priest
said
to
them, “What
are you
doing?”
\VS{19}They said
to him, “Shut up! Put
your hand
over
your mouth
and come
with
us! You can be
our adviser
and priest.
Wouldn’t it be better
to be
a priest
for a whole
Israelite
tribe
than for just
one
man’s
family?”
\VS{20}The priest
was happy.
He took
the ephod,
the personal idols,
and the carved image
and joined
the group.
\par }{\PP \VS{21}They turned
and went
on their way, but they walked
behind the children,
the cattle,
and their possessions.
\VS{22}After they
had gone a good distance
from Micah’s
house,
Micah’s
neighbors gathered together
and caught
up with the Danites.
\VS{23}When
they called
out
to
the Danites, the Danites
turned
around and said
to Micah,
“Why
have you gathered
together?”
\VS{24}He said,
“You stole
my gods
that
I made,
as well as this priest,
and then went
away. What
do I have left? How
can you have
the audacity
to say
to
me, ‘What do you want?’ ”
\VS{25}The Danites
said
to
him, “Don’t
say another word
to
us, or
some very angry
men
will attack
you, and you and your family
will die.”
\VS{26}The Danites
went
on their way;
when
Micah
realized
they were
too strong
to resist,
he turned
around and went
home.
\par }{\PP \VS{27}Now the
Danites took
what Micah
had
made,
as well as his priest,
and came
to Laish,
where the people
were undisturbed
and unsuspecting.
They struck
them down with the sword
and burned
the city.
\VS{28}No one
came to the rescue
because
the city was far
from Sidon
and they had no
dealings
with
anyone.
The city was
in a valley
near Beth Rehob.
The Danites rebuilt
the city
and occupied it.
\VS{29}They named
it Dan
after their ancestor,
who
was one of Israel’s
sons.
But
the city’s
name
used to be Laish.
\VS{30}The
Danites
worshiped
the carved image.
Jonathan,
descendant
of Gershom,
son
of Moses,
and his descendants
served
as priests
for the tribe
of Dan
until
the time
of the exile.
\VS{31}They worshiped
Micah’s
carved
image
the whole
time
God’s
authorized
shrine
was in Shiloh.

\par }\Chap{19}{\PP \VerseOne{1}In those
days
Israel
had no
king.
There was
a Levite
living temporarily
in the remote
region of the Ephraimite
hill country.
He acquired
a concubine
from Bethlehem
in Judah.
\VS{2}However, she
got angry
at him and went home
to
her father’s
house
in Bethlehem
in Judah.
When
she had been there
four
months,
\VS{3}her
husband
came
after
her, hoping
he could convince
her to return.
He brought with him
his servant
and a pair
of donkeys.
When she brought
him into her father’s
house
and the girl’s
father
saw
him,
he greeted
him warmly.
\VS{4}His father-in-law,
the girl’s
father,
persuaded
him to stay
with
him for three
days,
and they ate
and drank
together, and spent
the night there.
\VS{5}On
the fourth
day
they woke up early
and the Levite got
ready to leave.
But the girl’s
father
said
to
his son-in-law,
“Have
a bite
to eat
for some energy, then
you can
go.”
\VS{6}So the two
of them sat down
and had a meal
together.
Then the girl’s
father
said
to
the man,
“Why not
stay another night
and have a good
time!”
\VS{7}When the man
got
ready to leave,
his father-in-law
convinced
him to stay another night.
\VS{8}He woke up early
in the morning
on the fifth
day
so
he could
leave, but
the girl’s
father
said,
“Get some energy.
Wait
until
later
in the day
to leave!” So
they ate a meal together.
\VS{9}When the man
got
ready to leave
with his concubine
and his servant,
his father-in-law,
the girl’s
father,
said
to him, “Look! The day
is almost
over! Stay
another night! Since
the day
is over, stay
another night here
and have a good
time. You can
get up early
tomorrow
and start
your trip
home.”
\VS{10}But the man
did not
want
to stay
another night. He left
and traveled
as far as
Jebus
(that
is, Jerusalem). He had
with
him a pair
of saddled
donkeys
and his concubine.
\par }{\PP \VS{11}When they
got near
Jebus,
it was getting
quite
late
and the servant
said
to
his master,
“Come on, let’s
stop
at this
Jebusite
city
and spend the night in it.”
\VS{12}But his master
said
to him,
“We should not
stop
at
a foreign
city
where
non-Israelites
live. We will travel
on to
Gibeah.”
\VS{13}He said
to his servant, “Come
on, we will go
into one
of the other
towns and spend the night
in Gibeah
or
Ramah.”
\VS{14}So they traveled
on,
and the sun
went down
when they were near
Gibeah
in the territory of Benjamin.
\VS{15}They stopped
there
and decided
to spend
the night in Gibeah.
They came
into the city
and sat down
in the town square,
but no
one
invited
them to spend
the night.
\par }{\PP \VS{16}But then an old
man
passed by, returning
at the end of the day from
his work
in the field.
The man
was from the Ephraimite
hill country;
he
was living temporarily
in Gibeah.
(The residents
of the town
were Benjaminites.)
\VS{17}When he looked up
and saw
the traveler
in the town
square,
the old
man
said,
“Where
are you heading? Where
do you come from?”
\VS{18}The Levite said
to
him, “We
are traveling
from Bethlehem
in Judah
to the remote region
of the Ephraimite
hill country.
That’s where I’m
from. I had business in Bethlehem
in Judah,
but now I’m
heading
home.
But no
one
has invited
me into their home.
\VS{19}We have
enough
straw
and grain
for our donkeys,
and there is
enough
food
and wine
for me, your female servant,
and the young man
who is with
your servants.
We lack
nothing.”
\VS{20}The old
man
said,
“Everything
is just
fine! I will take care
of all
your needs.
But
don’t
spend
the night in the town square.”
\VS{21}So he brought
him to his house
and fed
the donkeys.
They washed
their feet
and had a meal.
\par }{\PP \VS{22}They
were having a good
time, when suddenly
some men
of the city,
some good-for-nothings,
surrounded
the
house
and kept beating
on
the door.
They said
to
the old
man
who owned
the house,
“Send out
the
man
who
came
to visit
you so we can have sex
with him.”
\VS{23}The man
who owned
the house
went
outside and said
to
them, “No,
my brothers! Don’t
do this wicked
thing! After
all, this
man
is a guest in my house.
Don’t
do
such a disgraceful thing!
\VS{24}Here
are my virgin
daughter
and my guest’s concubine.
I will send
them out
and you can abuse
them and do
to them whatever you like.
But don’t
do
such a disgraceful
thing
to this
man!”
\VS{25}The men
refused
to listen
to him, so the Levite
grabbed
his concubine
and made her go
outside.
They raped
her and abused
her all
night
long until
morning.
They let
her go
at dawn.
\VS{26}The woman
arrived
back at daybreak
and was sprawled
out on
the doorstep
of the house
where
her master
was staying
until
it became light.
\VS{27}When her master
got
up in the morning,
opened
the doors
of the house,
and went
outside to start on
his journey,
there
was the woman,
his concubine,
sprawled
out on
the doorstep
of the house
with her hands
on
the threshold.
\VS{28}He said
to her,
“Get
up, let’s
leave!” But there was no
response.
He put her
on
the donkey
and
went
home.
\VS{29}When he got
home,
he took
a knife,
grabbed
his concubine,
and carved
her up
into twelve
pieces.
Then he sent
the pieces throughout
Israel.
\VS{30}Everyone
who saw
the sight said,
“Nothing
like this has happened or
been witnessed
during
the entire
time
since
the Israelites
left the land
of Egypt! Take careful note of it! Discuss
it and speak!”

\par }\Chap{20}{\PP \VerseOne{1}All
the Israelites
from
Dan
to
Beer Sheba
and from the land
of Gilead
left
their homes and assembled
together
before
the {\ND{Lord}}
at Mizpah.
\VS{2}The leaders
of all
the people
from all
the tribes
of Israel
took their places in the assembly
of God’s
people,
which numbered four
hundred
thousand
sword-wielding
foot soldiers.
\VS{3}The Benjaminites
heard
that
the Israelites
had gone up
to Mizpah.
Then the Israelites
said, “Explain
how
this wicked
thing happened!”
\VS{4}The Levite,
the husband
of the murdered
woman,
spoke up,
“I
and my concubine
stopped
in Gibeah
in the territory of Benjamin
to spend the night.
\VS{5}The leaders
of Gibeah
attacked
me and at night
surrounded
the house
where I
was staying.
They wanted
to kill
me; instead they abused
my concubine
so badly
that she died.
\VS{6}I grabbed hold
of my concubine
and carved
her up and sent
the pieces throughout
the territory
occupied
by Israel,
because
they committed
such an unthinkable
atrocity
in Israel.
\VS{7}All
you Israelites,
make
a decision
here!”
\par }{\PP \VS{8}All
Israel
rose
up in unison
and said,
“Not
one of us will go
home! Not
one
of us will return
to his house!
\VS{9}Now
this
is what
we will do
to Gibeah: We will attack
the city as the lot dictates.
\VS{10}We will take
ten
of every group of a hundred
men
from all
the tribes
of Israel
(and a hundred
of every group of a thousand,
and a thousand
of every group of ten thousand) to get
supplies
for the army.
When they arrive
in Gibeah
of Benjamin
they will punish them for the atrocity
which
they committed
in Israel.”
\VS{11}So all
the men
of Israel
gathered together
at
the city
as allies.
\par }{\PP \VS{12}The tribes
of Israel
sent
men
throughout
the tribe
of Benjamin,
saying,
“How
could such a wicked
thing take place?
\VS{13}Now,
hand over
the good-for-nothings
in Gibeah
so we can execute
them and purge
Israel
of wickedness.”
But the Benjaminites
refused
to listen
to their Israelite
brothers.
\VS{14}The Benjaminites
came from
their cities
and assembled
at Gibeah
to make
war
against
the Israelites.
\VS{15}That day
the Benjaminites
mustered
from their cities
twenty-six
thousand
sword-wielding
soldiers, besides
seven
hundred
well-trained
soldiers from Gibeah.
\VS{16}Among this
army
were seven
hundred
specially-trained
left-handed
soldiers.
Each
one could sling
a stone
and hit even
the smallest
target.
\VS{17}The men
of Israel
(not counting
Benjamin) had mustered
four
hundred
thousand
sword-wielding
soldiers, every
one
an
experienced
warrior.
\par }{\PP \VS{18}The Israelites
went
up
to Bethel
and asked
God, “Who
should lead
the charge
against
the Benjaminites?” The
{\ND{Lord}}
said,
“Judah
should lead.”
\VS{19}The Israelites
got
up the next morning
and moved
against
Gibeah.
\VS{20}The men
of Israel
marched out
to fight
Benjamin;
they arranged
their
battle lines
against
Gibeah.
\VS{21}The Benjaminites
attacked
from
Gibeah
and struck down
twenty-two
thousand
Israelites
that
day.
\par }{\PP \VS{22}The Israelite
army
took heart
and once more
arranged
their battle
lines, in the same place
where
they had
taken their positions
the day
before.
\VS{23}The Israelites
went up
and wept
before
the {\ND{Lord}}
until
evening.
They asked
the {\ND{Lord}}, “Should we again
march out to fight
the Benjaminites,
our brothers?” The
{\ND{Lord}}
said,
“Attack
them!”
\VS{24}So the Israelites
marched
toward
the Benjaminites
the next
day.
\VS{25}The Benjaminites
again
attacked them
from
Gibeah
and struck down
eighteen
thousand
sword-wielding
Israelite
soldiers.
\par }{\PP \VS{26}So
all
the Israelites,
the whole
army,
went
up to Bethel.
They wept
and sat
there
before
the {\ND{Lord}}; they did not eat anything
that day
until
evening.
They offered
up burnt sacrifices
and tokens of peace
to the
{\ND{Lord}}.
\VS{27}The Israelites
asked
the {\ND{Lord}}
(for the ark
of God’s
covenant
was there
in those
days;
\VS{28}Phinehas
son
of Eleazar,
son
of Aaron,
was serving
the
{\ND{Lord}} in those
days), “Should we once more
march out
to fight
the Benjaminites
our brothers,
or
should we quit?” The
{\ND{Lord}}
said,
“Attack,
for
tomorrow
I will
hand them over to you.”
\par }{\PP \VS{29}So Israel
hid
men in ambush
outside
Gibeah.
\VS{30}The Israelites
attacked
the Benjaminites
the next day;
they took their positions
against Gibeah
just as
they had done before.
\VS{31}The Benjaminites
attacked
the army,
leaving
the city
unguarded.
They began
to strike
down their enemy
just as
they had done before.
On the main roads
(one
leads
to Bethel,
the other
to Gibeah) and in the field,
they struck down about thirty
Israelites.
\VS{32}Then the Benjaminites
said,
“They are defeated
just as before.”
But
the Israelites
said,
“Let’s retreat
and lure
them away from
the city
into
the main roads.”
\VS{33}All
the men
of Israel
got up
from their places
and took their positions
at Baal Tamar,
while the Israelites
hiding in ambush
jumped out
of their places
west of Gibeah.
\VS{34}Ten
thousand
men,
well-trained
soldiers from all
Israel,
then
made a frontal
assault
against Gibeah
– the battle
was fierce.
But the Benjaminites did not
realize
that
disaster
was at their doorstep.
\VS{35}The
{\ND{Lord}}
annihilated
Benjamin
before
Israel;
the Israelites
struck down
that day
25,100
sword-wielding
Benjaminites.
\VS{36}Then the Benjaminites
saw
they were defeated.
\par }{\PP The Israelites
retreated before Benjamin,
because
they had confidence
in the men they had
hid
in ambush
outside Gibeah.
\VS{37}The men hiding in ambush
made a mad
dash
to
Gibeah.
They
attacked
and put the sword
to the entire
city.
\VS{38}The Israelites
and the men
hiding in ambush
had arranged
a signal.
When the men hiding in ambush sent up
a smoke
signal
from
the city,
\VS{39}the Israelites
counterattacked.
Benjamin
had begun
to strike
down the Israelites;
they struck
down about thirty
men.
They said,
“There’s no doubt
about it! They are totally defeated
as in the earlier
battle.”
\VS{40}But when the signal,
a pillar
of smoke,
began
to rise up
from
the city,
the Benjaminites
turned
around
and saw
the whole
city
going up
in a cloud of smoke that rose high into the sky.
\VS{41}When the Israelites
turned around,
the Benjaminites
panicked
because
they could see
that
disaster
was on
their doorstep.
\VS{42}They retreated
before
the Israelites,
taking the road
to the wilderness.
But the battle
overtook
them as
men
from
the surrounding cities
struck
them down.
\VS{43}They surrounded
the Benjaminites,
chased
them from Nohah,
and annihilated
them all the way to
a spot east
of Geba.
\VS{44}Eighteen
thousand
Benjaminites,
all
of them capable warriors,
fell dead.
\VS{45}The rest turned
and ran
toward the wilderness,
heading toward
the cliff
of Rimmon.
But the Israelites caught
five
thousand
of them
on the main roads.
They stayed right on their heels
all the way
to Gidom
and struck down
two thousand more.
\VS{46}That day
twenty-five
thousand
sword-wielding
Benjaminites
fell
in battle,
all
of them capable
warriors.
\VS{47}Six
hundred
survivors
turned
and ran away
to the wilderness,
to
the cliff
of Rimmon.
They stayed
there four
months.
\VS{48}The Israelites
returned
to
the Benjaminite
towns
and put
the sword
to them. They wiped out
the cities, the animals,
and everything
they could find.
They set
fire
to every
city in their path.

\par }\Chap{21}{\PP \VerseOne{1}The Israelites
had taken an oath
in Mizpah,
saying,
“Not
one of us will allow
his daughter
to marry
a Benjaminite.”
\VS{2}So the people
came
to Bethel
and sat
there
before
God
until
evening,
weeping
loudly
and uncontrollably.
\VS{3}They said,
“Why,
O
{\ND{Lord}}
God
of Israel,
has this
happened in Israel?” An entire
tribe
has disappeared
from Israel
today!”
\par }{\PP \VS{4}The next
morning the people
got up early
and built
an altar
there.
They offered
up burnt sacrifices
and token of peace.
\VS{5}The Israelites
asked,
“Who
from all
the Israelite
tribes
has
not
assembled
before the
{\ND{Lord}}?” They had
made a solemn
oath that whoever
did not
assemble
before the
{\ND{Lord}}
at Mizpah
must
certainly be executed.
\VS{6}The Israelites
regretted
what had happened to their brother
Benjamin.
They said,
“Today
we cut off
an entire
tribe
from Israel!
\VS{7}How
can we find wives
for those who are left? After all, we
took an oath
in the
{\ND{Lord}}’s
name not
to give
them our daughters
as wives.”
\VS{8}So they asked, “Who
from all
the Israelite
tribes
did not
assemble
before the
{\ND{Lord}}
at Mizpah?” Now
it just so happened no
one
from Jabesh
Gilead
had come
to
the gathering.
\VS{9}When
they took roll
call, they noticed
none
of the inhabitants
of Jabesh
Gilead
were
there.
\VS{10}So
the assembly
sent
12,000
capable
warriors
against Jabesh Gilead. They commanded
them, “Go
and kill
with your swords
the inhabitants
of Jabesh
Gilead,
including
the women
and little children.
\VS{11}Do
this: exterminate
every
male,
as well as
every
woman
who has had sexual relations
with a male.
But spare the lives of any virgins.”
So they did as instructed.
\VS{12}They found
among the inhabitants
of Jabesh
Gilead
four
hundred
young girls
who were virgins
– they had
never
had sexual relations
with a male.
They brought
them back to
the camp
at Shiloh
in the land
of Canaan.
\par }{\PP \VS{13}The entire
assembly
sent
messengers
to
the Benjaminites
at the cliff
of Rimmon
and assured
them they would not be harmed.
\VS{14}The Benjaminites
returned
at that time,
and the Israelites gave
to them the women
they had
spared
from Jabesh
Gilead.
But there were
not
enough to go around.
\par }{\PP \VS{15}The people
regretted
what had happened to Benjamin
because
the {\ND{Lord}}
had weakened
the Israelite
tribes.
\VS{16}The leaders
of the assembly
said,
“How
can we find wives
for
those who are left? After all, the Benjaminite
women
have been wiped out.
\VS{17}The remnant
of Benjamin
must be preserved.
An entire Israelite
tribe
should not
be wiped out.
\VS{18}But we
can’t
allow
our daughters
to marry
them, for
the Israelites
took an oath,
saying,
‘Whoever gives
a woman
to a Benjaminite
will be destroyed!’
\VS{19}However, there is
an annual
festival
to the
{\ND{Lord}}
in Shiloh,
which
is north
of Bethel
(east
of the main road
that goes up
from Bethel
to Shechem) and south
of Lebonah.”
\VS{20}So they commanded
the Benjaminites,
“Go
hide
in the vineyards,
\VS{21}and keep your eyes open. When
you see
the daughters
of Shiloh
coming
out
to dance
in the celebration, jump out
from
the vineyards.
Each one
of you, catch
yourself a wife
from among the daughters
of Shiloh
and then go
home to the land
of Benjamin.
\VS{22}When
their fathers
or
brothers
come
and protest
to
us, we’ll say
to
them, “Do us a favor
and let them
be, for
we
could not
get
each
one a wife
through battle.
Don’t worry about breaking your oath! You
would only be guilty
if you had voluntarily
given
them wives.’ ”
\par }{\PP \VS{23}The Benjaminites
did
as instructed.
They abducted
two hundred of the dancing girls to be their wives.
They went
home
to
their own territory,
rebuilt
their cities,
and settled down.
\VS{24}Then
the Israelites
dispersed
from there
to their respective
tribal
and clan
territories.
Each
went
from there
to his own
property.
\VS{25}In those
days
Israel
had no
king.
Each man
did
what he considered to be right.
\par }