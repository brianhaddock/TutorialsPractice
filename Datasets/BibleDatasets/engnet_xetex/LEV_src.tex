\NormalFont\ShortTitle{Leviticus}
{\MT Leviticus

\par }\ChapOne{1}{\SH Introduction to the Sacrificial Regulations
\par }{\PP \VerseOne{1}Then the
{\ND{Lord}}
called
to
Moses
and spoke
to him
from the Meeting
Tent:
\VS{2}“Speak
to
the Israelites
and tell
them, ‘When someone
among you presents
an offering
to the
{\ND{Lord}}, you must present
your offering
from
the domesticated animals,
either from
the herd
or from
the flock.
\par }{\SH Burnt Offering Regulations: Animal from the Herd
\par }{\PP \VS{3}“‘If
his offering is a burnt offering
from
the herd
he must present
it as a flawless
male;
he must present
it at the entrance
of the Meeting
Tent
for its acceptance
before
the {\ND{Lord}}.
\VS{4}He must lay
his hand
on
the head
of the burnt offering,
and it will be accepted
for him to make atonement
on his behalf.
\VS{5}Then the one presenting the offering must slaughter
the
bull
before
the {\ND{Lord}},
and the sons
of Aaron,
the priests,
must present
the
blood
and splash
the blood
against
the sides
of the
altar
which
is at the entrance
of the Meeting
Tent.
\VS{6}Next, the one presenting
the offering must skin
the burnt offering
and cut
it into parts,
\VS{7}and the sons
of Aaron,
the priest,
must put
fire
on
the altar
and arrange
wood
on
the fire.
\VS{8}Then
the sons
of Aaron,
the priests,
must arrange
the
parts
with
the head
and the
suet
on
the wood
that
is in the fire
on
the altar.
\VS{9}Finally, the one presenting the offering must wash
its entrails
and its legs
in water
and the priest
must
offer
all
of it up in smoke on the altar –
it is a burnt offering,
a gift
of a soothing
aroma
to the
{\ND{Lord}}.
\par }{\SH Animal from the Flock
\par }{\PP \VS{10}“‘If
his offering is from
the flock
for a burnt offering –
from
the sheep
or
the goats
– he must present
a flawless
male,
\VS{11}and must slaughter
it on
the north
side
of the altar
before
the
{\ND{Lord}}, and the sons
of Aaron,
the priests,
will splash
its blood
against
the altar’s
sides.
\VS{12}Next, the one presenting
the offering must cut it into parts,
with
its head
and its suet,
and the priest
must arrange
them
on
the wood
which
is in the fire,
on
the altar.
\VS{13}Then the one presenting the offering must wash
the entrails
and the legs
in water,
and the priest
must present
all
of it and offer it up in smoke
on the altar
– it is
a burnt offering,
a gift
of a soothing
aroma
to the
{\ND{Lord}}.
\par }{\SH From the Birds
\par }{\PP \VS{14}“‘If
his offering
to the
{\ND{Lord}}
is a burnt offering
from
the birds,
he must present
his offering
from
the turtledoves
or
from
the young
pigeons.
\VS{15}The priest
must present
it at the altar,
pinch
off its head
and offer the head up in smoke
on the altar,
and its blood
must be drained
out against
the side
of the altar.
\VS{16}Then the priest must remove
its entrails
by cutting off its tail feathers,
and throw
them to the east
side
of the altar
into
the place
of fatty ashes,
\VS{17}and tear
it open by its wings
without
dividing
it into two parts. Finally, the priest
must offer it up in smoke
on
the altar
on
the wood
which
is in the fire
– it is a burnt offering,
a gift
of a soothing
aroma
to the
{\ND{Lord}}.

\par }\Chap{2}{\PP \VerseOne{1}“‘When
a person
presents
a grain offering
to the
{\ND{Lord}}, his offering must consist of choice wheat flour,
and he must pour
olive oil
on
it and put
frankincense on it.
\VS{2}Then he must bring
it to
the sons
of Aaron,
the priests, and the priest
must scoop
out from there
a handful
of its choice wheat flour
and some of its olive oil
in addition
to all
of its frankincense,
and the priest
must offer
its memorial
portion up in smoke on the altar
– it is a gift
of a soothing
aroma
to the
{\ND{Lord}}.
\VS{3}The remainder
of the grain offering
belongs to Aaron
and to his sons –
it is most
holy
from the gifts
of the {\ND{Lord}}.
\par }{\SH Processed Grain Offerings
\par }{\PP \VS{4}“‘When
you present
an offering
of grain
baked
in an oven,
it must be made of choice wheat flour
baked into unleavened
loaves
mixed
with olive oil
or unleavened
wafers
smeared
with olive oil.
\VS{5}If
your offering is a grain offering
made on
the griddle,
it must be choice wheat flour
mixed
with olive oil,
unleavened.
\VS{6}Crumble
it in pieces
and pour
olive oil
on it – it is a grain offering.
\VS{7}If
your offering
is a grain offering
made in a pan,
it must be made of choice wheat flour
deep fried
in olive oil.
\par }{\PP \VS{8}“‘You must bring
the grain offering
that
must be made
from these
to the
{\ND{Lord}}. Present
it to
the priest,
and he will bring
it to
the altar.
\VS{9}Then
the priest
must take up from
the grain offering
its memorial
portion and offer it up in smoke
on the altar
– it is a gift
of a soothing
aroma
to the
{\ND{Lord}}.
\VS{10}The remainder
of the grain offering
belongs to Aaron
and to his sons
– it is most
holy
from the gifts
of the
{\ND{Lord}}.
\par }{\SH Additional Grain Offering Regulations
\par }{\PP \VS{11}“‘No grain offering
which
you present
to the
{\ND{Lord}}
can be made
with yeast,
for
you must not
offer up in smoke
any
yeast
or
honey
as a gift
to the
{\ND{Lord}}.
\VS{12}You can present
them to the
{\ND{Lord}}
as an offering
of first fruit,
but they must not
go up
to
the altar
for a soothing
aroma.
\VS{13}Moreover,
you must season every one
of your
grain offerings
with salt;
you must not
allow
the salt
of the covenant
of your God
to be missing from your grain offering –
on
every one
of your
grain offerings
you must present
salt.
\par }{\PP \VS{14}“‘If
you present
a grain offering
of first ripe grain
to the
{\ND{Lord}}, you must present
your grain offering
of first ripe grain
as soft kernels
roasted
in fire
– crushed bits
of fresh grain.
\VS{15}And you
must put
olive oil
on it and set
frankincense
on it – it is a grain offering.
\VS{16}Then the priest
must offer its memorial portion up in smoke – some of its crushed bits, some of its olive oil, in addition to all of its frankincense – it is a gift to the
{\ND{Lord}}.

\par }\Chap{3}{\PP \VerseOne{1}“‘Now if
his offering is a peace offering
sacrifice,
if
he presents
an offering from
the herd,
he must
present
before
the {\ND{Lord}}
a flawless
male
or
a female.
\VS{2}He must lay
his hand
on
the head
of his offering
and slaughter
it at the entrance
of the Meeting
Tent,
and the sons
of Aaron,
the priests,
must splash
the blood
against
the altar’s sides.
\VS{3}Then the one presenting the offering must present
a gift
to the
{\ND{Lord}}
from the peace offering
sacrifice: He must remove the fat
that covers
the
entrails
and all
the fat
that
surrounds the entrails,
\VS{4}the two
kidneys
with the fat
on
their sinews,
and the protruding lobe
on
the liver
(which he is to remove
along with the kidneys).
\VS{5}Then the sons
of Aaron
must offer it up in smoke
on the altar
atop
the burnt offering
that
is on
the wood
in the fire
as a gift
of a soothing
aroma
to the
{\ND{Lord}}.
\par }{\SH Animal from the Flock
\par }{\PP \VS{6}“‘If
his offering
for a peace offering
sacrifice
to the
{\ND{Lord}}
is from
the flock,
he must present
a flawless
male
or
female.
\VS{7}If
he presents
a sheep
as his offering,
he must present
it before
the {\ND{Lord}}.
\VS{8}He must lay
his hand
on
the head
of his offering
and slaughter
it before
the Meeting
Tent,
and the sons
of Aaron
must splash
its blood
against
the altar’s sides.
\VS{9}Then he must present
a gift
to the
{\ND{Lord}}
from the peace offering
sacrifice: He must
remove
all the fatty tail
up to the
end of the spine,
the fat
covering
the
entrails,
and all
the fat
on
the entrails,
\VS{10}the two
kidneys
with the fat
on
their sinews,
and the protruding lobe
on
the liver
(which he is to remove
along with the kidneys).
\VS{11}Then the priest
must offer it up in smoke
on the altar
as a food
gift
to the
{\ND{Lord}}.
\par }{\PP \VS{12}“‘If
his offering
is a goat
he must present
it before
the {\ND{Lord}},
\VS{13}lay
his hand
on
its
head,
and slaughter
it before
the Meeting
Tent,
and the sons
of Aaron
must splash
its blood
against
the altar’s sides.
\VS{14}Then he must present
from
it his offering
as a gift
to the
{\ND{Lord}}: the fat
which
covers
the entrails
and all
the fat
on
the entrails,
\VS{15}the two
kidneys
with the fat
on
their sinews,
and the protruding lobe
on
the liver
(which he is to remove
along with the kidneys).
\VS{16}Then the priest
must offer them up in smoke
on the altar
as a food
gift
for a soothing
aroma
– all
the fat
belongs to the
{\ND{Lord}}.
\VS{17}This is a perpetual
statute
throughout your generations
in all
the places
where you live: You must never
eat
any
fat
or any
blood.’ ”

\par }\Chap{4}{\PP \VerseOne{1}Then the
{\ND{Lord}}
spoke
to
Moses:
\VS{2}“Tell
the Israelites,
‘When
a person
sins
by straying
unintentionally
from any
of the Lord’s
commandments
which
must not
be violated,
and violates
any
one of them –
\par }{\SH For the Priest
\par }{\PP \VS{3}“‘If
the high
priest
sins
so that the people
are guilty,
on
account of the sin
he has committed
he must present
a flawless
young
bull
to the
{\ND{Lord}}
for a sin offering.
\VS{4}He must bring
the bull
to
the entrance
of the Meeting
Tent
before
the {\ND{Lord}}, lay
his hand
on
the head
of the bull,
and slaughter
the
bull
before
the {\ND{Lord}}.
\VS{5}Then that high
priest
must take
some of the blood
of the bull
and bring
it to
the Meeting
Tent.
\VS{6}The
priest
must dip
his finger
in the blood
and sprinkle
some
of it seven
times
before
the {\ND{Lord}}
toward the front
of the veil-canopy
of the sanctuary.
\VS{7}The priest
must put
some
of the blood
on
the horns
of the altar
of fragrant
incense
that is before
the {\ND{Lord}}
in the Meeting
Tent,
and all
the rest of the bull’s
blood
he must pour
out at the base
of the altar
of burnt offering
that
is at the entrance
of the Meeting
Tent.
\par }{\PP \VS{8}“‘Then he must take up
all
the fat
from
the sin offering
bull: the fat
covering
the entrails
and all
the fat
surrounding the entrails,
\VS{9}the two
kidneys
with the fat
on
their sinews,
and the protruding lobe
on
the liver
(which he is to remove
along with the kidneys)
\VS{10}– just as it is taken from the ox of the peace offering sacrifice – and the priest must offer them up in smoke on the altar of burnt offering.
\VS{11}But the hide
of the bull,
all
its flesh
along
with its head
and its legs,
its entrails,
and its dung –
\VS{12}all
the rest of the bull –
he must bring outside
the camp
to
a ceremonially clean
place,
to
the fatty ash
pile,
and he must burn
it on
a wood
fire;
it must be burned
on
the fatty ash
pile.
\par }{\SH For the Whole Congregation
\par }{\PP \VS{13}“‘If
the whole
congregation
of Israel
strays
unintentionally
and the matter
is not
noticed
by the assembly,
and they violate
one
of the Lord’s
commandments, which
must not
be violated,
so they become guilty,
\VS{14}the assembly
must present
a young
bull
for a sin offering
when
the sin
they have committed
becomes known.
They must bring
it before
the Meeting
Tent,
\VS{15}the elders
of the congregation
must lay
their hands
on
the head
of the bull
before
the {\ND{Lord}}, and someone must slaughter
the
bull
before
the {\ND{Lord}}.
\VS{16}Then the high
priest
must bring
some of the blood
of the bull
to
the Meeting
Tent,
\VS{17}and that priest
must dip
his finger
in the blood
and sprinkle
some of the blood seven
times
before
the {\ND{Lord}}
toward the front
of the veil-canopy.
\VS{18}He
must put
some
of the blood
on
the horns
of the altar
which
is before
the {\ND{Lord}}
in the Meeting
Tent,
and all
the rest of the blood
he must pour
out at the base
of the altar
of burnt offering
that
is at the entrance
of the Meeting
Tent.
\par }{\PP \VS{19}“‘Then
the priest must take all
its fat
and offer
the fat
up in smoke
on the altar.
\VS{20}He must do
with the rest of the bull
just
as he did
with the bull
of the sin offering;
this is what
he must do
with it. So
the priest
will make
atonement
on
their behalf and they
will be forgiven.
\VS{21}He must bring
the rest of the bull
outside
the camp
and burn
it just
as he burned
the
first
bull
– it is
the sin offering
of the assembly.
\par }{\SH For the Leader
\par }{\PP \VS{22}“‘Whenever
a leader,
by straying unintentionally, sins
and violates
one
of the commandments
of the {\ND{Lord}}
his God
which
must not
be violated,
and he pleads
guilty,
\VS{23}or
his sin
that
he committed
is made known
to him,
he must bring
a flawless
male
goat
as his offering.
\VS{24}He must lay
his hand
on
the head
of the male goat
and slaughter
it in the
place
where
the burnt offering
is slaughtered
before
the {\ND{Lord}} –
it is
a sin offering.
\VS{25}Then the priest
must take
some of the blood
of the sin offering
with his finger
and put
it on
the horns
of the altar
of burnt offering,
and he must pour
out the rest of its blood
at the base
of the altar
of burnt offering.
\VS{26}Then the priest must offer all
of its fat
up in smoke
on the altar
like the fat
of the peace offering
sacrifice.
So the priest
will make atonement
on
his behalf for his sin
and he will be forgiven.
\par }{\SH For the Common Person
\par }{\PP \VS{27}“‘If
an ordinary individual
sins
by straying unintentionally
when he violates
one
of the Lord’s
commandments
which
must not
be violated,
and he pleads guilty
\VS{28}or
his sin
that
he committed
is made known
to him, he must bring
a flawless
female
goat
as his offering
for
the sin
that
he committed.
\VS{29}He must lay
his hand
on
the head
of the sin offering
and slaughter
the sin offering
in the place
where the burnt offering is slaughtered.
\VS{30}Then the priest
must take
some of its blood
with his finger
and put
it on
the horns
of the altar
of burnt offering,
and he must
pour
out all
the rest of its blood
at the base
of the altar.
\VS{31}Then he must remove
all
of its fat
(just
as fat
was removed
from the peace offering
sacrifice) and the priest
must offer it up in smoke
on the altar
for a soothing
aroma
to the
{\ND{Lord}}. So the priest
will make atonement
on
his behalf and he will be forgiven.
\par }{\PP \VS{32}“‘But if
he brings
a sheep
as his offering,
for a sin offering,
he must bring
a flawless
female.
\VS{33}He must lay
his hand
on
the head
of the sin offering
and slaughter
it for a sin offering
in the place
where
the
burnt offering
is slaughtered.
\VS{34}Then the priest
must take
some of the blood
of the sin offering
with his finger
and put
it on
the horns
of the altar
of burnt offering,
and he must
pour
out all
the rest of its blood
at the base
of the altar.
\VS{35}Then the one who brought the offering must remove
all
its fat
(just
as the fat
of the sheep
is removed
from the peace offering
sacrifice) and the priest
must offer them
up in smoke
on the altar
on
top of the other gifts
of the {\ND{Lord}}. So the priest
will make atonement
on
his behalf for his sin
which
he has committed
and he will be forgiven.

\par }\Chap{5}{\PP \VerseOne{1}“‘When
a person
sins
in that he hears
a public
curse
against one who fails to testify and he is a witness
(he either
saw
or
knew
what had happened ) and he does
not
make it known, then he will bear
his punishment for iniquity.
\VS{2}Or
when there is a person
who touches
anything
ceremonially unclean,
whether
the carcass
of an unclean
wild animal,
or
the carcass
of an unclean
domesticated animal,
or
the carcass
of an unclean
creeping thing,
even if he did not realize
it, but he himself has become unclean
and is guilty;
\VS{3}or
when
he touches
human
uncleanness
with regard to anything by which
he can become unclean,
even if he did not realize
it, but he himself has later come to know
it and is guilty;
\VS{4}or
when
a person
swears
an oath, speaking thoughtlessly
with his lips,
whether to do evil
or
to do good,
with regard to anything
which
the individual
might speak thoughtlessly
in an oath,
even if he did not realize
it, but he himself has later come to know
it and is guilty
with regard to one
of these oaths –
\VS{5}when
an individual becomes
guilty
with regard to one
of these
things he must confess
how
he has sinned,
\VS{6}and he must bring
his penalty for guilt
to the
{\ND{Lord}}
for
his sin
that
he has committed,
a female
from
the flock,
whether a female sheep
or
a female goat,
for a sin offering.
So the priest
will make atonement
on
his behalf for his sin.
\par }{\PP \VS{7}“‘If
he cannot
afford
an animal from the flock,
he must bring
his penalty
for guilt for his sin that
he has committed,
two
turtledoves
or
two
young
pigeons,
to the
{\ND{Lord}}, one
for a sin offering
and one
for a burnt offering.
\VS{8}He must bring
them to
the priest
and present
first
the one that
is for a sin offering.
The priest must pinch
its head
at the nape of its neck,
but must not
sever the head from the body.
\VS{9}Then he must sprinkle
some of the blood
of the sin offering
on
the wall
of the altar,
and the remainder
of the blood
must be squeezed
out at the base
of the altar
– it is
a sin offering.
\VS{10}The
second
bird he must make
a burnt offering
according to the standard regulation.
So the priest
will make atonement
on
behalf of this person for his sin
which
he has committed,
and he will be forgiven.
\par }{\PP \VS{11}“‘If
he cannot
afford
two
turtledoves
or
two
young
pigeons,
he must bring
as his offering
for his sin which
he has committed
a tenth
of an ephah
of choice wheat flour
for a sin offering.
He must not
place
olive oil
on
it and he must not
put
frankincense
on
it, because
it is a sin offering.
\VS{12}He must bring
it to
the priest
and the priest
must scoop
out from
it a handful
as its memorial
portion and offer it up in smoke
on the altar
on
top of the other gifts
of the {\ND{Lord}} –
it
is a sin offering.
\VS{13}So the priest
will make atonement
on
his behalf for his sin
which
he has committed
by doing one
of these
things, and he will be forgiven.
The remainder of the offering will belong to the priest
like the grain offering.’ ”
\par }{\SH Guilt Offering Regulations: Known Trespass
\par }{\PP \VS{14}Then the
{\ND{Lord}}
spoke
to
Moses:
\VS{15}“When a person
commits
a trespass
and sins
by straying unintentionally
from the regulations about the Lord’s
holy
things, then he must bring
his penalty for guilt
to the
{\ND{Lord}}, a flawless
ram
from
the flock,
convertible
into silver
shekels
according to the standard of the sanctuary
shekel,
for a guilt offering.
\VS{16}And whatever
holy
thing
he violated
he must restore
and must add
one fifth
to it and give
it to the
priest.
So the priest
will make atonement
on
his behalf with the guilt offering
ram
and he will be forgiven.”
\par }{\SH Unknown trespass
\par }{\PP \VS{17}“If
a person
sins
and violates
any
of the Lord’s
commandments
which
must not
be violated
(although he did not
know
it at the time, but later realizes he is guilty), then he will bear
his punishment for iniquity
\VS{18}and must bring
a flawless
ram
from
the flock,
convertible into silver shekels,
for a guilt offering
to
the priest.
So the priest
will make atonement
on
his behalf for his error
which
he committed (although
he himself had not
known
it) and he will be forgiven.
\VS{19}It is a guilt offering;
he
was surely guilty
before the
{\ND{Lord}}.”

\par }\Chap{6}{\PP \VerseOne{1} Then the
{\ND{Lord}}
spoke
to
Moses:
\VS{2}“When a person
sins
and commits
a trespass
against the
{\ND{Lord}}
by deceiving
his fellow citizen
in regard to something held in
trust, or
a pledge,
or
something stolen,
or
by extorting
something from his fellow citizen,
\VS{3}or
has found
something lost
and denies
it and swears
falsely
concerning any
one
of the things
that
someone
might do
to sin –
\VS{4}when it happens
that
he sins
and he is found guilty,
then he must return
whatever
he had stolen,
or
whatever
he had extorted,
or
the thing
that
he had held in trust,
or
the lost thing
that
he had found,
\VS{5}or
anything
about which
he swears
falsely.
He must restore
it in full
and add
one fifth
to it; he must give
it to its owner when
he is found guilty.
\VS{6}Then he must bring
his guilt offering
to the
{\ND{Lord}}, a flawless
ram
from
the flock,
convertible into silver shekels,
for a guilt offering
to
the priest.
\VS{7}So the priest
will make atonement
on
his behalf before
the {\ND{Lord}}
and he will be forgiven
for whatever
he has done
to become guilty.”
\par }{\SH Sacrificial Instructions for the Priests: The Burnt Offering
\par }{\PP \VS{8} Then the
{\ND{Lord}}
spoke
to
Moses:
\VS{9}“Command
Aaron
and his sons,
‘This
is the law
of the burnt offering.
The burnt offering
is
to remain on
the hearth
on
the altar
all
night
until
morning,
and the fire
of the altar
must be kept burning on it.
\VS{10}Then the priest
must put on
his
linen
robe and must put
linen
leggings
over
his bare flesh,
and he must take up
the
fatty ashes
of the burnt offering
that
the fire
consumed
on
the altar,
and he must place
them beside
the altar.
\VS{11}Then he must take off
his clothes
and put on
other
clothes,
and he must bring
the fatty ashes
outside
the camp
to
a ceremonially clean
place,
\VS{12}but the fire
which is on
the altar
must be kept burning
on it. It must not
be extinguished.
So the priest
must kindle
wood
on
it morning
by morning,
and he must arrange
the burnt offering
on
it and offer
the
fat
of the
peace offering up in smoke on it.
\VS{13}A continual
fire
must be kept burning
on
the altar.
It must not
be extinguished.
\par }{\SH The Grain Offering of the Common Person
\par }{\PP \VS{14}“‘This
is the law
of the grain offering.
The
sons
of Aaron
are to present it before
the {\ND{Lord}}
in front
of the altar,
\VS{15}and the priest must take up
with his hand
some of the choice wheat flour
of the
grain offering
and some of its olive oil,
and all
of the frankincense
that
is on
the grain offering,
and he must offer its memorial
portion up in smoke
on the altar
as a soothing
aroma
to the
{\ND{Lord}}.
\VS{16}Aaron
and his sons
are to eat
what is left over
from
it. It must be eaten
unleavened
in a holy
place;
they are to eat
it in the courtyard
of the Meeting
Tent.
\VS{17}It must not
be baked
with yeast.
I have given
it as their portion
from my gifts.
It is most
holy,
like the sin offering
and the guilt offering.
\VS{18}Every
male
among the sons
of Aaron
may eat
it. It is a perpetual
allotted portion
throughout your generations
from the gifts
of the {\ND{Lord}}. Anyone
who touches
these gifts must be holy.’ ”
\par }{\SH The Grain Offering of the Priests
\par }{\PP \VS{19}Then the
{\ND{Lord}}
spoke
to
Moses:
\VS{20}“This
is the offering
of Aaron
and his sons
which
they must present
to the
{\ND{Lord}}
on the day
when he is anointed: a tenth
of an ephah
of choice wheat flour
as a continual
grain offering,
half
of it in the morning
and half
of it in the evening.
\VS{21}It must be made with olive oil
on a griddle
and you must bring
it well soaked,
so you must present
a grain offering
of broken pieces
as a soothing
aroma
to the
{\ND{Lord}}.
\VS{22}The
high
priest
who succeeds him
from among his sons
must do
it. It is a perpetual
statute;
it must be offered up in smoke
as a whole offering
to the
{\ND{Lord}}.
\VS{23}Every
grain offering
of a priest
must be a whole offering;
it must not
be eaten.”
\par }{\SH The Sin Offering
\par }{\PP \VS{24}Then the
{\ND{Lord}}
spoke
to
Moses:
\VS{25}“Tell
Aaron
and his sons,
‘This
is the law
of the sin offering.
In the place
where
the burnt offering
is slaughtered
the sin offering
must be slaughtered
before
the {\ND{Lord}}. It is
most
holy.
\VS{26}The priest
who offers it for sin
is to eat
it. It must be eaten
in a holy
place,
in the court
of the Meeting
Tent.
\VS{27}Anyone
who touches
its meat
must be holy,
and whoever
spatters
some of its blood
on
a garment,
you must wash
whatever
he spatters
it on
in a holy
place.
\VS{28}Any clay vessel
it is boiled
in must be broken,
and if
it was boiled
in a bronze
vessel,
then that vessel must be rubbed
out and rinsed
in water.
\VS{29}Any
male
among the priests
may eat
it. It is
most
holy.
\VS{30}But any
sin offering
from which
some of its blood
is brought
into
the Meeting
Tent
to make atonement
in the sanctuary
must not
be eaten.
It must be burned up
in the fire.

\par }\Chap{7}{\PP \VerseOne{1}“‘This
is the law
of the guilt offering.
It is
most
holy.
\VS{2}In the place
where
they slaughter
the burnt offering
they must slaughter
the guilt offering,
and the officiating priest must
splash
the blood
against
the altar’s
sides.
\VS{3}Then
the one making the offering must present
all
its fat: the fatty tail,
the fat
covering
the entrails,
\VS{4}the two
kidneys
and the fat
on
their sinews,
and the protruding lobe
on
the liver
(which he must remove
along with the kidneys).
\VS{5}Then the priest
must offer them up in smoke
on the altar
as a gift
to the
{\ND{Lord}}. It is
a guilt offering.
\VS{6}Any
male
among the priests
may eat
it. It must be eaten
in a holy
place.
It is
most holy.
\VS{7}The law
is the same
for the sin offering
and the guilt offering;
it belongs to the priest
who
makes atonement
with it.
\par }{\SH Priestly Portions of Burnt and Grain Offerings
\par }{\PP \VS{8}“‘As for the priest
who presents
someone’s
burnt offering,
the hide
of that burnt offering
which
he presented
belongs to him.
\VS{9}Every
grain offering
which
is baked
in the oven
or made
in the pan
or on
the griddle
belongs
to the priest
who presented it.
\VS{10}Every
grain offering,
whether mixed
with olive oil
or dry,
belongs to all
the sons
of Aaron,
each one alike.
\par }{\SH The Peace Offering
\par }{\PP \VS{11}“‘This
is the law
of the peace offering
sacrifice
which
he is to present
to the
{\ND{Lord}}.
\VS{12}If
he presents
it on account of thanksgiving,
along with the thank offering sacrifice he must present unleavened loaves mixed with olive oil, unleavened wafers smeared with olive oil, and well soaked ring-shaped loaves made of choice wheat flour mixed with olive oil.
\VS{13}He must present
this grain offering
in addition
to ring-shaped loaves
of leavened
bread
which regularly accompany the sacrifice
of his thanksgiving
peace offering.
\VS{14}He must present
one
of each kind
of grain offering
as a contribution
offering to the
{\ND{Lord}}; it belongs to the priest
who splashes
the
blood
of the peace offering.
\VS{15}The meat
of his thanksgiving
peace offering
must be eaten
on the day
of his offering; he must not
set
any of it aside
until
morning.
\par }{\PP \VS{16}“‘If
his offering
is a votive
or
freewill
sacrifice,
it may be eaten
on the day
he presents
his sacrifice,
and also the leftovers
from
it may be eaten
on the next day,
\VS{17}but the leftovers
from the meat
of the sacrifice
must be burned up
in the fire
on
the third
day.
\VS{18}If
some of the meat
of his peace offering
sacrifice
is ever eaten
on the third
day
it will not
be accepted;
it will
not
be
accounted
to the one who presented
it, since it is spoiled,
and the person
who eats
from
it will bear
his punishment for iniquity.
\VS{19}The meat
which
touches
anything
ceremonially unclean
must not
be eaten;
it must be burned up
in the fire.
As for ceremonially clean meat,
everyone
who is ceremonially clean
may eat
the meat.
\VS{20}The person
who
eats
meat
from the peace offering
sacrifice
which
belongs to the
{\ND{Lord}}
while his uncleanness
persists will be cut off
from his people.
\VS{21}When a person
touches
anything
unclean
(whether human
uncleanness,
or
an unclean
animal,
or
an unclean
detestable
creature) and eats
some of the meat
of the peace offering
sacrifice
which
belongs to the
{\ND{Lord}}, that person
will be cut off
from his people.’ ”
\par }{\SH Sacrificial Instructions for the Common People: Fat and Blood
\par }{\PP \VS{22}Then the
{\ND{Lord}}
spoke
to
Moses:
\VS{23}“Tell
the Israelites,
‘You must not
eat
any
fat
of an ox,
sheep,
or goat.
\VS{24}Moreover, the fat
of an animal that has died of natural causes
and the fat
of an animal torn
by beasts may be used
for any other
purpose,
but you must certainly
never
eat it.
\VS{25}If
anyone
eats
fat
from
the animal
from which
he presents
a gift
to the
{\ND{Lord}}, that person will be cut off
from his people.
\VS{26}And you must not
eat
any
blood
of the birds
or the domesticated
land animals
in any
of the places where you live.
\VS{27}Any
person
who
eats
any
blood
– that person will be cut off
from his people.’ ”
\par }{\SH Priestly Portions of Peace Offerings
\par }{\PP \VS{28}Then the
{\ND{Lord}}
spoke
to
Moses:
\VS{29}“Tell
the Israelites,
‘The one who presents
his peace offering
sacrifice
to the
{\ND{Lord}}
must bring
his offering
to the
{\ND{Lord}}
from his peace offering sacrifice.
\VS{30}With his own hands
he must bring
the
{\ND{Lord}}’s
gifts.
He must bring
the fat
with the breast
to wave
the breast
as a wave offering
before
the {\ND{Lord}},
\VS{31}and the priest
must
offer
the fat
up in smoke on the altar,
but
the breast
will belong to Aaron
and his sons.
\VS{32}The right
thigh
you must give
as a contribution
offering to the priest
from your peace offering
sacrifices.
\VS{33}The one from Aaron’s
sons
who presents
the blood
of the peace offering
and fat
will
have the right
thigh
as his share,
\VS{34}for
the
breast
of the wave offering
and the thigh
of the contribution
offering I have taken
from the Israelites
out of their peace offering
sacrifices
and have given
them to Aaron
the priest
and to his sons
from the people of
Israel
as a perpetual
allotted portion.’ ”
\par }{\PP \VS{35}This
is the allotment
of Aaron
and the allotment
of his sons
from the
{\ND{Lord}}’s
gifts
on the day
Moses presented
them
to serve
as priests to the
{\ND{Lord}}.
\VS{36}This
is what
the {\ND{Lord}}
commanded
to give
to them from the Israelites
on the day
Moses anointed them – a perpetual allotted portion throughout their generations.
\par }{\SH Summary of Sacrificial Regulations in Leviticus 6:8-7:36
\par }{\PP \VS{37}This
is the law
for the burnt offering,
the grain offering,
the sin offering,
the guilt offering,
the ordination
offering, and the peace offering
sacrifice,
\VS{38}which
the {\ND{Lord}}
commanded
Moses
on Mount
Sinai
on the day
he commanded
the
Israelites
to present
their offerings
to the
{\ND{Lord}}
in the wilderness
of Sinai.

\par }\Chap{8}{\PP \VerseOne{1}Then the
{\ND{Lord}}
spoke
to
Moses:
\VS{2}“Take
Aaron
and his sons
with
him, and the garments,
the anointing
oil,
the
sin offering
bull,
the two
rams,
and the basket
of unleavened bread,
\VS{3}and assemble
the whole
congregation
at the entrance
of the Meeting
Tent.”
\VS{4}So Moses
did
just
as the
{\ND{Lord}}
commanded
him, and the congregation
assembled
at the entrance
of the Meeting
Tent.
\VS{5}Then Moses
said to
the congregation: “This
is what
the {\ND{Lord}}
has
commanded
to be done.”
\par }{\SH Clothing Aaron
\par }{\PP \VS{6}So
Moses
brought
Aaron
and his sons
forward and washed
them
with water.
\VS{7}Then he put
the tunic
on
Aaron, wrapped
the sash
around him, and clothed
him with
the robe.
Next he put
the ephod
on
him and placed on him
the decorated band
of the ephod, and fastened
the ephod closely to him with the band.
\VS{8}He then set
the breastpiece
on
him
and put
the Urim
and Thummim
into
the breastpiece.
\VS{9}Finally, he set
the
turban
on
his head
and attached
the gold
plate,
the holy
diadem,
to
the front
of the turban
just
as the
{\ND{Lord}}
had commanded
Moses.
\par }{\SH Anointing the Tabernacle and Aaron, and Clothing Aaron’s Sons
\par }{\PP \VS{10}Then
Moses
took
the anointing
oil
and anointed
the tabernacle
and everything
in it, and so consecrated them.
\VS{11}Next he sprinkled
some
of it on
the altar
seven
times
and so anointed
the altar,
all
its vessels,
and the wash basin
and its stand
to consecrate them.
\VS{12}He then poured
some of the anointing
oil
on
the head
of Aaron
and anointed
him to consecrate him.
\VS{13}Moses
also brought forward
Aaron’s
sons,
clothed
them with tunics,
wrapped
sashes
around them, and wrapped
headbands
on them just as
the {\ND{Lord}}
had commanded
Moses.
\par }{\SH Consecration Offerings
\par }{\PP \VS{14}Then he brought near
the sin offering
bull
and Aaron
and his
sons
laid their hands
on
the head
of the sin offering
bull,
\VS{15}and he slaughtered
it. Moses
then took
the
blood
and put
it all around
on
the horns
of the altar
with his finger
and decontaminated
the
altar,
and he poured
out the rest of the
blood
at the base
of the altar
and so consecrated
it to make atonement
on it.
\VS{16}Then he took
all
the fat
on
the entrails,
the
protruding lobe
of the liver,
and the
two
kidneys
and their fat,
and Moses
offered it all up in smoke
on the altar,
\VS{17}but the rest of the bull
– its hide,
its flesh,
and its dung
– he completely burned up
outside
the camp
just
as the
{\ND{Lord}}
had commanded
Moses.
\par }{\PP \VS{18}Then he presented
the burnt offering
ram
and Aaron
and his sons
laid
their hands
on
the head
of the ram,
\VS{19}and he slaughtered
it. Moses
then splashed
the blood
against
the altar’s
sides.
\VS{20}Then
he cut
the ram
into parts,
and Moses
offered
the head,
the parts,
and the suet up in smoke,
\VS{21}but the entrails
and the
legs
he washed
with water,
and Moses
offered
the
whole
ram
up in smoke on the altar
– it was
a burnt offering
for a soothing
aroma,
a gift
to the
{\ND{Lord}}, just
as the
{\ND{Lord}}
had commanded
Moses.
\par }{\PP \VS{22}Then he presented
the second
ram,
the ram
of ordination,
and Aaron
and his
sons
laid their hands
on
the head
of the ram
\VS{23}and he slaughtered
it. Moses
then took
some of its blood
and put
it on
Aaron’s
right
earlobe,
on
the thumb
of his right
hand,
and on
the big toe
of his right
foot.
\VS{24}Next he brought
Aaron’s
sons
forward, and Moses
put
some
of the blood
on
their right
earlobes,
on
their right
thumbs,
and on
the big toes
of their right
feet,
and Moses
splashed
the rest of the blood
against
the altar’s
sides.
\par }{\PP \VS{25}Then he took
the
fat
(the
fatty tail,
all
the fat
on
the entrails,
the
protruding lobe
of the liver,
and the
two
kidneys
and their fat
) and the
right
thigh,
\VS{26}and from the basket
of unleavened
bread that
was before
the {\ND{Lord}}
he took
one unleavened
loaf,
one
loaf
of bread
mixed with olive oil,
and one
wafer,
and placed
them on
the fat
parts and on
the right
thigh.
\VS{27}He then put
all
of them on
the palms
of Aaron
and his sons,
who waved
them as a wave offering
before
the {\ND{Lord}}.
\VS{28}Moses
then took
them from their palms
and offered them up in smoke
on the altar
on
top of the burnt offering
– they were
an ordination
offering for a soothing
aroma;
it was
a gift
to the
{\ND{Lord}}.
\VS{29}Finally, Moses
took
the breast
and waved
it as a wave offering
before
the {\ND{Lord}}
from the ram
of ordination.
It was Moses’
share
just
as the
{\ND{Lord}}
had commanded
Moses.
\par }{\SH Anointing Aaron, his Sons, and their Garments
\par }{\PP \VS{30}Then Moses
took
some of the anointing
oil
and some
of the blood
which
was on
the altar
and sprinkled
it on
Aaron
and his garments,
and on
his sons
and his sons’
garments
with
him. So he consecrated
Aaron,
his garments,
and his sons
and his sons’
garments
with him.
\VS{31}Then Moses
said to
Aaron
and his sons,
“Boil
the
meat
at the entrance
of the Meeting
Tent,
and there
you are to eat
it and the
bread
which
is in the ordination
offering basket,
just
as I have commanded,
saying,
‘Aaron
and his sons
are to eat it,’
\VS{32}but the remainder
of the meat
and the bread
you must burn
with fire.
\VS{33}And you must not
go out
from the entrance
of the Meeting
Tent
for seven
days,
until
the day
when
your days
of ordination
are completed,
because
you must be ordained
over a seven-day
period.
\VS{34}What has been
done
on
this
day
the {\ND{Lord}}
has commanded
to be done
to make atonement for you.
\VS{35}You must reside
at the entrance
of the
Meeting
Tent
day
and night
for seven
days
and keep
the charge
of the {\ND{Lord}}
so that you will not
die,
for
this
is what I have been commanded.”
\VS{36}So Aaron
and his sons
did
all
the things
the {\ND{Lord}}
had
commanded
through
Moses.

\par }\Chap{9}{\PP \VerseOne{1}On
the eighth
day
Moses
summoned
Aaron
and his sons
and the elders
of Israel,
\VS{2}and said
to
Aaron,
“Take
for yourself a bull
calf
for a sin offering
and a ram
for a burnt offering,
both flawless,
and present
them before
the {\ND{Lord}}.
\VS{3}Then tell
the Israelites: ‘Take
a male
goat
for a sin offering
and a calf
and lamb,
both a year
old and flawless,
for a burnt offering,
\VS{4}and an ox
and a ram
for peace offerings
to sacrifice
before
the {\ND{Lord}}, and a grain offering
mixed
with olive oil,
for
today
the {\ND{Lord}}
is going to
appear
to you.’ ”
\VS{5}So they took
what
Moses
had commanded
to
the front
of the Meeting
Tent
and the whole
congregation
presented
them and stood
before
the {\ND{Lord}}.
\VS{6}Then Moses
said,
“This
is what
the {\ND{Lord}}
has
commanded
you to do
so that the glory
of the {\ND{Lord}}
may appear
to you.”
\VS{7}Moses
then said
to
Aaron,
“Approach
the altar
and make
your sin offering
and your burnt offering,
and make atonement
on behalf
of yourself and on behalf
of the people;
and also make
the
people’s
offering
and make atonement
on behalf
of them just
as the
{\ND{Lord}}
has commanded.”
\par }{\SH The Sin Offering for the Priests
\par }{\PP \VS{8}So Aaron
approached
the altar
and slaughtered
the sin offering
calf
which was for himself.
\VS{9}Then Aaron’s
sons
presented
the
blood
to him
and he dipped
his finger
in the blood
and put
it on
the horns
of the altar,
and the rest of the blood
he poured
out at the base
of the altar.
\VS{10}The fat
and the
kidneys
and the
protruding lobe
of
the liver
from
the sin offering
he offered up in smoke
on the altar
just
as the
{\ND{Lord}}
had commanded
Moses,
\VS{11}but the flesh
and the hide
he completely burned up
outside
the camp.
\par }{\SH The Burnt Offering for the Priests
\par }{\PP \VS{12}He then slaughtered
the burnt offering,
and his
sons
handed
the blood
to
him and he splashed
it against
the altar’s
sides.
\VS{13}The burnt offering
itself they handed
to
him by its parts,
including the head,
and he offered them up in smoke
on
the altar,
\VS{14}and he washed
the entrails
and the legs
and offered them up in smoke
on
top of the burnt offering
on the altar.
\par }{\SH The Offerings for the People
\par }{\PP \VS{15}Then he presented
the people’s
offering.
He took
the sin offering
male goat
which
was for the people,
slaughtered
it, and performed a decontamination
rite with it like the first one.
\VS{16}He then presented
the burnt offering,
and did
it according to the standard regulation.
\VS{17}Next he presented
the grain offering,
filled
his hand
with some
of it, and offered it up in smoke
on
the altar
in addition
to the morning
burnt offering.
\VS{18}Then he slaughtered
the
ox
and the
ram
– the peace offering
sacrifices
which
were for the people
– and Aaron’s
sons
handed
the blood
to him
and he splashed
it against
the altar’s
sides.
\VS{19}As for the
fat
parts from
the ox
and from
the ram
(the fatty tail,
the fat covering
the entrails, the kidneys,
and the protruding lobe
of the liver),
\VS{20}they set
those
on
the breasts
and he offered the fat
parts up in smoke
on the altar.
\VS{21}Finally Aaron
waved
the breasts
and the right
thigh
as a wave offering
before
the
{\ND{Lord}}
just
as Moses
had commanded.
\par }{\PP \VS{22}Then Aaron
lifted
up his hands
toward
the people
and blessed
them and descended
from making
the sin offering,
the burnt offering,
and the peace offering.
\VS{23}Moses
and Aaron
then entered
into
the Meeting
Tent.
When they came out,
they blessed
the
people,
and the glory
of the {\ND{Lord}}
appeared
to
all
the people.
\VS{24}Then fire
went out
from the presence
of the {\ND{Lord}}
and consumed
the
burnt offering
and the
fat
parts on
the altar,
and all
the people
saw
it, so they shouted
loudly and fell
down with their faces to the ground.

\par }\Chap{10}{\PP \VerseOne{1}Then Aaron’s
sons,
Nadab
and Abihu,
each
took
his fire pan
and put
fire
in it, set
incense
on
it, and presented
strange
fire
before
the {\ND{Lord}}, which
he had not
commanded them to do.
\VS{2}So fire
went out
from the presence
of the {\ND{Lord}}
and consumed
them
so that they died
before
the {\ND{Lord}}.
\VS{3}Moses
then said
to
Aaron,
“This is
what
the {\ND{Lord}}
spoke: ‘Among the ones close
to me I will show myself holy,
and in the presence
of all
the people
I will be honored.’ ”
So Aaron
kept silent.
\VS{4}Moses
then called
to
Mishael
and Elzaphan,
the sons
of Uzziel,
Aaron’s
uncle,
and said
to
them, “Come near,
carry
your brothers
away from the front
of the sanctuary
to
a place outside
the camp.”
\VS{5}So they came near
and carried
them away in their tunics
to
a place outside
the camp
just
as Moses
had spoken.
\VS{6}Then Moses
said
to
Aaron
and to Eleazar
and Ithamar
his other two sons,
“Do not
dishevel
the hair of your heads
and do not
tear your garments,
so that you do not
die
and so that wrath does not come on
the whole
congregation.
Your brothers,
all
the house
of Israel,
are to mourn
the
burning
which
the {\ND{Lord}} has caused,
\VS{7}but you must not
go out
from the entrance
of the Meeting
Tent
lest
you die,
for
the Lord’s
anointing
oil
is on
you.” So
they acted
according to the word
of Moses.
\par }{\SH Perpetual Statutes the Lord Spoke to Aaron
\par }{\PP \VS{8}Then the
{\ND{Lord}}
spoke
to
Aaron,
\VS{9}“Do not
drink
wine
or strong drink,
you
and your sons
with
you, when you enter
into
the Meeting
Tent,
so that you do not
die,
which is a perpetual
statute
throughout your generations,
\VS{10}as well as to distinguish
between
the holy
and the common,
and between
the unclean
and the clean,
\VS{11}and to teach
the Israelites
all
the statutes
that
the {\ND{Lord}}
has spoken
to
them through
Moses.”
\par }{\SH Perpetual Statutes Moses spoke to Aaron
\par }{\PP \VS{12}Then
Moses
spoke to
Aaron
and to
Eleazar
and Ithamar,
his remaining
sons,
“Take
the
grain offering
which remains
from the gifts
of the {\ND{Lord}}
and eat
it unleavened
beside
the altar,
for
it is
most
holy.
\VS{13}You must eat
it in a holy
place
because
it is
your allotted
portion and the allotted portion
of your sons
from the gifts
of the {\ND{Lord}}, for
this
is what I have been commanded.
\VS{14}Also, the breast
of the wave offering
and the thigh
of the contribution
offering you must eat
in a ceremonially clean
place,
you
and your sons
and daughters
with
you, for
they have been given as your allotted portion
and the allotted portion
of your sons
from the peace offering
sacrifices
of the Israelites.
\VS{15}The thigh
of the contribution
offering and the breast
of the wave offering
they must bring
in addition
to the gifts
of the fat
parts to wave
them as a wave offering
before
the {\ND{Lord}}, and it will belong
to you and your sons
with
you for a perpetual
statute
just
as the
{\ND{Lord}}
has commanded.”
\par }{\SH The Problem with the Inaugural Sin Offering
\par }{\PP \VS{16}Later Moses
sought diligently
for the sin offering
male goat,
but it had actually
been burnt. So he became angry
at Eleazar
and Ithamar,
Aaron’s
remaining
sons,
saying,
\VS{17}“Why
did you not
eat
the
sin offering
in the sanctuary? For
it is most
holy
and he gave
it to you to bear
the
iniquity
of the congregation,
to make atonement
on
their behalf before
the {\ND{Lord}}.
\VS{18}See here! Its blood
was not
brought
into
the holy place
within! You should certainly have eaten
it in the sanctuary
just as I commanded!”
\VS{19}But
Aaron
spoke to
Moses,
“See here! Just today
they presented
their sin offering
and their burnt offering
before
the {\ND{Lord}}
and such
things as these
have happened to me! If I had eaten
a sin offering
today
would the
{\ND{Lord}}
have been pleased?”
\VS{20}When Moses
heard
this explanation,
he was satisfied.

\par }\Chap{11}{\PP \VerseOne{1}The
{\ND{Lord}}
spoke
to
Moses
and Aaron,
saying
to them,
\VS{2}“Tell
the Israelites: ‘This
is the kind of creature
you may eat
from among all
the animals
that
are on
the land.
\VS{3}You may eat any
among the animals
that has a divided
hoof
(the hooves
are completely split in two
) and that also chews
the cud.
\VS{4}However,
you must not
eat
these
from among those that
chew
the cud
and have divided
hooves: The camel
is unclean
to you because
it chews
the cud
even though
its hoof
is not
divided.
\VS{5}The rock badger
is unclean
to you because
it chews
the cud
even though
its hoof
is not
divided.
\VS{6}The hare
is unclean
to you because
it chews
the cud
even though
its hoof
is not
divided.
\VS{7}The pig
is unclean
to you because
its hoof
is divided
(the hoof
is completely split in two
), even though it does not
chew
the cud.
\VS{8}You must not
eat
from their meat
and you must not
touch
their carcasses;
they are
unclean to you.
\par }{\SH Clean and Unclean Water Creatures
\par }{\PP \VS{9}“‘These
you can eat
from all
creatures that
are in the water: Any
creatures
in the water
that have both fins
and scales,
whether in the seas
or in the streams,
you may eat.
\VS{10}But any
creatures that do not
have both fins
and scales,
whether in the seas
or in the streams,
from all
the swarming things
of the water
and from all
the living
creatures
that
are in the water,
are detestable to you.
\VS{11}Since they are detestable
to you, you must not
eat
their meat
and their carcass
you must detest.
\VS{12}Any
creature in the water
that
does not
have both fins
and scales
is detestable to you.
\par }{\SH Clean and Unclean Birds
\par }{\PP \VS{13}“‘These
you are to detest
from among
the birds
– they must not
be eaten,
because they
are detestable: the griffon vulture,
the bearded vulture,
the black vulture,
\VS{14}the kite,
the buzzard
of any kind,
\VS{15}every
kind
of crow,
\VS{16}the
eagle owl,
the
short-eared owl,
the
long-eared owl,
the
hawk
of any kind,
\VS{17}the little owl,
the cormorant,
the screech owl,
\VS{18}the white owl,
the scops owl,
the osprey,
\VS{19}the stork,
the heron
of any kind,
the hoopoe,
and the bat.
\par }{\SH Clean and Unclean Insects
\par }{\PP \VS{20}“‘Every
winged
swarming thing
that walks
on
all fours
is detestable to you.
\VS{21}However,
this
you may eat
from all
the winged
swarming things
that walk
on
all fours,
which
have jointed
legs
to hop
with
on
the land.
\VS{22}These
you may eat
from them: the locust
of any kind,
the bald locust
of any kind,
the cricket
of any kind,
the grasshopper
of any kind.
\VS{23}But any
other winged
swarming thing
that
has four
legs
is detestable to you.
\par }{\SH Carcass Uncleanness
\par }{\PP \VS{24}“‘By these
you defile
yourselves; anyone
who touches
their carcass
will be unclean
until
the evening,
\VS{25}and anyone
who carries
their carcass
must wash
his clothes
and will be unclean
until
the evening.
\par }{\SH Inedible Land Quadrupeds
\par }{\PP \VS{26}“‘All
animals
that
divide
the hoof
but it is not
completely split in two
and do not
chew
the cud
are unclean
to you; anyone
who touches
them becomes unclean.
\VS{27}All
that walk
on
their paws
among all
the creatures
that walk
on
all
fours
are unclean
to you. Anyone
who touches
their carcass
will be unclean
until
the evening,
\VS{28}and the one who carries
their carcass
must wash
his clothes
and be unclean
until
the evening;
they
are unclean to you.
\par }{\SH Creatures that Swarm on the Land
\par }{\PP \VS{29}“‘Now this
is what is unclean
to you among the swarming things
that swarm
on
the land: the rat,
the mouse,
the large lizard
of any kind,
\VS{30}the Mediterranean gecko,
the spotted lizard,
the wall gecko,
the skink, and the chameleon.
\VS{31}These
are the ones that are unclean
to you among all
the swarming things.
Anyone
who touches
them when they die
will be unclean
until
evening.
\VS{32}Also, anything
they fall
on
when they
die
will become unclean
– any
wood
vessel
or
garment
or
article of leather
or
sackcloth.
Any
such vessel
with which
work
is done
must be immersed
in water
and will be unclean
until
the evening.
Then it will become clean.
\VS{33}As for any
clay
vessel
they
fall
into,
everything
in it
will become unclean
and you must break it.
\VS{34}Any
food
that
may be eaten
which
becomes soaked
with water
will become unclean.
Anything
drinkable
in any
such vessel
will become unclean.
\VS{35}Anything
their carcass
may
fall
on
will become unclean.
An oven
or small stove
must be smashed
to pieces; they
are unclean,
and they will stay
unclean to you.
\VS{36}However,
a spring
or a cistern
which collects
water
will be
clean,
but one who touches
their carcass
will be unclean.
\VS{37}Now, if
such a carcass
falls
on
any
sowing
seed which
is to be sown,
it is
clean,
\VS{38}but if
water
is put
on
the seed
and such a carcass
falls
on
it, it is
unclean to you.
\par }{\SH Edible Land Quadrupeds
\par }{\PP \VS{39}“‘Now if
an animal
that
you may eat dies,
whoever touches
its
carcass
will be unclean
until
the evening.
\VS{40}One who eats
from its carcass
must wash
his clothes
and be unclean
until
the evening,
and whoever carries
its carcass
must wash
his clothes
and be unclean
until
the evening.
\VS{41}Every
swarming thing
that swarms
on
the land
is detestable;
it must not
be eaten.
\VS{42}You must not
eat
anything
that crawls
on
its belly
or anything
that walks
on
all fours
or on any
number
of legs
of all
the swarming things
that swarm
on
the land,
because
they
are detestable.
\VS{43}Do not
make yourselves
detestable
by any
of the swarming things.
You must not
defile
yourselves by them and become unclean by them,
\VS{44}for
I am
the {\ND{Lord}}
your God
and you are to sanctify
yourselves and be
holy
because
I am
holy.
You must not
defile
yourselves
by any
of the swarming things
that creep
on
the ground,
\VS{45}for
I am
the {\ND{Lord}}
who brought you up
from the land
of Egypt
to be
your God,
and you are to be
holy
because
I am
holy.
\VS{46}This
is the law
of the land animals,
the birds,
all
the living
creatures
that move
in the water,
and all
the creatures
that swarm
on
the land,
\VS{47}to distinguish
between
the unclean
and the clean,
between
the living
creatures that may be eaten
and the living
creatures that
must not
be eaten.’ ”

\par }\Chap{12}{\PP \VerseOne{1}The
{\ND{Lord}}
spoke
to
Moses:
\VS{2}“Tell
the Israelites,
‘When
a woman
produces offspring
and bears
a male
child, she
will be unclean
seven
days,
as she is
unclean
during
the days
of her menstruation.
\VS{3}On
the eighth
day
the flesh
of his foreskin
must be circumcised.
\VS{4}Then she will remain thirty-three
days
in
blood
purity.
She must not
touch
anything
holy
and she must not
enter
the sanctuary
until
the days
of her purification
are fulfilled.
\VS{5}If
she bears a female
child, she
will be impure
fourteen
days
as during her menstrual
flow, and she will remain sixty-six
days
in
blood
purity.
\par }{\PP \VS{6}“‘When the days
of her purification
are completed
for a son
or
for a daughter,
she must bring
a one year
old lamb
for a burnt offering
and a young
pigeon
or
turtledove
for a sin offering
to
the entrance
of the Meeting
Tent,
to
the priest.
\VS{7}The priest is to present
it before
the
{\ND{Lord}}
and make atonement
on
her behalf, and she will be clean
from her flow
of blood.
This
is the law
of the one who bears
a child, for the male
or
the female child.
\VS{8}If
she cannot
afford
a sheep,
then she must take
two
turtledoves
or
two
young
pigeons,
one
for a burnt offering
and one
for a sin offering,
and the priest
is to make atonement
on
her behalf, and she will be clean.’ ”

\par }\Chap{13}{\PP \VerseOne{1}The
{\ND{Lord}}
spoke
to
Moses
and Aaron:
\VS{2}“When
someone
has a swelling
or
a scab
or
a bright spot
on the skin
of his body
that
may become
a diseased
infection,
he must be brought
to
Aaron
the priest
or
one
of his sons,
the priests.
\VS{3}The priest
must then examine
the infection
on the skin
of the body,
and if the hair
in the infection
has turned
white
and the infection
appears
to be deeper
than the skin
of the body,
then it is a diseased
infection,
so when
the priest
examines
it he must pronounce the person unclean.
\par }{\SH A Bright Spot on the Skin
\par }{\PP \VS{4}“If
it is
a white
bright spot
on the skin
of his body,
but it does not
appear to be deeper
than
the skin,
and the hair
has not
turned
white,
then the priest
is
to quarantine
the
person with the infection
for seven
days.
\VS{5}The priest
must then examine
it on
the seventh
day,
and if, as far as
he can see,
the infection
has stayed
the same and has not
spread
on the skin,
then the priest
is to quarantine
the person for another
seven
days.
\VS{6}The
priest
must then examine
it again
on
the seventh
day,
and if the infection
has faded
and has not
spread
on the skin,
then the priest
is to pronounce the person clean.
It is a scab,
so he must wash
his clothes
and be clean.
\VS{7}If,
however,
the scab
is spreading
further
on the skin
after
he has shown
himself to
the priest
for his purification,
then he must show
himself to the priest
a second time.
\VS{8}The priest
must then examine
it, and if the scab
has spread
on the skin,
then the priest
is to pronounce the person unclean.
It is
a disease.
\par }{\SH A Swelling on the Skin
\par }{\PP \VS{9}“When
someone
has a diseased
infection,
he must be
brought
to
the priest.
\VS{10}The priest
will then examine
it, and if a white
swelling
is on the skin,
it
has turned
the hair
white,
and there is raw
flesh
in the swelling,
\VS{11}it is a chronic
disease
on the skin
of his body,
so the priest is to pronounce him unclean.
The priest
must not
merely quarantine
him, for
he is
unclean.
\VS{12}If,
however,
the disease
breaks
out on the skin
so that the disease
covers
all
the skin
of the person with the infection
from his head
to his feet,
as far
as the priest
can
see,
\VS{13}the priest
must then examine
it, and if the disease
covers
his whole
body,
he is to pronounce the person with the infection
clean.
He has turned
all
white,
so he is clean.
\VS{14}But
whenever raw
flesh
appears
in it he will be unclean,
\VS{15}so
the priest
is to examine
the raw
flesh
and pronounce
him unclean –
it is
diseased.
\VS{16}If, however,
the raw
flesh
once again
turns
white,
then he must come
to
the priest.
\VS{17}The priest
will then examine
it, and if the infection
has turned
white,
the priest
is to pronounce the person with the infection
clean –
he is
clean.
\par }{\SH A Boil on the Skin
\par }{\PP \VS{18}“When
someone’s body
has
a boil
on its skin
and it heals,
\VS{19}and in the place
of the boil
there is a white
swelling
or
a reddish
white
bright spot,
he must show
himself to
the priest.
\VS{20}The priest
will then examine
it, and if it appears
to be deeper
than
the skin
and its hair
has turned
white,
then the priest
is to pronounce the person unclean.
It is a diseased
infection
that has broken
out in the boil.
\VS{21}If,
however, the priest
examines
it, and there
is no
white
hair
in it, it is not
deeper
than
the skin,
and it has faded,
then the priest
is to quarantine
him for seven
days.
\VS{22}If
it is spreading
further
on the skin,
then the priest
is to pronounce him unclean.
It is
an infection.
\VS{23}But if
the bright spot
stays
in its place
and has not
spread,
it is the scar
of the boil,
so the priest
is to pronounce him clean.
\par }{\SH A Burn on the Skin
\par }{\PP \VS{24}“When
a body
has
a burn
on its skin
and the raw area
of the burn
becomes
a reddish
white
or
white
bright spot,
\VS{25}the priest
must examine
it, and if the hair
has turned
white
in the bright spot
and it appears
to be deeper
than
the skin,
it is a disease
that
has broken out
in the burn.
The priest
is to pronounce
the person unclean.
It is
a diseased
infection.
\VS{26}If,
however, the priest
examines
it and there
is no
white
hair
in the bright spot,
it is not
deeper
than
the skin,
and it has faded,
then the priest
is to quarantine
him for seven
days.
\VS{27}The priest
must then examine
it on
the seventh
day,
and if
it is spreading
further
on the skin,
then the priest
is to pronounce him unclean.
It is
a diseased
infection.
\VS{28}But if
the bright spot
stays
in its place, has not
spread
on the skin,
and it has faded,
then it is the swelling
of the burn,
so the priest
is to pronounce him clean,
because
it is the scar
of the burn.
\par }{\SH Scall on the Head or in the Beard
\par }{\PP \VS{29}“When a man
or
a woman
has an infection
on the head
or
in the beard,
\VS{30}the
priest
is to examine
the infection,
and if it appears
to be deeper
than
the skin
and the hair
in it is reddish yellow
and thin,
then the priest
is to pronounce the person unclean.
It is
scall,
a disease
of the head
or
the beard.
\VS{31}But
if
the priest
examines
the scall
infection
and it does not
appear
to be deeper
than
the skin,
and there is no
black
hair
in it, then
the priest
is to quarantine
the person with the scall
infection
for seven
days.
\VS{32}The priest
must then examine
the infection
on
the seventh
day,
and if the scall
has not
spread,
there is no
reddish yellow
hair
in it,
and the scall
does not
appear to be deeper
than
the skin,
\VS{33}then the
individual is to shave
himself, but he must not
shave
the
area affected by the scall,
and the priest
is to quarantine
the person with the scall
for another
seven
days.
\VS{34}The priest
must then examine
the scall
on
the seventh
day,
and if the scall
has not
spread
on the skin
and it does not
appear
to be deeper
than
the skin,
then the
priest
is to pronounce him clean.
So he is to wash
his clothes
and be clean.
\VS{35}If,
however,
the scall
spreads
further
on the skin
after
his purification,
\VS{36}then
the priest
is
to examine
it, and if the scall
has spread
on the skin
the priest
is not
to search
further for reddish yellow
hair.
The person is unclean.
\VS{37}If,
as far as the priest can see,
the scall
has stayed
the same and black
hair
has sprouted
in it, the scall
has been healed;
the person is clean.
So the priest
is to pronounce him clean.
\par }{\SH Bright White Spots on the Skin
\par }{\PP \VS{38}“When a man
or
a woman
has
bright spots
– white
bright spots
– on the skin
of their body,
\VS{39}the priest
is
to examine
them, and if the bright spots
on the skin
of their body
are faded
white,
it is a harmless
rash that has broken out
on the skin.
The person is clean.
\par }{\SH Baldness on the Head
\par }{\PP \VS{40}“When
a man’s
head
is bare so that
he is balding
in back, he is clean.
\VS{41}If
his head
is bare on the forehead so that he is balding
in front,
he is
clean.
\VS{42}But if
there
is a reddish
white
infection
in the back
or
front bald
area, it is a disease
breaking
out in his back
or
front bald area.
\VS{43}The
priest
is to examine
it, and if the swelling
of the infection
is reddish
white
in the back
or
front bald
area like the appearance
of a disease
on the skin
of the body,
\VS{44}he is a diseased
man.
He is unclean.
The priest
must surely pronounce him unclean because
of his infection
on his head.
\par }{\SH The Life of the Person with Skin Disease
\par }{\PP \VS{45}“As for the diseased person
who
has the infection,
his clothes
must be
torn,
the hair of his head
must be
unbound,
he must cover
his mustache,
and he must call
out ‘Unclean! Unclean!’
\VS{46}The whole
time
he has the infection
he will be continually unclean.
He
must live
in isolation,
and his place of residence
must be outside
the camp.
\par }{\SH Infections in Garments, Cloth, or Leather
\par }{\PP \VS{47}“When
a garment
has a diseased
infection
in it, whether a wool
or
linen
garment,
\VS{48}or
in the warp
or
woof
of the linen
or the wool,
or
in leather
or
anything
made
of leather,
\VS{49}if the infection
in the garment
or
leather
or
warp
or
woof
or
any
article
of leather
is yellowish green
or
reddish,
it is a diseased
infection
and it must be shown
to the priest.
\VS{50}The priest
is to examine
and then quarantine
the article with the infection
for seven
days.
\VS{51}He must then examine
the infection
on the seventh
day.
If
the infection
has spread
in the garment,
or
in the warp,
or
in the woof,
or
in the leather
– whatever
the article into which
the leather
was made –
the infection
is a malignant
disease.
It is
unclean.
\VS{52}He must burn
the garment
or
the warp
or
the woof,
whether wool
or
linen,
or
any
article
of leather
which
has the infection
in it. Because
it is a malignant
disease
it
must be burned up
in the fire.
\VS{53}But if
the priest
examines
it and
the infection
has not
spread
in the garment
or
in the warp
or
in the woof
or
in any
article
of leather,
\VS{54}the priest
is to command
that they wash
whatever
has the infection
and quarantine
it for another
seven
days.
\VS{55}The priest
must then examine
it after
the
infection
has been washed out,
and if
the infection
has not
changed
its appearance
even though the infection
has not
spread,
it is unclean.
You must burn
it up
in the fire.
It is
a fungus,
whether on the back side
or
front side of the article.
\VS{56}But if
the priest
has examined
it and
the infection
has faded
after
it has been washed,
he is to tear
it out of
the garment
or
the leather
or
the warp
or
the woof.
\VS{57}Then if
it still appears
again
in the garment
or
the warp
or
the woof,
or
in any
article
of leather,
it is an outbreak.
Whatever has the infection
in it you must burn up
in the fire.
\VS{58}But the garment
or
the warp
or
the woof
or
any
article
of leather
which
you wash
and infection
disappears
from it is to be washed
a second
time and it will be clean.”
\par }{\SH Summary of Infection Regulations
\par }{\PP \VS{59}This
is the law
of the diseased
infection
in the garment
of wool
or
linen,
or
the warp
or
woof,
or
any
article
of leather,
for pronouncing it clean
or
unclean.

\par }\Chap{14}{\PP \VerseOne{1}The
{\ND{Lord}}
spoke
to
Moses:
\VS{2}“This
is the law
of the diseased
person on the day
of his purification,
when he is brought
to
the priest.
\VS{3}The priest
is to
go
outside
the camp
and examine
the infection.
If the infection
of the diseased
person
has
been healed,
\VS{4}then the priest
will command
that two
live
clean
birds,
a piece of cedar
wood,
a scrap of crimson
fabric,
and some twigs of hyssop
be taken up
for the one being cleansed.
\VS{5}The priest
will then command
that one
bird
be slaughtered
into a clay vessel
over fresh
water.
\VS{6}Then he is to take
the live
bird
along with the piece of cedar
wood,
the scrap of crimson
fabric,
and the twigs of hyssop,
and he is to dip
them and the live
bird
in the blood
of the bird
slaughtered
over
the fresh
water,
\VS{7}and sprinkle
it seven
times
on
the one being cleansed
from
the disease,
pronounce him clean,
and send
the live
bird
away over
the open
countryside.
\par }{\SH The Seven Days of Purification
\par }{\PP \VS{8}“The one being cleansed
must then wash
his clothes,
shave
off all
his hair,
and bathe
in water,
and so be clean.
Then afterward
he may enter
the camp,
but he must live
outside
his tent
seven
days.
\VS{9}When
the seventh
day
comes he must shave
all
his hair
– his head,
his beard,
his eyebrows,
all
his hair
– and he
must wash
his clothes,
bathe
his body
in water,
and so be clean.
\par }{\SH The Eighth Day Atonement Rituals
\par }{\PP \VS{10}“On the eighth
day
he must take
two
flawless
male lambs,
one
flawless
yearling
female
lamb,
three-tenths of an ephah
of choice wheat flour
as a grain offering
mixed
with olive oil,
and one
log
of olive oil,
\VS{11}and the priest
who pronounces
him clean
will have the
man
who is being cleansed
stand along with
these offerings before
the
{\ND{Lord}}
at the entrance
of the Meeting
Tent.
\par }{\PP \VS{12}“The
priest
is to take
one
male lamb
and present
it for a guilt offering
along
with the log
of olive oil
and present them
as a wave offering
before
the {\ND{Lord}}.
\VS{13}He must then slaughter
the
male lamb
in the place
where
the sin offering
and the
burnt offering
are slaughtered,
in the sanctuary,
because,
like the sin offering,
the guilt offering
belongs to the priest;
it is
most
holy.
\VS{14}Then the priest
is to take
some of the blood
of the guilt offering
and put
it on
the right
earlobe
of the one being cleansed,
on
the thumb
of his right
hand,
and on
the big toe
of his right
foot.
\VS{15}The priest
will then take
some of the log
of olive oil
and pour
it into
his
own left
hand.
\VS{16}Then the priest
is to dip
his right
forefinger
into
the olive oil
that
is in his left
hand,
and sprinkle
some of
the olive oil
with his finger
seven
times
before
the {\ND{Lord}}.
\VS{17}The priest
will then put
some of the rest
of the olive oil
that
is in
his hand
on
the right
earlobe
of the one being cleansed,
on
the thumb
of his right
hand,
and on
the big toe
of his right
foot,
on
the blood
of the guilt offering,
\VS{18}and the remainder
of the olive oil
that
is in his hand
the priest
is to put
on
the head
of the one being cleansed.
So the priest
is to make atonement
for him before
the {\ND{Lord}}.
\par }{\PP \VS{19}“The priest
must then perform
the sin offering
and make atonement
for the one being cleansed
from his impurity.
After
that he is to slaughter
the burnt offering,
\VS{20}and the priest
is to offer
the burnt offering
and the
grain offering
on
the altar.
So the priest
is to make atonement
for him and he will be clean.
\par }{\SH The Eighth Day Atonement Rituals for the Poor Person
\par }{\PP \VS{21}“If
the person is poor
and does not
have sufficient
means,
he must take
one
male lamb
as a guilt offering
for a wave offering
to make atonement
for himself, one-tenth of an ephah
of choice wheat flour
mixed
with olive oil
for a grain offering,
a log
of olive oil,
\VS{22}and two
turtledoves
or
two
young
pigeons,
which
are within his means.
One
will be
a sin offering
and the other
a burnt offering.
\par }{\PP \VS{23}“On the eighth
day
he must bring
them for his purification
to
the priest
at
the entrance
of the Meeting
Tent
before
the {\ND{Lord}},
\VS{24}and the priest
is to take
the
male lamb
of the guilt offering
and the
log
of olive oil
and wave
them as a wave offering
before
the {\ND{Lord}}.
\VS{25}Then he is to slaughter
the male lamb
of the guilt offering,
and the priest
is to take
some of the blood
of the guilt offering
and put
it on
the right
earlobe
of the one being cleansed,
on the thumb
of his right
hand,
and on
the big toe
of his right
foot.
\VS{26}The priest
will then pour
some
of the olive oil
into
his own
left
hand,
\VS{27}and sprinkle
some of
the olive oil
that
is in
his left
hand
with his right
forefinger
seven
times
before
the {\ND{Lord}}.
\VS{28}Then the priest
is to put
some of the olive oil
that
is in
his hand
on
the right
earlobe
of the one being cleansed,
on
the thumb
of his right
hand,
and on
the big toe
of his right
foot,
on
the place
of the blood
of the guilt offering,
\VS{29}and the remainder
of the olive oil
that
is in the hand
of the priest
he is to put
on
the head
of the one being cleansed
to make atonement
for him
before
the {\ND{Lord}}.
\par }{\PP \VS{30}“He will then make
one
of the turtledoves
or
young
pigeons,
which
are within his means,
\VS{31}a sin offering
and the
other
a burnt offering
along
with the grain offering.
So the priest
is to make atonement
for the one
being cleansed
before
the {\ND{Lord}}.
\VS{32}This
is the law
of the one in whom
there is a diseased
infection,
who
does not
have sufficient
means for his purification.”
\par }{\SH Purification of Disease-Infected Houses
\par }{\PP \VS{33}The Lord
spoke
to
Moses
and Aaron:
\VS{34}“When
you enter
the land
of Canaan
which
I am
about to give
to you for a possession,
and I put
a diseased
infection
in a house
in the land
you are to possess,
\VS{35}then
whoever
owns the house
must come
and declare
to the priest,
‘Something like an infection
is visible
to me in the house.’
\VS{36}Then the priest
will command
that
the
house
be cleared
before
the priest
enters
to examine
the infection
so that
everything
in the house
does not
become unclean,
and afterward
the priest
will enter
to examine
the
house.
\VS{37}He is to examine
the infection,
and if the infection
in the walls
of the house
consists of yellowish green
or
reddish
eruptions,
and it appears
to be deeper
than
the surface of the wall,
\VS{38}then the priest
is to go out
of the house
to
the doorway
of the house
and quarantine
the
house
for seven
days.
\VS{39}The priest
must return
on
the seventh
day
and examine
it, and if the infection
has spread
in the walls
of the house,
\VS{40}then the priest
is to command
that the stones
that had
the infection
in them be pulled
and thrown
outside
the city
into
an unclean
place.
\VS{41}Then
he is to have the house
scraped
all around
on the inside,
and the plaster
which
is scraped off
must be dumped
outside
the city
into
an unclean
place.
\VS{42}They are then to take
other
stones
and replace
those stones,
and he is to take
other
plaster
and replaster
the house.
\par }{\PP \VS{43}“If
the infection
returns
and breaks
out in the house
after
he has pulled
out the stones,
scraped
the house,
and it is replastered,
\VS{44}the priest
is
to come
and examine
it, and if the infection
has spread
in the house,
it is a malignant
disease
in the house.
It is
unclean.
\VS{45}He must tear
down the
house,
its stones,
its wood,
and all
the plaster
of the house,
and bring
all of it outside
the city
to
an unclean
place.
\VS{46}Anyone who enters
the house
all
the days
the priest has quarantined
it will be unclean
until
evening.
\VS{47}Anyone who lies
down in the house
must wash
his clothes.
Anyone who eats
in the house
must wash
his clothes.
\par }{\PP \VS{48}“If,
however, the priest
enters
and examines
it, and the
infection
has not
spread
in the house
after
the house
has been replastered,
then the priest
is to pronounce the house
clean
because
the infection
has been healed.
\VS{49}Then he is to take
two
birds,
a piece of cedar
wood,
a scrap of crimson
fabric,
and some twigs of hyssop
to decontaminate
the house,
\VS{50}and he is to slaughter
one
bird
into
a clay vessel
over fresh
water.
\VS{51}He must then take
the piece of cedar
wood,
the twigs of hyssop,
the scrap of crimson
fabric,
and the live
bird,
and dip
them in the blood
of the slaughtered
bird
and in the fresh
water,
and sprinkle
the house
seven
times.
\VS{52}So he is to decontaminate
the house
with the blood
of the bird,
the fresh
water,
the live
bird,
the piece of cedar
wood,
the twigs of hyssop,
and the scrap of crimson
fabric,
\VS{53}and he is to send
the
live
bird
away outside
the city
into
the open
countryside.
So he is to make atonement
for the house
and it will be clean.
\par }{\SH Summary of Purification Regulations for Infections
\par }{\PP \VS{54}“This
is the law
for all
diseased
infections,
for scall,
\VS{55}for the diseased
garment,
for the house,
\VS{56}for the swelling,
for the scab,
and for the bright spot,
\VS{57}to teach
when
something is unclean
and when
it is clean.
This
is the law
for dealing with infectious disease.”

\par }\Chap{15}{\PP \VerseOne{1}The
{\ND{Lord}}
spoke
to
Moses
and Aaron:
\VS{2}“Speak
to
the Israelites
and tell
them,
‘When
any
man
has a discharge
from his body,
his discharge
is unclean.
\VS{3}Now this
is his uncleanness
in regard to his discharge –
whether his body
secretes
his discharge
or
blocks
his discharge, he is unclean. All the days that his body has a discharge or his body
blocks his discharge,
this is his uncleanness.
\par }{\PP \VS{4}“‘Any
bed
the man with a discharge
lies
on
will be unclean,
and any
furniture
he sits
on
will be unclean.
\VS{5}Anyone
who touches
his bed
must wash
his clothes,
bathe
in water,
and be unclean
until
evening.
\VS{6}The one who sits
on
the furniture
the man with a discharge
sits
on
must wash
his clothes,
bathe
in water,
and be unclean
until
evening.
\VS{7}The one who touches
the body
of the man with a discharge
must wash
his clothes,
bathe
in water,
and be unclean
until
evening.
\VS{8}If
the man with a discharge
spits
on a person who is ceremonially clean,
that person must wash
his clothes,
bathe
in water,
and be unclean
until
evening.
\VS{9}Any
means of riding
the man with a discharge
rides
on
will be unclean.
\VS{10}Anyone
who touches
anything
that
was under
him will be unclean
until
evening,
and the one who carries
those items must wash
his clothes,
bathe
in water,
and be unclean
until
evening.
\VS{11}Anyone
whom
the man with the discharge
touches
without
having rinsed
his hands
in water
must wash
his clothes,
bathe
in water,
and be unclean
until
evening.
\VS{12}A clay vessel
which
the man with the discharge
touches
must be broken,
and any
wooden
utensil
must be rinsed
in water.
\par }{\SH Purity Regulations for Male Bodily Discharges
\par }{\PP \VS{13}“‘When
the man with
the discharge becomes
clean
from his discharge
he is to count
off for himself seven
days
for his purification,
and he must wash
his clothes,
bathe
in fresh
water,
and be clean.
\VS{14}Then on the eighth
day
he is to take
for himself two
turtledoves
or
two
young
pigeons,
and he is to present
himself before
the {\ND{Lord}}
at
the entrance
of the Meeting
Tent
and give
them to
the priest,
\VS{15}and the priest
is to make one
of them a sin offering
and the other
a burnt offering.
So the priest
is to make atonement
for him before
the {\ND{Lord}}
for his discharge.
\par }{\PP \VS{16}“‘When
a man
has a seminal
emission,
he must bathe
his whole
body
in water
and be unclean
until
evening,
\VS{17}and he must wash
in water
any
clothing
or leather
that
has semen
on it,
and it will be unclean
until
evening.
\VS{18}When a man
has sexual intercourse
with
a woman
and there is a seminal
emission,
they must bathe
in water
and be unclean
until
evening.
\par }{\SH Female Bodily Discharges
\par }{\PP \VS{19}“‘When
a woman
has a discharge
and her discharge
is blood
from her body,
she is to be
in her menstruation
seven
days,
and anyone
who touches
her will be unclean
until
evening.
\VS{20}Anything
she lies
on
during her menstruation
will be unclean,
and anything
she sits
on
will be unclean.
\VS{21}Anyone
who touches
her bed
must wash
his clothes,
bathe
in water,
and be unclean
until
evening.
\VS{22}Anyone
who touches
any
furniture
she sits
on
must wash
his clothes,
bathe
in water,
and be unclean
until
evening.
\VS{23}If
there is something on
the bed
or
on
the furniture
she
sits
on,
when he touches
it he will be unclean
until
evening,
\VS{24}and if
a man
actually
has sexual
intercourse with
her so that her menstrual
impurity touches him, then
he will be unclean
seven
days
and any
bed
he lies
on
will be unclean.
\par }{\PP \VS{25}“‘When
a woman’s
discharge
of blood
flows
many
days
not
at the time
of her menstruation,
or
if
it flows
beyond the time of her menstruation,
all
the days
of her discharge
of impurity
will be like
the days
of her menstruation
– she is
unclean.
\VS{26}Any
bed
she lies
on
all
the days
of her discharge
will be
to her like the bed
of her menstruation,
any
furniture
she sits
on
will be
unclean
like the impurity
of her menstruation,
\VS{27}and anyone
who touches
them will be unclean,
and he must wash
his clothes,
bathe
in water,
and be unclean
until
evening.
\par }{\SH Purity Regulations from Female Bodily Discharges
\par }{\PP \VS{28}“‘If
she becomes clean
from her discharge,
then she is to count
off for herself seven
days,
and afterward
she will be clean.
\VS{29}Then on the eighth
day
she must take
for herself two
turtledoves
or
two
young
pigeons
and she must bring
them to
the priest
at
the entrance
of the Meeting
Tent,
\VS{30}and the priest
is to make
one
a sin offering
and the
other
a burnt offering.
So the priest
is to make atonement
for her before
the
{\ND{Lord}}
from her discharge
of impurity.
\par }{\SH Summary of Purification Regulations for Bodily Discharges
\par }{\PP \VS{31}“‘Thus
you are to set the
Israelites
apart from their impurity
so that they do not
die
in their impurity
by defiling
my tabernacle
which
is in their midst.
\VS{32}This
is the law
of the one with a discharge: the one who
has
a seminal
emission
and becomes unclean by it,
\VS{33}the one who is sick
in her menstruation,
the one with
a discharge,
whether male
or female,
and a man
who
has sexual
intercourse with
an unclean woman.’ ”

\par }\Chap{16}{\PP \VerseOne{1}The
{\ND{Lord}}
spoke
to
Moses
after
the death
of Aaron’s
two
sons
when they approached
the presence
of the {\ND{Lord}}
and died,
\VS{2}and the
{\ND{Lord}}
said
to
Moses: “Tell
Aaron
your brother
that he must not
enter
at any
time
into
the holy
place inside the veil-canopy
in front
of the atonement plate
that
is on
the ark
so that
he may not
die,
for
I will appear
in the cloud
over
the atonement plate.
\par }{\SH Day of Atonement Offerings
\par }{\PP \VS{3}“In this
way Aaron
is to
enter
into the sanctuary
– with a young
bull
for a sin offering
and a ram
for a burnt offering.
\VS{4}He must put on
a holy
linen
tunic,
linen
leggings
are to cover his body,
and he is to wrap
himself with a linen
sash
and wrap
his head
with a linen
turban.
They
are holy
garments,
so he must bathe
his body
in water
and put
them on.
\VS{5}He must also take
two
male
goats
from the congregation
of the Israelites
for a sin offering
and one
ram
for a burnt offering.
\VS{6}Then Aaron
is to present
the sin offering
bull
which
is for himself and is to make atonement
on behalf
of himself
and his household.
\VS{7}He must then take
the two
goats
and stand
them before
the {\ND{Lord}}
at the entrance
of the Meeting
Tent,
\VS{8}and Aaron
is to cast lots
over
the two
goats,
one
lot
for the
{\ND{Lord}}
and one
lot
for Azazel.
\VS{9}Aaron
must then present
the goat
which
has been designated
by lot
for the
{\ND{Lord}}, and he is to make
it a sin offering,
\VS{10}but the goat
which
has been designated
by lot
for Azazel
is to be stood
alive
before
the {\ND{Lord}}
to make atonement
on
it by sending
it away to Azazel
into the wilderness.
\par }{\SH The Sin Offering Sacrificial Procedures
\par }{\PP \VS{11}“Aaron
is to present
the
sin offering
bull
which
is for himself, and he is to make atonement
on behalf
of himself and his household.
He is to slaughter
the
sin offering
bull
which is for himself,
\VS{12}and take
a censer
full
of coals
of fire
from the altar
before
the {\ND{Lord}}
and a full
double handful
of finely ground fragrant
incense,
and bring
them inside the veil-canopy.
\VS{13}He
must then put
the incense
on
the fire
before
the {\ND{Lord}}, and the cloud
of incense
will cover
the
atonement plate
which
is above
the ark of the testimony,
so that he will not
die.
\VS{14}Then he is to take
some of the blood
of the bull
and sprinkle
it with his finger
on
the eastern
face
of the atonement plate,
and in front
of the atonement plate
he is to sprinkle
some
of the blood
seven
times
with his finger.
\par }{\PP \VS{15}“He must then slaughter
the
sin offering
goat
which
is for the people.
He is to bring
its blood
inside the veil-canopy,
and he is to do
with
its blood
just
as he did
to the blood
of the bull: He is to sprinkle
it on
the atonement plate
and in front
of the atonement plate.
\VS{16}So he is to make atonement
for the holy
place from the impurities
of the Israelites
and from their transgressions
with regard to all
their sins,
and thus
he is to do
for the Meeting
Tent
which resides
with
them in the midst
of their impurities.
\VS{17}Nobody
is to be
in the Meeting
Tent
when he enters
to make atonement
in the holy place
until
he goes out,
and he has made atonement
on
his behalf,
on behalf
of his household,
and on behalf
of the whole
assembly
of Israel.
\par }{\PP \VS{18}“Then he is to
go out
to
the altar
which
is before
the
{\ND{Lord}}
and make atonement
for it.
He is to take
some of the blood
of the bull
and some of the blood
of the goat,
and put
it all around
on
the horns
of the altar.
\VS{19}Then he is to sprinkle
on
it some
of the blood
with his finger
seven
times,
and cleanse
and consecrate
it from the impurities
of the Israelites.
\par }{\SH The Live Goat Ritual Procedures
\par }{\PP \VS{20}“When he has finished
purifying
the holy
place, the Meeting
Tent,
and the altar,
he is to present
the live
goat.
\VS{21}Aaron
is to lay
his two
hands
on
the head
of the live
goat
and confess
over
it all
the iniquities
of the Israelites
and all
their transgressions
in regard to all
their sins,
and thus he is to put
them on
the head
of the goat
and send
it away into the wilderness
by the hand
of a man
standing ready.
\VS{22}The goat
is to bear
on
itself all
their iniquities
into
an inaccessible
land,
so he is to send
the goat
away in the wilderness.
\par }{\SH The Concluding Rituals
\par }{\PP \VS{23}“Aaron
must then enter
the Meeting
Tent
and take off
the linen
garments
which
he had put on
when he entered
the sanctuary,
and leave
them there.
\VS{24}Then he must bathe
his body
in water
in a holy
place,
put on
his clothes,
and go out
and make
his burnt offering
and the
people’s
burnt offering.
So he is to make atonement
on behalf
of himself and the people.
\par }{\PP \VS{25}“Then
he is to offer up the fat
of the sin offering
in smoke
on the altar,
\VS{26}and the one who sent
the goat
away to Azazel
must wash
his clothes,
bathe
his body
in water,
and afterward
he may reenter
the camp.
\VS{27}The
bull
of the sin offering
and the
goat
of the sin offering,
whose
blood
was brought
to make atonement
in the holy
place, must be brought
outside
the camp
and their hide,
their flesh,
and their dung
must be burned up,
\VS{28}and the one who burns
them must wash
his clothes
and bathe
his body
in water,
and afterward
he may reenter
the camp.
\par }{\SH Review of the Day of Atonement
\par }{\PP \VS{29}“This is to be
a perpetual
statute
for you. In the seventh
month,
on the tenth
day of the month,
you must humble
yourselves
and do no
work
of any
kind, both the native citizen
and the foreigner
who resides
in your midst,
\VS{30}for
on this
day
atonement
is to be made for you to cleanse
you from all
your sins;
you must be clean
before
the {\ND{Lord}}.
\VS{31}It is to be a Sabbath
of complete rest
for you, and you must humble
yourselves.
It is a perpetual
statute.
\par }{\PP \VS{32}“The priest
who
is anointed
and ordained
to act as high priest in place
of his father
is to make atonement.
He is to put on
the linen
garments,
the
holy
garments,
\VS{33}and he is
to purify
the
Most Holy
Place,
he is to purify the
Meeting
Tent
and the
altar,
and he is to make atonement
for the priests
and for all
the people
of the assembly.
\VS{34}This
is to be a perpetual
statute
for you to make atonement
for the Israelites
for all
their sins
once
a year.”
So he did
just
as the
{\ND{Lord}}
had commanded
Moses.

\par }\Chap{17}{\PP \VerseOne{1}The
{\ND{Lord}}
spoke
to
Moses:
\VS{2}“Speak
to
Aaron,
his sons,
and all
the Israelites,
and tell
them: ‘This
is the word
that
the {\ND{Lord}}
has commanded:
\VS{3}“Blood guilt will be accounted
to any man
from the house
of Israel
who
slaughters
an ox
or
a lamb
or
a goat
inside the camp
or
outside
the camp,
\VS{4}but has not
brought
it to
the entrance
of the Meeting
Tent
to present
it as an offering
to the
{\ND{Lord}}
before
the tabernacle
of the {\ND{Lord}}. He
has shed
blood,
so that man
will be cut off
from the midst
of his people.
\VS{5}This is so that
the Israelites
will bring
their sacrifices
that
they
are sacrificing
in
the open field
to the
{\ND{Lord}}
at
the entrance
of the Meeting
Tent
to
the priest
and sacrifice
them there as peace offering
sacrifices
to the
{\ND{Lord}}.
\VS{6}The priest
is to splash
the
blood
on
the altar
of the {\ND{Lord}}
at the entrance
of the Meeting
Tent,
and offer
the fat
up in smoke for a soothing
aroma
to the
{\ND{Lord}}.
\VS{7}So they must no
longer
offer
their sacrifices
to the goat
demons, acting
like prostitutes
by going after
them. This
is to be a perpetual
statute
for them
throughout their generations.
\par }{\PP \VS{8}“You are to
say
to them: ‘Any
man
from the house
of Israel
or from
the foreigners
who
reside
in their midst,
who
offers
a burnt offering
or
a sacrifice
\VS{9}but does not
bring
it
to
the entrance
of the Meeting
Tent
to offer
it to the
{\ND{Lord}} –
that person
will be cut off
from his people.
\par }{\SH Prohibition against Eating Blood
\par }{\PP \VS{10}“‘Any
man
from the house
of Israel
or from
the foreigners
who reside
in their midst
who
eats
any
blood,
I will set
my face
against that person
who eats
the
blood,
and I will cut
him off
from the
midst
of his people,
\VS{11}for
the life
of every living
thing is
in the blood.
So I myself
have assigned
it to you on
the altar
to make atonement
for your lives,
for
the blood
makes atonement
by means of the life.
\VS{12}Therefore,
I have said
to the Israelites: No
person
among you is to eat
blood,
and no
resident foreigner
who lives
among
you is to eat
blood.
\par }{\PP \VS{13}“‘Any
man
from the Israelites
or from
the foreigners
who reside
in their midst
who
hunts
a wild animal
or
a bird
that
may be eaten
must pour out
its blood
and cover
it with soil,
\VS{14}for
the life
of all
flesh
is its blood.
So I
have said
to the Israelites: You must not
eat
the blood
of any
living
thing because
the life
of every
living
thing is
its blood
– all
who eat
it will be cut off.
\par }{\SH Regulations for Eating Carcasses
\par }{\PP \VS{15}“‘Any
person
who
eats
an animal that has died of natural causes
or an animal torn
by beasts, whether a native citizen
or a foreigner,
must wash
his clothes,
bathe
in water,
and be unclean
until
evening;
then he becomes clean.
\VS{16}But if
he does not
wash
his clothes
and does not
bathe
his body,
he will bear
his punishment for iniquity.’ ”

\par }\Chap{18}{\PP \VerseOne{1}The
{\ND{Lord}}
spoke
to
Moses:
\VS{2}“Speak
to
the Israelites
and tell
them,
‘I am
the {\ND{Lord}}
your God!
\VS{3}You must not do
as they do
in the land
of Egypt
where
you have been living,
and you must not
do
as they do
in the land
of Canaan
into which
I am
about to bring
you; you
must not
walk
in their statutes.
\VS{4}You must observe
my regulations
and you must be sure
to walk
in my statutes.
I am
the {\ND{Lord}}
your God.
\VS{5}So you must keep
my statutes
and my regulations;
anyone
who does
so will live by keeping
them. I am
the {\ND{Lord}}.
\par }{\SH Laws of Sexual Relations
\par }{\PP \VS{6}“‘No
man
is to
approach
any
close relative
to have sexual intercourse
with her. I am
the
{\ND{Lord}}.
\VS{7}You must not expose
your father’s
nakedness
by having sexual intercourse with your mother.
She
is your mother;
you must not
have intercourse with her.
\VS{8}You must not
have sexual intercourse with your father’s
wife;
she is your father’s
nakedness.
\VS{9}You must not have sexual intercourse
with your sister,
whether she is your father’s
daughter
or
your mother’s
daughter,
whether she is born
in the same household
or
born
outside
it; you must not
have sexual intercourse with either of them.
\VS{10}You must not
expose
the nakedness
of your son’s
daughter
or
your daughter’s
daughter
by having sexual intercourse
with them, because
they are your own nakedness.
\VS{11}You must not have sexual intercourse
with the daughter
of your father’s
wife
born
of your father;
she is your sister.
You must not
have intercourse with her.
\VS{12}You must not
have sexual intercourse
with your father’s
sister;
she is your father’s
flesh.
\VS{13}You must not
have sexual intercourse
with your mother’s
sister,
because
she is
your mother’s
flesh.
\VS{14}You must not expose
the nakedness
of your father’s
brother;
you must not
approach
his wife
to have sexual intercourse with her. She is
your aunt.
\VS{15}You must not have sexual intercourse
with your daughter-in-law;
she is your son’s
wife.
You must not
have intercourse with her.
\VS{16}You must not
have sexual intercourse
with your brother’s
wife;
she
is your brother’s
nakedness.
\VS{17}You must not have
sexual intercourse
with both a woman
and her daughter;
you must not
take
as wife either her son’s
daughter
or her daughter’s
daughter
to have
intercourse
with them. They are
closely related
to her –
it is lewdness.
\VS{18}You must not
take
a woman
in marriage and then marry her sister
as a rival
wife while she is still alive,
to have sexual intercourse with her.
\par }{\PP \VS{19}“‘You must not
approach
a woman
in her menstrual
impurity
to have sexual intercourse with her.
\VS{20}You must not
have sexual
intercourse
with the wife
of your fellow citizen
to become unclean with her.
\VS{21}You must not
give
any of your children
as an offering to Molech,
so that you do not
profane
the
name
of your God.
I am
the {\ND{Lord}}!
\VS{22}You must not
have sexual intercourse
with
a male
as one has sexual intercourse
with a woman;
it is
a detestable act.
\VS{23}You must not
have sexual
intercourse
with any
animal
to become defiled
with it, and a woman
must not
stand
before
an animal
to have sexual
intercourse with it; it is
a perversion.
\par }{\SH Warning against the Abominations of the Nations
\par }{\PP \VS{24}“‘Do not
defile
yourselves with any
of these
things, for
the nations
which
I am
about to drive
out before
you have been defiled
with all
these things.
\VS{25}Therefore the land
has become unclean
and I have brought the punishment
for its iniquity
upon
it, so that the land
has vomited
out its inhabitants.
\VS{26}You yourselves
must obey
my statutes
and my regulations
and must not
do
any
of these
abominations,
both the native citizen
and the resident
foreigner
in your midst,
\VS{27}for
the people
who were in the land
before
you have done
all
these
abominations,
and the land
has become unclean.
\VS{28}So do not
make the land
vomit
you out because you defile
it just
as it has vomited
out the nations
that
were before you.
\VS{29}For if
anyone
does
any
of these
abominations,
the persons
who do
them will be cut off
from the midst
of their people.
\VS{30}You must obey
my charge
to not
practice
any of the abominable
statutes
that have been
done
before
you, so that you do not
defile
yourselves by them. I am
the {\ND{Lord}}
your God.’ ”

\par }\Chap{19}{\PP \VerseOne{1}The
{\ND{Lord}}
spoke
to
Moses:
\VS{2}“Speak
to
the whole
congregation
of the Israelites
and tell
them, ‘You must be
holy
because
I,
the {\ND{Lord}}
your God,
am holy.
\VS{3}Each
of you must respect
his mother
and his father,
and you must keep
my Sabbaths.
I am
the {\ND{Lord}}
your God.
\VS{4}Do not
turn
to
idols,
and you must not
make
for yourselves gods
of cast metal.
I am
the {\ND{Lord}}
your God.
\par }{\SH Eating the Peace Offering
\par }{\PP \VS{5}“‘When
you sacrifice
a peace offering
sacrifice to the
{\ND{Lord}}, you must sacrifice
it so that it is accepted for you.
\VS{6}It must be eaten
on the day
of your sacrifice
and on the following
day, but what is left
over until
the third
day
must be burned up.
\VS{7}If,
however, it is eaten
on
the third
day,
it is spoiled,
it will not
be accepted,
\VS{8}and the one who eats
it will bear
his punishment for iniquity
because
he has profaned
what is
holy
to the
{\ND{Lord}}. That person will be cut off
from his people.
\par }{\SH Leaving the Gleanings
\par }{\PP \VS{9}“‘When you gather
in the harvest
of your land,
you must not
completely harvest
the corner
of your field,
and you must not
gather
up the gleanings
of your harvest.
\VS{10}You must not
pick
your vineyard
bare, and you must not
gather
up the fallen grapes
of your vineyard.
You must leave
them for the
poor
and the foreigner.
I am
the
{\ND{Lord}}
your God.
\par }{\SH Dealing Honestly
\par }{\PP \VS{11}“‘You must not
steal,
you must not
tell lies,
and you must not
deal falsely
with your fellow
citizen.
\VS{12}You must not
swear
falsely
in my name,
so that you do not profane
the name
of your God.
I am
the {\ND{Lord}}.
\VS{13}You must not
oppress
your neighbor
or
commit robbery
against him. You must not
withhold
the wages of the hired
laborer overnight
until
morning.
\VS{14}You must not
curse
a deaf
person or put
a stumbling block
in front
of a blind
person. You must fear
your God;
I am
the {\ND{Lord}}.
\par }{\SH Justice, Love, and Propriety
\par }{\PP \VS{15}“‘You must not
deal
unjustly
in judgment: you must neither show partiality
to the poor
nor
honor the rich.
You must judge
your fellow citizen
fairly.
\VS{16}You must not
go
about as a slanderer
among your people.
You must not
stand
idly
by when your neighbor’s
life is at stake. I am
the {\ND{Lord}}.
\VS{17}You must not
hate
your brother
in your heart.
You must surely reprove
your fellow citizen
so that you do not
incur
sin
on account of him.
\VS{18}You must not
take vengeance
or
bear a grudge
against the children
of your people,
but you must love
your neighbor
as yourself.
I am
the {\ND{Lord}}.
\VS{19}You must keep
my statutes.
You must not
allow two different kinds
of your animals
to breed,
you must not
sow
your field
with two different kinds
of seed,
and you must not
wear
a garment
made of two different kinds
of fabric.
\par }{\SH Lying with a Slave Woman
\par }{\PP \VS{20}“‘When a man
has sexual intercourse
with
a woman,
although she is a slave woman
designated for another man
and she has not
yet been ransomed,
or
freedom
has not
been granted
to her, there will be an obligation to pay compensation.
They must not
be put to death,
because
she was not
free.
\VS{21}He must bring
his guilt offering
to the
{\ND{Lord}}
at the entrance
of the Meeting
Tent,
a guilt offering
ram,
\VS{22}and the priest
is to make atonement
for him
with the ram
of the guilt offering
before
the {\ND{Lord}}
for his sin
that
he has committed,
and he will be forgiven
of his sin
that
he has committed.
\par }{\SH The Produce of Fruit Trees
\par }{\PP \VS{23}“‘When
you enter
the land
and plant
any
fruit
tree,
you must consider its fruit
to be forbidden.
Three
years
it will be
forbidden
to you; it must not
be eaten.
\VS{24}In the fourth
year
all
its fruit
will be
holy,
praise
offerings to the
{\ND{Lord}}.
\VS{25}Then in the fifth
year
you may eat
its fruit
to add
its produce
to your harvest. I am
the {\ND{Lord}}
your God.
\par }{\SH Blood, Hair, and Body
\par }{\PP \VS{26}“‘You must not
eat
anything with the blood
still
in it. You must not
practice
either divination
or
soothsaying.
\VS{27}You must not
round
off the corners
of the hair on
your head
or
ruin
the corners
of your beard.
\VS{28}You must not
slash
your body
for a dead person
or incise
a tattoo
on yourself. I am
the {\ND{Lord}}.
\VS{29}Do not
profane
your daughter
by making her a prostitute,
so that the land
does not
practice prostitution
and become full
of lewdness.
\par }{\SH Purity, Honor, Respect, and Honesty
\par }{\PP \VS{30}“‘You must keep
my Sabbaths
and fear
my sanctuary.
I am
the {\ND{Lord}}.
\VS{31}Do not
turn
to
the spirits
of the dead and do not
seek
familiar spirits
to become unclean
by them. I am
the {\ND{Lord}}
your God.
\VS{32}You must stand up
in the presence
of the aged,
honor
the presence
of an elder,
and fear
your God.
I
am the
{\ND{Lord}}.
\VS{33}When
a foreigner
resides
with
you in your land,
you must not
oppress him.
\VS{34}The foreigner
who resides
with
you must be to you like a native citizen
among
you; so you must love
him as yourself,
because
you were foreigners
in the land
of Egypt.
I am
the {\ND{Lord}}
your God.
\VS{35}You must not
do
injustice
in the regulation
of measures,
whether of length, weight,
or volume.
\VS{36}You must have honest
balances,
honest
weights,
an honest
ephah,
and an honest
hin.
I am
the {\ND{Lord}}
your God
who
brought you out
from the land
of Egypt.
\VS{37}You must be sure to obey
all
my statutes
and regulations.
I am
the {\ND{Lord}}.’ ”

\par }\Chap{20}{\PP \VerseOne{1}The
{\ND{Lord}}
spoke
to
Moses:
\VS{2}“You are to say
to
the Israelites,
‘Any
man
from the Israelites
or from
the foreigners
who reside
in Israel
who
gives
any of his children
to Molech
must
be put to death;
the people
of the land
must pelt
him with stones.
\VS{3}I myself
will set
my face
against that man
and cut
him off
from the
midst
of his people,
because
he has given
some of his children
to Molech
and thereby
defiled
my sanctuary
and profaned
my holy
name.
\VS{4}If,
however,
the people
of the land
shut
their eyes
to that man
when he
gives
some of his children
to Molech
so that they do not put him to death,
\VS{5}I myself
will set
my face
against that man
and his clan.
I will cut off
from the midst
of their people
both him and all
who follow
after
him in spiritual prostitution,
to commit prostitution
by worshiping Molech.
\par }{\SH Prohibition against Spiritists and Mediums
\par }{\PP \VS{6}“‘The person
who
turns
to
the spirits
of the dead and familiar spirits
to commit prostitution
by going after
them, I will set
my
face
against that person
and cut
him off
from the midst
of his people.
\par }{\SH Exhortation to Holiness and Obedience
\par }{\PP \VS{7}“‘You must sanctify
yourselves and be
holy,
because
I am
the
{\ND{Lord}}
your God.
\VS{8}You must be sure
to obey
my statutes.
I am
the {\ND{Lord}}
who sanctifies you.
\par }{\SH Family Life and Sexual Prohibitions
\par }{\PP \VS{9}“‘If
anyone
curses
his father
and mother
he must be put
to death.
He has cursed
his father
and mother;
his blood guilt is on himself.
\VS{10}If a man
commits adultery
with
his neighbor’s
wife,
both
the adulterer
and the adulteress
must be put to death.
\VS{11}If a man
has
sexual
intercourse with
his father’s
wife,
he has exposed his father’s
nakedness.
Both
of them must be put to death;
their blood guilt is on themselves.
\VS{12}If a man
has
sexual
intercourse with
his daughter-in-law,
both
of them must be put to death.
They have committed perversion;
their blood guilt is on themselves.
\VS{13}If
a man
has sexual intercourse
with
a male
as one has sexual intercourse
with a woman,
the two
of them have committed
an abomination.
They must be put
to death;
their blood guilt is on themselves.
\VS{14}If a man
has
sexual
intercourse with both a woman
and her
mother,
it is lewdness.
Both he and they must be burned
to death, so there
is no
lewdness
in your midst.
\VS{15}If a man
has sexual
intercourse
with any animal,
he must
be put to death,
and you must kill
the animal.
\VS{16}If a woman
approaches
any
animal
to have sexual
intercourse with
it, you must kill
the woman,
and the animal
must
be put to death;
their blood guilt is on themselves.
\par }{\PP \VS{17}“‘If a man
has
sexual
intercourse with his sister,
whether the daughter
of his father
or
his mother,
so that he sees
her nakedness
and she sees
his nakedness,
it is a disgrace.
They must be cut off
in the sight
of the children
of their people.
He has exposed his sister’s
nakedness;
he will bear
his punishment for iniquity.
\VS{18}If a man
has
sexual
intercourse with
a menstruating
woman
and uncovers
her nakedness,
he has laid bare
her fountain
of blood and she
has exposed
the
fountain
of her blood,
so both
of them must be cut off
from the midst
of their people.
\VS{19}You must not
expose
the nakedness
of your mother’s
sister
and your father’s
sister, for
such a person has laid bare
his own close relative.
They must bear
their punishment
for iniquity.
\VS{20}If a man
has
sexual
intercourse with
his aunt,
he has exposed his uncle’s
nakedness;
they must bear
responsibility
for their sin, they will die
childless.
\VS{21}If a man
has sexual intercourse with
his brother’s
wife,
it is indecency.
He has exposed
his brother’s
nakedness;
they will be
childless.
\par }{\SH Exhortation to Holiness and Obedience
\par }{\PP \VS{22}“‘You must be sure to obey
all
my statutes
and regulations,
so that the land
to which
I am
about to bring
you to take up residence
there
does
not
vomit you out.
\VS{23}You must not
walk
in the statutes
of the nation
which
I am
about to drive
out before
you, because
they have done
all
these
things and I am filled with disgust against them.
\VS{24}So I have said
to you: You yourselves
will possess
their land
and I myself
will give
it to you for a possession,
a land
flowing
with milk
and honey.
I am
the {\ND{Lord}}
your God
who has
set you apart
from
the other peoples.
\VS{25}Therefore you must distinguish
between
the clean
animal
and the unclean,
and between
the unclean
bird
and the clean,
and you must not
make yourselves detestable
by means
of an animal
or bird
or anything
that
creeps
on the ground
– creatures I have distinguished
for you as unclean.
\VS{26}You must be
holy
to me because
I, the
{\ND{Lord}},
am
holy,
and I have set you apart
from
the other peoples
to be mine.
\par }{\SH Prohibition against Spiritists and Mediums
\par }{\PP \VS{27}“‘A man
or
woman
who has
in them a spirit
of the dead or
a familiar spirit
must
be put to death.
They must pelt
them with stones;
their blood guilt is on themselves.’ ”

\par }\Chap{21}{\PP \VerseOne{1}The
{\ND{Lord}}
said
to
Moses: “Say
to
the priests,
the sons
of Aaron
– say
to them,
‘For a dead person
no
priest is to defile
himself among his people,
\VS{2}except
for
his close relative
who is near
to him: his mother,
his father,
his son,
his daughter,
his brother,
\VS{3}and his virgin
sister
who is near
to
him, who has
no
husband;
he may defile himself for her.
\VS{4}He must not
defile
himself as a husband
among his people
so as to profane himself.
\VS{5}Priests must not
have a bald
spot shaved
on
their head,
they must not
shave
the corner
of their beard,
and they must not
cut slashes
in their body.
\par }{\PP \VS{6}“‘They must be holy
to their God,
and they must not
profane
the name
of their God,
because
they are the ones who present
the
{\ND{Lord}}’s
gifts,
the food
of their God.
Therefore they
must be
holy.
\VS{7}They must not
take
a wife
defiled by prostitution,
nor are they to take
a wife
divorced
from her
husband,
for
the priest is holy
to his God.
\VS{8}You must sanctify
him because
he presents
the food
of your God.
He must be holy
to you because
I, the
{\ND{Lord}}
who sanctifies
you all, am
holy.
\VS{9}If
a daughter
of a priest
profanes
herself by engaging in prostitution,
she is profaning
her father.
She
must be burned to death.
\par }{\SH Rules for the High Priest
\par }{\PP \VS{10}“‘The high priest
– who is greater
than his brothers,
on
whose
head
the anointing
oil
is poured,
who has
been ordained
to wear
the priestly garments
– must neither
dishevel
the hair of his head
nor
tear his garments.
\VS{11}He must not
go
where there is any
dead
person;
he must not
defile
himself even for his father
and his mother.
\VS{12}He
must not
go out
from
the sanctuary
and must not
profane
the
sanctuary
of his God,
because
the dedication
of the anointing
oil
of his
God
is on
him. I am
the {\ND{Lord}}.
\VS{13}He must take
a wife
who is a virgin.
\VS{14}He must not
marry
a widow,
a divorced
woman, or one profaned by prostitution;
he may
only
take
a virgin
from his people
as a wife.
\VS{15}He must not
profane
his children
among his people,
for
I am
the
{\ND{Lord}}
who sanctifies him.’ ”
\par }{\SH Rules for the Priesthood
\par }{\PP \VS{16}The
{\ND{Lord}}
spoke
to
Moses:
\VS{17}“Tell
Aaron,
‘No man
from your descendants
throughout their generations
who
has a physical flaw
is to approach
to present
the food
of his God.
\VS{18}Certainly
no man
who
has a physical flaw
is to approach: a blind
man,
or
one who is lame,
or
one with a slit
nose, or
a limb too long,
\VS{19}or
a man
who
has had a broken
leg
or
arm,
\VS{20}or
a hunchback,
or
a dwarf,
or
one with a spot
in his eye,
or
a festering eruption,
or
a feverish rash,
or
a crushed
testicle.
\VS{21}No man
from the descendants
of Aaron
the priest
who
has a physical flaw
may step forward
to present
the
{\ND{Lord}}’s
gifts;
he has a physical flaw,
so
he must not
step forward
to present
the
food
of his God.
\VS{22}He may eat
both the most holy
and the holy
food
of his God,
\VS{23}but
he must not
go
into the veil-canopy
or
step forward
to
the altar
because
he has a physical flaw.
Thus he must not
profane
my holy places,
for
I am
the {\ND{Lord}}
who sanctifies them.’ ”
\par }{\PP \VS{24}So Moses
spoke
these things to
Aaron,
his sons,
and all
the Israelites.

\par }\Chap{22}{\PP \VerseOne{1}The
{\ND{Lord}}
spoke
to
Moses:
\VS{2}“Tell
Aaron
and his sons
that they must deal respectfully
with the holy
offerings of the Israelites,
which
they
consecrate
to me, so that they do not
profane
my holy
name.
I am
the {\ND{Lord}}.
\VS{3}Say
to them,
‘Throughout your generations,
if any
man
from all
your descendants
approaches
the holy
offerings which
the Israelites
consecrate
to the
{\ND{Lord}}
while he is impure,
that person must be cut off
from before
me. I am
the {\ND{Lord}}.
\VS{4}No man
from the descendants
of Aaron
who is diseased
or
has a discharge
may eat
the holy offerings
until
he becomes clean.
The one who touches
anything
made unclean
by contact with a dead person,
or
a man
who
has
a seminal
emission,
\VS{5}or
a man
who touches
a swarming thing
by which
he becomes unclean,
or
touches a person
by which
he becomes unclean,
whatever
that person’s impurity –
\VS{6}the person
who
touches
any of these will be unclean
until
evening
and must not
eat
from
the holy
offerings unless
he has bathed
his body
in water.
\VS{7}When the sun
goes down
he will be clean,
and afterward
he may eat
from
the holy
offerings, because
they are his food.
\VS{8}He must not
eat
an animal that has died of natural causes
or an animal torn
by beasts and thus become unclean
by it. I am
the {\ND{Lord}}.
\VS{9}They must keep
my charge
so that they do not
incur
sin
on
account of it and therefore die
because
they profane
it. I am
the {\ND{Lord}}
who sanctifies them.
\par }{\PP \VS{10}“‘No
lay
person may eat
anything holy.
Neither a priest’s
lodger
nor
a hired laborer
may eat
anything holy,
\VS{11}but if
a priest
buys
a person
with his own money,
that
person may eat
the holy offerings, and those born
in the priest’s own house
may eat
his food.
\VS{12}If
a priest’s
daughter
marries a lay
person, she
may not
eat
the holy
contribution offerings,
\VS{13}but if
a priest’s
daughter
is a widow
or divorced,
and she has no
children
so that she returns
to
live in her father’s
house
as in her youth,
she may eat
from her father’s
food,
but no
lay person
may eat it.
\par }{\PP \VS{14}“‘If
a man
eats
a holy offering
by mistake,
he must add
one fifth
to
it and give
the holy offering
to the priest.
\VS{15}They must not
profane
the holy
offerings which
the Israelites
contribute
to the
{\ND{Lord}},
\VS{16}and so cause them to incur
a penalty
for guilt
when they eat
their holy
offerings, for
I am
the {\ND{Lord}}
who sanctifies them.’ ”
\par }{\SH Regulations for Offering Votive and Freewill Offerings
\par }{\PP \VS{17}The
{\ND{Lord}}
spoke
to
Moses:
\VS{18}“Speak
to
Aaron,
his sons,
and all
the Israelites
and tell
them,
‘When any
man
from the house
of Israel
or from
the foreigners
in Israel
presents
his offering
for any
of the votive
or freewill
offerings which
they present
to the
{\ND{Lord}}
as a burnt offering,
\VS{19}if it is to be acceptable for your benefit
it must be a flawless
male
from the cattle,
sheep,
or goats.
\VS{20}You must not
present
anything
that
has a flaw,
because
it will not
be acceptable for your benefit.
\VS{21}If
a man
presents
a peace offering
sacrifice
to the
{\ND{Lord}}
for
a special
votive offering
or
for a freewill
offering from the herd
or
the flock,
it must be flawless
to be
acceptable;
it must have no
flaw.
\par }{\PP \VS{22}“‘You must not
present
to the
{\ND{Lord}}
something blind,
or
with a broken
bone, or
mutilated,
or
with a running sore,
or
with a festering eruption,
or
with a feverish rash.
You must not
give
any of these
as a gift
on
the altar
to the
{\ND{Lord}}.
\VS{23}As for an ox
or a sheep
with a limb
too long or stunted,
you may present
it as a freewill offering,
but it will not
be acceptable
for a votive offering.
\VS{24}You must not
present
to the
{\ND{Lord}}
something with testicles
that are bruised,
crushed, torn, or cut off;
you must not
do
this in your land.
\VS{25}Even from a foreigner
you must not
present
the
food
of your God
from such animals
as these,
for
they are ruined
and flawed;
they will not
be acceptable for your benefit.’ ”
\par }{\PP \VS{26}The
{\ND{Lord}}
spoke
to
Moses:
\VS{27}“When
an ox,
lamb,
or
goat
is born,
it must be
under
the care of its mother
seven
days,
but from the eighth
day
onward
it will be acceptable
as an offering
gift
to the
{\ND{Lord}}.
\VS{28}You must not
slaughter
an ox
or
a sheep
and its young
on the same
day.
\VS{29}When
you sacrifice
a thanksgiving
offering to the
{\ND{Lord}}, you must sacrifice it so that it is acceptable for your benefit.
\VS{30}On that very day
it must be eaten;
you must not
leave any
part
of it
over until
morning.
I am
the {\ND{Lord}}.
\par }{\PP \VS{31}“You must be sure
to do
my commandments.
I am
the {\ND{Lord}}.
\VS{32}You must not
profane
my holy
name,
and I will be sanctified
in the midst
of the Israelites.
I am
the {\ND{Lord}}
who sanctifies you,
\VS{33}the one who brought you out
from the land
of Egypt
to be
your God.
I am
the {\ND{Lord}}.”

\par }\Chap{23}{\PP \VerseOne{1}The
{\ND{Lord}}
spoke
to
Moses:
\VS{2}“Speak
to
the Israelites
and tell
them,
‘These
are
the
{\ND{Lord}}’s
appointed
times which
you must proclaim
as holy
assemblies
– my appointed times:
\par }{\SH The Weekly Sabbath
\par }{\PP \VS{3}“‘Six
days
work
may be done,
but on the seventh
day
there must be a Sabbath
of complete rest,
a holy
assembly.
You must not
do
any
work;
it is a Sabbath
to the
{\ND{Lord}}
in all
the places where you live.
\par }{\SH The Festival of Passover and Unleavened Bread
\par }{\PP \VS{4}“‘These
are the
{\ND{Lord}}’s
appointed
times, holy
assemblies,
which
you must proclaim
at their appointed time.
\VS{5}In the first
month,
on the fourteenth
day of the month,
at twilight,
is a Passover offering
to the
{\ND{Lord}}.
\VS{6}Then on the fifteenth
day
of the same
month
will be the festival
of unleavened
bread to the
{\ND{Lord}}; seven
days
you must eat
unleavened bread.
\VS{7}On the first
day
there will be a holy
assembly
for you; you must not
do
any
regular
work.
\VS{8}You must present
a gift
to the
{\ND{Lord}}
for seven
days,
and the seventh
day
is a holy
assembly;
you must not
do
any
regular
work.’ ”
\par }{\SH The Presentation of First Fruits
\par }{\PP \VS{9}The
{\ND{Lord}}
spoke
to
Moses:
\VS{10}“Speak
to
the Israelites
and tell
them, ‘When you enter
the land
that
I
am about to give
to you and you gather
in its harvest,
then you must bring
the sheaf
of the first portion
of your harvest
to
the priest,
\VS{11}and he must wave
the sheaf
before
the {\ND{Lord}}
to be accepted for your benefit – on the day after the Sabbath the priest is to wave it.
\VS{12}On the day
you wave
the sheaf
you must also offer a flawless
yearling
lamb
for a burnt offering
to the
{\ND{Lord}},
\VS{13}along with its grain offering,
two
tenths of an ephah
of choice wheat flour
mixed
with olive oil,
as a gift
to the
{\ND{Lord}}, a soothing
aroma,
and its drink offering,
one fourth
of a hin
of wine.
\VS{14}You must not
eat
bread,
roasted
grain, or fresh grain
until
this
very
day,
until
you bring
the offering
of your God.
This is a perpetual
statute
throughout your generations
in all
the places where you live.
\par }{\SH The Festival of Weeks
\par }{\PP \VS{15}“‘You must count
for yourselves seven
weeks from the day after
the Sabbath,
from the day
you bring
the wave offering
sheaf;
they must be complete
weeks.
\VS{16}You must count
fifty
days
– until
the day after
the seventh
Sabbath
– and then you must present
a new
grain offering
to the
{\ND{Lord}}.
\VS{17}From the places
where you live you must bring
two
loaves of bread
for a wave offering;
they must be
made from two
tenths of an ephah
of fine wheat flour,
baked
with yeast,
as first fruits
to the
{\ND{Lord}}.
\VS{18}Along with the loaves of bread,
you must also present seven
flawless
yearling
lambs,
one
young
bull,
and two
rams.
They are to be
a burnt offering
to the
{\ND{Lord}}
along with their grain offering
and drink offerings,
a gift
of a soothing
aroma
to the
{\ND{Lord}}.
\VS{19}You must also offer
one male
goat
for a
sin offering
and two
yearling
lambs
for a peace offering
sacrifice,
\VS{20}and the priest
is to wave them – the two lambs – along with the bread of the first fruits, as a wave offering before the
{\ND{Lord}}; they will be holy to the
{\ND{Lord}} for the priest.
\par }{\PP \VS{21}“‘On
this
very
day
you must proclaim
an assembly;
it is to be a holy
assembly for you. You must not
do
any
regular work.
This is a perpetual
statute
in all
the places
where you live throughout your generations.
\VS{22}When
you gather
in the harvest
of your land,
you must not
completely
harvest
the corner
of your field,
and you must not
gather
up the gleanings
of your harvest.
You must leave
them for the poor
and the foreigner.
I am
the {\ND{Lord}}
your God.’ ”
\par }{\SH The Festival of Horn Blasts
\par }{\PP \VS{23}The
{\ND{Lord}}
spoke
to
Moses:
\VS{24}“Tell
the Israelites,
‘In the seventh
month,
on the first
day of the month,
you must have a complete rest,
a memorial
announced by loud horn blasts,
a holy
assembly.
\VS{25}You must not
do
any
regular
work,
but you must present
a gift
to the
{\ND{Lord}}.’ ”
\par }{\SH The Day of Atonement
\par }{\PP \VS{26}The
{\ND{Lord}}
spoke
to
Moses:
\VS{27}“The tenth
day of this
seventh
month
is the Day
of Atonement.
It is
to be a holy
assembly
for you, and you must humble
yourselves
and present
a gift
to the
{\ND{Lord}}.
\VS{28}You must not
do
any
work
on this
particular
day,
because
it is
a day
of atonement
to make atonement
for yourselves before
the {\ND{Lord}}
your God.
\VS{29}Indeed,
any
person
who
does not
behave with humility
on this
particular
day
will be cut off
from his people.
\VS{30}As for any
person
who
does
any
work
on this
particular
day,
I will exterminate
that
person
from the midst
of his people!
\VS{31}You must not
do
any
work.
This is a perpetual
statute
throughout your generations
in all
the places where you live.
\VS{32}It is a Sabbath
of complete rest
for you, and you must humble
yourselves
on the ninth
day of the month
in the evening,
from evening
until
evening
you must observe
your Sabbath.”
\par }{\SH The Festival of Booths
\par }{\PP \VS{33}The
{\ND{Lord}}
spoke
to
Moses:
\VS{34}“Tell
the Israelites,
‘On
the fifteenth
day
of this
seventh
month
is the Festival
of Temporary Shelters
for seven
days
to the
{\ND{Lord}}.
\VS{35}On the first
day
is a holy
assembly;
you must do
no
regular
work.
\VS{36}For seven
days
you must present
a gift
to the
{\ND{Lord}}. On
the eighth
day
there is to be a holy
assembly
for you, and you must present
a gift
to the
{\ND{Lord}}. It is
a solemn assembly
day; you must not
do
any
regular
work.
\par }{\PP \VS{37}“‘These
are the appointed
times of the
{\ND{Lord}}
that
you must proclaim
as holy
assemblies
to present
a gift
to the
{\ND{Lord}} –
burnt offering,
grain offering,
sacrifice,
and drink offerings,
each day
according to its regulation,
\VS{38}besides
the Sabbaths
of the {\ND{Lord}}
and all
your gifts,
votive offerings,
and freewill
offerings which
you must give
to the
{\ND{Lord}}.
\par }{\PP \VS{39}“‘On
the fifteenth
day
of the seventh
month,
when you gather
in the produce
of the land,
you must celebrate
a pilgrim festival
of the {\ND{Lord}}
for seven
days.
On the first
day
is a complete rest
and on the eighth
day
is complete rest.
\VS{40}On the first
day
you must take
for yourselves branches
from majestic
trees –
palm
branches,
branches
of leafy
trees,
and willows
of the brook
– and you must rejoice
before
the {\ND{Lord}}
your God
for seven
days.
\VS{41}You must celebrate it as a pilgrim
festival
to the
{\ND{Lord}}
for seven
days
in the year.
This
is a perpetual
statute
throughout your generations;
you must celebrate
it in the seventh
month.
\VS{42}You must live
in temporary shelters
for seven
days;
every
native citizen
in Israel
must live
in temporary shelters,
\VS{43}so
that your future generations
may know
that
I made the Israelites
live
in temporary shelters
when
I brought them out
from the land
of Egypt.
I am
the {\ND{Lord}}
your God.’ ”
\par }{\PP \VS{44}So Moses
spoke
to
the Israelites
about the appointed
times of the
{\ND{Lord}}.

\par }\Chap{24}{\PP \VerseOne{1}The
{\ND{Lord}}
spoke
to
Moses:
\VS{2}“Command
the Israelites
to bring
to
you pure
oil
of beaten
olives
for the light,
to make a lamp
burn
continually.
\VS{3}Outside
the veil-canopy
of the congregation
in the Meeting
Tent
Aaron
must arrange
it from evening
until
morning
before
the {\ND{Lord}}
continually.
This is a perpetual
statute
throughout your generations.
\VS{4}On
the ceremonially pure
lampstand
he must arrange
the lamps
before
the {\ND{Lord}}
continually.
\par }{\PP \VS{5}“You must take
choice wheat flour
and bake
twelve
loaves;
there must be two
tenths of an ephah
of flour in each
loaf,
\VS{6}and you must set
them
in two
rows,
six
in a row,
on
the ceremonially pure
table
before
the {\ND{Lord}}.
\VS{7}You must put
pure
frankincense
on
each row,
and it will become
a memorial
portion for the bread,
a gift
to the
{\ND{Lord}}.
\VS{8}Each Sabbath
day
Aaron must arrange
it before
the {\ND{Lord}}
continually;
this portion is from the Israelites
as a perpetual
covenant.
\VS{9}It will belong
to Aaron
and his sons,
and they must eat
it in
a holy
place
because
it is
most
holy
to him, a perpetual
allotted portion
from the gifts
of the {\ND{Lord}}.”
\par }{\SH A Case of Blaspheming the Name
\par }{\PP \VS{10}Now an Israelite woman’s
son
whose father
was an Egyptian
went out
among
the Israelites,
and the Israelite woman’s
son
and an Israelite
man
had a fight
in the camp.
\VS{11}The Israelite
woman’s
son
misused
the Name
and cursed,
so they brought
him to
Moses.
(Now his mother’s
name
was Shelomith
daughter
of Dibri,
of the tribe
of Dan.)
\VS{12}So they placed
him in custody
until they were
able to make a clear
legal decision for themselves
based on
words from the mouth
of the
{\ND{Lord}}.
\par }{\PP \VS{13}Then the
{\ND{Lord}}
spoke
to
Moses:
\VS{14}“Bring
the
one who cursed
outside
the camp,
and all
who heard
him are to lay
their hands
on
his head,
and the
whole
congregation
is to stone him to death.
\VS{15}Moreover,
you are to
tell
the Israelites,
‘If
any man
curses
his God
he will bear
responsibility for his sin,
\VS{16}and one who misuses
the name
of the {\ND{Lord}}
must surely be put
to death.
The whole
congregation
must surely stone him,
whether he is a foreigner
or a native citizen;
when he misuses
the Name
he must be put to death.
\par }{\PP \VS{17}“‘If
a man
beats
any
person
to death,
he must be put to death.
\VS{18}One who beats
an animal
to death
must make restitution
for it, life
for
life.
\VS{19}If
a man
inflicts
an injury
on his fellow citizen,
just
as he has done
it must be done to him –
\VS{20}fracture
for
fracture,
eye
for
eye,
tooth
for
tooth
– just
as he inflicts
an injury
on another person
that same
injury must be inflicted on him.
\VS{21}One who beats
an animal
to death must make restitution
for it, but one who beats
a person
to death must be put to death.
\VS{22}There will be
one
regulation
for you, whether a foreigner
or a native citizen,
for
I am
the {\ND{Lord}}
your God.’ ”
\par }{\PP \VS{23}Then
Moses
spoke
to
the Israelites
and they brought
the one who cursed
outside
the camp
and stoned
him with stones.
So the Israelites
did
just
as the
{\ND{Lord}}
had commanded
Moses.

\par }\Chap{25}{\PP \VerseOne{1}The
{\ND{Lord}}
spoke
to
Moses
at Mount
Sinai:
\VS{2}“Speak
to
the Israelites
and tell
them,
‘When
you enter
the land
that
I am
giving
you, the land
must observe
a Sabbath
to the
{\ND{Lord}}.
\VS{3}Six
years
you may sow
your field,
and six
years
you may prune
your vineyard
and gather
the produce,
\VS{4}but in the seventh
year
the land
must have a Sabbath
of complete rest –
a Sabbath
to the
{\ND{Lord}}. You must not
sow
your field
or prune
your vineyard.
\VS{5}You must not
gather
in the aftergrowth
of your harvest
and you must not
pick
the grapes
of your unpruned vines;
the land
must have a year
of complete rest.
\VS{6}You may have the Sabbath
produce
of the land
to eat
– you, your male servant,
your female servant,
your hired worker,
the resident foreigner
who stays
with you,
\VS{7}your cattle,
and the wild animals
that
are in your land
– all
its produce
will be
for you to eat.
\par }{\SH Regulations for the Jubilee Year of Release
\par }{\PP \VS{8}“‘You must count
off seven
weeks
of years,
seven
times
seven
years,
and the days
of the seven
weeks
of years
will amount to forty-nine
years.
\VS{9}You must sound
loud horn
blasts –
in the seventh
month,
on the tenth
day of the month,
on the Day
of Atonement
– you must sound
the horn
in your entire
land.
\VS{10}So you must consecrate
the
fiftieth
year,
and you must proclaim
a release
in the land
for all
its inhabitants.
That year will be
your jubilee;
each
one of you must return
to
his property
and each
one of you must return
to
his clan.
\VS{11}That
fiftieth
year
will be
your jubilee;
you must not
sow
the land, harvest
its aftergrowth,
or
pick the grapes
of its unpruned vines.
\VS{12}Because
that year is a jubilee,
it
will be holy to you – you may eat its produce from the field.
\par }{\SH Release of Landed Property
\par }{\PP \VS{13}“‘In this
year
of jubilee
you must each
return
to
your property.
\VS{14}If
you make
a sale
to your fellow citizen
or
buy
from your fellow citizen,
no
one
is to wrong
his brother.
\VS{15}You may buy
it from your fellow citizen
according to the number
of years
since
the last jubilee;
he may sell
it to you according to
the years
of produce that are left.
\VS{16}The more years
there are, the more
you may make its purchase
price, and the fewer
years
there are, the less
you must make its
purchase
price, because
he is
only selling
to you a number
of years of produce.
\VS{17}No
one is to oppress
his fellow
citizen,
but you must fear
your God,
because
I am
the {\ND{Lord}}
your God.
\VS{18}You must obey
my statutes
and my regulations;
you must be sure
to keep
them so
that you may live
securely
in
the land.
\par }{\PP \VS{19}“‘The land
will give
its fruit
and you may eat
until you are satisfied,
and you may live
securely in the land.
\VS{20}If
you say,
‘What
will we eat
in the seventh
year
if
we do not
sow
and gather
our produce?’
\VS{21}I will command
my blessing
for you in the sixth
year
so
that it may yield
the produce
for three
years,
\VS{22}and you may sow
the eighth
year
and eat
from
that sixth year’s produce –
old
produce. Until
you bring in
the ninth
year’s
produce,
you may eat
old produce.
\VS{23}The land
must not
be sold
without reclaim
because
the land
belongs to me, for
you
are foreigners
and residents
with me.
\VS{24}In all
your landed
property
you must provide
for the right of redemption
of the land.
\par }{\PP \VS{25}“‘If
your brother
becomes impoverished
and sells
some of his property,
his near
redeemer
is to
come
to you and redeem
what his brother
sold.
\VS{26}If a man
has
no
redeemer,
but he prospers
and gains
enough
for its redemption,
\VS{27}he
is to calculate
the value
of the
years
it was sold,
refund
the balance
to the
man
to whom
he had sold
it, and return
to his property.
\VS{28}If
he has not
prospered
enough
to refund
a balance
to him, then
what he sold
will belong to the one who bought
it until
the jubilee
year,
but it must revert
in the jubilee
and the original owner may return
to his property.
\par }{\SH Release of Houses
\par }{\PP \VS{29}“‘If
a man
sells
a residential
house
in a walled
city,
its right of redemption
must extend until
one full year
from its sale;
its right of redemption
must extend
to a full calendar year.
\VS{30}If
it is not
redeemed
before
the full
calendar
year
is ended, the house
in the walled
city
will belong
without
reclaim
to the one who bought
it throughout his generations;
it will not
revert
in the jubilee.
\VS{31}The houses
of villages,
however, which
have no
wall
surrounding
them must be considered
as the field
of the land;
they will have the right of redemption
and must revert
in the jubilee.
\VS{32}As for the cities
of the Levites,
the houses
in the cities
which they possess,
the Levites
must have a perpetual
right of redemption.
\VS{33}Whatever
someone among the Levites
might redeem
– the sale
of a house
which is his property
in a city
– must revert in the jubilee,
because
the houses
of the cities
of the Levites
are their property
in the midst
of the Israelites.
\VS{34}Moreover, the open
field
areas
of their cities
must not
be sold,
because
that is their perpetual
possession.
\par }{\SH Debt and Slave Regulations
\par }{\PP \VS{35}“‘If
your brother
becomes impoverished
and is indebted to you,
you must support
him; he must live
with
you like a foreign
resident.
\VS{36}Do not
take
interest
or profit
from him, but you must fear
your God
and your brother
must live
with you.
\VS{37}You must not
lend
him your money
at interest
and you must not
sell
him food
for profit.
\VS{38}I am
the {\ND{Lord}}
your God
who
brought you out
from the land
of Egypt
to give
you the land
of Canaan
– to be
your God.
\par }{\PP \VS{39}“‘If
your brother
becomes impoverished
with
regard to you so that he sells
himself to you, you must not
subject
him to slave
service.
\VS{40}He must be
with you
as a hired worker,
as a resident foreigner;
he must serve
with you
until
the year
of jubilee,
\VS{41}but then he may
go
free,
he and his children
with
him, and may return
to
his family
and to
the property
of his ancestors.
\VS{42}Since
they are
my servants
whom
I brought out
from the land
of Egypt,
they must not
be sold
in a slave
sale.
\VS{43}You must not
rule
over him harshly,
but you must fear
your God.
\par }{\PP \VS{44}“‘As for your male
and female
slaves
who
may belong to you – you may buy male and female slaves from the nations all around you.
\VS{45}Also
you may buy
slaves from the children
of the foreigners
who reside
with
you, and from their
families
that
are with
you, whom
they have fathered
in your land,
they may become
your property.
\VS{46}You may give them as inheritance
to your children
after
you to possess
as property.
You may enslave
them perpetually.
However, as for your brothers
the Israelites,
no
man
may rule
over his brother
harshly.
\par }{\PP \VS{47}“‘If
a resident
foreigner
who is with
you prospers
and your brother
becomes impoverished
with
regard to him so that he sells
himself to a resident
foreigner
who is with
you or
to a member
of a foreigner’s
family,
\VS{48}after
he has sold
himself he retains a right of redemption.
One
of his brothers
may redeem him,
\VS{49}or
his uncle
or
his cousin
may redeem
him, or
anyone of the rest of his blood relatives
– his family –
may redeem
him, or
if he prospers
he may redeem
himself.
\VS{50}He
must calculate
with
the one who bought
him the number of years
from the year he sold
himself to him until
the jubilee
year,
and the cost
of his sale
must correspond to the number
of years,
according to the rate
of wages a hired worker
would have earned while with him.
\VS{51}If
there are still
many
years,
in keeping
with them he must refund
most of the cost
of his purchase
for his redemption,
\VS{52}but if
only a few
years
remain
until
the jubilee,
he must calculate
for himself in keeping with the remaining
years
and refund
it for his redemption.
\VS{53}He must be
with
the one who bought him like a yearly
hired worker.
The one who bought him must not
rule
over him harshly
in your sight.
\VS{54}If,
however, he is not
redeemed
in these
ways, he must go free
in the jubilee
year,
he and his children
with him,
\VS{55}because
the Israelites
are my own servants;
they are
my servants
whom
I brought out
from the land
of Egypt.
I am
the {\ND{Lord}}
your God.

\par }\Chap{26}{\PP \VerseOne{1}“‘You must not
make
for yourselves idols,
so you must not
set
up for yourselves a carved image
or a pillar,
and you must not
place a sculpted
stone
in your land
to bow
down before it,
for
I am
the {\ND{Lord}}
your God.
\VS{2}You must keep
my Sabbaths
and reverence
my sanctuary.
I am
the {\ND{Lord}}.
\par }{\SH The Benefits of Obedience
\par }{\PP \VS{3}“‘If
you walk
in my statutes
and are sure
to obey
my commandments,
\VS{4}I will give
you your rains
in their time
so that the land
will give
its yield
and the trees
of the field
will produce
their fruit.
\VS{5}Threshing
season will extend
for you until
the season for harvesting grapes,
and the season for harvesting grapes
will extend until
sowing
season, so you will eat
your bread
until you are satisfied,
and you will live
securely
in your land.
\VS{6}I will grant
peace
in the land
so that you will lie down to sleep
without
anyone terrifying
you. I will remove
harmful
animals
from
the land,
and no
sword
of war will pass through
your land.
\VS{7}You will pursue
your enemies
and they will fall
before
you by the sword.
\VS{8}Five
of you will pursue
a hundred,
and a hundred
of you will pursue
ten thousand,
and your enemies
will fall
before
you by the sword.
\VS{9}I will turn
to
you, make you fruitful,
multiply
you, and maintain
my covenant
with you.
\VS{10}You will still be eating
stored
produce from the previous year
and will have to clean out
what is stored
from the previous year to make room for new.
\par }{\PP \VS{11}“‘I will put
my tabernacle
in your midst
and I will not
abhor you.
\VS{12}I will walk
among
you, and I will be
your God
and you
will be
my people.
\VS{13}I am
the {\ND{Lord}}
your God
who
brought you out
from the
land
of Egypt,
from being their slaves,
and I broke
the bars
of your yoke
and caused you to walk
upright.
\par }{\SH The Consequences of Disobedience
\par }{\PP \VS{14}“‘If,
however, you do not
obey
me and keep
all
these
commandments –
\VS{15}if
you reject
my statutes
and abhor
my
regulations
so that you do not
keep
all
my commandments
and you break
my covenant –
\VS{16}I
for my part
will do
this
to you: I will inflict
horror
on
you, consumption
and fever,
which diminish
eyesight
and drain
away the vitality of life.
You will sow
your seed
in vain
because your
enemies
will eat it.
\VS{17}I will set
my face
against you. You will be struck
down before
your enemies,
those who hate
you will rule
over you, and you will flee
when there is no
one pursuing you.
\par }{\PP \VS{18}“‘If,
in spite
of all
these
things, you do not
obey
me,
I will discipline
you seven
times more
on
account of your sins.
\VS{19}I will break
your strong
pride
and make
your sky
like iron
and your land
like bronze.
\VS{20}Your strength
will be used up in vain,
your land
will not
give
its yield,
and the trees
of the land
will not
produce
their fruit.
\par }{\PP \VS{21}“‘If
you walk
in hostility
against me and are not
willing
to obey
me,
I will increase
your affliction
seven
times according to your sins.
\VS{22}I will send
the
wild
animals
against you and they will bereave
you of your children, annihilate
your cattle,
and diminish
your population so that your roads
will become deserted.
\par }{\PP \VS{23}“‘If
in spite of these
things you do not
allow yourselves to be disciplined
and you walk
in hostility against me,
\VS{24}I myself
will also walk
in hostility
against you and strike
you seven
times on
account of your sins.
\VS{25}I will bring
on
you an avenging
sword,
a covenant
vengeance.
Although you will gather together
into
your cities,
I will send
pestilence
among
you and you will be given
into enemy
hands.
\VS{26}When I break
off your supply
of bread,
ten
women
will bake
your bread
in one
oven;
they will ration
your bread
by weight,
and you will eat
and not
be satisfied.
\par }{\PP \VS{27}“‘If
in spite of this
you do not
obey
me
but walk
in hostility against me,
\VS{28}I will walk
in hostile
rage
against you and I myself
will also discipline
you seven
times on
account of your sins.
\VS{29}You will eat
the flesh
of your sons
and the flesh
of your daughters.
\VS{30}I will destroy
your high places
and cut down
your incense altars,
and I will stack
your dead bodies
on
top of the lifeless bodies
of your idols.
I will abhor
you.
\VS{31}I will lay
your cities
waste
and make your sanctuaries
desolate,
and I will refuse
to smell
your soothing
aromas.
\VS{32}I myself
will make
the land
desolate
and your enemies
who live
in it will be appalled.
\VS{33}I will scatter
you among the nations
and unsheathe
the sword
after
you, so
your land
will become desolate
and your cities
will become
a waste.
\par }{\PP \VS{34}“‘Then
the land
will make up
for its
Sabbaths
all
the days
it lies desolate
while you
are in the land
of your enemies;
then
the land
will rest
and make up its
Sabbaths.
\VS{35}All
the days
of the desolation
it will have the rest
it did not
have on your Sabbaths
when you lived
on it.
\par }{\PP \VS{36}“‘As
for the ones who remain
among you, I will bring
despair
into their hearts
in the lands
of their enemies.
The
sound
of a blowing leaf
will pursue
them, and they will flee
as one who flees
the sword
and fall
down even though there is no
pursuer.
\VS{37}They will stumble
over each
other
as those who flee before
a sword,
though there is no
pursuer,
and there will be
no
one to take a stand
for you before
your enemies.
\VS{38}You will perish
among the nations;
the land
of your enemies
will consume you.
\par }{\SH Restoration through Confession and Repentance
\par }{\PP \VS{39}“‘As for the ones who remain
among you, they will rot
away because of their iniquity
in the lands
of your enemies,
and they will also
rot
away because of their ancestors’
iniquities
which are with them.
\VS{40}However, when
they confess
their iniquity
and their ancestors’
iniquity
which
they committed by trespassing
against me, by which
they also walked
in hostility against me
\VS{41}(and
I myself
will walk
in hostility
against them and bring
them into
the land
of their enemies), and then
their uncircumcised
hearts
become humbled
and they make up
for
their iniquity,
\VS{42}I will remember
my covenant
with Jacob
and also
my covenant
with Isaac
and also
my covenant
with Abraham,
and I will remember
the land.
\VS{43}The land
will be abandoned
by them
in order that it may make up for its Sabbaths
while it is made desolate
without them,
and they
will make up
for their iniquity
because
they have
rejected
my regulations
and have abhorred
my statutes.
\VS{44}In spite
of this,
however, when
they are in the land
of their enemies
I will not
reject
them and abhor
them
to make a complete
end of them, to break
my covenant
with
them, for
I am
the {\ND{Lord}}
their God.
\VS{45}I will remember
for them the covenant
with their ancestors
whom
I brought out
from the
land
of Egypt
in the sight
of the nations
to be
their God.
I am
the {\ND{Lord}}.’ ”
\par }{\SH Summary Colophon
\par }{\PP \VS{46}These
are the statutes,
regulations,
and instructions
which
the {\ND{Lord}}
established between
himself and the Israelites
at Mount
Sinai
through
Moses.

\par }\Chap{27}{\PP \VerseOne{1}The
{\ND{Lord}}
spoke
to
Moses:
\VS{2}“Speak
to
the Israelites
and tell
them,
‘When
a man
makes a special
votive offering
based on the conversion value
of persons
to the
{\ND{Lord}},
\VS{3}the conversion value
of the male
from twenty
years
old
up to
sixty
years
old
is
fifty
shekels
by the standard of the sanctuary
shekel.
\VS{4}If
the person is a female,
the conversion value
is thirty
shekels.
\VS{5}If
the person is from five
years
old up to
twenty
years
old, the conversion value
of the male
is twenty
shekels,
and for the female
ten
shekels.
\VS{6}If
the person is one month
old up to
five
years
old, the conversion value
of the male
is five
shekels
of silver,
and for the female
the conversion value
is three
shekels
of silver.
\VS{7}If
the person is from sixty
years
old and older,
if
he is a male
the conversion value
is fifteen
shekels,
and for the female
ten
shekels.
\VS{8}If
he is
too poor
to pay the conversion value,
he must stand
the person before
the priest
and the priest
will establish his conversion value;
according
to what the man who
made the vow
can afford,
the priest
will establish his conversion value.
\par }{\SH Redemption of Vowed Animals
\par }{\PP \VS{9}“‘If
what is vowed is a kind of animal
from
which
an offering
may be presented
to the
{\ND{Lord}}, anything
which
he gives
to the
{\ND{Lord}}
from
this kind of animal will be
holy.
\VS{10}He must not
replace
or
exchange
it, good
for bad
or
bad
for good,
and if
he does indeed exchange
one animal
for another animal,
then both the original animal and its substitute
will be
holy.
\VS{11}If
what is vowed
is an unclean
animal
from
which
an offering
must not
be presented
to the
{\ND{Lord}}, then he must stand
the animal
before
the priest,
\VS{12}and the priest
will establish its conversion value,
whether
good
or
bad.
According to the assessed conversion value
of the priest,
thus
it will be.
\VS{13}If,
however,
the person who
made the vow redeems
the animal, he must add
one fifth
to its conversion value.
\par }{\SH Redemption of Vowed Houses
\par }{\PP \VS{14}“‘If
a man
consecrates
his house
as holy
to the
{\ND{Lord}}, the priest
will establish its conversion
value, whether
good
or bad.
Just
as the
priest
establishes its conversion
value, thus
it will stand.
\VS{15}If
the one who consecrates
it redeems
his house,
he must add
to it one fifth
of its conversion value
in silver, and it will belong to him.
\par }{\SH Redemption of Vowed Fields
\par }{\PP \VS{16}“‘If
a man
consecrates
to the
{\ND{Lord}}
some of his own landed
property,
the conversion value
must be calculated in accordance
with the amount of seed
needed to sow it, a homer
of barley
seed
being priced at fifty
shekels
of silver.
\VS{17}If
he consecrates
his field
in the jubilee
year,
the conversion value
will stand,
\VS{18}but if
he consecrates
his field
after
the jubilee,
the priest
will calculate
the price
for him
according
to the years
that are left
until
the next jubilee
year,
and it will be deducted
from the conversion value.
\VS{19}If,
however, the one who consecrated
the field
redeems
it, he must add
to it one fifth
of the conversion price
and it will belong to him.
\VS{20}If
he does not
redeem
the field,
but
sells
the field
to someone
else,
he may never
redeem it.
\VS{21}When
it reverts
in the jubilee,
the field
will be holy
to the
{\ND{Lord}}
like a permanently dedicated
field;
it will become
the priest’s
property.
\par }{\PP \VS{22}“‘If
he consecrates
to the
{\ND{Lord}}
a field
he has purchased,
which
is not
part of his own landed
property,
\VS{23}the
priest
will calculate
for him the amount
of its conversion value
until
the jubilee
year,
and he must pay
the
conversion value
on that jubilee day
as something that
is holy
to the
{\ND{Lord}}.
\VS{24}In the jubilee
year
the field
will return
to the one from whom
he bought
it, the one to whom
it belongs as landed
property.
\VS{25}Every
conversion value
must be calculated by the standard of the sanctuary
shekel;
twenty
gerahs
to the shekel.
\par }{\SH Redemption of the Firstborn
\par }{\PP \VS{26}“‘Surely
no
man
may consecrate
a firstborn
that already belongs to the
{\ND{Lord}}
as
a firstborn
among the animals;
whether
it is an ox
or
a sheep,
it belongs to the
{\ND{Lord}}.
\VS{27}If,
however, it is among the unclean
animals,
he may ransom
it according to its conversion value
and must add
one fifth
to
it, but if
it is not
redeemed
it must be sold
according to its conversion value.
\par }{\SH Things Permanently Dedicated to the Lord
\par }{\PP \VS{28}“‘Surely
anything
which
a man
permanently dedicates
to the
{\ND{Lord}}
from all
that belongs
to him, whether from people,
animals,
or his landed
property,
must be neither sold
nor
redeemed;
anything
permanently dedicated
is
most
holy
to the
{\ND{Lord}}.
\VS{29}Any
human being who is permanently dedicated
must not be ransomed;
such a person
must
be put to death.
\par }{\SH Redemption of the Tithe
\par }{\PP \VS{30}“‘Any
tithe
of the land,
from the grain
of the land
or from the fruit
of the trees,
belongs to the
{\ND{Lord}}; it is
holy
to the
{\ND{Lord}}.
\VS{31}If
a man
redeems
part of his tithe,
however, he must add
one fifth to it.
\VS{32}All
the tithe
of herd
or flock,
everything
which
passes
under
the rod,
the tenth
one will be
holy
to the
{\ND{Lord}}.
\VS{33}The owner must not
examine
the animals to distinguish between
good
and bad,
and he must not
exchange
it. If,
however, he does exchange
it, both the original animal and its substitute
will be
holy.
It must not
be redeemed.’ ”
\par }{\SH Final Colophon
\par }{\PP \VS{34}These
are the commandments
which
the {\ND{Lord}}
commanded
Moses
to tell
the Israelites
at Mount
Sinai.
\par }