\NormalFont\ShortTitle{1 Chronicles}
{\MT 1 Chronicles

\par }\ChapOne{1}{\SH Adam’s Descendants
\par }{\PP \VerseOne{1}Adam,
Seth,
Enosh,
\VS{2}Kenan,
Mahalalel,
Jered,
\VS{3}Enoch,
Methuselah,
Lamech,
\VS{4}Noah,
Shem,
Ham,
and Japheth.
\par }{\SH Japheth’s Descendants
\par }{\PP \VS{5}The sons
of Japheth:
\par }{\PP Gomer,
Magog,
Madai,
Javan,
Tubal,
Meshech,
and Tiras.
\par }{\PP \VS{6}The sons
of Gomer:
\par }{\PP Ashkenaz,
Riphath,
and Togarmah.
\par }{\PP \VS{7}The sons
of Javan:
\par }{\PP Elishah,
Tarshish,
the Kittites,
and the Rodanites.
\par }{\SH Ham’s Descendants
\par }{\PP \VS{8}The sons
of Ham:
\par }{\PP Cush,
Mizraim,
Put,
and Canaan.
\par }{\PP \VS{9}The sons
of Cush:
\par }{\PP Seba,
Havilah,
Sabta,
Raamah,
and Sabteca.
\par }{\PP The sons of
Raamah:
\par }{\PP Sheba
and Dedan.
\par }{\PP \VS{10}Cush
was the father
of Nimrod,
who established himself as
a mighty warrior
on earth.
\par }{\PP \VS{11}Mizraim
was the father
of the Ludites,
Anamites,
Lehabites,
Naphtuhites,
\VS{12}Pathrusites,
Casluhites
(from whom
the Philistines
descended
), and the Caphtorites.
\par }{\PP \VS{13}Canaan
was the father
of Sidon
– his firstborn
– and Heth,
\VS{14}as well as the Jebusites,
Amorites,
Girgashites,
\VS{15}Hivites,
Arkites,
Sinites,
\VS{16}Arvadites,
Zemarites,
and Hamathites.
\par }{\SH Shem’s Descendants
\par }{\PP \VS{17}The sons
of Shem:
\par }{\PP Elam,
Asshur,
Arphaxad,
Lud,
and Aram.
\par }{\PP The sons of Aram:
\par }{\PP Uz,
Hul,
Gether,
and Meshech.
\par }{\PP \VS{18}Arphaxad
was the father
of Shelah,
and Shelah
was the father
of Eber.
\VS{19}Two
sons
were born
to Eber: the first
was named
Peleg,
for
during his lifetime
the earth
was divided;
his brother’s
name
was Joktan.
\par }{\PP \VS{20}Joktan
was the father
of Almodad,
Sheleph,
Hazarmaveth,
Jerah,
\VS{21}Hadoram,
Uzal,
Diklah,
\VS{22}Ebal,
Abimael,
Sheba,
\VS{23}Ophir,
Havilah,
and Jobab.
All
these
were the sons
of Joktan.
\par }{\PP \VS{24}Shem,
Arphaxad,
Shelah,
\VS{25}Eber,
Peleg,
Reu,
\VS{26}Serug,
Nahor,
Terah,
\VS{27}Abram
(that
is, Abraham).
\par }{\PP \VS{28}The sons
of Abraham:
\par }{\PP Isaac
and Ishmael.
\par }{\PP \VS{29}These
were their descendants:
\par }{\SH Ishmael’s Descendants
\par }{\PP Ishmael’s firstborn son was Nebaioth; the others were Kedar, Adbeel, Mibsam,
\VS{30}Mishma,
Dumah,
Massa,
Hadad,
Tema,
\VS{31}Jetur,
Naphish,
and Kedemah.
These
were
the sons
of Ishmael.
\par }{\SH Keturah’s Descendants
\par }{\PP \VS{32}The sons
to whom Keturah,
Abraham’s
concubine,
gave birth:
\par }{\PP Zimran,
Jokshan,
Medan,
Midian,
Ishbak,
Shuah.
\par }{\PP The sons
of Jokshan:
\par }{\PP Sheba
and Dedan.
\par }{\PP \VS{33}The sons
of Midian:
\par }{\PP Ephah,
Epher,
Hanoch,
Abida,
and Eldaah.
All
these
were the sons
of Keturah.
\par }{\SH Isaac’s Descendants
\par }{\PP \VS{34}Abraham
was the father
of Isaac.
The sons
of Isaac:
\par }{\PP Esau
and Israel.
\par }{\SH Esau’s Descendants
\par }{\PP \VS{35}The sons
of Esau:
\par }{\PP Eliphaz,
Reuel,
Jeush,
Jalam,
and Korah.
\par }{\PP \VS{36}The sons
of Eliphaz:
\par }{\PP Teman,
Omar,
Zephi,
Gatam,
Kenaz,
and (by Timna) Amalek.
\par }{\PP \VS{37}The sons
of Reuel:
\par }{\PP Nahath,
Zerah,
Shammah,
and Mizzah.
\par }{\SH The Descendants of Seir
\par }{\PP \VS{38}The sons
of Seir:
\par }{\PP Lotan,
Shobal,
Zibeon,
Anah,
Dishon,
Ezer,
and Dishan.
\par }{\PP \VS{39}The sons
of Lotan:
\par }{\PP Hori
and Homam.
(Timna
was Lotan’s
sister.)
\par }{\PP \VS{40}The sons
of Shobal:
\par }{\PP Alyan,
Manahath,
Ebal,
Shephi,
and Onam.
\par }{\PP The sons
of Zibeon:
\par }{\PP Aiah
and Anah.
\par }{\PP \VS{41}The son
of Anah:
\par }{\PP Dishon.
\par }{\PP The sons
of Dishon:
\par }{\PP Hamran,
Eshban,
Ithran,
and Keran.
\par }{\PP \VS{42}The sons
of Ezer:
\par }{\PP Bilhan,
Zaavan,
Jaakan.
\par }{\PP The sons of
Dishan:
\par }{\PP Uz
and Aran.
\par }{\SH Kings of Edom
\par }{\PP \VS{43}These
were the kings
who
reigned
in the land
of Edom
before
any king
ruled
over the Israelites:
\par }{\PP Bela
son
of Beor;
the name
of his city
was Dinhabah.
\par }{\PP \VS{44}When Bela
died,
Jobab
son
of Zerah
from Bozrah,
succeeded him.
\par }{\PP \VS{45}When Jobab
died,
Husham
from the land
of the Temanites
succeeded him.
\par }{\PP \VS{46}When Husham
died,
Hadad
son
of Bedad
succeeded him. He struck
down the Midianites
in the plains
of Moab;
the name
of his city
was Avith.
\par }{\PP \VS{47}When Hadad
died,
Samlah
from Masrekah
succeeded him.
\par }{\PP \VS{48}When Samlah
died,
Shaul
from Rehoboth
on the river
succeeded him.
\par }{\PP \VS{49}When Shaul
died,
Baal-Hanan
son
of Achbor
succeeded him.
\par }{\PP \VS{50}When Baal-Hanan
died,
Hadad
succeeded
him; the name
of his city
was Pai.
His wife
was Mehetabel,
daughter
of Matred,
daughter
of Me-Zahab.
\par }{\PP \VS{51}Hadad
died.
\par }{\SH Tribal Chiefs of Edom
\par }{\PP The tribal chiefs
of Edom
were:
\par }{\PP Timna,
Alvah,
Jetheth,
\VS{52}Oholibamah,
Elah,
Pinon,
\VS{53}Kenaz,
Teman,
Mibzar,
\VS{54}Magdiel,
Iram.
These
were the tribal chiefs
of Edom.

\par }\Chap{2}{\PP \VerseOne{1}These
were the sons
of Israel:
\par }{\PP Reuben,
Simeon,
Levi,
and Judah;
\par }{\PP Issachar
and Zebulun;
\par }{\PP \VS{2}Dan,
Joseph,
and Benjamin;
\par }{\PP Naphtali,
Gad,
and Asher.
\par }{\SH Judah’s Descendants
\par }{\PP \VS{3}The sons
of Judah:
\par }{\PP Er,
Onan,
and Shelah.
These three
were born
to him by Bathshua,
a Canaanite
woman.
Er,
Judah’s
firstborn,
displeased
the
{\ND{Lord}}, so the
{\ND{Lord}} killed him.
\par }{\PP \VS{4}Tamar,
Judah’s daughter-in-law,
bore
to him Perez
and Zerah.
Judah
had five
sons
in all.
\par }{\PP \VS{5}The sons
of Perez:
\par }{\PP Hezron
and Hamul.
\par }{\PP \VS{6}The sons
of Zerah:
\par }{\PP Zimri,
Ethan,
Heman,
Kalkol,
Dara –
five
in all.
\par }{\PP \VS{7}The son
of Carmi:
\par }{\PP Achan,
who brought the disaster
on Israel
when he stole
what was devoted to God.
\par }{\PP \VS{8}The son
of Ethan:
\par }{\PP Azariah.
\par }{\PP \VS{9}The sons
born
to Hezron:
\par }{\PP Jerahmeel,
Ram,
and Caleb.
\par }{\SH Ram’s Descendants
\par }{\PP \VS{10}Ram
was the father
of Amminadab,
and Amminadab
was the father
of Nahshon,
the tribal chief of Judah.
\VS{11}Nahshon
was the father
of Salma,
and Salma
was the father
of Boaz.
\VS{12}Boaz
was the father
of Obed,
and Obed
was the father
of Jesse.
\par }{\PP \VS{13}Jesse
was the father
of Eliab,
his firstborn;
Abinadab
was born second,
Shimea
third,
\VS{14}Nethanel
fourth,
Raddai
fifth,
\VS{15}Ozem
sixth,
David
seventh.
\VS{16}Their sisters
were Zeruiah
and Abigail.
Zeruiah’s
three
sons
were Abshai,
Joab,
and Asahel.
\VS{17}Abigail
bore
Amasa,
whose father
was Jether
the Ishmaelite.
\par }{\SH Caleb’s Descendants
\par }{\PP \VS{18}Caleb
son
of Hezron
fathered sons
by his wife
Azubah
(also known as Jerioth). Her sons
were Jesher,
Shobab,
and Ardon.
\VS{19}When Azubah
died,
Caleb
married
Ephrath,
who bore
him Hur.
\VS{20}Hur
was the father
of Uri,
and Uri
was the father
of Bezalel.
\par }{\PP \VS{21}Later
Hezron
had sexual relations with
the daughter
of Makir,
the father
of Gilead.
(He
had married
her
when he
was sixty
years
old.) She bore
him Segub.
\VS{22}Segub
was the father
of Jair,
who owned twenty-three
cities
in the land
of Gilead.
\VS{23}(Geshur
and Aram
captured
the towns
of Jair,
along with
Kenath
and its sixty
surrounding
towns.) All
these
were descendants
of Makir,
the father
of Gilead.
\par }{\PP \VS{24}After
Hezron’s
death,
Caleb
had sexual relations with Ephrath,
his father Hezron’s
widow,
and she bore
to him Ashhur
the father
of Tekoa.
\par }{\SH Jerahmeel’s Descendants
\par }{\PP \VS{25}The sons
of Jerahmeel,
Hezron’s
firstborn,
were Ram,
the firstborn,
Bunah,
Oren,
Ozem,
and Ahijah.
\VS{26}Jerahmeel
had another
wife
named
Atarah;
she
was Onam’s
mother.
\par }{\PP \VS{27}The sons
of Ram,
Jerahmeel’s
firstborn,
were Maaz,
Jamin,
and Eker.
\par }{\PP \VS{28}The sons
of Onam
were Shammai
and Jada.
\par }{\PP The sons
of Shammai:
\par }{\PP Nadab
and Abishur.
\par }{\PP \VS{29}Abishur’s
wife
was Abihail,
who bore
him Ahban
and Molid.
\par }{\PP \VS{30}The sons
of Nadab:
\par }{\PP Seled
and Appaim.
(Seled
died
without having sons.)
\par }{\PP \VS{31}The son
of Appaim:
\par }{\PP Ishi.
\par }{\PP The son
of Ishi:
\par }{\PP Sheshan.
\par }{\PP The son
of Sheshan:
\par }{\PP Ahlai.
\par }{\PP \VS{32}The sons
of Jada,
Shammai’s
brother:
\par }{\PP Jether
and Jonathan.
(Jether
died
without
having sons.)
\par }{\PP \VS{33}The sons
of Jonathan:
\par }{\PP Peleth
and Zaza.
\par }{\PP These
were the descendants
of Jerahmeel.
\par }{\PP \VS{34}Sheshan
had no
sons,
only
daughters.
Sheshan
had an Egyptian
servant
named
Jarha.
\VS{35}Sheshan
gave
his daughter
to his servant
Jarha
as a wife;
she bore
him Attai.
\par }{\PP \VS{36}Attai
was the father
of Nathan,
and Nathan
was the father
of Zabad.
\VS{37}Zabad
was the father
of Ephlal,
and Ephlal
was the father
of Obed.
\VS{38}Obed
was the father
of Jehu,
and Jehu
was the father
of Azariah.
\VS{39}Azariah
was the father
of Helez,
and Helez
was the father
of Eleasah.
\VS{40}Eleasah
was the father
of Sismai,
and Sismai
was the father
of Shallum.
\VS{41}Shallum
was the father
of Jekamiah,
and Jekamiah
was the father
of Elishama.
\par }{\SH More of Caleb’s Descendants
\par }{\PP \VS{42}The sons
of Caleb,
Jerahmeel’s
brother:
\par }{\PP His firstborn
Mesha,
the father
of Ziph,
and his second son
Mareshah,
the father
of Hebron.
\par }{\PP \VS{43}The sons
of Hebron:
\par }{\PP Korah,
Tappuah,
Rekem,
and Shema.
\par }{\PP \VS{44}Shema
was
the father
of Raham,
the father
of Jorkeam.
Rekem
was the father
of Shammai.
\VS{45}Shammai’s
son
was Maon,
who was the father
of Beth-Zur.
\par }{\PP \VS{46}Caleb’s
concubine
Ephah
bore
Haran,
Moza,
and Gazez.
Haran
was the father
of Gazez.
\par }{\PP \VS{47}The sons
of Jahdai:
\par }{\PP Regem,
Jotham,
Geshan,
Pelet,
Ephah,
and Shaaph.
\par }{\PP \VS{48}Caleb’s
concubine
Maacah
bore
Sheber
and Tirhanah.
\VS{49}She also bore
Shaaph
the father
of Madmannah
and Sheva
the father
of Machbenah
and Gibea.
Caleb’s
daughter
was Achsah.
\par }{\PP \VS{50}These
were the descendants
of Caleb.
\par }{\PP The sons
of Hur,
the firstborn
of Ephrath:
\par }{\PP Shobal,
the father
of Kiriath Jearim,
\VS{51}Salma,
the father
of Bethlehem,
Hareph,
the father
of Beth-Gader.
\par }{\PP \VS{52}The sons
of Shobal,
the father
of Kiriath Jearim,
were Haroeh,
half
of the Manahathites,
\VS{53}the clans
of Kiriath Jearim
– the Ithrites,
Puthites,
Shumathites,
and Mishraites.
(The Zorathites
and Eshtaolites
descended from
these groups.)
\par }{\PP \VS{54}The sons
of Salma:
\par }{\PP Bethlehem,
the Netophathites,
Atroth Beth-Joab,
half
the Manahathites,
the Zorites,
\VS{55}and the clans
of the scribes
who lived
in Jabez: the Tirathites,
Shimeathites,
and Sucathites.
These are the Kenites
who descended
from Hammath,
the father
of Beth-Rechab.

\par }\Chap{3}{\PP \VerseOne{1}These
were
the sons
of David
who
were born
to him in Hebron:
\par }{\PP The firstborn
was Amnon,
whose mother
was Ahinoam
from Jezreel;
\par }{\PP the second
was Daniel,
whose mother
was Abigail
from Carmel;
\par }{\PP \VS{2}the third
was Absalom
whose mother was Maacah,
daughter
of King
Talmai
of Geshur;
\par }{\PP the fourth
was Adonijah,
whose mother was Haggith;
\par }{\PP \VS{3}the fifth
was Shephatiah,
whose mother was Abital;
\par }{\PP the sixth
was Ithream,
whose mother
was Eglah.
\par }{\PP \VS{4}These six
were born
to David in Hebron,
where
he ruled
for seven
years
and six
months.
\par }{\PP He ruled
thirty-three
years
in Jerusalem.
\VS{5}These
were the sons born
to him in Jerusalem:
\par }{\PP Shimea,
Shobab,
Nathan,
and Solomon
– the mother of these four
was Bathsheba
the daughter
of Ammiel.
\par }{\PP \VS{6}The other nine were Ibhar,
Elishua,
Elpelet,
\VS{7}Nogah,
Nepheg,
Japhia,
\VS{8}Elishama,
Eliada,
and Eliphelet.
\par }{\PP \VS{9}These were all
the sons
of David,
not counting
the sons
of his concubines.
Tamar
was their sister.
\par }{\SH Solomon’s Descendants
\par }{\PP \VS{10}Solomon’s
son
was Rehoboam,
\par }{\PP followed by Abijah
his son,
\par }{\PP Asa
his son,
\par }{\PP Jehoshaphat
his son,
\par }{\PP \VS{11}Joram
his son,
\par }{\PP Ahaziah
his son,
\par }{\PP Joash
his son,
\par }{\PP \VS{12}Amaziah
his son,
\par }{\PP Azariah
his son,
\par }{\PP Jotham
his son,
\par }{\PP \VS{13}Ahaz
his son,
\par }{\PP Hezekiah
his son,
\par }{\PP Manasseh
his son,
\par }{\PP \VS{14}Amon
his son,
\par }{\PP Josiah
his son.
\par }{\PP \VS{15}The sons
of Josiah:
\par }{\PP Johanan
was the firstborn;
Jehoiakim
was born second;
Zedekiah
third;
and Shallum
fourth.
\par }{\PP \VS{16}The sons
of Jehoiakim:
\par }{\PP his son
Jehoiachin
and his son
Zedekiah.
\par }{\PP \VS{17}The sons
of Jehoiachin
the exile:
\par }{\PP Shealtiel
his son,
\VS{18}Malkiram,
Pedaiah,
Shenazzar,
Jekamiah,
Hoshama,
and Nedabiah.
\par }{\PP \VS{19}The sons
of Pedaiah:
\par }{\PP Zerubbabel
and Shimei.
\par }{\PP The sons
of Zerubbabel:
\par }{\PP Meshullam
and Hananiah.
Shelomith
was their sister.
\par }{\PP \VS{20}The five
others were Hashubah,
Ohel,
Berechiah,
Hasadiah,
and Jushab-Hesed.
\par }{\PP \VS{21}The descendants
of Hananiah:
\par }{\PP Pelatiah,
Jeshaiah,
the sons
of Rephaiah,
of Arnan,
of Obadiah,
and of Shecaniah.
\par }{\PP \VS{22}The descendants
of Shecaniah:
\par }{\PP Shemaiah
and his sons: Hattush,
Igal,
Bariah,
Neariah,
and Shaphat
– six in all.
\par }{\PP \VS{23}The sons
of Neariah:
\par }{\PP Elioenai,
Hizkiah,
and Azrikam
– three in all.
\par }{\PP \VS{24}The sons
of Elioenai:
\par }{\PP Hodaviah,
Eliashib,
Pelaiah,
Akkub,
Johanan,
Delaiah,
and Anani
– seven in all.

\par }\Chap{4}{\PP \VerseOne{1}The descendants
of Judah:
\par }{\PP Perez,
Hezron,
Carmi,
Hur,
and Shobal.
\par }{\PP \VS{2}Reaiah
the son
of Shobal
was the father
of Jahath,
and Jahath
was the father
of Ahumai
and Lahad.
These
were the clans
of the Zorathites.
\par }{\PP \VS{3}These
were the sons
of Etam:
\par }{\PP Jezreel,
Ishma,
and Idbash.
Their sister
was Hazzelelponi.
\par }{\PP \VS{4}Penuel
was the father
of Gedor,
and Ezer
was the father
of Hushah.
These
were the descendants
of Hur,
the firstborn
of Ephrathah
and the father
of Bethlehem.
\par }{\PP \VS{5}Ashhur
the father
of Tekoa
had two
wives,
Helah
and Naarah.
\VS{6}Naarah
bore
him Ahuzzam,
Hepher,
Temeni,
and Haahashtari.
These
were the sons
of Naarah.
\VS{7}The sons
of Helah: Zereth,
Zohar,
Ethnan,
\VS{8}and Koz,
who was the father
of Anub,
Hazzobebah,
and the clans
of Aharhel
the son
of Harum.
\par }{\PP \VS{9}Jabez
was
more respected
than his brothers.
His mother
had named
him Jabez,
for she said,
“I experienced pain
when
I gave birth to him.”
\VS{10}Jabez
called
out to the God
of Israel,
“If
only you would greatly
bless
me and expand my territory! May
your hand
be with
me! Keep
me from harm
so I might not
endure pain!” God
answered
his prayer.
\par }{\PP \VS{11}Kelub,
the brother
of Shuhah,
was the father
of Mehir,
who was
the father
of Eshton.
\VS{12}Eshton
was the father
of Beth-Rapha,
Paseah,
and Tehinnah,
the father
of Ir Nahash.
These
were the men
of Recah.
\par }{\PP \VS{13}The sons
of Kenaz:
\par }{\PP Othniel
and Seraiah.
\par }{\PP The sons
of Othniel:
\par }{\PP Hathath and Meonothai.
\VS{14}Meonothai
was the father
of Ophrah.
\par }{\PP Seraiah
was the father
of Joab,
the father
of those who live in Ge
Harashim,
who were craftsmen.
\par }{\PP \VS{15}The sons
of Caleb
son
of Jephunneh:
\par }{\PP Iru,
Elah,
and Naam.
\par }{\PP The son
of Elah:
\par }{\PP Kenaz.
\par }{\PP \VS{16}The sons
of Jehallelel:
\par }{\PP Ziph,
Ziphah,
Tiria,
and Asarel.
\par }{\PP \VS{17}The sons
of Ezrah:
\par }{\PP Jether,
Mered,
Epher,
and Jalon.
\par }{\PP Mered’s
wife Bithiah gave birth to Miriam,
Shammai,
and Ishbah,
the father
of Eshtemoa.
\VS{18}(His Judahite
wife
gave birth
to Jered
the father
of Gedor,
Heber
the father
of Soco,
and Jekuthiel
the father
of Zanoah.) These
were the sons
of Pharaoh’s
daughter
Bithiah,
whom
Mered
married.
\par }{\PP \VS{19}The sons
of Hodiah’s
wife,
the sister
of Naham:
\par }{\PP the father
of Keilah
the Garmite,
and Eshtemoa
the Maacathite.
\par }{\PP \VS{20}The sons
of Shimon:
\par }{\PP Amnon,
Rinnah,
Ben-Hanan,
and Tilon.
\par }{\PP The descendants
of Ishi:
\par }{\PP Zoheth
and Ben Zoheth.
\par }{\PP \VS{21}The sons
of Shelah
son
of Judah:
\par }{\PP Er
the father
of Lecah,
Laadah
the father
of Mareshah,
the clans
of the linen
workers
at Beth-Ashbea,
\VS{22}Jokim,
the men
of Cozeba,
and Joash
and Saraph,
both of whom
ruled in Moab
and Jashubi Lehem.
(This information
is from ancient records.)
\VS{23}They
were the potters
who lived
in Netaim
and Gederah;
they lived
there
and worked
for the king.
\par }{\SH Simeon’s Descendants
\par }{\PP \VS{24}The descendants
of Simeon:
\par }{\PP Nemuel,
Jamin,
Jarib,
Zerah,
Shaul,
\VS{25}his son
Shallum,
his son
Mibsam,
and his son
Mishma.
\par }{\PP \VS{26}The descendants
of Mishma:
\par }{\PP His son Hammuel,
his son
Zaccur,
and his son
Shimei.
\par }{\PP \VS{27}Shimei
had sixteen
sons
and six
daughters.
But his brothers
did not
have many
sons,
so their whole
clan
was not
as numerous
as
the sons
of Judah.
\VS{28}They lived
in Beer Sheba,
Moladah,
Hazar Shual,
\VS{29}Bilhah,
Ezem,
Tolad,
\VS{30}Bethuel,
Hormah,
Ziklag,
\VS{31}Beth Marcaboth,
Hazar Susim,
Beth Biri,
and Shaaraim.
These
were their towns
until
the reign
of David.
\VS{32}Their settlements
also included Etam,
Ain,
Rimmon,
Tochen,
and Ashan
– five
towns.
\VS{33}They also lived
in all
the settlements
that
surrounded
these
towns
as far
as Baal.
These were their settlements;
they kept genealogical records.
\par }{\PP \VS{34}Their clan leaders were:

\par }{\PP Meshobab,
Jamlech,
Joshah
son
of Amaziah,
\VS{35}Joel,
Jehu
son
of Joshibiah
(son
of Seraiah,
son
of Asiel),
\VS{36}Eleoenai,
Jaakobah,
Jeshohaiah,
Asaiah,
Adiel,
Jesimiel,
Benaiah,
\VS{37}Ziza
son
of Shipi
(son
of Allon,
son
of Jedaiah,
son
of Shimri,
son
of Shemaiah).
\VS{38}These
who are named
above were the leaders
of their clans.
\par }{\PP Their extended families
increased
greatly in numbers.
\VS{39}They went
to the entrance
of Gedor,
to
the east
of the valley,
looking
for pasture
for their sheep.
\VS{40}They found
fertile
and rich
pasture;
the land
was very broad,
undisturbed
and peaceful.
Indeed
some
Hamites
had been living
there
prior to that.
\VS{41}The men whose names
are listed
came
during the time
of King
Hezekiah
of Judah
and attacked
the Hamites’ settlements,
as well as the Meunites
they discovered
there,
and they wiped
them out
to
this
very day.
They dispossessed
them, for
they found pasture
for their sheep
there.
\VS{42}Five
hundred
men
of Simeon,
led by Pelatiah,
Neariah,
Rephaiah,
and Uzziel,
the sons
of Ishi,
went
to the hill country
of Seir
\VS{43}and defeated
the rest
of the Amalekite
refugees; they live
there
to
this
very
day.

\par }\Chap{5}{\PP \VerseOne{1}The sons
of Reuben,
Israel’s
firstborn –
\par }{\PP (Now
he was the firstborn,
but when
he defiled
his father’s
bed,
his rights as firstborn
were given
to the sons
of Joseph,
Israel’s
son.
So Reuben is not
listed
as firstborn
in the genealogical records.
\VS{2}Though
Judah
was the strongest
among his brothers
and a leader
descended from
him, the right of the firstborn
belonged to Joseph.)
\par }{\PP \VS{3}The sons
of Reuben,
Israel’s
firstborn:
\par }{\PP Hanoch,
Pallu,
Hezron,
and Carmi.
\par }{\PP \VS{4}The descendants
of Joel:
\par }{\PP His son
Shemaiah,
his son
Gog,
his son
Shimei,
\VS{5}his son
Micah,
his son
Reaiah,
his son
Baal,
\VS{6}and his son
Beerah,
whom
King
Tiglath-pileser
of Assyria
carried into exile.
Beerah was the tribal leader
of Reuben.
\par }{\PP \VS{7}His brothers
by their clans,
as listed in their genealogical records:
\par }{\PP The leader
Jeiel,
Zechariah,
\VS{8}and Bela
son
of Azaz,
son
of Shema,
son
of Joel.
\par }{\PP They lived
in Aroer
as far
as Nebo
and Baal Meon.
\VS{9}In the east
they settled
as far
as the entrance
to the desert
that stretches to
the Euphrates
River,
for
their cattle
had increased
in numbers in the land
of Gilead.
\VS{10}During the time
of Saul
they attacked
the Hagrites
and defeated
them. They took over
their territory
in
the entire
eastern
region of Gilead.
\par }{\SH Gad’s Descendants
\par }{\PP \VS{11}The descendants
of Gad
lived
near
them in the land
of Bashan,
as far
as Salecah.
\par }{\PP \VS{12}They included Joel
the leader,
Shapham
the second
in command, Janai,
and Shaphat
in Bashan.
\VS{13}Their relatives,
listed according to their families,
included Michael,
Meshullam,
Sheba,
Jorai,
Jacan,
Zia,
and Eber
– seven in all.
\par }{\PP \VS{14}These
were the sons
of Abihail
son
of Huri,
son
of Jaroah,
son
of Gilead,
son
of Michael,
son
of Jeshishai,
son
of Jahdo,
son
of Buz.
\VS{15}Ahi
son
of Abdiel,
son
of Guni,
was the leader
of the family.
\VS{16}They lived
in Gilead,
in Bashan
and its surrounding
settlements, and in the pasturelands
of Sharon
to
their very borders.
\VS{17}All
of them were listed in the genealogical records
in the time
of King
Jotham
of Judah
and in the time
of King
Jeroboam
of Israel.
\par }{\PP \VS{18}The Reubenites,
Gadites,
and the half-tribe
of Manasseh
had 44,760
men
in their combined armies, warriors
who carried
shields
and swords,
were equipped
with bows,
and were trained
for
war.
\VS{19}They attacked
the Hagrites,
Jetur,
Naphish,
and Nodab.
\VS{20}They received divine help
in fighting them, and the Hagrites
and all
their allies
were handed
over to them. They cried
out to God
during the battle;
he responded to their prayers
because
they trusted in him.
\VS{21}They seized
the Hagrites’ animals,
including 50,000
camels,
250,000
sheep,
and 2,000
donkeys.
They also took captive 100,000
people.
\VS{22}Because
God
fought for them, they killed
many
of the enemy. They dispossessed
the Hagrites
and lived
in their land
until
the exile.
\par }{\SH The Half-Tribe of Manasseh
\par }{\PP \VS{23}The half-tribe
of Manasseh
settled
in the land
from Bashan
as far
as Baal Hermon,
Senir,
and Mount
Hermon.
They
grew
in number.
\par }{\PP \VS{24}These
were the leaders
of their families:
\par }{\PP Epher,
Ishi,
Eliel,
Azriel,
Jeremiah,
Hodaviah,
and Jahdiel.
They were skilled
warriors,
men
of reputation,
and leaders
of their families.
\VS{25}But they were unfaithful
to the God
of their ancestors
and worshiped
instead the gods
of the native
peoples
whom
God
had destroyed
before them.
\VS{26}So the God
of Israel
stirred
up King
Pul
of Assyria
(that is, King
Tiglath-pileser
of Assyria), and he carried
away the Reubenites,
Gadites,
and half-tribe
of Manasseh
and took them to Halah,
Habor,
Hara,
and the river
of Gozan,
where they remain to this
very day.

\par }\Chap{6}{\PP \VerseOne{1} The sons
of Levi:
\par }{\PP Gershon,
Kohath,
and Merari.
\par }{\PP \VS{2}The sons
of Kohath:
\par }{\PP Amram,
Izhar,
Hebron,
and Uzziel.
\par }{\PP \VS{3}The children
of Amram:
\par }{\PP Aaron,
Moses,
and Miriam.
\par }{\PP The sons
of Aaron:
\par }{\PP Nadab,
Abihu,
Eleazar,
and Ithamar.
\par }{\PP \VS{4}Eleazar
was the father
of Phinehas,
and Phinehas
was the father
of Abishua.
\VS{5}Abishua
was the father
of Bukki,
and Bukki
was the father
of Uzzi.
\VS{6}Uzzi
was the father
of Zerahiah,
and Zerahiah
was the father
of Meraioth.
\VS{7}Meraioth
was the father
of Amariah,
and Amariah
was the father
of Ahitub.
\VS{8}Ahitub
was the father
of Zadok,
and Zadok
was the father
of Ahimaaz.
\VS{9}Ahimaaz
was the father
of Azariah,
and Azariah
was the father
of Johanan.
\VS{10}Johanan
was the father
of Azariah,
who
served
as a priest in the temple
Solomon
built
in Jerusalem.
\VS{11}Azariah
was the father
of Amariah,
and Amariah
was the father
of Ahitub.
\VS{12}Ahitub
was the father
of Zadok,
and Zadok
was the father
of Shallum.
\VS{13}Shallum
was the father
of Hilkiah,
and Hilkiah
was the father
of Azariah.
\VS{14}Azariah
was the father
of Seraiah,
and Seraiah
was the father
of Jehozadak.
\VS{15}Jehozadak
went
into exile
when the
{\ND{Lord}}
sent the people of Judah
and Jerusalem
into exile by the hand
of Nebuchadnezzar.
\par }{\PP \VS{16}The sons
of Levi:
\par }{\PP Gershom,
Kohath,
and Merari.
\par }{\PP \VS{17}These
are the names
of the sons
Gershom:
\par }{\PP Libni
and Shimei.
\par }{\PP \VS{18}The sons
of Kohath:
\par }{\PP Amram,
Izhar,
Hebron,
and Uzziel.
\par }{\PP \VS{19}The sons
of Merari:
\par }{\PP Mahli
and Mushi.
\par }{\PP These
are the clans
of the Levites
by their families.
\par }{\PP \VS{20}To Gershom:
\par }{\PP His son
Libni,
his son
Jahath,
his son
Zimmah,
\VS{21}his son
Joah,
his son
Iddo,
his son
Zerah,
and his son
Jeatherai.
\par }{\PP \VS{22}The sons
of Kohath:
\par }{\PP His son
Amminadab,
his son
Korah,
his son
Assir,
\VS{23}his son
Elkanah,
his son
Ebiasaph,
his son
Assir,
\VS{24}his son Tahath,
his son
Uriel,
his son
Uzziah,
and his son
Shaul.
\par }{\PP \VS{25}The sons
of Elkanah:
\par }{\PP Amasai,
Ahimoth,
\VS{26}his son
Elkanah,
his son
Zophai,
his son
Nahath,
\VS{27}his son
Eliab,
his son
Jeroham,
and his son
Elkanah.
\par }{\PP \VS{28}The sons
of Samuel:
\par }{\PP Joel
the firstborn
and Abijah the second oldest.
\par }{\PP \VS{29}The descendants
of Merari:
\par }{\PP Mahli,
his son
Libni,
his son
Shimei,
his son
Uzzah,
\VS{30}his son
Shimea,
his son
Haggiah,
and his son
Asaiah.
\par }{\SH Professional Musicians
\par }{\PP \VS{31}These
are the men David
put in charge
of music
in the
{\ND{Lord}}’s
sanctuary,
after the ark
was placed there.
\VS{32}They performed
music
before
the sanctuary
of the meeting
tent
until
Solomon
built
the
{\ND{Lord}}’s
temple
in Jerusalem.
They carried out their tasks
according to regulations.
\par }{\PP \VS{33}These
are the ones who served
along with their sons:
\par }{\PP From the Kohathites:
\par }{\PP Heman
the musician,
son
of Joel,
son
of Samuel,
\VS{34}son
of Elkanah,
son
of Jeroham,
son
of Eliel,
son
of Toah,
\VS{35}son
of Zuph,
son
of Elkanah,
son
of Mahath,
son
of Amasai,
\VS{36}son
of Elkanah,
son
of Joel,
son
of Azariah,
son
of Zephaniah,
\VS{37}son
of Tahath,
son
of Assir,
son
of Ebiasaph,
son
of Korah,
\VS{38}son
of Izhar,
son
of Kohath,
son
of Levi,
son
of Israel.
\par }{\PP \VS{39}Serving beside
him
was his fellow
Levite Asaph,
son
of Berechiah,
son
of Shimea,
\VS{40}son
of Michael,
son
of Baaseiah,
son
of Malkijah,
\VS{41}son
of Ethni,
son
of Zerah,
son
of Adaiah,
\VS{42}son
of Ethan,
son
of Zimmah,
son
of Shimei,
\VS{43}son
of Jahath,
son
of Gershom,
son
of Levi.
\par }{\PP \VS{44}Serving beside
them were their fellow
Levites, the descendants
of Merari,
led by Ethan,
son
of Kishi,
son
of Abdi,
son
of Malluch,
\VS{45}son
of Hashabiah,
son
of Amaziah,
son
of Hilkiah,
\VS{46}son
of Amzi,
son
of Bani,
son
of Shemer,
\VS{47}son
of Mahli,
son
of Mushi,
son
of Merari,
son
of Levi.
\par }{\PP \VS{48}The rest of their fellow
Levites
were assigned
to perform the remaining
tasks
at God’s
sanctuary.
\VS{49}But Aaron
and his descendants
offered
sacrifices on
the altar
for burnt offerings
and on
the altar
for incense
as they had been assigned
to do in the most holy
sanctuary.
They made atonement
for Israel,
just
as God’s
servant
Moses
had ordered.
\par }{\PP \VS{50}These
were the descendants
of Aaron:
\par }{\PP His son
Eleazar,
his son
Phinehas,
his son
Abishua,
\VS{51}his son
Bukki,
his son
Uzzi,
his son
Zerahiah,
\VS{52}his son Meraioth,
his son
Amariah,
his son
Ahitub,
\VS{53}his son
Zadok,
and his son
Ahimaaz.
\par }{\PP \VS{54}These
were the areas where
Aaron’s
descendants
lived:

\par }{\PP The following belonged to the Kohathite
clan,
for
they
received the first allotment:
\par }{\PP \VS{55}They
were allotted
Hebron
in the territory
of Judah,
as well as its surrounding
pasturelands.
\VS{56}(But the city’s land
and nearby towns
were allotted
to Caleb
son
of Jephunneh.)
\VS{57}The descendants
of Aaron
were also allotted
as cities
of refuge
Hebron,
Libnah
and its pasturelands,
Jattir,
Eshtemoa
and its pasturelands,
\VS{58}Hilez
and its pasturelands,
Debir
and its pasturelands,
\VS{59}Ashan
and its pasturelands,
and Beth Shemesh
and its pasturelands.
\par }{\PP \VS{60}Within the territory of the tribe
of Benjamin
they were allotted Geba
and its pasturelands,
Alemeth
and its pasturelands,
and Anathoth
and its
pasturelands.
Their clans
were allotted
thirteen
cities
in all.
\VS{61}The rest of Kohath’s
descendants
were allotted
ten
cities
in the territory of the half-tribe
of Manasseh.
\par }{\PP \VS{62}The clans
of
Gershom’s
descendants
received thirteen
cities
within the territory of the tribes
of Issachar,
Asher,
Naphtali,
and Manasseh
(in Bashan).
\par }{\PP \VS{63}The clans
of
Merari’s
descendants
were allotted
twelve
cities
within the territory of the tribes
of Reuben,
Gad,
and Zebulun.
\par }{\PP \VS{64}So the Israelites
gave
to the Levites
these cities
and their pasturelands.
\VS{65}They allotted
these
previously named
cities
from the territory of the tribes
of Judah,
Simeon,
and Benjamin.
\par }{\PP \VS{66}The clans
of Kohath’s
descendants
also received territory
within the tribe
of Ephraim.
\VS{67}They were allotted
as cities
of refuge
Shechem
and its
pasturelands
(in the hill country
of Ephraim), Gezer
and its
pasturelands,
\VS{68}Jokmeam
and its pasturelands,
Beth Horon
and its pasturelands,
\VS{69}Aijalon
and its pasturelands,
and Gath Rimmon
and its pasturelands.
\par }{\PP \VS{70}Within the territory of the half-tribe
of Manasseh,
the rest of Kohath’s
descendants
received
Aner
and its pasturelands
and Bileam
and its pasturelands.
\par }{\PP \VS{71}The following belonged to Gershom’s
descendants:
\par }{\PP Within the territory
of the half-tribe
of Manasseh: Golan
in Bashan
and its pasturelands
and Ashtaroth
and its
pasturelands.
\par }{\PP \VS{72}Within the territory of the tribe
of Issachar: Kedesh
and its pasturelands,
Daberath
and its pasturelands,
\VS{73}Ramoth
and its
pasturelands,
and Anem
and its pasturelands.
\par }{\PP \VS{74}Within the territory of the tribe
of Asher: Mashal
and its pasturelands,
Abdon
and its pasturelands,
\VS{75}Hukok
and its pasturelands,
and Rehob
and its pasturelands.
\par }{\PP \VS{76}Within the territory of the tribe
of Naphtali: Kedesh
in Galilee
and its
pasturelands,
Hammon
and its
pasturelands,
and Kiriathaim
and its
pasturelands.
\par }{\PP \VS{77}The following belonged to the rest
of Merari’s
descendants:
\par }{\PP Within the territory of the tribe
of Zebulun: Rimmono
and its pasturelands,
and Tabor
and its pasturelands.
\par }{\PP \VS{78}Within the territory of the tribe
of Reuben
across
the Jordan River
east
of Jericho: Bezer
in the desert
and its pasturelands,
Jahzah
and its pasturelands,
\VS{79}Kedemoth
and its pasturelands,
and Mephaath
and its pasturelands.
\par }{\PP \VS{80}Within the territory of the tribe
of Gad: Ramoth
in Gilead
and its pasturelands,
Mahanaim
and its pasturelands,
\VS{81}Heshbon
and its
pasturelands,
and Jazer
and its pasturelands.

\par }\Chap{7}{\PP \VerseOne{1}The sons
of Issachar:
\par }{\PP Tola,
Puah,
Jashub,
and Shimron
– four in all.
\par }{\PP \VS{2}The sons
of Tola:
\par }{\PP Uzzi,
Rephaiah,
Jeriel,
Jahmai,
Jibsam,
and Samuel.
They were leaders
of their families.
In the time
of David
there were 22,600
warriors
listed
in Tola’s
genealogical records.
\par }{\PP \VS{3}The son
of Uzzi:
\par }{\PP Izrachiah.
\par }{\PP The sons
of Izrahiah:
\par }{\PP Michael,
Obadiah,
Joel,
and Isshiah.
All
five
were leaders.
\par }{\PP \VS{4}According to
the genealogical records
of their families,
they had 36,000
warriors
available for battle,
for
they had numerous
wives
and sons.
\VS{5}Altogether
the genealogical records
of the clans
of Issachar
listed
87,000
warriors.
\par }{\SH Benjamin’s Descendants
\par }{\PP \VS{6}The sons of Benjamin:
\par }{\PP Bela,
Beker,
and Jediael
– three in all.
\par }{\PP \VS{7}The sons
of Bela:
\par }{\PP Ezbon,
Uzzi,
Uzziel,
Jerimoth,
and Iri.
The five
of them were leaders
of their families.
There were 22,034
warriors
listed in their genealogical records.
\par }{\PP \VS{8}The sons
of Beker:
\par }{\PP Zemirah,
Joash,
Eliezer,
Elioenai,
Omri,
Jeremoth,
Abijah,
Anathoth,
and Alameth.
All
these
were the sons
of Beker.
\VS{9}There were 20,200
family
leaders
and warriors
listed
in their genealogical records.
\par }{\PP \VS{10}The son
of Jediael:
\par }{\PP Bilhan.
\par }{\PP The sons
of Bilhan:
\par }{\PP Jeush,
Benjamin,
Ehud,
Kenaanah,
Zethan,
Tarshish,
and Ahishahar.
\VS{11}All
these
were the sons
of Jediael.
Listed in their genealogical records were 17,200
family
leaders
and warriors
who were capable of marching out
to battle.
\par }{\PP \VS{12}The Shuppites
and Huppites
were descendants
of Ir;
the Hushites
were descendants
of Aher.
\par }{\SH Naphtali’s Descendants
\par }{\PP \VS{13}The sons
of Naphtali:
\par }{\PP Jahziel,
Guni,
Jezer,
and Shallum –
sons
of Bilhah.
\par }{\SH Manasseh’s Descendants
\par }{\PP \VS{14}The sons
of Manasseh:
\par }{\PP Asriel,
who
was born
to Manasseh’s Aramean
concubine.
She also gave birth
to Makir
the father
of Gilead.
\VS{15}Now Makir
married
a wife
from the Huppites
and Shuppites.
(His sister’s
name
was Maacah.)
\par }{\PP Zelophehad
was Manasseh’s second
son; he
had only
daughters.
\par }{\PP \VS{16}Maacah,
Makir’s
wife,
gave birth
to a son,
whom she named
Peresh.
His brother
was Sheresh,
and his sons
were Ulam
and Rekem.
\par }{\PP \VS{17}The son
of Ulam:
\par }{\PP Bedan.
\par }{\PP These
were the sons
of Gilead,
son
of Makir,
son
of Manasseh.
\VS{18}His sister
Hammoleketh
gave birth
to Ishhod,
Abiezer,
and Mahlah.
\par }{\PP \VS{19}The sons
of Shemida
were
Ahian,
Shechem,
Likhi,
and Aniam.
\par }{\SH Ephraim’s Descendants
\par }{\PP \VS{20}The descendants
of Ephraim:
\par }{\PP Shuthelah,
his son
Bered,
his son
Tahath,
his son
Eleadah,
his son
Tahath,
\VS{21}his son
Zabad,
his son
Shuthelah
\par }{\PP (Ezer
and Elead
were killed
by the men
of Gath,
who were natives
of the land,
when
they went down
to steal
their cattle.
\VS{22}Their father
Ephraim
mourned
for them many
days
and his brothers
came
to console him.
\VS{23}He had sexual relations with
his wife;
she became pregnant
and gave birth
to a son.
Ephraim named
him Beriah
because
tragedy
had come
to his family.
\VS{24}His daughter
was Sheerah,
who built
Lower
and Upper
Beth Horon,
as well as Uzzen Sheerah),
\par }{\PP \VS{25}his son
Rephah,
his son
Resheph,
his son
Telah,
his son
Tahan,
\VS{26}his son
Ladan,
his son
Ammihud,
his son
Elishama,
\VS{27}his son
Nun,
and his son
Joshua.
\par }{\PP \VS{28}Their property
and settlements
included Bethel
and its surrounding towns,
Naaran
to the east,
Gezer
and its surrounding towns
to the west,
and Shechem
and its surrounding towns
as far
as Ayyah
and its surrounding towns.
\VS{29}On
the border
of Manasseh’s
territory were Beth-Shean
and its surrounding towns,
Taanach
and its surrounding towns,
Megiddo
and its surrounding towns,
and Dor
and its surrounding towns.
The descendants
of Joseph,
Israel’s
son,
lived here.
\par }{\SH Asher’s Descendants
\par }{\PP \VS{30}The sons
of Asher:
\par }{\PP Imnah,
Ishvah,
Ishvi,
and Beriah.
Serah
was their sister.
\par }{\PP \VS{31}The sons
of Beriah:
\par }{\PP Heber
and Malkiel,
who was the father
of Birzaith.
\par }{\PP \VS{32}Heber
was the father
of Japhlet,
Shomer,
Hotham,
and Shua
their sister.
\par }{\PP \VS{33}The sons
of Japhlet:
\par }{\PP Pasach,
Bimhal,
and Ashvath.
These
were Japhlet’s
sons.
\par }{\PP \VS{34}The sons
of his brother
Shemer:
\par }{\PP Rohgah,
Hubbah,
and Aram.
\par }{\PP \VS{35}The sons
of his brother
Helem:
\par }{\PP Zophah,
Imna,
Shelesh,
and Amal.
\par }{\PP \VS{36}The sons
of Zophah:
\par }{\PP Suah,
Harnepher,
Shual,
Beri,
Imrah,
\VS{37}Bezer,
Hod,
Shamma,
Shilshah,
Ithran,
and Beera.
\par }{\PP \VS{38}The sons
of Jether:
\par }{\PP Jephunneh,
Pispah,
and Ara.
\par }{\PP \VS{39}The sons
of Ulla:
\par }{\PP Arah,
Hanniel,
and Rizia.
\par }{\PP \VS{40}All
these
were the descendants
of Asher.
They were the leaders
of their families,
the most capable
men, who were warriors
and served as head
chiefs.
There were
26,000
warriors
listed in their genealogical records
as capable of doing battle.

\par }\Chap{8}{\PP \VerseOne{1}Benjamin
was the father
of Bela,
his firstborn;
Ashbel
was born second,
Aharah
third,
\VS{2}Nohah
fourth,
and Rapha
fifth.
\par }{\PP \VS{3}Bela’s
sons
were
Addar,
Gera,
Abihud,
\VS{4}Abishua,
Naaman,
Ahoah,
\VS{5}Gera,
Shephuphan,
and Huram.
\par }{\PP \VS{6}These
were the descendants
of Ehud
who were leaders
of the families
living in
Geba
who were forced to move
to
Manahath:
\VS{7}Naaman,
Ahijah,
and Gera,
who moved them. Gera
was the father
of Uzzah
and Ahihud.
\par }{\PP \VS{8}Shaharaim
fathered
sons in Moab
after
he divorced
his wives
Hushim
and Baara.
\VS{9}By his wife
Hodesh
he fathered
Jobab,
Zibia,
Mesha,
Malkam,
\VS{10}Jeuz,
Sakia,
and Mirmah.
These
were his sons;
they were family
leaders.
\VS{11}By Hushim
he fathered
Abitub
and Elpaal.
\par }{\PP \VS{12}The sons
of Elpaal:
\par }{\PP Eber,
Misham,
Shemed
(who built
Ono
and Lod,
as well as its surrounding towns),
\VS{13}Beriah,
and Shema.
They
were leaders
of the families
living
in Aijalon
and chased
out the inhabitants
of Gath.
\par }{\PP \VS{14}Ahio,
Shashak,
Jeremoth,
\VS{15}Zebadiah,
Arad,
Eder,
\VS{16}Michael,
Ishpah,
and Joha
were the sons
of Beriah.
\par }{\PP \VS{17}Zebadiah,
Meshullam,
Hizki,
Heber,
\VS{18}Ishmerai,
Izliah,
and Jobab
were the sons
of Elpaal.
\par }{\PP \VS{19}Jakim,
Zikri,
Zabdi,
\VS{20}Elienai,
Zillethai,
Eliel,
\VS{21}Adaiah,
Beraiah,
and Shimrath
were the sons
of Shimei.
\par }{\PP \VS{22}Ishpan,
Eber,
Eliel,
\VS{23}Abdon,
Zikri,
Hanan,
\VS{24}Hananiah,
Elam,
Anthothijah,
\VS{25}Iphdeiah,
and Penuel
were the sons
of Shashak.
\par }{\PP \VS{26}Shamsherai,
Shechariah,
Athaliah,
\VS{27}Jaareshiah,
Elijah,
and Zikri
were the sons
of Jeroham.
\VS{28}These
were the family
leaders
listed in the genealogical records;
they lived
in Jerusalem.
\par }{\PP \VS{29}The father
of Gibeon
lived
in Gibeon;
his wife’s
name
was Maacah.
\VS{30}His firstborn
son
was Abdon,
followed by Zur,
Kish,
Baal,
Nadab,
\VS{31}Gedor,
Ahio,
Zeker, and Mikloth.
\par }{\PP \VS{32}Mikloth
was the father
of Shimeah.
They
also lived
near
their relatives
in Jerusalem.
\par }{\PP \VS{33}Ner
was the father
of Kish,
and Kish
was the father
of Saul.
Saul
was the father
of Jonathan,
Malki-Shua,
Abinadab,
and Eshbaal.
\par }{\PP \VS{34}The son
of Jonathan:
\par }{\PP Meribbaal.
\par }{\PP Meribbaal
was the father
of Micah.
\par }{\PP \VS{35}The sons
of Micah:
\par }{\PP Pithon,
Melech,
Tarea,
and Ahaz.
\par }{\PP \VS{36}Ahaz
was the father of Jehoaddah,
and Jehoaddah
was the father
of Alemeth,
Azmaveth,
and Zimri.
Zimri
was the father
of Moza,
\VS{37}and Moza
was the father
of Binea.
His son was Raphah,
whose son
was Eleasah,
whose son
was Azel.
\par }{\PP \VS{38}Azel
had six
sons: Azrikam
his firstborn,
followed by Ishmael,
Sheariah,
Obadiah,
and Hanan.
All
these
were the sons
of Azel.
\par }{\PP \VS{39}The sons
of his brother
Eshek:
\par }{\PP Ulam
was his firstborn,
Jeush
second,
and Eliphelet
third.
\VS{40}The sons
of Ulam
were
warriors
who were adept archers.
They had many
sons
and grandsons,
a total of 150.
\par }{\PP All
these
were the descendants
of Benjamin.

\par }\Chap{9}{\PP \VerseOne{1}Genealogical records were kept
for all
Israel;
they are recorded
in the Scroll
of the Kings
of Israel.
\par }{\SH Exiles Who Resettled in Jerusalem
\par }{\PP The people of Judah
were carried
away to Babylon
because of their unfaithfulness.
\VS{2}The first
to resettle
on their property
and in their cities
were some Israelites,
priests,
Levites,
and temple servants.
\VS{3}Some
from
the tribes of Judah,
Benjamin,
and Ephraim
and Manasseh
settled
in Jerusalem.
\par }{\PP \VS{4}The settlers included: Uthai
son
of Ammihud,
son
of Omri,
son
of Imri,
son
of Bani,
who was a descendant of Perez
son
of Judah.
\par }{\PP \VS{5}From
the Shilonites: Asaiah
the firstborn
and his sons.
\par }{\PP \VS{6}From
the descendants
of Zerah: Jeuel.
\par }{\PP Their relatives
numbered 690.
\par }{\PP \VS{7}From
the descendants
of Benjamin:
\par }{\PP Sallu
son
of Meshullam,
son
of Hodaviah,
son
of Hassenuah;
\VS{8}Ibneiah
son
of Jeroham;
Elah
son
of Uzzi,
son
of Mikri;
and Meshullam
son
of Shephatiah,
son
of Reuel,
son
of Ibnijah.
\par }{\PP \VS{9}Their relatives,
listed in their genealogical records,
numbered 956.
All
these
men
were leaders
of their families.
\par }{\PP \VS{10}From
the priests:
\par }{\PP Jedaiah;
Jehoiarib;
Jakin;
\VS{11}Azariah
son
of Hilkiah,
son
of Meshullam,
son
of Zadok,
son
of Meraioth,
son
of Ahitub
the leader
in God’s
temple;
\VS{12}Adaiah
son
of Jeroham,
son
of Pashhur,
son
of Malkijah;
and Maasai
son
of Adiel,
son
of Jahzerah,
son
of Meshullam,
son
of Meshillemith,
son
of Immer.
\par }{\PP \VS{13}Their relatives,
who were leaders
of their families,
numbered 1,760.
They were
capable
men who were assigned
to carry out the various tasks
of service
in God’s
temple.
\par }{\PP \VS{14}From
the Levites:
\par }{\PP Shemaiah
son
of Hasshub,
son
of Azrikam,
son
of Hashabiah
a descendant of
Merari;
\VS{15}Bakbakkar;
Heresh;
Galal;
Mattaniah
son
of Mika,
son
of Zikri,
son
of Asaph;
\VS{16}Obadiah
son
of Shemaiah,
son
of Galal,
son
of Jeduthun;
and Berechiah
son
of Asa,
son
of Elkanah,
who lived
among the settlements
of the Netophathites.
\par }{\PP \VS{17}The gatekeepers
were:
\par }{\PP Shallum,
Akkub,
Talmon,
Ahiman,
and their brothers.
Shallum
was the leader;
\VS{18}he serves
to this day at the King’s
Gate
on the east.
These
were the gatekeepers
from the camp
of the descendants
of Levi.
\par }{\PP \VS{19}Shallum
son
of Kore,
son
of Ebiasaph,
son
of
Korah,
and his relatives
from his family
(the Korahites) were assigned
to guard
the entrance
to the sanctuary.
Their ancestors
had guarded
the entrance
to the
{\ND{Lord}}’s
dwelling place.
\VS{20}Phinehas
son
of Eleazar
had been their leader
in earlier
times, and the
{\ND{Lord}}
was with him.
\VS{21}Zechariah
son
of Meshelemiah
was the guard
at the entrance
to the meeting
tent.
\par }{\PP \VS{22}All
those selected
to be gatekeepers
at the entrances
numbered
212.
Their
names were recorded in the genealogical records
of their settlements.
David
and Samuel
the prophet
had appointed
them
to their positions.
\VS{23}They
and their descendants
were assigned
to guard
the gates
of the
{\ND{Lord}}’s
sanctuary
(that is, the tabernacle).
\VS{24}The gatekeepers
were posted on all four
sides
– east,
west,
north,
and south.
\VS{25}Their relatives,
who lived in their settlements,
came
from time
to
time
and served with
them for seven-day periods.
\VS{26}The four
head gatekeepers,
who were
Levites,
were assigned to guard the storerooms
and treasuries
in God’s sanctuary.
\VS{27}They would spend the night in their posts
all around
God’s
sanctuary,
for
they
were assigned to guard
it and would open
it with the key every morning.
\VS{28}Some of them
were in charge
of the articles
used by those who served;
they counted
them when they brought
them in and when they brought them out.
\VS{29}Some of them
were in charge
of the equipment
and articles
of the sanctuary,
as well as the flour,
wine,
olive oil,
incense,
and spices.
\VS{30}(But some
of the priests
mixed
the spices.)
\VS{31}Mattithiah,
a Levite,
the firstborn son
of Shallum
the Korahite,
was in charge
of baking
the bread for offerings.
\VS{32}Some
of the Kohathites,
their relatives,
were in charge of preparing
the bread
that is displayed
each Sabbath.
\par }{\PP \VS{33}The musicians
and Levite
family
leaders
stayed in rooms
at the sanctuary and were exempt
from other duties, for
day
and night
they had to carry out their assigned tasks.
\VS{34}These
were the family
leaders
of the Levites,
as listed in their genealogical records.
They lived
in Jerusalem.
\par }{\SH Jeiel’s Descendants
\par }{\PP \VS{35}Jeiel
(the father
of Gibeon) lived
in Gibeon.
His wife
was Maacah.
\VS{36}His firstborn
son
was Abdon,
followed by Zur,
Kish,
Baal,
Ner,
Nadab,
\VS{37}Gedor,
Ahio,
Zechariah,
and Mikloth.
\VS{38}Mikloth
was the father
of Shimeam.
They
also lived
near
their relatives
in Jerusalem.
\par }{\PP \VS{39}Ner
was the father
of Kish,
and Kish
was the father
of Saul.
Saul
was the father
of Jonathan,
Malki-Shua,
Abinadab,
and Eshbaal.
\par }{\PP \VS{40}The son
of Jonathan:
\par }{\PP Meribbaal,
who was the father
of Micah.
\par }{\PP \VS{41}The sons
of Micah:
\par }{\PP Pithon,
Melech,
Tahrea, and Ahaz.
\par }{\PP \VS{42}Ahaz
was the father
of Jarah,
and Jarah
was the father
of Alemeth,
Azmaveth,
and Zimri.
Zimri
was the father
of Moza,
\VS{43}and Moza
was the father
of Binea.
His son was Rephaiah,
whose son
was Eleasah,
whose son
was Azel.
\par }{\PP \VS{44}Azel
had six
sons: Azrikam
his firstborn,
followed by Ishmael,
Sheariah,
Obadiah,
and Hanan.
These
were the sons
of Azel.

\par }\Chap{10}{\PP \VerseOne{1}Now the Philistines
fought
against Israel.
The Israelites
fled
before
the Philistines
and many of them fell
dead
on Mount
Gilboa.
\VS{2}The Philistines
stayed right on the heels
of Saul
and his sons.
They struck
down Saul’s
sons
Jonathan,
Abinadab,
and Malki-Shua.
\VS{3}The battle
was thick around
Saul;
the archers
spotted him
and wounded him.
\VS{4}Saul
told
his armor
bearer,
“Draw
your sword
and stab
me with it. Otherwise
these
uncircumcised people
will come
and torture
me.” But his armor
bearer
refused
to do it, because
he was very
afraid.
So Saul
took
the sword
and fell
on it.
\VS{5}When his armor
bearer
saw
that
Saul
was dead,
he also
fell
on
his sword
and died.
\VS{6}So
Saul
and his three
sons
died;
his whole
household
died
together.
\VS{7}When
all
the Israelites
who
were in the valley
saw that
the army had fled
and that
Saul
and his sons
were dead,
they abandoned
their cities
and fled.
The Philistines
came
and occupied them.
\par }{\PP \VS{8}The next
day, when the Philistines
came
to strip
loot from the corpses,
they discovered
Saul
and his sons
lying dead
on Mount
Gilboa.
\VS{9}They stripped
his corpse, and then carried
off his head
and his armor.
They sent
messengers throughout the land
of the Philistines
proclaiming the news
to their idols
and
their people.
\VS{10}They placed
his armor
in the temple
of their gods
and hung his head
in the temple
of Dagon.
\VS{11}When all
the residents of Jabesh
Gilead
heard
about everything
the Philistines
had
done
to Saul,
\VS{12}all
the warriors
went
and recovered
the bodies
of Saul
and his sons
and brought
them to Jabesh.
They buried
their remains
under
the oak tree
in Jabesh
and fasted
for seven
days.
\par }{\PP \VS{13}So Saul
died
because he was unfaithful
to the
{\ND{Lord}}
and did not
obey
the
{\ND{Lord}}’s
instructions;
he even
tried to conjure
up underworld spirits.
\VS{14}He did not
seek
the
{\ND{Lord}}’s guidance, so the
{\ND{Lord}}
killed
him and transferred
the kingdom
to David
son
of Jesse.

\par }\Chap{11}{\PP \VerseOne{1}All
Israel
joined
David
at Hebron
and said,
“Look,
we
are your very
flesh and blood!
\VS{2}In the past,
even
when
Saul
was king,
you
were Israel’s
commanding general. The
{\ND{Lord}}
your God
said
to you,
‘You will shepherd
my people
Israel;
you
will rule
over
my people
Israel.’ ”
\VS{3}When all
the leaders
of Israel
came
to
the king
at Hebron,
David
made
an agreement
with them in Hebron
before
the {\ND{Lord}}. They anointed
David
king
over
Israel,
just as the
{\ND{Lord}}
had announced
through
Samuel.
\par }{\SH David Conquers Jerusalem
\par }{\PP \VS{4}David
and the whole
Israelite
army advanced to Jerusalem
(that is,
Jebus). (The Jebusites,
the land’s
original inhabitants, lived there.)
\VS{5}The residents
of Jebus
said
to David,
“You cannot
invade
this place!” But David
captured
the fortress
of Zion
(that is,
the City
of David).
\VS{6}David
said,
“Whoever
attacks
the Jebusites
first
will become
commanding
general!” So Joab
son
of Zeruiah
attacked
first
and became
commander.
\VS{7}David
lived
in the fortress;
for this reason
it is called
the City
of David.
\VS{8}He built
up the city
around
it, from
the terrace
to the surrounding
walls; Joab
restored
the rest
of the city.
\VS{9}David’s
power steadily
grew, for the
{\ND{Lord}}
who commands armies
was with him.
\par }{\SH David’s Warriors
\par }{\PP \VS{10}These
were the leaders
of David’s
warriors
who
helped establish
and stabilize his rule over
all
Israel,
in accordance
with the
{\ND{Lord}}’s
word.
\VS{11}This
is the list
of David’s
warriors:


\par }{\PP Jashobeam,
a Hacmonite,
was head
of the officers.
He killed
three
hundred
men with
his spear
in a single battle.
\par }{\PP \VS{12}Next in command
was Eleazar
son
of Dodo
the Ahohite.
He was one of the three
elite warriors.
\VS{13}He
was
with
David
in Pas Dammim
when the Philistines
assembled
there
for battle.
In an area
of the field
that was full
of barley,
the army
retreated
before
the Philistines,
\VS{14}but then they made a stand
in the middle
of that area.
They defended
it and defeated
the
Philistines;
the {\ND{Lord}}
gave
them a great
victory.
\par }{\PP \VS{15}Three
of the thirty
leaders
went down
to
David
at the rocky
cliff at the cave
of Adullam,
while a Philistine
force was camped
in the Valley
of Rephaim.
\VS{16}David
was in the stronghold
at the time,
while a Philistine
garrison
was in Bethlehem.
\VS{17}David
was thirsty
and said,
“How I wish
someone would
give me some water
to drink
from the cistern
in Bethlehem
near the city gate!”
\VS{18}So the three
elite warriors broke through
the Philistine
forces
and drew
some water
from the cistern
in Bethlehem
near the city gate.
They carried
it back
to
David,
but David
refused
to drink
it. He poured
it out
as a drink offering to the
{\ND{Lord}}
\VS{19}and said,
“God
forbid
that I should do
this! Should I drink
the blood
of these
men
who risked their lives?” Because
they risked their lives
to bring
it to him, he refused
to drink
it. Such were the exploits
of the three
elite warriors.
\par }{\PP \VS{20}Abishai
the brother
of Joab
was
head
of the three
elite
warriors. He
killed three
hundred
men with his spear
and gained
fame
along with the three elite warriors.
\VS{21}From
the three
he was given double
honor
and he became
their officer,
even
though he was not one of them.
\par }{\PP \VS{22}Benaiah
son
of Jehoiada
was a brave
warrior
from Kabzeel
who performed great exploits. He struck
down the two
sons of Ariel
of Moab;
he also went down
and killed
a lion
inside
a cistern
on a snowy
day.
\VS{23}He
even killed
an Egyptian
who was seven and a half feet
tall.
The Egyptian
had a spear
as big as the crossbeam
of a weaver’s
loom; Benaiah attacked
him with
a club.
He grabbed
the spear
out of the Egyptian’s
hand
and killed
him with his own spear.
\VS{24}Such
were the exploits
of Benaiah
son
of Jehoiada,
who gained fame
along with the three
elite warriors.
\VS{25}He received honor
from
the thirty
warriors, though he was not
one of the three
elite warriors. David
put
him in charge of
his bodyguard.
\par }{\PP \VS{26}The mighty
warriors were:
\par }{\PP Asahel
the brother
of Joab,
\par }{\PP Elhanan
son
of Dodo,
from Bethlehem,
\par }{\PP \VS{27}Shammoth
the Harorite,
\par }{\PP Helez
the Pelonite,
\par }{\PP \VS{28}Ira
son
of Ikkesh
the Tekoite,
\par }{\PP Abiezer
the Anathothite,
\par }{\PP \VS{29}Sibbekai
the Hushathite,
\par }{\PP Ilai
the Ahohite,
\par }{\PP \VS{30}Maharai
the Netophathite,
\par }{\PP Heled
son
of Baanah
the Netophathite,
\par }{\PP \VS{31}Ithai
son
of Ribai
from Gibeah
in Benjaminite
territory,
\par }{\PP Benaiah
the Pirathonite,
\par }{\PP \VS{32}Hurai
from the valleys
of Gaash,
\par }{\PP Abiel
the Arbathite,
\par }{\PP \VS{33}Azmaveth
the Baharumite,
\par }{\PP Eliahba
the Shaalbonite,
\par }{\PP \VS{34}the sons
of Hashem
the Gizonite,
\par }{\PP Jonathan
son
of Shageh
the Hararite,
\par }{\PP \VS{35}Ahiam
son
of Sakar
the Hararite,
\par }{\PP Eliphal
son
of Ur,
\par }{\PP \VS{36}Hepher
the Mekerathite,
\par }{\PP Ahijah
the Pelonite,
\par }{\PP \VS{37}Hezro
the Carmelite,
\par }{\PP Naarai
son
of Ezbai,
\par }{\PP \VS{38}Joel
the brother
of Nathan,
\par }{\PP Mibhar
son
of Hagri,
\par }{\PP \VS{39}Zelek
the Ammonite,
\par }{\PP Naharai
the Beerothite,
the armor-bearer
of Joab
son
of Zeruiah,
\par }{\PP \VS{40}Ira
the Ithrite,
\par }{\PP Gareb
the Ithrite,
\par }{\PP \VS{41}Uriah
the Hittite,
\par }{\PP Zabad
son
of Achli,
\par }{\PP \VS{42}Adina
son
of Shiza
the Reubenite,
leader
of the Reubenites
and the thirty warriors with him,
\par }{\PP \VS{43}Hanan
son
of Maacah,
\par }{\PP Joshaphat
the Mithnite,
\par }{\PP \VS{44}Uzzia
the Ashterathite,
\par }{\PP Shama
and Jeiel,
the sons
of Hotham
the Aroerite,
\par }{\PP \VS{45}Jediael
son
of Shimri,
\par }{\PP and Joha
his brother,
the Tizite,
\par }{\PP \VS{46}Eliel
the Mahavite,
\par }{\PP and Jeribai
and Joshaviah,
the sons
of Elnaam,
\par }{\PP and Ithmah
the Moabite,
\par }{\PP \VS{47}Eliel,
\par }{\PP and Obed,
\par }{\PP and Jaasiel
the Mezobaite.

\par }\Chap{12}{\PP \VerseOne{1}These
were the men who joined
David
in Ziklag,
when he was banished
from the presence
of Saul
son
of Kish.
(They
were among the warriors
who assisted
him in battle.
\VS{2}They were armed
with bows
and could shoot arrows
or sling
stones
right or
left-handed. They were fellow
tribesmen
of Saul
from Benjamin. ) These were:
\par }{\PP \VS{3}Ahiezer,
the leader,
and Joash,
the sons
of Shemaah
the Gibeathite;
Jeziel
and Pelet,
the sons
of Azmaveth;
\par }{\PP Berachah,
\par }{\PP Jehu
the Anathothite,
\par }{\PP \VS{4}Ishmaiah
the Gibeonite,
one of the thirty
warriors
and their leader,

\par }{\PP Jeremiah,
\par }{\PP Jahaziel,
\par }{\PP Johanan,
\par }{\PP Jozabad
the
Gederathite,
\par }{\PP \VS{5}Eluzai,
\par }{\PP Jerimoth,
\par }{\PP Bealiah,
\par }{\PP Shemariah,
\par }{\PP Shephatiah
the Haruphite,
\par }{\PP \VS{6}Elkanah,
Isshiah,
Azarel,
Joezer,
and Jashobeam,
who were Korahites,
\par }{\PP \VS{7}and Joelah
and Zebadiah,
the sons
of Jeroham
from Gedor.
\par }{\PP \VS{8}Some of the Gadites
joined
David
at the stronghold
in the desert.
They were warriors
who
were trained
for battle;
they carried
shields
and spears.
They were as fierce
as lions
and could run as quickly
as gazelles
across the hills.
\VS{9}Ezer
was the leader,
Obadiah
the second
in command, Eliab
the third,
\VS{10}Mishmannah
the fourth,
Jeremiah
the fifth,
\VS{11}Attai
the sixth,
Eliel
the seventh,
\VS{12}Johanan
the eighth,
Elzabad
the ninth,
\VS{13}Jeremiah
the tenth,
and Machbannai
the eleventh.
\VS{14}These
Gadites
were military leaders;
the least
led a hundred
men, the greatest
a
thousand.
\VS{15}They
crossed
the Jordan River
in the first
month,
when it was overflowing
its banks,
and routed
those living in all
the valleys
to the east
and west.
\par }{\PP \VS{16}Some
from Benjamin
and Judah
also came
to David’s
stronghold.
\VS{17}David
went out
to meet
them and said, “If
you come
to me in peace
and want to help
me, then I will make
an alliance
with you. But if
you come to betray
me to my enemies
when I have not
harmed
you, may the God
of our ancestors
take notice
and judge!”
\VS{18}But a spirit
empowered
Amasai,
the leader
of the thirty
warriors, and he said:
\par }{\PP “We are yours, O
David!
\par }{\PP We support
you, O
son
of Jesse!
\par }{\PP May
you greatly prosper!
\par }{\PP May those who help
you prosper!
\par }{\PP Indeed
your God
helps
you!”
\par }{\PP So David
accepted them and made
them leaders
of raiding bands.
\par }{\PP \VS{19}Some
men from Manasseh
joined
David
when he went
with the Philistines
to fight
against
Saul.
(But in the end they did not help
the Philistines because, after taking counsel,
the Philistine
lords
sent
David away, saying: “It would be disastrous
for us if he deserts to his master
Saul.”)
\VS{20}When David went
to Ziklag,
the men of Manasseh
who joined
him were Adnach,
Jozabad,
Jediael,
Michael,
Jozabad,
Elihu,
and Zillethai,
leaders
of a thousand
soldiers each in the tribe of Manasseh.
\VS{21}They helped
David
fight against
raiding bands,
for all
of them were
warriors
and leaders
in the army.
\VS{22}Each
day
men
came
to help
David
until
his army
became very large.
\par }{\SH Support for David in Hebron
\par }{\PP \VS{23}The following
is a record
of the armed
warriors
who came
with their leaders
and joined David
in Hebron
in order to make
David king
in Saul’s
place, in accordance with the
{\ND{Lord}}’s
decree:
\par }{\PP \VS{24}From Judah
came 6,800
trained
warriors
carrying
shields
and spears.
\par }{\PP \VS{25}From Simeon
there were 7,100
warriors.
\par }{\PP \VS{26}From Levi
there were 4,600.
\VS{27}Jehoiada,
the leader
of Aaron’s descendants,
brought 3,700 men with him,
\VS{28}along with Zadok,
a young
warrior,
and twenty-two
leaders
from his family.
\par }{\PP \VS{29}From Benjamin,
Saul’s
tribe,
there were 3,000,
most of whom,
up to
that time,
had been loyal
to Saul.
\par }{\PP \VS{30}From Ephraim
there were 20,800
warriors,
who had brought fame
to their families.
\par }{\PP \VS{31}From the half
tribe
of Manasseh
there were 18,000
who had
been designated
by name
to come
and make
David
king.
\par }{\PP \VS{32}From Issachar
there were 200
leaders
and all
their relatives
at their command
– they understood
the times
and knew
what
Israel
should do.
\par }{\PP \VS{33}From Zebulun
there were 50,000
warriors
who were prepared
for battle,
equipped with all
kinds of weapons,
and ready to give their undivided loyalty.
\par }{\PP \VS{34}From Naphtali
there were 1,000
officers,
along with 37,000
men carrying shields
and spears.
\par }{\PP \VS{35}From Dan
there were 28,600
men prepared
for battle.
\par }{\PP \VS{36}From Asher
there were 40,000
warriors prepared
for battle.
\par }{\PP \VS{37}From the other side
of the Jordan,
from Reuben,
Gad,
and the half
tribe
of Manasseh,
there were 120,000
men
armed
with all kinds
of weapons.
\par }{\PP \VS{38}All
these
men
were warriors
who were ready to march.
They came
to Hebron
to make
David
king
over
all
Israel
by acclamation;
all
the rest
of the Israelites
also
were in agreement
that David
should become king.
\VS{39}They spent
three
days
feasting
there
with David,
for their relatives
had given them provisions.
\VS{40}Also their neighbors,
from as far
away as
Issachar,
Zebulun,
and Naphtali,
were bringing
food
on donkeys,
camels,
mules,
and oxen.
There were large
supplies
of flour,
fig
cakes,
raisins,
wine,
olive oil,
beef,
and lamb,
for
Israel
was celebrating.

\par }\Chap{13}{\PP \VerseOne{1}David
consulted
with
his military officers,
including those
who led groups of a thousand
and those who led groups
of a hundred.
\VS{2}David
said
to the whole
Israelite
assembly,
“If
you so desire
and the
{\ND{Lord}}
our God
approves, let’s spread
the word to
our brothers
who remain
in all
the regions
of Israel,
and to
the priests
and Levites
in their cities,
so they may join us.
\VS{3}Let’s move
the ark
of our God
back here, for we
did not
seek
his will throughout
Saul’s
reign.”
\VS{4}The whole
assembly
agreed to do
this,
for
the proposal
seemed
right to all
the people.
\VS{5}So David
assembled
all
Israel
from
the Shihor River
in Egypt
to
Lebo Hamath,
to bring
the ark
of God
from Kiriath Jearim.
\VS{6}David
and all
Israel
went up
to Baalah
(that is, Kiriath Jearim) in Judah
to bring up
from there
the ark
of God
the {\ND{Lord}}, who sits enthroned
between the cherubim
– the ark that
is called
by his name.
\par }{\PP \VS{7}They transported
the ark
on
a new
cart
from the house
of Abinadab;
Uzzah
and Ahio
were guiding
the cart,
\VS{8}while David
and all
Israel
were energetically
celebrating
before
God,
singing
and playing various
stringed instruments,
tambourines,
cymbals,
and trumpets.
\VS{9}When
they arrived
at the threshing floor
of Kidon,
Uzzah
reached out
his hand
to take hold
of the ark,
because
the oxen
stumbled.
\VS{10}The
{\ND{Lord}}
was so furious
with Uzzah,
he killed
him, because he reached
out his hand
and touched the ark.
He died
right there
before
God.
\par }{\PP \VS{11}David
was angry
because
the {\ND{Lord}}
attacked
Uzzah;
so he called
that place
Perez Uzzah,
which remains
its name to this
very day.
\VS{12}David
was afraid
of God
that day
and said,
“How will I ever
be able to bring
the ark
of God up here?”
\VS{13}So David
did not
move
the ark
to
the City
of David;
he left
it in the house
of Obed-Edom
the Gittite.
\VS{14}The ark
of God
remained
in Obed-Edom’s
house
for three
months;
the {\ND{Lord}}
blessed
Obed-Edom’s
family
and everything
that belonged to him.

\par }\Chap{14}{\PP \VerseOne{1}King
Hiram
of Tyre
sent
messengers
to
David,
along with cedar
logs,
stonemasons,
and carpenters
to build
a palace for him.
\VS{2}David
realized
that
the {\ND{Lord}}
had established
him as king
over
Israel
and that
he had elevated
his kingdom
for the sake of
his people
Israel.
\par }{\PP \VS{3}In Jerusalem
David
married
more
wives
and fathered
more
sons
and daughters.
\VS{4}These
are the names
of children born
to him in Jerusalem: Shammua,
Shobab,
Nathan,
Solomon,
\VS{5}Ibhar,
Elishua,
Elpelet,
\VS{6}Nogah,
Nepheg,
Japhia,
\VS{7}Elishama,
Beeliada,
and Eliphelet.
\par }{\PP \VS{8}When the Philistines
heard
that
David
had been anointed
king
of
all
Israel,
all
the Philistines
marched up
to confront
him. When David
heard
about it, he
marched out
against them.
\VS{9}Now the Philistines
had come
and raided
the Valley
of Rephaim.
\VS{10}David
asked
God,
“Should I march up
against
the Philistines? Will you
hand
them over to me?” The
{\ND{Lord}}
said
to him, “March up! I will
hand them over to you!”
\VS{11}So they marched against
Baal Perazim
and David
defeated
them there.
David
said,
“Using me as his instrument, God
has burst out
against my enemies
like water
bursts out.”
So that
place
is called
Baal Perazim.
\VS{12}The Philistines left
their idols there,
so David
ordered that they be burned.
\par }{\PP \VS{13}The Philistines
again
raided
the valley.
\VS{14}So David
again
asked
God
what he should do. This time God
told him, “Don’t
march up
after
them; circle
around them and come
against them
in front
of the trees.
\VS{15}When
you hear
the
sound
of marching
in the tops
of the trees,
then
attack.
For
at that moment
the
{\ND{Lord}} is going before
you to strike
down the
army
of the Philistines.”
\VS{16}David
did
just
as God
commanded
him, and they struck
down the Philistine
army
from Gibeon
to Gezer.
\par }{\PP \VS{17}So
David
became
famous
in all
the lands;
the {\ND{Lord}}
caused
all
the nations
to fear him.

\par }\Chap{15}{\PP \VerseOne{1}David constructed
buildings
in the City
of David;
he then prepared
a place
for the ark
of God
and pitched
a tent for it.
\VS{2}Then
David
said,
“Only
the Levites
may carry
the ark
of God,
for
the {\ND{Lord}}
chose
them to carry
the ark
of the {\ND{Lord}}
and to serve
before
him perpetually.
\VS{3}David
assembled
all
Israel
at
Jerusalem
to bring
the ark
of the {\ND{Lord}}
up to
the place
he had
prepared for it.
\VS{4}David
gathered together
the descendants
of Aaron
and the Levites:
\par }{\PP \VS{5}From the descendants
of Kohath: Uriel
the leader
and 120
of his relatives.
\par }{\PP \VS{6}From the descendants
of Merari: Asaiah
the leader
and 220
of his relatives.
\par }{\PP \VS{7}From the descendants
of Gershom: Joel
the leader
and 130
of his relatives.
\par }{\PP \VS{8}From the descendants
of Elizaphan: Shemaiah
the leader
and 200
of his relatives.
\par }{\PP \VS{9}From the descendants
of Hebron: Eliel
the leader
and 80
of his relatives.
\par }{\PP \VS{10}From the descendants
of Uzziel: Amminadab
the leader
and 112
of his relatives.
\par }{\PP \VS{11}David
summoned
the priests
Zadok
and Abiathar,
along with the Levites
Uriel,
Asaiah,
Joel,
Shemaiah,
Eliel,
and Amminadab.
\VS{12}He told
them: “You
are the leaders
of the Levites’
families.
You
and your relatives
must consecrate
yourselves
and bring
the
ark
of the {\ND{Lord}}
God
of Israel
up to
the place I have prepared for it.
\VS{13}The first
time you
did not
carry it; that is why the
{\ND{Lord}}
God
attacked
us, because
we did not
ask
him about the proper way to carry it.”
\VS{14}The priests
and Levites
consecrated
themselves so they could bring up
the ark
of the {\ND{Lord}}
God
of Israel.
\VS{15}The descendants
of Levi
carried
the ark
of God
on their shoulders
with poles, just
as Moses
had
ordered
according to the divine command.
\par }{\PP \VS{16}David
told
the leaders
of the Levites
to appoint
some of their relatives
as musicians;
they were
to play various
instruments,
including stringed instruments
and cymbals,
and to sing
loudly
and joyfully.
\VS{17}So the Levites
appointed
Heman
son
of Joel;
one of his relatives,
Asaph
son
of Berechiah;
one of the descendants
of Merari,
Ethan
son
of Kushaiah;
\VS{18}along with
some of their relatives
who were second in rank,
including Zechariah,
Jaaziel,
Shemiramoth,
Jehiel,
Unni,
Eliab,
Benaiah,
Maaseiah,
Mattithiah,
Eliphelehu,
Mikneiah,
Obed-Edom,
and Jeiel,
the gatekeepers.
\par }{\PP \VS{19}The musicians
Heman,
Asaph,
and Ethan
were to sound the bronze
cymbals;
\VS{20}Zechariah,
Aziel,
Shemiramoth,
Jehiel,
Unni,
Eliab,
Maaseiah,
and Benaiah
were to play the harps
according to the
{\IT{
alamoth}} style;
\VS{21}Mattithiah,
Eliphelehu,
Mikneiah,
Obed-Edom,
Jeiel,
and Azaziah
were to play the lyres
according
to the
{\IT{
sheminith}}
style, as led by the director;
\VS{22}Kenaniah,
the leader
of the Levites,
was in charge of transport,
for
he was well-informed on this matter;
\VS{23}Berechiah
and Elkanah
were guardians
of the ark;
\VS{24}Shebaniah,
Joshaphat,
Nethanel,
Amasai,
Zechariah,
Benaiah,
and Eliezer
the priests
were
to blow
the trumpets
before
the ark
of God;
Obed-Edom
and Jehiel
were also guardians
of the ark.
\par }{\PP \VS{25}So
David,
the leaders
of Israel,
and the commanders
of units of a thousand
went
to bring up
the ark
of the
{\ND{Lord}}’s
covenant
from
the house
of Obed-Edom
with celebration.
\VS{26}When
God
helped
the Levites
who were carrying
the ark
of the
{\ND{Lord}}’s
covenant,
they sacrificed
seven
bulls
and seven
rams.
\VS{27}David
was wrapped
in a linen
robe,
as were all
the Levites
carrying
the ark,
the musicians,
and Kenaniah
the supervisor
of transport
and the musicians;
David
also wore a linen
ephod.
\VS{28}All
Israel
brought up
the ark
of the
{\ND{Lord}}’s
covenant;
they were shouting,
blowing
trumpets,
sounding cymbals,
and playing
stringed instruments.
\VS{29}As the ark
of the
{\ND{Lord}}’s
covenant
entered
the City
of David,
Michal,
Saul’s
daughter,
looked
out the window.
When she saw
King
David
jumping
and celebrating,
she despised him.

\par }\Chap{16}{\PP \VerseOne{1}They brought
the ark
of God
and put
it in the middle
of the tent
David
had
pitched
for it. Then
they offered
burnt sacrifices
and peace offerings
before
God.
\VS{2}When David
finished
offering
burnt sacrifices
and peace offerings,
he pronounced a blessing
over the people
in the
{\ND{Lord}}’s
name.
\VS{3}He then handed
out to each
Israelite
man
and woman
a loaf
of bread,
a date cake,
and a raisin cake.
\VS{4}He appointed
some
of the Levites
to serve
before
the ark
of the {\ND{Lord}}, to offer
prayers, songs
of thanks,
and hymns
to the
{\ND{Lord}}
God
of Israel.
\VS{5}Asaph
was the leader
and Zechariah
second
in command, followed by Jeiel,
Shemiramoth,
Jehiel,
Mattithiah,
Eliab,
Benaiah,
Obed-Edom,
and Jeiel.
They were to play
stringed instruments;
Asaph
was to sound
the cymbals;
\VS{6}and the priests
Benaiah
and Jahaziel
were to blow trumpets
regularly
before
the ark
of God’s
covenant.
\par }{\SH David Thanks God
\par }{\PP \VS{7}That
day
David
first
gave
to Asaph
and his colleagues
this song of thanks
to the
{\ND{Lord}}:
\par }{\Q \VS{8}Give thanks
to the
{\ND{Lord}}!
\par }{\Q Call
on his name!
\par }{\Q Make known
his accomplishments
among the nations!
\par }{\Q \VS{9}Sing
to him! Make music
to him!
\par }{\Q Tell about all
his miraculous deeds!
\par }{\Q \VS{10}Boast
about his holy
name!
\par }{\Q Let the hearts
of those who seek
the {\ND{Lord}}
rejoice!
\par }{\Q \VS{11}Seek
the {\ND{Lord}}
and the strength
he gives!
\par }{\Q Seek
his presence
continually!
\par }{\Q \VS{12}Recall
the miraculous
deeds he performed,
\par }{\Q his mighty acts
and the judgments
he decreed,
\par }{\Q \VS{13}O children
of Israel,
God’s servant,
\par }{\Q you descendants
of Jacob,
God’s chosen ones!
\par }{\Q \VS{14}He is
the {\ND{Lord}}
our God;
\par }{\Q he carries out judgment
throughout
the earth.
\par }{\Q \VS{15}Remember
continually
his covenantal
decree,
\par }{\Q the promise
he made to a thousand
generations –
\par }{\Q \VS{16}the promise
he made
to Abraham,
\par }{\Q the promise he made by oath
to Isaac!
\par }{\Q \VS{17}He gave
it to Jacob
as a decree,
\par }{\Q to Israel
as a lasting
promise,
\par }{\Q \VS{18}saying,
“To you I will give
the land
of Canaan
\par }{\Q as the portion
of your inheritance.”
\par }{\Q \VS{19}When
they were few
in number,
\par }{\Q just a
very few,
and foreign residents within it,
\par }{\Q \VS{20}they wandered
from nation
to
nation,
\par }{\Q and from one kingdom
to
another.
\par }{\Q \VS{21}He let no
one
oppress
them,
\par }{\Q he disciplined
kings
for their sake,
\par }{\Q \VS{22}saying, “Don’t
touch
my anointed ones!
\par }{\Q Don’t
harm
my prophets!”
\par }{\Q \VS{23}Sing
to the
{\ND{Lord}}, all
the earth!
\par }{\Q Announce
every day
how he delivers!
\par }{\Q \VS{24}Tell
the nations
about his splendor,
\par }{\Q tell all
the nations
about his miraculous deeds!
\par }{\Q \VS{25}For
the {\ND{Lord}}
is great
and certainly worthy
of praise,
\par }{\Q he is
more
awesome
than all
gods.
\par }{\Q \VS{26}For
all
the gods
of the nations
are worthless,
\par }{\Q but the
{\ND{Lord}}
made
the heavens.
\par }{\Q \VS{27}Majestic
splendor
emanates
from him,

\par }{\Q he is the source
of strength
and joy.
\par }{\Q \VS{28}Ascribe
to the
{\ND{Lord}}, O families
of the nations,
\par }{\Q ascribe
to the
{\ND{Lord}}
splendor
and strength!
\par }{\Q \VS{29}Ascribe
to the
{\ND{Lord}}
the splendor
he deserves!

\par }{\Q Bring
an offering
and enter
his presence!
\par }{\Q Worship
the {\ND{Lord}}
in holy
attire!
\par }{\Q \VS{30}Tremble
before
him, all
the earth!
\par }{\Q The world
is established,
it cannot
be moved.
\par }{\Q \VS{31}Let
the heavens
rejoice,
and the earth
be happy!
\par }{\Q Let the nations
say, ‘The
{\ND{Lord}}
reigns!’
\par }{\Q \VS{32}Let
the sea
and everything
in it shout!
\par }{\Q Let
the fields
and everything
in them celebrate!
\par }{\Q \VS{33}Then
let the trees
of the forest
shout
with joy before
the {\ND{Lord}},
\par }{\Q for
he comes
to judge
the earth!
\par }{\Q \VS{34}Give thanks
to the
{\ND{Lord}}, for
he is good
\par }{\Q and his loyal love
endures.
\par }{\Q \VS{35}Say
this prayer: “Deliver
us, O God
who delivers
us!
\par }{\Q Gather
us! Rescue
us from
the nations!
\par }{\Q Then we will give thanks
to your holy
name,
\par }{\Q and boast
about your praiseworthy deeds.”
\par }{\Q \VS{36}May the
{\ND{Lord}}
God
of Israel
be praised,
\par }{\Q in the future
and forevermore.
\par }{\Q Then all
the people
said, “We agree! Praise
the {\ND{Lord}}!”
\par }{\SH David Appoints Worship Leaders
\par }{\PP \VS{37}David left
Asaph
and his colleagues
there
before
the ark
of the
{\ND{Lord}}’s
covenant
to serve
before
the ark
regularly
and fulfill
each day’s
requirements,
\VS{38}including Obed-Edom
and sixty-eight
colleagues.
Obed-Edom
son
of Jeduthun
and Hosah
were gatekeepers.
\VS{39}Zadok
the priest
and his fellow
priests
served before
the
{\ND{Lord}}’s
tabernacle
at the worship center
in Gibeon,
\VS{40}regularly
offering
burnt sacrifices
to the
{\ND{Lord}}
on
the altar
for burnt sacrifice,
morning
and evening,
according to what is prescribed
in the law
of the {\ND{Lord}}
which
he charged
Israel to observe.
\VS{41}Joining
them were Heman,
Jeduthun,
and the rest
of those chosen
and designated
by name
to give thanks
to the
{\ND{Lord}}. (For
his loyal love
endures!)
\VS{42}Heman
and Jeduthun
were in charge of the music, including the trumpets,
cymbals,
and the other musical
instruments
used
in praising
God.
The sons
of Jeduthun
guarded the entrance.
\par }{\PP \VS{43}Then
all
the people
returned to their homes,
and David
went to pronounce a blessing
on his family.

\par }\Chap{17}{\PP \VerseOne{1}When
David
had settled
into his palace,
he
said
to
Nathan
the prophet,
“Look,
I
am living
in a palace
made from cedar,
while the ark
of the
{\ND{Lord}}’s
covenant
is under
a tent.”
\VS{2}Nathan
said
to
David,
“You should do
whatever
you have in mind,
for
God
is with you.”
\par }{\PP \VS{3}That night
God
told
Nathan the prophet,
\VS{4}“Go,
tell
my servant
David: ‘This is what
the {\ND{Lord}}
says: “You
must not
build
me a house
in which to live.
\VS{5}For
I have not
lived
in a house
from
the time
I brought
Israel
up
from Egypt to
the present day.
I have lived
in a tent
that has been in
various places.
\VS{6}Wherever
I moved
throughout
Israel,
I did not say
to any
of the leaders
whom
I appointed
to care
for my people
Israel, ‘Why
have you not
built
me a house
made from cedar?’ ” ’
\par }{\PP \VS{7}“So now,
say
this
to my servant
David: ‘This is what
the {\ND{Lord}}
who commands armies
says: “I
took
you from
the pasture
and from
your
work as a shepherd
to make you a leader
of my people
Israel.
\VS{8}I was
with
you wherever
you went
and I defeated
all
your enemies
before
you. Now I will make
you as famous
as the great men
of the earth.
\VS{9}I will establish
a place
for my people
Israel
and settle
them there; they will live there and not
be disturbed
anymore.
Violent men
will not
oppress
them again,
as
they did in the beginning
\VS{10}and during the time
when I appointed
judges
to lead my people
Israel.
I will subdue
all
your enemies.
\par }{\PP “‘ “I declare
to you that the
{\ND{Lord}}
will build
a dynastic house for you!
\VS{11}When
the time
comes
for
you to die,
I will raise up
your descendant,
one of your own sons,
to succeed you, and I will establish
his kingdom.
\VS{12}He
will build
me a house,
and I
will make his dynasty
permanent.
\VS{13}I
will become
his father
and he
will become
my son.
I will never
withhold
my loyal love
from him,
as
I withheld
it from
the one
who
ruled
before you.
\VS{14}I will put
him in permanent
charge
of my house
and my kingdom;
his dynasty
will be
permanent.” ’ ”
\VS{15}Nathan
told
David
all
these
words that were revealed to him.
\par }{\SH David Praises God
\par }{\PP \VS{16}David
went
in, sat
before
the {\ND{Lord}}, and said: “Who
am
I, O
{\ND{Lord}}
God,
and what
is my family,
that
you should have brought
me to this point?
\VS{17}And you did not stop there, O God! You have also spoken
about the future
of your servant’s
family.
You have revealed
to me what men
long to know,
O
{\ND{Lord}}
God.
\VS{18}What
more
can David
say
to
you? You have honored
your servant;
you
have given your servant
special recognition.
\VS{19}O
{\ND{Lord}}, for the sake of
your servant
and according
to your will, you have done
this
great thing
in order to reveal
your greatness.
\VS{20}O
{\ND{Lord}}, there is none
like
you; there is no
God
besides
you! What we heard
is true!
\VS{21}And who
is like your people,
Israel,
a unique
nation
in the earth? Their God
went
to claim
a nation
for himself! You made
a name
for yourself by doing great
and awesome
deeds when you drove
out nations before
your people
whom
you had delivered
from the Egyptian
empire and its gods.
\VS{22}You made
Israel
your very
own
nation
for all time. You,
O
{\ND{Lord}}, became
their God.
\VS{23}So now,
O
{\ND{Lord}}, may the promise
you made
about your servant
and his family
become
a permanent
reality! Do
as
you promised,
\VS{24}so it may become a reality
and you may gain lasting
fame,
as people say, ‘The
{\ND{Lord}}
who commands armies
is the God
of Israel.’
David’s
dynasty
will be established
before you,
\VS{25}for
you,
my God,
have revealed
to your servant
that you will build
a dynasty
for him. That is why
your servant
has had
the courage
to pray
to you.
\VS{26}Now,
O
{\ND{Lord}}, you
are the true God;
you have
made this
good
promise
to your servant.
\VS{27}Now
you are willing
to bless
your servant’s
dynasty
so that it may
stand permanently
before
you, for
you,
O
{\ND{Lord}}, have blessed
it and it will be blessed
from now on into the future.”

\par }\Chap{18}{\PP \VerseOne{1}Later
David
defeated
the Philistines
and subdued
them. He took
Gath
and its surrounding towns
away from
the Philistines.
\par }{\PP \VS{2}He defeated
the Moabites;
the Moabites
became David’s
subjects
and brought
tribute.
\par }{\PP \VS{3}David
defeated
King
Hadadezer
of Zobah
as far as Hamath,
when he went
to extend
his authority
to the Euphrates
River.
\VS{4}David
seized
from
him 1,000
chariots,
7,000
charioteers,
and 20,000
infantrymen.
David
cut the hamstrings
of all
but
a hundred
of
Hadadezer’s chariot horses.
\VS{5}The Arameans
of Damascus
came to help
King
Hadadezer
of Zobah,
but David
killed
22,000
of the Arameans.
\VS{6}David
placed
garrisons in the territory of the Arameans
of Damascus;
the Arameans
became David’s
subjects
and brought
tribute.
The
{\ND{Lord}}
protected
David
wherever
he campaigned.
\VS{7}David
took
the golden
shields
which
Hadadezer’s
servants
had carried and brought
them to Jerusalem.
\VS{8}From Tibhath
and Kun,
Hadadezer’s
cities,
David
took
a great
deal
of bronze.
(Solomon
used it to make
the big bronze basin called “The Sea,”
the pillars,
and other bronze
items.
\par }{\PP \VS{9}When
King
Tou
of Hamath
heard that
David
had defeated
the entire
army
of King
Hadadezer
of Zobah,
\VS{10}he sent
his son
Hadoram
to
King
David
to
extend his best wishes and to pronounce a blessing
on
him
for
his victory
over
Hadadezer,
for
Tou
had been at war
with Hadadezer.
He also sent various
items
made of gold,
silver,
and bronze.
\VS{11}King
David
dedicated
these things to the
{\ND{Lord}}, along with
the silver
and gold
which
he had carried
off from all
the nations,
including Edom,
Moab,
the Ammonites,
the Philistines,
and Amalek.
\par }{\PP \VS{12}Abishai
son
of Zeruiah
killed
18,000
Edomites
in the Valley
of Salt.
\VS{13}He placed
garrisons
in Edom,
and all
the Edomites
became David’s
subjects.
The
{\ND{Lord}}
protected
David
wherever
he campaigned.
\par }{\SH David’s Officials
\par }{\PP \VS{14}David
reigned
over
all
Israel;
he guaranteed
justice
for all
his people.
\VS{15}Joab
son
of Zeruiah
was commanding general
of the army; Jehoshaphat
son
of Ahilud
was secretary;
\VS{16}Zadok
son
of Ahitub
and Abimelech
son
of Abiathar
were priests;
Shavsha
was scribe;
\VS{17}Benaiah
son
of Jehoiada
supervised
the Kerethites
and Pelethites;
and David’s
sons
were
the king’s
leading officials.

\par }\Chap{19}{\PP \VerseOne{1}Later
King
Nahash
of the Ammonites
died
and his son
succeeded him.
\VS{2}David
said,
“I will express
my loyalty
to Hanun
son
of Nahash,
for
his father
was loyal
to me.” So
David
sent
messengers
to
express
his sympathy over
his father’s
death. When
David’s
servants
entered
Ammonite
territory
to visit
Hanun
and express the king’s sympathy,
\VS{3}the Ammonite
officials
said
to Hanun,
“Do you really think David
is trying to honor
your father
by sending
these messengers to express
his sympathy? No,
his servants
have come
to
you so they can
get information
and spy
out the land!”
\VS{4}So Hanun
seized
David’s
servants
and shaved their beards off.
He cut off
the lower part
of their robes
so that
their buttocks
were exposed and then sent
them away.
\VS{5}Messengers came
and told
David
what had happened to the men,
so
he summoned
them, for
the men
were thoroughly
humiliated.
The king
said,
“Stay
in Jericho
until
your beards
grow
again; then you may come back.”
\par }{\PP \VS{6}When the Ammonites
realized
that
David
was disgusted
with them, Hanun
and the Ammonites
sent
1,000
talents
of silver
to hire
chariots
and charioteers
from
Aram Naharaim,
Aram
Maacah,
and Zobah.
\VS{7}They hired
32,000
chariots,
along with the king
of Maacah
and his army,
who came
and camped
in front
of Medeba.
The Ammonites
also assembled
from their cities
and marched
out to do battle.
\par }{\PP \VS{8}When David
heard
the news, he sent
Joab
and the entire
army
to meet them.
\VS{9}The Ammonites
marched out and were deployed
for battle
at the entrance
to the city,
while the kings
who had
come
were by themselves
in the field.
\VS{10}When Joab
saw
that
the battle
would be
fought on two fronts,
he chose
some of Israel’s
best men
and deployed
them against
the Arameans.
\VS{11}He put
his
brother
Abishai
in charge of the rest
of the army
and they were deployed
against
the Ammonites.
\VS{12}Joab said,
“If
the Arameans
start to overpower
me, you come
to my rescue.
If
the Ammonites
start to overpower
you, I will come to your rescue.
\VS{13}Be strong! Let’s
fight bravely
for the sake of our people
and the cities
of our God! The
{\ND{Lord}}
will do
what he decides
is best!”
\VS{14}So
Joab
and his men
marched
toward
the Arameans
to do battle,
and they fled
before him.
\VS{15}When the Ammonites
saw
the Arameans
flee,
they
fled
before
Joab’s brother
Abishai
and withdrew
into the city.
Joab
went back
to Jerusalem.
\par }{\PP \VS{16}When
the Arameans
realized they had been defeated
by Israel,
they sent
for reinforcements
from beyond
the Euphrates
River, led by Shophach
the commanding general
of Hadadezer’s
army.
\VS{17}When David
was informed,
he gathered
all
Israel,
crossed
the Jordan River,
and marched
against
them. David
deployed
his army against
the Arameans
for battle
and they fought
against him.
\VS{18}The Arameans
fled
before
Israel.
David
killed
7,000
Aramean
charioteers
and 40,000
infantrymen;
he also killed
Shophach
the commanding
general.
\VS{19}When
Hadadezer’s
subjects
saw
they were defeated
by Israel,
they made peace
with
David
and became his subjects.
The Arameans
were no
longer
willing
to help
the Ammonites.

\par }\Chap{20}{\PP \VerseOne{1}In the spring,
at the time
when kings
normally conduct
wars, Joab
led
the army
into battle
and devastated
the land
of the Ammonites.
He went
and besieged
Rabbah,
while David
stayed
in Jerusalem.
Joab
defeated
Rabbah
and tore
it down.
\VS{2}David
took
the crown
from
the head
of their king
and wore
it (its weight
was a talent
of gold
and it was set
with
precious
stones). He
took a large amount
of plunder
from the city.
\VS{3}He removed
the city’s residents
and made them do hard labor
with saws,
iron
picks,
and axes.
This was his
policy
with all
the Ammonite
cities.
Then David
and all
the army
returned
to Jerusalem.
\par }{\SH Battles with the Philistines
\par }{\PP \VS{4}Later
there was
a battle
with
the Philistines
in Gezer.
At that time
Sibbekai
the Hushathite
killed
Sippai,
one of the descendants
of the Rephaim,
and the Philistines were subdued.
\par }{\PP \VS{5}There was
another
battle
with
the Philistines
in which Elhanan
son
of Jair
the Bethlehemite
killed
the brother
of Goliath
the Gittite,
whose spear
had a shaft
as big as the crossbeam
of a weaver’s loom.
\par }{\PP \VS{6}In a battle
in Gath
there was
a large man
who had six
fingers
on each hand and six toes on each foot – twenty-four in all! He too was a descendant of Rapha.
\VS{7}When he taunted
Israel,
Jonathan
son
of Shimea,
David’s
brother,
killed him.
\par }{\PP \VS{8}These
were
the descendants of Rapha
who lived in Gath;
they were killed
by the hand
of David
and his soldiers.

\par }\Chap{21}{\PP \VerseOne{1}An adversary
opposed
Israel,
inciting
David
to count
how many warriors Israel had.
\VS{2}David
told
Joab
and the leaders
of the army, “Go,
count
the number
of warriors from Beer Sheba
to
Dan.
Then bring
back a report to
me so I may know how many we have.”
\VS{3}Joab
replied,
“May the
{\ND{Lord}}
make his army
a hundred
times
larger! My master,
O king,
do not
all
of them serve
my master? Why
does my master
want
to do this? Why
bring
judgment
on Israel?”
\par }{\PP \VS{4}But the king’s
edict
stood, despite Joab’s
objections.
So Joab
left
and traveled
throughout
Israel
before returning
to Jerusalem.
\VS{5}Joab
reported
to
David
the number
of warriors.
In all
Israel
there were
1,100,000
sword-wielding
soldiers;
Judah
alone had 470,000
sword-wielding
soldiers.
\VS{6}Now Joab
did not
number
Levi
and Benjamin,
for
the king’s
edict
disgusted
him.
\VS{7}God
was also offended
by it, so he attacked
Israel.
\par }{\PP \VS{8}David
said
to
God,
“I have sinned
greatly
by doing
this! Now,
please
remove
the guilt
of your servant,
for
I have acted very
foolishly.”
\VS{9}The
{\ND{Lord}}
told
Gad,
David’s
prophet,
\VS{10}“Go,
tell
David,
‘This is what
the {\ND{Lord}}
says: “I am
offering
you three
forms
of judgment
from which
to choose.
Pick
one of them.” ’ ”
\VS{11}Gad
went
to
David
and told
him, “This is what
the {\ND{Lord}}
says: ‘Pick one of these:
\VS{12}three
years
of famine,
or
three
months
being chased
by your enemies
and struck down by their
swords,
or
three
days
being struck down by
the {\ND{Lord}}, during which
a plague
will invade the land
and the
{\ND{Lord}}’s
messenger
will destroy
throughout
Israel’s
territory.’
Now,
decide
what
I should
tell the one
who sent me.”
\VS{13}David
said
to
Gad,
“I am
very
upset! I prefer
to be attacked
by the
{\ND{Lord}}, for
his mercy
is very
great;
I do not
want to be attacked
by men!”
\VS{14}So
the {\ND{Lord}}
sent a plague
through Israel,
and 70,000
Israelite
men
died.
\par }{\PP \VS{15}God
sent
an angel
to ravage
Jerusalem.
As he was
doing
so,
the {\ND{Lord}}
watched
and relented
from his judgment.
He told
the angel
who was destroying,
“That’s
enough! Stop now!”
\par }{\PP Now the
{\ND{Lord}}’s
angel
was standing
near
the threshing floor
of Ornan
the Jebusite.
\VS{16}David
looked
up and saw
the
{\ND{Lord}}’s
messenger
standing
between
the earth
and sky
with his sword
drawn
and in his hand,
stretched
out over
Jerusalem.
David
and the leaders,
covered
with sackcloth,
threw
themselves down with their faces to the ground.
\VS{17}David
said
to
God,
“Was I
not
the one who decided
to number
the army? I
am the one who
sinned
and committed this awful
deed! As for these
sheep
– what
have they done? O
{\ND{Lord}}
my God,
attack me
and my family,
but
remove the plague
from your people!”
\par }{\PP \VS{18}So the
{\ND{Lord}}’s
messenger
told
Gad
to
instruct
David
to go up
and build
an altar
for the
{\ND{Lord}}
on the threshing floor
of Ornan
the Jebusite.
\VS{19}So David
went up
as
Gad
instructed
him to do
in the name
of the {\ND{Lord}}.
\VS{20}While Ornan
was threshing
wheat,
he turned
and saw
the messenger,
and he
and his four
sons
hid themselves.
\VS{21}When David
came
to
Ornan,
Ornan
looked
and saw
David;
he came out
from
the threshing floor
and bowed
to David
with his face
to the ground.
\VS{22}David
said
to
Ornan,
“Sell
me the threshing floor
so I can build
on it an altar
for the
{\ND{Lord}} –
I’ll pay
top price –
so that the plague
may be removed
from
the people.”
\VS{23}Ornan
told
David,
“You can have
it! My master,
the king,
may do
what he wants.
Look,
I am giving
you the oxen
for burnt sacrifices,
the threshing sledges
for wood,
and the wheat
for an offering.
I give
it all to you.”
\VS{24}King
David
replied
to Ornan,
“No,
I insist
on buying
it
for
top price.
I will not
offer to the
{\ND{Lord}}
what belongs to you or offer
a burnt sacrifice
that cost me nothing.
\VS{25}So David
bought
the place
from Ornan
for 600
pieces
of gold.
\VS{26}David
built
there
an altar
to the
{\ND{Lord}}
and offered
burnt sacrifices
and peace offerings.
He called
out to
the {\ND{Lord}}, and the
{\ND{Lord}} responded
by sending fire
from
the sky
and consuming the burnt sacrifice
on
the altar.
\VS{27}The
{\ND{Lord}}
ordered
the messenger
to put his sword
back
into its sheath.
\par }{\PP \VS{28}At that time,
when David
saw
that
the {\ND{Lord}}
responded
to him at the threshing floor
of Ornan
the Jebusite,
he sacrificed
there.
\VS{29}Now the
{\ND{Lord}}’s
tabernacle
(which
Moses
had made
in the wilderness) and the altar
for burnt sacrifices
were at that time
at the worship center
in Gibeon.
\VS{30}But David
could
not
go
before
it to seek
God’s
will, for
he was afraid
of the sword
of the
{\ND{Lord}}’s
messenger.

\par }\Chap{22}{\PP \VerseOne{1}David
then said,
“This
is
the place where the temple
of the {\ND{Lord}}
God
will be, along
with the altar
for burnt sacrifices
for Israel.”
\par }{\SH David Orders a Temple to Be Built
\par }{\PP \VS{2}David
ordered
the
resident foreigners
in the land
of Israel
to be called together.
He appointed
some of them to be stonecutters
to chisel
stones
for the building
of God’s
temple.
\VS{3}David
supplied
a large amount
of iron
for the nails
of the doors
of the gates
and for braces,
more
bronze
than could be weighed,
\VS{4}and more cedar
logs
than could
be counted.
(The Sidonians
and Tyrians
had brought a large amount
of cedar
logs
to David.)
\par }{\PP \VS{5}David
said,
“My son
Solomon
is just an inexperienced
young man,
and the temple
to be built
for the
{\ND{Lord}}
must be especially magnificent
so it will become famous
and be considered splendid
by all
the nations.
Therefore I will make preparations
for its construction.” So David
made extensive
preparations
before
he died.
\par }{\PP \VS{6}He summoned
his son
Solomon
and charged
him to build
a temple
for the
{\ND{Lord}}
God
of Israel.
\VS{7}David
said
to Solomon: “My son,
I
really wanted
to build
a temple
to honor
the {\ND{Lord}}
my God.
\VS{8}But
the {\ND{Lord}}
said
to me: ‘You have spilled
a great
deal
of blood
and fought many
battles.
You must not
build
a temple
to honor
me, for
you have spilled
a great
deal of blood
on the ground
before me.
\VS{9}Look,
you will have
a son,
who will be
a peaceful
man.
I will give him rest
from all
his enemies
on every side.
Indeed,
Solomon
will be
his name;
I will give
Israel
peace
and quiet
during his reign.
\VS{10}He will build
a temple
to honor
me; he
will become
my son,
and I
will become his father.
I
will grant
to his dynasty
permanent
rule
over
Israel.’
\par }{\PP \VS{11}“Now,
my son,
may
the {\ND{Lord}}
be with
you! May you succeed
and build
a temple
for the
{\ND{Lord}}
your God,
just
as he announced you would.
\VS{12}Only
may the
{\ND{Lord}}
give
you insight
and understanding
when he places you in charge of
Israel,
so you may obey
the law
of the {\ND{Lord}}
your God.
\VS{13}Then
you will succeed,
if
you carefully
obey
the rules
and regulations
which
the {\ND{Lord}}
ordered
Moses
to give to Israel.
Be strong
and brave! Don’t
be afraid
and don’t
panic!
\VS{14}Now, look,
I have made every effort
to supply what is needed to
build the
{\ND{Lord}}’s
temple.
I have stored up 100,000
talents
of gold,
1,000,000
talents
of silver,
and so much
bronze
and iron
it cannot
be
weighed,
as well as wood
and stones.
Feel
free to add more!
\VS{15}You also have available
many
workers,
including stonecutters,
masons, carpenters,
and an innumerable array
of workers
who are skilled
\VS{16}in using gold,
silver,
bronze,
and iron.
Get up
and begin
the work! May
the {\ND{Lord}}
be with you!”
\par }{\PP \VS{17}David
ordered
all
the officials
of Israel
to support
his son
Solomon.
\VS{18}He told them, “The
{\ND{Lord}}
your God
is with
you! He has made you secure
on every side,
for
he handed
over to me the
inhabitants
of the region
and the region
is subdued
before
the {\ND{Lord}}
and his people.
\VS{19}Now
seek
the {\ND{Lord}}
your God
wholeheartedly
and with your entire being! Get
up and build
the
sanctuary
of the {\ND{Lord}}
God! Then you can bring
the
ark
of the
{\ND{Lord}}’s
covenant
and the holy
items
dedicated to God’s
service into the temple
that is built
to honor
the {\ND{Lord}}.”

\par }\Chap{23}{\PP \VerseOne{1}When David
was old
and approaching
the end of his life,
he made
his son
Solomon
king over
Israel.
\par }{\PP \VS{2}David assembled
all
the leaders
of Israel,
along with the priests
and the Levites.
\VS{3}The Levites
who were thirty
years
old and up
were
counted;
there were 38,000
men.
\VS{4}David said, “Of these,
24,000
are to direct
the work
of the
{\ND{Lord}}’s
temple;
6,000
are to be officials
and judges;
\VS{5}4,000
are to be gatekeepers;
and 4,000
are to praise
the {\ND{Lord}}
with the instruments
I supplied for worship.”
\VS{6}David
divided
them into groups corresponding
to the sons
of Levi: Gershon,
Kohath,
and Merari.
\par }{\PP \VS{7}The Gershonites
included Ladan
and Shimei.
\par }{\PP \VS{8}The sons
of Ladan:
\par }{\PP Jehiel
the oldest,
Zetham,
and Joel
– three in all.
\par }{\PP \VS{9}The sons
of Shimei:
\par }{\PP Shelomoth,
Haziel,
and Haran
– three
in all.
\par }{\PP These
were the leaders
of the family
of Ladan.
\par }{\PP \VS{10}The sons
of Shimei:
\par }{\PP Jahath,
Zina,
Jeush,
and Beriah.
These
were Shimei’s
sons
– four in all.
\VS{11}Jahath
was
the oldest
and Zizah
the second
oldest. Jeush
and Beriah
did not
have many
sons,
so
they were considered one family
with one
responsibility.
\par }{\PP \VS{12}The sons
of Kohath:
\par }{\PP Amram,
Izhar,
Hebron,
and Uzziel
– four in all.
\par }{\PP \VS{13}The sons
of Amram:
\par }{\PP Aaron
and Moses.
\par }{\PP Aaron
and his descendants
were chosen on a permanent
basis
to consecrate
the most
holy items,
to
offer sacrifices
before
the {\ND{Lord}}, to serve
him, and to praise
his name.
\VS{14}The descendants
of Moses
the man
of God
were considered
Levites.
\par }{\PP \VS{15}The sons
of Moses:
\par }{\PP Gershom
and Eliezer.
\par }{\PP \VS{16}The son
of Gershom:
\par }{\PP Shebuel
the oldest.
\par }{\PP \VS{17}The son
of Eliezer
was Rehabiah,
the oldest.
Eliezer
had no
other
sons,
but Rehabiah
had many
descendants.
\par }{\PP \VS{18}The son
of Izhar:
\par }{\PP Shelomith
the oldest.
\par }{\PP \VS{19}The sons
of Hebron:
\par }{\PP Jeriah
the oldest,
Amariah
the second,
Jahaziel
the third,
and Jekameam
the fourth.
\par }{\PP \VS{20}The sons
of Uzziel:
\par }{\PP Micah
the oldest,
and Isshiah
the second.
\par }{\PP \VS{21}The sons
of Merari:
\par }{\PP Mahli
and Mushi.
\par }{\PP The sons
of Mahli:
\par }{\PP Eleazar
and Kish.
\par }{\PP \VS{22}Eleazar
died
without
having sons;
he had
only
daughters.
The sons
of Kish,
their cousins,
married them.
\par }{\PP \VS{23}The sons
of Mushi:
\par }{\PP Mahli,
Eder,
and Jeremoth
– three in all.
\par }{\PP \VS{24}These
were the descendants
of Levi
according to their families,
that is, the leaders
of families
as counted
and individually
listed who carried out
assigned
tasks
in the
{\ND{Lord}}’s
temple
and were twenty
years
old
and up.
\VS{25}For
David
said,
“The
{\ND{Lord}}
God
of Israel
has given his people
rest
and has permanently
settled
in Jerusalem.
\VS{26}So
the Levites
no
longer need to carry
the tabernacle
or any
of the items
used in its service.”
\VS{27}According to David’s
final
instructions,
the Levites
twenty
years
old and up were
counted.
\par }{\PP \VS{28}Their job was to help Aaron’s
descendants
in the service
of the
{\ND{Lord}}’s
temple.
They were to take care of the courtyards,
the rooms,
ceremonial
purification
of all
holy
items, and other jobs
related to the service
of God’s
temple.
\VS{29}They also took care of the bread
that is displayed,
the flour
for offerings,
the unleavened
wafers,
the round cakes,
the mixing,
and all
the measuring.
\VS{30}They also
stood
in a designated place every morning
and offered thanks
and praise
to the
{\ND{Lord}}. They also
did this
in the evening
\VS{31}and whenever
burnt sacrifices
were offered
to the
{\ND{Lord}}
on the Sabbath
and at new moon
festivals
and assemblies.
A designated number
were to serve before
the {\ND{Lord}}
regularly
in accordance with regulations.
\VS{32}They were in charge
of the meeting
tent
and the
holy
place, and helped
their relatives,
the descendants
of Aaron,
in the service
of the
{\ND{Lord}}’s
temple.

\par }\Chap{24}{\PP \VerseOne{1}The divisions
of Aaron’s
descendants
were as follows:
\par }{\PP The sons
of Aaron:
\par }{\PP Nadab,
Abihu,
Eleazar,
and Ithamar.
\par }{\PP \VS{2}Nadab
and Abihu
died
before
their father
did; they
had no
sons.
Eleazar
and Ithamar
served as priests.
\par }{\PP \VS{3}David,
Zadok
(a descendant
of Eleazar), and Ahimelech
(a descendant
of Ithamar) divided
them into groups to carry out their
assigned responsibilities.
\VS{4}The descendants
of Eleazar
had more
leaders
than
the descendants
of Ithamar,
so they divided
them up
accordingly; the descendants
of Eleazar
had sixteen
leaders,
while the descendants
of Ithamar
had eight.
\VS{5}They divided
them by lots,
for
there
were
officials
of the holy
place and officials
designated by God
among the descendants
of both Eleazar
and Ithamar.
\VS{6}The scribe
Shemaiah
son
of Nethanel,
a Levite,
wrote down
their names before
the king,
the officials,
Zadok
the priest,
Ahimelech
son
of Abiathar,
and the leaders
of the priestly
and Levite
families. One
family
was drawn
by lot from Eleazar,
and then the next from Ithamar.
\par }{\PP \VS{7}The first
lot
went to Jehoiarib,
\par }{\PP the second
to Jedaiah,
\par }{\PP \VS{8}the third
to Harim,
\par }{\PP the fourth
to Seorim,
\par }{\PP \VS{9}the fifth
to Malkijah,
\par }{\PP the sixth
to Mijamin,
\par }{\PP \VS{10}the seventh
to Hakkoz,
\par }{\PP the eighth
to Abijah,
\par }{\PP \VS{11}the ninth
to Jeshua,
\par }{\PP the tenth
to Shecaniah,
\par }{\PP \VS{12}the eleventh
to Eliashib,
\par }{\PP the twelfth
to Jakim,
\par }{\PP \VS{13}the thirteenth
to Huppah,
\par }{\PP the fourteenth
to Jeshebeab,
\par }{\PP \VS{14}the fifteenth
to Bilgah,
\par }{\PP the sixteenth
to Immer,
\par }{\PP \VS{15}the seventeenth
to Hezir,
\par }{\PP the eighteenth
to Happizzez,
\par }{\PP \VS{16}the nineteenth
to Pethahiah,
\par }{\PP the twentieth
to Jehezkel,
\par }{\PP \VS{17}the twenty-first
to Jakin,
\par }{\PP the twenty-second
to Gamul,
\par }{\PP \VS{18}the twenty-third
to Delaiah,
\par }{\PP the twenty-fourth
to Maaziah.
\par }{\PP \VS{19}This
was the order
in which they carried out their assigned responsibilities
when they entered
the
{\ND{Lord}}’s
temple,
according to the regulations
given them by
their ancestor
Aaron,
just
as the
{\ND{Lord}}
God
of Israel
had instructed him.
\par }{\SH Remaining Levites
\par }{\PP \VS{20}The rest
of the Levites
included:
\par }{\PP Shubael
from the sons
of Amram,
\par }{\PP Jehdeiah
from the sons
of Shubael,
\par }{\PP \VS{21}the firstborn Isshiah
from Rehabiah
and the sons
of Rehabiah,
\par }{\PP \VS{22}Shelomoth
from the Izharites,
\par }{\PP Jahath
from the sons
of Shelomoth.
\par }{\PP \VS{23}The sons
of Hebron:
\par }{\PP Jeriah,
Amariah
the second,
Jahaziel
the third,
and Jekameam
the fourth.
\par }{\PP \VS{24}The son
of Uzziel:
\par }{\PP Micah;
\par }{\PP Shamir
from the sons
of Micah.
\par }{\PP \VS{25}The brother
of Micah:
\par }{\PP Isshiah.
\par }{\PP Zechariah
from the sons
of Isshiah.
\par }{\PP \VS{26}The sons
of Merari:
\par }{\PP Mahli
and Mushi.
\par }{\PP The son
of Jaaziah:
\par }{\PP Beno.
\par }{\PP \VS{27}The sons
of Merari,
from Jaaziah:
\par }{\PP Beno, Shoham,
Zaccur,
and Ibri.
\par }{\PP \VS{28}From Mahli:
\par }{\PP Eleazar,
who had no
sons.
\par }{\PP \VS{29}From Kish:
\par }{\PP Jerahmeel.
\par }{\PP \VS{30}The sons
of Mushi:
\par }{\PP Mahli,
Eder,
and Jerimoth.
\par }{\PP These
were the Levites,
listed
by their families.
\par }{\PP \VS{31}Just like their relatives,
the descendants
of Aaron,
they
also
cast
lots
before
King
David,
Zadok,
Ahimelech,
the leaders
of families,
the priests,
and the Levites.
The families
of the oldest
son cast lots along with
the those of the youngest.

\par }\Chap{25}{\PP \VerseOne{1}David
and the army
officers
selected
some of the sons
of Asaph,
Heman,
and Jeduthun
to prophesy
as they played stringed instruments
and cymbals.
The following
men
were assigned this responsibility:
\par }{\PP \VS{2}From the sons
of Asaph: Zaccur,
Joseph,
Nethaniah,
and Asarelah.
The sons
of Asaph
were supervised
by Asaph,
who prophesied
under the king’s
supervision.
\par }{\PP \VS{3}From the sons
of Jeduthun: Gedaliah,
Zeri,
Jeshaiah,
Hashabiah,
and Mattithiah
– six
in all, under supervision
of their father
Jeduthun,
who prophesied
as he played a harp,
giving thanks
and praise
to the
{\ND{Lord}}.
\par }{\PP \VS{4}From the sons
of Heman: Bukkiah,
Mattaniah,
Uzziel,
Shebuel,
Jerimoth,
Hananiah,
Hanani,
Eliathah,
Giddalti,
Romamti-Ezer,
Joshbekashah,
Mallothi,
Hothir,
and Mahazioth.
\VS{5}All
these
were the sons
of Heman,
the king’s
prophet.
God
had promised him these sons in order to make him prestigious.
God
gave
Heman
fourteen
sons
and three
daughters.
\par }{\PP \VS{6}All
of these
were under the supervision
of their fathers;
they were musicians
in the
{\ND{Lord}}’s
temple,
playing cymbals
and stringed instruments
as they served
in God’s
temple.
Asaph,
Jeduthun,
and Heman
were under the supervision
of the king.
\VS{7}They and their relatives,
all
of them skilled
and trained
to make music
to the
{\ND{Lord}}, numbered
two hundred
eighty-eight.
\par }{\PP \VS{8}They cast
lots
to determine their responsibilities
– oldest
as well as youngest,
teacher as well as student.
\par }{\PP \VS{9}The first
lot
went
to Asaph’s
son Joseph
and his relatives
and sons
– twelve in all,
\par }{\PP the second to Gedaliah and his relatives and sons – twelve in all,
\par }{\PP \VS{10}the third
to Zaccur
and his sons
and relatives
– twelve in all,
\par }{\PP \VS{11}the fourth
to Izri
and his sons
and relatives
– twelve in all,
\par }{\PP \VS{12}the fifth
to Nethaniah
and his sons
and relatives
– twelve in all,
\par }{\PP \VS{13}the sixth
to Bukkiah
and his sons
and relatives
– twelve in all,
\par }{\PP \VS{14}the seventh
to Jesharelah
and his sons
and relatives
– twelve in all,
\par }{\PP \VS{15}the eighth
to Jeshaiah
and his sons
and relatives
– twelve in all,
\par }{\PP \VS{16}the ninth
to Mattaniah
and his sons
and relatives
– twelve in all,
\par }{\PP \VS{17}the tenth
to Shimei
and his sons
and relatives
– twelve in all,
\par }{\PP \VS{18}the eleventh
to Azarel
and his sons
and relatives
– twelve in all,
\par }{\PP \VS{19}the twelfth
to Hashabiah
and his sons
and relatives
– twelve in all,
\par }{\PP \VS{20}the thirteenth
to Shubael
and his sons
and relatives
– twelve in all,
\par }{\PP \VS{21}the fourteenth
to Mattithiah
and his sons
and relatives
– twelve in all,
\par }{\PP \VS{22}the fifteenth
to Jerimoth
and his sons
and relatives
– twelve in all,
\par }{\PP \VS{23}the sixteenth
to Hananiah
and his sons
and relatives
– twelve in all,
\par }{\PP \VS{24}the seventeenth
to Joshbekashah
and his sons
and relatives
– twelve in all,
\par }{\PP \VS{25}the eighteenth
to Hanani
and his sons
and relatives
– twelve in all,
\par }{\PP \VS{26}the nineteenth
to Mallothi
and his sons
and relatives
– twelve in all,
\par }{\PP \VS{27}the twentieth
to Eliathah
and his sons
and relatives
– twelve in all,
\par }{\PP \VS{28}the twenty-first
to Hothir
and his sons
and relatives
– twelve in all,
\par }{\PP \VS{29}the twenty-second
to Giddalti
and his sons
and relatives
– twelve in all,
\par }{\PP \VS{30}the twenty-third
to Mahazioth
and his sons
and relatives
– twelve in all,
\par }{\PP \VS{31}the twenty-fourth
to Romamti-Ezer
and his sons
and relatives
– twelve in all.

\par }\Chap{26}{\PP \VerseOne{1}The divisions
of the gatekeepers:
\par }{\PP From the Korahites: Meshelemiah,
son
of Kore,
one of the sons
of Asaph.
\par }{\PP \VS{2}Meshelemiah’s
sons:
\par }{\PP The firstborn
Zechariah,
the second
Jediael,
the third
Zebadiah,
the fourth
Jathniel,
\VS{3}the fifth
Elam,
the sixth
Jehohanan,
and the seventh
Elihoenai.
\par }{\PP \VS{4}Obed-Edom’s
sons:
\par }{\PP The firstborn
Shemaiah,
the second
Jehozabad,
the third
Joah,
the fourth
Sakar,
the fifth
Nethanel,
\VS{5}the sixth
Ammiel,
the seventh
Issachar,
and the eighth
Peullethai.
(Indeed,
God
blessed Obed-Edom.)
\par }{\PP \VS{6}His son
Shemaiah
also had sons,
who were leaders
of their families,
for
they
were highly
respected.
\VS{7}The sons
of Shemaiah:
\par }{\PP Othni,
Rephael,
Obed,
and Elzabad.
His relatives
Elihu
and Semakiah
were also respected.
\par }{\PP \VS{8}All
these
were the descendants
of Obed-Edom.
They
and their sons
and relatives
were respected men,
capable
of doing their responsibilities.
There were sixty-two
of them related to Obed-Edom.
\par }{\PP \VS{9}Meshelemiah
had sons
and relatives
who were respected
– eighteen in all.
\par }{\PP \VS{10}Hosah,
one of the descendants
of Merari,
had sons:
\par }{\PP The firstborn Shimri
(he was not
actually
the firstborn,
but
his father gave him that status),
\VS{11}the second
Hilkiah,
the third
Tebaliah,
and the fourth
Zechariah.
All
of Hosah’s
sons
and relatives
numbered thirteen.
\par }{\PP \VS{12}These
divisions
of the gatekeepers,
corresponding to their leaders,
had assigned responsibilities,
like
their relatives,
as they served
in the
{\ND{Lord}}’s
temple.
\par }{\PP \VS{13}They cast
lots,
both young
and old,
according to their families,
to determine which gate they would be responsible for.
\VS{14}The lot
for the east
gate went
to Shelemiah.
They then cast
lots
for his son
Zechariah,
a wise
adviser,
and the lot
for the north
gate went to him.
\VS{15}Obed-Edom
was assigned the south
gate, and his sons
were assigned
the storehouses.
\VS{16}Shuppim
and Hosah
were assigned the west
gate,
along with
the Shalleketh
gate on the upper road. One guard was adjacent to another.
\VS{17}Each day
there were six
Levites
posted on the east,
four
on the north,
and four
on the south.
At the storehouses
they were posted in pairs.
\VS{18}At the court
on the west
there were four
posted on the road
and two
at the court.
\VS{19}These
were the divisions
of the gatekeepers
who were descendants
of Korah
and Merari.
\par }{\SH Supervisors of the Storehouses
\par }{\PP \VS{20}Their fellow Levites
were in charge
of the storehouses
in God’s
temple
and the storehouses
containing consecrated items.
\VS{21}The descendants
of Ladan,
who were descended
from Gershon
through Ladan
and were leaders
of the families
of Ladan
the Gershonite,
included Jehieli
\VS{22}and the sons
of Jehieli,
Zetham
and his brother
Joel.
They were in charge
of the storehouses
in the
{\ND{Lord}}’s
temple.
\par }{\PP \VS{23}As for the Amramites,
Izharites,
Hebronites,
and Uzzielites:
\par }{\PP \VS{24}Shebuel
son
of Gershom,
the son
of Moses,
was the supervisor
of the storehouses.
\VS{25}His relatives
through Eliezer
included: Rehabiah
his son,
Jeshaiah
his son,
Joram
his son,
Zikri
his son,
and Shelomith
his son.
\VS{26}Shelomith
and his relatives
were in charge of
all
the storehouses
containing the consecrated items
dedicated
by King
David,
the family
leaders
who led units
of a thousand
and a hundred,
and the army
officers.
\VS{27}They had dedicated
some
of the plunder
taken in battles
to be used for repairs
on the
{\ND{Lord}}’s
temple.
\VS{28}They were also in charge of everything
dedicated
by Samuel
the prophet,
Saul
son
of Kish,
Abner
son
of Ner,
and Joab
son
of Zeruiah;
Shelomith
and
his relatives
were in charge of everything
that had been dedicated.
\par }{\PP \VS{29}As for the Izharites: Kenaniah
and his sons
were given responsibilities
outside
the temple as officers
and judges
over
Israel.
\par }{\PP \VS{30}As for the Hebronites: Hashabiah
and his relatives,
1,700
respected men,
were assigned responsibilities
in
Israel
west
of the Jordan;
they did the
{\ND{Lord}}’s
work
and the king’s
service.
\par }{\PP \VS{31}As for the Hebronites: Jeriah
was the leader
of the Hebronites
according to the genealogical records.
In the fortieth
year
of David’s
reign,
they examined
the records and discovered
there were highly
respected men
in Jazer
in Gilead.
\VS{32}Jeriah had 2,700
relatives
who were respected
family
leaders.
King
David
placed them in
charge of the Reubenites,
the Gadites,
and the half-tribe
of Manasseh;
they took care of all
matters pertaining
to God
and the king.

\par }\Chap{27}{\PP \VerseOne{1}What follows is a list
of Israelite
family
leaders
and commanders of units
of a thousand
and a hundred,
as well as their officers
who served
the
king
in various
matters.
Each division
was assigned
to serve for
one month
during
the year;
each
consisted
of 24,000 men.
\par }{\PP \VS{2}Jashobeam
son
of Zabdiel
was in charge of the first
division,
which was assigned the first
month.
His division
consisted of 24,000 men.
\VS{3}He was a descendant
of Perez;
he was in charge
of all
the army
officers
for the first
month.
\par }{\PP \VS{4}Dodai
the Ahohite
was in
charge of the division
assigned the second
month;
Mikloth
was the next in rank.
His division
consisted
of 24,000 men.
\par }{\PP \VS{5}The third
army
commander,
assigned the third
month,
was Benaiah
son
of Jehoiada
the priest.
He was the leader
of his division,
which consisted
of 24,000 men.
\VS{6}Benaiah
was the leader of the thirty
warriors
and his division;
his son
was Ammizabad.
\par }{\PP \VS{7}The fourth,
assigned the fourth
month,
was Asahel,
brother
of Joab;
his son
Zebadiah
succeeded
him. His division
consisted
of 24,000 men.
\par }{\PP \VS{8}The fifth,
assigned the fifth
month,
was the commander
Shamhuth
the Izrahite.
His division
consisted of 24,000 men.
\par }{\PP \VS{9}The sixth,
assigned the sixth
month,
was Ira
son
of Ikkesh
the Tekoite.
His division
consisted
of 24,000 men.
\par }{\PP \VS{10}The seventh,
assigned the seventh
month,
was Helez
the Pelonite,
an Ephraimite.
His division
consisted
of 24,000 men.
\par }{\PP \VS{11}The eighth,
assigned the eighth
month,
was Sibbekai
the Hushathite,
a Zerahite.
His division
consisted
of 24,000 men.
\par }{\PP \VS{12}The ninth,
assigned the ninth
month,
was Abiezer
the Anathothite,
a Benjaminite.
His division
consisted
of 24,000 men.
\par }{\PP \VS{13}The tenth,
assigned the tenth
month,
was Maharai
the Netophathite,
a Zerahite.
His division
consisted
of 24,000 men.
\par }{\PP \VS{14}The eleventh,
assigned the eleventh
month,
was Benaiah
the Pirathonite,
an Ephraimite.
His division
consisted
of 24,000 men.
\par }{\PP \VS{15}The twelfth,
assigned the twelfth
month,
was Heldai
the Netophathite,
a descendant of Othniel.
His division
consisted
of 24,000 men.
\par }{\PP \VS{16}The officers of the Israelite
tribes:
\par }{\PP Eliezer
son
of Zikri
was the leader
of the Reubenites,
\par }{\PP Shephatiah
son
of Maacah
led the Simeonites,
\par }{\PP \VS{17}Hashabiah
son
of Kemuel
led the Levites,
\par }{\PP Zadok
led the descendants of Aaron,
\par }{\PP \VS{18}Elihu,
a brother
of David,
led Judah,
\par }{\PP Omri
son
of Michael
led Issachar,
\par }{\PP \VS{19}Ishmaiah
son
of Obadiah
led Zebulun,
\par }{\PP Jerimoth
son
of Azriel
led Naphtali,
\par }{\PP \VS{20}Hoshea
son
of Azaziah
led the Ephraimites,
\par }{\PP Joel
son
of Pedaiah
led the half-tribe
of Manasseh,
\par }{\PP \VS{21}Iddo
son
of Zechariah
led the half-tribe
of Manasseh
in Gilead,
\par }{\PP Jaasiel
son
of Abner
led Benjamin,
\par }{\PP \VS{22}Azarel
son
of Jeroham
led Dan.
\par }{\PP These
were the commanders
of the Israelite
tribes.
\par }{\PP \VS{23}David
did not
count
the males
twenty
years
old and under,
for
the
{\ND{Lord}}
had promised
to make
Israel
as numerous
as the stars
in the sky.
\VS{24}Joab
son
of Zeruiah
started
to count
the men but did not
finish.
God was angry
with Israel
because of this,
so
the number
was not
recorded in the scroll called The Annals
of King
David.
\par }{\SH Royal Officials
\par }{\PP \VS{25}Azmaveth
son
of Adiel
was in charge of
the king’s
storehouses;
\par }{\PP Jonathan
son
of Uzziah
was in charge of
the storehouses
in the field,
in the cities,
in the towns,
and in the towers.
\par }{\PP \VS{26}Ezri
son
of Kelub
was in charge
of the field
workers
who farmed
the land.
\par }{\PP \VS{27}Shimei
the Ramathite
was in charge
of the vineyards;
\par }{\PP Zabdi
the Shiphmite
was in charge of the wine
stored
in the vineyards.
\par }{\PP \VS{28}Baal-Hanan
the Gederite
was in
charge of the olive
and sycamore
trees in the lowlands;
\par }{\PP Joash
was in
charge of the storehouses
of olive oil.
\par }{\PP \VS{29}Shitrai
the Sharonite
was in charge
of the cattle
grazing
in Sharon;
\par }{\PP Shaphat
son
of Adlai
was in charge
of the cattle
in the valleys.
\par }{\PP \VS{30}Obil
the Ishmaelite
was in charge
of the camels;
\par }{\PP Jehdeiah
the Meronothite
was in charge
of the donkeys.
\par }{\PP \VS{31}Jaziz
the Hagrite
was in charge
of the sheep.
\par }{\PP All
these
were the officials
in charge of King
David’s
property.
\par }{\PP \VS{32}Jonathan,
David’s
uncle,
was a wise adviser
and scribe;
\par }{\PP Jehiel
son
of Hacmoni
cared for the king’s
sons.
\par }{\PP \VS{33}Ahithophel
was the king’s
adviser;
\par }{\PP Hushai
the Arkite
was the king’s
confidant.
\par }{\PP \VS{34}Ahithophel
was succeeded
by Jehoiada
son
of Benaiah
and by Abiathar.
\par }{\PP Joab
was the commanding general
of the king’s
army.

\par }\Chap{28}{\PP \VerseOne{1}David
assembled
in Jerusalem
all
the officials
of Israel,
including the commanders
of the tribes,
the commanders
of the army divisions
that served
the king,
the commanders
of units of a thousand
and a hundred,
the officials
who were in charge
of all
the property
and livestock
of the king
and his sons,
the eunuchs,
and the warriors,
including the most skilled of them.
\par }{\PP \VS{2}King
David
rose
to his feet
and said: “Listen
to me, my brothers
and my people.
I
wanted
to build
a temple
where the ark
of the
{\ND{Lord}}’s
covenant
could be placed as a footstool
for our God.
I have made the preparations
for building it.
\VS{3}But God
said
to me, ‘You must not
build
a temple
to honor
me, for
you are
a warrior
and have spilled
blood.’
\VS{4}The
{\ND{Lord}}
God
of Israel
chose
me out of my father’s
entire
family
to become
king
over
Israel
and have a permanent
dynasty. Indeed,
he chose
Judah
as leader,
and my father’s
family within
Judah,
and then he picked
me out from among my father’s
sons
and made me king
over
all
Israel.
\VS{5}From all
the many
sons
the {\ND{Lord}}
has given
me, he chose
Solomon
my son
to rule
on his
behalf over
Israel.
\VS{6}He said
to me, ‘Solomon
your son
is
the one who will build
my temple
and my courts,
for
I have chosen
him to become my son
and I
will become
his father.
\VS{7}I
will establish
his kingdom
permanently,
if
he remains
committed
to obeying my commands
and regulations,
as you are doing
this
day.’
\VS{8}So now,
in the sight
of all
Israel,
the
{\ND{Lord}}’s
assembly,
and in the hearing
of our God,
I say this: Carefully
observe all
the commands
of the {\ND{Lord}}
your God,
so that you may
possess
this good
land
and may leave it as a permanent
inheritance
for your children
after you.
\par }{\PP \VS{9}“And you,
Solomon
my son,
obey
the
God
of your father
and serve
him with a submissive
attitude
and a willing
spirit,
for
the {\ND{Lord}}
examines
all
minds
and understands
every
motive
of one’s thoughts.
If
you seek
him, he will let you find
him, but if
you abandon
him, he will reject
you permanently.
\VS{10}Realize
now
that
the {\ND{Lord}}
has chosen
you to build
a temple
as his sanctuary.
Be strong
and do it!”
\par }{\PP \VS{11}David
gave
to his son
Solomon
the blueprints
for the temple
porch,
its buildings, its treasuries,
its upper
areas, its inner
rooms,
and the room
for atonement.
\VS{12}He gave him the blueprints
of all
he envisioned
for the courts
of the
{\ND{Lord}}’s
temple,
all
the surrounding
rooms,
the storehouses
of God’s
temple,
and the storehouses
for the holy items.
\par }{\PP \VS{13}He gave him the regulations for the divisions
of priests
and Levites,
for all
the assigned
responsibilities
within the
{\ND{Lord}}’s
temple,
and for all
the items
used in the service
of the
{\ND{Lord}}’s
temple.
\par }{\PP \VS{14}He gave him the prescribed weight
for all
the gold
items
to be used in various
types of service
in the
{\ND{Lord}}’s
temple, for all
the silver
items
to be used in various
types
of service,
\VS{15}for
the gold
lampstands
and their gold
lamps,
including the weight
of each
lampstand
and its lamps,
for the silver
lampstands,
including the weight
of each lampstand
and its lamps,
according to the prescribed use
of each
lampstand,
\VS{16}for the
gold
used in the display
tables,
including the amount
to be used in each table,
for the silver
to be used in the silver
tables,
\VS{17}for the pure
gold
used for the meat forks,
bowls,
and jars,
for the small gold
bowls,
including the weight
for each
bowl,
for the small silver
bowls,
including the weight
for each
bowl,
\VS{18}and for the refined
gold
of the incense
altar.
\par }{\PP He gave him the blueprint
for the seat
of the gold
cherubim
that spread
their wings and provide shelter
for the ark
of the
{\ND{Lord}}’s
covenant.
\par }{\PP \VS{19}David said, “All
of this I put in writing
as the
{\ND{Lord}}
directed me and gave me insight
regarding the details
of the blueprints.”
\par }{\PP \VS{20}David
said
to his son
Solomon: “Be strong
and brave! Do
it! Don’t
be afraid
and don’t
panic! For
the {\ND{Lord}}
God,
my God,
is with
you. He will not
leave
you or
abandon
you before
all
the work
for the service
of the
{\ND{Lord}}’s
temple
is finished.
\VS{21}Here
are the divisions
of the priests
and Levites
who will perform all
the service
of God’s
temple.
All
the willing
and skilled
men are ready to assist you in all
the work
and perform
their service.
The officials
and all
the people
are ready to follow your instructions.”

\par }\Chap{29}{\PP \VerseOne{1}King
David
said
to the entire
assembly: “My son
Solomon,
the one
whom God
has chosen,
is just an inexperienced
young man,
and the task
is great,
for
this palace
is not
for man,
but for
the {\ND{Lord}}
God.
\VS{2}So I have made every
effort
to provide
what is needed
for the temple
of my God,
including the gold,
silver,
bronze,
iron,
wood,
as well as a large amount
of onyx,
settings
of antimony
and other stones,
all
kinds of precious
stones,
and alabaster.
\VS{3}Now, to show
my commitment
to the temple
of my God,
I donate
my personal treasure
of gold
and silver
to the temple
of my God,
in addition
to all
that I have already supplied
for this holy
temple.
\VS{4}This includes 3,000
talents
of gold
from Ophir
and 7,000
talents
of refined silver
for overlaying
the walls
of the buildings,
\VS{5}for gold
and silver
items, and for all
the work
of the craftsmen.
Who
else wants to contribute
to the
{\ND{Lord}}
today?”
\par }{\PP \VS{6}The leaders
of the families,
the leaders
of the Israelite
tribes,
the commanders
of units of a thousand
and a hundred,
and the supervisors
of the king’s
work
contributed willingly.
\VS{7}They donated
for the service
of God’s
temple
5,000
talents
and ten thousand
darics
of gold,
10,000
talents
of silver,
18,000
talents
of bronze,
and 100,000
talents
of iron.
\VS{8}All who possessed
precious
stones
donated them
to the treasury
of the
{\ND{Lord}}’s
temple,
which was under the supervision
of Jehiel
the Gershonite.
\VS{9}The people
were delighted
with
their donations,
for
they contributed
to the
{\ND{Lord}}
with a willing attitude;
King
David
was also
very
happy.
\par }{\SH David Praises the Lord
\par }{\PP \VS{10}David
praised
the {\ND{Lord}}
before the entire
assembly:
\par }{\PP “O
{\ND{Lord}}
God
of our father
Israel,
you deserve praise
forevermore!
\VS{11}O
{\ND{Lord}}, you are great,
mighty,
majestic,
magnificent,
glorious,
and sovereign over all
the sky
and earth! You
have dominion
and exalt
yourself as the ruler
of all.
\VS{12}You are
the source of wealth
and honor;
you
rule
over all.
You possess strength
and might
to magnify
and give strength
to all.
\VS{13}Now,
our God,
we
give thanks
to you and praise
your majestic
name!
\par }{\PP \VS{14}“But
who
am
I and who
are my people,
that
we should be
in a position
to contribute
this
much? Indeed,
everything
comes from
you, and we have simply
given
back to you what is yours.
\VS{15}For
we
are resident foreigners
and nomads
in your presence,
like all
our ancestors;
our days
are like a shadow
on
the earth,
without
security.
\VS{16}O
{\ND{Lord}}
our God,
all
this
wealth,
which
we have collected
to build
a temple
for you to honor
your holy
name, comes from you; it
all belongs to you.
\VS{17}I know,
my God,
that
you
examine
thoughts
and are pleased
with integrity.
With pure
motives
I
contribute
all
this;
and now
I look
with joy
as your people
who have gathered
here
contribute to you.
\VS{18}O
{\ND{Lord}}
God
of our ancestors
Abraham,
Isaac,
and Israel,
maintain
the motives
of your people
and keep
them devoted
to you.
\VS{19}Make
my son
Solomon
willing
to obey
your commands,
rules,
and regulations,
and to complete building
the palace
for which
I have made preparations.”
\par }{\PP \VS{20}David
told
the entire
assembly: “Praise
the

{\ND{Lord}}
your God!” So
the entire
assembly
praised
the {\ND{Lord}}
God
of their ancestors;
they bowed
down and stretched out flat on the ground before
the {\ND{Lord}}
and the king.
\par }{\SH David Designates Solomon King
\par }{\PP \VS{21}The next
day
they made sacrifices
and offered
burnt sacrifices
to the
{\ND{Lord}}
(1,000
bulls,
1,000
rams,
1,000
lambs), along with their accompanying drink offerings
and many
other sacrifices for all
Israel.
\VS{22}They held
a feast
before
the {\ND{Lord}}
that day
and celebrated.
\par }{\PP Then they designated Solomon,
David’s
son,
as king
a second
time; before the
{\ND{Lord}}
they anointed
him as ruler
and Zadok
as priest.
\VS{23}Solomon
sat
on
the
{\ND{Lord}}’s
throne
as king
in place
of his father
David;
he was successful
and all
Israel
was loyal to him.
\VS{24}All
the officers
and warriors,
as well
as all
of King
David’s
sons,
pledged
their allegiance
to King
Solomon.
\VS{25}The
{\ND{Lord}}
greatly magnified
Solomon
before
all
Israel
and bestowed
on
him greater
majesty
than any
king
of Israel
before him.
\par }{\SH David’s Reign Comes to an End
\par }{\PP \VS{26}David
son
of Jesse
reigned
over
all
Israel.
\VS{27}He reigned
over
Israel
forty
years;
he reigned
in Hebron
seven
years
and in Jerusalem
thirty-three
years.
\VS{28}He died
at a good
old age,
having enjoyed
long life,
wealth,
and honor.
His son
Solomon
succeeded him.
\VS{29}King
David’s
accomplishments, from start
to finish,
are recorded
in the Annals
of Samuel
the prophet,
the Annals
of Nathan
the prophet,
and the Annals
of Gad the prophet.
\VS{30}Recorded there are all
the facts about his reign
and accomplishments,
and an account of the events that
involved him,
Israel,
and all
the neighboring
kingdoms.
\par }