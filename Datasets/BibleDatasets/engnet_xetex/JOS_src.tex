\NormalFont\ShortTitle{Joshua}
{\MT Joshua

\par }\ChapOne{1}{\SH The
{\ND{Lord}} Commissions Joshua
\par }{\PP \VerseOne{1}After
Moses
the
{\ND{Lord}}’s
servant
died,
the {\ND{Lord}}
said
to
Joshua
son
of Nun,
Moses’
assistant:
\VS{2}“Moses
my servant
is dead.
Get
ready! Cross
the Jordan River! Lead these
people
into
the land
which
I am
ready to hand over to them.
\VS{3}I am handing over
to you every
place
you set
foot,
as
I promised
Moses.
\VS{4}Your territory
will extend
from the wilderness
in the south to Lebanon
in the north. It will extend all the way
to the great
River
Euphrates
in the east (including all
of Syria) and all the way
to the Mediterranean
Sea
in the west.
\VS{5}No
one will be able to resist
you all
the days
of your life.
As I was
with
Moses,
so I will be
with
you. I will not
abandon
you or
leave you alone.
\VS{6}Be strong
and brave! You
must lead these people
in the conquest
of this
land
that
I solemnly promised
their ancestors
I would hand over to them.
\VS{7}Make sure
you are very
strong
and brave! Carefully obey
all
the law
my servant
Moses
charged
you to keep! Do not
swerve
from
it to the right
or to the left,
so that
you may be successful
in all
you do.
\VS{8}This
law
scroll
must not
leave
your lips! You must memorize
it day
and night
so
you can carefully
obey
all
that is written
in it. Then
you will prosper
and be successful.
\VS{9}I repeat,
be strong
and brave! Don’t
be afraid
and don’t
panic,
for
I, the
{\ND{Lord}}
your God,
am with
you in all
you do.”
\par }{\SH Joshua Prepares for the Invasion
\par }{\PP \VS{10}Joshua
instructed
the leaders
of the people:
\VS{11}“Go
through
the camp
and command
the people,
‘Prepare
your supplies,
for
within three
days
you
will cross
the
Jordan River
and begin
the conquest
of the land
the {\ND{Lord}}
your God
is ready to hand over
to you.’ ”
\par }{\PP \VS{12}Joshua
told
the Reubenites,
Gadites,
and the half
tribe
of Manasseh:
\VS{13}“Remember
what
Moses
the
{\ND{Lord}}’s
servant
commanded
you. The

{\ND{Lord}}
your God
is giving
you a place to settle
and is handing
this
land
over
to you.
\VS{14}Your wives,
children
and cattle
may stay
in the land
that
Moses
assigned
to you
east of the Jordan River.
But all
you
warriors
must cross
over armed for battle
ahead
of your brothers.
You must help
them
\VS{15}until
the {\ND{Lord}}
gives
your brothers
a place like yours to settle and they
conquer
the land
the {\ND{Lord}}
your God
is ready to hand over
to them. Then you may go back
to your allotted
land
and occupy
the land Moses
the
{\ND{Lord}}’s
servant
assigned
you east
of the Jordan.”
\par }{\PP \VS{16}They told
Joshua,
“We will do
everything
you say. We will go
wherever
you send us.
\VS{17}Just
as we obeyed
Moses,
so
we will obey
you. But
may
the {\ND{Lord}}
your God
be with
you as
he was
with
Moses!
\VS{18}Any
man
who
rebels
against what you say
and does not
obey
all
your commands
will be executed.
But
be strong
and brave!”

\par }\Chap{2}{\PP \VerseOne{1}Joshua
son
of Nun
sent
two
spies out
from
Shittim
secretly
and instructed
them: “Find out
what you can about the land,
especially Jericho.”
They stopped
at the house
of a prostitute
named
Rahab
and spent the night
there.
\VS{2}The king
of Jericho
received this
report: “Note
well! Israelite
men
have come
here
tonight
to spy
on the land.”
\VS{3}So the king
of Jericho
sent
this order to
Rahab: “Turn over
the men
who came
to you – the ones who came to your house – for they have come to spy on the whole land!”
\VS{4}But the woman
hid
the two
men
and replied,
“Yes,
these men
were clients of mine,
but I didn’t
know
where
they came from.
\VS{5}When
it was time to shut
the city gate
for the night,
the men
left.
I don’t
know
where
they were heading.
Chase
after
them quickly,
for
you have time to catch them!”
\VS{6}(Now she
had taken them up
to the roof
and had hidden
them in the stalks
of flax
she had spread out
on
the roof.)
\VS{7}Meanwhile the king’s men
tried to find them
on the road
to the Jordan River
near the fords.
The city gate
was shut
as soon as
they set out
in pursuit
of them.
\par }{\PP \VS{8}Now before
the spies went to sleep,
Rahab went up
to the roof.
\VS{9}She said
to
the men,
“I know
the {\ND{Lord}}
is handing
this land
over to you. We are absolutely terrified
of you, and all
who live
in the land
are cringing
before you.
\VS{10}For
we heard
how
the {\ND{Lord}}
dried
up the
water
of the Red
Sea
before
you when you left
Egypt
and how
you annihilated
the
two
Amorite
kings,
Sihon
and Og,
on the other side
of the Jordan.
\VS{11}When we heard
the news we lost our courage
and no
one
could
even breathe
for fear of you. For
the {\ND{Lord}}
your God
is God
in heaven
above
and on
earth
below!
\VS{12}So now,
promise
me this with an oath
sworn in the
{\ND{Lord}}’s
name. Because
I have shown
allegiance
to you,
show
allegiance
to
my family.
Give
me a solemn
pledge
\VS{13}that you will spare
the lives of my father,
mother,
brothers,
sisters,
and all
who belong
to them, and rescue
us from death.”
\VS{14}The men
said
to her, “If you die, may we die
too! If
you do not
report
what
we’ve
been up to, then, when
the {\ND{Lord}}
hands the land
over
to us, we
will show
unswerving
allegiance to you.”
\par }{\PP \VS{15}Then Rahab let them down
by a rope
through the window.
(Her house
was built as part of the city
wall;
she
lived
in the wall.)
\VS{16}She told
them, “Head to the hill country,
so
the ones chasing
you don’t find
you. Hide
from them there
for three
days,
long
enough for those
chasing
you to
return.
Then
you can
be on your way.”
\VS{17}The men
said
to
her,
“We
are not bound
by this
oath you made us swear unless the following conditions are met:
\VS{18}When we
invade
the land
, tie
this
red
rope
in the window
through which
you let us down,
and gather together
in your house your father,
mother,
brothers,
and all
who live in your father’s
house.
\VS{19}Anyone
who
leaves
your house
will be responsible for his own death – we are innocent in that case! But if anyone with you in the house is harmed, we will be responsible.
\VS{20}If
you should report
what
we’ve
been
up to, we are not bound
by this oath you made us swear.”
\VS{21}She said,
“I agree
to these conditions.”
She sent
them on their way
and then tied
the red
rope
in the window.
\VS{22}They went
to the hill country
and stayed
there
for three
days,
long enough
for those chasing
them to return.
Their pursuers
looked
all
along the way
but did not
find them.
\VS{23}Then the two
men
returned
– they came down
from the hills,
crossed
the river, came
to
Joshua
son
of Nun,
and reported
to him all
they had discovered.
\VS{24}They told
Joshua,
“Surely
the {\ND{Lord}}
is handing
over all
the land
to us! All
who live
in the land
are cringing
before us!”

\par }\Chap{3}{\PP \VerseOne{1}Bright and early
the next morning
Joshua
and the Israelites
left Shittim
and came
to
the Jordan.
They camped
there
before
crossing the river.
\VS{2}After
three
days
the leaders
went through
the camp
\VS{3}and commanded
the
people: “When you see
the
ark
of the covenant
of the {\ND{Lord}}
your God
being carried
by the Levitical
priests,
you
must leave
here
and walk
behind it.
\VS{4}But
stay about three thousand feet
behind
it. Keep
your distance
so
you can
see which
way
you should go,
for
you have not
traveled
this
way
before.”
\par }{\PP \VS{5}Joshua
told
the people,
“Ritually consecrate
yourselves, for
tomorrow
the {\ND{Lord}}
will perform
miraculous
deeds among you.”
\VS{6}Joshua
told
the priests,
“Pick
up the
ark
of the covenant
and pass
on ahead
of the people.”
So they picked
up the
ark
of the covenant
and went
ahead
of the people.
\par }{\PP \VS{7}The
{\ND{Lord}}
told
Joshua,
“This
very day
I will begin
to honor
you before
all
Israel
so they will know
that
I am
with you
just
as I was
with
Moses.
\VS{8}Instruct
the priests
carrying
the ark
of the covenant,
‘When you reach
the bank
of the Jordan River,
wade
into the water.’ ”
\par }{\PP \VS{9}Joshua
told
the Israelites,
“Come
here
and listen
to the words
of the {\ND{Lord}}
your God!”
\VS{10}Joshua
continued, “This
is how you will know
the living
God
is among
you and that he will truly drive
out
before
you the
Canaanites,
Hittites,
Hivites,
Perizzites,
Girgashites,
Amorites,
and Jebusites.
\VS{11}Look! The ark
of the covenant
of the Ruler
of the whole
earth
is ready to enter
the Jordan
ahead of you.
\VS{12}Now
select
for yourselves twelve
men
from the tribes
of Israel,
one per
tribe.
\VS{13}When the feet
of the priests
carrying
the ark
of the {\ND{Lord}}, the Ruler
of the whole
earth,
touch
the water
of the Jordan,
the water
coming downstream
toward
you will stop flowing
and pile up.”
\par }{\PP \VS{14}So when
the people
left
their tents
to cross
the Jordan,
the priests
carrying
the ark
of the covenant
went ahead of them.
\VS{15}When the ones carrying
the ark
reached
the Jordan
and the feet
of the priests
carrying
the ark
touched
the surface
of the water
– (the Jordan
is at flood stage
all
during harvest
time) –
\VS{16}the water
coming downstream
toward them stopped
flowing. It piled up
far
upstream
at Adam
(the city
near
Zarethan); there was no water at all flowing to the sea
of the Arabah
(the Salt
Sea). The people
crossed
the river opposite
Jericho.
\VS{17}The priests
carrying
the ark
of the covenant
of the {\ND{Lord}}
stood
firmly
on dry ground
in the middle
of the Jordan.
All
Israel
crossed
over on dry ground
until
the entire
nation
was on the other
side.

\par }\Chap{4}{\PP \VerseOne{1}When
the entire
nation
was on the other
side,
the {\ND{Lord}}
told
Joshua,
\VS{2}“Select
for yourselves twelve
men
from
the people,
one per
tribe.
\VS{3}Instruct
them, ‘Pick
up twelve
stones
from the middle
of the Jordan,
from the very place where
the priests
stand
firmly, and carry them over
with
you and put
them in the
place where you camp
tonight.’ ”
\par }{\PP \VS{4}Joshua
summoned
the twelve
men
he had
appointed
from the Israelites,
one per
tribe.
\VS{5}Joshua
told
them, “Go
in front
of the ark
of the {\ND{Lord}}
your God
to
the middle
of the Jordan.
Each
of you is to put a stone
on
his shoulder,
according to the number
of the Israelite
tribes.
\VS{6}The stones
will be
a reminder
to you. When
your children
ask
someday,
‘Why
are these
stones important to you?’
\VS{7}tell
them how
the water
of the Jordan
stopped flowing before
the ark
of the covenant
of the {\ND{Lord}}. When it crossed
the Jordan,
the water
of the Jordan
stopped flowing. These
stones
will be
a lasting
memorial
for the Israelites.”
\par }{\PP \VS{8}The Israelites
did
just
as Joshua
commanded.
They picked
up twelve
stones,
according to the number
of the Israelite
tribes,
from the middle
of the Jordan
as
the {\ND{Lord}}
had instructed
Joshua.
They carried them
over
with
them to
the camp
and put
them there.
\VS{9}Joshua
also set up
twelve
stones
in the middle
of the Jordan
in the very place where
the priests
carrying
the ark
of the covenant
stood.
They remain
there
to
this
very
day.
\par }{\PP \VS{10}Now the priests
carrying
the ark
of the covenant were standing
in the middle
of the Jordan
until
everything
the Lord
had
commanded
Joshua
to tell
the people
was accomplished, in accordance with all
that
Moses
had commanded
Joshua.
The people
went across
quickly,
\VS{11}and when
all
the people
had finished crossing,
the ark
of the {\ND{Lord}}
and the priests
crossed as the people looked on.
\VS{12}The Reubenites,
Gadites,
and the half-tribe
of Manasseh
crossed
over armed for battle
ahead
of the Israelites,
just
as Moses
had instructed them.
\VS{13}About forty
thousand
battle-ready
troops
marched past
the {\ND{Lord}}
to fight
on the plains
of Jericho.
\VS{14}That day
the {\ND{Lord}}
brought honor
to Joshua
before
all
Israel.
They respected
him all
his life,
just
as they had respected
Moses.
\par }{\PP \VS{15}The
{\ND{Lord}}
told
Joshua,
\VS{16}“Instruct
the priests
carrying
the ark
of the covenantal laws
to come up
from
the Jordan.”
\VS{17}So Joshua
instructed
the priests,
“Come up
from
the Jordan!”
\VS{18}The priests
carrying
the ark
of the covenant
of the {\ND{Lord}}
came up
from the middle
of the Jordan,
and as soon
as they
set foot
on dry land,
the water
of the Jordan
flowed again and returned
to flood stage.
\par }{\PP \VS{19}The people
went up
from
the Jordan
on the tenth
day of the first
month
and camped
in Gilgal
on the eastern
border
of Jericho.
\VS{20}Now Joshua
set
up in Gilgal
the twelve
stones
they
had
taken
from
the Jordan.
\VS{21}He told
the Israelites,
“When your children
someday
ask
their fathers,
‘What
do these
stones represent?’
\VS{22}explain
to your children,
‘Israel
crossed
the Jordan River
on dry ground.’
\VS{23}For the
{\ND{Lord}}
your God
dried
up the
water
of the Jordan
before
you while
you crossed
over. It was just
like when
the {\ND{Lord}}
your God
dried
up the Red
Sea
before
us while
we crossed it.
\VS{24}He has done
this so
all
the nations
of the earth
might
recognize the
{\ND{Lord}}’s
power
and so
you might always obey
the {\ND{Lord}}
your God.”

\par }\Chap{5}{\PP \VerseOne{1}When
all
the Amorite
kings
on the west
side
of the Jordan
and all
the Canaanite
kings
along the seacoast
heard how
the {\ND{Lord}}
had
dried
up the
water
of the Jordan
before
the Israelites
while
they crossed,
they lost their courage
and could not
even breathe
for fear of the Israelites.
\par }{\SH A New Generation is Circumcised
\par }{\PP \VS{2}At that time
the {\ND{Lord}}
told
Joshua,
“Make
flint
knives
and circumcise
the Israelites
once again.”
\VS{3}So Joshua
made
flint
knives
and circumcised
the Israelites
on the Hill
of the Foreskins.
\VS{4}This
is why Joshua
had to circumcise
them: All
the men
old enough to fight
when they left
Egypt
died
on the journey
through the desert
after they left
Egypt.
\VS{5}Now
all
the men
who left
were circumcised,
but all
the sons
born
on the journey
through the desert
after they left
Egypt
were uncircumcised.
\VS{6}Indeed,
for forty
years
the Israelites
traveled
through the desert
until
all
the men
old enough to fight
when they left
Egypt,
the ones who
had disobeyed
the {\ND{Lord}}, died off. For the
{\ND{Lord}}
had
sworn
a solemn oath to them that he would not
let them see
the land
he
had
sworn
on oath to give
them,
a land
rich
in milk
and honey.
\VS{7}He replaced
them with their sons,
whom Joshua
circumcised.
They were uncircumcised;
their fathers had
not
circumcised
them along
the way.
\VS{8}When
all
the men had been circumcised,
they stayed
there in the camp
until they had healed.
\VS{9}The
{\ND{Lord}}
said
to
Joshua,
“Today
I have taken away
the disgrace
of Egypt
from you.” So that place
is called
Gilgal
even to
this
day.
\par }{\PP \VS{10}So the Israelites
camped
in Gilgal
and celebrated
the Passover
in the evening
of the fourteenth
day
of the month
on the plains
of Jericho.
\VS{11}They ate
some of the produce
of the land
the day
after
the Passover,
including unleavened
bread and roasted grain.
\VS{12}The manna
stopped appearing
the day
they ate
some of the produce
of the land;
the Israelites
never
ate
manna
again.
\par }{\SH Israel Conquers Jericho
\par }{\PP \VS{13}When
Joshua
was near Jericho,
he looked
up and saw
a man
standing
in front
of him holding
a drawn
sword.
Joshua
approached him
and asked
him, “Are you
on our side or allied
with our enemies?”
\VS{14}He answered, “Truly
I am
the commander
of the
{\ND{Lord}}’s
army.
Now
I have arrived!” Joshua
bowed down
with his face
to the ground
and asked,
“What
does my master
want to say
to
his servant?”
\VS{15}The commander
of the
{\ND{Lord}}’s
army
answered Joshua,
“Remove your sandals
from your feet,
because
the place
where
you
stand
is holy.”
Joshua
did
so.

\par }\Chap{6}{\PP \VerseOne{1}Now Jericho
was shut
tightly
because
of the Israelites.
No
one was allowed to leave
or
enter.
\VS{2}The
{\ND{Lord}}
told
Joshua,
“See,
I
am about to defeat
Jericho
for you, along
with its king
and its warriors.
\VS{3}Have all
the warriors
march around
the city
one
time;
do
this
for six
days.
\VS{4}Have seven
priests
carry
seven
rams’
horns
in front
of the ark.
On the seventh
day
march around
the city
seven
times,
while the priests
blow
the horns.
\VS{5}When
you hear
the
signal
from the ram’s
horn,
have the whole
army
give
a loud
battle cry.
Then the city
wall
will collapse
and the warriors
should charge
straight ahead.”
\par }{\PP \VS{6}So Joshua
son
of Nun
summoned
the priests
and instructed
them,
“Pick up
the ark
of the covenant,
and seven
priests
must carry
seven
rams’
horns
in front
of the ark
of the {\ND{Lord}}.”
\VS{7}And he told
the army, “Move ahead
and march around
the city,
with armed
troops going
ahead
of the ark
of the {\ND{Lord}}.”
\par }{\PP \VS{8}When
Joshua
gave the army
its orders,
the seven
priests
carrying
the seven
rams’
horns
before
the {\ND{Lord}}
moved ahead
and blew
the horns
as the ark
of the covenant
of the {\ND{Lord}}
followed
behind.
\VS{9}Armed
troops marched
ahead
of the priests
blowing
the horns,
while the rear guard
followed along
behind
the ark
blowing
rams’ horns.
\VS{10}Now Joshua
had instructed
the army, “Do not
give a battle cry
or
raise
your voices;
say nothing
until
the day
I tell
you, ‘Give the battle cry.’
Then give the battle cry!”
\VS{11}So Joshua made sure
they marched
the ark
of the {\ND{Lord}}
around
the city
one
time.
Then they went
back to the camp
and spent
the night there.
\par }{\PP \VS{12}Bright and early
the next morning
Joshua
had the priests
pick
up the ark
of the {\ND{Lord}}.
\VS{13}The seven
priests
carrying
the seven
rams’
horns
before
the ark
of the {\ND{Lord}}
marched
along
blowing
their horns.
Armed
troops marched
ahead
of them, while the rear guard
followed
along behind
the ark
of the {\ND{Lord}}
blowing
rams’ horns.
\VS{14}They marched
around the city
one
time
on the second
day,
then returned
to the camp.
They did
this
six
days in all.
\par }{\PP \VS{15}On
the seventh
day
they were up
at the crack
of dawn
and marched
around the city
as before
– only
this
time
they marched
around it seven
times.
\VS{16}The seventh
time
around,
the priests
blew
the rams’ horns
and Joshua
told
the army, “Give the battle cry,
for
the {\ND{Lord}}
is handing
the
city over to you!
\VS{17}The city
and all
that
is in it must be set apart for the
{\ND{Lord}}, except
for Rahab
the prostitute
and all
who
are with
her in her house,
because
she hid
the
spies
we sent.
\VS{18}But
be careful
when you are
setting
apart the riches
for the
{\ND{Lord}}. If you
take
any
of it, you will make
the
Israelite
camp
subject to annihilation
and cause a disaster.
\VS{19}All
the silver
and gold,
as well as bronze
and iron
items, belong
to the
{\ND{Lord}}. They must go
into the
{\ND{Lord}}’s
treasury.”
\par }{\PP \VS{20}The rams’ horns
sounded
and when
the army
heard
the signal,
they
gave
a loud
battle cry.
The wall
collapsed
and the warriors
charged
straight ahead
into the city
and captured it.
\VS{21}They annihilated
with the sword
everything
that
breathed in the city,
including men
and women,
young
and old,
as well as
cattle,
sheep,
and donkeys.
\VS{22}Joshua
told
the two
men
who had spied
on the land,
“Enter
the prostitute’s
house
and bring out
the woman
and all
who belong
to her as
you promised her.”
\VS{23}So the young
spies
went
and brought out
Rahab,
her father,
mother,
brothers,
and all
who
belonged to her. They brought out
her whole
family
and took
them to a place outside
the Israelite
camp.
\VS{24}But they burned
the city
and all
that
was in it, except
for the silver,
gold,
and bronze
and iron
items
they put
in the treasury
of the
{\ND{Lord}}’s
house.
\VS{25}Yet Joshua
spared
Rahab
the prostitute,
her father’s
family,
and all
who belonged
to her. She lives
in
Israel
to this very
day
because
she hid
the messengers
Joshua
sent
to spy
on Jericho.
\VS{26}At that time
Joshua
made this solemn
declaration: “The man
who
attempts to rebuild
this
city
of Jericho
will stand
condemned before
the {\ND{Lord}}. He will lose his firstborn son
when he lays its foundations
and his youngest
son when he erects
its gates!”
\VS{27}The
{\ND{Lord}}
was with
Joshua
and he became
famous
throughout
the land.

\par }\Chap{7}{\PP \VerseOne{1}But the Israelites
disobeyed
the command
about the city’s riches.
Achan
son
of Carmi,
son
of Zabdi,
son
of Zerah,
from the tribe
of Judah,
stole
some
of the riches.
The
{\ND{Lord}}
was furious
with the Israelites.
\par }{\PP \VS{2}Joshua
sent
men
from Jericho
) and instructed
them, “Go up
and spy
on the land.”
So the men
went up
and spied
on Ai.
\VS{3}They returned
and reported
to
Joshua, “Don’t
send
the whole
army.
About two or
three
thousand
men
are adequate to defeat
Ai.
Don’t
tire out
the whole
army,
for
Ai
is small.”
\par }{\PP \VS{4}So
about
three
thousand
men
went up,
but they fled
from the men
of Ai.
\VS{5}The men
of Ai
killed
about thirty-six
of them
and chased
them
from in front
of the city gate
all the way
to the fissures
and defeated
them on the steep slope.
The people’s
courage
melted away
like
water.
\par }{\PP \VS{6}Joshua
tore
his clothes;
he and the leaders
of Israel
lay
face
down
on
the ground
before
the ark
of the {\ND{Lord}}
until
evening
and threw
dirt
on
their heads.
\VS{7}Joshua
prayed, “O,
Master,

{\ND{Lord}}! Why
did you bring
these
people
across
the Jordan
to hand
us over
to the Amorites
so they could destroy us?
\VS{8}If only we had been satisfied to live on the other side of the Jordan! O
Lord,
what
can I say
now
that
Israel
has retreated
before
its enemies?
\VS{9}When
the Canaanites
and all
who live
in the land
hear about this, they will turn
against
us and destroy
the
very memory
of us from
the earth.
What
will you do
to protect your great
reputation?”
\par }{\PP \VS{10}The
{\ND{Lord}}
responded
to
Joshua,
“Get
up! Why
are you
lying
there face down?
\VS{11}Israel
has sinned;
they have
violated
my covenantal
commandment! They have
taken
some
of the riches;
they have
stolen
them and deceitfully
put
them among their own
possessions.
\VS{12}The Israelites
are unable
to stand
before
their enemies;
they retreat
because
they have become
subject to annihilation.
I will no
longer
be
with
you, unless
you destroy
what has contaminated you.
\VS{13}Get
up! Ritually consecrate
the
people
and tell
them this: ‘Ritually consecrate
yourselves for tomorrow,
because
the {\ND{Lord}}
God
of Israel
says, “You are contaminated,
O Israel! You will not
be able
to stand
before
your enemies
until
you remove
what is contaminating
you.”
\VS{14}In the morning
you must approach
in tribal
order.
The tribe
the {\ND{Lord}}
selects
must approach
by clans.
The clan
the {\ND{Lord}}
selects
must approach
by families.
The family
the {\ND{Lord}}
selects
must approach
man by man.
\VS{15}The one caught
with the riches
must be burned up
along
with all
who belong
to him, because
he violated
the
{\ND{Lord}}’s
covenant
and did
such a disgraceful thing
in Israel.’ ”
\par }{\PP \VS{16}Bright and early
the next morning
Joshua
made Israel
approach
in tribal
order
and the tribe
of Judah
was selected.
\VS{17}He then
made the clans
of Judah
approach
and the clan of the Zerahites was selected.
He made the clan
of the Zerahites
approach
and Zabdi
was selected.
\VS{18}He then
made Zabdi’s family
approach
man
by man and Achan
son
of Carmi,
son
of Zabdi,
son
of Zerah,
from the tribe
of Judah, was selected.
\VS{19}So Joshua
said
to
Achan,
“My son,
honor
the {\ND{Lord}}
God
of Israel
and give
him praise! Tell
me
what
you did;
don’t
hide
anything from me!”
\VS{20}Achan
told
Joshua,
“It is true.
I
have sinned
against the
{\ND{Lord}}
God
of Israel
in this
way:
\VS{21}I saw
among the goods
we seized
a
nice
robe
from Babylon,
two hundred
silver
pieces,
and a
bar
of gold
weighing
fifty
shekels.
I wanted
them, so
I took
them. They are hidden
in the ground
right in the middle
of my tent
with the silver
underneath.”
\par }{\PP \VS{22}Joshua
sent
messengers
who ran
to the tent.
The things were hidden
right in his tent,
with the silver
underneath.
\VS{23}They took
it all from the middle
of the tent,
brought
it to
Joshua
and all
the Israelites,
and placed
it before
the {\ND{Lord}}.
\VS{24}Then Joshua
and all
Israel
took
Achan,
son
of Zerah,
along with the
silver,
the
robe,
the bar
of gold,
his sons,
daughters,
ox,
donkey,
sheep,
tent,
and all
that belonged
to
him and brought
them up
to the
Valley
of Disaster.
\VS{25}Joshua
said,
“Why
have you brought disaster
on us? The
{\ND{Lord}}
will bring disaster
on you today!” All
Israel
stoned
him to death. (They also stoned
and burned the others.)
\VS{26}Then they erected
over
him a large
pile
of stones
(it remains to
this
very day
) and the
{\ND{Lord}}’s
anger
subsided.
So
that place
is
called
the Valley
of Disaster
to
this
very day.

\par }\Chap{8}{\PP \VerseOne{1}The
{\ND{Lord}}
told
Joshua,
“Don’t
be afraid
and don’t
panic! Take
the
whole
army
with you and march
against Ai! See,
I
am handing over
to you the
king
of Ai,
along
with his people,
city,
and land.
\VS{2}Do
to Ai
and its king
what
you did
to Jericho
and its king,
except
you may plunder
its goods
and cattle.
Set
an ambush
behind
the city!”
\par }{\PP \VS{3}Joshua
and the whole
army
marched
against Ai.
Joshua
selected
thirty
thousand
brave
warriors
and sent
them out at night.
\VS{4}He told
them, “Look,
set an ambush
behind
the city.
Don’t
go very
far
from
the city;
all
of you be ready!
\VS{5}I
and all
the troops
who
are with
me will approach
the city.
When
they come out
to fight
us like
before,
we will retreat from them.
\VS{6}They will attack us until
we have lured
them from
the city,
for
they will say,
‘They are retreating from us
like
before.’
We will retreat from them.
\VS{7}Then you
rise up
from your hiding place
and seize
the city.
The
{\ND{Lord}}
your God
will hand
it over to you.
\VS{8}When
you capture
the city,
set
it on fire.
Do
as the
{\ND{Lord}}
says! See,
I have given you orders.”
\VS{9}Joshua
sent
them away and they went
to
their hiding
place west
of Ai,
between
Bethel
and Ai.
Joshua
spent
that
night
with the army.
\par }{\PP \VS{10}Bright and early
the next morning
Joshua
gathered
the
army,
and he
and the leaders
of Israel
marched
at the head of it to Ai.
\VS{11}All
the troops
that
were with him marched up
and drew
near
the city.
They camped
north
of Ai
on the other side
of the valley.
\VS{12}He took
five
thousand
men
and set
an ambush
west
of the city
between
Bethel
and Ai.
\VS{13}The army
was
in position
– the
main army
north
of the city
and the
rear
guard west
of the city.
That night
Joshua
went
into the middle
of the valley.
\par }{\PP \VS{14}When
the king
of Ai
saw
Israel,
he
and his whole
army
quickly
got up
the next day and went out
to fight
Israel
at the meeting
place near
the Arabah.
But he
did not
realize
men were hiding
behind
the city.
\VS{15}Joshua
and all
Israel
pretended
to be defeated
by them and they retreated
along the way
to the desert.
\VS{16}All
the reinforcements
in Ai were ordered to chase
them;
they chased
Joshua
and were lured
away from
the city.
\VS{17}No
men
were left
in Ai
or
Bethel;
they all went out
after
Israel.
They left
the city
wide open
and chased
Israel.
\par }{\PP \VS{18}The
{\ND{Lord}}
told
Joshua,
“Hold out
toward
Ai
the curved
sword in your hand,
for
I am handing
the city
over
to you.” So
Joshua
held
out toward Ai the curved
sword in his hand.
\VS{19}When he held
out his hand,
the men waiting
in ambush
rose
up quickly
from their place
and attacked. They entered
the city,
captured
it, and immediately
set
it on fire.
\VS{20}When the men
of Ai
turned
around,
they saw
the smoke
from the city
ascending
into
the sky
and were
so shocked
they were unable
to flee
in any direction.
In the meantime
the men
who were retreating
to the desert
turned
against their pursuers.
\VS{21}When Joshua
and all
Israel
saw
that
the men in ambush
had captured
the
city
and that
the city
was going up
in smoke,
they turned
around and struck
down the
men
of Ai.
\VS{22}At the same time the men
who had taken the city
came out
to fight,
and the men
of Ai were
trapped in the middle.
The Israelites
struck
them down,
leaving
no
survivors
or refugees.
\VS{23}But they captured
the king
of Ai
alive
and brought
him to
Joshua.
\par }{\PP \VS{24}When
Israel
had finished
killing
all
the men of Ai
who had
chased
them toward the desert
(they all
fell
by the sword), all
Israel
returned
to Ai
and put
the sword to it.
\VS{25}Twelve
thousand
men
and women
died
that day,
including
all
the men
of Ai.
\VS{26}Joshua
kept holding
out
his curved
sword until
Israel had
annihilated
all
who lived
in Ai.
\VS{27}But
Israel
did
plunder the cattle
and the goods
of the city,
in accordance
with the
{\ND{Lord}}’s
orders
to Joshua.
\VS{28}Joshua
burned
Ai
and made
it a permanently
uninhabited
mound
(it remains that way to
this
very day).
\VS{29}He hung
the king
of Ai
on
a tree,
leaving him exposed until
evening.
At sunset
Joshua
ordered
that his corpse
be taken down
from
the tree.
They threw
it down
at
the entrance
of the city
gate
and erected
over
it a large
pile
of stones
(it remains to this
very day).
\par }{\SH Covenant Renewal
\par }{\PP \VS{30}Then
Joshua
built
an altar
for the
{\ND{Lord}}
God
of Israel
on Mount
Ebal,
\VS{31}just
as Moses
the
{\ND{Lord}}’s
servant
had commanded
the Israelites.
As described in
the law
scroll
of Moses,
it was made with uncut
stones
untouched by an iron
tool.
They offered
burnt sacrifices
on
it
and sacrificed
tokens of peace.
\VS{32}There,
in the presence
of the Israelites,
Joshua inscribed
on
the stones
a duplicate
of the law
written
by Moses.
\VS{33}All
the people, rulers, leaders,
and judges
were
standing
on either
side of the ark,
in front
of the Levitical
priests
who carried
the ark
of the covenant
of the {\ND{Lord}}. Both resident foreigners
and native
Israelites
were
there. Half
the people stood in front
of Mount
Gerizim
and the other half
in front
of Mount
Ebal,
as
Moses
the
{\ND{Lord}}’s
servant
had
previously instructed
to them
to do
for the formal
blessing ceremony.
\VS{34}Then Joshua read
aloud
all
the words
of the law,
including the blessings
and the curses,
just
as they are written
in the law
scroll.
\VS{35}Joshua
read
aloud every
commandment
Moses
had given
before
the whole
assembly
of Israel,
including the women,
children,
and resident foreigners
who lived
among them.

\par }\Chap{9}{\PP \VerseOne{1}When
the news
reached all
the kings
on the west side
of the Jordan –
in the hill country,
the lowlands,
and all
along the Mediterranean
coast
as far as Lebanon
(including the Hittites,
Amorites,
Canaanites,
Perizzites,
Hivites,
and Jebusites) –
\VS{2}they formed
an alliance
to fight
against
Joshua
and Israel.
\par }{\PP \VS{3}When the residents
of Gibeon
heard
what
Joshua
did
to Jericho
and Ai,
\VS{4}they
did
something clever.
They collected
some provisions
and put
worn-out
sacks
on their donkeys,
along with worn-out
wineskins
that were ripped
and patched.
\VS{5}They had worn-out,
patched
sandals
on their feet
and dressed in
worn-out
clothes.
All
their bread
was
dry
and hard.
\VS{6}They came
to
Joshua
at
the camp
in Gilgal
and said
to
him and the men
of Israel,
“We have come
from a distant
land.
Make
a treaty with us.”
\VS{7}The men
of Israel
said
to
the Hivites,
“Perhaps
you
live
near
us. So how
can we make
a treaty with you?”
\VS{8}But they said
to
Joshua,
“We
are willing to be your subjects.”
So Joshua
said
to
them, “Who
are you
and where
do you come from?”
\VS{9}They told
him, “Your subjects
have come
from a very
distant
land
because of the reputation
of the {\ND{Lord}}
your God,
for
we have heard
the news
about all
he did
in Egypt
\VS{10}and all
he did
to the two
Amorite
kings
on the other side
of the Jordan
– King
Sihon
of Heshbon
and King
Og
of Bashan
in Ashtaroth.
\VS{11}Our leaders
and all
who live
in our land
told
us, ‘Take
provisions
for your journey
and go
meet
them. Tell
them, “We
are willing to
be your subjects.
Make
a treaty with us.” ’
\VS{12}This
bread
of ours was warm
when we packed
it in
our homes
the day
we started out
to
meet
you, but
now
it is
dry
and hard.
\VS{13}These
wineskins
we filled
were brand new,
but look
how they have ripped.
Our clothes
and sandals
have worn out
because it has been a very
long
journey.”
\VS{14}The men
examined
some of their provisions,
but they failed
to ask
the
{\ND{Lord}}’s
advice.
\VS{15}Joshua
made
a peace
treaty with them and agreed
to let them live.
The leaders
of the community
sealed it with an oath.
\par }{\PP \VS{16}Three
days
after
they made
the treaty
with them,
the Israelites found out
they
were from the local area
and lived
nearby.
\VS{17}So the Israelites
set out
and on the third
day
arrived
at
their cities
– Gibeon,
Kephirah,
Beeroth,
and Kiriath Jearim.
\VS{18}The Israelites
did not
attack
them because
the leaders
of the community
had sworn an oath
to them in the name of the
{\ND{Lord}}
God
of Israel.
The whole
community
criticized
the leaders,
\VS{19}but all
the leaders
told
the whole
community,
“We
swore an oath
to them in the name of the
{\ND{Lord}}
God
of Israel.
So now
we can’t
hurt them!
\VS{20}We must
let them live
so we can escape
the curse
attached to
the oath
we swore to them.”
\VS{21}The leaders
then added, “Let them live.”
So they became
woodcutters
and water
carriers
for the whole
community,
as
the leaders
had decided.
\VS{22}\par }{\PP Joshua
summoned
the Gibeonites and said
to
them, “Why
did you trick
us by saying,
‘We
live far away
from
you,’ when you really
live
nearby?
\VS{23}Now
you
are condemned
to perpetual
servitude
as woodcutters
and water
carriers
for the house
of my God.”
\VS{24}They said
to Joshua,
“It was carefully reported
to your subjects
how
the {\ND{Lord}}
your God
commanded
Moses
his servant
to assign
you the
whole
land
and to destroy
all
who live
in the land
from before
you. Because of you we were terrified
we would lose our lives,
so we did
this
thing.
\VS{25}So now
we are in your power.
Do
to us what you think is good
and appropriate.
\VS{26}Joshua did
as
they said; he kept
the Israelites
from killing them
\VS{27}and that day
made
them
woodcutters
and water
carriers
for the community
and for the altar
of the {\ND{Lord}}
at the divinely chosen
site.
(They continue in that capacity to this
very
day.)

\par }\Chap{10}{\PP \VerseOne{1}Adoni-Zedek,
king
of Jerusalem,
and its king.
He also heard how the people
of Gibeon
made peace
with
Israel
and lived
among them.
\VS{2}All Jerusalem was terrified
because
Gibeon
was a large
city,
like one
of the royal
cities.
It
was larger
than
Ai
and all
its men
were warriors.
\VS{3}So King
Adoni-Zedek
of Jerusalem
sent
this message to
King
Hoham
of Hebron,
King
Piram
of Jarmuth,
King
Japhia
of Lachish,
and King
Debir
of Eglon:
\VS{4}“Come
to
my aid
so we can attack
Gibeon,
for
it has made peace
with
Joshua
and the Israelites.”
\VS{5}So the five
Amorite
kings
(the kings
of Jerusalem,
Hebron,
Jarmuth,
Lachish,
and Eglon) and all
their
troops
gathered together
and advanced.
They deployed
their troops and fought
against
Gibeon.
\par }{\PP \VS{6}The men
of Gibeon
sent
this message to
Joshua
at the camp
in Gilgal,
“Do not
abandon
your subjects! Rescue
us! Help
us! For
all
the Amorite
kings
living
in the hill country
are attacking us.”
\VS{7}So Joshua
and his whole
army, including the bravest warriors,
marched up
from
Gilgal.
\VS{8}The
{\ND{Lord}}
told
Joshua,
“Don’t
be afraid
of them,
for
I am
handing
them over to you.
Not
one
of them
can resist you.”
\VS{9}Joshua
attacked
them by surprise
after marching
all
night
from
Gilgal.
\VS{10}The
{\ND{Lord}}
routed
them before
Israel.
Israel thoroughly defeated them at Gibeon. They chased them up the road to the pass of Beth Horon and struck them down all the way to Azekah and Makkedah.
\VS{11}As
they fled
from
Israel
on the slope leading down
from Beth Horon,
the {\ND{Lord}}
threw down
on
them large
hailstones
from
the sky,
all the way
to Azekah.
They died
– in fact, more
died
from the hailstones
than the Israelites
killed
with the sword.
\par }{\PP \VS{12}The day
the {\ND{Lord}}
delivered
the Amorites
over
to the Israelites,
Joshua
prayed to the
{\ND{Lord}}
before
Israel:

\par }{\Q “O sun,
stand still
over Gibeon!
\par }{\Q O moon,
over the Valley
of Aijalon!”
\par }{\PP \VS{13}The sun
stood still
and the moon
stood
motionless while
the nation
took vengeance
on its enemies.
The event is recorded
in
the Scroll
of the Upright
One. The sun
stood
motionless in the middle
of the sky
and did not
set
for about a full
day.
\VS{14}There has not
been
a day
like it
before
or since.
The
{\ND{Lord}}
obeyed
a man,
for
the {\ND{Lord}}
fought
for Israel!
\VS{15}Then Joshua
and all
Israel
returned
to
the camp
at Gilgal.
\par }{\PP \VS{16}The five
Amorite
kings
ran
away and hid
in the cave
at Makkedah.
\VS{17}Joshua
was told,
“The five
kings
have been found
hiding
in the cave
at Makkedah.”
\VS{18}Joshua
said,
“Roll
large
stones
over the mouth
of the cave
and post
guards in front of it.
\VS{19}But don’t
you
delay! Chase
your enemies
and catch
them! Don’t
allow
them to retreat
to
their cities,
for
the {\ND{Lord}}
your God
is handing
them over
to you.”
\VS{20}Joshua
and the Israelites
almost totally wiped
them out,
but some survivors
did escape
to
the fortified
cities.
\VS{21}Then
the whole
army
safely
returned
to
Joshua
at the camp
in Makkedah.
No
one dared
threaten the Israelites.
\VS{22}Joshua
said,
“Open
the cave’s
mouth
and bring
the five
kings
out
of the cave to me.”
\VS{23}They did
as ordered;
they brought
the five
kings
out
of the cave to him – the kings of Jerusalem, Hebron, Jarmuth, Lachish, and Eglon.
\VS{24}When
they brought
the
kings
out to
Joshua,
he
summoned
all
the men
of Israel
and said
to
the commanders
of the troops
who
accompanied
him,
“Come
here and put
your feet
on
the necks
of these
kings.”
So they came
up and put
their feet
on
their necks.
\VS{25}Then Joshua
said
to
them, “Don’t
be afraid
and don’t
panic! Be strong
and brave,
for
the {\ND{Lord}}
will do
the same thing
to all
your enemies
you
fight.
\VS{26}Then
Joshua
executed
them and hung
them on
five
trees.
They were left hanging
on
the trees
until
evening.
\VS{27}At sunset
Joshua
ordered
his men to take
them down
from
the trees.
They threw
them into
the cave
where
they had hidden
and piled
large
stones
over
the mouth
of the cave.
(They remain
to this
very day.)
\par }{\SH Joshua Launches a Southern Campaign
\par }{\PP \VS{28}That day
Joshua
captured
Makkedah
and put
the sword
to it and its king.
He annihilated
everyone
who lived
in it; he left
no
survivors.
He did
to its king
what
he had done
to the king
of Jericho.
\par }{\PP \VS{29}Joshua
and all
Israel
marched from Makkedah
to Libnah
and fought
against it.
\VS{30}The
{\ND{Lord}}
handed
it and its
king
over
to Israel,
and Israel put
the sword
to all
who
lived
there; they left no
survivors.
They did
to its king
what they had
done
to the king
of Jericho.
\par }{\PP \VS{31}Joshua
and all
Israel
marched
from Libnah
to Lachish.
He deployed
his troops and fought against it.
\VS{32}The
{\ND{Lord}}
handed
Lachish
over to Israel
and they captured
it on the second
day.
They put
the sword
to all
who
lived
there, just
as they had
done
to Libnah.
\VS{33}Then
King
Horam
of Gezer
came up to help
Lachish,
but Joshua
struck
down him and his army
until
no
survivors
remained.
\par }{\PP \VS{34}Joshua
and all
Israel
marched
from Lachish
to Eglon.
They deployed
troops and fought
against it.
\VS{35}That day
they captured
it and put
the sword
to all
who
lived
there. That
day
they annihilated
it just
as they had
done
to Lachish.
\par }{\PP \VS{36}Joshua
and all
Israel
marched
up from
Eglon
to Hebron
and fought
against it.
\VS{37}They captured
it and put
the sword
to its king,
all
its surrounding cities,
and all
who
lived
in it; they left no
survivors.
As they had
done
at Eglon,
they annihilated
it and all
who
lived there.
\par }{\PP \VS{38}Joshua
and all
Israel
turned
to Debir
and fought
against it.
\VS{39}They captured
it, its
king,
and all
its surrounding cities
and put
the sword
to them. They annihilated
everyone
who
lived
there; they left no
survivors.
They did
to Debir
and its king
what they had
done
to Libnah
and its king
and to Hebron.
\par }{\PP \VS{40}Joshua
defeated
the
whole
land,
including the hill country,
the Negev,
the lowlands,
the slopes,
and all
their kings.
He left
no
survivors.
He annihilated
everything
that breathed,
just
as the
{\ND{Lord}}
God
of Israel
had commanded.
\VS{41}Joshua
conquered
the area between
Kadesh Barnea
and Gaza
and the whole
region
of Goshen,
all the way
to Gibeon.
\VS{42}Joshua
captured
in one
campaign
all
these
kings
and their lands,
for
the {\ND{Lord}}
God
of Israel
fought
for Israel.
\VS{43}Then Joshua
and all
Israel
returned
to
the camp
at Gilgal.

\par }\Chap{11}{\PP \VerseOne{1}When
King
Jabin
of Hazor
heard
the news, he organized a coalition,
including King
Jobab
of Madon,
the king
of Shimron,
the king
of Acshaph,
\VS{2}and the northern
kings
who
ruled in the hill country,
the Arabah
south
of Kinnereth,
the lowlands,
and the heights
of Dor
to the west.
\VS{3}Canaanites
came from the east
and west;
Amorites,
Hittites,
Perizzites,
and Jebusites
from the hill country;
and Hivites
from below
Hermon
in the area
of Mizpah.
\VS{4}These kings came out
with their armies;
they were as numerous
as the sand
on
the seashore
and had a large number
of horses
and chariots.
\VS{5}All
these
kings
gathered
and joined
forces
at
the Waters
of Merom
to
fight
Israel.
\par }{\PP \VS{6}The
{\ND{Lord}}
told
Joshua,
“Don’t
be afraid
of them, for
about this
time
tomorrow
I
will cause
all
of them to lie dead
before
Israel.
You must
hamstring
their horses
and burn
their chariots.”
\VS{7}Joshua
and his whole
army
caught
them by surprise
at the Waters
of Merom
and attacked them.
\VS{8}The
{\ND{Lord}}
handed
them over to Israel
and they struck
them down and chased
them all
the way
to Greater
Sidon,
Misrephoth
Maim, and the Mizpah
Valley
to the east.
They struck
them down until
no
survivors
remained.
\VS{9}Joshua
did
to them as
the {\ND{Lord}}
had
commanded
him; he hamstrung
their horses
and burned
their chariots.
\par }{\PP \VS{10}At that time
Joshua
turned,
captured
Hazor,
and struck
down its king
with the sword,
for
Hazor
was at that
time the leader
of all
these
kingdoms.
\VS{11}They annihilated
everyone
who
lived
there with the sword –
no
one who breathed
remained
– and burned
Hazor.
\par }{\PP \VS{12}Joshua
captured
all
these
royal
cities
and all
their kings
and annihilated
them with the sword,
as Moses
the
{\ND{Lord}}’s
servant
had commanded.
\VS{13}But
Israel
did not
burn
any
of the cities
located
on
mounds,
except
for Hazor;
it was the only
one Joshua
burned.
\VS{14}The Israelites
plundered all
the goods
of these
cities
and the cattle,
but they
totally destroyed
all
the people
and allowed no
one who breathed
to live.
\VS{15}Moses
the
{\ND{Lord}}’s
servant
passed on
the
{\ND{Lord}}’s commands
to Joshua,
and Joshua
did
as
he was told.
He did
not
ignore
any
of the commands
the {\ND{Lord}}
had given
Moses.
\par }{\SH A Summary of Israel’s Victories
\par }{\PP \VS{16}Joshua
conquered
the whole
land,
including
the hill country,
all
the Negev,
all
the land
of Goshen,
the lowlands,
the Arabah,
the
hill country
of Israel
and its lowlands,
\VS{17}from
Mount
Halak
on up
to Seir,
as far
as Baal Gad
in the Lebanon
Valley
below
Mount
Hermon.
He captured
all
their kings
and executed them.
\VS{18}Joshua
campaigned
against
these
kings
for quite
some time.
\VS{19}No
city
made peace
with the Israelites
(except
the Hivites
living
in Gibeon); they had to conquer
all of them,
\VS{20}for
the {\ND{Lord}}
determined
to make them obstinate
so
they would
attack
Israel.
He wanted
Israel to annihilate
them
without
mercy,
as
he had
instructed
Moses.
\par }{\PP \VS{21}At that time
Joshua
attacked
and eliminated
the Anakites
from
the hill country –
from
Hebron,
Debir,
Anab,
and all
the hill country
of Judah
and Israel.
Joshua
annihilated
them and their cities.
\VS{22}No
Anakites
were left
in Israelite
territory, though some remained
in Gaza,
Gath,
and Ashdod.
\VS{23}Joshua
conquered
the whole
land,
just
as the
{\ND{Lord}}
had promised
Moses,
and he
assigned
Israel
their tribal
portions.
Then the land
was free
of war.

\par }\Chap{12}{\PP \VerseOne{1}Now these
are the kings
of the land
whom
the Israelites
defeated
and drove
from their land
on the east
side
of the Jordan,
from the Arnon
Valley
to
Mount
Hermon,
including all
the eastern
Arabah:
\par }{\PP \VS{2}King
Sihon
of the Amorites
who lived
in Heshbon
and ruled
from Aroer
(on
the edge
of the Arnon
Valley) – including the city in the middle of the valley and half of Gilead – all the way to the Jabbok Valley bordering Ammonite territory.
\VS{3}His kingdom included the eastern
Arabah
from
the Sea
of Kinnereth
to
the Sea
of the Arabah
(the Salt
Sea), including the route
to Beth Jeshimoth
and the area southward
below
the slopes
of Pisgah.
\par }{\PP \VS{4}The territory
of King
Og
of Bashan,
one of
the few remaining
Rephaites,
who lived
in Ashtaroth
and Edrei
\VS{5}and ruled
over Mount
Hermon,
Salecah,
all
of Bashan
to
the border
of the Geshurites
and Maacathites,
and half
of Gilead
as far as the border
of King
Sihon
of Heshbon.
\par }{\PP \VS{6}Moses
the
{\ND{Lord}}’s
servant
and the Israelites
defeated
them and Moses
the
{\ND{Lord}}’s
servant
assigned
their land
to Reuben,
Gad,
and the half
tribe
of Manasseh.
\par }{\PP \VS{7}These
are the kings
of the land
whom
Joshua
and the Israelites
defeated
on the west
side
of the Jordan,
from Baal Gad
in the Lebanon
Valley
to
Mount
Halak
on up to
Seir.
Joshua
assigned
this territory
to the Israelite
tribes,
\VS{8}including the hill country,
the lowlands,
the Arabah,
the slopes,
the wilderness,
and the Negev
– the land of the Hittites,
Amorites,
Canaanites,
Perizzites,
Hivites,
and Jebusites:
\par }{\Q \VS{9}the king
of Jericho
(one),
\par }{\Q the king
of Ai
– located
near Bethel
– (one),
\par }{\Q \VS{10}the king
of Jerusalem
(one),
\par }{\Q the king
of Hebron
(one),
\par }{\Q \VS{11}the king
of Jarmuth
(one),
\par }{\Q the king
of Lachish
(one),
\par }{\Q \VS{12}the king
of Eglon
(one),
\par }{\Q the king
of Gezer
(one),
\par }{\Q \VS{13}the king
of Debir
(one),
\par }{\Q the king
of Geder
(one),
\par }{\Q \VS{14}the king
of Hormah
(one),
\par }{\Q the king
of Arad
(one),
\par }{\Q \VS{15}the king
of Libnah
(one),
\par }{\Q the king
of Adullam
(one),
\par }{\Q \VS{16}the king
of Makkedah
(one),
\par }{\Q the king
of Bethel
(one),
\par }{\Q \VS{17}the king
of Tappuah
(one),
\par }{\Q the king
of Hepher
(one),
\par }{\Q \VS{18}the king
of Aphek
(one),
\par }{\Q the king
of Lasharon
(one),
\par }{\Q \VS{19}the king
of Madon
(one),
\par }{\Q the king
of Hazor
(one),
\par }{\Q \VS{20}the king
of Shimron Meron
(one),
\par }{\Q the king
of Acshaph
(one),
\par }{\Q \VS{21}the king
of Taanach
(one),
\par }{\Q the king
of Megiddo
(one),
\par }{\Q \VS{22}the king
of Kedesh
(one),
\par }{\Q the king
of Jokneam
near Carmel
(one),
\par }{\Q \VS{23}the king
of Dor
– near Naphath Dor
– (one),
\par }{\Q the king
of Goyim
– near Gilgal
– (one),
\par }{\Q \VS{24}the king
of Tirzah
(one),
\par }{\Q a total
of thirty-one
kings.

\par }\Chap{13}{\PP \VerseOne{1}When Joshua
was very old,
the {\ND{Lord}}
told
him, “You
are very old,
and a great
deal
of land
remains
to be conquered.
\VS{2}This
is the land
that remains: all
the territory
of the Philistines
and all
the Geshurites,
\VS{3}from
the Shihor River
east
of Egypt
northward
to the territory
of Ekron
(it is regarded as
Canaanite
territory), including the area belonging to the five
Philistine
lords
who ruled in Gaza,
Ashdod,
Ashkelon,
Gath,
and Ekron,
as well as Avvite land
\VS{4}to the south;
all
the Canaanite
territory,
from Arah
in the region of Sidon
to
Aphek,
as far
as Amorite
territory;
\VS{5}the territory
of Byblos
and all
Lebanon
to the east,
from Baal Gad
below
Mount
Hermon
to
Lebo Hamath.
\VS{6}I
will drive
out before
the Israelites
all
who live
in the hill country
from
Lebanon
to
Misrephoth
Maim, all
the Sidonians;
you be sure
to parcel
it out to Israel
as I instructed you.”
\VS{7}Now,
divide
up this
land
among
the nine
tribes
and the half-tribe
of Manasseh.”
\par }{\SH Tribal Lands East of the Jordan
\par }{\PP \VS{8}The other half of Manasseh, Reuben,
and Gad
received
their allotted tribal lands
beyond
the Jordan,
just
as Moses,
the
{\ND{Lord}}’s
servant,
had assigned them.
\VS{9}Their territory started from Aroer
(on
the edge
of the Arnon
Valley), included the city
in the middle
of the valley,
the whole
plain
of Medeba
as far
as Dibon,
\VS{10}and all
the cities
of King
Sihon
of the Amorites
who
ruled
in Heshbon,
and ended
at the Ammonite
border.
\VS{11}Their territory
also included Gilead,
Geshurite
and Maacathite
territory, all
Mount
Hermon,
and all
Bashan
to Salecah –
\VS{12}the whole
kingdom
of Og
in Bashan,
who
ruled
in Ashtaroth
and Edrei.
(He was
one of
the few remaining
Rephaites.) Moses
defeated
them and took their lands.
\VS{13}But the Israelites
did not
conquer
the Geshurites
and Maacathites;
Geshur
and Maacah
live
among
Israel
to
this
very
day.
\VS{14}However,
Moses did not
assign
land as an inheritance
to the Levites;
their inheritance
is the sacrificial offerings
made to the
{\ND{Lord}}
God
of Israel,
as he instructed them.
\par }{\PP \VS{15}Moses
assigned
land to the tribe
of Reuben
by its clans.
\VS{16}Their
territory
started at Aroer
(on
the edge
of the Arnon
Valley) and included the city
in the middle
of the valley,
the whole
plain
of Medeba,
\VS{17}Heshbon
and all
its surrounding cities
on the plain,
including Dibon,
Bamoth Baal,
Beth Baal Meon,
\VS{18}Jahaz,
Kedemoth,
Mephaath,
\VS{19}Kiriathaim,
Sibmah,
Zereth Shahar
on the hill
in the valley,
\VS{20}Beth Peor,
the slopes
of Pisgah,
and Beth Jeshimoth.
\VS{21}It encompassed all
the cities
of the plain
and the whole
realm
of King
Sihon
of the Amorites
who
ruled
in Heshbon.
Moses
defeated
him and the
Midianite
leaders
Evi,
Rekem,
Zur,
Hur,
and Reba
(they were subjects
of Sihon
and lived
in his territory).
\VS{22}The Israelites
killed
Balaam
son
of Beor,
the omen
reader, along with the others.
\VS{23}The border
of the tribe of
Reuben
was the Jordan.
The land
allotted
to the tribe of
Reuben
by its clans
included these cities
and their towns.
\par }{\PP \VS{24}Moses
assigned
land to the tribe
of Gad
by its clans.
\VS{25}Their territory
included Jazer,
all
the cities
of Gilead,
and half
of Ammonite
territory
as far
as Aroer
near
Rabbah.
\VS{26}Their territory ran from
Heshbon
to Ramath Mizpah
and Betonim,
and from Mahanaim
to
the territory
of Debir.
\VS{27}It included the valley
of Beth Haram,
Beth Nimrah,
Succoth,
and Zaphon,
and the rest
of the realm
of King
Sihon
of Heshbon,
the area
east
of the Jordan
to
the end
of the Sea
of Kinnereth.
\VS{28}The land allotted
to the tribe of
Gad
by its clans
included these cities
and their towns.
\par }{\PP \VS{29}Moses
assigned
land to the half-tribe
of Manasseh
by
its clans.
\VS{30}Their territory
started
at Mahanaim
and encompassed all
Bashan,
the whole
realm
of King
Og
of Bashan,
including all
sixty
cities
in Havvoth
Jair
in Bashan.
\VS{31}Half
of Gilead,
Ashtaroth,
and Edrei,
cities
in the kingdom
of Og
in Bashan,
were assigned to the descendants
of Makir
son
of Manasseh,
to half
the descendants
of Makir
by their clans.
\par }{\PP \VS{32}These
are the land assignments
made by Moses
on the plains
of Moab
east
of the Jordan River
opposite Jericho.
\VS{33}However, Moses
did not
assign land as an inheritance
to the Levites;
their inheritance
is the
{\ND{Lord}}
God
of Israel,
as he instructed them.

\par }\Chap{14}{\PP \VerseOne{1}The following
is a record of the territory assigned
to the Israelites
in the land
of Canaan
by
Eleazar
the priest,
Joshua
son
of Nun,
and the Israelite
tribal
leaders.
\VS{2}The land assignments
to the nine-and-a-half
tribes
were made by drawing lots,
as the
{\ND{Lord}}
had instructed
Moses.
\VS{3}Now
Moses
had assigned
land
to the two-and-a-half
tribes
east
of the Jordan,
but he assigned
no
land
to the Levites.
\VS{4}The descendants
of Joseph
were considered as two
tribes,
Manasseh
and Ephraim.
The Levites
were allotted
no territory,
though
they were assigned cities
in which to live,
along with the grazing areas
for their cattle
and possessions.
\VS{5}The Israelites
followed the
{\ND{Lord}}’s
instructions
to Moses
and divided
up the land.
\par }{\PP \VS{6}The men
of Judah
approached
Joshua
in Gilgal,
and Caleb
son
of Jephunneh
the Kenizzite
said
to
him, “You
know
what
the
{\ND{Lord}}
said
about you and me to
Moses,
the man
of God,
at Kadesh Barnea.
\VS{7}I
was forty
years
old
when Moses,
the
{\ND{Lord}}’s
servant,
sent
me from Kadesh Barnea
to spy
on the land
and I brought back
to him
an honest
report.
\VS{8}My countrymen
who
accompanied
me frightened
the people,
but I
remained loyal
to the
{\ND{Lord}}
my God.
\VS{9}That day
Moses
made this solemn
promise: ‘Surely
the land
on which
you walked
will belong
to you and your
descendants
permanently,
for
you remained
loyal
to the
{\ND{Lord}}
your God.’
\VS{10}So now,
look,
the {\ND{Lord}}
has preserved
my life, just
as he promised,
these
past forty-five
years
since
the {\ND{Lord}}
spoke
these
words
to
Moses,
during which
Israel
traveled
through the wilderness.
Now
look,
I am
today
eighty-five
years old.
\VS{11}Today
I am still
as strong
as
when
Moses
sent
me out.
I can
fight
and go
about
my daily activities with the same energy
I had then.
\VS{12}Now,
assign
me this
hill country
which
the {\ND{Lord}}
promised
me at that time! No doubt you
heard
at that time
that
the Anakites
live there
in large,
fortified
cities.
But, assuming
the {\ND{Lord}}
is with
me, I will conquer
them, as
the {\ND{Lord}}
promised.”
\VS{13}Joshua
asked God to empower
Caleb
son
of Jephunneh
and assigned
him Hebron.
\VS{14}So
Hebron
remains the assigned land
of Caleb
son
of Jephunneh
the Kenizzite
to this very
day
because
he remained
loyal
to the
{\ND{Lord}}
God
of Israel.
\VS{15}(Hebron
used to be called
Kiriath Arba.
Arba was a famous
Anakite.
) Then the land
was free
of war.

\par }\Chap{15}{\PP \VerseOne{1}The land allotted
to the tribe
of Judah
by its clans
reached to
the border
of Edom,
to the Wilderness
of Zin
in the Negev
far to the south.
\VS{2}Their
southern
border
started at the southern
tip
of the Salt
Sea,
\VS{3}extended
south
of the Scorpion Ascent,
crossed
to Zin,
went up
from the south
to Kadesh Barnea,
crossed
to Hezron,
went up
to Addar,
and turned toward
Karka.
\VS{4}It then crossed
to Azmon,
extended
to the Stream
of Egypt,
and ended
at the sea.
This
was
their southern
border.
\par }{\PP \VS{5}The eastern
border
was the Salt
Sea
to
the mouth
of the Jordan River.
\par }{\PP The northern border
started
north
of the Salt Sea
at the mouth
of the Jordan,
\VS{6}went up
to Beth Hoglah,
crossed
north
of Beth Arabah,
and went up
to the Stone
of Bohan
son
of Reuben.
\VS{7}It then went up
to Debir
from the Valley
of Achor,
turning
northward
to
Gilgal
(which
is opposite
the Pass
of Adummim
south
of the valley), crossed
to
the waters
of En Shemesh
and extended
to
En Rogel.
\VS{8}It then went up
the Valley
of Ben
Hinnom
to
the slope
of the Jebusites
on the south
(that is,
Jerusalem), going up
to
the top
of the hill
opposite
the Valley
of Ben Hinnom
to the west,
which
is at the end
of the Valley
of the Rephaites
to the north.
\VS{9}It then went
from the top
of the hill
to
the spring
of the waters
of Nephtoah,
extended
to
the cities
of Mount
Ephron,
and went
to Baalah
(that
is, Kiriath Jearim).
\VS{10}It then turned
from Baalah
westward
to
Mount
Seir,
crossed
to
the slope
of Mount
Jearim
on the north
(that is
Kesalon), descended
to Beth Shemesh,
and crossed
to Timnah.
\VS{11}It then extended
to
the slope
of Ekron
to the north,
went
toward Shikkeron,
crossed
to Mount
Baalah,
extended
to Jabneel,
and ended
at the sea.
\par }{\PP \VS{12}The western
border
was the Mediterranean
Sea.
These
were the borders
of the tribe
of Judah
and its
clans.
\par }{\PP \VS{13}Caleb
son
of Jephunneh
was assigned
Kiriath Arba
(that is
Hebron) within
the tribe
of Judah,
according
to the
{\ND{Lord}}’s
instructions
to Joshua.
(Arba was the father
of Anak.)
\VS{14}Caleb
drove
out from there
three
Anakites
– Sheshai,
Ahiman,
and Talmai,
descendants
of Anak.
\VS{15}From there
he attacked
the people
of Debir.
(Debir
used to be
called
Kiriath Sepher.)
\VS{16}Caleb
said,
“To the man who attacks
and captures
Kiriath Sepher
I will give
my daughter
Acsah
as a wife.”
\VS{17}When Othniel
son
of Kenaz,
Caleb’s brother,
captured
it, Caleb
gave
Acsah
his daughter
to him as a wife.
\par }{\PP \VS{18}One time Acsah came
and charmed
her father
so that she could ask him
for some land.
When she got down from her donkey,
Caleb
said
to her, “What would you like?”
\VS{19}She answered,
“Please give
me a special present.
Since
you have given
me land
in the Negev,
now give
me springs
of water.
So he gave
her both upper
and lower
springs.
\par }{\PP \VS{20}This
is the land assigned
to the tribe
of Judah
by its clans:
\VS{21}These cities
were located at the southern
extremity of Judah’s
tribal
land near the border
of Edom: Kabzeel,
Eder,
Jagur,
\VS{22}Kinah,
Dimonah,
Adadah,
\VS{23}Kedesh,
Hazor,
Ithnan,
\VS{24}Ziph,
Telem,
Bealoth,
\VS{25}Hazor Hadattah,
Kerioth
Hezron
(that
is, Hazor),
\VS{26}Amam,
Shema,
Moladah,
\VS{27}Hazar Gaddah,
Heshbon,
Beth Pelet,
\VS{28}Hazar Shual,
Beer Sheba,
Biziothiah,
\VS{29}Baalah,
Iim,
Ezem,
\VS{30}Eltolad,
Kesil,
Hormah,
\VS{31}Ziklag,
Madmannah,
Sansannah,
\VS{32}Lebaoth,
Shilhim,
Ain,
and Rimmon
– a total
of twenty-nine
cities
and their towns.
\par }{\PP \VS{33}These cities were in the lowlands: Eshtaol,
Zorah,
Ashnah,
\VS{34}Zanoah,
En Gannim,
Tappuah,
Enam,
\VS{35}Jarmuth,
Adullam,
Socoh,
Azekah,
\VS{36}Shaaraim,
Adithaim,
and Gederah
(or Gederothaim) – a total of fourteen cities and their towns.
\par }{\PP \VS{37}Zenan,
Hadashah,
Migdal Gad,
\VS{38}Dilean,
Mizpah,
Joktheel,
\VS{39}Lachish,
Bozkath,
Eglon,
\VS{40}Cabbon,
Lahmas,
Kitlish,
\VS{41}Gederoth,
Beth Dagon,
Naamah,
and Makkedah
– a total of sixteen
cities
and their towns.
\par }{\PP \VS{42}Libnah,
Ether,
Ashan,
\VS{43}Iphtah,
Ashnah,
Nezib,
\VS{44}Keilah,
Aczib,
and Mareshah
– a total of nine
cities
and their towns.
\par }{\PP \VS{45}Ekron
and its surrounding towns
and settlements;
\VS{46}from Ekron
westward,
all
those in the vicinity
of Ashdod
and their towns;
\VS{47}Ashdod
with its surrounding towns
and settlements,
and Gaza
with its surrounding towns
and settlements,
as far
as the Stream
of Egypt
and the border
at the Mediterranean
Sea.
\par }{\PP \VS{48}These cities were in the hill country: Shamir,
Jattir,
Socoh,
\VS{49}Dannah,
Kiriath Sannah
(that
is, Debir),
\VS{50}Anab,
Eshtemoh,
Anim,
\VS{51}Goshen,
Holon,
and Giloh
– a total of eleven
cities
and their towns.
\par }{\PP \VS{52}Arab,
Dumah,
Eshan,
\VS{53}Janim,
Beth Tappuah,
Aphekah,
\VS{54}Humtah,
Kiriath Arba
(that is,
Hebron), and Zior
– a total of nine
cities
and their towns.
\par }{\PP \VS{55}Maon,
Carmel,
Ziph,
Juttah,
\VS{56}Jezreel,
Jokdeam,
Zanoah,
\VS{57}Kain,
Gibeah,
and Timnah
– a total of ten
cities
and their towns.
\par }{\PP \VS{58}Halhul,
Beth Zur,
Gedor,
\VS{59}Maarath,
Beth Anoth,
and Eltekon
– a total of six
cities
and their towns.
\par }{\PP \VS{60}Kiriath Baal
(that is,
Kiriath Jearim) and Rabbah
– a total of two
cities
and their towns.
\par }{\PP \VS{61}These cities were in the desert: Beth Arabah,
Middin,
Secacah,
\VS{62}Nibshan,
the city of Salt,
and En Gedi
– a total of six
cities
and their towns.
\par }{\PP \VS{63}The men
of Judah
were unable
to conquer
the Jebusites
living in
Jerusalem.
The Jebusites
live
with
the people of Judah
in Jerusalem
to
this
very
day.

\par }\Chap{16}{\PP \VerseOne{1}The land allotted
to Joseph’s
descendants
extended
from the Jordan
at Jericho
\VS{2}The southern border extended
from Bethel
to Luz,
and crossed
to
Arkite
territory
at Ataroth.
\VS{3}It then descended
westward
to
Japhletite
territory,
as far as
the territory
of lower
Beth Horon
and Gezer,
and ended
at the sea.
\par }{\PP \VS{4}Joseph’s
descendants,
Manasseh
and Ephraim,
were assigned their land.
\VS{5}The territory
of the tribe
of Ephraim
by its clans
included the following: The border
of their assigned land
to the east
was Ataroth Addar
as far as
upper
Beth Horon.
\VS{6}It
then extended
on to the sea,
with Micmethath
on the north.
It
turned
eastward
to Taanath Shiloh
and crossed
it on the east
to Janoah.
\VS{7}It then descended
from Janoah
to Ataroth
and Naarah,
touched
Jericho,
and extended
to the Jordan River.
\VS{8}From Tappuah
it went
westward
to the Valley
of Kanah
and ended
at the sea.
This
is the land assigned to
the tribe
of Ephraim
by its clans.
\VS{9}Also included were the cities
set apart
for the tribe
of Ephraim
within
Manasseh’s
territory,
along with
their towns.
\par }{\PP \VS{10}The Ephraimites did not
conquer
the Canaanites
living
in Gezer.
The Canaanites
live
among
the Ephraimites
to
this
very day
and do hard labor
as their servants.

\par }\Chap{17}{\PP \VerseOne{1}The tribe
of Manasseh,
Joseph’s
firstborn son,
was
also allotted
land. The descendants of Makir,
Manasseh’s
firstborn
and the father
of Gilead,
received land, for
they were warriors.
They were
assigned Gilead
and Bashan.
\VS{2}The rest
of Manasseh’s
descendants
were
also assigned
land by their clans,
including the descendants
of Abiezer,
Helek,
Asriel,
Shechem,
Hepher,
and Shemida.
These
are the male
descendants
of Manasseh
son
of Joseph
by their clans.
\par }{\PP \VS{3}Now Zelophehad
son
of Hepher,
son
of Gilead,
son
of Makir,
son
of Manasseh,
had no
sons,
only
daughters.
These
are the names
of his daughters: Mahlah,
Noah,
Hoglah,
Milcah,
and Tirzah.
\VS{4}They went
before
Eleazar
the priest,
Joshua
son
of Nun,
and the leaders
and said,
“The
{\ND{Lord}}
told
Moses
to assign
us land
among
our relatives.”
So Joshua assigned
them land
among
their uncles,
as the
{\ND{Lord}}
had commanded.
\VS{5}Manasseh
was allotted
ten
shares
of land, in addition
to the land
of Gilead
and Bashan
east
of the Jordan,
\VS{6}for
the daughters
of Manasseh
were assigned
land among
his sons.
The land
of Gilead
belonged
to the rest
of the descendants
of Manasseh.
\par }{\PP \VS{7}The border
of Manasseh
went from Asher
to Micmethath
which
is near
Shechem.
It then went
south
toward
those who live
in Tappuah.
\VS{8}(The land
of Tappuah
belonged
to Manasseh,
but Tappuah,
located on the border
of Manasseh,
belonged to the tribe of
Ephraim.)
\VS{9}The border
then descended
southward
to the Valley of Kanah.
Ephraim
was assigned cities
there among
the cities
of Manasseh,
but the border
of Manasseh
was north
of the valley
and ended
at the sea.
\VS{10}Ephraim’s
territory was to the south,
and Manasseh’s to the north.
The sea
was Manasseh’s
western
border
and their territory touched
Asher
on the north
and Issachar
on the east.
\VS{11}Within Issachar’s
and Asher’s
territory Manasseh
was
assigned Beth Shean,
Ibleam,
the
residents
of Dor,
En Dor,
the residents
of Taanach,
the residents
of Megiddo,
the three
of Napheth,
and the towns surrounding all these cities.
\VS{12}But
the men
of Manasseh
were unable
to conquer
these
cities;
the Canaanites
managed
to remain
in those areas.
\VS{13}Whenever
the Israelites
were strong
militarily, they forced
the Canaanites
to do hard labor,
but they never
totally
conquered them.
\par }{\PP \VS{14}The descendants
of Joseph
said
to Joshua,
“Why
have you assigned
us only one
tribal allotment? After all,
we have
many
people,
for until
now the

{\ND{Lord}}
has enabled us to increase in number.”
\VS{15}Joshua
replied
to
them, “Since
you
have so many
people,
go up
into the forest
and clear out
a place
to live in the land
of the Perizzites
and Rephaites,
for
the hill country
of Ephraim
is too small for you.”
\VS{16}The descendants
of Joseph
said,
“The whole
hill country
is inadequate
for us, and the Canaanites
living
down in the valley
in Beth Shean
and its surrounding towns
and in the Valley
of Jezreel
have
chariots
with iron-rimmed wheels.”
\VS{17}Joshua
said
to
the family
of Joseph
– to both Ephraim
and Manasseh: “You have many
people
and great
military
strength.
You will not
have just one
tribal allotment.
\VS{18}The whole hill country
will be
yours; though it is a forest,
you can clear
it and it will be
entirely yours.
You can conquer
the Canaanites,
though they have chariots
with iron-rimmed
wheels and are strong.”

\par }\Chap{18}{\PP \VerseOne{1}The entire
Israelite
community
assembled
at Shiloh
and there
they set up the tent
of meeting.
Though they had subdued
the land,
\VS{2}seven
Israelite
tribes
had
not
been assigned
their allotted land.
\VS{3}So Joshua
said
to
the Israelites: “How
long
do you
intend to put off
occupying
the land
the {\ND{Lord}}
God
of your ancestors
has
given you?
\VS{4}Pick
three
men
from each tribe.
I will send
them out
to walk
through the land
and make a map of
it for me.
\VS{5}Divide
it into seven
regions.
Judah
will stay
in its territory
in the south,
and the family
of Joseph
in its
territory
in the north.
\VS{6}But as for you,
map
out the land
into seven
regions
and bring
it to
me.
I will
draw
lots
for you here
before
the {\ND{Lord}}
our God.
\VS{7}But
the Levites
will not
have an allotted portion
among
you, for
their inheritance
is to serve
the {\ND{Lord}}. Gad,
Reuben,
and the half-tribe
of Manasseh
have already received
their allotted
land east
of the Jordan
which
Moses
the
{\ND{Lord}}’s
servant
assigned them.”
\par }{\PP \VS{8}When
the men
started out,
Joshua
told
those going
to map
out the land,
“Go,
walk
through the land,
map
it out, and return
to me.
Then I will draw lots
for you before
the {\ND{Lord}}
here
in Shiloh.”
\VS{9}The men
journeyed
through
the land
and mapped it
and its cities
out
into seven
regions
on
a scroll.
Then they came
to
Joshua
at
the camp
in Shiloh.
\VS{10}Joshua
drew
lots
for them in Shiloh
before
the {\ND{Lord}}
and divided
the land
among the Israelites
according to their allotted portions.
\par }{\SH Benjamin’s Tribal Lands
\par }{\PP \VS{11}The first lot
belonged
to the tribe
of Benjamin
by its clans.
Their allotted
territory
was between
Judah
and Joseph.
\VS{12}Their northern
border
started
at
the Jordan,
went up
to
the slope
of Jericho
on the north,
ascended
westward
to the hill country,
and extended
to the desert
of Beth Aven.
\VS{13}It then crossed
from there
to Luz,
to
the slope
of Luz
to the south
(that
is, Bethel), and descended
to Ataroth Addar
located on
the hill
that
is south
of lower
Beth Horon.
\VS{14}It then turned on
the west
side
southward
from
the hill
near
Beth Horon
on the south
and extended
to
Kiriath Baal
(that
is, Kiriath Jearim), a city
belonging to the tribe
of Judah.
This
is the western
border.
\VS{15}The southern
side started on
the edge
of Kiriath Jearim
and extended
westward
to
the spring
of the waters
of Nephtoah.
\VS{16}The border
then descended
to
the edge
of the hill country
near
the Valley
of Ben Hinnom
located
in the Valley
of the Rephaites
to the north.
It descended
through the Valley
of Hinnom
to
the slope
of the Jebusites
to the south
and then down
to En Rogel.
\VS{17}It went
northward,
extending
to En Shemesh
and Geliloth
opposite
the Pass
of Adummim,
and descended
to the Stone
of Bohan
son
of Reuben.
\VS{18}It crossed
to
the slope
in front
of the Arabah
to the north
and descended
into the Arabah.
\VS{19}It then crossed
to
the slope
of Beth Hoglah
to the north
and ended
at
the northern
tip
of the Salt
Sea
at
the mouth
of the Jordan
River. This
was the southern
border.
\VS{20}The Jordan River
borders
it on
the east.
These
were the borders
of the land assigned
to the tribe of
Benjamin
by its clans.
\par }{\PP \VS{21}These cities
belonged to the tribe
of Benjamin
by its clans: Jericho,
Beth Hoglah,
Emek
Keziz,
\VS{22}Beth Arabah,
Zemaraim,
Bethel,
\VS{23}Avvim,
Parah,
Ophrah,
\VS{24}Kephar Ammoni,
Ophni,
and Geba
– a total of twelve
cities
and their towns.
\par }{\PP \VS{25}Gibeon,
Ramah,
Beeroth,
\VS{26}Mizpah,
Kephirah,
Mozah,
\VS{27}Rekem,
Irpeel,
Taralah,
\VS{28}Zelah,
Haeleph,
the Jebusite
city (that
is, Jerusalem), Gibeah,
and Kiriath
– a total of fourteen
cities
and their towns.
This
was the land assigned
to the tribe
of Benjamin
by its clans.

\par }\Chap{19}{\PP \VerseOne{1}The second
lot
belonged
to the tribe
of Simeon
by its clans.
\VS{2}Their
assigned
land
included Beer Sheba,
Moladah,
\VS{3}Hazar Shual,
Balah,
Ezem,
\VS{4}Eltolad,
Bethul,
Hormah,
\VS{5}Ziklag,
Beth Marcaboth,
Hazar Susah,
\VS{6}Beth Lebaoth,
and Sharuhen
– a total of thirteen
cities
and their towns,
\VS{7}Ain,
Rimmon,
Ether,
and Ashan
– a total of four
cities
and their towns,
\VS{8}as well as all
the towns
around
these
cities
as far as
Baalath Beer
(Ramah
of the Negev). This
was the land assigned
to the tribe
of Simeon
by its clans.
\VS{9}Simeon’s
assigned land
was taken from Judah’s
allotted portion,
for
Judah’s
territory
was too large
for them;
so Simeon
was assigned land
within Judah.
\par }{\SH Zebulun’s Tribal Lands
\par }{\PP \VS{10}The third
lot
belonged
to the tribe
of Zebulun
by its clans.
The border
of their
territory extended to Sarid.
\VS{11}Their border
went up
westward
to Maralah
and touched
Dabbesheth
and the valley
near
Jokneam.
\VS{12}From Sarid
it turned
eastward
to
the territory
of Kisloth Tabor,
extended
to
Daberath,
and went up
to Japhia.
\VS{13}From there
it crossed
eastward
to Gath Hepher
and Eth Kazin
and extended
to Rimmon,
turning toward
Neah.
\VS{14}It then turned
on the north
to Hannathon
and ended
at the Valley
of Iphtah El.
\VS{15}Their territory included Kattah,
Nahalal,
Shimron,
Idalah,
and Bethlehem;
in all they had twelve
cities
and their towns.
\VS{16}This
was the land assigned
to the tribe
of Zebulun
by its clans,
including these
cities
and their towns.
\par }{\SH Issachar’s Tribal Lands
\par }{\PP \VS{17}The fourth
lot
belonged
to the tribe
of Issachar
by its clans.
\VS{18}Their assigned land included
Jezreel,
Kesulloth,
Shunem,
\VS{19}Hapharaim,
Shion,
Anaharath,
\VS{20}Rabbith,
Kishion,
Ebez,
\VS{21}Remeth,
En Gannim,
En Haddah
and Beth Pazzez.
\VS{22}Their border
touched
Tabor,
Shahazumah,
and Beth Shemesh,
and ended
at the Jordan.
They had sixteen
cities
and their towns.
\VS{23}This
was the land assigned
to the tribe
of Issachar
by its clans,
including the cities
and their towns.
\par }{\SH Asher’s Tribal Lands
\par }{\PP \VS{24}The fifth
lot
belonged
to the tribe
of Asher
by its clans.
\VS{25}Their territory
included Helkath,
Hali,
Beten,
Acshaph,
\VS{26}Alammelech,
Amad,
and Mishal.
Their border touched
Carmel
to the west
and Shihor Libnath.
\VS{27}It turned
eastward
toward
Beth Dagon,
touched
Zebulun
and the Valley
of Iphtah El
to the north,
as well as the Valley of Emek
and Neiel,
and extended
to
Cabul
on the north
\VS{28}and on to Ebron,
Rehob,
Hammon,
and Kanah,
as far
as Greater
Sidon.
\VS{29}It then turned
toward Ramah
as far as
the fortified
city
of Tyre,
turned
to Hosah,
and ended
at the sea
near Hebel, Aczib,
\VS{30}Umah,
Aphek,
and Rehob.
In all they had twenty-two
cities
and their towns.
\VS{31}This
was the land assigned
to the tribe
of Asher
by its clans,
including these
cities
and their towns.
\par }{\SH Naphtali’s Tribal Lands
\par }{\PP \VS{32}The sixth
lot
belonged
to the tribe
of Naphtali
by its clans.
\VS{33}Their border
started
at Heleph
and the oak
of Zaanannim,
went to Adami
Nekeb,
Jabneel
and on to
Lakkum,
and ended at
the Jordan River.
\VS{34}It turned
westward
to Aznoth Tabor,
extended from
there
to Hukok,
touched
Zebulun
on
the south,
Asher
on the west,
and the Jordan
on the east.
\VS{35}The fortified
cities
included Ziddim,
Zer,
Hammath,
Rakkath,
Kinnereth,
\VS{36}Adamah,
Ramah,
Hazor,
\VS{37}Kedesh,
Edrei,
En Hazor,
\VS{38}Yiron,
Migdal El,
Horem,
Beth Anath,
and Beth Shemesh.
In all they had nineteen
cities
and their towns.
\VS{39}This
was the land assigned
to the tribe
of Naphtali
by its clans,
including the cities
and their towns.
\par }{\SH Dan’s Tribal Lands
\par }{\PP \VS{40}The seventh
lot
belonged
to the tribe
of Dan
by its clans.
\VS{41}Their assigned land
included
Zorah,
Eshtaol,
Ir Shemesh,
\VS{42}Shaalabbin,
Aijalon,
Ithlah,
\VS{43}Elon,
Timnah,
Ekron,
\VS{44}Eltekeh,
Gibbethon,
Baalath,
\VS{45}Jehud,
Bene Berak,
Gath Rimmon,
\VS{46}the waters of Jarkon,
and Rakkon,
including the territory
in front
of Joppa.
\VS{47}(The Danites
failed
to conquer their territory,
so they
went up
and fought
with
Leshem
and captured
it. They put
the sword
to it, took possession
of it, and lived
in it.
They renamed
it Dan
after their ancestor. )
\VS{48}This
was the land assigned
to the tribe
of Dan
by its clans,
including these
cities
and their towns.
\par }{\SH Joshua Receives Land
\par }{\PP \VS{49}When they finished
dividing
the land
into its regions,
the Israelites
gave
Joshua
son
of Nun
some land.
\VS{50}As
the {\ND{Lord}}
had instructed,
they gave
him the city
he requested
– Timnath
Serah in the Ephraimite
hill country.
He built
up the city
and lived in it.
\par }{\PP \VS{51}These
are the land
assignments
which
Eleazar
the priest,
Joshua
son
of Nun,
and the Israelite
tribal leaders
made by drawing lots
in Shiloh
before
the {\ND{Lord}}
at the entrance
of the tent
of meeting.
So they finished
dividing up
the land.

\par }\Chap{20}{\PP \VerseOne{1}The
{\ND{Lord}}
instructed
Joshua:
\VS{2}“Have
the Israelites
select
the cities
of refuge
that
I told
you about through
Moses.
\VS{3}Anyone who accidentally
kills
someone can escape
there;
these cities will be
a place of asylum
from the avenger
of blood.
\VS{4}The one who committed manslaughter should escape
to
one
of these
cities,
stand
at the entrance
of the city
gate,
and present
his case
to the leaders
of that city.
They should then bring
him into
the city,
give
him a place
to stay, and let him live there.
\VS{5}When
the avenger
of blood
comes
after
him, they must not
hand over
to him
the one who committed
manslaughter,
for
he accidentally
killed
his fellow
man without
premeditation.
\VS{6}He must remain
in that city
until
his case
is decided by the assembly
and the high
priest
dies.
Then
the one who committed manslaughter
may return
home
to
the city
from which
he escaped.”
\par }{\PP \VS{7}So
they selected
Kedesh
in Galilee
in the hill country
of Naphtali,
Shechem
in the hill country
of Ephraim,
and Kiriath Arba
(that
is, Hebron) in the hill country
of Judah.
\VS{8}Beyond
the Jordan
east
of Jericho
they selected
Bezer
in the desert
on the plain
belonging to the tribe
of Reuben,
Ramoth
in Gilead
belonging to the tribe
of Gad,
and Golan
in Bashan
belonging to the tribe
of Manasseh.
\VS{9}These
were
the cities
of refuge appointed
for all
the Israelites
and for resident foreigners
living
among
them. Anyone
who accidentally
killed
someone
could escape
there
and not
be executed
by
the avenger
of blood,
at least until
his case was reviewed
by the assembly.

\par }\Chap{21}{\PP \VerseOne{1}The tribal
leaders
of the Levites
went
before
Eleazar
the priest
and Joshua
son
of Nun
and the Israelite
tribal
leaders
\VS{2}in Shiloh
in the land
of Canaan
and said,
“The
{\ND{Lord}}
told
Moses
to assign
us cities
in which to live
along with the grazing areas
for our cattle.”
\VS{3}So
the Israelites
assigned
these
cities
and their grazing areas
to the Levites
from their own
holdings,
as the
{\ND{Lord}} had instructed.
\par }{\PP \VS{4}The first lot
belonged
to the Kohathite
clans.
The Levites
who were descendants
of Aaron
the priest
were
allotted
thirteen
cities
from
the tribes
of Judah,
Simeon,
and Benjamin.
\VS{5}The rest of Kohath’s
descendants
were allotted
ten
cities
from the clans
of the tribe
of Ephraim,
and from the tribe
of Dan
and the half-tribe
of Manasseh.
\VS{6}Gershon’s
descendants
were allotted
thirteen
cities
from the clans
of the tribe
of Issachar,
and from the tribes
of Asher
and Naphtali
and the half-tribe
of Manasseh
in Bashan.
\VS{7}Merari’s
descendants
by their clans
were allotted twelve
cities
from the tribes
of Reuben,
Gad,
and Zebulun.
\VS{8}So the Israelites
assigned
to the Levites
by lot
these
cities
and their grazing areas,
as
the {\ND{Lord}}
had
instructed
Moses.
\par }{\PP \VS{9}They assigned
from the tribes
of Judah
and Simeon
the cities
listed below.
\VS{10}They were assigned to the Kohathite
clans
of the Levites
who were descendants
of Aaron,
for
the first
lot
belonged
to them.
\VS{11}They
assigned
them Kiriath Arba
(Arba was the father
of Anak), that
is, Hebron,
in the hill country
of Judah,
along with
its surrounding
grazing areas.
\VS{12}(Now the city’s
fields
and surrounding
towns
they had assigned
to Caleb
son
of Jephunneh
as his property.)
\VS{13}So to the descendants
of Aaron
the priest
they assigned
Hebron
(a city
of refuge
for one who committed manslaughter), Libnah,
\VS{14}Jattir,
Eshtemoa,
\VS{15}Holon,
Debir,
\VS{16}Ain,
Juttah,
and Beth Shemesh,
along
with the grazing areas of each – a total of nine cities taken from these two tribes.
\VS{17}From the tribe
of Benjamin
they assigned
Gibeon,
Geba,
\VS{18}Anathoth,
and
Almon,
along with the grazing areas of each – a total of four cities.
\VS{19}The priests
descended
from Aaron
received
thirteen
cities
and their grazing areas.
\par }{\PP \VS{20}The rest
of the Kohathite
clans
of the Levites
were
allotted
cities
from the tribe
of Ephraim.
\VS{21}They assigned
them Shechem
(a city
of refuge
for one who committed manslaughter) in the hill country
of Ephraim,
Gezer,
\VS{22}Kibzaim,
and Beth Horon,
along with the grazing areas of each – a total of four cities.
\VS{23}From the tribe
of Dan
they assigned Eltekeh,
Gibbethon,
\VS{24}Aijalon,
and Gath Rimmon,
along
with the grazing areas of each – a total of four cities.
\VS{25}From the half-tribe
of Manasseh
they assigned Taanach
and Gath Rimmon,
along
with the grazing areas of each – a total of two cities.
\VS{26}The rest
of the Kohathite
clans
received ten
cities
and their grazing areas.
\par }{\PP \VS{27}They assigned
to the
Gershonite
clans
of the Levites
the following cities: from the half-tribe
of Manasseh: Golan
in Bashan
(a city
of refuge
for one who
committed manslaughter) and Beeshtarah,
along with the grazing areas of each – a total of two cities;
\VS{28}from the tribe
of Issachar: Kishon,
Daberath,
\VS{29}Jarmuth,
and En Gannim,
along
with the grazing areas of each – a total of four cities;
\VS{30}from the tribe
of Asher: Mishal,
Abdon,
\VS{31}Helkath,
and
Rehob,
along with the grazing areas of each – a total of four cities;
\VS{32}from the tribe
of Naphtali: Kedesh
in Galilee
(a city
of refuge
for one who committed manslaughter), Hammoth Dor,
and Kartan,
along
with the grazing areas of each – a total of three cities.
\VS{33}The Gershonite
clans
received thirteen
cities
and their grazing areas.
\par }{\PP \VS{34}They assigned to the Merarite
clans
(the remaining
Levites) the following cities: from the tribe
of Zebulun: Jokneam,
Kartah,
\VS{35}Dimnah,
and Nahalal,
along with the grazing areas of each – a total of four cities;
\VS{36}from the tribe
of Reuben: Bezer,
Jahaz,
\VS{37}Kedemoth,
and
Mephaath,
along with the grazing areas of each – a total of four cities;
\VS{38}from the tribe
of Gad: Ramoth
in Gilead
(a city
of refuge
for one who committed manslaughter), Mahanaim,
\VS{39}Heshbon,
and
Jazer,
along with the grazing areas of each – a total of four cities.
\VS{40}The Merarite
clans
(the remaining
Levites) were
allotted
twelve
cities.
\par }{\PP \VS{41}The Levites
received within
the land
owned
by the Israelites
forty-eight
cities
in all
and their grazing areas.
\VS{42}Each of these
cities
had grazing areas
around
it; they were
alike
in this
regard.
\par }{\PP \VS{43}So the
{\ND{Lord}}
gave
Israel
all
the land
he had
solemnly
promised
to their ancestors,
and they conquered
it and lived in it.
\VS{44}The
{\ND{Lord}}
made them secure,
in fulfillment
of all
he had
solemnly
promised their ancestors.
None
of their enemies
could resist them.
\VS{45}Not one
of the
{\ND{Lord}}’s
faithful
promises
to
the family
of Israel
was left unfulfilled;
every
one was realized.

\par }\Chap{22}{\PP \VerseOne{1}Then
Joshua
summoned
the Reubenites,
Gadites,
and the half-tribe
of Manasseh
\VS{2}and told
them: “You
have carried
out all
the instructions
of Moses
the
{\ND{Lord}}’s
servant,
and you have
obeyed
all
I have
told you.
\VS{3}You have not
abandoned
your fellow
Israelites this
entire time,
right
up
to this
very
day.
You have completed
the task
given
you by
the {\ND{Lord}}
your God.
\VS{4}Now
the {\ND{Lord}}
your God
has made your fellow
Israelites secure,
just
as he promised
them. So now
you may turn
around and go
to
your homes
in your own
land
which
Moses
the
{\ND{Lord}}’s
servant
assigned to you east
of the Jordan.
\VS{5}But
carefully
obey
the commands
and instructions
Moses
the
{\ND{Lord}}’s
servant
gave
you. Love
the

{\ND{Lord}}
your God,
follow
all
his instructions,
obey
his commands,
be loyal
to him, and serve
him with all
your heart
and being!”
\par }{\PP \VS{6}Joshua
rewarded
them and sent
them on
their way;
they returned
to
their homes.
\VS{7}(Now to one half-tribe
of Manasseh,
Moses
had assigned
land in Bashan;
and to the other half
Joshua
had assigned
land on the west
side
of the Jordan
with
their fellow
Israelites.) When
Joshua
sent
them
home,
he rewarded
them,
\VS{8}saying,
“Take home
great
wealth,
a lot
of cattle,
silver,
gold,
bronze,
iron,
and a lot of clothing.
Divide
up the goods
captured from your enemies
with
your brothers.”
\VS{9}So the Reubenites,
Gadites,
and half-tribe
of Manasseh
left
the Israelites
in Shiloh
in the land
of Canaan
and headed
home to
their own
land
in Gilead,
which
they acquired
by the
{\ND{Lord}}’s
command
through
Moses.
\par }{\SH Civil War is Averted
\par }{\PP \VS{10}The Reubenites,
Gadites,
and half-tribe
of Manasseh
came
to
Geliloth
near the Jordan
in the land
of Canaan
and built
there,
near the Jordan,
an impressive
altar.
\VS{11}The Israelites
received
this report: “Look,
the Reubenites,
Gadites,
and half-tribe
of Manasseh
have built
an altar
at
the entrance
to the land
of Canaan,
at Geliloth
near the Jordan
on the Israelite
side.”
\VS{12}When the Israelites
heard
this, the entire
Israelite
community
assembled
at Shiloh
to launch
an attack
against them.
\par }{\PP \VS{13}The Israelites
sent
Phinehas,
son
of Eleazar,
the priest,
to
the land
of Gilead
to
the Reubenites,
Gadites,
and the half-tribe
of Manasseh.
\VS{14}He was accompanied
by ten
leaders,
one
from each
of the Israelite
tribes,
each one
a family
leader
among the Israelite clans.
\VS{15}They went
to
the land
of Gilead
to
the Reubenites,
Gadites,
and the half-tribe
of Manasseh,
and said
to them:
\VS{16}“The
entire
community
of the {\ND{Lord}}
says,
‘Why
have you disobeyed
the God
of Israel
by turning
back
today
from following
the {\ND{Lord}}? You built
an altar
for yourselves and have rebelled
today
against the
{\ND{Lord}}.
\VS{17}The sin
we committed at Peor
was bad enough.
To this
very
day
we have
not
purified
ourselves;
it even
brought a plague
on the community
of the {\ND{Lord}}.
\VS{18}Now today
you
dare to turn back
from following
the {\ND{Lord}}! You
are rebelling
today
against the
{\ND{Lord}}; tomorrow
he may break out in anger
against the entire
community
of Israel.
\VS{19}But
if
your own
land
is impure,
cross over
to
the
{\ND{Lord}}’s
own
land,
where the
{\ND{Lord}}
himself lives,
and settle down
among
us. But don’t
rebel
against
the {\ND{Lord}}
or
us
by building
for yourselves an altar
aside
from the altar
of the {\ND{Lord}}
our God.
\VS{20}When Achan
son
of Zerah
disobeyed
the command about the city’s
riches,
the entire
Israelite
community
was judged,
though
only one
man
had sinned. He most certainly
died
for his sin!’ ”
\par }{\PP \VS{21}The Reubenites,
Gadites,
and the half-tribe
of Manasseh
answered
the leaders
of the Israelite clans:
\VS{22}“El,
God,
the {\ND{Lord}}! El,
God,
the {\ND{Lord}}! He knows
the truth! Israel
must also know! If
we have rebelled
or
disobeyed
the {\ND{Lord}}, don’t
spare
us today!
\VS{23}If we have built
an altar
for ourselves to turn back
from following
the {\ND{Lord}}
by making burnt sacrifices
and grain offerings
on it,
or
by
offering tokens of peace
on
it, the
{\ND{Lord}}
himself will punish us.
\VS{24}We swear we have done
this
because we were worried
that in the future
your descendants
would say
to our descendants,
‘What
relationship do you have with the
{\ND{Lord}}
God
of Israel?
\VS{25}The
{\ND{Lord}}
made the Jordan
a boundary
between
us and you Reubenites
and Gadites.
You have no
right to worship
the {\ND{Lord}}.’ In this way
your descendants
might cause our descendants
to stop
obeying
the

{\ND{Lord}}.
\VS{26}So
we decided
to build
this altar,
not
for burnt offerings
and sacrifices,
\VS{27}but
as a reminder
to us and you, and to our descendants
who follow
us,
that we
will honor
the {\ND{Lord}}
in his very presence
with burnt offerings,
sacrifices,
and tokens of peace.
Then in the future
your descendants
will not be
able to say
to our descendants,
‘You have no
right to worship
the {\ND{Lord}}.’
\VS{28}We said,
‘If
in the future
they say
such a thing to
us or to our descendants, we
will reply,
“See
the
model
of the
{\ND{Lord}}’s
altar
that
our ancestors
made,
not
for burnt offerings
or
sacrifices,
but
as a reminder to us and you.” ’
\VS{29}Far be it
from
us to rebel
against the
{\ND{Lord}}
by turning back
today
from following after
the {\ND{Lord}}
by building
an altar
for burnt offerings,
sacrifices,
and tokens of peace
aside
from the altar
of the {\ND{Lord}}
our God
located
in front
of his dwelling place!”
\par }{\PP \VS{30}When
Phinehas
the priest
and the community
leaders
and clan
leaders
who
accompanied
him heard
the defense
of the Reubenites,
Gadites,
and the Manassehites,
they were satisfied.
\VS{31}Phinehas,
son
of Eleazar,
the priest,
said
to
the Reubenites,
Gadites,
and the Manassehites, “Today
we know
that
the {\ND{Lord}}
is among
us, because you have
not
disobeyed
the {\ND{Lord}}
in this.
Now
you have rescued
the
Israelites
from the
{\ND{Lord}}’s
judgment.”
\par }{\PP \VS{32}Phinehas,
son
of Eleazar,
the priest,
and the leaders
left the Reubenites
and Gadites
in the land
of Gilead
and reported
back
to
the Israelites
in the land
of Canaan.
\VS{33}The Israelites
were satisfied
with their report
and gave thanks
to God.
They said
nothing
more about launching
an attack
to destroy
the
land
in which
the Reubenites
and Gadites
lived.
\VS{34}The Reubenites
and Gadites
named
the altar,
“Surely
it is
a Reminder
to us that
the {\ND{Lord}}
is God.”

\par }\Chap{23}{\PP \VerseOne{1}A long
time
passed
after
the {\ND{Lord}}
made Israel
secure
from all
their enemies,
and Joshua
was very old.
\VS{2}So Joshua
summoned
all
Israel,
including the elders,
rulers,
judges,
and leaders,
and told
them: “I
am very old.
\VS{3}You
saw
everything
the {\ND{Lord}}
your God
did
to all
these
nations
on your behalf,
for
the {\ND{Lord}}
your God
fights for you.
\VS{4}See,
I have parceled out
to your tribes
these
remaining
nations,
from
the Jordan
to the Mediterranean Sea
in the west,
including all
the nations
I defeated.
\VS{5}The
{\ND{Lord}}
your God
will drive
them out from before
you and remove them, so you can occupy their
land
as
the {\ND{Lord}}
your God
promised you.
\VS{6}Be very
strong! Carefully
obey
all
that is written
in the law
scroll
of Moses
so you won’t swerve
from
it to the right
or the left,
\VS{7}or
associate
with these
nations
that remain
near you. You must not
invoke
or
make solemn declarations
by the names
of their gods! You must not
worship
or
bow down to them!
\VS{8}But
you must
be loyal
to the
{\ND{Lord}}
your God,
as you have been
to
this
very
day.
\par }{\PP \VS{9}“The
{\ND{Lord}}
drove
out from before
you
great
and mighty
nations;
no
one has been able to resist
you
to this very
day.
\VS{10}One
of you makes
a thousand
run
away, for
the {\ND{Lord}}
your God
fights
for you as
he promised you he would.
\VS{11}Watch
yourselves
carefully! Love
the {\ND{Lord}}
your God!
\VS{12}But if
you ever turn
away
and make alliances
with these
nations
that remain
near you, and intermarry
with
them and establish friendly relations
with them,
\VS{13}know
for
certain
that
the {\ND{Lord}}
our God
will no
longer
drive out
these
nations
from before
you. They will trap
and ensnare
you; they will be
a whip
that
tears your sides
and thorns
that blind your eyes
until
you disappear
from this
good
land
the {\ND{Lord}}
your God
gave you.
\par }{\PP \VS{14}“Look,
today
I
am about to die.
You know
with all
your heart
and being
that
not
even one
of all
the faithful
promises
the {\ND{Lord}}
your God
made
to you is left unfulfilled;
every
one was realized
– not one
promise
is unfulfilled!
\VS{15}But
in the same way
every
faithful promise
the {\ND{Lord}}
your God
made to
you has been
realized,
it is just as certain, if you disobey, that
the {\ND{Lord}}
will bring
on
you every
judgment
until
he destroys
you from this
good
land
which
the {\ND{Lord}}
your God
gave you.
\VS{16}If you violate
the covenantal
laws of the
{\ND{Lord}}
your God
which
he commanded
you to keep,
and follow, worship,
and bow down
to other
gods,
the {\ND{Lord}}
will be very angry
with you and you will disappear
quickly
from the good
land
which
he gave to you.”

\par }\Chap{24}{\PP \VerseOne{1}Joshua
assembled
all
the Israelite
tribes
at Shechem.
He summoned
Israel’s
elders,
rulers,
judges,
and leaders,
and they appeared
before
God.
\VS{2}Joshua
told
all
the people,
“Here is what
the {\ND{Lord}}
God
of Israel
says: ‘In the distant past
your ancestors
lived
beyond
the Euphrates River,
including Terah
the father
of Abraham
and Nahor.
They worshiped
other
gods,
\VS{3}but I took
your father
Abraham
from beyond
the Euphrates
and brought
him into the entire
land
of Canaan.
I made his descendants
numerous;
I gave
him Isaac,
\VS{4}and to Isaac
I gave
Jacob
and Esau.
To Esau
I assigned
Mount
Seir,
while Jacob
and his sons
went down
to Egypt.
\VS{5}I sent
Moses
and Aaron,
and I struck
Egypt
down when I intervened
in their
land. Then I brought you out.
\VS{6}When I brought
your fathers
out of Egypt,
you arrived
at the sea.
The Egyptians
chased
your fathers
with chariots
and horsemen
to the Red
Sea.
\VS{7}Your fathers cried out
for help to
the {\ND{Lord}}; he made
the area between
you and the Egyptians
dark,
and then drowned
them in
the sea.
You witnessed
with your very own eyes
what
I did
in Egypt.
You lived
in the wilderness
for a long
time.
\VS{8}Then I brought
you to
the land
of the Amorites
who lived
east
of the Jordan.
They fought
with
you, but I handed
them over
to you; you conquered
their land
and I destroyed
them from before you.
\VS{9}Balak
son
of Zippor,
king
of Moab,
launched an attack against
Israel.
He summoned
Balaam
son
of Beor
to call down judgment on you.
\VS{10}I refused
to respond
to Balaam;
he kept
prophesying
good things about you, and I rescued
you from his power.
\VS{11}You crossed
the
Jordan
and came
to
Jericho.
The leaders
of Jericho,
as well as the Amorites,
Perizzites,
Canaanites,
Hittites,
Girgashites,
Hivites,
and Jebusites,
fought
with you, but I handed
them over to you.
\VS{12}I sent
terror
ahead
of you to drive out
before
you the two
Amorite
kings.
I gave you the victory; it was not
by your swords
or
bows.
\VS{13}I gave
you a land
in which
you had not
worked
hard; you took up residence
in cities
you did not
build
and you
are eating
the produce of vineyards
and olive groves
you did not
plant.’
\par }{\PP \VS{14}Now
obey
the {\ND{Lord}}
and worship
him with integrity
and loyalty.
Put aside
the
gods
your ancestors
worshiped
beyond
the Euphrates
and in Egypt
and worship
the {\ND{Lord}}.
\VS{15}If
you have
no desire
to worship
the

{\ND{Lord}}, choose
today
whom
you will worship,
whether
it be the
gods
whom
your ancestors
worshiped
beyond
the Euphrates,
or
the
gods
of the Amorites
in whose
land
you
are living.
But I
and my family
will worship
the

{\ND{Lord}}!”
\par }{\PP \VS{16}The people
responded,
“Far be it
from us to abandon
the {\ND{Lord}}
so we can worship
other
gods!
\VS{17}For
the {\ND{Lord}}
our God
took
us
and our fathers
out of slavery
in the land
of Egypt
and performed
these
awesome
miracles
before our very eyes.
He continually protected
us as
we traveled
and when we passed
through
nations.
\VS{18}The
{\ND{Lord}}
drove
out from before
us all
the nations,
including the Amorites
who lived
in the land.
So we
too
will worship
the {\ND{Lord}}, for
he is
our God!”
\par }{\PP \VS{19}Joshua
warned
the people,
“You will not
keep worshiping
the {\ND{Lord}}, for
he is
a holy
God.
He is
a jealous
God
who will not
forgive
your rebellion
or your sins.
\VS{20}If
you abandon
the

{\ND{Lord}}
and worship
foreign
gods,
he will turn
against you; he will bring disaster
on you and destroy
you, though he once treated
you well.”
\par }{\PP \VS{21}The people
said
to
Joshua,
“No! We really
will worship
the {\ND{Lord}}!”
\VS{22}Joshua
said
to
the people,
“Do you agree to be witnesses
against yourselves that
you
have chosen
to worship
the {\ND{Lord}}?” They replied,
“We are witnesses!”
\VS{23}Joshua said, “Now
put aside
the foreign
gods
that
are among
you and submit
to
the {\ND{Lord}}
God
of Israel.”
\par }{\PP \VS{24}The people
said
to
Joshua,
“We will worship
the {\ND{Lord}}
our God
and obey him.”
\par }{\PP \VS{25}That day
Joshua
drew up
an agreement
for the people,
and he established
rules
and regulations
for them in Shechem.
\VS{26}Joshua
wrote
these
words
in the Law
Scroll
of God.
He then took
a large
stone
and set
it up there
under
the oak
tree near the
{\ND{Lord}}’s
shrine.
\VS{27}Joshua
said
to
all
the people,
“Look,
this
stone
will be
a witness
against you, for
it
has heard
everything
the {\ND{Lord}}
said
to us.
It will be
a witness against
you if you
deny
your God.”
\VS{28}When
Joshua
dismissed
the people,
they went
to their allotted
portions of land.
\par }{\SH An Era Ends
\par }{\PP \VS{29}After
all this
Joshua
son
of Nun,
the
{\ND{Lord}}’s
servant,
died
at the age
of one hundred
ten.
\VS{30}They buried
him in his allotted
territory
in Timnath
Serah
in the hill country
of Ephraim,
north
of Mount
Gaash.
\VS{31}Israel
worshiped
the {\ND{Lord}}
throughout
Joshua’s
lifetime
and as long as
the elderly men
who
outlived
him
remained alive. These men had
experienced
firsthand
everything
the {\ND{Lord}}
had
done
for Israel.
\par }{\PP \VS{32}The
bones
of Joseph,
which
the Israelites
had brought up
from Egypt,
were buried
at Shechem
in the part
of the field
that
Jacob
bought
from the
sons
of Hamor,
the father
of Shechem,
for one hundred
pieces of money.
So it became
the inheritance
of the tribe
of Joseph.
\par }{\PP \VS{33}Eleazar
son
of Aaron
died,
and they buried
him in Gibeah
in the hill country
of Ephraim,
where his son
Phinehas
had
been assigned land.
\par }