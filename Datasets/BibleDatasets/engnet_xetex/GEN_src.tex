\NormalFont\ShortTitle{Genesis}
{\MT Genesis

\par }\ChapOne{1}{\SH The Creation of the World
\par }{\PP \VerseOne{1}In the beginning
God
created
the heavens
and the
earth.
\par }{\PP \VS{2}Now
the earth
was without shape
and empty,
and darkness
was over
the surface
of the watery deep,
but the Spirit
of God
was moving
over
the surface
of the water.
\VS{3}God
said, “Let there be
light.”
And there
was light!
\VS{4}God
saw
that
the light
was good,
so God
separated
the light
from the darkness.
\VS{5}God
called
the light
“day”
and the darkness “night.”
There was
evening,
and there was
morning,
marking the first
day.
\par }{\PP \VS{6}God
said,
“Let there be
an expanse
in the midst
of the waters
and let it
separate
water
from water.
\VS{7}So God
made
the expanse
and separated
the water
under
the expanse
from the water
above
it.
It was
so.
\VS{8}God
called
the expanse
“sky.”
There was
evening,
and there was
morning,
a second
day.
\par }{\PP \VS{9}God
said,
“Let
the water
under
the sky
be gathered to
one
place
and let dry ground
appear.”
It was
so.
\VS{10}God
called
the dry ground
“land”
and the gathered
waters
he called
“seas.”
God
saw
that
it was good.
\par }{\PP \VS{11}God
said,
“Let
the land
produce vegetation: plants
yielding
seeds
according to their kinds,
and trees
bearing
fruit
with seed
in it according
to their kinds.”
It was
so.
\VS{12}The land
produced
vegetation
– plants
yielding
seeds
according to their kinds,
and trees
bearing
fruit
with seed
in it according to their kinds.
God
saw
that
it was good.
\VS{13}There was
evening,
and there was
morning,
a third
day.
\par }{\PP \VS{14}God
said,
“Let there be
lights
in the expanse
of the sky
to separate
the day
from the night,
and let them be
signs
to indicate
seasons and days
and years,
\VS{15}and let them serve as lights
in the expanse
of the sky
to give light
on
the earth.”
It was
so.
\VS{16}God
made
two
great
lights –
the greater
light
to rule over
the day
and the lesser
light
to rule over
the night.
He made the stars also.
\VS{17}God
placed
the lights in the expanse
of the sky
to shine
on
the earth,
\VS{18}to preside
over the day
and the night,
and to separate
the light
from the darkness.
God
saw
that
it was good.
\VS{19}There was
evening,
and there was
morning,
a fourth
day.
\par }{\PP \VS{20}God
said,
“Let the water
swarm
with swarms
of living
creatures
and let birds
fly
above
the earth
across
the expanse
of the sky.”
\VS{21}God
created
the great
sea creatures
and every
living
and moving thing
with which
the water
swarmed,
according to their kinds,
and every
winged
bird
according to its kind.
God
saw
that
it was good.
\VS{22}God
blessed
them
and said,
“Be fruitful
and multiply
and fill
the
water
in the seas,
and let the birds
multiply
on the earth.”
\VS{23}There was
evening,
and there was
morning,
a fifth
day.
\par }{\PP \VS{24}God
said,
“Let the land
produce living
creatures
according to their kinds: cattle,
creeping things,
and wild animals,
each according to its kind.”
It was so.
\VS{25}God
made
the wild animals
according
to their kinds,
the cattle
according to their kinds,
and all
the creatures that creep
along the ground
according to their kinds.
God
saw
that
it was good.
\par }{\PP \VS{26}Then God
said,
“Let us make
\par }{\PP humankind
in our image,
after our likeness,
so they may rule
over the fish
of the sea
and the birds
of the air,
over the cattle,
and over all
the earth,
and over all
the creatures
that move
on
the earth.”
\par }{\Q \VS{27}God
created
humankind
in his own image,
\par }{\Q in the image
of God
he created
them,

\par }{\Q male
and female
he created them.
\par }{\PP \VS{28}God
blessed
them
and said
to them, “Be fruitful
and multiply! Fill
the
earth
and subdue
it! Rule
over the fish
of the sea
and the birds
of the air
and every
creature
that moves
on
the ground.”
\VS{29}Then God
said,
“I now
give
you every
seed-bearing
plant
on
the face
of the entire
earth
and every
tree
that has
fruit
with seed
in it. They will be
yours for food.
\VS{30}And to all
the animals
of the earth,
and to every
bird
of the air,
and to all
the creatures that move
on
the ground
– everything that has
the breath
of life in it – I give every green plant for food.” It was so.
\par }{\PP \VS{31}God
saw
all
that he had
made
– and it was
very
good! There was
evening,
and there was
morning,
the sixth
day.

\par }\Chap{2}{\PP \VerseOne{1}The heavens
and the earth
were completed
with everything
that was in them.
\VS{2}By the seventh
day
God
finished
the work
that he had
been doing,
and he ceased
on
the seventh
day
all
the work
that
he had been
doing.
\VS{3}God
blessed
the seventh
day
and made
it holy
because
on it he ceased
all
the work
that he had
been
doing
in creation.
\par }{\SH The Creation of Man and Woman
\par }{\PP \VS{4}This
is the account
of the heavens
and
the earth
when they were created
– when
the {\ND{Lord}}
God
made
the earth
and heavens.
\par }{\PP \VS{5}Now no shrub
of the field
had yet
grown on the earth,
and no plant
of the field
had yet
sprouted,
for
the {\ND{Lord}}
God
had not
caused it to rain
on
the earth,
and there was no
man
to cultivate
the
ground.
\VS{6}Springs
would well up
from
the earth
and water
the whole
surface
of the ground.
\VS{7}The
{\ND{Lord}}
God
formed
the
man
from
the soil
of the ground
and breathed
into his nostrils
the breath
of life,
and the man
became a living
being.
\par }{\PP \VS{8}The
{\ND{Lord}}
God
planted
an orchard
in the east,
in Eden;
and there
he placed
the man
he had
formed.
\VS{9}The
{\ND{Lord}}
God
made all
kinds of trees grow
from
the soil,
every
tree
that was pleasing
to look
at and good
for food.
(Now the tree
of life
and the tree
of the knowledge
of good
and evil
were in the middle
of the orchard.)
\par }{\PP \VS{10}Now a river
flows
from Eden
to
water
the orchard,
and from there
it divides
into four
headstreams.
\VS{11}The name
of the first
is Pishon;
it runs
through the entire
land
of Havilah,
where
there
is gold.
\VS{12}(The gold
of that
land
is pure;
pearls
and lapis
lazuli
are also there).
\VS{13}The name
of the second
river
is
Gihon;
it
runs
through the entire
land
of Cush.
\VS{14}The name
of the third
river
is
Tigris;
it
runs
along the east
side of Assyria.
The fourth
river
is
the Euphrates.
\par }{\PP \VS{15}The
{\ND{Lord}}
God
took
the man
and placed
him in the orchard
in Eden
to care for it and to maintain it.
\VS{16}Then the
{\ND{Lord}}
God
commanded
the man,
“You may freely
eat
fruit from every
tree
of the orchard,
\VS{17}but you must not
eat
from
the tree
of the knowledge
of good
and evil,
for
when
you eat
from
it you will surely
die.”
\par }{\PP \VS{18}The
{\ND{Lord}}
God
said,
“It is not
good
for the man
to be alone.
I will make
a companion for him
who corresponds to him.”
\VS{19}The
{\ND{Lord}}
God
formed
out
of the ground
every
living
animal of the field
and every
bird
of the air.
He brought
them to
the man
to see
what
he would name
them, and whatever
the man
called
each living creature,
that was
its name.
\VS{20}So
the man
named
all
the animals,
the birds
of the air,
and the living creatures
of the field,
but for Adam
no
companion who
corresponded to him
was found.
\VS{21}So the
{\ND{Lord}}
God
caused the man
to fall
into a deep sleep,
and while he was asleep,
he took
part of the man’s
side
and closed up
the place
with flesh.
\VS{22}Then
the {\ND{Lord}}
God
made a woman
from
the part
he had
taken
out
of the man,
and he brought
her to
the man.
\VS{23}Then the man
said,
\par }{\Q “This
one at last
is bone
of my bones
\par }{\Q and flesh
of my flesh;
\par }{\Q this
one will be called ‘woman,’
\par }{\Q for
she was taken
out of man.”
\par }{\PP \VS{24}That is why
a man
leaves
his father
and mother
and unites
with his wife,
and they become a new family.
\VS{25}The man
and his wife
were both
naked,
but they were not
ashamed.

\par }\Chap{3}{\PP \VerseOne{1}Now the serpent
was
more shrewd
\par }{\PP than any
of the wild
animals
that
the {\ND{Lord}}
God
had
made. He
said
to
the woman,
“Is it really
true that
God
said,
‘You must not
eat
from any
tree
of the orchard’?”
\VS{2}The woman
said
to
the serpent,
“We may eat
of the fruit
from the trees
of the orchard;
\VS{3}but concerning the fruit
of the tree
that
is in the middle
of the orchard
God
said,
‘You must not
eat
from
it, and you must not
touch
it, or else
you will die.’ ”
\VS{4}The serpent
said
to
the woman,
“Surely
you will not
die,
\VS{5}for
God
knows
that
when
you eat
from
it your eyes
will open
and you will be like
divine
beings who know
good
and evil.”
\par }{\PP \VS{6}When the woman
saw
that
the tree
produced fruit
that
was good
for
food, was attractive
to the eye,
and was desirable
for making
one wise,
she took
some of its fruit
and ate
it. She also
gave
some
of it to her husband who
was with
her, and he ate it.
\VS{7}Then the eyes
of both
of them opened,
and they knew
they
were naked;
so they sewed
fig
leaves
together and made
coverings for themselves.
\par }{\SH The Judgment Oracles of God at the Fall
\par }{\PP \VS{8}Then
the man
and his wife
heard
the sound
of the {\ND{Lord}}
God
moving
about in the orchard
at the breezy
time of the day,
and they hid
from the
{\ND{Lord}}
God
among
the trees
of the orchard.
\VS{9}But the
{\ND{Lord}}
God
called
to
the man
and said
to him, “Where are you?”
\VS{10}The man replied, “I heard
you moving
about in the orchard,
and I was afraid
because
I was naked,
so I
hid.”
\VS{11}And the
{\ND{Lord}} God said,
“Who
told
you that
you were naked? Did you
eat
from
the tree
that
I commanded
you not
to eat
from?”
\VS{12}The man
said,
“The woman
whom
you gave
me,
she
gave
me some
fruit from the tree
and I ate it.”
\VS{13}So the
{\ND{Lord}}
God
said to the woman,
“What
is this
you have done?” And the woman
replied,
“The serpent
tricked
me, and I ate.”
\par }{\Q \VS{14}The
{\ND{Lord}}
God
said
to
the serpent,
\par }{\Q “Because
you have done
this,
\par }{\Q cursed
are you
above all
the wild beasts
\par }{\Q and all
the living
creatures of the field!
\par }{\Q On
your belly
you will crawl
\par }{\Q and dust
you will eat
all
the days
of your life.
\par }{\Q \VS{15}And I will put
hostility
between
you and the woman
\par }{\Q and between
your offspring
and her offspring;
\par }{\Q her offspring will attack
your head,
\par }{\Q and you
will attack
her offspring’s heel.”
\par }{\Q \VS{16}To
the woman
he said,
\par }{\Q “I will greatly
increase
your labor pains;
\par }{\Q with pain
you will give birth
to children.
\par }{\Q You will want
to control your husband,
\par }{\Q but he will dominate you.”
\par }{\Q \VS{17}But to Adam
he said,
\par }{\Q “Because
you obeyed
your wife
\par }{\Q and ate
from
the tree
about which
I commanded
you,
\par }{\Q ‘You must not
eat
from
it,’
\par }{\Q cursed
is the ground
thanks to you;

\par }{\Q in painful
toil
you will eat
of it all
the days
of your life.
\par }{\Q \VS{18}It will produce
thorns
and thistles
for you,
\par }{\Q but you will eat
the grain
of the field.
\par }{\Q \VS{19}By the sweat
of your brow
you will eat
food
\par }{\Q until
you return
to
the ground,
\par }{\Q for
out
of it
you were taken;
\par }{\Q for
you
are dust,
and to
dust
you will return.”
\par }{\PP \VS{20}The man
named
his wife
Eve,
because
she
was the mother
of all
the living.
\VS{21}The
{\ND{Lord}}
God
made
garments
from skin
for Adam
and his wife,
and clothed them.
\VS{22}And the
{\ND{Lord}}
God
said,
“Now
that the man
has become
like one
of us, knowing
good
and evil,
he must not be allowed
to stretch
out his hand
and take
also
from the tree
of life
and eat,
and live
forever.”
\VS{23}So
the {\ND{Lord}}
God
expelled
him from the orchard
in Eden
to cultivate
the ground
from which
he had been taken.
\VS{24}When he drove
the
man
out, he placed
on the eastern
side of the orchard
in Eden
angelic sentries
who used the flame
of a whirling
sword
to guard
the
way
to the tree
of life.

\par }\Chap{4}{\PP \VerseOne{1}Now the man
had marital relations
with his wife
Eve,
and she became pregnant
and gave birth
to Cain.
Then she said,
“I
have created
a man
just as the
{\ND{Lord}} did!”
\VS{2}Then
she gave birth
to his brother
Abel.
Abel
took care
of the flocks,
while Cain
cultivated
the ground.
\par }{\PP \VS{3}At the designated
time
Cain
brought
some of the fruit
of the ground
for an offering
to the
{\ND{Lord}}.
\VS{4}But Abel
brought
some of the firstborn
of his flock
– even
the fattest
of them.
And the
{\ND{Lord}}
was pleased with Abel
and his offering,
\VS{5}but with Cain
and his offering
he was not
pleased.
So Cain
became very
angry,
and his expression
was downcast.
\par }{\PP \VS{6}Then the
{\ND{Lord}}
said to
Cain,
“Why
are you angry,
and why
is your expression downcast?
\VS{7}Is it not
true
that if
you do what is right,
you will be fine? But if
you do not
do what is right,
sin
is crouching
at the door.
It desires
to
dominate you, but you
must subdue it.”
\par }{\PP \VS{8}Cain
said
to
his brother
Abel,
“Let’s go out to the field.” While
they were in the field,
Cain
attacked
his brother
Abel
and killed him.
\par }{\PP \VS{9}Then the
{\ND{Lord}}
said
to Cain,
“Where
is your brother
Abel?” And he replied,
“I don’t
know! Am
I my brother’s
guardian?”
\VS{10}But the
{\ND{Lord}} said,
“What
have you done? The voice
of your brother’s
blood
is crying out
to me
from
the ground!
\VS{11}So now,
you
are banished
from
the ground,
which
has opened
its mouth to receive
your brother’s
blood
from your hand.
\VS{12}When
you try to cultivate
the
\par }{\PP ground
it will
no
longer
yield
its best for you. You will be
a homeless wanderer
on the earth.”
\VS{13}Then Cain
said
to
the {\ND{Lord}}, “My punishment
is too great
to endure!
\VS{14}Look! You are driving
me off the land
today,
and I must hide
from your presence.
I will be
a homeless wanderer
on the earth;
whoever
finds
me will
kill me.”
\VS{15}But the
{\ND{Lord}}
said
to him, “All
right then, if
anyone kills
Cain,
Cain will be avenged seven times as much.” Then the
{\ND{Lord}} put a special mark on Cain so that no one who found him would strike him down.
\VS{16}So Cain
went out
from the presence
of the {\ND{Lord}}
and lived
in the land
of Nod,
east
of Eden.
\par }{\SH The Beginning of Civilization
\par }{\PP \VS{17}Cain
had marital relations
with his wife,
and she became pregnant
and gave birth
to Enoch.
Cain was
building
a city,
and he named
the city
after his
son
Enoch.
\VS{18}To Enoch
was born
Irad,
and Irad
was the father
of Mehujael.
Mehujael
was the father
of Methushael,
and Methushael
was the father
of Lamech.
\par }{\PP \VS{19}Lamech
took
two
wives
for himself; the name
of the first
was Adah,
and the name
of the second
was Zillah.
\VS{20}Adah
gave birth
to Jabal;
he
was
the first
of those who live in
tents
and keep livestock.
\VS{21}The name
of his brother
was Jubal;
he
was
the first
of all
who play
the harp
and the flute.
\VS{22}Now Zillah
also
gave birth
to Tubal-Cain,
who heated metal and shaped
all
kinds of tools
made of bronze
and iron.
The sister
of Tubal-Cain
was Naamah.
\par }{\PP \VS{23}Lamech
said
to his wives,
\par }{\Q “Adah
and Zillah! Listen
to me!
\par }{\Q You wives
of Lamech,
hear
my words!
\par }{\Q I have killed
a man
for
wounding
me,
\par }{\Q a young man
for hurting me.
\par }{\Q \VS{24}If
Cain
is to be avenged
seven times
as much,
\par }{\Q then Lamech
seventy-seven times!”
\par }{\PP \VS{25}And Adam
had marital relations
with his wife
again,
and she gave birth
to a son.
She
named
him Seth,
saying, “God
has given me another
child
in place
of Abel
because
Cain
killed him.”
\VS{26}And a son
was also
born
to Seth,
whom he named
Enosh.
At that time
people began
to worship
the {\ND{Lord}}.

\par }\Chap{5}{\PP \VerseOne{1}This
is the record
of the family line
of Adam.
\par }{\PP When
God
created
humankind,
he made
them in the likeness
of God.
\VS{2}He created
them male
and female;
when
they were created,
he blessed
them and named
them “humankind.”
\par }{\PP \VS{3}When
Adam
had lived
130
years
he fathered
a son
in his own likeness,
according to his image,
and he named
him Seth.
\VS{4}The length of time
Adam
lived after
he became the father
of Seth
was 800
years;
during this time he had other
sons
and daughters.
\VS{5}The entire
lifetime
of Adam
was 930
years,
and then he died.
\par }{\PP \VS{6}When
Seth
had lived
105
years,
he became the father
of Enosh.
\VS{7}Seth
lived
807
years
after
he became the father
of Enosh,
and he had other
sons
and daughters.
\VS{8}The entire
lifetime
of Seth
was 912
years,
and then he died.
\par }{\PP \VS{9}When
Enosh
had lived
90
years,
he became the father
of Kenan.
\VS{10}Enosh
lived
815
years
after
he became the father
of Kenan,
and he had other
sons
and daughters.
\VS{11}The entire
lifetime
of Enosh
was 905
years,
and then he died.
\par }{\PP \VS{12}When
Kenan
had lived
70
years,
he became the father
of Mahalalel.
\VS{13}Kenan
lived
840
years
after
he became the father
of Mahalalel,
and he had
other sons
and daughters.
\VS{14}The entire
lifetime
of Kenan
was 910
years,
and then he died.
\par }{\PP \VS{15}When
Mahalalel
had lived
65
years,
he became the father
of Jared.
\VS{16}Mahalalel
lived
830
years
after
he became the father
of Jared,
and he had other
sons
and daughters.
\VS{17}The entire
lifetime
of Mahalalel
was
895
years,
and then he died.
\par }{\PP \VS{18}When
Jared
had lived
162
years,
he became the father
of Enoch.
\VS{19}Jared
lived
800
years
after
he became the father
of Enoch,
and he had other
sons
and daughters.
\VS{20}The entire
lifetime
of Jared
was 962
years,
and then he died.
\par }{\PP \VS{21}When
Enoch
had lived
65
years,
he became the father
of Methuselah.
\VS{22}After
he became the father
of Methuselah,
Enoch
walked
with
God
for 300
years,
and he had other
sons
and daughters.
\VS{23}The entire
lifetime
of Enoch
was 365
years.
\VS{24}Enoch
walked
with
God,
and then he disappeared
because
God
took him away.
\par }{\PP \VS{25}When
Methuselah
had lived 187
years,
he became the father
of Lamech.
\VS{26}Methuselah
lived 782
years
after
he became the father
of Lamech,
and he had other
sons
and daughters.
\VS{27}The entire
lifetime
of Methuselah
was 969
years,
and then he died.
\par }{\PP \VS{28}When
Lamech
had lived
182
years,
he had a son.
\VS{29}He named
him Noah,
saying,
“This
one will bring us comfort
from our labor
and from
the painful toil
of our hands
because of the ground
that
the {\ND{Lord}}
has
cursed.”
\VS{30}Lamech
lived
595
years
after
he became the father
of Noah,
and he had other
sons
and daughters.
\VS{31}The entire
lifetime
of Lamech
was 777
years,
and then he died.
\par }{\PP \VS{32}After
Noah
was 500
years
old,
he
became the father
of Shem,
Ham,
and Japheth.

\par }\Chap{6}{\PP \VerseOne{1}When
humankind
began
to multiply
on
the face
of the earth,
and daughters
were born to them,
\VS{2}the sons
of God
saw
that the daughters
of humankind
were beautiful.
Thus
they took
wives
for themselves from any
they chose.
\VS{3}So the
{\ND{Lord}}
said,
“My spirit
will not
remain
in humankind
indefinitely,
since
they
are
mortal.
They will remain for
120
more years.”
\par }{\PP \VS{4}The Nephilim
were
on the earth
in those
days
(and also
after
this) when the sons
of God
were having sexual relations
with
the daughters
of humankind,
who gave birth
to their children. They
were the mighty heroes
of old,
the famous
men.
\par }{\PP \VS{5}But the
{\ND{Lord}}
saw
that
the wickedness
of humankind
had become great
on the earth.
Every
inclination
of the thoughts
of their minds
was only
evil
all
the time.
\VS{6}The
{\ND{Lord}}
regretted
that
he had made
humankind
on the earth,
and he was highly
offended.
\VS{7}So the
{\ND{Lord}}
said,
“I will wipe
humankind,
whom
I have created,
from the face
of the earth
– everything from humankind
to
animals,
including
creatures that move on the ground
and birds
of the air,
for
I regret
that
I have made them.”
\par }{\PP \VS{8}But Noah
found
favor
in the sight
of the
{\ND{Lord}}.
\par }{\SH The Judgment of the Flood
\par }{\PP \VS{9}This
is the account
of Noah.
\par }{\PP Noah
was a godly
man;
he was blameless
\par }{\PP among
his contemporaries. He
walked
with
God.
\VS{10}Noah
had three
sons: Shem,
Ham,
and Japheth.
\par }{\PP \VS{11}The earth
was ruined
in the sight
of God;
the earth
was filled
with violence.
\VS{12}God
saw
the earth,
and indeed
it was ruined,
for
all
living creatures
on
the earth
were sinful.
\VS{13}So God
said
to Noah,
“I have decided
that all
living creatures
must die,
for
the earth
is filled
with violence
because of them. Now
I am about to destroy
them and the earth.
\VS{14}Make
for yourself an ark
of cypress
wood.
Make
rooms in the ark,
and cover it with
pitch
inside
and out.
\VS{15}This
is how
you should make
it: The ark
is to be 450 feet
long,
75 feet
wide,
and 45 feet
high.
\VS{16}Make
a roof for the ark
and finish
it, leaving 18 inches
from the top.
Put
a door
in the side
of the ark,
and make
lower,
middle, and upper decks.
\VS{17}I
am about
to bring
floodwaters
on
the earth
to destroy
from under
the sky
all
the living creatures
that have
the breath
of life
in them. Everything
that
is on the earth
will die,
\VS{18}but I will confirm
my covenant
with
you. You will enter
the ark
– you,
your sons,
your wife,
and your sons’
wives
with you.
\VS{19}You must bring
into
the ark
two
of every
kind of living creature
from all
flesh,
male
and female,
to keep them alive
with you.
\VS{20}Of the birds
after their kinds,
and of the cattle
after their kinds,
and of every
creeping thing
of the ground
after its kind,
two
of every
kind will come
to
you so you can keep them alive.
\VS{21}And you
must take
for yourself every
kind of food
that
is eaten,
and gather
it together.
It will be
food for you and for them.
\par }{\PP \VS{22}And Noah
did
all
that
God
commanded him – he did indeed.

\par }\Chap{7}{\PP \VerseOne{1}The
{\ND{Lord}}
said
to Noah,
“Come
into the ark,
you
and all
your household,
for
I consider
you godly
among this
generation.
\VS{2}You must take
with you seven
of every
kind of clean
animal,
the male
and its mate,
two
of every kind of unclean
animal,
the male
and its mate,
\VS{3}and also
seven
of every kind of bird
in the sky,
male
and female,
to preserve
their offspring
on
the face
of the earth.
\VS{4}For
in seven
days
I
will cause
it to rain
on
the earth
for forty
days
and forty
nights,
and I will wipe
from the
face
of the ground
every
living thing
that
I have
made.”
\par }{\PP \VS{5}And Noah
did
all
that
the {\ND{Lord}}
commanded him.
\par }{\PP \VS{6}Noah
was 600
years
old
when the floodwaters
engulfed
the earth.
\VS{7}Noah
entered
the ark
along with
his sons,
his wife,
and his sons’
wives
because of the floodwaters.
\VS{8}Pairs of clean
animals,
of unclean
animals,
of birds,
and of everything
that
creeps
along
the ground,
\VS{9}male
and female,
came
into
the ark
to
Noah,
just
as God
had commanded
him.
\VS{10}And after
seven
days
the floodwaters
engulfed the earth.
\par }{\PP \VS{11}In the six
hundredth
year
of Noah’s
life,
in the second
month,
on the seventeenth
day
of the month
– on that day
all
the fountains
of the great
deep
burst
open and the floodgates
of the heavens
were opened.
\VS{12}And the rain
fell on
the earth
forty
days
and forty
nights.
\par }{\PP \VS{13}On that
very
day
Noah
entered
the ark,
accompanied by
his
sons
Shem,
Ham,
and Japheth,
along with his
wife
and his sons’
three
wives.
\VS{14}They
entered, along with every
living creature
after its kind,
every
animal
after its kind,
every
creeping thing
that creeps
on
the earth
after its kind,
and every
bird
after its kind,
everything
with wings.
\VS{15}Pairs
of all
creatures
that
have the breath
of life
came
into
the ark
to
Noah.
\VS{16}Those that entered
were male
and female,
just
as God
commanded
him. Then the
{\ND{Lord}}
shut him in.
\par }{\PP \VS{17}The flood
engulfed the earth
for forty
days.
As the waters
increased,
they lifted
the ark
and raised
it above
the earth.
\VS{18}The waters
completely
overwhelmed
the earth,
and the ark
floated on
the surface
of the waters.
\VS{19}The waters
completely inundated
the earth
so that even all
the high
mountains
under
the entire
sky
were covered.
\VS{20}The waters
rose
more than twenty feet
above
the mountains.
\VS{21}And all
living things
that moved
on
the earth
died,
including the birds,
domestic animals,
wild animals,
all
the creatures
that swarm
over
the earth,
and all
humankind.
\VS{22}Everything
on dry land
that had
the breath
of life
in its nostrils
died.
\VS{23}So the
{\ND{Lord}} destroyed
every
living thing
that
was on
the surface
of the ground,
including people,
animals,
creatures that creep along the ground,
and birds
of the sky.
They were wiped off
the earth.
Only
Noah
and those who
were with
him in the ark
survived.
\VS{24}The waters
prevailed
over
the earth
for 150
days.

\par }\Chap{8}{\PP \VerseOne{1}But God
remembered
Noah
and all
the wild animals
and domestic animals
that
were with
him in the ark.
God
caused a wind
to blow
over
the earth
and the waters
receded.
\VS{2}The fountains
of the deep
and the floodgates
of heaven
were closed,
and the rain
stopped falling
from
the sky.
\VS{3}The waters
kept receding
steadily
from the earth,
so
that they had gone down by the end
of the 150
days.
\VS{4}On
the seventeenth
day
of the seventh
month,
the ark
came to rest
on
one of the mountains
of Ararat.
\VS{5}The waters
kept on
receding
until
the tenth
month.
On the first
day of the tenth
month,
the tops
of the mountains
became visible.
\par }{\PP \VS{6}At the end
of forty
days,
Noah
opened
the window
he had
made
in the ark
\VS{7}and sent
out a raven;
it kept flying
back
and forth
until
the waters
had dried
up on
the earth.
\par }{\PP \VS{8}Then Noah sent
out a dove
to see
if
the waters
had receded
from the surface
of the ground.
\VS{9}The dove
could not
find
a resting place
for its feet
because
water
still covered the surface
of the entire
earth,
and so
it returned
to
Noah in the ark.
He stretched out
his hand,
took
the dove, and brought
it back into
the
ark.
\VS{10}He waited
seven
more
days
and then
sent
out the dove
again from
the ark.
\VS{11}When the dove
returned
to him
in the evening,
there was
a freshly
plucked
olive
leaf
in its beak! Noah
knew
that
the waters
had receded
from
the earth.
\VS{12}He waited
another
seven
days
and sent
the dove
out again,
but it did not
return
to him this time.
\par }{\PP \VS{13}In Noah’s six
hundred
and first
year,
in the first
day of the first
month,
the waters
had dried
up from
the earth,
and Noah
removed
the covering
from the ark
and saw
that
the surface
of the ground
was dry.
\VS{14}And by the twenty-seventh
day
of the second
month
the earth
was dry.
\par }{\PP \VS{15}Then
God
spoke
to Noah
and said,
\VS{16}“Come out
of the ark,
you,
your wife,
your sons,
and your sons’
wives
with you.
\VS{17}Bring out
with
you all
the living creatures
that
are with
you. Bring out every
living thing,
including the birds,
animals,
and every
creeping thing
that creeps
on
the earth.
Let them increase
and be fruitful
and multiply
on
the earth!”
\par }{\PP \VS{18}Noah
went out
along with
his sons,
his wife,
and his sons’
wives.
\VS{19}Every
living creature,
every
creeping thing,
every
bird,
and everything
that moves
on
the earth
went out
of
the ark
in their groups.
\par }{\PP \VS{20}Noah
built
an altar
to the
{\ND{Lord}}. He then took
some of every
kind of clean
animal
and clean
bird
and offered
burnt offerings
on the altar.
\VS{21}And the
{\ND{Lord}}
smelled
the soothing
aroma
and said
to
himself, “I will never
again
curse
the ground
because
of humankind,
even though
the inclination
of their minds
is evil
from childhood
on. I will never
again
destroy
everything
that lives,
as I have just
done.
\par }{\Q \VS{22}“While
the earth
continues to exist,

\par }{\Q planting
time
and harvest,
\par }{\Q cold
and heat,
\par }{\Q summer
and winter,
\par }{\Q and day
and night
will not
cease.”

\par }\Chap{9}{\PP \VerseOne{1}Then God
blessed
Noah
and his sons
and said
to them, “Be fruitful
and multiply
and fill
the earth.
\VS{2}Every
living
creature of the earth
and every
bird
of the sky
will be
terrified
of you. Everything
that
creeps
on the ground
and all
the fish
of the sea
are under your authority.
\VS{3}You may eat any
moving thing
that
lives.
As I gave
you the green
plants,
I now give
you everything.
\par }{\PP \VS{4}But
you must not
eat
meat
with its life
(that is, its blood) in it.
\VS{5}For your lifeblood
I will surely
exact
punishment,
from every
living creature
I will exact
punishment.
From each person
I will exact
punishment
for the life
of the individual
since the man
was his relative.
\par }{\Q \VS{6}“Whoever sheds
human
blood,
\par }{\Q by other humans
\par }{\Q must his blood
be shed;
\par }{\Q for
in God’s image
\par }{\Q God
has made
humankind.”
\par }{\PP \VS{7}But as for you,
be fruitful
and multiply;
increase abundantly
on the earth
and multiply on it.”
\par }{\PP \VS{8}God
said
to
Noah
and his sons,
\VS{9}“Look! I
now confirm
my covenant
with
you and your descendants
after you
\VS{10}and with
every
living creature
that
is with
you, including the birds,
the domestic animals,
and every
living creature
of the earth
with
you, all
those that
came out
of the ark with you – every living creature of the earth.
\VS{11}I confirm
my covenant
with
you: Never
again
will all
living
things be
wiped
out by the waters
of a flood;
never
again
will a flood
destroy
the earth.”
\par }{\PP \VS{12}And God
said,
“This
is the guarantee
of the covenant
I am
making with you and every
living creature
with
you, a covenant for all subsequent
generations:
\VS{13}I will
place my rainbow
in the clouds,
and it will become
a guarantee
of the covenant
between
me and the earth.
\VS{14}Whenever
I bring clouds
over
the earth
and the rainbow
appears
in the clouds,
\VS{15}then I will remember
my covenant
with you and with all
living
creatures
of all
kinds.
Never
again
will the waters
become a flood
and destroy
all
living things.
\VS{16}When
the rainbow
is in the clouds,
I
will notice
it and remember
the perpetual
covenant
between
God
and all
living
creatures
of all
kinds
that
are on
the earth.”
\par }{\PP \VS{17}So God
said
to
Noah,
“This
is the guarantee
of the covenant
that
I am
confirming
between
me and all
living
things that
are on
the earth.”
\par }{\SH The Curse of Canaan
\par }{\PP \VS{18}The sons
of Noah
who came out
of the ark
were Shem,
Ham,
and Japheth.
(Now Ham
was
the father
of Canaan.)
\VS{19}These
were the sons
of Noah,
and from them the whole
earth
was populated.
\par }{\PP \VS{20}Noah,
a man
of the soil,
began
to plant
a vineyard.
\VS{21}When he drank
some
of the wine,
he got drunk
and uncovered
himself inside
his tent.
\VS{22}Ham,
the father
of Canaan,
saw
his father’s
nakedness
and told
his two
brothers
who were outside.
\VS{23}Shem
and Japheth
took
the garment
and placed
it on
their shoulders.
Then
they walked
in backwards
and covered
up their father’s
nakedness.
Their faces
were turned the other way
so they did not
see
their father’s
nakedness.
\par }{\PP \VS{24}When Noah
awoke
from his drunken stupor
he learned what
his youngest
son
had
done to him.
\VS{25}So he said,
\par }{\Q “Cursed
be Canaan!

\par }{\Q The lowest of
slaves
\par }{\Q he will be
to his brothers.”
\par }{\PP \VS{26}He also said,
\par }{\Q “Worthy
of praise is the
{\ND{Lord}}, the God
of Shem!
\par }{\Q May Canaan
be
the slave of Shem!
\par }{\Q \VS{27}May God
enlarge
Japheth’s
territory and numbers!

\par }{\Q May he live
in the tents
of Shem
\par }{\Q and may Canaan
be his slave!”
\par }{\PP \VS{28}After
the flood
Noah
lived
350
years.
\VS{29}The entire
lifetime
of Noah
was 950
years,
and then he died.

\par }\Chap{10}{\PP \VerseOne{1}This
is the account
of Noah’s
sons
Shem,
Ham,
and Japheth.
Sons
were born
to them after
the flood.
\par }{\PP \VS{2}The sons
of Japheth
were Gomer,
Magog,
Madai,
Javan,
Tubal,
Meshech,
and Tiras.
\VS{3}The sons
of Gomer
were Askenaz,
Riphath,
and Togarmah.
\VS{4}The sons
of Javan
were Elishah,
Tarshish,
the Kittim,
and the Dodanim.
\VS{5}From these
the coastlands
of the nations
were separated
into their lands,
every one
according to its language,
according to their families,
by their nations.
\par }{\PP \VS{6}The sons
of Ham
were Cush,
Mizraim,
Put,
and Canaan.
\VS{7}The sons
of Cush
were Seba,
Havilah,
Sabtah,
Raamah,
and Sabteca.
The sons of
Raamah
were Sheba
and Dedan.
\par }{\PP \VS{8}Cush
was the father
of Nimrod;
he began
to be a valiant warrior
on the earth.
\VS{9}He was
a mighty
hunter
before
the {\ND{Lord}}. (That is why
it is said,
“Like Nimrod,
a mighty
hunter
before
the {\ND{Lord}}.”)
\VS{10}The primary
regions of his kingdom
were Babel,
Erech,
Akkad,
and Calneh
in the land
of Shinar.
\VS{11}From
that land
he
went
to Assyria,
where he built
Nineveh,
Rehoboth-Ir,
Calah,
\VS{12}and Resen,
which is between
Nineveh
and the great
city
Calah.
\par }{\PP \VS{13}Mizraim
was the father
of the Ludites,
Anamites,
Lehabites,
Naphtuhites,
\VS{14}Pathrusites,
Casluhites
(from whom
the Philistines
came), and Caphtorites.
\par }{\PP \VS{15}Canaan
was the father
of Sidon
his firstborn,
Heth,
\VS{16}the Jebusites,
Amorites,
Girgashites,
\VS{17}Hivites,
Arkites,
Sinites,
\VS{18}Arvadites,
Zemarites,
and Hamathites.
Eventually
the families
of the Canaanites
were scattered
\VS{19}and the borders
of Canaan
extended
from Sidon
all the way to
Gerar
as far as
Gaza,
and all the way to
Sodom,
Gomorrah,
Admah,
and Zeboiim,
as far as
Lasha.
\VS{20}These
are the sons
of Ham,
according to their families,
according to their languages,
by their lands,
and by their nations.
\par }{\PP \VS{21}And sons were also
born
to Shem
(the older
brother
of Japheth), the father
of all
the sons
of Eber.
\par }{\PP \VS{22}The sons
of Shem
were Elam,
Asshur,
Arphaxad,
Lud,
and Aram.
\VS{23}The sons
of Aram
were Uz,
Hul,
Gether,
and Mash.
\VS{24}Arphaxad
was the father
of Shelah,
and Shelah
was the father
of Eber.
\VS{25}Two
sons
were born
to Eber: One
was named
Peleg
because
in his days
the earth
was divided,
and his brother’s
name
was Joktan.
\VS{26}Joktan
was the father
of Almodad,
Sheleph,
Hazarmaveth,
Jerah,
\VS{27}Hadoram,
Uzal,
Diklah,
\VS{28}Obal,
Abimael,
Sheba,
\VS{29}Ophir,
Havilah,
and Jobab.
All
these
were sons
of Joktan.
\VS{30}Their dwelling
place was
from Mesha
all the way to
Sephar
in the eastern
hills.
\VS{31}These
are the sons
of Shem
according to their families,
according to their languages,
by their lands,
and according to their nations.
\par }{\PP \VS{32}These
are the families
of the sons
of Noah,
according to their genealogies,
by their nations,
and from these
the nations
spread
over the earth
after
the flood.

\par }\Chap{11}{\PP \VerseOne{1}The whole
earth
had a common language
and a common vocabulary.
\VS{2}When
the people
moved eastward,
they found
a plain
in Shinar
and settled
there.
\VS{3}Then they said
to
one
another, “Come, let’s
make
bricks
and bake
them thoroughly.”
(They had brick
instead of stone
and tar
instead of mortar.)
\VS{4}Then they said,
“Come, let’s
build
ourselves a city
and a tower
with its top
in the heavens
so that we may make
a name
for ourselves. Otherwise
we will be scattered
across the face
of the entire
earth.”
\par }{\PP \VS{5}But the
{\ND{Lord}}
came down
to see
the city
and the tower
that
the people
had
started building.
\VS{6}And the
{\ND{Lord}}
said,
“If
as one
people
all
sharing
a common
language
they have begun
to do
this,
then
nothing they
plan to do will be beyond them.
\VS{7}Come,
let’s go down
and confuse
their language
so they won’t
be able to understand
each other.”
\par }{\PP \VS{8}So the
{\ND{Lord}}
scattered
them from there
across the face
of the entire
earth,
and they stopped
building
the city.
\VS{9}That is why
its name
was called
Babel –
because
there
the {\ND{Lord}}
confused
the language
of the entire
world,
and from there
the {\ND{Lord}}
scattered
them across the face
of the entire
earth.
\par }{\SH The Genealogy of Shem
\par }{\PP \VS{10}This
is the account
of Shem.
\par }{\PP Shem
was
100
old
when he became the father
of Arphaxad,
two years
after
the flood.
\VS{11}And after
becoming the father
of Arphaxad,
Shem
lived
500
years
and had other
sons
and daughters.
\par }{\PP \VS{12}When Arphaxad
had lived 35
years,
he became the father
of Shelah.
\VS{13}And after
he became the father
of Shelah,
Arphaxad
lived
403
years
and had other
sons
and daughters.
\par }{\PP \VS{14}When Shelah
had lived
30
years,
he became the father
of Eber.
\VS{15}And after
he became the father
of Eber,
Shelah
lived
403
years
and had other
sons
and daughters.
\par }{\PP \VS{16}When
Eber
had lived
34
years,
he became the father
of Peleg.
\VS{17}And after
he became the father
of Peleg,
Eber
lived
430
years
and had other
sons
and daughters.
\par }{\PP \VS{18}When
Peleg
had lived
30
years,
he became the father
of Reu.
\VS{19}And after
he became the father
of Reu,
Peleg
lived
209
years
and had other sons
and daughters.
\par }{\PP \VS{20}When
Reu
had lived
32
years,
he became the father
of Serug.
\VS{21}And after
he became the father
of Serug,
Reu
lived
207
years
and had other sons
and daughters.
\par }{\PP \VS{22}When
Serug
had lived
30
years,
he became the father
of Nahor.
\VS{23}And after
he became the father
of Nahor,
Serug
lived
200
years
and had other
sons
and daughters.
\par }{\PP \VS{24}When
Nahor
had lived
29
years,
he became the father
of Terah.
\VS{25}And after
he became the father
of Terah,
Nahor
lived
119
years
and had other sons
and daughters.
\par }{\PP \VS{26}When
Terah
had lived
70
years,
he became the father
of Abram,
Nahor,
and Haran.
\par }{\SH The Record of Terah
\par }{\PP \VS{27}This
is the account
of Terah.
\par }{\PP Terah
became the father
of Abram,
Nahor,
and Haran.
And Haran
became the father
of Lot.
\VS{28}Haran
died
in the land
of his birth,
in Ur
of the Chaldeans,
while his father
Terah was still alive.
\VS{29}And Abram
and Nahor
took
wives
for themselves. The name
of Abram’s
wife
was Sarai,
and the name
of Nahor’s
wife
was Milcah;
she was the daughter
of Haran,
the father
of both Milcah
and Iscah.
\VS{30}But Sarai
was barren;
she had no
children.
\par }{\PP \VS{31}Terah
took
his son
Abram,
his grandson
Lot
(the son
of Haran), and his daughter-in-law
Sarai,
his son
Abram’s
wife,
and with them
he set out
from Ur
of the Chaldeans
to go
to Canaan.
When they came
to
Haran,
they settled
there.
\VS{32}The lifetime
of Terah
was 205
years,
and he
died
in Haran.

\par }\Chap{12}{\PP \par }{\Q \VerseOne{1}Now the
{\ND{Lord}}
said
to
Abram,
\par }{\Q “Go out
from your country,
your relatives,
and your father’s
household
\par }{\Q to
the land
that
I will show you.
\par }{\Q \VS{2}Then I will make
you into a great nation,
and I will bless
you,

\par }{\Q and I will make
your name
great,
\par }{\Q so that you will exemplify divine blessing.
\par }{\Q \VS{3}I will bless
those who bless
you,

\par }{\Q but the one who treats
you lightly
I must curse,
\par }{\Q and all
the families
of the earth
will bless one another by your name.”
\par }{\PP \VS{4}So
Abram
left, just
as the
{\ND{Lord}}
had told
him
to do, and Lot
went
with
him. (Now Abram
was 75
years
old
when
he departed
from Haran.)
\VS{5}And Abram
took
his wife
Sarai,
his nephew
Lot,
and all
the possessions
they had
accumulated
and the people they had
acquired
in Haran,
and they left
for the land
of Canaan.
They entered
the land
of Canaan.
\par }{\PP \VS{6}Abram
traveled
through the land
as far
as the oak tree
of Moreh
at Shechem.
(At that time
the Canaanites
were in the land.)
\VS{7}The
{\ND{Lord}}
appeared
to
Abram
and said,
“To your descendants
I will give
this
land.”
So Abram built
an altar
there
to the
{\ND{Lord}},
who had appeared
to him.
\par }{\PP \VS{8}Then he moved
from there
to the hill country
east
of Bethel
and pitched
his tent,
with Bethel
on the west
and Ai
on the east.
There
he built
an altar
to the
{\ND{Lord}}
and worshiped
the {\ND{Lord}}.
\VS{9}Abram
continually journeyed
by stages
down to the Negev.
\par }{\SH The Promised Blessing Jeopardized
\par }{\PP \VS{10}There was
a famine
in the land,
so Abram
went down
to Egypt
to stay
for a
while because
the famine
was severe.
\VS{11}As he approached
Egypt,
he said
to
his wife
Sarai,
“Look,
I know
that
you
are a beautiful
woman.
\VS{12}When
the Egyptians
see
you they will say,
‘This
is his wife.’
Then they will kill
me but will keep you alive.
\VS{13}So
tell
them you
are my sister
so that it may
go well
for me because
of you and my life
will be spared
on account of you.”
\par }{\PP \VS{14}When
Abram
entered
Egypt,
the Egyptians
saw
that the woman
was very
beautiful.
\VS{15}When
Pharaoh’s
officials
saw her, they praised
her to
Pharaoh.
So Abram’s
wife
was taken
into the household
of Pharaoh,
\VS{16}and he did treat
Abram
well
on account of
her. Abram received
sheep
and cattle,
male donkeys,
male servants,
female servants,
female donkeys,
and camels.
\par }{\PP \VS{17}But the
{\ND{Lord}}
struck
Pharaoh
and his household
with severe
diseases
because of Sarai,
Abram’s
wife.
\VS{18}So Pharaoh
summoned
Abram
and said,
“What
is this
you have done
to me? Why
didn’t
you tell
me that
she was
your wife?
\VS{19}Why
did you say,
‘She is my sister,’
so that I took
her to be my wife? Here
is your wife! Take
her and go!”
\VS{20}Pharaoh
gave his men
orders
about Abram, and so they expelled
him, along
with his wife
and all
his possessions.

\par }\Chap{13}{\PP \VerseOne{1}So Abram
went up
from Egypt
into the Negev.
He took his wife
and all
his possessions
with
him, as well as Lot.
\VS{2}(Now Abram
was
very
wealthy
in livestock,
silver,
and gold.)
\par }{\PP \VS{3}And he journeyed
from place to place from the Negev
as far
as Bethel.
He returned to
the place
where
he had
pitched his tent
at the beginning,
between
Bethel
and Ai.
\VS{4}This was the place
where
he had first
built
the altar,
and there
Abram
worshiped
the {\ND{Lord}}.
\par }{\PP \VS{5}Now
Lot,
who was traveling
with
Abram,
also
had flocks,
herds,
and tents.
\VS{6}But the land
could not support
them while they were living
side by side.
Because
their possessions
were so great,
they were not
able
to live
alongside one another.
\VS{7}So there were
quarrels
between
Abram’s
herdsmen
and Lot’s
herdsmen.
(Now the Canaanites
and the Perizzites
were living
in the land
at that time.)
\par }{\PP \VS{8}Abram
said
to
Lot,
“Let there
be
no
quarreling
between
me and you, and between
my herdsmen
and your herdsmen,
for
we
are close relatives.
\VS{9}Is not
the whole
land
before
you? Separate
yourself
now
from
me. If
you go to the
left,
then I’ll
go to the right, but if
you go to the right,
then I’ll go to the left.”
\par }{\PP \VS{10}Lot
looked up
and saw
the whole
region
of the Jordan.
He noticed that
all
of it was well-watered
(before
the {\ND{Lord}}
obliterated
Sodom
and Gomorrah) like the garden
of the {\ND{Lord}}, like the land
of Egypt,
all the way
to Zoar.
\VS{11}Lot
chose
for himself the whole
region
of the Jordan
and traveled
toward the east.
\par }{\PP So the relatives
separated
from each other.
\VS{12}Abram
settled
in the land
of Canaan,
but Lot
settled
among the cities
of
the Jordan plain
and pitched
his tents next to Sodom.
\VS{13}(Now
the people
of Sodom
were extremely
wicked
rebels
against the
{\ND{Lord}}.)
\par }{\PP \VS{14}After
Lot
had departed,
the {\ND{Lord}}
said
to
Abram, “Look
from
the place
where
you
stand to the north,
south,
east,
and west.
\VS{15}I will give
all
the land
that
you
see
to you and your descendants
forever.
\VS{16}And I will make
your descendants
like the dust
of the earth,
so that
if
anyone is able
to count
the
dust
of the earth,
then your descendants
also
can be counted.
\VS{17}Get up
and walk
throughout
the land,
for
I will give it to you.”
\par }{\PP \VS{18}So Abram
moved his tents
and went
to live
by the oaks
of Mamre
in Hebron,
and he built
an altar
to the
{\ND{Lord}}
there.

\par }\Chap{14}{\PP \VerseOne{1}At that time
Amraphel
king
of Shinar,
Arioch
king
of Ellasar,
Kedorlaomer
king
of Elam,
and Tidal
king
of nations
\VS{2}went to war
against Bera
king
of Sodom,
Birsha
king
of Gomorrah,
Shinab
king
of Admah,
Shemeber
king
of Zeboiim,
and the king
of Bela
(that is,
Zoar).
\VS{3}These
last five kings joined
forces
in the Valley
of Siddim
(that is,
the Salt
Sea).
\VS{4}For twelve
years
they had served
Kedorlaomer,
but in the thirteenth
year
they rebelled.
\VS{5}In the fourteenth
year,
Kedorlaomer
and the kings
who
were his allies
came
and defeated
the Rephaites
in Ashteroth Karnaim,
the Zuzites
in Ham,
the Emites
in Shaveh Kiriathaim,
\VS{6}and the Horites
in their hill country
of Seir,
as far
as El Paran,
which
is near the desert.
\VS{7}Then
they attacked
En Mishpat
(that
is, Kadesh) again, and they conquered
all
the territory
of the Amalekites,
as well
as the Amorites
who were living
in Hazazon Tamar.
\par }{\PP \VS{8}Then
the king
of Sodom,
the king
of Gomorrah,
the king
of Admah,
the king
of Zeboiim,
and the king
of Bela
(that is,
Zoar) went out and prepared
for battle.
In the Valley
of Siddim they met
\VS{9}Kedorlaomer
king
of Elam,
Tidal
king
of nations,
Amraphel
king
of Shinar,
and Arioch
king
of Ellasar.
Four
kings
fought
against five.
\VS{10}Now the Valley
of Siddim
was full of tar
pits.
When the kings
of Sodom
and Gomorrah
fled,
they fell
into them, but some
survivors
fled
to the hills.
\VS{11}The four victorious kings took
all
the possessions
and food
of Sodom
and Gomorrah
and left.
\VS{12}They also took
Abram’s
nephew
Lot
and his possessions
when they left,
for Lot
was living
in Sodom.
\par }{\PP \VS{13}A fugitive
came
and told
Abram
the Hebrew.
Now Abram was living
by the oaks
of Mamre
the Amorite,
the brother
of Eshcol
and Aner.
(All these
were allied
by treaty
with Abram.)
\VS{14}When Abram
heard
that
his nephew
had been taken captive,
he mobilized
his 318
trained
men who had been born
in his household,
and he pursued
the invaders as far as
Dan.
\VS{15}Then, during the night,
Abram divided
his forces
against them and defeated
them. He chased
them as far
as Hobah,
which
is north
of Damascus.
\VS{16}He retrieved
all
the stolen property.
He also
brought back
his nephew
Lot
and his possessions,
as well as
the women
and the rest of the people.
\par }{\PP \VS{17}After
Abram returned from
defeating
Kedorlaomer
and the kings
who
were with
him, the king
of Sodom
went out to
meet
Abram in
the Valley
of Shaveh
(known as the King’s
Valley).
\VS{18}Melchizedek
king
of Salem
brought out
bread
and wine.
(Now he was
the priest
of the Most
High God.)
\VS{19}He blessed
Abram, saying,
\par }{\Q “Blessed
be Abram
by the Most
High God,
\par }{\Q Creator
of heaven
and earth.
\par }{\Q \VS{20}Worthy
of praise is the Most
High God,
\par }{\Q who
delivered your enemies
into your hand.”
\par }{\PP Abram gave
Melchizedek
a tenth
of everything.
\par }{\PP \VS{21}Then the king
of Sodom
said to
Abram,
“Give
me
the people
and take
the possessions for yourself.”
\VS{22}But Abram
replied
to
the king
of Sodom,
“I raise
my hand
to
the {\ND{Lord}}, the Most
High God,
Creator
of heaven
and earth, and vow
\VS{23}that I will take
nothing belonging to you, not even a thread
or the strap
of a sandal.
That way you can
never say,
‘It is I
who made Abram
rich.’
\VS{24}I will take nothing except
compensation for what
the young men
have eaten.
As for the share
of the men
who
went
with me – Aner, Eshcol, and Mamre – let them take their share.”

\par }\Chap{15}{\PP \VerseOne{1}After
these
things
the word
of the {\ND{Lord}}
came
to
Abram
in a vision: “Fear
not,
Abram! I
am your shield
and the one who will reward
you in great
abundance.”
\par }{\PP \VS{2}But Abram
said,
“O sovereign

{\ND{Lord}}, what
will you give
me since I
continue
to be childless,
and my heir
is
Eliezer
of Damascus?”
\VS{3}Abram
added, “Since
you have not
given
me a descendant,
then look,
one born in my house
will be my heir!”
\par }{\PP \VS{4}But look,
the word
of the {\ND{Lord}}
came to
him: “This man
will not
be your heir,
but instead
a son who comes
from your own body
will be your heir.”
\VS{5}The
{\ND{Lord}} took
him outside
and said,
“Gaze
into the sky
and count
the stars
– if
you are able
to count
them!” Then he said
to him, “So
will your descendants be.”
\par }{\PP \VS{6}Abram believed
the {\ND{Lord}}, and the
{\ND{Lord}} considered
his response of faith as proof of genuine loyalty.
\par }{\PP \VS{7}The
{\ND{Lord}} said
to
him, “I am
the {\ND{Lord}}
who
brought you out
from Ur
of the Chaldeans
to give
you this
land
to possess.”
\VS{8}But Abram said,
“O sovereign

{\ND{Lord}}, by what
can I know
that
I am
to possess it?”
\par }{\PP \VS{9}The
{\ND{Lord}} said
to him,
“Take
for me a heifer,
a goat,
and a ram,
each three years old,
along with a dove
and a young pigeon.”
\VS{10}So Abram took
all
these
for him and then cut
them in two
and placed
each
half
opposite
the other,
but he did not
cut the birds
in half.
\VS{11}When birds of prey
came down
on
the carcasses,
Abram drove them away.
\par }{\PP \VS{12}When
the sun
went down,
Abram
fell
sound asleep,
and great
terror
overwhelmed
him.
\VS{13}Then the
{\ND{Lord}} said
to Abram,
“Know
for
certain
that your descendants
will be
strangers in a foreign
country.
They will be enslaved
and oppressed
for four
hundred
years.
\VS{14}But I
will execute judgment on
the nation
that
they will serve.
Afterward
they will come out
with many possessions.
\VS{15}But as for you,
you will go
to
your ancestors
in peace
and be buried
at a good
old age.
\VS{16}In the fourth
generation
your descendants will return
here,
for
the sin
of the Amorites
has not
yet
reached its limit.”
\par }{\PP \VS{17}When
the sun
had gone
down and it was
dark,
a smoking
firepot
with a flaming
torch
passed
between
the animal parts.
\VS{18}That
day
the {\ND{Lord}}
made
a covenant
with
Abram: “To your descendants
I give
this
land,
from the river
of Egypt
to
the great
river,
the Euphrates
River –
\VS{19}the land of the Kenites,
Kenizzites,
Kadmonites,
\VS{20}Hittites,
Perizzites,
Rephaites,
\VS{21}Amorites,
Canaanites,
Girgashites,
and Jebusites.”

\par }\Chap{16}{\PP \VerseOne{1}Now Sarai,
Abram’s
wife,
had not
given birth
to any children, but she had an Egyptian
servant
named
Hagar.
\VS{2}So Sarai
said
to
Abram,
“Since
the {\ND{Lord}}
has prevented
me from having children,
have sexual relations
with
my servant.
Perhaps
I can have a family
by
her.” Abram
did what Sarai told him.
\par }{\PP \VS{3}So after
Abram
had lived
in Canaan
for ten
years,
Sarai,
Abram’s
wife,
gave
Hagar,
her Egyptian
servant,
to her husband
to be
his
wife.
\VS{4}He had sexual relations
with
Hagar,
and she became pregnant.
Once Hagar realized
she was pregnant,
she despised
Sarai.
\VS{5}Then Sarai
said
to
Abram,
“You have brought this wrong
on
me! I
allowed
my servant
to have
sexual relations
with you, but when she realized
that
she was pregnant,
she despised
me. May the
{\ND{Lord}}
judge
between you and me!”
\par }{\PP \VS{6}Abram
said
to
Sarai,
“Since
your servant
is under your authority,
do
to her whatever
you think
best.” Then Sarai
treated Hagar harshly, so she ran away from
Sarai.
\par }{\PP \VS{7}The
{\ND{Lord’s}}
angel
found
Hagar near a spring
of water
in the desert
– the spring
that is along the road
to Shur.
\VS{8}He said,
“Hagar,
servant
of Sarai,
where
have you come
from, and where
are you going?” She
replied,
“I’m
running away from
my mistress,
Sarai.”
\par }{\PP \VS{9}Then the
{\ND{Lord’s}}
angel
said
to her, “Return
to
your mistress
and submit
to her authority.
\VS{10}I will greatly
multiply
your descendants,”
the
{\ND{Lord’s}}
angel
added, “so
that they will be too numerous
to count.”
\VS{11}Then the
{\ND{Lord’s}}
angel
said
to her,
\par }{\Q “You are now
pregnant
\par }{\Q and are about to give birth
to a son.
\par }{\Q You are to name
him Ishmael,
\par }{\Q for
the {\ND{Lord}}
has heard
your painful groans.
\par }{\Q \VS{12}He
will be
a wild donkey
of a man.
\par }{\Q He will be hostile
to everyone,
\par }{\Q and everyone
will be hostile to him.

\par }{\Q He will live
away from his brothers.”
\par }{\PP \VS{13}So
Hagar named
the {\ND{Lord}}
who spoke
to her,
“You
are the God
who sees
me,” for
she said,
“Here
I have seen
one who sees me!”
\VS{14}That is why
the well
was called
Beer Lahai Roi.
(It is located
between
Kadesh
and Bered.)
\par }{\PP \VS{15}So Hagar
gave birth
to Abram’s
son,
whom Abram
named
Ishmael.
\VS{16}(Now Abram
was 86
years
old
when Hagar
gave birth
to Ishmael.)

\par }\Chap{17}{\PP \VerseOne{1}When
Abram
was 99
years
old,
the {\ND{Lord}}
appeared
to
him
and said, “I
am the sovereign
God.
Walk
before
me and be
blameless.
\VS{2}Then I will confirm
my covenant
between
me and you, and I will give
you a multitude
of descendants.”
\par }{\PP \VS{3}Abram
bowed down
with
his face
to the ground,
and God
said to him,
\VS{4}“As for me,
this is
my covenant
with
you: You will be
the father
of a multitude
of nations.
\VS{5}No
longer
will your name
be Abram.
Instead, your name
will be
Abraham
because
I will make
you the father
of a multitude
of nations.
\VS{6}I will make you extremely
fruitful.
I will make
nations
of you, and kings
will descend
from you.
\VS{7}I
will confirm
my covenant
as a perpetual
covenant
between
me and you. It will extend to your descendants
after
you throughout their generations.
I will be
your God and the God
of your descendants
after you.
\VS{8}I will give
the whole
land
of Canaan
– the land
where you are now residing –
to you and your descendants
after
you as a permanent
possession.
I will be
their God.”
\par }{\PP \VS{9}Then God
said
to Abraham,
“As for you,
you must keep
the covenantal
requirement I am imposing
on you
and your descendants
after
you throughout their generations.
\VS{10}This
is my requirement
that
you and your descendants
after
you must keep: Every
male
among you must be circumcised.
\VS{11}You must circumcise
the flesh
of your foreskins.
This will be
a reminder
of the covenant
between me and you.
\VS{12}Throughout your generations
every
male
among you who is eight
days
old must be circumcised,
whether born
in your house
or bought
with money
from any
foreigner
who
is not
one of your descendants.
\VS{13}They must indeed be
circumcised,
whether born
in your house
or bought
with money.
The sign of my covenant
will be
visible in your flesh
as a permanent
reminder.
\VS{14}Any uncircumcised
male
who
has not
been circumcised
in the flesh
of his foreskin
will be cut off
from his people
– he has failed
to carry out my requirement.”
\par }{\PP \VS{15}Then God
said
to
Abraham,
“As for your wife,
you must no longer
call
her Sarai;
Sarah
will be her name.
\VS{16}I will bless
her and will give
you a son
through her. I will bless
her and she will become
a mother of nations.
Kings
of countries
will come
from her!”
\par }{\PP \VS{17}Then
Abraham
bowed
down with his face
to the ground and laughed
as he said
to himself, “Can a son
be born
to a man who is a hundred
years
old? Can
Sarah
bear a child
at the age
of ninety?”
\VS{18}Abraham
said
to
God,
“O that Ishmael
might
live
before you!”
\par }{\PP \VS{19}God
said,
“No,
Sarah
your wife
is going to bear
you a son,
and you will name
him Isaac.
I will confirm
my covenant
with
him as a perpetual
covenant
for his descendants
after him.
\VS{20}As for Ishmael,
I have heard
you. I will indeed
bless
him, make him fruitful,
and give him a multitude
of descendants. He will become the
father
of twelve
princes;
I will make him
into a great
nation.
\VS{21}But I will establish
my covenant
with
Isaac,
whom
Sarah
will bear
to you at this
set time next
year.”
\VS{22}When he finished
speaking
with
Abraham,
God
went up from him.
\par }{\PP \VS{23}Abraham
took
his son
Ishmael
and every
male
in his household
(whether born
in his house
or bought
with money) and circumcised
them on that
very same day,
just
as God
had told him to do.
\VS{24}Now Abraham
was 99
years
old
when he was circumcised;
\VS{25}his son
Ishmael
was thirteen
years
old when
he was circumcised.
\VS{26}Abraham
and his son
Ishmael
were circumcised
on the very
same day.
\VS{27}All
the men
of his household,
whether born
in his household
or bought
with money
from a foreigner,
were circumcised
with him.

\par }\Chap{18}{\PP \VerseOne{1}The
{\ND{Lord}}
appeared
to
Abraham by the oaks
of Mamre
while he was sitting
at the entrance
to his tent
during the hottest
time of the day.
\VS{2}Abraham looked
up and saw
three
men
standing
across from him.
When he saw
them he ran
from the entrance
of the tent
to meet
them and bowed
low to the ground.
\par }{\PP \VS{3}He said,
“My lord,
if
I have found
favor
in your sight,
do not
pass
by and leave
your servant.
\VS{4}Let
a little
water
be brought
so that you may all wash
your feet
and rest
under
the tree.
\VS{5}And let me get
a bit
of food
so
that
you may refresh
yourselves since
you have passed
by your servant’s
home. After
that you may
be on
your way.” “All
right,”
they replied,
“you may
do
as
you say.”
\par }{\PP \VS{6}So Abraham
hurried
into the tent
and said
to
Sarah,
“Quick! Take three
measures
of fine flour,
knead
it, and make
bread.”
\VS{7}Then
Abraham
ran
to the herd
and chose
a fine,
tender
calf,
and gave
it to
a servant,
who quickly
prepared it.
\VS{8}Abraham then took
some curds
and milk,
along with the calf
that had
been prepared,
and placed
the food before
them. They ate
while he was standing
near them under
a tree.
\par }{\PP \VS{9}Then they asked
him,
“Where
is Sarah
your wife?” He replied,
“There,
in the tent.”
\VS{10}One of them said,
“I
will surely return
to
you when
the season
comes round
again, and your wife
Sarah
will have a son!” (Now Sarah
was listening
at the entrance
to the tent,
not far behind him.
\VS{11}Abraham
and Sarah
were old
and advancing
in years;
Sarah
had long since passed menopause.)
\VS{12}So Sarah
laughed
to herself,
thinking, “After
I am worn out
will I have pleasure,
especially when my husband
is old too?”
\par }{\PP \VS{13}The
{\ND{Lord}}
said
to
Abraham,
“Why
did Sarah
laugh
and say,
‘Will I really
have a child
when I am
old?’
\VS{14}Is anything impossible for
the {\ND{Lord}}? I will return
to
you when
the season
comes round again and Sarah
will have a son.”
\VS{15}Then
Sarah
lied,
saying,
“I did not
laugh,”
because
she was afraid.
But the
{\ND{Lord}} said,
“No! You did laugh.”
\par }{\SH Abraham Pleads for Sodom
\par }{\PP \VS{16}When the men
got
up to leave,
they looked
out over
Sodom.
(Now Abraham
was walking
with
them to see them on their way.)
\VS{17}Then the
{\ND{Lord}}
said,
“Should
I
hide
from Abraham
what
I
am
about to do?
\VS{18}After all, Abraham
will surely become
a great
and powerful
nation,
and all
the nations
on the earth
will pronounce blessings on one another using his name.
\VS{19}I have chosen
him so that he may command
his children
and his household
after
him to keep
the way
of the {\ND{Lord}}
by doing
what is right
and just.
Then the
{\ND{Lord}}
will give
to
Abraham
what
he promised him.”
\par }{\PP \VS{20}So the
{\ND{Lord}}
said,
“The outcry
against Sodom
and Gomorrah
is so
great
and their sin
so blatant
\VS{21}that I must go down
and see
if they are as wicked as the outcry
suggests.
If
not,
I want
to know.”
\par }{\PP \VS{22}The two men
turned
and headed
toward Sodom,
but Abraham
was still
standing
before
the {\ND{Lord}}.
\VS{23}Abraham
approached
and said,
“Will you sweep
away the godly
along with
the wicked?
\VS{24}What if
there
are fifty
godly
people in
the city? Will you really
wipe
it out and not
spare
the place
for
the sake of the fifty
godly
people who are in it?
\VS{25}Far be it
from you to do
such a thing
– to kill
the godly
with
the wicked,
treating the godly
and the wicked
alike! Far be it
from you! Will not
the judge
of the whole
earth
do
what is right?”
\par }{\PP \VS{26}So the
{\ND{Lord}}
replied,
“If
I find
in
the city
of Sodom
fifty
godly
people, I will spare
the whole
place
for their sake.”
\par }{\PP \VS{27}Then
Abraham
asked,
“Since
I have undertaken
to speak
to
the Lord
(although I am
but dust
and ashes),
\VS{28}what if
there are five
less
than the fifty
godly
people? Will you destroy
the whole
city
because five are lacking?” He replied,
“I will not
destroy
it if
I find
forty-five
there.”
\par }{\PP \VS{29}Abraham
spoke
to him
again, “What if
forty
are found
there?” He replied,
“I will not
do
it for the sake of
the forty.”
\par }{\PP \VS{30}Then Abraham said,
“May the Lord
not
be angry
so that I may speak! What if
thirty
are found
there?” He replied,
“I will not
do
it if
I find
thirty
there.”
\par }{\PP \VS{31}Abraham said,
“Since I
have undertaken
to speak
to
the Lord,
what if
only twenty
are found
there?” He replied,
“I will not
destroy
it for the sake of
the twenty.”
\par }{\PP \VS{32}Finally Abraham said,
“May
the Lord
not
be angry
so that I may speak
just
once
more. What if
ten
are found
there?” He replied,
“I will not
destroy
it for the sake of
the ten.”
\par }{\PP \VS{33}The
{\ND{Lord}}
went
on his way when he had
finished
speaking
to Abraham.
Then Abraham
returned
home.

\par }\Chap{19}{\PP \VerseOne{1}The two
angels
came
to Sodom
in the evening
while Lot
was sitting
in the city’s gateway.
When Lot
saw
them, he got up
to meet
them and bowed down
with his face toward
the ground.
\par }{\PP \VS{2}He said,
“Here,
my lords,
please
turn
aside
to
your servant’s
house.
Stay
the night and wash
your feet.
Then
you can be on your way
early
in the morning.” “No,”
they replied,
“we’ll spend
the night in the town square.”
\par }{\PP \VS{3}But he urged
them persistently,
so they turned aside
with him
and entered
his house.
He prepared
a feast
for them, including bread
baked
without yeast,
and they ate.
\VS{4}Before
they could lie
down to sleep, all
the men
– both young
and old,
from every
part
of the city
of Sodom
– surrounded
the house.
\VS{5}They shouted
to
Lot, “Where
are the men
who
came
to
you tonight? Bring them out
to
us so we can have sex with them!”
\par }{\PP \VS{6}Lot
went outside
to them, shutting the door
behind him.
\VS{7}He said,
“No, my brothers! Don’t
act
so wickedly!
\VS{8}Look,
I have two
daughters
who have
never
had
sexual relations
with a man.
Let
me bring them out
to
you, and you can do
to them whatever
you please.
Only
don’t
do
anything
to these
men,
for
they have
come
under the protection
of my roof.”
\par }{\PP \VS{9}“Out
of our way!” they cried,
and “This man
came
to live here as a foreigner,
and now
he dares to judge
us! We’ll
do more harm
to you than to them!” They
kept pressing
in on Lot
until they were close enough to break
down the door.
\par }{\PP \VS{10}So
the men
inside
reached
out and pulled Lot
back into
the house
as they shut
the door.
\VS{11}Then they struck
the men
who
were at the door
of the house,
from the youngest
to the oldest,
with blindness.
The men outside wore themselves out trying
to find
the door.
\VS{12}Then the two visitors
said
to
Lot,
“Who
else
do you have here? Do you have any sons-in-law,
sons,
daughters,
or other relatives in the city? Get them out
of this place
\VS{13}because
we
are about to destroy
it. The outcry
against this
place
is so great
before
the {\ND{Lord}}
that he has sent
us to destroy it.”
\par }{\PP \VS{14}Then Lot
went out
and spoke
to
his sons-in-law
who were going to marry
his daughters.
He said,
“Quick, get
out
of this
place
because
the {\ND{Lord}}
is about to destroy
the
city!” But
his sons-in-law
thought he was ridiculing them.
\par }{\PP \VS{15}At dawn
the angels
hurried
Lot
along, saying,
“Get
going! Take
your wife
and your two
daughters
who
are here, or else
you will be destroyed
when the city
is judged!”
\VS{16}When
Lot hesitated,
the men
grabbed
his hand
and the hands
of his wife
and two
daughters
because the
{\ND{Lord}}
had compassion
on
them. They led
them away and placed
them outside
the city.
\VS{17}When
they had brought
them outside,
they said,
“Run
for your lives! Don’t
look
behind
you
or stop
anywhere
in the valley! Escape
to the mountains
or
you will be destroyed!”
\par }{\PP \VS{18}But Lot
said
to
them, “No,
please,
Lord!
\VS{19}Your
servant
has
found
favor
with you, and you have
shown me great
kindness
by sparing
my life.
But I am
not
able
to escape
to the mountains
because
this disaster
will overtake
me and I’ll
die.
\VS{20}Look,
this town
over here is close
enough to escape
to, and it’s just a little
one.
Let
me go there.
It’s just a little place,
isn’t
it? Then I’ll survive.”
\par }{\PP \VS{21}“Very well,” he replied, “I
will grant
this
request
too
and will not
overthrow
the town
you mentioned.
\VS{22}Run
there
quickly,
for
I cannot
do
anything
until
you arrive
there.”
(This incident
explains
why the town
was called
Zoar.)
\par }{\PP \VS{23}The sun
had just risen
over
the land
as Lot
reached
Zoar.
\VS{24}Then the
{\ND{Lord}}
rained
down sulfur
and fire
on
Sodom
and Gomorrah.
It was sent down from
the sky
by the
{\ND{Lord}}.
\VS{25}So he overthrew
those
cities
and all
that region,
including all
the inhabitants
of the cities
and the vegetation that grew
from the ground.
\VS{26}But Lot’s
wife
looked
back
longingly and was turned into a pillar
of salt.
\par }{\PP \VS{27}Abraham
got up early
in the morning
and went to
the place
where
he had
stood
before
the {\ND{Lord}}.
\VS{28}He looked
out toward
Sodom
and Gomorrah
and all
the land
of that region.
As he did so, he saw
the smoke
rising up from
the land
like smoke
from a furnace.
\par }{\PP \VS{29}So when
God
destroyed
the cities
of the region,
God
honored
Abraham’s
request.
He removed Lot
from the midst
of the destruction
when he destroyed
the cities
Lot
had
lived in.
\par }{\PP \VS{30}Lot
went up
from Zoar
with
his two
daughters
and settled
in the mountains
because
he was afraid
to live
in Zoar.
So he lived
in a cave
with his two
daughters.
\VS{31}Later the older daughter
said
to
the younger,
“Our father
is old,
and there is no
man
anywhere nearby
to have sexual relations
with us, according to
the way
of all
the world.
\VS{32}Come, let’s
make our father
drunk
with wine
so we can have sexual
relations with
him and preserve
our family
line through our father.”
\par }{\PP \VS{33}So
that night
they made
their father
drunk
with wine,
and the older daughter
came
and had sexual
relations with
her father.
But he was not
aware
that she had sexual
relations with him and then got up.
\VS{34}So in the morning
the older daughter
said
to
the younger,
“Since I
had sexual
relations with
my father
last night,
let’s make him drunk
again tonight.
Then you go
and have sexual
relations with
him so we can preserve
our family line
through our father.”
\VS{35}So they made
their father
drunk
that night
as well,
and
the younger
one came and had sexual
relations with
him. But he was not
aware
that she had sexual
relations with him and then got up.
\par }{\PP \VS{36}In this way both
of Lot’s
daughters
became pregnant
by their father.
\VS{37}The older daughter
gave birth
to a son
and named
him Moab.
He is
the ancestor
of the Moabites
of today.
\VS{38}The younger
daughter also
gave birth
to a son
and named
him Ben-Ammi.
He is
the ancestor
of the Ammonites
of today.

\par }\Chap{20}{\PP \VerseOne{1}Abraham
journeyed
from there
to the Negev
region
and settled
between
Kadesh
and Shur.
While he lived as a temporary resident
in Gerar,
\VS{2}Abraham
said
about
his wife
Sarah,
“She is my sister.”
So
Abimelech,
king
of Gerar,
sent for
Sarah
and took her.
\par }{\PP \VS{3}But God
appeared
to
Abimelech
in a dream
at night
and said
to him, “You are as good as dead
because of the woman
you have
taken,
for
she
is someone else’s
wife.”
\par }{\PP \VS{4}Now Abimelech
had not
gone near
her. He said,
“Lord,
would you really
slaughter
an innocent
nation?
\VS{5}Did Abraham not
say
to me, ‘She is my sister’? And she herself said, ‘He is my brother.’
I have done
this
with a clear conscience
and with innocent
hands!”
\par }{\PP \VS{6}Then in the dream
God
replied
to
him, “Yes, I
know
that
you have done
this
with a clear conscience.
That is why I
have kept
you from sinning
against
me and why
I did not
allow
you to
touch her.
\VS{7}But now
give back
the man’s
wife.
Indeed
he is a prophet
and he will pray
for
you; thus
you will live.
But if
you don’t
give her back,
know
that
you will surely
die
along with all
who
belong to you.”
\par }{\PP \VS{8}Early
in the morning
Abimelech
summoned
all
his servants.
When he told
them about all
these
things,
they
were terrified.
\VS{9}Abimelech
summoned
Abraham
and said
to him, “What
have you done
to us? What
sin
did I commit against you that
would cause you to bring
such
great
guilt
on
me and my kingdom? You have done
things to me
that
should not
be done!”
\VS{10}Then Abimelech
asked
Abraham,
“What
prompted
you to do
this
thing?”
\par }{\PP \VS{11}Abraham
replied,
“Because
I thought, ‘Surely
no
one fears
God
in this
place.
They will kill
me because
of my wife.’
\VS{12}What’s more,
she is indeed
my sister,
my father’s
daughter,
but
not
my mother’s
daughter.
She became
my wife.
\VS{13}When
God
made me wander
from my father’s
house,
I told
her, ‘This
is what
you can do
to show your loyalty
to me: Every
place
we go,
say
about
me, “He is
my brother.” ’ ”
\par }{\PP \VS{14}So Abimelech
gave
sheep,
cattle,
and male and female
servants
to Abraham.
He also gave
his wife
Sarah
back to him.
\VS{15}Then Abimelech
said,
“Look,
my land
is before
you; live
wherever
you please.”
\par }{\PP \VS{16}To Sarah
he said,
“Look,
I have given
a thousand
pieces of silver
to your ‘brother.’
This is
compensation
for you so that you will stand vindicated
before
all
who
are with you.”
\par }{\PP \VS{17}Abraham
prayed
to
God,
and God
healed
Abimelech,
as well as his wife
and female slaves
so that they were able to have children.
\VS{18}For
the {\ND{Lord}}
had caused
infertility
to
strike every
woman in the household
of Abimelech
because
he took Sarah,
Abraham’s
wife.

\par }\Chap{21}{\PP \VerseOne{1}The
{\ND{Lord}}
visited
Sarah
just as
he had said
he
would and did
for Sarah
what
he had promised.
\VS{2}So Sarah
became pregnant
and bore
Abraham
a son
in his old age
at the appointed
time that
God
had
told
him.
\VS{3}Abraham
named
his son
– whom
Sarah
bore to him – Isaac.
\VS{4}When his son
Isaac
was eight
days
old, Abraham
circumcised
him just
as God
had commanded him to do.
\VS{5}(Now Abraham
was a hundred
years
old when his son
Isaac
was born to him.)
\par }{\PP \VS{6}Sarah
said,
“God
has made
me laugh.
Everyone
who hears
about this will laugh with me.”
\VS{7}She went on to say, “Who
would
have said to Abraham
that
Sarah
would nurse
children? Yet
I have given birth
to a son
for him in his old age!”
\par }{\PP \VS{8}The child
grew
and was weaned.
Abraham
prepared
a great
feast
on the day
that Isaac
was weaned.
\VS{9}But
Sarah
noticed
the son
of Hagar
the Egyptian
– the son whom
Hagar had
borne
to Abraham
– mocking.
\VS{10}So she said
to Abraham,
“Banish
that slave woman
and her son,
for
the son
of that slave woman
will not
be an heir
along with
my son
Isaac!”
\par }{\PP \VS{11}Sarah’s
demand
displeased
Abraham
greatly
because
Ishmael was his son.
\VS{12}But God
said
to
Abraham,
“Do not
be upset
about
the boy
or your slave wife.
Do all
that
Sarah
is telling
you because
through Isaac
your descendants
will be counted.
\VS{13}But I will also
make
the son
of the slave wife
into a great nation,
for
he is
your descendant too.”
\par }{\PP \VS{14}Early
in the morning
Abraham
took
some food
and a skin
of water
and gave
them to
Hagar.
He put
them on
her shoulders,
gave her the
child,
and sent
her away. So she went
wandering aimlessly
through the wilderness
of Beer Sheba.
\VS{15}When
the water
in the skin
was gone, she shoved
the child
under
one
of the shrubs.
\VS{16}Then she went
and sat down
by herself across from
him at quite a distance, about a bowshot
away;
for
she thought, “I refuse
to watch
the child
die.”
So she sat
across from
him and wept uncontrollably.
\par }{\PP \VS{17}But God
heard
the boy’s
voice.
The angel
of God
called
to
Hagar
from
heaven
and asked
her, “What
is the matter, Hagar? Don’t
be afraid,
for
God
has heard
the boy’s
voice
right where
he is crying.
\VS{18}Get
up! Help
the boy
up and hold
him by the hand,
for
I will make
him into a great
nation.”
\VS{19}Then
God
enabled
Hagar to see
a well
of water.
She went
over and filled
the skin
with water,
and then gave the boy
a drink.
\par }{\PP \VS{20}God
was
with
the boy
as he grew.
He lived
in the wilderness
and became
an archer.
\VS{21}He lived
in the wilderness
of Paran.
His mother
found a wife
for him from the land
of Egypt.
\par }{\PP \VS{22}At that time
Abimelech
and Phicol,
the commander
of his army,
said
to
Abraham,
“God
is with
you in all
that
you
do.
\VS{23}Now
swear
to me right here
in God’s
name that you will not deceive
me, my
children,
or my
descendants.
Show me,
and the land
where
you are staying,
the same loyalty
that
I have
shown you.”
\par }{\PP \VS{24}Abraham
said,
“I
swear to do this.”
\VS{25}But Abraham
lodged a complaint
against
Abimelech
concerning
a well
that
Abimelech’s
servants
had seized.
\VS{26}“I do not
know
who
has done
this
thing,”
Abimelech
replied.
“Moreover,
you
did not
tell
me. I
did not
hear
about it until
today.”
\par }{\PP \VS{27}Abraham
took
some sheep
and cattle
and gave
them to Abimelech.
The two
of them made
a treaty.
\VS{28}Then Abraham
set
seven
ewe
lambs apart from the flock
by themselves.
\VS{29}Abimelech
asked
Abraham,
“What
is the meaning of
these
seven
ewe
lambs that
you have set
apart?”
\VS{30}He replied,
“You must
take
these seven
ewe
lambs from my hand
as legal proof
that
I dug
this
well.”
\VS{31}That is why
he named
that place
Beer Sheba,
because
the two
of them swore
an oath there.
\par }{\PP \VS{32}So they made
a treaty
at Beer Sheba.
Then
Abimelech
and Phicol,
the commander
of his army,
returned
to
the land
of the Philistines.
\VS{33}Abraham planted
a tamarisk tree
in Beer Sheba.
There
he worshiped
the {\ND{Lord}}, the eternal
God.
\VS{34}So Abraham
stayed
in the land
of the Philistines
for quite
some time.

\par }\Chap{22}{\PP \VerseOne{1}Some time
after
these
things
God
tested
Abraham.
He said
to him,
“Abraham!” “Here
I am!” Abraham replied.
\VS{2}God said,
“Take
your
son
– your only
son, whom
you love,
Isaac –
and go
to
the land
of Moriah! Offer him up
there
as a burnt offering
on
one
of the mountains
which
I will indicate
to you.”
\par }{\PP \VS{3}Early
in the morning
Abraham
got up and saddled
his donkey.
He took
two
of his young servants
with
him, along
with his son
Isaac.
When he had cut
the wood
for the burnt offering,
he started
out for the place
God
had spoken to him about.
\par }{\PP \VS{4}On
the third
day
Abraham
caught
sight
of the place
in the distance.
\VS{5}So he
said
to
his servants,
“You two stay
here
with
the donkey
while
the boy
and I
go
up there.
We will worship
and then return
to you.”
\par }{\PP \VS{6}Abraham
took
the
wood
for the burnt offering
and put
it on
his son
Isaac.
Then he took
the
fire
and the
knife
in his hand,
and the two
of them walked on
together.
\VS{7}Isaac
said
to
his father
Abraham, “My father?” “What is it, my son?” he replied.
“Here is
the fire
and the wood,”
Isaac said, “but where
is the lamb
for the burnt offering?”
\VS{8}“God
will provide
for himself the lamb
for the burnt offering,
my son,”
Abraham
replied.
The two
of them continued
on together.
\par }{\PP \VS{9}When they came
to
the place
God
had
told
him about,
Abraham
built
the
altar
there
and arranged
the
wood
on it. Next he tied
up his son
Isaac
and placed
him on
the altar
on top
of the wood.
\VS{10}Then Abraham
reached
out his hand,
took
the knife,
and prepared to slaughter
his son.
\VS{11}But the
{\ND{Lord’s}}
angel
called
to him
from
heaven,
“Abraham! Abraham!” “Here
I am!” he answered.
\VS{12}“Do not
harm
the boy!” the angel said. “Do not
do
anything
to him, for
now
I know
that
you fear
God
because
you
did not
withhold
your son,
your only son,
from me.”
\par }{\PP \VS{13}Abraham
looked up
and saw
behind
him a ram
caught
in the bushes
by its horns.
So he
went over
and got
the ram
and offered it up
as a burnt offering
instead
of his son.
\VS{14}And Abraham
called
the name
of that place
“The
{\ND{Lord}}
provides.”
It is
said
to this day, “In the mountain
of the {\ND{Lord}}
provision will be made.”
\par }{\PP \VS{15}The
{\ND{Lord}}’s
angel
called
to
Abraham
a second
time from
heaven
\VS{16}and said,
“ ‘I solemnly swear
by my own name,’ decrees the
{\ND{Lord}}, ‘that because
you have
done
this
and have not
withheld
your son,
your only son,
\VS{17}I will indeed
bless
you, and I will greatly multiply
your descendants
so that they will be as countless as the stars
in the sky
or the grains of sand
on
the seashore.
Your descendants
will take possession
of the strongholds
of their enemies.
\VS{18}Because
you have
obeyed
me,
all
the nations
of the earth
will pronounce
blessings
on one another using the name of your descendants.’ ”
\par }{\PP \VS{19}Then Abraham
returned
to
his servants,
and they set out
together
for Beer Sheba
where Abraham
stayed.
\par }{\PP \VS{20}After
these
things
Abraham
was
told,
“Milcah
also
has borne children
to your brother
Nahor –
\VS{21}Uz
the firstborn,
his brother
Buz,
Kemuel
(the father
of Aram),
\VS{22}Kesed,
Hazo,
Pildash,
Jidlaph,
and Bethuel.”
\VS{23}(Now Bethuel
became the father
of Rebekah.) These
were the eight
sons Milcah
bore
to Abraham’s
brother
Nahor.
\VS{24}His concubine,
whose name
was Reumah,
also
bore him children – Tebah, Gaham, Tahash, and Maacah.

\par }\Chap{23}{\PP \VerseOne{1}Sarah
lived
127
years.
\VS{2}Then she
died
in Kiriath Arba
(that
is, Hebron) in the land
of Canaan.
Abraham
went
to mourn
for Sarah
and to weep for her.
\par }{\PP \VS{3}Then Abraham
got up
from
mourning his dead
wife and said
to
the sons
of Heth,
\VS{4}“I
am a temporary
settler
among you. Grant
me ownership
of a burial site
among
you so that I may bury
my dead.”
\par }{\PP \VS{5}The sons
of Heth
answered
Abraham,
\VS{6}“Listen,
sir,
you are
a mighty prince
among
us! You
may bury
your dead
in the choicest
of our tombs.
None
of us will refuse
you his tomb
to prevent
you from
burying
your dead.”
\par }{\PP \VS{7}Abraham
got
up and bowed
down to the local
people,
the sons
of Heth.
\VS{8}Then
he said
to them, “If
you agree
that I
may
bury
my dead,
then hear
me out. Ask
Ephron
the son
of Zohar
\VS{9}if he will sell
me the
cave
of Machpelah
that
belongs to
him; it is at the end
of his field.
Let him sell
it to me publicly
for the full
price,
so that I may own
it as a burial site.”
\par }{\PP \VS{10}(Now Ephron
was sitting
among
the sons
of Heth.) Ephron
the Hethite
replied
to Abraham
in the hearing
of the sons
of Heth
– before all
who entered
the gate
of his city –
\VS{11}“No,
my lord! Hear
me out. I sell
you both the field
and the cave
that
is in it. In the presence
of my people
I sell
it to you. Bury
your dead.”
\par }{\PP \VS{12}Abraham
bowed
before
the local
people
\VS{13}and said
to
Ephron
in their hearing,
“Hear
me, if
you will. I
pay
to you
the price
of the field.
Take
it
from
me so that I may bury
my dead
there.”
\par }{\PP \VS{14}Ephron
answered
Abraham,
saying to him,
\VS{15}“Hear
me, my lord.
The land
is worth 400
pieces
of silver,
but what
is that
between
me and you? So bury
your dead.”
\par }{\PP \VS{16}So Abraham
agreed
to
Ephron’s
price and weighed out
for him
the price
that Ephron
had
quoted
in the hearing
of the sons
of Heth
– 400
pieces
of silver,
according to the standard
measurement at the time.
\par }{\PP \VS{17}So
Abraham secured
Ephron’s
field
in Machpelah,
next to Mamre,
including the field,
the cave
that
was in it, and all
the trees
that
were in the field
and all
around
its border,
\VS{18}as his
property
in the presence
of the sons
of Heth
before all
who entered
the gate
of Ephron’s city.
\par }{\PP \VS{19}After
this
Abraham
buried
his wife
Sarah
in the cave
in the field
of Machpelah
next to Mamre
(that is,
Hebron) in the land
of Canaan.
\VS{20}So Abraham
secured
the field
and the cave
that
was in it as a burial
site
from the sons
of Heth.

\par }\Chap{24}{\PP \VerseOne{1}Now Abraham
was old,
well advanced
in years,
and the
{\ND{Lord}}
had blessed
him
in everything.
\VS{2}Abraham
said
to
his servant,
the senior
one in his household
who was in charge
of everything
he had,
“Put
your
hand
under
my thigh
\VS{3}so that I may make you solemnly
promise by the
{\ND{Lord}}, the God
of heaven
and the God
of the earth: You must not
acquire
a wife
for my son
from the daughters
of the Canaanites,
among
whom
I am
living.
\VS{4}You must go
instead
to my country
and to
my relatives
to find
a wife
for my son
Isaac.”
\par }{\PP \VS{5}The servant
asked
him,
“What if
the woman
is not
willing
to come
back
with me
to
this
land? Must I then take your son
back
to
the land
from which
you came?”
\par }{\PP \VS{6}“Be careful
never
to take my son
back
there!” Abraham
told
him.
\VS{7}“The
{\ND{Lord}}, the God
of heaven,
who
took
me from my father’s
house
and the land
of my relatives,
promised
me with a solemn
oath, ‘To your descendants
I will give
this
land.’
He
will send
his angel
before
you so that you may find
a wife
for my son
from there.
\VS{8}But if
the woman
is not
willing
to come
back
with you, you will be free
from this
oath
of mine. But you must not
take my son
back
there!”
\VS{9}So the servant
placed
his hand
under
the thigh
of his master
Abraham
and gave his solemn
promise
he would carry out his wishes.
\par }{\PP \VS{10}Then the servant
took
ten
of his master’s
camels
and departed
with all
kinds of gifts
from his
master
at his disposal.
He journeyed
to
the region of Aram Naharaim
and the city
of Nahor.
\VS{11}He made the camels
kneel
down by the well
outside
the city.
It was evening,
the time
when the women would go out
to draw
water.
\VS{12}He prayed,
“O
{\ND{Lord}}, God
of my master
Abraham,
guide
me
today.
Be faithful
to my master
Abraham.
\VS{13}Here
I am,
standing
by the spring,
and the daughters
of the people
who live in the town
are coming out
to draw
water.
\VS{14}I will
say
to
a young
woman, ‘Please
lower
your jar
so I may drink.’
May the
one you have chosen
for your servant
Isaac
reply,
‘Drink,
and I’ll
give your camels
water
too.’
In this way I will know
that
you have been faithful
to my master.”
\par }{\PP \VS{15}Before
he had finished
praying,
there
came
Rebekah
with her water jug
on
her shoulder.
She was
the daughter of Bethuel
son
of Milcah
(Milcah
was the wife
of Abraham’s
brother
Nahor).
\VS{16}Now the young
woman was very
beautiful.
She was a virgin;
no
man
had ever had sexual relations
with her. She went down
to the spring,
filled
her jug,
and came back up.
\VS{17}Abraham’s servant
ran
to meet
her and said,
“Please
give me
a
sip
of water
from your jug.”
\VS{18}“Drink,
my lord,”
she replied,
and quickly
lowering
her jug
to her hands,
she gave him a drink.
\VS{19}When
she
had done so, she
said,
“I’ll
draw
water
for your camels
too, until
they have drunk as much as they want.”
\VS{20}She quickly
emptied
her jug
into
the watering trough
and ran
back to
the well
to draw
more water until she had drawn
enough for all
his camels.
\VS{21}Silently
the man
watched her with interest
to determine
if the
{\ND{Lord}}
had made his journey
successful
or not.
\par }{\PP \VS{22}After
the camels
had finished
drinking,
the man
took
out a gold
nose ring
weighing
a beka
and two
gold
bracelets
weighing
ten shekels and gave them to her.
\VS{23}“Whose
daughter
are you?” he asked. “Tell
me,
is there
room in your father’s
house
for us to spend the night?”
\par }{\PP \VS{24}She said
to him,
“I am
the daughter
of Bethuel
the son
of Milcah,
whom
Milcah bore
to Nahor.
\VS{25}We have
plenty
of straw
and feed,”
she added, “and room
for you
to spend the night.”
\par }{\PP \VS{26}The man
bowed
his head and worshiped
the {\ND{Lord}},
\VS{27}saying
“Praised
be the
{\ND{Lord}}, the God
of my master
Abraham,
who has
not
abandoned
his faithful
love for my master! The
{\ND{Lord}}
has led
me to the house
of my master’s
relatives!”
\par }{\PP \VS{28}The young
woman ran
and told
her mother’s
household
all about
these things.
\VS{29}(Now Rebekah
had a brother
named
Laban.) Laban
rushed
out
to
meet the man
at the spring.
\VS{30}When
he saw
the
bracelets
on
his
sister’s
wrists
and the nose ring
and heard
his
sister
Rebekah
say, “This is what
the man
said
to
me,” he went
out to
meet the man.
There
he was, standing
by the camels
near the spring.
\VS{31}Laban said
to him, “Come,
you who are blessed
by the
{\ND{Lord}}! Why
are you standing
out here
when I
have prepared
the house
and a place
for the camels?”
\par }{\PP \VS{32}So
Abraham’s servant went
to the house
and unloaded
the camels.
Straw
and feed
were given to the camels,
and water
was provided
so that
he and the men
who were with
him could wash
their feet.
\VS{33}When food
was served,
he said,
“I will not
eat
until
I have said
what I want to say.” “Tell
us,” Laban said.
\par }{\PP \VS{34}“I am
the servant
of Abraham,”
he began.
\VS{35}“The
{\ND{Lord}}
has richly blessed
my master
and he has become
very
wealthy.
The Lord has given
him sheep
and cattle,
silver
and gold,
male and female
servants,
and camels
and donkeys.
\VS{36}My master’s
wife
Sarah
bore
a son
to him
when she was
old, and my master
has given
him everything
he owns.
\VS{37}My master
made me swear
an oath. He said,
‘You must not
acquire
a wife
for my son
from the daughters
of the Canaanites,
among whom
I am
living,
\VS{38}but
you must
go
to
the family
of my father
and to
my relatives
to find
a wife
for my son.’
\VS{39}But I said
to
my master,
‘What if
the woman
does not
want to go with me?’
\VS{40}He answered,
‘The
{\ND{Lord}}, before
whom
I have walked,
will send
his angel
with
you. He will make your journey
a success
and you will find
a wife
for my son
from among my relatives,
from my father’s
family.
\VS{41}You will be free
from your oath
if
you go
to
my relatives
and they will
not
give
her to you. Then you will be
free
from your oath.’
\VS{42}When I came
to the spring
today,
I prayed,
‘O
{\ND{Lord}}, God
of my master
Abraham,
if
you have
decided
to make my journey
successful,
may events unfold as
follows:
\VS{43}Here
I am,
standing
by the spring.
When
the young woman goes
out
to draw
water,
I’ll say,
“Give
me
a little
water
to drink
from your jug.”
\VS{44}Then she will reply to
me, “Drink,
and I’ll
draw
water for your camels
too.” May that woman
be the one whom
the {\ND{Lord}}
has chosen
for my master’s
son.’
\par }{\PP \VS{45}“Before
I
finished
praying
in my heart,
along came
Rebekah
with her water jug
on
her shoulder! She went down
to the spring
and drew
water. So I said
to
her, ‘Please
give me a drink.’
\VS{46}She quickly
lowered
her jug
from her shoulder and said,
‘Drink,
and I’ll
give your camels
water
too.’
So I drank,
and she also
gave the camels
water.
\VS{47}Then I asked
her, ‘Whose
daughter
are you?’ She replied,
‘The daughter
of Bethuel
the son
of Nahor,
whom
Milcah
bore
to Nahor.’ I put
the ring
in her nose
and the bracelets
on
her wrists.
\VS{48}Then I bowed
down and worshiped
the {\ND{Lord}}. I praised
the

{\ND{Lord}}, the God
of my master
Abraham,
who had
led
me on the right
path
to find
the
granddaughter
of my master’s
brother
for his son.
\VS{49}Now,
if
you will show
faithful
love
to
my master,
tell
me. But if
not,
tell
me as well, so
that I may go
on
my way.”
\par }{\PP \VS{50}Then Laban
and Bethuel
replied,
“This is the
{\ND{Lord}}’s
doing.
Our wishes
are of
no
concern.
\VS{51}Rebekah
stands here before
you. Take
her and go
so that she may become
the wife
of your master’s
son,
just
as the
{\ND{Lord}}
has decided.”
\par }{\PP \VS{52}When
Abraham’s
servant
heard
their words,
he bowed
down to the ground
before the
{\ND{Lord}}.
\VS{53}Then he
brought out
gold,
silver
jewelry,
and clothing
and gave
them to Rebekah.
He also gave
valuable
gifts to her brother
and to her mother.
\VS{54}After
this, he and the men
who
were with
him ate
a meal
and stayed there overnight.
\par }{\PP When they got
up in the morning,
he said,
“Let
me leave now so I can return to my master.”
\VS{55}But Rebekah’s brother
and her mother
replied,
“Let the girl
stay
with
us a few more days,
perhaps ten.
Then
she can go.”
\VS{56}But he said
to
them, “Don’t
detain me – the
{\ND{Lord}} has granted me success on my journey. Let me leave now so I may return to my master.”
\VS{57}Then they said,
“We’ll call
the girl
and find
out what
she wants
to do.”
\VS{58}So they called
Rebekah
and asked
her, “Do you want to
go
with
this
man?” She replied,
“I want to go.”
\par }{\PP \VS{59}So they sent
their sister
Rebekah
on her way, accompanied by her female attendant,
with Abraham’s
servant
and his men.
\VS{60}They blessed
Rebekah
with these words:

\par }{\Q “Our sister,
may you
become
the mother of thousands
of ten thousands!
\par }{\Q May your descendants
possess
the strongholds
of their enemies.”
\par }{\PP \VS{61}Then
Rebekah
and her female
servants mounted
the camels
and rode away
with the man.
So Abraham’s servant
took
Rebekah
and left.
\par }{\PP \VS{62}Now Isaac
came
from
Beer Lahai Roi,
for he
was living
in the Negev.
\VS{63}He
went out
to relax
in the field
in the early evening.
Then he looked
up
and saw
that there
were camels
approaching.
\VS{64}Rebekah
looked
up and saw
Isaac.
She got down
from her camel
\VS{65}and asked
Abraham’s servant, “Who
is that man
walking
in the field
toward
us?” “That is my master,”
the servant
replied.
So she took
her veil
and covered herself.
\par }{\PP \VS{66}The servant
told
Isaac
everything
that had
happened.
\VS{67}Then Isaac
brought
Rebekah
into his mother
Sarah’s
tent.
He took
her as his wife
and loved
her. So Isaac
was comforted
after
his mother’s death.

\par }\Chap{25}{\PP \VerseOne{1}Abraham
had taken
another wife,
named
Keturah.
\VS{2}She bore
him Zimran,
Jokshan,
Medan,
Midian,
Ishbak,
and Shuah.
\VS{3}Jokshan
became the father
of Sheba
and Dedan.
The descendants
of Dedan
were
the Asshurites,
Letushites,
and Leummites.
\VS{4}The sons
of Midian
were Ephah,
Epher,
Hanoch,
Abida,
and Eldaah.
All
these
were descendants
of Keturah.
\par }{\PP \VS{5}Everything
he owned
Abraham
left to his son Isaac.
\VS{6}But while
he
was still
alive,
Abraham
gave
gifts
to the sons
of his concubines
and sent
them off to
the east,
away from his son
Isaac.
\par }{\PP \VS{7}Abraham
lived a total
of 175
years.
\VS{8}Then Abraham
breathed
his last
and died
at a good
old age,
an old man
who had lived a full
life. He joined
his ancestors.
\VS{9}His sons
Isaac
and Ishmael
buried
him in the cave
of Machpelah
near Mamre,
in the field
of Ephron
the son
of Zohar,
the Hethite.
\VS{10}This was the field
Abraham
had
purchased
from the sons
of Heth.
There
Abraham
was buried
with his wife
Sarah.
\VS{11}After
Abraham’s
death,
God
blessed
his son
Isaac.
Isaac
lived
near
Beer Lahai Roi.
\par }{\SH The Sons of Ishmael
\par }{\PP \VS{12}This
is the account
of Abraham’s
son
Ishmael,
whom
Hagar
the Egyptian,
Sarah’s
servant,
bore
to Abraham.
\par }{\PP \VS{13}These
are the names
of Ishmael’s
sons,
by their names
according to their records: Nebaioth
(Ishmael’s
firstborn), Kedar,
Adbeel,
Mibsam,
\VS{14}Mishma,
Dumah,
Massa,
\VS{15}Hadad,
Tema,
Jetur,
Naphish,
and Kedemah.
\VS{16}These
are
the sons
of Ishmael,
and these
are their names
by their settlements
and their camps
– twelve
princes
according to their clans.
\par }{\PP \VS{17}Ishmael
lived
a total
of 137
years.
He breathed his last
and died;
then he joined
his ancestors.
\VS{18}His descendants settled
from Havilah
to Shur,
which
runs next
to Egypt
all the way
to Asshur.
They settled
away from
all
their relatives.
\par }{\SH Jacob and Esau
\par }{\PP \VS{19}This
is the account
of Isaac,
the son
of Abraham.
\par }{\PP Abraham
became the father
of Isaac.
\VS{20}When
Isaac
was forty
years
old, he married
Rebekah,
the daughter
of Bethuel
the Aramean
from Paddan Aram
and sister
of Laban
the Aramean.
\par }{\PP \VS{21}Isaac
prayed
to the
{\ND{Lord}}
on behalf
of his wife
because
she was
childless.
The
{\ND{Lord}}
answered
his prayer,
and his wife
Rebekah
became pregnant.
\VS{22}But the children
struggled
inside her,
and she said,
“If
it is going
to be like
this,
I’m
not so
sure I want
to be pregnant!” So
she asked
the {\ND{Lord}},
\VS{23}and the
{\ND{Lord}}
said
to her,
\par }{\Q “Two
nations
are in your womb,
\par }{\Q and two
peoples
will be separated
from within
you.
\par }{\Q One people
will be stronger
than the other,
\par }{\Q and the older
will serve
the younger.”
\par }{\PP \VS{24}When
the time
came for Rebekah to give birth,
there
were twins
in her womb.
\VS{25}The first
came out
reddish
all
over, like a hairy
garment,
so
they named
him Esau.
\VS{26}When his brother
came out
with his hand
clutching
Esau’s
heel,
they named
him Jacob.
Isaac
was sixty
years
old
when they were born.
\par }{\PP \VS{27}When the boys
grew
up, Esau
became a skilled
hunter,
a man
of the open fields,
but Jacob
was an even-tempered
man,
living
in tents.
\VS{28}Isaac
loved
Esau
because
he had a taste for fresh game,
but Rebekah
loved
Jacob.
\par }{\PP \VS{29}Now
Jacob
cooked
some stew,
and when Esau
came
in from
the open fields,
he was
famished.
\VS{30}So Esau
said
to
Jacob,
“Feed
me some of the red stuff – yes, this red stuff – because I’m starving!” (That is why he was also called Edom.)
\par }{\PP \VS{31}But Jacob
replied,
“First sell
me your birthright.”
\VS{32}“Look,”
said
Esau,
“I’m
about
to die! What
use is the birthright to me?”
\VS{33}But Jacob
said,
“Swear
an oath to me now.”
So Esau swore
an oath to him and sold
his birthright
to Jacob.
\par }{\PP \VS{34}Then Jacob
gave
Esau
some bread
and lentil
stew;
Esau ate
and drank,
then got
up and went
out. So Esau
despised
his birthright.

\par }\Chap{26}{\PP \VerseOne{1}There was
a famine
in the land,
subsequent
to the earlier
famine
that
occurred
in the days
of Abraham.
Isaac
went
to
Abimelech
king
of the Philistines
at Gerar.
\VS{2}The
{\ND{Lord}}
appeared
to
Isaac and said,
“Do not
go down
to Egypt;
settle
down in the land
that I will point out to you.
\VS{3}Stay
in this
land.
Then I will be
with
you and will bless
you, for
I will give
all
these lands
to you and to your descendants,
and I will fulfill
the
solemn
promise I made
to your father
Abraham.
\VS{4}I will multiply
your descendants
so they will be as numerous as the stars
in the sky,
and I will give
them
all
these lands.
All
the nations
of the earth
will pronounce
blessings
on one another using
the name
of your descendants.
\VS{5}All this will come to pass
because Abraham
obeyed
me and kept
my charge,
my commandments,
my statutes,
and my laws.”
\VS{6}So Isaac
settled
in Gerar.
\par }{\PP \VS{7}When the men
of that place
asked
him about his wife,
he replied,
“She is my sister.”
He was afraid
to say,
“She is my wife,”
for he thought
to himself, “The men
of this place
will kill
me to get Rebekah
because
she is
very beautiful.”
\par }{\PP \VS{8}After Isaac had been
there
a long
time,
Abimelech
king
of the Philistines
happened to look out
a window
and observed
Isaac
caressing
his wife
Rebekah.
\VS{9}So Abimelech
summoned
Isaac
and said,
“She is really
your wife! Why
did you say,
‘She
is my sister’?” Isaac
replied,
“Because
I thought
someone might
kill me to get her.”
\par }{\PP \VS{10}Then Abimelech
exclaimed,
“What
in the world have you done
to us? One
of the men
might easily have
had sexual
relations with
your wife,
and you would have brought
guilt
on us!”
\VS{11}So Abimelech
commanded
all
the people,
“Whoever touches
this
man
or his wife
will surely be put to death.”
\par }{\PP \VS{12}When Isaac
planted
in that land,
he reaped
in the same year
a hundred
times what he had sown,
because the
{\ND{Lord}}
blessed him.
\VS{13}The man
became wealthy.
His influence
continued
to grow
until
he became very
prominent.
\VS{14}He had so many
sheep
and cattle
and such a great
household
of servants that the Philistines
became jealous of him.
\VS{15}So the Philistines
took dirt
and filled
up
all
the wells
that his father’s
servants
had dug
back in the days
of his father
Abraham.
\par }{\PP \VS{16}Then Abimelech
said
to
Isaac,
“Leave us and go
elsewhere, for
you have become much more powerful
than we are.”
\VS{17}So
Isaac
left
there
and settled
in
the Gerar
Valley.
\VS{18}Isaac
reopened
the wells
that had
been dug
back in the days
of his father
Abraham,
for the Philistines
had stopped
them up after
Abraham
died.
Isaac gave
these wells the same names
his father
had
given them.
\par }{\PP \VS{19}When Isaac’s
servants
dug
in the valley
and discovered
a well
with fresh flowing
water
there,
\VS{20}the herdsmen
of Gerar
quarreled
with
Isaac’s
herdsmen,
saying,
“The water
belongs to us!” So Isaac named
the well
Esek
because
they argued
with him about it.
\VS{21}His servants dug
another
well,
but they quarreled
over
it too,
so Isaac named
it Sitnah.
\VS{22}Then he moved
away from there
and dug
another
well.
They did not
quarrel
over
it, so
Isaac named
it Rehoboth,
saying,
“For
now
the {\ND{Lord}}
has made
room
for us, and we will prosper
in the land.”
\par }{\PP \VS{23}From there
Isaac went up
to Beer Sheba.
\VS{24}The
{\ND{Lord}}
appeared
to him
that night
and said,
“I am
the God
of your father
Abraham.
Do not
be afraid,
for
I am with
you. I
will bless
you and multiply
your descendants
for the sake of
my servant
Abraham.”
\VS{25}Then Isaac
built
an altar
there
and worshiped
the {\ND{Lord}}. He pitched
his tent
there,
and his servants
dug
a well.
\par }{\PP \VS{26}Now Abimelech
had come
to him
from Gerar
along with Ahuzzah
his friend
and Phicol
the commander
of his army.
\VS{27}Isaac
asked
them, “Why
have you come
to
me? You
hate
me and sent me away from you.”
\VS{28}They replied,
“We could plainly
see
that
the {\ND{Lord}}
is with
you. So we decided there should be
a pact
between us – between us and you. Allow us to make a treaty with you
\VS{29}so that you will not
do
us any harm,
just as we have not
harmed
you, but have always
treated
you well
before sending
you away
in peace.
Now
you
are blessed
by the
{\ND{Lord}}.”
\par }{\PP \VS{30}So Isaac held
a feast
for them and they celebrated.
\VS{31}Early
in the morning
the men made a treaty
with each
other.
Isaac
sent
them off; they separated
on good terms.
\par }{\PP \VS{32}That
day
Isaac’s
servants
came
and told
him about the well
they had
dug.
“We’ve found
water,”
they reported.
\VS{33}So
he named
it Shibah;
that is why
the name
of the city
has been Beer Sheba
to this
day.
\par }{\PP \VS{34}When
Esau
was forty
years
old, he married
Judith
the daughter
of Beeri
the Hittite,
as well as Basemath
the daughter
of Elon
the Hittite.
\VS{35}They caused
Isaac
and Rebekah
great anxiety.

\par }\Chap{27}{\PP \VerseOne{1}When
Isaac
was old
and his eyes
were so weak that he was almost
blind,
he called
his older
son
Esau
and said
to him,
“My son!” “Here I am!” Esau replied.
\VS{2}Isaac said,
“Since
I am so old,
I could
die
at any time.
\VS{3}Therefore,
take
your weapons
– your quiver
and your bow
– and go out
into the open fields
and hunt
down some wild game for me.
\VS{4}Then prepare
for me some tasty food,
the kind I love,
and bring
it to me. Then I will eat
it so that I may bless
you before
I die.”
\par }{\PP \VS{5}Now Rebekah
had been listening
while Isaac
spoke
to
his son
Esau.
When Esau
went
out to the open fields
to hunt down
some wild game
and bring it back,
\VS{6}Rebekah
said
to
her son
Jacob,
“Look,
I overheard
your father
tell
your brother
Esau,
\VS{7}‘Bring
me some wild game
and prepare
for me some tasty food.
Then I will eat
it and bless
you in the presence
of the {\ND{Lord}}
before
I die.’
\VS{8}Now
then,
my son,
do exactly
what
I tell you!
\VS{9}Go
to
the flock
and get
me two
of the best
young
goats.
I’ll prepare
them in a tasty
way for your father,
just
the way he loves them.
\VS{10}Then
you will take
it to your father.
Thus
he will eat
it and bless
you before
he dies.”
\par }{\PP \VS{11}“But Esau
my brother
is a hairy
man,”
Jacob
protested to
his mother
Rebekah,
“and I
have smooth
skin!
\VS{12}My father
may
touch
me! Then
he’ll think
I’m mocking
him and I’ll bring
a curse
on
myself instead
of a blessing.”
\VS{13}So his mother
told
him,
“Any curse
against you will fall on me, my son! Just
obey
me! Go
and get them for me!”
\par }{\PP \VS{14}So he went
and got
the goats and brought
them to his mother.
She prepared some tasty food,
just
the way his father
loved it.
\VS{15}Then
Rebekah
took
her older
son
Esau’s
best
clothes,
which
she had with
her in the house,
and put
them on her younger
son
Jacob.
\VS{16}She put
the skins
of the young
goats
on
his hands
and the smooth part
of his neck.
\VS{17}Then she handed
the tasty food
and the bread
she had
made
to her
son
Jacob.
\par }{\PP \VS{18}He went
to
his father
and said,
“My father!” Isaac replied,
“Here
I am. Which
are you,
my son?”
\VS{19}Jacob
said
to
his father,
“I am
Esau,
your firstborn.
I’ve done
as
you told
me.
Now
sit
up
and eat
some of my wild game
so that you can bless
me.”
\VS{20}But Isaac
asked his son,
“How
in the world
did you find
it so quickly,
my son?” “Because
the {\ND{Lord}}
your God
brought
it to me,”
he replied.
\VS{21}Then Isaac
said
to Jacob,
“Come closer
so I can
touch
you, my son,
and know for certain if
you really
are my son
Esau.”
\VS{22}So
Jacob
went over to
his father
Isaac,
who felt
him and said,
“The voice
is Jacob’s,
but the hands
are Esau’s.”
\VS{23}He did not
recognize
him because
his hands
were hairy,
like his brother
Esau’s
hands.
So Isaac blessed Jacob.
\VS{24}Then he asked,
“Are you
really
my son
Esau?” “I am,”
Jacob replied.
\VS{25}Isaac said,
“Bring some of the wild game
for me to eat,
my son.
Then
I will
bless
you.” So Jacob brought it to him, and he ate
it. He also brought
him wine,
and Isaac drank.
\VS{26}Then his father
Isaac
said
to him, “Come here
and kiss
me, my son.”
\VS{27}So Jacob went over
and kissed
him. When Isaac caught the scent
of his clothing,
he blessed
him, saying,
\par }{\Q “Yes,
my son
smells
\par }{\Q like the scent
of an open field
\par }{\Q which
the {\ND{Lord}}
has blessed.
\par }{\Q \VS{28}May God
give
you
\par }{\Q the dew
of the sky
\par }{\Q and the richness
of the earth,
\par }{\Q and plenty
of grain
and new wine.
\par }{\Q \VS{29}May peoples
serve
you
\par }{\Q and nations
bow
down to you.
\par }{\Q You will be
lord
over your brothers,
\par }{\Q and the sons
of your mother
will bow down
to you.

\par }{\Q May those who curse
you be cursed,
\par }{\Q and those who bless
you be blessed.”
\par }{\PP \VS{30}Isaac
had just
finished
blessing
Jacob,
and Jacob
had scarcely left
his
father’s
presence,
when his brother
Esau
returned
from the hunt.
\VS{31}He also
prepared
some tasty food
and brought
it to his
father.
Esau said
to him, “My father,
get up
and eat
some of your son’s
wild game.
Then you can bless me.”
\VS{32}His father
Isaac
asked, “Who
are you?” “I am
your firstborn
son,”
he replied,
“Esau!”
\VS{33}Isaac
began to shake
violently
and asked,
“Then who
else
hunted
game
and brought
it to me? I ate
all
of it just before
you arrived,
and I blessed
him. He will
indeed
be
blessed!”
\par }{\PP \VS{34}When Esau
heard
his father’s
words,
he wailed
loudly
and bitterly.
He said
to his father,
“Bless
me too,
my
father!”
\VS{35}But Isaac replied,
“Your brother
came
in here deceitfully
and took
away your blessing.”
\VS{36}Esau exclaimed,
“ ‘Jacob’
is the right name
for
him! He has tripped
me up
two times! He took away
my birthright,
and now,
look,
he has taken away
my blessing!” Then he asked,
“Have you not
kept back
a blessing for me?”
\par }{\PP \VS{37}Isaac
replied
to Esau,
“Look! I have made him lord
over you. I have made
all
his relatives
his servants
and provided
him with grain
and new wine.
What
is left that I can do
for you, my son?”
\VS{38}Esau
said
to
his
father,
“Do you have only that one
blessing,
my father? Bless
me
too!” Then Esau
wept
loudly.
\par }{\PP \VS{39}So
his father
Isaac
said
to him,
\par }{\Q “Indeed,
your home will be
\par }{\Q away from the richness
of the earth,
\par }{\Q and away from the dew
of the sky
above.
\par }{\Q \VS{40}You will live
by your sword
\par }{\Q but you will serve
your brother.
\par }{\Q When
you grow restless,
\par }{\Q you will tear off
his yoke
\par }{\Q from your neck.”
\par }{\PP \VS{41}So Esau
hated
Jacob
because
of the blessing
his father
had
given to his brother. Esau
said
privately, “The time
of mourning
for my father
is near; then I will kill
my brother
Jacob!”
\par }{\PP \VS{42}When Rebekah
heard what
her older
son
Esau
had said, she quickly summoned
her younger
son
Jacob
and told
him,
“Look,
your brother
Esau
is planning
to get revenge by killing you.
\VS{43}Now
then, my son,
do what I say. Run away immediately
to
my brother
Laban
in Haran.
\VS{44}Live
with
him for a little while until
your brother’s
rage
subsides.
\VS{45}Stay there until
your brother’s
anger
against you subsides
and he forgets what
you did
to him. Then I’ll send
someone to bring
you back from
there.
Why
should I lose
both
of you in one
day?”
\par }{\PP \VS{46}Then Rebekah
said to
Isaac,
“I am deeply
depressed
because of these daughters
of Heth.
If
Jacob
were to marry
one of these
daughters
of Heth
who live
in this land,
I would want to die!”

\par }\Chap{28}{\PP \VerseOne{1}So Isaac
called
for
Jacob
and blessed
him. Then he commanded
him, “You must not
marry
a Canaanite
woman!
\VS{2}Leave
immediately
for Paddan Aram! Go to the house
of Bethuel,
your mother’s
father,
and find
yourself a wife
there,
among the daughters
of Laban,
your mother’s
brother.
\VS{3}May the sovereign
God
bless
you! May he make you fruitful
and give you a multitude
of descendants! Then you will become
a large
nation.
\VS{4}May he give
you and your descendants
the
blessing
he gave to Abraham
so that
you may possess
the
land
God
gave
to Abraham,
the land where
you have been living as a temporary resident.”
\VS{5}So Isaac
sent
Jacob
on his way, and he went
to Paddan Aram,
to
Laban
son
of Bethuel
the Aramean
and brother
of Rebekah,
the mother
of Jacob
and Esau.
\par }{\PP \VS{6}Esau
saw
that
Isaac
had blessed
Jacob
and sent
him off
to Paddan Aram
to find
a wife
there.
As he blessed
him, Isaac commanded
him, “You must not
marry
a Canaanite
woman.”
\VS{7}Jacob
obeyed
his father
and mother
and left
for Paddan Aram.
\VS{8}Then Esau
realized
that
the Canaanite
women
were displeasing
to his father
Isaac.
\VS{9}So Esau
went
to
Ishmael
and married
Mahalath,
the sister
of Nebaioth
and daughter
of Abraham’s
son
Ishmael,
along with the wives he already had.
\par }{\SH Jacob’s Dream at Bethel
\par }{\PP \VS{10}Meanwhile
Jacob
left
Beer Sheba
and set
out
for Haran.
\VS{11}He reached
a certain place
where he decided to camp
because
the sun
had gone down.
He took
one of the stones
and placed
it near
his head.
Then he fell asleep
in that place
\VS{12}and had a dream.
He saw
a stairway
erected
on the earth
with its top
reaching
to the heavens.
The angels
of God
were going up
and coming down it
\VS{13}and the
{\ND{Lord}}
stood
at its top. He said,
“I am
the {\ND{Lord}}, the God
of your grandfather Abraham
and the God
of your father
Isaac.
I will give
you and your descendants
the ground
you
are lying
on.
\VS{14}Your descendants
will be like
the dust
of the earth,
and you will spread out
to the west,
east,
north,
and south.
All
the families
of the earth
will pronounce blessings
on one another using your name and that of your descendants.
\VS{15}I
am
with
you! I will protect
you wherever
you go
and will bring you back
to
this
land.
I will not
leave
you until
I have
done
what
I promised you!”
\par }{\PP \VS{16}Then Jacob
woke up
and thought, “Surely
the {\ND{Lord}}
is in this
place,
but I
did not
realize it!”
\VS{17}He was afraid
and said,
“What
an awesome
place
this
is! This
is nothing
else than
the house
of God! This
is the gate
of heaven!”
\par }{\PP \VS{18}Early
in the morning
Jacob
took
the stone
he had
placed
near
his head
and set
it up as a sacred stone.
Then he poured
oil
on
top of it.
\VS{19}He called
that
place
Bethel,
although
the former
name
of the town
was Luz.
\VS{20}Then Jacob
made
a vow,
saying,
“If
God
is with me
and protects
me on this
journey
I am
taking
and gives
me food
to eat
and clothing
to wear,
\VS{21}and I return
safely
to
my father’s
home,
then the
{\ND{Lord}}
will become my God.
\VS{22}Then this
stone
that
I have set
up as a sacred stone
will be
the house
of God,
and I
will surely
give
you back a tenth
of everything you give me.”

\par }\Chap{29}{\PP \VerseOne{1}So
Jacob
moved
on and came to the land
of the eastern people.
\VS{2}He saw
in the field
a well
with three
flocks
of sheep
lying
beside it,
because
the flocks
were watered
from
that well.
Now a large
stone
covered the mouth
of the well.
\VS{3}When all
the flocks
were gathered
there,
the shepherds would roll
the stone
off
the mouth
of the well
and water
the sheep.
Then they would put
the stone
back
in its place
over
the well’s
mouth.
\par }{\PP \VS{4}Jacob
asked
them, “My brothers,
where
are you
from?” They replied,
“We’re
from Haran.”
\VS{5}So he said
to them, “Do you know
Laban,
the grandson
of Nahor?” “We know
him,” they said.
\VS{6}“Is he well?” Jacob asked.
They replied,
“He is well.
Now look,
here comes
his daughter
Rachel
with
the sheep.”
\VS{7}Then Jacob said,
“Since
it is still
the middle of the day,
it is not
time
for the flocks
to be gathered.
You should water
the sheep
and then go
and let them graze some more.”
\VS{8}“We can’t,”
they said,
“until
all
the flocks
are gathered
and the stone
is rolled
off
the mouth
of the well.
Then we water
the sheep.”
\par }{\PP \VS{9}While he was still
speaking
with
them, Rachel
arrived
with
her father’s
sheep,
for
she was
tending them.
\VS{10}When
Jacob
saw
Rachel,
the daughter
of his uncle
Laban,
and the
sheep
of his uncle
Laban,
he
went over
and rolled
the
stone
off the
mouth
of the well
and watered
the
sheep
of his uncle
Laban.
\VS{11}Then Jacob
kissed
Rachel
and began to weep
loudly.
\VS{12}When Jacob
explained to Rachel
that
he was
a relative
of her father
and the son
of Rebekah,
she
ran
and told
her father.
\VS{13}When
Laban
heard
this news
about Jacob,
his sister’s
son,
he rushed out
to meet
him. He embraced
him and kissed
him and brought
him to
his house.
Jacob told
Laban how he was related to him.
\VS{14}Then Laban
said
to him, “You are indeed
my
own flesh
and blood.” So Jacob stayed
with
him for a month.
\par }{\PP \VS{15}Then Laban
said
to Jacob,
“Should you
work
for me for nothing
because
you are my relative? Tell
me what
your wages should be.”
\VS{16}(Now Laban
had two
daughters;
the older
one was named
Leah,
and the younger
one Rachel.
\VS{17}Leah’s
eyes
were tender,
but Rachel
had a lovely
figure
and beautiful
appearance.)
\VS{18}Since Jacob
had fallen in love
with Rachel,
he said,
“I’ll serve
you seven
years
in exchange for your younger
daughter
Rachel.”
\VS{19}Laban
replied,
“I’d rather
give
her to you than to another
man.
Stay
with me.”
\VS{20}So Jacob
worked
for seven
years
to acquire Rachel.
But they seemed
like only a few
days
to him because his love for her was so great.
\par }{\PP \VS{21}Finally Jacob
said
to
Laban,
“Give
me my wife,
for
my time
of service is up.
I want to have marital relations
with her.”
\VS{22}So Laban
invited
all
the people
of that place
and prepared
a feast.
\VS{23}In the evening
he brought
his daughter
Leah
to Jacob, and Jacob had marital relations
with her.
\VS{24}(Laban
gave
his female servant
Zilpah
to his daughter
Leah
to be her servant.)
\par }{\PP \VS{25}In the morning
Jacob discovered
it was
Leah! So Jacob said
to
Laban,
“What
in the world
have you done
to me! Didn’t
I work for you in exchange for Rachel? Why
have you tricked me?”
\VS{26}“It is not
our custom
here,”
Laban
replied,
“to give
the younger
daughter in marriage before
the firstborn.
\VS{27}Complete my older daughter’s
bridal
week.
Then we will give
you the younger one too,
in exchange
for seven
more
years
of work.”
\par }{\PP \VS{28}Jacob did
as Laban said. When
Jacob
completed
Leah’s
bridal week,
Laban gave
him his daughter
Rachel
to be his wife.
\VS{29}(Laban
gave
his female servant
Bilhah
to his daughter
Rachel
to be her servant.)
\VS{30}Jacob had marital relations
with
Rachel
as well.
He loved
Rachel
more than Leah,
so he worked
for
Laban for seven
more
years.
\par }{\SH The Family of Jacob
\par }{\PP \VS{31}When the
{\ND{Lord}}
saw
that
Leah
was unloved,
he enabled
her to become pregnant
while Rachel
remained childless.
\VS{32}So Leah
became pregnant
and gave birth
to a son.
She
named
him Reuben,
for
she said,
“The
{\ND{Lord}}
has looked
with pity on my oppressed condition.
Surely
my husband
will love
me now.”
\par }{\PP \VS{33}She became pregnant
again
and had
another son.
She said,
“Because
the {\ND{Lord}}
heard
that
I was unloved, he gave
me this
one too.”
So she
named
him Simeon.
\par }{\PP \VS{34}She became pregnant
again
and had
another son.
She said,
“Now
this time
my husband
will show me affection,
because
I have given birth
to three
sons
for
him.” That is why
he was named
Levi.
\par }{\PP \VS{35}She became pregnant
again
and had
another son.
She said,
“This time
I will praise
the {\ND{Lord}}.” That is why
she
named
him Judah.
Then she stopped
having children.

\par }\Chap{30}{\PP \VerseOne{1}When Rachel
saw
that
she could not
give Jacob
children,
she became jealous
of her sister.
She said
to Jacob,
“Give
me children
or
I’ll
die!”
\VS{2}Jacob
became furious
with Rachel
and exclaimed,
“Am
I in the place
of God,
who has
kept
you from
having children?”
\VS{3}She replied,
“Here
is my servant
Bilhah! Have sexual relations
with
her so that she can bear children
for me
and I
can have a family through her.”
\par }{\PP \VS{4}So Rachel gave
him her servant
Bilhah
as a wife,
and Jacob
had marital relations
with her.
\VS{5}Bilhah
became pregnant
and gave
Jacob
a son.
\VS{6}Then Rachel
said,
“God
has vindicated
me. He has responded
to my prayer
and given
me a son.”
That is why
she
named
him Dan.
\par }{\PP \VS{7}Bilhah,
Rachel’s
servant,
became pregnant
again
and gave
Jacob
another
son.
\VS{8}Then Rachel
said,
“I have fought a desperate struggle
with
my sister,
but
I have won.”
So she
named
him Naphtali.
\par }{\PP \VS{9}When Leah
saw
that
she had stopped
having children, she
gave her
servant
Zilpah
to Jacob
as a wife.
\VS{10}Soon
Leah’s
servant
Zilpah
gave Jacob
a son.
\VS{11}Leah
said,
“How fortunate!” So she named
him Gad.
\par }{\PP \VS{12}Then Leah’s
servant
Zilpah
gave
Jacob
another
son.
\VS{13}Leah
said,
“How happy
I am, for
women
will call
me happy!” So she named
him Asher.
\par }{\PP \VS{14}At
the time
of the wheat
harvest
Reuben
went
out and found
some mandrake
plants in a field
and brought
them to
his mother
Leah.
Rachel
said
to
Leah,
“Give
me
some of your son’s
mandrakes.”
\VS{15}But Leah replied, “Wasn’t it enough
that you’ve taken away
my husband? Would you take away
my son’s
mandrakes
too?” “All right,” Rachel
said,
“he may sleep
with
you tonight
in exchange
for your son’s
mandrakes.”
\VS{16}When Jacob
came
in from
the fields
that evening,
Leah
went out
to meet
him and said,
“You must sleep
with
me because
I have paid
for your services
with my son’s
mandrakes.”
So he had marital relations
with
her that night.
\VS{17}God
paid attention
to
Leah;
she became pregnant
and gave
Jacob
a son
for the fifth time.
\VS{18}Then Leah
said,
“God
has granted
me a reward
because I gave
my servant
to my husband
as a wife.” So she named
him Issachar.
\par }{\PP \VS{19}Leah
became pregnant
again
and gave
Jacob
a son
for the sixth time.
\VS{20}Then Leah
said,
“God
has given
me a good
gift.
Now
my husband
will honor me
because
I have given
him six
sons.”
So she named
him Zebulun.
\par }{\PP \VS{21}After
that she gave birth
to a daughter
and named
her Dinah.
\par }{\PP \VS{22}Then God
took
note
of Rachel.
He
paid attention
to
her and enabled
her to become pregnant.
\VS{23}She became pregnant
and gave birth
to a son.
Then she said,
“God
has taken away
my shame.”
\VS{24}She
named
him Joseph,
saying,
“May
the {\ND{Lord}}
give me yet another
son.”
\par }{\SH The Flocks of Jacob
\par }{\PP \VS{25}After
Rachel
had
given birth
to Joseph,
Jacob
said
to
Laban,
“Send
me on my way so that I can go
home
to my own country.
\VS{26}Let
me take my wives
and my children
whom
I have acquired by working
for you. Then I’ll depart,
because
you
know
how hard
I’ve worked for you.”
\par }{\PP \VS{27}But Laban
said
to him,
“If
I have found
favor
in your sight,
please
stay here, for I have learned by divination
that the
{\ND{Lord}}
has blessed
me on account of you.”
\VS{28}He added,
“Just name
your wages
– I’ll pay whatever you want.”
\par }{\PP \VS{29}“You
know
how I have
worked
for you,” Jacob replied, “and how well
your livestock
have fared under my care.
\VS{30}Indeed,
you had little
before
I arrived, but now your
possessions have increased
many
times over. The
{\ND{Lord}}
has blessed
you wherever
I worked. But now,
how long
must it be
before I
do
something for my own family
too?”
\par }{\PP \VS{31}So Laban asked, “What
should I give
you?” “You don’t need to give
me a thing,” Jacob
replied, “but if
you agree
to this
one condition, I will continue to care
for your flocks
and protect them:
\VS{32}Let me walk among
all
your flocks
today
and remove
from them every
speckled
or spotted
sheep,
every
dark-colored
lamb,
and the spotted
or speckled
goats.
These animals will be
my wages.
\VS{33}My integrity
will testify
for me later on.
When you come
to verify that I’ve taken only the wages
we agreed on, if I have in my possession any
goat
that
is not
speckled
or spotted
or any sheep
that
is not dark-colored,
it will be considered stolen.”
\VS{34}“Agreed!” said
Laban,
“It will be
as you say.”
\par }{\PP \VS{35}So
that day
Laban removed
the male goats
that were streaked
or spotted,
all
the female goats
that were speckled
or spotted
(all
that had
any white
on them), and all
the dark-colored
lambs,
and put
them in the care
of his
sons.
\VS{36}Then he separated
them from Jacob
by a three-day
journey,
while Jacob
was taking care
of the rest
of Laban’s
flocks.
\par }{\PP \VS{37}But Jacob
took
fresh-cut
branches
from poplar,
almond,
and plane trees.
He made
white
streaks
by peeling them, making the white
inner wood in
the branches
visible.
\VS{38}Then he set up
the peeled
branches
in all the watering
troughs
where
the flocks
came
to drink.
He set up the branches
in front of
the flocks
when they were in heat
and came
to drink.
\VS{39}When the sheep
mated
in front of the branches,
they gave birth
to young that were streaked
or speckled
or spotted.
\VS{40}Jacob
removed
these lambs,
but he made
the rest of the flock
face
the streaked
and completely
dark-colored
animals in Laban’s
flock.
So he made separate flocks
for himself
and did not
mix them with Laban’s
flocks.
\VS{41}When
the stronger
females
were in heat,
Jacob
would set up
the branches
in the troughs
in front
of the flock,
so they would mate
near the branches.
\VS{42}But if the animals were weaker,
he did not
set
the branches there. So the weaker
animals ended up belonging to Laban
and the stronger
animals to Jacob.
\VS{43}In this way Jacob became
extremely
prosperous.
He owned
large
flocks,
male
and female
servants,
camels,
and donkeys.

\par }\Chap{31}{\PP \VerseOne{1}Jacob heard
that Laban’s
sons
were complaining, “Jacob
has taken
everything
that belonged
to our father! He has gotten
rich
at our father’s expense!”
\VS{2}When
Jacob
saw
the look
on Laban’s
face,
he could
tell his attitude
toward him had changed.
\par }{\PP \VS{3}The
{\ND{Lord}}
said
to
Jacob,
“Return
to
the land
of your fathers
and to your relatives.
I will be
with you.”
\VS{4}So Jacob
sent
a message
for Rachel
and Leah
to come to the field
where his flocks were.
\VS{5}There he said
to them,
“I
can
tell that
your father’s
attitude
toward me has changed,
but the God
of my father
has been
with me.
\VS{6}You
know
that
I’ve
worked
for your father as hard as I could,
\VS{7}but your father
has humiliated
me and changed
my wages
ten
times.
But God
has not
permitted
him to do me
any harm.
\VS{8}If
he said, ‘The speckled
animals will be
your wage,’
then the entire
flock
gave birth
to speckled offspring.
But if
he said,
‘The streaked
animals will be
your wage,’
then the entire
flock
gave birth
to streaked offspring.
\VS{9}In this way God
has snatched
away your father’s
livestock
and given them to me.
\par }{\PP \VS{10}“Once during
breeding
season
I saw
in a dream
that the male goats
mating
with the flock
were streaked,
speckled,
and spotted.
\VS{11}In the dream
the angel
of God
said to me,
‘Jacob!’ ‘Here I am!’ I replied.
\VS{12}Then he said,
‘Observe
that all
the male goats
mating
with the flock
are streaked,
speckled,
or spotted,
for
I have observed
all
that
Laban
has done to you.
\VS{13}I am
the God
of Bethel,
where
you anointed
the sacred stone
and made a vow
to me. Now
leave
this
land
immediately
and return
to
your native
land.’ ”
\par }{\PP \VS{14}Then
Rachel
and Leah
replied
to him, “Do we still
have any portion
or inheritance
in our father’s
house?
\VS{15}Hasn’t
he treated
us like foreigners? He not only sold
us, but completely wasted
the money paid for us!
\VS{16}Surely
all
the wealth
that
God
snatched
away from our father
belongs to us and to our children.
So now
do
everything
God
has
told you.”
\par }{\PP \VS{17}So
Jacob
immediately put
his children
and his wives
on
the camels.
\VS{18}He took away
all
the livestock
he had acquired
in Paddan Aram
and all
his moveable property
that
he had accumulated.
Then he set out
toward the land
of Canaan
to return to
his father
Isaac.
\par }{\PP \VS{19}While Laban
had gone
to shear
his sheep,
Rachel
stole
the household idols
that
belonged to her father.
\VS{20}Jacob
also deceived
Laban
the Aramean
by not telling
him that
he was leaving.
\VS{21}He left
with all
he owned.
He quickly crossed
the Euphrates
River and headed
for the hill country
of Gilead.
\par }{\PP \VS{22}Three days
later
Laban
discovered
Jacob
had left.
\VS{23}So he took
his relatives
with
him and pursued
Jacob for seven
days.
He caught
up with him in the hill country
of Gilead.
\VS{24}But God
came
to
Laban
the Aramean
in a dream
at night
and warned
him, “Be careful
that you
neither
bless nor curse
Jacob.”
\par }{\PP \VS{25}Laban
overtook
Jacob,
and when
Jacob
pitched
his tent
in the hill country
of Gilead,
Laban
and his relatives
set up camp there too.
\VS{26}“What
have you done?” Laban
demanded of Jacob.
“You’ve
deceived me and carried away my daughters
as if they were captives
of war!
\VS{27}Why
did you run away
secretly
and deceive
me? Why didn’t
you tell
me so I could send
you off with a celebration
complete with singing,
tambourines,
and harps?
\VS{28}You didn’t even allow me
to kiss
my daughters
and my grandchildren
good-bye. You have acted
foolishly!
\VS{29}I have
the power
to do
you
harm,
but the God
of your father
told
me last night,
‘Be careful
that you neither
bless nor curse
Jacob.’
\VS{30}Now
I understand that you have gone
away
because
you longed
desperately
for your father’s
house.
Yet
why
did you steal
my gods?”
\par }{\PP \VS{31}“I left secretly
because
I was afraid!” Jacob
replied
to Laban.
“I thought you might
take your daughters
away
from me by force.
\VS{32}Whoever has
taken
your gods
will be put to death! In the presence
of our relatives
identify
whatever
is yours
and take
it.” (Now Jacob
did not
know
that
Rachel
had stolen them.)
\par }{\PP \VS{33}So Laban
entered
Jacob’s
tent,
and Leah’s
tent,
and the tent
of the two
female servants,
but he did not
find
the idols. Then he left
Leah’s
tent
and entered
Rachel’s.
\VS{34}(Now Rachel
had taken
the idols
and put
them inside her camel’s
saddle and sat
on
them.) Laban
searched
the whole
tent,
but did not
find them.
\VS{35}Rachel said
to
her father,
“Don’t
be angry,
my lord.
I cannot
stand up
in your presence
because
I am having
my period.”
So he searched
thoroughly, but did not
find
the idols.
\par }{\PP \VS{36}Jacob
became angry
and argued
with Laban.
“What
did I do wrong?” he demanded
of Laban. “What
sin
of mine prompted you to chase
after
me in hot pursuit?
\VS{37}When
you searched
through all
my goods,
did you find
anything
that belonged to you? Set
it here before
my relatives
and yours,
and let them settle the dispute
between
the two of us!
\par }{\PP \VS{38}“I
have been with you for the past
twenty
years.
Your ewes
and female goats
have not
miscarried,
nor
have I
eaten
rams
from your flocks.
\VS{39}Animals torn
by wild beasts I never
brought
to
you; I
always absorbed
the loss myself. You always made me pay
for every missing
animal, whether it was taken
by day
or
at night.
\VS{40}I was consumed
by scorching heat
during the day
and by piercing cold
at night,
and I went without sleep.
\VS{41}This
was my lot for twenty
years
in your house: I worked like a slave for you – fourteen years for your two daughters and six years for your flocks, but you changed my wages ten times!
\VS{42}If
the God
of my father
– the God
of Abraham,
the one whom
Isaac
fears –
had not been
with me, you would certainly
have
sent
me away empty-handed! But God
saw
how I was oppressed
and how hard I worked,
and he rebuked
you last night.”
\par }{\PP \VS{43}Laban
replied
to
Jacob,
“These women
are my daughters,
these children
are my grandchildren,
and these flocks
are my flocks.
All
that
you
see
belongs to me. But how
can I harm
these
daughters
of mine today
or
the children
to whom
they have given birth?
\VS{44}So now,
come, let’s
make a formal agreement,
you and I,
and it will be
proof that we have made peace.”
\par }{\PP \VS{45}So Jacob
took
a stone
and set
it up
as a memorial pillar.
\VS{46}Then he
said
to his relatives,
“Gather
stones.”
So they brought
stones
and put
them in
a pile.
They ate
there
by
the pile of stones.
\VS{47}Laban
called
it Jegar Sahadutha,
but Jacob
called
it Galeed.
\par }{\PP \VS{48}Laban
said,
“This
pile
of stones is a witness
of our agreement today.”
That is why
it was called
Galeed.
\VS{49}It was also called Mizpah
because
he said,
“May the
{\ND{Lord}}
watch
between
us when
we are
out
of sight
of one another.
\VS{50}If
you mistreat
my daughters
or if
you take
wives
besides
my daughters,
although no
one
else is with
us, realize
that God
is witness
to your actions.”
\par }{\PP \VS{51}“Here
is this pile
of stones and this
pillar
I have set
up between
me and you,” Laban
said
to Jacob.
\VS{52}“This
pile
of stones
and the pillar
are reminders
that I
will not
pass
beyond
this
pile
to come to harm you and that you
will not
pass
beyond
this
pile
and this
pillar
to come to harm me.
\VS{53}May the God
of Abraham
and the god
of Nahor,
the gods
of their father,
judge
between
us.” Jacob
took an oath
by the God whom his father
Isaac
feared.
\VS{54}Then Jacob
offered
a sacrifice
on the mountain
and invited
his relatives
to eat
the meal.
They ate
the meal
and spent
the night on the mountain.
\par }{\PP \VS{55}Early
in the morning
Laban
kissed
his grandchildren
and his daughters
goodbye and blessed
them. Then Laban
left
and returned
home.

\par }\Chap{32}{\PP \VerseOne{1}So Jacob
went
on his way
and the angels
of God
met him.
\VS{2}When Jacob
saw
them, he exclaimed, “This
is the camp
of God!” So
he named
that place
Mahanaim.
\par }{\PP \VS{3}Jacob
sent
messengers
on ahead
to
his brother
Esau
in the land
of Seir,
the region
of Edom.
\VS{4}He commanded
them, “This is what
you must say
to my lord
Esau: ‘This is what
your servant
Jacob
says: I have been staying
with
Laban
until
now.
\VS{5}I have oxen,
donkeys,
sheep,
and male and female
servants.
I have sent
this message to inform
my lord,
so that I may find
favor
in your sight.’ ”
\par }{\PP \VS{6}The messengers
returned
to
Jacob
and said,
“We went
to
your brother
Esau.
He is coming to meet
you and has four
hundred
men
with him.”
\VS{7}Jacob
was very
afraid
and upset.
So he divided
the people
who
were with
him into two
camps,
as well
as the flocks,
herds,
and camels.
\VS{8}“If
Esau
attacks
one camp,”
he thought, “then the other
camp
will be
able
to escape.”
\par }{\PP \VS{9}Then Jacob
prayed, “O God
of my father
Abraham,
God
of my father
Isaac,
O
{\ND{Lord}}, you said
to me,
‘Return
to your land
and to your relatives
and I will make you prosper.’
\VS{10}I am not worthy
of all
the faithful love
you have shown
your servant.
With only my walking stick
I crossed
the Jordan,
but now
I have become
two
camps.
\VS{11}Rescue
me,
I pray,
from the hand
of my brother
Esau,
for
I am
afraid
he will come
and attack
me, as well as the mothers
with their children.
\VS{12}But you
said,
‘I will certainly make you prosper
and will make
your descendants
like the sand
on the seashore,
too numerous
to count.’ ”
\par }{\PP \VS{13}Jacob stayed
there
that night.
Then he
sent
as a gift
to his brother
Esau
\VS{14}two hundred
female goats
and twenty
male goats,
two hundred
ewes
and twenty
rams,
\VS{15}thirty
female
camels
with their young,
forty
cows
and ten
bulls,
and twenty
female donkeys
and ten
male donkeys.
\VS{16}He entrusted
them to
his servants,
who divided them into herds.
He told
his servants,
“Pass over
before
me, and keep
some distance
between
one herd and the next.”
\VS{17}He instructed
the servant leading the first
herd, “When
my brother
Esau
meets
you and asks,
‘To whom
do you
belong? Where
are you going? Whose
herds are you driving?’
\VS{18}then you must say, ‘They belong to your servant
Jacob.
They have been sent
as
a gift
to my lord
Esau.
In fact
Jacob himself
is
behind us.’ ”
\par }{\PP \VS{19}He also
gave these instructions
to the
second
and third
servants, as well
as all
those
who were following
the herds,
saying,
“You must say
the same
thing
to
Esau
when
you meet him.
\VS{20}You must also
say,
‘In fact
your servant
Jacob
is behind
us.’ ” Jacob thought, “I will first appease
him by
sending a gift
ahead
of me. After
that
I will meet
him. Perhaps
he will accept me.”
\VS{21}So
the gifts
were sent on
ahead
of him while
he spent
that night
in the camp.
\par }{\PP \VS{22}During
the night
Jacob quickly took
his two
wives,
his two
female servants,
and his eleven
sons
and crossed
the ford
of the Jabbok.
\VS{23}He took
them
and sent them across
the stream
along
with all his possessions.
\VS{24}So Jacob
was left
alone.
Then a man wrestled
with
him until
daybreak.
\VS{25}When the man saw
that
he could
not
defeat Jacob, he struck
the socket
of his hip
so the socket
of Jacob’s
hip
was dislocated
while he wrestled
with him.
\par }{\PP \VS{26}Then the man said,
“Let me
go,
for
the dawn
is breaking.” “I will not
let
you go,”
Jacob replied, “unless
you bless me.”
\VS{27}The man asked
him, “What
is your name?” He answered,
“Jacob.”
\VS{28}“No
longer
will your name
be Jacob,”
the man told
him, “but
Israel,
because
you have fought
with
God
and with
men
and have prevailed.”
\par }{\PP \VS{29}Then Jacob
asked,
“Please tell
me
your name.” “Why
do you ask
my name?” the man
replied. Then he blessed
Jacob there.
\VS{30}So
Jacob
named
the place
Peniel,
explaining, “Certainly
I have seen
God
face
to
face
and have survived.”
\par }{\PP \VS{31}The sun
rose
over him as he crossed over
Penuel,
but he was limping
because of his hip.
\VS{32}That is why
to this
day
the Israelites
do not
eat
the
sinew
which
is attached
to the
socket
of the hip,
because
he struck
the socket
of Jacob’s
hip
near the attached sinew.

\par }\Chap{33}{\PP \VerseOne{1}Jacob
looked
up and saw
that Esau
was coming
along with
four
hundred
men.
So he divided
the children
among Leah,
Rachel,
and the two
female servants.
\VS{2}He put
the servants
and their children
in front,
with Leah
and her children
behind
them,
and Rachel
and Joseph
behind them.
\VS{3}But Jacob himself went
on ahead
of them, and he bowed
toward the ground
seven
times
as
he approached
his brother.
\VS{4}But Esau
ran
to meet
him, embraced
him,
hugged
his neck,
and kissed
him. Then they both wept.
\VS{5}When Esau looked
up and saw
the women
and the children,
he asked,
“Who
are these
people with you?” Jacob replied,
“The children
whom
God
has graciously
given your servant.”
\VS{6}The female servants
came forward
with their children
and bowed down.
\VS{7}Then Leah
came forward
with her children
and they bowed
down. Finally
Joseph
and Rachel
came forward and bowed down.
\par }{\PP \VS{8}Esau then asked, “What
did you intend
by sending
all
these
herds to meet
me?” Jacob replied,
“To find
favor
in your sight,
my lord.”
\VS{9}But Esau
said,
“I have
plenty,
my brother.
Keep what belongs to you.”
\VS{10}“No,
please
take them,” Jacob
said. “If
I have found
favor
in your sight,
accept
my gift
from my hand.
Now
that
I
have seen
your face
and you have accepted
me, it is as if I have seen
the face
of God.
\VS{11}Please
take
my present
that
was brought
to you, for
God
has been generous
to me and I have
all
I need.” When Jacob urged
him, he took it.
\par }{\PP \VS{12}Then Esau
said,
“Let’s
be on our way! I will go
in front of you.”
\VS{13}But Jacob said
to him,
“My lord
knows
that
the children
are young,
and that I have to look after
the sheep
and cattle
that are nursing
their young. If
they are driven too hard
for even
a single
day,
all
the animals
will die.
\VS{14}Let
my lord
go on
ahead
of his servant.
I
will travel
more slowly,
at the pace
of the herds
and the children,
until
I come
to
my lord
at Seir.”
\par }{\PP \VS{15}So Esau
said,
“Let
me leave
some
of my men
with
you.” “Why
do that?” Jacob replied. “My lord
has already
been kind
enough to me.”
\par }{\PP \VS{16}So
that same
day
Esau
made his way
back to Seir.
\VS{17}But Jacob
traveled
to Succoth
where he built
himself a house
and made
shelters
for his livestock.
That is why
the place
was called
Succoth.
\par }{\PP \VS{18}After he left
Paddan Aram,
Jacob
came
safely
to the city
of Shechem
in the land
of Canaan,
and he camped
near
the city.
\VS{19}Then he purchased
the portion
of the field
where
he had pitched
his tent;
he bought
it from the sons
of Hamor,
Shechem’s
father,
for a hundred
pieces of money.
\VS{20}There
he set
up an altar
and called
it “The God
of Israel
is God.”

\par }\Chap{34}{\PP \VerseOne{1}Now Dinah,
Leah’s
daughter
whom
she bore
to Jacob,
went to meet
the young women
of the land.
\VS{2}When
Shechem
son
of Hamor
the Hivite,
who ruled that area,
saw
her,
he grabbed
her, forced himself on her, and sexually
assaulted her.
\VS{3}Then he became very attached
to Dinah,
Jacob’s
daughter.
He fell in love
with the young
woman and spoke
romantically to her.
\VS{4}Shechem
said
to
his father
Hamor,
“Acquire
this
young girl
as my wife.”
\VS{5}When Jacob
heard
that
Shechem had violated
his daughter
Dinah,
his sons
were with
the livestock
in the field.
So Jacob
remained silent
until
they came in.
\par }{\PP \VS{6}Then Shechem’s
father
Hamor
went
to speak
with
Jacob about Dinah.
\VS{7}Now Jacob’s
sons
had come
in from
the field
when they heard
the news. They
were offended
and very
angry
because
Shechem had disgraced
Israel
by sexually assaulting
Jacob’s
daughter,
a crime that should not
be committed.
\par }{\PP \VS{8}But Hamor
made
this appeal to
them: “My son
Shechem
is in love
with your daughter.
Please
give
her to him as his wife.
\VS{9}Intermarry
with
us. Let
us marry your daughters,
and take
our daughters as wives for yourselves.
\VS{10}You may live
among
us, and the land
will be
open
to you. Live
in it, travel freely
in it, and acquire property in it.”
\par }{\PP \VS{11}Then Shechem
said
to
Dinah’s father
and brothers,
“Let me find
favor
in your sight,
and whatever
you require
of me I’ll give.
\VS{12}You can make
the bride price
and the gift
I must bring very
expensive, and I’ll give
whatever
you ask
of me.
Just give
me the
young
woman as my wife!”
\par }{\PP \VS{13}Jacob’s
sons
answered
Shechem
and his father
Hamor
deceitfully
when they spoke
because Shechem had
violated
their sister
Dinah.
\VS{14}They said
to
them, “We cannot
give
our sister
to a man
who
is not circumcised,
for
it would be a disgrace to us.
\VS{15}We will give you our consent
on this
one condition: You must
become
like us
by circumcising
all
your males.
\VS{16}Then we will give
you our daughters
to marry,
and we will take
your daughters
as wives for ourselves, and we will live
among you
and become
one
people.
\VS{17}But if
you do not
agree
to our terms by being circumcised,
then
we will take
our sister
and depart.”
\par }{\PP \VS{18}Their offer pleased
Hamor
and his son
Shechem.
\VS{19}The young man
did not
delay
in doing
what
they asked because
he wanted
Jacob’s
daughter
Dinah badly. (Now he was more important than
anyone
in his father’s
household.)
\VS{20}So Hamor
and his son
Shechem
went
to
the gate
of their city
and spoke
to
the men
of their city,
\VS{21}“These
men
are at peace
with
us. So let them
live
in the land
and travel freely
in it, for the
land
is
wide
enough
for them. We will take
their daughters
for wives,
and we will give
them our daughters to marry.
\VS{22}Only
on this
one condition
will these men
consent to live
with
us and become
one
people: They demand that every
male
among us be circumcised
just
as they
are circumcised.
\VS{23}If we do so, won’t their
livestock,
their property,
and all
their
animals
become ours? So let’s consent
to their demand,
so they will live
among us.”
\par }{\PP \VS{24}All
the men who assembled
at the city
gate
agreed
with Hamor
and his son
Shechem.
Every
male
who assembled
at the city
gate
was circumcised.
\VS{25}In three
days,
when
they were still
in pain,
two
of Jacob’s
sons,
Simeon
and Levi,
Dinah’s
brothers,
each
took
his sword
and went
to
the unsuspecting
city
and slaughtered
every
male.
\VS{26}They killed
Hamor
and his
son
Shechem
with the sword,
took
Dinah
from Shechem’s
house,
and left.
\VS{27}Jacob’s
sons
killed
them and looted
the city
because
their sister
had been violated.
\VS{28}They took
their flocks,
herds,
and donkeys,
as well as everything
in the city
and in the surrounding fields.
\VS{29}They captured
as plunder
all
their wealth,
all
their little ones,
and their wives,
including everything
in the houses.
\par }{\PP \VS{30}Then Jacob
said
to
Simeon
and Levi,
“You have brought ruin
on
me by making me a foul odor
among the inhabitants
of the land
– among the Canaanites
and the Perizzites.
I
am few
in number;
they will join forces
against
me and attack
me, and both I
and my family
will be destroyed!”
\VS{31}But Simeon and Levi replied, “Should
he treat
our sister
like a common prostitute?”

\par }\Chap{35}{\PP \VerseOne{1}Then God
said
to Jacob,
“Go
up
at once to Bethel
and live
there.
Make
an altar
there
to God,
who appeared
to
you when you fled
from your brother
Esau.”
\VS{2}So Jacob
told
his household
and all
who
were with
him, “Get rid
of the foreign
gods
you have
among
you. Purify
yourselves
and change
your clothes.
\VS{3}Let us go up
at once to Bethel.
Then I will make
an altar
there
to God,
who responded
to me in my time
of distress
and has been
with me
wherever
I went.”
\par }{\PP \VS{4}So they gave
Jacob
all
the foreign
gods
that
were in their possession
and the rings
that
were in their ears.
Jacob
buried
them under
the oak
near
Shechem
\VS{5}and they
started on their journey.
The surrounding
cities
were afraid
of God,
and they did not
pursue
the sons
of Jacob.
\par }{\PP \VS{6}Jacob
and all
those who
were with
him arrived
at Luz
(that
is,
Bethel) in the land
of Canaan.
\VS{7}He built
an altar
there
and named
the place
El
Bethel
because
there
God
had revealed
himself to
him when he was fleeing
from
his brother.
\VS{8}(Deborah,
Rebekah’s
nurse,
died
and was buried
under
the oak
below
Bethel;
thus it was named
Oak of Weeping.)
\par }{\PP \VS{9}God
appeared
to
Jacob
again
after he returned
from Paddan Aram
and blessed him.
\VS{10}God
said
to him, “Your name
is Jacob,
but your name
will no longer
be called
Jacob;
Israel
will be
your name.”
So
God named
him Israel.
\VS{11}Then God
said
to him, “I am
the sovereign
God.
Be fruitful
and multiply! A nation
– even a company
of nations
– will descend
from
you; kings
will be among your descendants!
\VS{12}The land
I gave
to Abraham
and Isaac
I will give
to you. To your descendants
I will
also
give
this land.”
\VS{13}Then God
went up
from the place
where
he spoke
with him.
\VS{14}So Jacob
set up
a sacred
stone
pillar
in the place
where
God spoke
with
him. He poured out
a drink offering
on it, and then he poured
oil
on it.
\VS{15}Jacob
named
the place
where
God
spoke
with
him Bethel.
\par }{\PP \VS{16}They traveled
on from Bethel,
and when Ephrath
was still
some distance away,
Rachel
went
into labor –
and her labor
was hard.
\VS{17}When
her labor was at its hardest,
the midwife
said
to her, “Don’t
be afraid,
for
you are having
another
son.”
\VS{18}With her dying
breath, she named
him Ben-Oni.
But his father
called
him Benjamin instead.
\VS{19}So Rachel
died
and was buried
on the way
to Ephrath
(that
is, Bethlehem).
\VS{20}Jacob
set
up a marker
over
her grave;
it is
the Marker
of Rachel’s
Grave
to
this day.
\par }{\PP \VS{21}Then Israel
traveled
on and pitched
his tent
beyond
Migdal Eder.
\VS{22}While
Israel
was living in that land,
Reuben
had sexual
relations with Bilhah,
his father’s
concubine,
and Israel
heard
about it.
\par }{\PP Jacob
had twelve
sons:
\par }{\Q \VS{23}The sons
of Leah
were Reuben,
Jacob’s
firstborn,
as well as Simeon,
Levi,
Judah,
Issachar,
and Zebulun.
\par }{\Q \VS{24}The sons
of Rachel
were Joseph
and Benjamin.
\par }{\Q \VS{25}The sons
of Bilhah,
Rachel’s
servant,
were Dan
and Naphtali.
\par }{\Q \VS{26}The sons
of Zilpah,
Leah’s
servant,
were Gad
and Asher.
\par }{\PP These
were the sons
of Jacob
who
were born
to him in Paddan Aram.
\par }{\PP \VS{27}So Jacob
came
back to
his father
Isaac
in Mamre,
to Kiriath Arba
(that is,
Hebron), where
Abraham
and Isaac
had stayed.
\VS{28}Isaac
lived to be 180
years old.
\VS{29}Then Isaac
breathed his last
and joined
his ancestors.
He died
an old man
who had lived a full
life.
His sons
Esau
and Jacob
buried him.

\par }\Chap{36}{\PP \VerseOne{1}What follows is the account
of Esau
(also known as Edom).
\par }{\PP \VS{2}Esau
took
his wives
from the Canaanites: Adah
the daughter
of Elon
the Hittite,
and Oholibamah
the daughter
of Anah
and granddaughter
of Zibeon
the Hivite,
\VS{3}in addition to Basemath
the daughter
of Ishmael
and sister
of Nebaioth.
\par }{\PP \VS{4}Adah
bore
Eliphaz
to Esau,
Basemath
bore
Reuel,
\VS{5}and Oholibamah
bore
Jeush,
Jalam,
and Korah.
These
were the sons
of Esau
who
were born
to him in the land
of Canaan.
\par }{\PP \VS{6}Esau
took
his wives,
his sons,
his daughters,
all
the people
in his household,
his livestock,
his animals,
and all
his possessions
which
he had acquired
in the land
of Canaan
and went
to
a land
some distance away from Jacob
his brother
\VS{7}because
they had
too many
possessions
to be able to stay
together
and the land
where they had settled
was not
able
to support
them because
of their livestock.
\VS{8}So Esau
(also known as
Edom) lived in
the hill country
of Seir.
\par }{\PP \VS{9}This
is the account
of Esau,
the father
of the Edomites,
in the hill country
of Seir.
\par }{\PP \VS{10}These
were the names
of Esau’s
sons:
\par }{\PP Eliphaz,
the son
of Esau’s
wife
Adah,
and Reuel,
the son
of Esau’s
wife
Basemath.
\par }{\PP \VS{11}The sons
of Eliphaz
were:
\par }{\PP Teman,
Omar,
Zepho,
Gatam,
and Kenaz.
\par }{\PP \VS{12}Timna,
a concubine
of Esau’s
son
Eliphaz,
bore
Amalek
to Eliphaz.
These
were the sons
of Esau’s
wife
Adah.
\par }{\PP \VS{13}These
were the sons
of Reuel: Nahath,
Zerah,
Shammah,
and Mizzah.
These
were the sons
of Esau’s
wife
Basemath.
\par }{\PP \VS{14}These
were the sons
of Esau’s
wife
Oholibamah
the daughter
of Anah
and granddaughter
of Zibeon: She bore
Jeush,
Jalam,
and Korah
to Esau.
\par }{\PP \VS{15}These
were the chiefs
among the descendants
of Esau,
the sons
of Eliphaz,
Esau’s
firstborn: chief
Teman,
chief
Omar,
chief
Zepho,
chief
Kenaz,
\VS{16}chief
Korah,
chief
Gatam,
chief
Amalek.
These
were the chiefs
descended from Eliphaz
in the land
of Edom;
these
were the sons
of Adah.
\par }{\PP \VS{17}These
were the sons
of Esau’s
son
Reuel: chief
Nahath,
chief
Zerah,
chief
Shammah,
chief
Mizzah.
These
were the chiefs
descended from Reuel
in the land
of Edom;
these
were the sons
of Esau’s
wife
Basemath.
\par }{\PP \VS{18}These
were the sons
of Esau’s
wife
Oholibamah: chief
Jeush,
chief
Jalam,
chief
Korah.
These
were the chiefs
descended from Esau’s
wife
Oholibamah,
the daughter
of Anah.
\par }{\PP \VS{19}These
were the sons
of Esau
(also known as Edom), and these
were their chiefs.
\par }{\PP \VS{20}These
were the sons
of Seir
the Horite,
who were living
in the land: Lotan,
Shobal,
Zibeon,
Anah,
\VS{21}Dishon,
Ezer,
and Dishan.
These
were the chiefs
of the Horites,
the descendants
of Seir
in the land
of Edom.
\par }{\PP \VS{22}The sons
of Lotan
were Hori
and Homam;
Lotan’s
sister
was Timna.
\par }{\PP \VS{23}These
were the sons
of Shobal: Alvan,
Manahath,
Ebal,
Shepho,
and Onam.
\par }{\PP \VS{24}These
were the sons
of Zibeon: Aiah
and Anah
(who discovered
the hot springs
in the wilderness
as he pastured
the donkeys
of his father
Zibeon).
\par }{\PP \VS{25}These
were the children
of Anah: Dishon
and Oholibamah,
the daughter
of Anah.
\par }{\PP \VS{26}These
were the sons
of Dishon: Hemdan,
Eshban,
Ithran,
and Keran.
\par }{\PP \VS{27}These
were the sons
of Ezer: Bilhan,
Zaavan,
and Akan.
\par }{\PP \VS{28}These
were the sons
of Dishan: Uz
and Aran.
\par }{\PP \VS{29}These
were the chiefs
of the Horites: chief
Lotan,
chief
Shobal,
chief
Zibeon,
chief
Anah,
\VS{30}chief
Dishon,
chief
Ezer,
chief
Dishan.
These
were the chiefs
of the Horites,
according to their chief
lists in the land
of Seir.
\par }{\PP \VS{31}These
were the kings
who
reigned
in the land
of Edom
before
any king
ruled
over the Israelites:
\par }{\PP \VS{32}Bela
the son
of Beor
reigned
in Edom;
the name
of his city
was Dinhabah.
\par }{\PP \VS{33}When Bela
died,
Jobab
the son
of Zerah
from Bozrah
reigned
in his place.
\par }{\PP \VS{34}When Jobab
died,
Husham
from the land
of the Temanites
reigned
in his place.
\par }{\PP \VS{35}When Husham
died,
Hadad
the son
of Bedad,
who defeated
the Midianites
in the land
of Moab,
reigned in his place; the name
of his city
was Avith.
\par }{\PP \VS{36}When Hadad
died,
Samlah
from Masrekah
reigned
in his place.
\par }{\PP \VS{37}When Samlah
died,
Shaul
from Rehoboth
by the River
reigned
in his place.
\par }{\PP \VS{38}When Shaul
died,
Baal-Hanan
the son
of Achbor
reigned
in his place.
\par }{\PP \VS{39}When Baal-Hanan
the son
of Achbor
died,
Hadad
reigned
in his place;
the name
of his city
was Pau.
His wife’s
name
was Mehetabel,
the daughter
of Matred,
the daughter
of Me-Zahab.
\par }{\PP \VS{40}These
were the names
of the chiefs
of Esau,
according to their families,
according to their places,
by their names: chief
Timna,
chief
Alvah,
chief
Jetheth,
\VS{41}chief
Oholibamah,
chief
Elah,
chief
Pinon,
\VS{42}chief
Kenaz,
chief
Teman,
chief
Mibzar,
\VS{43}chief
Magdiel,
chief
Iram.
These
were the chiefs
of Edom,
according to their settlements in
the land
they possessed.
This was
Esau,
the father
of the Edomites.

\par }\Chap{37}{\PP \VerseOne{1}But Jacob
lived
in the land
where his father
had stayed,
in the land
of Canaan.
\par }{\PP \VS{2}This
is the account
of Jacob.
\par }{\PP Joseph,
his seventeen-year-old
son,
was
taking care
of the flocks
with
his brothers.
Now he was
a youngster
working with
the sons
of Bilhah
and Zilpah,
his father’s
wives.
Joseph
brought
back a bad
report
about them to
their father.
\par }{\PP \VS{3}Now Israel
loved
Joseph
more than all
his sons
because
he was
a son
born to him late in life,
and he made
a special
tunic for him.
\VS{4}When Joseph’s brothers
saw
that
their father
loved
him more than any
of them,
they hated
Joseph and were not
able
to speak
to him kindly.
\par }{\PP \VS{5}Joseph
had
a dream,
and when he told
his brothers
about it, they hated
him even more.
\VS{6}He said
to
them, “Listen
to this
dream
I had:
\VS{7}There
we
were,
binding
sheaves
of grain in the middle
of the field.
Suddenly
my sheaf
rose up
and stood upright
and your sheaves
surrounded
my sheaf
and bowed down to it!”
\VS{8}Then his brothers
asked him,
“Do you really think
you will rule
over us or
have dominion
over us?” They hated
him even more
because
of his dream
and because
of what he said.
\par }{\PP \VS{9}Then he had another
dream,
and told
it to his brothers.
“Look,”
he said.
“I had another
dream.
The sun,
the moon,
and eleven
stars
were bowing down to me.”
\VS{10}When he told
his father
and his brothers,
his father
rebuked
him, saying,
“What
is this
dream
that
you had? Will I,
your mother,
and your brothers
really come
and bow down
to you?”
\VS{11}His brothers
were jealous
of him, but his father
kept
in mind what Joseph said.
\par }{\PP \VS{12}When
his brothers
had gone
to graze
their father’s
flocks
near Shechem,
\VS{13}Israel
said
to
Joseph,
“Your brothers
are grazing
the flocks near Shechem.
Come,
I will send
you to
them.” “I’m ready,” Joseph replied.
\VS{14}So Jacob said
to him, “Go
now
and check
on the welfare
of your brothers
and of the
flocks,
and bring
me word.”
So Jacob sent
him from the valley
of Hebron.
\par }{\PP \VS{15}When Joseph reached Shechem, a man
found
him wandering
in the field,
so
the man
asked him,
“What
are you looking for?”
\VS{16}He replied,
“I’m
looking
for my brothers.
Please
tell
me where
they are
grazing their flocks.”
\VS{17}The man
said,
“They left
this
area, for
I heard
them say, ‘Let’s
go
to Dothan.’ ”
So Joseph
went after
his brothers
and found
them at Dothan.
\par }{\PP \VS{18}Now Joseph’s brothers saw
him from a distance,
and before
he reached
them, they plotted
to kill him.
\VS{19}They said
to
one
another,
“Here
comes
this master
of dreams!
\VS{20}Come now,
let’s
kill
him, throw
him into one
of the cisterns,
and then say
that a wild animal
ate
him. Then we’ll see
how
his dreams turn out!”
\par }{\PP \VS{21}When Reuben
heard
this, he rescued
Joseph from their hands,
saying, “Let’s
not
take his life!”
\VS{22}Reuben
continued, “Don’t
shed
blood! Throw
him into
this cistern
that is here
in the wilderness,
but don’t
lay
a hand
on him.”
(Reuben said
this so
he could rescue
Joseph from them and take him back
to
his father.)
\par }{\PP \VS{23}When
Joseph
reached
his brothers,
they stripped
him of his
tunic,
the special
tunic
that
he wore.
\VS{24}Then they took
him and threw
him into the cistern.
(Now the cistern
was empty;
there was no
water in it.)
\par }{\PP \VS{25}When they sat
down to eat
their food,
they looked
up and saw
a caravan
of Ishmaelites
coming
from Gilead.
Their camels
were carrying
spices,
balm,
and myrrh
down
to Egypt.
\VS{26}Then Judah
said to
his brothers,
“What
profit
is there if
we kill
our brother
and cover
up his blood?
\VS{27}Come, let’s
sell
him to the Ishmaelites,
but let’s not
lay a hand
on him, for
after all, he is our brother,
our own flesh.”
His brothers
agreed.
\VS{28}So when the Midianite
merchants
passed by,
Joseph’s
brothers pulled
him out
of the cistern
and sold
him to the Ishmaelites
for twenty
pieces of silver.
The Ishmaelites then took
Joseph
to Egypt.
\par }{\PP \VS{29}Later Reuben
returned
to
the cistern
to find that Joseph
was not
in it! He tore
his clothes,
\VS{30}returned
to
his brothers,
and said,
“The boy
isn’t there! And I,
where
can I
go?”
\VS{31}So they took
Joseph’s
tunic,
killed
a young
goat,
and dipped
the tunic
in the blood.
\VS{32}Then they brought
the special
tunic
to
their father
and said,
“We found
this.
Determine
now
whether
it is
your son’s
tunic
or
not.”
\par }{\PP \VS{33}He recognized
it and exclaimed,
“It is my son’s
tunic! A wild animal
has eaten
him! Joseph
has surely been torn
to pieces!”
\VS{34}Then Jacob
tore
his clothes,
put on
sackcloth,
and mourned
for
his son
many
days.
\VS{35}All
his sons
and daughters
stood by him to
console
him, but he refused
to
be consoled.
“No,” he said,
“I will go
to the grave
mourning
my son.”
So Joseph’s
father
wept for him.
\par }{\PP \VS{36}Now in Egypt
the Midianites
sold
Joseph to
Potiphar,
one of Pharaoh’s
officials,
the captain
of the guard.

\par }\Chap{38}{\PP \VerseOne{1}At that time
Judah
left
his brothers
and stayed
with an Adullamite
man
named
Hirah.
\par }{\PP \VS{2}There
Judah
saw
the daughter
of a Canaanite
man
named
Shua.
Judah acquired
her as a wife and had marital relations
with her.
\VS{3}She became pregnant
and had
a son.
Judah named
him Er.
\VS{4}She became pregnant
again
and had
another son,
whom
she named
Onan.
\VS{5}Then
she had
yet
another
son,
whom she named
Shelah.
She gave birth
to him in Kezib.
\par }{\PP \VS{6}Judah
acquired
a wife
for Er
his firstborn;
her name
was Tamar.
\VS{7}But Er,
Judah’s
firstborn,
was evil
in the
{\ND{Lord}}’s
sight,
so the
{\ND{Lord}}
killed him.
\par }{\PP \VS{8}Then Judah
said
to Onan,
“Have sexual relations
with
your brother’s
wife
and fulfill the duty of a brother-in-law
to her so that you may raise up
a descendant
for your brother.”
\VS{9}But Onan
knew
that
the child
would not
be
considered
his. So whenever he had sexual relations
with
his brother’s
wife,
he withdrew
prematurely
so as not
to give
his brother
a descendant.
\VS{10}What
he did
was evil
in the
{\ND{Lord}}’s sight,
so the
{\ND{Lord}}
killed
him too.
\par }{\PP \VS{11}Then Judah
said
to his daughter-in-law
Tamar,
“Live
as a widow
in your father’s
house
until
Shelah
my son
grows up.” For
he thought, “I don’t want him to die
like his brothers.”
So Tamar
went
and lived
in her father’s
house.
\par }{\PP \VS{12}After some time
Judah’s
wife,
the daughter
of Shua,
died.
After Judah
was consoled,
he
left for
Timnah
to
visit his
sheepshearers,
along with his friend
Hirah
the Adullamite.
\VS{13}Tamar
was told, “Look,
your father-in-law
is going up
to Timnah
to shear
his sheep.”
\VS{14}So
she removed
her widow’s
clothes
and covered
herself with a veil.
She wrapped
herself and sat
at the entrance
to Enaim
which
is on
the way
to Timnah.
(She did this because
she saw
that
she
had not
been given
to Shelah
as a wife,
even though
he had now grown up.)
\par }{\PP \VS{15}When Judah
saw
her, he thought
she was a prostitute
because
she had covered
her face.
\VS{16}He turned aside
to
her along
the road
and said,
“Come on! I want to
have sex
with you.” (He did not
realize
it was his daughter-in-law.) She asked,
“What
will you give
me in exchange for
having sex
with you?”
\VS{17}He replied,
“I’ll
send
you a young
goat
from
the flock.”
She asked,
“Will you
give
me a pledge
until
you send it?”
\VS{18}He said,
“What
pledge
should I give
you?” She replied,
“Your seal,
your cord,
and the staff
that’s
in your hand.”
So he gave
them to
her and had sex
with her. She became pregnant by him.
\VS{19}She left
immediately,
removed
her veil,
and put on
her widow’s
clothes.
\par }{\PP \VS{20}Then
Judah
had his friend
Hirah
the Adullamite
take
a young
goat
to get back from the woman
the items he had given in pledge,
but Hirah
could not
find her.
\VS{21}He asked
the men
who were there, “Where
is the cult prostitute
who
was at Enaim
by
the road?” But they replied,
“There has been
no
cult prostitute
here.”
\VS{22}So he returned
to
Judah
and said,
“I couldn’t
find
her. Moreover,
the men
of the place
said,
‘There has been no
cult prostitute
here.’ ”
\VS{23}Judah
said,
“Let her keep
the things for herself. Otherwise
we will appear to be
dishonest.
I did indeed
send
this
young goat,
but you
couldn’t
find her.”
\par }{\PP \VS{24}After
three
months
Judah
was told, “Your daughter-in-law
Tamar
has turned to prostitution,
and as a result she has become pregnant.”
Judah
said,
“Bring her out
and let her be burned!”
\VS{25}While
they were bringing her out,
she
sent
word
to
her father-in-law: “I am
pregnant
by the man
to whom
these
belong.” Then she said,
“Identify
the one to whom
the seal,
cord,
and staff
belong.”
\VS{26}Judah
recognized
them and said,
“She is more
upright than
I am,
because
I wouldn’t
give
her to Shelah
my son.”
He did not
have sexual relations
with her again.
\par }{\PP \VS{27}When
it was time
for her to give birth,
there
were twins
in her womb.
\VS{28}While
she was giving birth,
one child put
out his hand,
and the midwife
took
a scarlet
thread and tied
it on
his hand,
saying,
“This
one came out
first.”
\VS{29}But then he drew back
his hand,
and his brother
came out
before him. She said,
“How
you have broken out
of the womb!” So
he was named
Perez.
\VS{30}Afterward
his brother
came out
– the one who had
the scarlet
thread on
his hand
– and he was named
Zerah.

\par }\Chap{39}{\PP \VerseOne{1}Now Joseph
had been brought down
to Egypt.
An Egyptian
named Potiphar,
an official
of Pharaoh
and the captain
of the guard,
purchased
him from the Ishmaelites
who had
brought him there.
\VS{2}The
{\ND{Lord}}
was with
Joseph.
He was
successful
and lived in the household
of his Egyptian
master.
\VS{3}His master
observed
that
the {\ND{Lord}}
was with
him and that the
{\ND{Lord}}
made everything
he was
doing
successful.
\VS{4}So Joseph
found
favor
in his sight
and became his personal attendant.
Potiphar appointed
Joseph overseer of his household
and put
him in charge of everything
he owned.
\VS{5}From the time
Potiphar appointed
him over his household
and over
all
that
he owned,
the {\ND{Lord}}
blessed
the Egyptian’s
household
for Joseph’s
sake. The blessing
of the {\ND{Lord}}
was on everything
that he had,
both
in his house
and in his fields.
\VS{6}So Potiphar left
everything
he had in Joseph’s
care;
he gave no
thought
to anything
except
the food
he
ate.
\par }{\PP Now
Joseph
was well
built
and good-looking.
\VS{7}Soon after
these
things,
his master’s
wife
took
notice
of Joseph
and said,
“Have sex
with me.”
\VS{8}But he refused,
saying
to
his master’s
wife,
“Look,
my master
does not
give any thought
to his household
with
me here,
and everything
that
he owns
he has put
into my care.
\VS{9}There is no
one greater
in this
household
than
I am. He has withheld
nothing
from me
except
you because
you
are his wife.
So how
could I do
such
a great
evil
and sin
against God?”
\VS{10}Even though she continued
to speak
to Joseph
day
after day,
he did not
respond
to her invitation
to
have sex
with her.
\par }{\PP \VS{11}One
day
he went
into the house
to do
his work
when
none
of the household
servants were there
in the house.
\VS{12}She grabbed
him by his outer garment,
saying,
“Have sex
with
me!” But he left
his outer garment
in her hand
and ran
outside.
\VS{13}When
she saw
that
he had left
his outer garment
in her hand
and had run
outside,
\VS{14}she called
for her household
servants
and said
to them, “See,
my husband brought
in a Hebrew
man
to us to humiliate
us. He tried
to
have sex
with
me, but I screamed
loudly.
\VS{15}When
he heard
me raise
my voice
and scream,
he left
his outer garment
beside
me and ran
outside.”
\par }{\PP \VS{16}So she laid
his outer garment
beside
her until
his master
came
home.
\VS{17}This
is what she said
to
him: “That Hebrew
slave
you brought
to
us tried to humiliate me,
\VS{18}but when
I raised
my voice
and screamed,
he left
his outer
garment
and ran
outside.”
\par }{\PP \VS{19}When
his master
heard
his wife
say, “This
is the way
your slave
treated me,” he became furious.
\VS{20}Joseph’s
master
took
him and threw
him into
the prison,
the place
where
the king’s
prisoners
were confined.
So he was there
in the prison.
\par }{\PP \VS{21}But the
{\ND{Lord}}
was
with
Joseph
and showed
him kindness.
He granted
him favor
in the sight
of the prison
warden.
\VS{22}The warden
put
all
the prisoners
under Joseph’s
care.
He was
in charge
of whatever
they were doing.
\VS{23}The warden
did not
concern himself with
anything
that was in Joseph’s care
because
the {\ND{Lord}}
was with
him and whatever
he was
doing
the {\ND{Lord}}
was making successful.

\par }\Chap{40}{\PP \VerseOne{1}After
these
things
happened,
the cupbearer
to the king
of Egypt
and the royal baker
offended their master,
the king
of Egypt.
\VS{2}Pharaoh
was enraged
with his two
officials,
the cupbearer
and the baker,
\VS{3}so he imprisoned
them in the house
of the captain
of the guard
in
the same facility
where
Joseph
was confined.
\VS{4}The captain
of the guard
appointed
Joseph
to be
their attendant,
and he served
them.

\par }{\PP They spent
some time
in custody.
\VS{5}Both of them,
the cupbearer
and the baker
of the king
of Egypt,
who
were confined
in the prison,
had a dream
the same night.
Each
man’s
dream
had its own meaning.
\VS{6}When Joseph
came
to
them in the morning,
he saw
that they were looking
depressed.
\VS{7}So he asked
Pharaoh’s
officials,
who
were with
him in custody
in his master’s
house,
“Why
do you look so sad
today?”
\VS{8}They told
him,
“We both had
dreams,
but there is no
one to
interpret
them.” Joseph
responded,
“Don’t
interpretations
belong to God? Tell
them to me.”
\par }{\PP \VS{9}So
the chief
cupbearer
told
his dream
to Joseph: “In my dream,
there
was a vine
in front of me.
\VS{10}On the vine
there were three
branches.
As it
budded,
its blossoms
opened
and its clusters
ripened
into grapes.
\VS{11}Now Pharaoh’s
cup
was in my hand,
so I took
the grapes,
squeezed
them into
his
cup,
and put
the cup
in Pharaoh’s
hand.”
\par }{\PP \VS{12}“This
is its meaning,”
Joseph
said
to him. “The three
branches
represent three
days.
\VS{13}In three
more
days
Pharaoh
will reinstate
you and restore
you to your office.
You will put
Pharaoh’s
cup
in his hand,
just
as
you did before
when
you were cupbearer.
\VS{14}But
remember
me when
it goes
well
for you, and show
me
kindness.
Make mention
of me to
Pharaoh
and bring me out
of this
prison,
\VS{15}for
I really was kidnapped
from the land
of the Hebrews
and I have
done
nothing
wrong here
for
which they should put
me in a dungeon.”
\par }{\PP \VS{16}When the chief
baker
saw
that
the interpretation
of the first dream was favorable,
he said
to
Joseph,
“I
also
appeared in my dream
and there were three
baskets
of white bread
on
my head.
\VS{17}In the top
basket
there were baked
goods
of every
kind for Pharaoh,
but the birds
were eating
them from
the basket
that was on
my head.”
\par }{\PP \VS{18}Joseph
replied,
“This
is its meaning: The three
baskets
represent three
days.
\VS{19}In three
more
days
Pharaoh
will decapitate
you and impale
you on
a pole.
Then the birds
will eat
your flesh from you.”
\par }{\PP \VS{20}On the third
day
it was Pharaoh’s
birthday,
so he gave
a feast
for all
his servants.
He “lifted
up” the head
of the chief
cupbearer
and the head
of the chief
baker
in the midst
of his servants.
\VS{21}He restored
the chief
cupbearer
to his former position
so that he placed
the cup
in
Pharaoh’s
hand,
\VS{22}but the chief
baker
he impaled,
just as
Joseph
had predicted.
\VS{23}But the chief
cupbearer
did not
remember
Joseph
– he forgot him.

\par }\Chap{41}{\PP \VerseOne{1}At the end
of two full years
Pharaoh
had a dream.
As he was standing
by
the Nile,
\VS{2}seven
fine-looking,
fat
cows
were coming up
out of
the Nile,
and they grazed
in the reeds.
\VS{3}Then
seven
bad-looking,
thin
cows
were coming up
after
them from
the Nile,
and they stood
beside
the other cows
at
the edge
of the river.
\VS{4}The bad-looking,
thin
cows
ate
the seven
fine-looking,
fat
cows.
Then Pharaoh
woke up.
\par }{\PP \VS{5}Then he fell asleep
again and had a second
dream: There
were seven
heads of grain
growing
on one
stalk,
healthy
and good.
\VS{6}Then
seven
heads of grain,
thin
and burned
by the east wind,
were sprouting
up after them.
\VS{7}The thin
heads
swallowed
up the seven
healthy
and full
heads.
Then Pharaoh
woke
up and realized
it was a dream.
\par }{\PP \VS{8}In the morning
he was troubled,
so
he called
for all
the diviner-priests
of Egypt
and all
its wise
men. Pharaoh
told
them his dreams,
but no
one could interpret
them
for him.
\VS{9}Then
the chief
cupbearer
said
to Pharaoh,
“Today
I
recall
my failures.
\VS{10}Pharaoh
was enraged
with his servants,
and he put
me in prison
in the house
of the captain
of the guards
– me and the chief
baker.
\VS{11}We each had
a dream
one
night;
each
of us had
a dream
with its own meaning.
\VS{12}Now a young man,
a Hebrew,
a servant
of the captain
of the guards,
was with
us there.
We told
him our dreams,
and he interpreted
the meaning of each of our respective dreams for us.
\VS{13}It happened
just as
he had said to us – Pharaoh restored me to my office, but he impaled the baker.”
\par }{\PP \VS{14}Then
Pharaoh
summoned
Joseph.
So they brought him quickly
out
of the dungeon;
he shaved
himself, changed
his clothes,
and came
before
Pharaoh.
\VS{15}Pharaoh
said
to
Joseph,
“I had a dream,
and there is no
one who can interpret
it. But I
have heard
about
you, that you can interpret
dreams.”
\VS{16}Joseph
replied
to Pharaoh,
“It is not within
my power, but God
will speak concerning
the welfare
of Pharaoh.”
\par }{\PP \VS{17}Then
Pharaoh
said to
Joseph,
“In my dream
I
was standing
by
the edge
of the Nile.
\VS{18}Then
seven
fat
and fine-looking
cows
were coming up
out of
the Nile,
and they grazed
in the reeds.
\VS{19}Then
seven
other
cows
came up
after
them; they were scrawny,
very
bad-looking,
and lean.
I had never
seen
such bad-looking
cows as these in all
the land
of Egypt!
\VS{20}The lean,
bad-looking
cows
ate
up the seven
fat
cows.
\VS{21}When they had eaten
them, no
one would have known
that
they had done
so, for they were just as bad-looking
as before.
Then I woke up.
\VS{22}I also saw
in my dream
seven
heads of grain
growing
on one
stalk,
full
and good.
\VS{23}Then
seven
heads of grain,
withered
and thin
and burned
with the east wind,
were sprouting up
after them.
\VS{24}The thin
heads of grain
swallowed
up the seven
good
heads of grain.
So I told
all this to the diviner-priests,
but no
one could tell me its meaning.”
\par }{\PP \VS{25}Then Joseph
said to
Pharaoh,
“Both dreams
of Pharaoh
have
the same
meaning. God
has revealed
to Pharaoh
what he is about to do.
\VS{26}The seven
good
cows
represent seven
years,
and the seven
good
heads of grain
represent
seven
years.
Both dreams
have the same
meaning.
\VS{27}The seven
lean,
bad-looking
cows
that came up
after
them represent seven
years,
as do the seven
empty
heads of grain
burned
with the east wind.
They represent
seven
years
of famine.
\VS{28}This is
just
what
I told
Pharaoh: God
has shown
Pharaoh
what
he is about
to do.
\VS{29}Seven
years
of great
abundance
are coming
throughout the whole
land
of Egypt.
\VS{30}But seven
years
of famine
will occur
after
them, and all
the abundance
will be forgotten
in the land
of Egypt.
The famine
will devastate
the land.
\VS{31}The previous
abundance
of the land
will not
be remembered because
of the famine
that
follows,
for
the famine will be very
severe.
\VS{32}The dream
was repeated
to
Pharaoh
because
the matter
has been decreed
by God,
and God
will make it happen soon.
\par }{\PP \VS{33}“So now
Pharaoh
should look
for a wise
and discerning
man
and give him authority
over
all the land
of Egypt.
\VS{34}Pharaoh
should do this – he should appoint officials throughout the land to collect one-fifth of the produce of the land of Egypt during the seven years of abundance.
\VS{35}They should gather
all
the excess food
during these
good
years
that are coming.
By Pharaoh’s
authority they should store up
grain
so the cities
will have food,
and they should preserve it.
\VS{36}This food
should be
held in storage
for the land
in preparation for the seven
years
of famine
that
will
occur throughout the land
of Egypt.
In this way the land
will survive the famine.”
\par }{\PP \VS{37}This advice made sense
to Pharaoh
and all
his officials.
\VS{38}So Pharaoh
asked
his officials,
“Can we find
a man
like Joseph, one
in whom
the Spirit
of God is present?”
\VS{39}So Pharaoh
said
to Joseph,
“Because God
has enabled
you to know
all
this,
there is no
one as wise
and discerning
as you are!
\VS{40}You
will
oversee
my
household,
and all
my people
will submit to your commands. Only
I, the king,
will be greater
than you.
\par }{\PP \VS{41}“See
here,” Pharaoh
said
to
Joseph,
“I place
you in authority over
all
the land
of Egypt.”
\VS{42}Then Pharaoh
took
his signet ring
from his own hand
and put
it on
Joseph’s.
He clothed
him with fine linen
clothes
and put
a gold
chain
around
his neck.
\VS{43}Pharaoh had him ride
in the chariot
used by
his second-in-command,
and they cried out
before
him, “Kneel
down!” So he placed
him over
all
the land
of Egypt.
\VS{44}Pharaoh
also said
to
Joseph,
“I am
Pharaoh,
but without your permission
no
one
will move
his hand
or his foot
in all
the land
of Egypt.”
\VS{45}Pharaoh
gave Joseph
the name
Zaphenath-Paneah.
He also gave
him Asenath
daughter
of Potiphera,
priest
of On,
to be his wife.
So
Joseph
took charge
of all the land
of Egypt.
\par }{\PP \VS{46}Now Joseph
was 30
years
old when
he began serving
Pharaoh
king
of Egypt.
Joseph
was commissioned by
Pharaoh
and was in charge of all
the land
of Egypt.
\VS{47}During the seven
years
of abundance
the land
produced
large, bountiful harvests.
\VS{48}Joseph collected
all
the excess food
in the land
of Egypt
during the seven
years
and stored
it
in the cities.
In every city
he put the food
gathered
from the fields
around
it.
\VS{49}Joseph
stored up
a vast
amount
of grain,
like the sand
of the sea,
until
he stopped
measuring
it because
it was impossible
to measure.
\par }{\PP \VS{50}Two
sons
were born
to Joseph
before
the famine
came.
Asenath
daughter
of Potiphera,
priest
of On,
was their mother.
\VS{51}Joseph
named
the firstborn
Manasseh,
saying, “Certainly
God
has made me forget
all
my trouble
and all
my father’s
house.”
\VS{52}He named
the second
child
Ephraim,
saying, “Certainly
God
has made me fruitful
in the land
of my suffering.”
\par }{\PP \VS{53}The seven
years
of abundance
in the land
of Egypt
came to an end.
\VS{54}Then the seven
years
of famine
began,
just
as Joseph
had predicted.
There was
famine
in all
the other lands,
but throughout
the land
of Egypt
there was
food.
\VS{55}When all
the land
of Egypt
experienced
the famine, the people
cried out
to
Pharaoh
for food.
Pharaoh
said
to all
the people of Egypt, “Go
to
Joseph
and do
whatever
he tells you.”
\par }{\PP \VS{56}While the famine
was
over
all
the earth,
Joseph
opened
the storehouses and sold
grain to the Egyptians.
The famine
was severe
throughout the land
of Egypt.
\VS{57}People from every
country
came
to
Joseph
in Egypt
to buy
grain because
the famine
was severe
throughout
the earth.

\par }\Chap{42}{\PP \VerseOne{1}When
Jacob
heard
there was
grain
in Egypt,
he
said
to his sons,
“Why
are you looking at each other?”
\VS{2}He then said,
“Look,
I hear
that
there
is grain
in Egypt.
Go down
there
and buy
grain for us so that we may live
and not
die.”
\par }{\PP \VS{3}So ten
of Joseph’s
brothers
went down
to buy
grain
from Egypt.
\VS{4}But Jacob
did not
send
Joseph’s
brother
Benjamin
with
his brothers,
for
he said, “What if some accident happens to him?”
\VS{5}So Israel’s
sons
came
to buy
grain among
the other travelers,
for
the famine
was
severe in the land
of Canaan.
\par }{\PP \VS{6}Now Joseph
was
the ruler
of the country,
the one who sold
grain to all
the people
of the country.
Joseph’s
brothers
came
and bowed
down before him with their faces
to the ground.
\VS{7}When Joseph
saw
his brothers,
he recognized
them, but he pretended to be a stranger
to them
and spoke
to
them
harshly.
He asked,
“Where
do you come
from?” They answered, “From the land
of Canaan,
to buy
grain for food.”
\par }{\PP \VS{8}Joseph
recognized
his brothers,
but they
did not
recognize him.
\VS{9}Then Joseph
remembered
the dreams
he had
dreamed
about them, and he said
to
them, “You are spies;
you have come
to see
if our land
is vulnerable!”
\par }{\PP \VS{10}But they exclaimed, “No,
my lord! Your servants
have come
to buy
grain for food!
\VS{11}We are
all
the sons
of one
man;
we
are honest
men! Your servants
are not
spies.”
\par }{\PP \VS{12}“No,”
he insisted,
“but you have come
to see
if
our land
is vulnerable.”
\VS{13}They replied,
“Your servants
are from a family of twelve
brothers.
We
are the sons
of one
man
in the land
of Canaan.
The youngest
is with
our father
at this time,
and one
is no longer alive.”
\par }{\PP \VS{14}But Joseph
told
them, “It is
just
as I said
to
you: You are
spies!
\VS{15}You will be tested
in this
way: As surely as Pharaoh
lives,
you will not
depart
from this
place unless
your youngest
brother
comes
here.
\VS{16}One
of you must go
and get
your brother,
while the rest of you
remain in prison.
In this way your words
may be tested
to see if you are telling the truth.
If
not,
then, as surely as Pharaoh
lives,
you
are spies!”
\VS{17}He imprisoned
them all for three
days.
\VS{18}On
the third
day
Joseph
said
to them,
“Do
as I say and you will live,
for I
fear
God.
\VS{19}If
you
are honest
men, leave one
of your brothers
confined
here in prison
while the rest of you
go
and take
grain
back
for your hungry
families.
\VS{20}But you must bring
your youngest
brother
to
me. Then your words
will be verified
and you will not
die.”
They did
as he said.
\par }{\PP \VS{21}They said
to
one
other, “Surely
we’re
being punished
because of our brother,
because
we saw
how distressed
he was when he cried
to us for mercy,
but we refused
to listen.
That is why
this distress
has come
on
us!”
\VS{22}Reuben
said
to them,
“Didn’t
I say
to you, ‘Don’t
sin
against the boy,’
but you wouldn’t listen? So now
we must pay for shedding
his blood!”
\VS{23}(Now they
did not
know
that
Joseph
could understand
them, for
he was speaking through an interpreter.)
\VS{24}He turned away
from them and wept.
When he turned around
and spoke
to
them again, he had Simeon
taken
from them and tied up
before their eyes.
\par }{\PP \VS{25}Then Joseph
gave orders
to fill
their bags
with grain,
to return
each man’s
money
to
his sack,
and to give
them provisions
for the journey.
His orders were carried out.
\VS{26}So
they loaded
their grain
on
their donkeys
and left.
\par }{\PP \VS{27}When one
of them
opened
his sack
to get feed
for his donkey
at their resting
place, he saw
his money
in the mouth
of his sack.
\VS{28}He said
to
his brothers,
“My money
was returned! Here
it is in my sack!” They were dismayed;
they turned trembling
one
to
another
and said,
“What
in the world has God
done to us?”
\par }{\PP \VS{29}They returned
to
their father
Jacob
in the land
of Canaan
and told
him all
the things that had happened
to them, saying,
\VS{30}“The man,
the lord
of the land,
spoke
harshly
to us and treated
us as if we were spying on
the land.
\VS{31}But we said
to him,
‘We are honest
men; we
are not
spies!
\VS{32}We are from a family of twelve
brothers;
we
are the sons
of one
father.
One
is no
longer alive, and the youngest
is with
our father
at this time
in the land
of Canaan.’
\par }{\PP \VS{33}“Then the man,
the lord
of the land,
said
to us, ‘This
is how I will find out
if
you
are honest
men. Leave
one
of your brothers
with
me, and take
grain for your hungry
households
and go.
\VS{34}But bring
your youngest
brother
back to
me so I will know
that
you
are honest
men and not
spies.
Then
I will give
your brother
back to you
and you may move about freely
in the land.’ ”
\par }{\PP \VS{35}When
they
were emptying
their sacks,
there was each man’s
bag
of money
in his sack! When they and their father
saw
the bags
of money,
they
were afraid.
\VS{36}Their father
Jacob
said
to
them, “You are making me childless! Joseph
is gone.
Simeon
is gone.
And now you want to take
Benjamin! Everything is against me.”
\par }{\PP \VS{37}Then Reuben
said
to
his father,
“You may put my two
sons
to death
if
I do not
bring
him back to
you. Put
him
in my care
and I
will bring him back
to you.”
\VS{38}But Jacob replied,
“My son
will not
go down
there with
you, for
his brother
is dead
and he alone
is left.
If an accident
happens to him on the journey
you have to make,
then you will bring down
my gray hair
in sorrow
to the grave.”

\par }\Chap{43}{\PP \VerseOne{1}Now the famine
was severe
in the land.
\VS{2}When
they finished
eating
the grain
they had
brought
from Egypt,
their father
said
to
them, “Return,
buy
us a little
more food.”
\par }{\PP \VS{3}But Judah
said
to him, “The man
solemnly
warned
us, ‘You will not
see
my face
unless
your brother
is with you.’
\VS{4}If
you send
our
brother
with
us, we’ll go down
and buy
food for you.
\VS{5}But if
you will not
send
him, we won’t go down
there because
the man
said
to us,
‘You will not
see
my face
unless
your brother
is with you.’ ”
\par }{\PP \VS{6}Israel
said,
“Why
did you bring this trouble
on me by telling
the man
you had one more
brother?”
\par }{\PP \VS{7}They replied,
“The man
questioned
us thoroughly about
ourselves
and our family, saying,
‘Is your father
still
alive? Do you have
another brother?’ So we answered him in this way.
How
could we
possibly
know
that
he would say, ‘Bring your brother
down’?”
\par }{\PP \VS{8}Then Judah
said to
his father
Israel,
“Send
the boy
with
me and we will go
immediately.
Then we will live
and not
die
– we
and you
and our little ones.
\VS{9}I
myself pledge
security for him;
you may hold
me liable.
If
I do not
bring
him
back to
you and place him here
before
you, I will bear the blame
before you all
my life.
\VS{10}But if
we had not
delayed,
we could have traveled there and back
twice
by now!”
\par }{\PP \VS{11}Then their father
Israel
said
to
them, “If
it must be so,
then
do
this: Take
some of the best products
of the land
in your bags,
and take a gift
down
to the man
– a little
balm
and a little
honey,
spices
and myrrh,
pistachios
and almonds.
\VS{12}Take double
the money
with you; you must take
back
the money
that was returned
in the mouths
of your sacks
– perhaps
it was
an oversight.
\VS{13}Take
your brother
too, and go
right away
to
the man.
\VS{14}May the sovereign
God
grant
you mercy
before
the man
so that he may release
your other
brother
and Benjamin! As
for me, if I lose my children I
lose them.”
\par }{\PP \VS{15}So
the men
took
these
gifts,
and they took
double
the money
with them, along
with Benjamin.
Then
they hurried
down
to Egypt
and stood
before
Joseph.
\VS{16}When Joseph
saw
Benjamin
with
them, he said
to the servant who
was over
his household,
“Bring
the
men
to the house.
Slaughter
an animal
and prepare
it, for
the men
will eat
with
me at noon.”
\VS{17}The man
did
just
as Joseph
said;
he brought
the men
into Joseph’s
house.
\par }{\PP \VS{18}But the men
were afraid
when
they were brought
to Joseph’s
house.
They said,
“We
are being brought
in because
of the money
that was returned
in our sacks
last time.
He wants
to
capture
us, make
us slaves,
and take our donkeys!”
\VS{19}So they approached
the man
who
was in charge
of Joseph’s
household
and spoke
to
him at the entrance
to the house.
\VS{20}They said,
“My lord,
we did indeed come down
the first time
to buy
food.
\VS{21}But when
we
came
to
the place where we spent
the night, we opened
our sacks
and each
of us found his money
– the full amount –
in the mouth
of his sack.
So we have returned it.
\VS{22}We have brought additional
money
with us to buy
food.
We do not
know
who
put
the money
in our sacks!”
\par }{\PP \VS{23}“Everything is fine,”
the man in charge of Joseph’s household told
them. “Don’t
be afraid.
Your God
and the God
of your father
has given
you treasure
in your sacks.
I had your money.”
Then he brought
Simeon
out
to them.
\par }{\PP \VS{24}The servant in charge brought
the men
into Joseph’s
house.
He gave
them water,
and they washed
their feet.
Then he gave
food
to their donkeys.
\VS{25}They got their gifts
ready
for Joseph’s
arrival
at noon,
for
they had heard
that
they were to have a meal
there.
\par }{\PP \VS{26}When Joseph
came
home,
they presented
him with the gifts
they had
brought
inside,
and they bowed down
to the ground before him.
\VS{27}He asked
them how they were doing.
Then he said,
“Is your aging
father
well,
the one you spoke
about? Is he still
alive?”
\VS{28}“Your servant
our father
is well,”
they replied.
“He is still
alive.”
They bowed
down in humility.
\par }{\PP \VS{29}When Joseph looked
up and saw
his brother
Benjamin,
his mother’s
son,
he said,
“Is this
your youngest
brother,
whom
you told
me
about?” Then he said,
“May God
be gracious
to you, my son.”
\VS{30}Joseph
hurried
out, for
he was overcome by affection
for his brother
and was at the point of tears.
So he went
to his room
and wept
there.
\par }{\PP \VS{31}Then he washed
his face
and came out.
With composure
he said, “Set out
the food.”
\VS{32}They set
a place for him, a separate
place for
his brothers, and another for the Egyptians
who were eating
with
him. (The Egyptians
are not
able
to eat
with Hebrews,
for
the Egyptians
think it is
disgusting
to do so.)
\VS{33}They sat
before
him, arranged by order of birth, beginning
with the firstborn
and ending
with the youngest.
The men
looked at each
other
in astonishment.
\VS{34}He gave
them portions
of the food set before
him,
but the portion
for Benjamin
was five
times
greater
than the portions
for any
of the others. They drank
with
Joseph until they all became drunk.

\par }\Chap{44}{\PP \VerseOne{1}He instructed
the
servant who
was over
his household,
“Fill
the
sacks
of the men
with as much food
as
they can
carry
and put
each man’s
money
in the mouth
of his sack.
\VS{2}Then put
my cup
– the silver
cup
– in the mouth
of the youngest
one’s sack,
along
with the money
for his grain.”
He did
as
Joseph
instructed.
\par }{\PP \VS{3}When morning
came,
the men
and their
donkeys
were
sent off.
\VS{4}They
had not
gone very far
from the city
when
Joseph
said
to the servant who
was over
his household,
“Pursue
the men
at once! When you overtake
them, say
to
them, ‘Why
have you repaid
good
with evil?
\VS{5}Doesn’t
my master
drink
from this
cup and use
it for divination? You have
done
wrong!’ ”
\par }{\PP \VS{6}When the man overtook
them, he spoke
these
words to them.
\VS{7}They answered
him,
“Why
does my lord
say
such
things? Far be it
from your servants
to do
such a thing!
\VS{8}Look,
the money
that
we found
in the mouths
of our sacks
we brought back
to
you from the land
of Canaan.
Why
then would we steal
silver
or
gold
from your master’s
house?
\VS{9}If one of us has it, he will die,
and the rest of us
will become
my lord’s
slaves!”
\par }{\PP \VS{10}He replied,
“You have suggested
your own punishment! The one
who
has it will become
my slave,
but the rest of you
will go
free.”
\VS{11}So each man
quickly
lowered
his sack
to the ground
and opened
it.
\VS{12}Then the man searched.
He began
with the oldest
and finished
with the youngest.
The cup
was found
in Benjamin’s
sack!
\VS{13}They all tore
their clothes! Then each man
loaded
his donkey,
and they returned
to the city.
\par }{\PP \VS{14}So Judah
and his brothers
came back
to Joseph’s
house.
He
was still
there,
and they threw
themselves to the ground
before him.
\VS{15}Joseph
said
to them, “What
did you think
you were doing? Don’t
you know
that
a man
like
me can find out things
like this
by divination?”
\par }{\PP \VS{16}Judah
replied,
“What
can we say
to my lord? What
can we speak? How
can we clear
ourselves?
 God
has exposed
the sin
of your servants! We are now
my lord’s
slaves,
we
and the one in whose
possession the cup
was found.”
\par }{\PP \VS{17}But Joseph said,
“Far be it
from me to do
this! The man
in whose
hand
the cup
was found
will become
my slave,
but the rest of you
may go back
to
your father
in peace.”
\par }{\PP \VS{18}Then Judah
approached
him and said,
“My lord,
please allow
your servant
to speak
a word
with you.
Please do not
get angry
with your servant,
for
you are just like
Pharaoh.
\VS{19}My lord
asked
his servants,
‘Do you have
a father
or
a brother?’
\VS{20}We said
to
my lord,
‘We have
an aged
father,
and there is a young
boy
who was born when our father was old.
The boy’s brother
is dead.
He is the only
one of his mother’s
sons left,
and his father
loves him.’
\par }{\PP \VS{21}“Then you told
your servants,
‘Bring him down
to
me so
I can
see him.’
\VS{22}We said
to my lord,
‘The boy
cannot
leave
his father.
If he leaves
his father, his father
will die.’
\VS{23}But you said
to
your servants,
‘If
your youngest
brother
does not
come down
with
you, you will not
see
my face
again.’
\VS{24}When
we
returned to
your servant
my father,
we told
him the words
of my lord.
\par }{\PP \VS{25}“Then our father
said,
‘Go back
and buy
us a little
food.’
\VS{26}But we replied,
‘We cannot
go down
there.
If
our youngest
brother
is with
us, then
we will go,
for
we won’t
be permitted
to see
the man’s
face
if
our youngest
brother
is not
with us.’
\par }{\PP \VS{27}“Then your servant
my father
said to us, ‘You
know
that
my wife
gave me two
sons.
\VS{28}The first
disappeared
and I said,
“He has surely
been torn
to pieces.”
I have not
seen
him since.
\VS{29}If you take
this
one from me too
and an accident
happens
to him, then you will bring down
my gray hair
in tragedy
to the grave.’
\par }{\PP \VS{30}“So now,
when I return
to
your servant
my father,
and the boy
is not
with us – his very life is bound up in his son’s life.
\VS{31}When
he sees
the boy
is not
with us, he will die,
and your servants
will bring down
the gray hair
of your servant
our father
in sorrow
to the grave.
\VS{32}Indeed,
your servant
pledged security
for the boy
with
my father,
saying,
‘If
I do not
bring
him back to you, then I will bear the blame
before my father
all
my life.’
\par }{\PP \VS{33}“So now,
please
let your servant
remain
as my lord’s
slave
instead
of the boy.
As for the boy,
let him go back
with
his brothers.
\VS{34}For
how
can I go
back to
my father
if
the boy
is not
with
me? I couldn’t
bear to see
my father’s pain.”

\par }\Chap{45}{\PP \VerseOne{1}Joseph
was no
longer able
to control
himself before all
his attendants,
so he cried out,
“Make
everyone
go out
from my presence!” No
one
remained
with
Joseph
when he made himself known
to
his brothers.
\VS{2}He wept
loudly;
the Egyptians
heard
it and Pharaoh’s
household
heard about it.
\par }{\PP \VS{3}Joseph
said
to
his brothers,
“I am
Joseph! Is my father
still
alive?” His brothers
could
not
answer
him because
they were dumbfounded
before him.
\VS{4}Joseph
said
to
his brothers,
“Come closer
to me,”
so they came
near.
Then he said,
“I am
Joseph
your brother,
whom
you sold
into Egypt.
\VS{5}Now,
do not
be upset
and do not
be angry
with yourselves
because
you sold
me here,
for
God
sent
me ahead
of you to preserve life!
\VS{6}For
these
past two years
there has been famine
in
the land
and for five
more
years
there
will be neither
plowing
nor harvesting.
\VS{7}God
sent
me ahead
of you to preserve
you on the earth
and to save your lives by a great
deliverance.
\VS{8}So now,
it is not
you
who sent
me here,
but God.
He has made
me an adviser
to Pharaoh,
lord
over all
his household,
and ruler
over all
the land
of Egypt.
\VS{9}Now go up
to my father
quickly
and tell
him,
‘This is what
your son
Joseph
says: “God
has made
me lord
of all
Egypt.
Come down
to me;
do not
delay!
\VS{10}You will live
in the land
of Goshen,
and you will be
near
me
– you,
your children,
your grandchildren,
your flocks,
your herds,
and everything
you have.
\VS{11}I will provide
you with food there
because
there will be five
more
years
of famine.
Otherwise
you would become poor
– you,
your household,
and everyone
who belongs to you.” ’
\VS{12}You and my brother
Benjamin
can
certainly
see
with your own eyes
that
I really
am the one who speaks
to you.
\VS{13}So tell
my father
about all
my honor
in Egypt
and about everything
you have
seen.
But bring my father
down
here
quickly!”
\par }{\PP \VS{14}Then he threw
himself on
the neck
of his brother
Benjamin
and wept,
and Benjamin
wept
on
his neck.
\VS{15}He kissed
all
his brothers
and wept
over
them. After
this
his brothers
talked
with him.
\par }{\PP \VS{16}Now
it was reported
in the household
of Pharaoh,
“Joseph’s
brothers
have arrived.”
It pleased
Pharaoh
and his servants.
\VS{17}Pharaoh
said
to
Joseph,
“Say
to your brothers,
‘Do
this: Load
your animals
and go
to the land
of Canaan!
\VS{18}Get
your father
and your households
and come
to
me! Then I will give
you the best
land
in Egypt
and you will eat
the
best
of the land.’
\VS{19}You
are also commanded
to say, ‘Do
this: Take
for yourselves wagons
from the land
of Egypt
for your little ones
and for your wives.
Bring
your father
and come.
\VS{20}Don’t
worry
about your belongings,
for
the best
of all
the land
of Egypt will be yours.’ ”
\par }{\PP \VS{21}So
the sons
of Israel
did
as he said. Joseph
gave
them wagons
as Pharaoh
had instructed,
and he gave
them provisions
for the journey.
\VS{22}He gave
sets
of clothes
to each one of them, but to Benjamin
he gave
three
hundred
pieces of silver
and five
sets
of clothes.
\VS{23}To his father
he sent
the following: ten
donkeys
loaded
with the best products
of Egypt
and ten
female donkeys
loaded
with grain,
food,
and provisions
for his father’s
journey.
\VS{24}Then he sent
his brothers
on
their way
and they left.
He said
to them,
“As you travel
don’t
be overcome with fear.”
\par }{\PP \VS{25}So they went up
from Egypt
and came
to their father
Jacob
in the land
of Canaan.
\VS{26}They told
him, “Joseph
is still
alive
and he is ruler
over all
the land
of Egypt!” Jacob was stunned,
for
he did not
believe them.
\VS{27}But when
they related
to
him everything
Joseph
had
said
to them,
and when he saw
the wagons
that
Joseph
had
sent
to transport
him, their father
Jacob’s
spirit
revived.
\VS{28}Then Israel
said,
“Enough! My son
Joseph
is still
alive! I will go
and see
him before
I die.”

\par }\Chap{46}{\PP \VerseOne{1}So Israel
began his journey,
taking with him all
that he had.
When he came
to Beer Sheba
he offered
sacrifices
to the God
of his father
Isaac.
\VS{2}God
spoke
to Israel
in a vision
during the night
and said,
“Jacob,
Jacob!” He replied,
“Here I am!”
\VS{3}He said,
“I am
God,
the God
of your father.
Do not
be afraid
to go down
to Egypt,
for
I will make
you into a great
nation
there.
\VS{4}I
will go down
with
you to Egypt
and I
myself
will certainly bring
you back
from there. Joseph
will close
your eyes.”
\par }{\PP \VS{5}Then
Jacob
started out from Beer Sheba,
and the sons
of Israel
carried
their father
Jacob,
their little children,
and their wives
in the wagons
that
Pharaoh
had
sent
along to transport him.
\VS{6}Jacob
and all
his descendants
took
their livestock
and the possessions
they had
acquired
in the land
of Canaan,
and they went
to Egypt.
\VS{7}He brought
with
him to Egypt
his sons
and grandsons, his daughters
and granddaughters
– all
his descendants.
\par }{\PP \VS{8}These
are the names
of the sons
of Israel
who went
to Egypt
– Jacob
and his sons:
\par }{\Q Reuben,
the firstborn
of Jacob.
\par }{\Q \VS{9}The sons
of Reuben:
\par }{\Q Hanoch,
Pallu,
Hezron,
and Carmi.
\par }{\Q \VS{10}The sons
of Simeon:
\par }{\Q Jemuel,
Jamin,
Ohad,
Jakin,
Zohar,
\par }{\Q and Shaul
(the son
of a Canaanite woman).
\par }{\Q \VS{11}The sons
of Levi:
\par }{\Q Gershon,
Kohath,
and Merari.
\par }{\Q \VS{12}The sons
of Judah:
\par }{\Q Er,
Onan,
Shelah,
Perez,
and Zerah
\par }{\Q (but Er
and Onan
died
in the land
of Canaan).
\par }{\Q The sons
of Perez
were Hezron
and Hamul.
\par }{\Q \VS{13}The sons
of Issachar:
\par }{\Q Tola,
Puah,
Jashub,
and Shimron.
\par }{\Q \VS{14}The sons
of Zebulun:
\par }{\Q Sered,
Elon,
and Jahleel.
\par }{\Q \VS{15}These
were the sons
of Leah,
whom
she bore
to Jacob
in Paddan Aram,
along with
Dinah
his daughter.
His sons
and daughters
numbered thirty-three
in all.
\par }{\Q \VS{16}The sons
of Gad:
\par }{\Q Zephon,
Haggi,
Shuni,
Ezbon,
Eri,
Arodi,
and Areli.
\par }{\Q \VS{17}The sons
of Asher:
\par }{\Q Imnah,
Ishvah,
Ishvi,
Beriah,
and Serah
their sister.
\par }{\Q The sons
of Beriah
were Heber
and Malkiel.
\par }{\Q \VS{18}These
were the sons
of Zilpah,
whom
Laban
gave
to Leah
his daughter.
She bore
these
to Jacob,
sixteen in all.
\par }{\Q \VS{19}The sons
of Rachel
the wife
of Jacob:
\par }{\Q Joseph
and Benjamin.
\par }{\Q \VS{20}Manasseh
and Ephraim
were born
to Joseph
in the land
of Egypt.
Asenath
daughter
of Potiphera,
priest
of On,
bore them to him.
\par }{\Q \VS{21}The sons
of Benjamin:

\par }{\Q Bela,
Beker,
Ashbel,
Gera,
Naaman,
Ehi,
Rosh,
Muppim,
Huppim
and Ard.
\par }{\Q \VS{22}These
were the sons
of Rachel
who
were born
to Jacob,
fourteen
in all.
\par }{\Q \VS{23}The son
of Dan: Hushim.
\par }{\Q \VS{24}The sons
of Naphtali:
\par }{\Q Jahziel,
Guni,
Jezer,
and Shillem.
\par }{\Q \VS{25}These
were the sons
of Bilhah,
whom
Laban
gave
to Rachel
his daughter.
She bore
these
to Jacob,
seven
in all.
\par }{\PP \VS{26}All
the direct
descendants of Jacob
who
went
to Egypt
with him were sixty-six
in number. (This number does not include
the wives
of Jacob’s
sons.)
\VS{27}Counting the two
sons
of Joseph
who
were born
to him in Egypt,
all
the people
of the household
of Jacob
who were
in Egypt
numbered seventy.
\par }{\PP \VS{28}Jacob sent
Judah
before
him to
Joseph
to accompany him
to Goshen.
So they came
to the land
of Goshen.
\VS{29}Joseph
harnessed
his chariot
and went up
to meet
his father
Israel
in Goshen.
When he met him, he hugged
his neck
and wept
on
his neck
for quite some time.
\par }{\PP \VS{30}Israel
said
to
Joseph,
“Now let me die
since
I have seen
your face
and know that
you are still
alive.”
\VS{31}Then Joseph
said to
his brothers
and his father’s
household,
“I will go up
and tell
Pharaoh, ‘My brothers
and my father’s
household
who
were in the land
of Canaan
have come
to me.
\VS{32}The men
are shepherds;
they take care of livestock.
They have brought
their flocks
and their herds
and all
that they have.’
\VS{33}Pharaoh
will
summon
you and say,
‘What
is your occupation?’
\VS{34}Tell
him, ‘Your servants
have taken care
of cattle
from our youth
until
now,
both
we
and our fathers,’
so that you may
live
in the land
of Goshen,
for
everyone
who takes care
of sheep
is disgusting
to the Egyptians.”

\par }\Chap{47}{\PP \VerseOne{1}Joseph
went
and told
Pharaoh,
“My father,
my brothers,
their flocks
and herds,
and all
that
they own have arrived
from the land
of
\par }{\PP Canaan.
They are now
in the land
of Goshen.”
\VS{2}He took
five
of his brothers
and introduced
them to Pharaoh.
\par }{\PP \VS{3}Pharaoh
said
to
Joseph’s brothers,
“What
is your occupation?” They said
to
Pharaoh,
“Your servants
take care
of flocks,
just
as our
ancestors did.”
\VS{4}Then they said
to
Pharaoh,
“We
have come
to live as temporary residents
in the land.
There is no
pasture
for your servants’
flocks
because
the famine
is severe
in the land
of Canaan.
So now,
please
let your servants
live
in the land
of Goshen.”
\par }{\PP \VS{5}Pharaoh
said
to
Joseph,
“Your father
and your brothers
have come
to you.
\VS{6}The land
of Egypt
is before
you; settle
your father
and your brothers
in the best
region of the land.
They may live
in the land
of Goshen.
If
you know
of any highly capable men
among them, put
them in charge
of my livestock.”
\par }{\PP \VS{7}Then Joseph
brought
in his father
Jacob
and presented
him before
Pharaoh.
Jacob
blessed
Pharaoh.
\VS{8}Pharaoh
said
to
Jacob,
“How
long
have you lived?”
\VS{9}Jacob
said
to
Pharaoh,
“All
the years
of my travels
are 130.
All
the years
of my life
have been few
and painful;
the years
of my travels
are not
as long as those of my ancestors.”
\VS{10}Then Jacob
blessed
Pharaoh
and went out
from his
presence.
\par }{\PP \VS{11}So Joseph
settled
his father
and his brothers.
He gave
them territory
in the land
of Egypt,
in the best
region of the land,
the land
of Rameses,
just
as Pharaoh
had commanded.
\VS{12}Joseph
also provided
food
for his father,
his brothers,
and all
his father’s
household,
according
to the number of their little children.
\par }{\PP \VS{13}But there was no
food
in all
the land
because
the famine
was very
severe; the land
of Egypt
and the land
of Canaan
wasted away
because of the famine.
\VS{14}Joseph
collected
all
the money
that could be found
in the land
of Egypt
and in the land
of Canaan
as payment for the grain
they
were buying.
Then Joseph
brought
the money
into Pharaoh’s
palace.
\VS{15}When the money
from the lands
of Egypt
and Canaan
was used up,
all
the Egyptians
came
to Joseph
and said,
“Give
us food! Why
should we die
before
your very eyes because
our money
has run out?”
\par }{\PP \VS{16}Then Joseph
said,
“If
your money
is gone,
bring
your livestock,
and I will give
you food in exchange for your livestock.”
\VS{17}So they brought
their livestock
to
Joseph,
and Joseph
gave them
food
in exchange for their horses,
the livestock
of their flocks
and herds,
and their donkeys.
He got them through
that year
by giving them
food
in exchange
for livestock.
\par }{\PP \VS{18}When that year
was over, they came
to him
the next
year
and said
to him, “We cannot
hide
from our lord
that
the money
is used up and the livestock
and the animals
belong to
our lord.
Nothing
remains
before
our lord
except
our bodies
and our land.
\VS{19}Why
should we die
before your very eyes,
both
we
and our land? Buy
us and our land
in exchange for food,
and we,
with our land,
will become
Pharaoh’s
slaves.
Give
us seed
that we may live
and not
die.
Then the land
will not
become desolate.”
\par }{\PP \VS{20}So Joseph
bought
all
the land
of Egypt
for
Pharaoh.
Each of the Egyptians
sold
his field,
for
the famine
was severe.
So
the land
became Pharaoh’s.
\VS{21}Joseph made all the people
slaves from one end
of Egypt’s
border
to the other
end of it.
\VS{22}But
he did not
purchase
the land
of the priests
because
the priests
had
an allotment
from Pharaoh
and they ate
from their allotment
that
Pharaoh
gave
them.
That is why
they did not
sell
their land.
\par }{\PP \VS{23}Joseph
said
to
the people,
“Since
I have bought
you and your land
today
for Pharaoh,
here is
seed
for you. Cultivate
the land.
\VS{24}When
you gather in the crop,
give
one-fifth
of it to Pharaoh,
and the rest
will be
yours for seed
for the fields
and for you to eat,
including those in your households
and your little children.”
\VS{25}They replied,
“You have saved our lives! You are showing
us favor,
and we will be
Pharaoh’s
slaves.”
\par }{\PP \VS{26}So
Joseph
made it
a statute,
which is in effect
to this
day
throughout the land
of Egypt: One-fifth
belongs to Pharaoh.
Only
the land
of the priests
did not
become
Pharaoh’s.
\par }{\PP \VS{27}Israel
settled
in the land
of Egypt,
in the land
of Goshen,
and they owned
land there. They were fruitful
and increased
rapidly
in number.
\par }{\PP \VS{28}Jacob
lived
in the land
of Egypt
seventeen
years;
the years
of Jacob’s
life
were 147 in all.
\VS{29}The time
for Israel
to die
approached,
so he called
for his son
Joseph
and said
to him, “If
now
I have found
favor
in your sight,
put
your
hand
under
my thigh
and show
me kindness
and faithfulness.
Do not
bury
me
in Egypt,
\VS{30}but when
I rest
with
my fathers,
carry
me out of Egypt
and bury
me in their burial
place.” Joseph said,
“I
will do
as
you say.”
\par }{\PP \VS{31}Jacob said,
“Swear
to me that you will do
so.” So Joseph gave him his word. Then Israel
bowed
down at the head
of his bed.

\par }\Chap{48}{\PP \VerseOne{1}After
these
things
Joseph
was told, “Your father
is weakening.”
So he took
his two
sons
Manasseh
and Ephraim
with him.
\VS{2}When Jacob
was told, “Your son
Joseph
has just
come
to
you,” Israel
regained
strength and sat
up on
his bed.
\VS{3}Jacob
said
to
Joseph,
“The sovereign
God
appeared
to me
at Luz
in the land
of Canaan
and blessed me.
\VS{4}He said
to me,
‘I am going
to make you fruitful
and will multiply
you. I will make
you into a group
of nations,
and I will give
this
land
to your descendants
as an everlasting
possession.’
\par }{\PP \VS{5}“Now,
as for your two
sons,
who were born
to you in the land
of Egypt
before
I came
to
you in Egypt,
they
will be mine. Ephraim
and Manasseh
will be mine just as Reuben
and Simeon are.
\VS{6}Any children
that
you father
after
them will be yours; they will be listed
under the names
of their brothers
in their inheritance.
\VS{7}But as for me, when I
was returning
from Paddan,
Rachel
died – to my sorrow – in the land of Canaan. It happened along the way, some distance from Ephrath. So I buried her there on the way to Ephrath” (that is, Bethlehem).
\par }{\PP \VS{8}When Israel
saw
Joseph’s
sons,
he asked,
“Who
are these?”
\VS{9}Joseph
said
to
his father,
“They are
the sons
God
has
given
me in this
place.” His father said,
“Bring
them to me
so I may
bless them.”
\VS{10}Now Israel’s
eyes
were failing
because of his age;
he was not
able
to see well.
So Joseph brought
his sons near
to
him, and his father kissed
them and embraced them.
\VS{11}Israel
said
to
Joseph,
“I never
expected
to see
you again, but now
God
has allowed me to see
your children
too.”
\par }{\PP \VS{12}So
Joseph
moved
them from Israel’s knees
and bowed
down with his face
to the ground.
\VS{13}Joseph
positioned
them; he put Ephraim
on his right hand
across from Israel’s
left hand,
and Manasseh
on his left hand
across from Israel’s
right hand.
Then Joseph brought them closer
to his father.
\VS{14}Israel
stretched out
his right hand
and placed
it on
Ephraim’s
head,
although he was the younger.
Crossing his hands, he put his left hand
on
Manasseh’s
head,
for
Manasseh
was the firstborn.
\par }{\Q \VS{15}Then he blessed
Joseph
and said,
\par }{\Q “May the God
before
whom
my fathers
\par }{\Q Abraham
and Isaac
walked
–
\par }{\Q the God
who has been my shepherd
\par }{\Q all
my life
long
to this
day,
\par }{\Q \VS{16}the Angel
who has protected
me

\par }{\Q from all
harm
–
\par }{\Q bless
these boys.
\par }{\Q May my name
be named
in them,
\par }{\Q and the name
of my fathers
Abraham
and Isaac.
\par }{\Q May they grow
into a multitude
on the earth.”
\par }{\PP \VS{17}When Joseph
saw
that
his father
placed
his right hand
on
Ephraim’s
head,
it displeased
him. So he took
his father’s
hand
to move
it from Ephraim’s
head
to Manasseh’s
head.
\VS{18}Joseph
said
to
his father,
“Not
so,
my father,
for
this
is the firstborn.
Put
your right hand
on
his head.”
\par }{\PP \VS{19}But his father
refused
and said,
“I know,
my son,
I know.
He too
will become
a nation
and he too
will become great.
In spite
of this, his younger
brother
will be even greater
and his descendants
will become
a multitude
of nations.”
\VS{20}So he blessed
them that day,
saying,
\par }{\Q “By you will Israel
bless,
saying,
\par }{\Q ‘May God
make
you like Ephraim
and Manasseh.’ ”
\par }{\PP So he put
Ephraim
before
Manasseh.
\par }{\PP \VS{21}Then Israel
said to
Joseph,
“I
am
about to die,
but God
will be
with
you and will bring you back
to
the land
of your fathers.
\VS{22}As one
who is above
your brothers,
I
give
to you the mountain slope,
which
I took
from
the Amorites
with my sword
and my bow.”

\par }\Chap{49}{\PP \VerseOne{1}Jacob
called
for his sons
and said,
“Gather together
so I can tell
you what
will happen
to you in the future.
\par }{\Q \VS{2}“Assemble
and listen,
you sons
of Jacob;
\par }{\Q listen
to
Israel,
your father.
\par }{\Q \VS{3}Reuben,
you
are my firstborn,
\par }{\Q my might
and the beginning
of my strength,
\par }{\Q outstanding in dignity, outstanding in power.
\par }{\Q \VS{4}You are destructive
like water
and will not
excel,
\par }{\Q for
you got
on your father’s
bed,
\par }{\Q then
you defiled it – he got on my couch!
\par }{\Q \VS{5}Simeon
and Levi
are brothers,
\par }{\Q weapons
of violence
are their knives!
\par }{\Q \VS{6}O my
soul,
do not
come
into their council,
\par }{\Q do
not
be united
to their assembly,
my heart,
\par }{\Q for
in their anger
they have killed
men,
\par }{\Q and for pleasure
they have hamstrung
oxen.
\par }{\Q \VS{7}Cursed
be their anger,
for
it was fierce,
\par }{\Q and their fury,
for
it was cruel.
\par }{\Q I will divide
them in Jacob,
\par }{\Q and scatter
them in Israel!
\par }{\Q \VS{8}Judah,
your
brothers
will praise
you.
\par }{\Q Your hand
will be on the neck
of your enemies,
\par }{\Q your father’s
sons
will bow down before you.
\par }{\Q \VS{9}You are a lion’s
cub,
Judah,
\par }{\Q from the prey,
my son,
you have gone up.
\par }{\Q He crouches
and lies down like a lion;
\par }{\Q like a lioness
– who
will rouse him?
\par }{\Q \VS{10}The scepter
will not
depart from
Judah,
\par }{\Q nor the ruler’s staff
from between
his feet,
\par }{\Q until
he
comes
to whom it belongs;

\par }{\Q the nations
will obey him.
\par }{\Q \VS{11}Binding
his foal
to the vine,
\par }{\Q and his colt
to the choicest vine,
\par }{\Q he will wash
his garments
in wine,
\par }{\Q his robes
in the blood
of grapes.
\par }{\Q \VS{12}His eyes
will be dark
from wine,
\par }{\Q and his teeth
white
from milk.
\par }{\Q \VS{13}Zebulun
will live
by the haven
of the sea
\par }{\Q and become a haven
for ships;
\par }{\Q his border will extend to Sidon.
\par }{\Q \VS{14}Issachar
is a strong-boned
donkey
\par }{\Q lying down
between
two saddlebags.
\par }{\Q \VS{15}When he sees
a good
resting place,
\par }{\Q and the pleasant
land,
\par }{\Q he will bend
his shoulder
to the burden
\par }{\Q and become
a slave
laborer.
\par }{\Q \VS{16}Dan
will judge
his people
\par }{\Q as one
of the tribes
of Israel.
\par }{\Q \VS{17}May
Dan
be a snake
beside
the road,
\par }{\Q a viper
by the path,
\par }{\Q that bites
the heels
of the horse
\par }{\Q so that its rider
falls
backward.
\par }{\Q \VS{18}I wait
for your deliverance,
O
{\ND{Lord}}.
\par }{\Q \VS{19}Gad
will be raided
by marauding bands,
\par }{\Q but he will attack
them at their heels.
\par }{\Q \VS{20}Asher’s
food
will be rich,
\par }{\Q and he
will provide
delicacies
to royalty.
\par }{\Q \VS{21}Naphtali
is a free running
doe,
\par }{\Q he speaks
delightful
words.
\par }{\Q \VS{22}Joseph
is a fruitful
bough,
\par }{\Q a fruitful
bough near a spring
\par }{\Q whose branches
climb over the wall.
\par }{\Q \VS{23}The archers
will attack
him,

\par }{\Q they will shoot
at him and oppose him.
\par }{\Q \VS{24}But his bow
will remain
steady,
\par }{\Q and his hands
will be skillful;
\par }{\Q because of the hands
of the Mighty One
of Jacob,
\par }{\Q because
of the Shepherd,
the Rock
of Israel,
\par }{\Q \VS{25}because of the God
of your father,
\par }{\Q who will help
you,

\par }{\Q because of the sovereign
God,

\par }{\Q who will bless
you

\par }{\Q with blessings
from the sky
above,
\par }{\Q blessings
from the deep
that lies
below,
\par }{\Q and blessings
of the breasts
and womb.
\par }{\Q \VS{26}The blessings
of your father
are greater
\par }{\Q than
the blessings
of the eternal
mountains
\par }{\Q or the desirable things
of the age-old
hills.
\par }{\Q They will be
on the head
of Joseph
\par }{\Q and on the brow
of the prince
of his brothers.
\par }{\Q \VS{27}Benjamin
is a ravenous
wolf;
\par }{\Q in the morning
devouring
the prey,
\par }{\Q and in the evening
dividing
the plunder.”
\par }{\PP \VS{28}These
are the twelve
tribes
of Israel.
This
is what
their father
said
to them when he blessed
them. He gave each
of them an appropriate
blessing.
\par }{\PP \VS{29}Then he instructed
them, “I
am about to
go
to my people.
Bury
me with
my fathers
in
the cave
in the field
of Ephron
the Hittite.
\VS{30}It is the cave
in the field
of Machpelah,
near
Mamre
in the land
of Canaan,
which
Abraham
bought
for a burial
plot
from Ephron
the Hittite.
\VS{31}There
they buried
Abraham
and his wife
Sarah;
there
they buried
Isaac
and his wife
Rebekah;
and there
I buried
Leah.
\VS{32}The field
and the cave
in it were acquired
from the sons
of Heth.”
\par }{\PP \VS{33}When Jacob
finished
giving these instructions
to his sons,
he pulled
his feet
up onto
the bed,
breathed
his last breath,
and went
to
his people.

\par }\Chap{50}{\PP \VerseOne{1}Then
Joseph
hugged
his father’s
face.
He wept
over
him and kissed him.
\VS{2}Joseph
instructed
the physicians
in his service to embalm
his father,
so the physicians
embalmed
Israel.
\VS{3}They took
forty
days,
for
that is
the full
time
needed for embalming.
The Egyptians
mourned for him seventy
days.
\par }{\PP \VS{4}When the days
of mourning
had passed,
Joseph
said
to
Pharaoh’s
royal court, “If
I have found
favor
in your sight,
please
say
to Pharaoh,
\VS{5}‘My father
made me swear
an oath. He said, “I am
about to die.
Bury
me in my tomb
that
I dug
for myself there
in the land
of Canaan.”
Now
let
me go
and bury
my father;
then I will return.’ ”
\VS{6}So Pharaoh
said,
“Go
and bury
your father,
just
as he made you swear to do.”
\par }{\PP \VS{7}So Joseph
went up
to bury
his father;
all
Pharaoh’s
officials
went
with him – the senior courtiers of his household, all the senior officials of the land of Egypt,
\VS{8}all
Joseph’s
household,
his brothers,
and his father’s
household.
But
they left
their little children
and their flocks
and herds
in the land
of Goshen.
\VS{9}Chariots
and horsemen
also
went up
with
him, so it was a very
large
entourage.
\par }{\PP \VS{10}When they came
to
the threshing floor
of Atad
on the other side
of the Jordan,
they mourned
there
with very
great
and bitter
sorrow.
There Joseph observed
a seven
day
period of mourning
for his father.
\VS{11}When the Canaanites
who lived
in the land
saw
them mourning
at the threshing floor
of Atad,
they said,
“This
is a very
sad
occasion for
the Egyptians.”
That is why
its name
was called
Abel Mizraim,
which
is beyond
the Jordan.
\par }{\PP \VS{12}So
the sons
of Jacob did
for him just
as he had
instructed them.
\VS{13}His sons
carried
him to the
land
of Canaan
and buried
him in the cave
of the field
of Machpelah,
near
Mamre.
This is the field
Abraham
purchased
as a burial
plot
from Ephron
the Hittite.
\VS{14}After
he buried
his father,
Joseph
returned
to Egypt,
along with
his brothers
and all
who had accompanied
him to bury
his father.
\par }{\PP \VS{15}When Joseph’s
brothers
saw
that
their father
was dead,
they said,
“What if
Joseph
bears
a grudge
and wants to repay us in full for all
the harm
we
did
to him?”
\VS{16}So
they sent
word to Joseph,
saying,
“Your father
gave these instructions
before
he died:
\VS{17}‘Tell
Joseph
this: Please
forgive
the sin
of your brothers
and the wrong
they did when they treated
you so badly.’ Now
please
forgive
the sin of the servants
of the God
of your father.”
When this message was reported
to him,
Joseph
wept.
\VS{18}Then
his brothers
also
came and threw
themselves down before
him; they said,
“Here
we are; we are your slaves.”
\VS{19}But Joseph
answered
them,
“Don’t
be afraid.
Am
I in the place
of God?
\VS{20}As for you,
you meant
to harm
me, but God
intended
it for a good
purpose, so
he
could preserve
the lives of many
people,
as you can see this
day.
\VS{21}So now,
don’t
be afraid.
I
will provide
for you and your little children.”
Then he consoled
them and spoke
kindly to them.
\par }{\PP \VS{22}Joseph
lived
in Egypt,
along with his father’s
family.
Joseph
lived
110
years.
\VS{23}Joseph
saw
the descendants
of Ephraim
to the third generation.
He also
saw the children of
Makir
the son
of Manasseh;
they were
given special
inheritance rights by Joseph.
\par }{\PP \VS{24}Then Joseph
said to
his brothers,
“I am
about to die.
But God
will surely
come to you and lead
you up
from
this
land
to
the land
he swore
on oath to give to Abraham,
Isaac,
and Jacob.”
\VS{25}Joseph
made the sons
of Israel
swear
an oath.
He said,
“God
will surely
come
to you. Then
you must
carry my bones
up
from this place.”
\VS{26}So Joseph
died
at the age of 110.
After they embalmed
him, his body was placed
in a coffin
in Egypt.
\par }