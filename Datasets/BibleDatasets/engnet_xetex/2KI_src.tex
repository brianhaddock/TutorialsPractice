\NormalFont\ShortTitle{2 Kings}
{\MT 2 Kings

\par }\ChapOne{1}{\SH Elijah Confronts the King and His Commanders
\par }{\PP \VerseOne{1}After
Ahab
died,
Moab
rebelled
against Israel.
\VS{2}Ahaziah
fell
through
a window lattice
in his upper chamber
in Samaria
and was injured.
He sent
messengers
with these orders, “Go,
ask
Baal Zebub,
the god
of Ekron,
if
I will survive
this
injury.”
\par }{\PP \VS{3}But the
{\ND{Lord}}’s
angelic messenger
told
Elijah
the Tishbite,
“Get
up,
go to meet
the messengers
from the king
of Samaria.
Say
this to
them: ‘You must think
there is no
God
in Israel! That explains why you
are on your way
to seek
an oracle from Baal Zebub
the god
of Ekron.
\VS{4}Therefore
this is what
the {\ND{Lord}}
says,
“You will not
leave
the bed
you lie on, for
you will certainly
die!” ’ ” So Elijah
went on his way.
\par }{\PP \VS{5}When the messengers
returned
to
the king,
he asked
them,
“Why
have you returned?”
\VS{6}They replied, “A man
came up
to
meet
us. He told
us,
“Go
back
to
the king
who
sent
you and tell
him, ‘This is what
the {\ND{Lord}}
says: “You must think there is no
God
in Israel! That explains why you
are sending
for an oracle
from Baal Zebub,
the god
of Ekron.
Therefore
you will not
leave
the bed
you lie on, for
you will certainly
die.” ’ ”
\VS{7}The king asked
them, “Describe
the appearance of this man
who
came up
to meet
you and told
you these
things.”
\VS{8}They replied, “He was a hairy
man
and had a leather
belt
tied around
his waist.”
The king said,
“He is Elijah
the Tishbite.”
\par }{\PP \VS{9}The king sent
a captain
and his fifty
soldiers to retrieve Elijah. The captain went up
to him,
while
he was sitting
on
the top
of a hill.
He told
him,
“Prophet,
the king
says,
‘Come down!’ ”
\VS{10}Elijah
replied
to
the captain, “If
I
am indeed a prophet,
may fire
come down
from
the sky
and consume
you and your fifty
soldiers!” Fire
then came down
from the
sky
and consumed
him and his fifty soldiers.
\par }{\PP \VS{11}The king
sent
another
captain
and his fifty soldiers to retrieve Elijah. He went up
and told
him, “Prophet,
this is what
the king
says,
‘Come down
at once!’ ”
\VS{12}Elijah
replied
to
them, “If
I am
indeed a prophet,
may fire
come down
from
the sky
and consume
you and your fifty
soldiers!” Fire
from God
came down
from
the sky
and consumed
him
and his fifty soldiers.
\par }{\PP \VS{13}The king
sent
a third
captain
and his fifty
soldiers. This third
captain
went
up
and fell
on
his knees
before
Elijah.
He begged
for mercy, “Prophet,
please
have respect
for my life
and for the lives
of these
fifty
servants
of yours.
\VS{14}Indeed,
fire
came down
from
the sky
and consumed
the two
captains
who came before me, along with their men.
So now,
please have respect
for my life.”
\VS{15}The
{\ND{Lord}}’s
angelic
messenger
said
to Elijah,
“Go down
with
him. Don’t
be afraid
of him.” So he got
up and went down
with
him to
the king.
\par }{\PP \VS{16}Elijah said
to
the king, “This is what the
{\ND{Lord}}
says,
‘You sent
messengers
to seek
an oracle from Baal Zebub,
the god
of Ekron.
You must think
there is no
God
in Israel
from whom
you can seek
an oracle! Therefore
you will not
leave
the bed
you lie
on, for
you will certainly
die.’ ”
\par }{\PP \VS{17}He died
just
as the
{\ND{Lord}}
had
prophesied
through Elijah.
In the second
year
of the reign
of King Jehoram
son
of Jehoshaphat
over Judah,
Ahaziah’s brother Jehoram
replaced
him as king
of Israel, because
he had no
son.
\VS{18}The rest
of the events
of Ahaziah’s
reign, including
his accomplishments,
are recorded
in the scroll
called the Annals
of the Kings
of Israel.

\par }\Chap{2}{\PP \VerseOne{1}Just before the
{\ND{Lord}}
took Elijah
up
to heaven
in a windstorm,
Elijah
and Elisha
were traveling
from
Gilgal.
\VS{2}Elijah
told
Elisha,
“Stay
here,
for
the {\ND{Lord}}
has sent
me to Bethel.”
But Elisha
said,
“As certainly as the
{\ND{Lord}}
lives
and as you live,
I will not leave
you.” So they went down
to Bethel.
\VS{3}Some members
of the prophetic guild
in Bethel
came out
to
Elisha
and said,
“Do you know
that
today
the {\ND{Lord}}
is going to take
your master
from you?” He answered,
“Yes,
I
know.
Be quiet.”
\par }{\PP \VS{4}Elijah
said
to him, “Elisha,
stay
here,
for
the {\ND{Lord}}
has sent
me to Jericho.”
But he replied,
“As certainly as the
{\ND{Lord}}
lives
and as you live,
I will not leave
you.” So they went
to Jericho.
\VS{5}Some members of the prophetic guild
in Jericho
approached
Elisha
and said,
“Do you know
that
today
the {\ND{Lord}}
is going to take
your master
from you?” He answered,
“Yes,
I
know.
Be quiet.”
\par }{\PP \VS{6}Elijah
said
to him, “Stay
here,
for
the {\ND{Lord}}
has sent
me to the Jordan.”
But he replied,
“As certainly as the
{\ND{Lord}}
lives
and as you live,
I will not leave
you.” So they traveled on together.
\VS{7}The fifty
members
of the prophetic guild
went
and stood
opposite
them at a distance,
while Elijah and Elisha stood
by the Jordan.
\VS{8}Elijah
took
his cloak,
folded
it up, and hit
the water
with it. The water divided,
and the two
of them crossed over
on dry ground.
\par }{\PP \VS{9}When
they had crossed
over, Elijah
said
to
Elisha,
“What
can I do
for you, before
I am taken
away
from you?” Elisha
answered,
“May
I receive a double
portion of the prophetic spirit
that energizes you.”
\VS{10}Elijah replied,
“That’s a difficult
request! If
you see
me taken
from you, may it be
so,
but if
you don’t,
it will not
happen.”
\par }{\PP \VS{11}As they
were walking along
and talking,
suddenly
a fiery chariot
pulled by fiery
horses
appeared. They went between
Elijah and Elisha, and Elijah
went up
to heaven
in a windstorm.
\VS{12}While Elisha
was watching,
he
was crying out,
“My father,
my father! The chariot
and horsemen
of Israel!” Then he could no
longer
see
him. He grabbed
his clothes
and tore
them in two.
\VS{13}He picked up
Elijah’s
cloak,
which
had fallen off
him, and went back
and stood
on
the shore
of the Jordan.
\VS{14}He took
the
cloak
that had
fallen
off Elijah,
hit
the
water
with it, and said,
“Where
is the
{\ND{Lord}},
the God
of Elijah?” When he hit
the water,
it divided
and Elisha
crossed over.
\par }{\PP \VS{15}When
the members of the prophetic guild
in Jericho,
who
were standing at a distance,
saw him do this, they said,
“The spirit
that energized Elijah
rests
upon
Elisha.”
They went
to meet
him and bowed
down to the ground before him.
\VS{16}They said
to him,
“Look,
there
are fifty
capable
men
with
your servants.
Let
them go
and look
for your master,
for the wind
sent from the
{\ND{Lord}}
may have carried
him away and dropped
him on one
of the hills
or
in one
of the valleys.”
But Elisha replied,
“Don’t
send them out.”
\VS{17}But they were so insistent,
he became embarrassed.
So he said,
“Send
them out.” They sent
the fifty
men
out and they looked
for three
days,
but could not
find Elijah.
\VS{18}When they came
back,
Elisha was staying
in Jericho.
He said
to
them, “Didn’t
I tell
you, ‘Don’t
go’?”
\par }{\SH Elisha Demonstrates His Authority
\par }{\PP \VS{19}The men
of the city
said to
Elisha,
“Look,
the city
has a good
location,
as
our master
can
see.
But the water
is bad
and the land
doesn’t produce crops.”
\VS{20}Elisha said,
“Get
me a new
jar
and put
some
salt
in it.” So
they got it.
\VS{21}He went out
to
the spring
and threw
the salt
in. Then he said,
“This is what
the {\ND{Lord}}
says,
‘I have purified
this
water.
It will
no
longer
cause death
or fail to produce crops.”
\VS{22}The water
has been pure
to this very
day,
just
as Elisha
prophesied.
\par }{\PP \VS{23}He went up
from there
to Bethel.
As he
was traveling up
the road,
some young
boys
came out
of the city
and made fun
of him, saying,
“Go on up,
baldy! Go on up,
baldy!”
\VS{24}When he turned
around
and saw
them,
he called
God’s judgment
down on them. Two
female bears
came out
of the woods
and ripped
forty-two
of the boys to pieces.
\VS{25}From there
he traveled
to
Mount
Carmel
and then back
to Samaria.

\par }\Chap{3}{\PP \VerseOne{1}In the eighteenth
year
of King
Jehoshaphat’s
reign over Judah,
Ahab’s
son
Jehoram
became king
over
Israel
in Samaria;
he ruled
for twelve
years.
\VS{2}He did
evil
in the sight
of the
{\ND{Lord}}, but not
to the same degree
as his father
and mother.
He did remove
the sacred pillar
of Baal
that
his father
had
made.
\VS{3}Yet he persisted
in the sins
of Jeroboam
son
of Nebat,
who encouraged
Israel
to sin;
he did not
turn
from them.
\par }{\PP \VS{4}Now King
Mesha
of Moab
was
a sheep breeder.
He would send as tribute
to the king
of Israel
100,000
male lambs
and the wool
of 100,000
rams.
\VS{5}When
Ahab
died,
the king
of Moab
rebelled
against the king
of Israel.
\VS{6}At that time
King
Jehoram
left
Samaria
and assembled
all
Israel for war.
\VS{7}He sent
this message to
King
Jehoshaphat
of Judah: “The king
of Moab
has rebelled against
me. Will you fight
with
me against
Moab?” Jehoshaphat replied,
“I will join
you in the campaign; my army
and horses are at your disposal.”
\VS{8}He then asked, “Which invasion
route
are we going to take?” Jehoram answered,
“By the road
through the Desert
of Edom.”
\VS{9}So
the kings
of Israel,
Judah,
and Edom
set out together. They wandered around
on the road
for seven
days
and finally
ran out of water
for the men
and animals
they had
with them.
\VS{10}The king
of Israel
said,
“Oh
no! Certainly
the
{\ND{Lord}}
has summoned
these
three
kings
so
that he can hand
them over to the king of Moab!”
\VS{11}Jehoshaphat
asked,
“Is there no
prophet
of the {\ND{Lord}}
here
that we might seek
the
{\ND{Lord}}’s
direction?” One
of the servants
of the king
of Israel
answered,
“Elisha
son
of Shapat
is here;
he used to be Elijah’s servant.”
\VS{12}Jehoshaphat
said,
“The
{\ND{Lord}}
speaks
through him.” So the king
of Israel
and Jehoshaphat
and the king
of Edom
went down
to visit him.
\par }{\PP \VS{13}Elisha
said
to
the king
of Israel,
“Why
are you here? Go
to
your father’s
prophets
or your mother’s
prophets!” The king
of Israel
replied
to him, “No,
for
the {\ND{Lord}}
is the one who summoned
these
three
kings
so
that he can hand
them over to Moab.”
\VS{14}Elisha
said,
“As certainly as the
{\ND{Lord}}
who rules over all

lives
(whom
I serve), if
I did not respect
King
Jehoshaphat
of Judah,
I
would not
pay attention to you or
acknowledge you.
\VS{15}But now,
get
me a musician.”
When
the musician
played,
the {\ND{Lord}}
energized him,
\VS{16}and he said,
“This is what
the {\ND{Lord}}
says,
‘Make
many
cisterns
in this
valley,’
\VS{17}for
this is what
the {\ND{Lord}}
says, ‘You will not
feel
any wind
or
see
any rain,
but this valley
will be full
of water
and you
and your cattle
and animals
will drink.’
\VS{18}This
is an easy
task for the
{\ND{Lord}}; he will
also
hand
Moab
over to you.
\VS{19}You will defeat
every
fortified
city
and every
important
city.
You must chop down
every
productive
tree,
stop up
all
the springs,
and cover
all
the cultivated
land
with stones.”
\par }{\PP \VS{20}Sure enough, the next
morning,
at
the time
of the morning
sacrifice,
water
came
flowing
down
from Edom
and filled
the land.
\VS{21}Now all
Moab
had heard
that
the kings
were attacking,
so everyone
old enough
to fight
was mustered
and placed
at
the border.
\VS{22}When they got up early
the next morning,
the sun
was shining
on
the water.
To the Moabites,
who were some distance
away, the water
looked red
like blood.
\VS{23}The Moabites said,
“It’s
blood! The kings
are totally
destroyed!
 They have struck
one
another
down! Now,
Moab,
seize the plunder!”
\VS{24}When they approached
the Israelite
camp,
the Israelites
rose up
and struck down
the Moabites,
who then ran
from
them. The Israelites thoroughly defeated
Moab.
\VS{25}They tore down
the cities
and each man
threw
a stone
into every
cultivated
field
until they were covered.
They stopped up
every
spring
and chopped down
every
productive
tree.
\par }{\PP Only
Kir Hareseth
was left
intact,
but the slingers
surrounded
it and attacked it.
\VS{26}When
the king
of Moab
realized
he was losing
the battle,
he and 700
swordsmen
tried
to
break through
and attack
the king
of Edom,
but
they failed.
\VS{27}So he took
his firstborn
son,
who
was to succeed him as king,
and offered
him up
as a burnt sacrifice
on
the wall.
There
was an outburst
of divine anger against
Israel,
so they broke
off the attack
and
returned
to their homeland.

\par }\Chap{4}{\PP \VerseOne{1}Now
a wife
of one
of the prophets
appealed
to
Elisha
for help,
saying,
“Your servant,
my husband
is dead.
You
know
that
your servant
was
a loyal follower
of the {\ND{Lord}}. Now the creditor
is coming
to take
away my two
boys
to be his servants.”
\VS{2}Elisha
said
to
her, “What
can I do
for you? Tell
me, what
do you have
in the house?” She answered,
“Your servant
has nothing
in the house
except
a small jar
of olive oil.”
\VS{3}He said,
“Go
and ask
all
your neighbors
for empty
containers.
Get as many as you can.
\VS{4}Go
and close
the door
behind
you and your sons.
Pour
the olive oil into all
the containers;
set
aside each one when you have filled it.”
\VS{5}So she left
him and closed
the door
behind
her and her
sons.
As they
were bringing the containers to her,
she
was pouring the olive oil.
\VS{6}When
the containers
were full,
she said
to one of her
sons, “Bring
me
another
container.”
But he answered
her,
“There
are no
more.”
Then
the olive oil
stopped
flowing.
\VS{7}She went
and told
the prophet.
He said,
“Go,
sell
the olive oil.
Repay
your creditor,
and then you
and your sons
can live
off the rest of the profit.”
\par }{\SH Elisha Gives Life to a Boy
\par }{\PP \VS{8}One day
Elisha
traveled
to
Shunem,
where
a prominent
woman
lived. She insisted
that he stop for a meal.
So
whenever
he was passing through,
he would stop
in there
for a meal.
\VS{9}She said
to
her husband,
“Look,
I’m
sure
that
the man
who regularly
passes
through
here is a very special
prophet.
\VS{10}Let’s make
a small
private
upper room
and furnish
it with a bed,
table,
chair,
and lamp.
When he visits
us, he can stay
there.”
\par }{\PP \VS{11}One day
Elisha came
for a visit;
he went into the upper
room and rested.
\VS{12}He told
his servant
Gehazi,
“Ask the Shunammite
woman to come here.” So he did so and she came to him.
\VS{13}Elisha said
to Gehazi, “Tell
her,
‘Look,
you have treated
us with
such
great respect.
What
can I do
for you? Can
I put in a good word
for you with the king
or
the commander
of the army?’ ” She replied,
“I’m
quite
secure.”
\VS{14}So he asked
Gehazi, “What
can I do
for her?” Gehazi
replied,
“She has no
son,
and her husband
is old.”
\VS{15}Elisha told
him, “Ask her to come here.” So he did so and she came and stood
in the doorway.
\VS{16}He said,
“About this
time
next year
you
will be holding
a son.”
She said,
“No,
my master! O
prophet,
do not
lie
to your servant!”
\VS{17}The woman
did conceive,
and at
the specified
time
the next year she gave birth
to a son,
just
as Elisha
had told
her.
\par }{\PP \VS{18}The boy
grew
and one day
he went out
to
see his father
who was with the harvest workers.
\VS{19}He said
to
his father,
“My head! My head!” His father told
a servant,
“Carry
him to
his mother.”
\VS{20}So he picked
him up
and took
him to
his mother.
He sat
on
her lap
until
noon
and then died.
\VS{21}She
went up
and laid
him down
on
the prophet’s
bed.
She shut
the door behind
her and left.
\VS{22}She called
to
her husband,
“Send
me
one
of the servants
and one
of the donkeys,
so
I can go see
the prophet
quickly
and then return.”
\VS{23}He said,
“Why
do you
want to go
see him
today? It is not
the new moon
or
the Sabbath.”
She said,
“Everything’s fine.”
\VS{24}She saddled
the donkey
and told
her
servant,
“Lead
on.
Do not
stop unless
I say so.”
\par }{\PP \VS{25}So she went
to visit
the prophet
at Mount
Carmel.
When
he saw
her at a distance,
he said
to
his servant
Gehazi,
“Look,
it’s
the Shunammite woman.
\VS{26}Now,
run
to meet
her and ask
her, ‘Are you well? Are your husband
and the boy
well?’ ” She told
Gehazi, “Everything’s fine.”
\VS{27}But when she reached
the prophet
on
the mountain,
she grabbed hold
of his feet.
Gehazi
came near
to push
her away,
but the prophet
said,
“Leave
her alone,
for
she is very upset.
The
{\ND{Lord}}
has kept the matter hidden
from
me; he didn’t
tell me about it.”
\VS{28}She said,
“Did I ask
my master
for a son? Didn’t
I say, ‘Don’t mislead me?’ ”
\VS{29}Elisha told
Gehazi,
“Tuck
your robes into your belt,
take
my staff,
and go! Don’t
stop to exchange greetings with anyone! Place
my staff
on
the child’s
face.”
\VS{30}The mother
of the child
said,
“As certainly as the
{\ND{Lord}}
lives
and as you live,
I will not leave
you.” So Elisha got
up and followed
her back.
\par }{\PP \VS{31}Now Gehazi
went on
ahead
of them. He placed
the staff
on
the child’s
face,
but there was no
sound
or
response.
When
he came
back
to Elisha
he told
him, “The child
did not
wake up.”
\VS{32}When Elisha
arrived
at the house,
there
was the child
lying
dead
on
his bed.
\VS{33}He went
in by himself
and closed
the door.
Then he prayed
to
the {\ND{Lord}}.
\VS{34}He got up
on
the bed
and spread his body out over
the boy;
he put
his mouth
on
the boy’s mouth,
his eyes
over
the boy’s eyes,
and the palms
of his hands against
the boy’s palms.
He bent
down over him,
and the boy’s
skin
grew warm.
\VS{35}Elisha went back
and walked
around
in the house.
Then he got up
on the bed again and bent down
over
him. The child
sneezed
seven
times
and opened
his eyes.
\VS{36}Elisha called
to
Gehazi
and said,
“Get the Shunammite
woman.” So he did so and she came
to him.
He said
to her, “Take
your son.”
\VS{37}She came
in, fell
at his feet,
and bowed
down. Then she picked
up her son
and left.
\par }{\SH Elisha Makes a Meal Edible
\par }{\PP \VS{38}Now Elisha
went back
to Gilgal,
while there was famine
in the land.
Some of the prophets
were visiting
him and he told
his servant,
“Put
the big
pot
on
the fire and boil
some stew
for the prophets.”
\VS{39}Someone
went out
to
the field
to gather
some herbs
and found
a wild
vine.
He picked
some
of its fruit,
enough to fill
up the fold of his robe.
He came back,
cut it up,
and threw the slices into
the stew
pot,
not
knowing they were harmful.
\VS{40}The stew was poured
out for the men
to eat.
When
they ate
some of the stew,
they
cried out,
“Death
is in the pot,
O
prophet!” They could
not
eat it.
\VS{41}He said,
“Get
some flour.”
Then he threw
it into
the pot
and said,
“Now pour
some out for the men
so they may eat.”
There was
no
longer anything
harmful
in the pot.
\par }{\SH Elisha Miraculously Feeds a Hundred People
\par }{\PP \VS{42}Now a man
from Baal Shalisha
brought
some food
for the prophet –
twenty
loaves of bread
made from the firstfruits
of the barley
harvest, as well as fresh ears of grain.
Elisha said,
“Set it before
the people
so they may eat.”
\VS{43}But his attendant
said,
“How
can I
feed a hundred
men
with this?” He replied,
“Set it before
the people
so they may eat,
for
this is what
the {\ND{Lord}}
says,
‘They will eat
and have some left over.’ ”
\VS{44}So
he set
it before
them; they ate
and had some left
over, just
as the
{\ND{Lord}} predicted.

\par }\Chap{5}{\PP \VerseOne{1}Now Naaman,
the commander
of the king
of Syria’s
army,
was
esteemed
and respected
by his master,
for
through him the
{\ND{Lord}}
had given
Syria
military victories.
But this great
warrior
had a skin disease.
\VS{2}Raiding parties
went out
from Syria
and took captive
from the land
of Israel
a young
girl,
who became a servant
to Naaman’s
wife.
\VS{3}She told
her mistress,
“If only
my master
were in the presence
of the prophet
who
is in Samaria! Then
he would cure
him of his skin disease.”
\par }{\PP \VS{4}Naaman went
and told
his master
what
the girl
from the land
of Israel
had said.
\VS{5}The king
of Syria
said,
“Go! I will send
a letter
to
the king
of Israel.”
So Naaman went,
taking
with him ten
talents
of silver,
six
thousand
shekels of gold,
and ten
suits
of clothes.
\VS{6}He brought
the letter
to
king
of Israel.
It read: “This is a letter
of introduction
for my servant
Naaman,
whom I have sent
to
be cured
of his skin disease.”
\VS{7}When
the king
of Israel
read
the letter,
he tore
his clothes
and said,
“Am I God? Can I
kill
or restore life? Why does he ask
me to
cure
a man
of his skin disease? Certainly
you must
see
that
he is looking for
an excuse to fight me!”
\par }{\PP \VS{8}When
Elisha
the prophet
heard
that
the king
had torn
his clothes,
he sent
this message to the king,
“Why
did you tear
your clothes? Send him to me so
he may
know
there
is a prophet
in Israel.”
\VS{9}So Naaman
came
with his horses
and chariots
and stood
in the doorway
of Elisha’s
house.
\VS{10}Elisha
sent
out a messenger
who told
him, “Go
and wash
seven
times
in the Jordan;
your skin
will be restored
and you will be healed.”
\VS{11}Naaman
went away
angry.
He said,
“Look,
I thought
for sure he would come
out,
stand
there, invoke
the name
of the {\ND{Lord}}
his God,
wave
his hand
over
the area,
and cure
the skin disease.
\VS{12}The rivers
of Damascus,
the Abana
and Pharpar,
are better
than
any
of the waters
of Israel! Could I not
wash
in them and be healed?” So he turned around
and went away
angry.
\VS{13}His servants
approached
and said
to him,
“O master,
if
the prophet
had told
you to
do
some difficult task,
you would have been willing to do
it. It seems you should be happy that he simply said,
“Wash
and you will be healed.”
\VS{14}So he went down
and dipped
in the Jordan
seven
times,
as the prophet
had instructed.
His skin
became
as smooth
as a young
child’s
and he was healed.
\par }{\PP \VS{15}He
and his entire
entourage
returned
to
the prophet.
Naaman came
and stood
before
him. He said,
“For sure
I know
that
there is no
God
in all
the earth
except
in Israel! Now,
please
accept
a gift
from your servant.”
\VS{16}But Elisha replied,
“As certainly as the
{\ND{Lord}}
lives
(whom
I serve), I will
take
nothing from you.” Naaman insisted
that he take
it, but he refused.
\VS{17}Naaman
said,
“If not,
then
please
give
your servant
a load
of dirt,
enough for
a pair
of mules
to carry,
for
your servant
will never
again
offer a burnt offering
or sacrifice
to a god
other
than the
{\ND{Lord}}.
\VS{18}May the
{\ND{Lord}}
forgive
your servant
for this
one thing: When my master
enters
the temple
of Rimmon
to worship,
and he
leans
on
my arm
and I bow down
in the temple
of Rimmon,
may
the {\ND{Lord}}
forgive
your servant
for this.”
\VS{19}Elisha said
to him, “Go
in peace.”
\par }{\PP When he had gone
a short distance,
\VS{20}Gehazi,
the prophet
Elisha’s
servant,
thought, “Look,
my master
did not accept
what this
Syrian
Naaman
offered him. As certainly as the
{\ND{Lord}}
lives,
I will run
after
him and accept
something from him.”
\VS{21}So Gehazi
ran after
Naaman.
When Naaman
saw
someone running
after
him, he got down
from his chariot
to meet
him and asked,
“Is everything all right?”
\VS{22}He answered,
“Everything
is fine. My master
sent
me with this message, ‘Look,
two
servants
of the prophets
just
arrived
from the Ephraimite
hill country.
Please
give
them a talent
of silver
and two
suits
of clothes.’ ”
\VS{23}Naaman
said,
“Please
accept
two talents
of silver. He insisted,
and tied up
two talents
of silver
in two
bags,
along with two
suits
of clothes.
He gave
them to
two
of his servants
and they carried
them for Gehazi.
\VS{24}When he arrived
at the hill,
he took
them from
the servants
and put them
in the house.
Then he sent
the
men
on their way.
\par }{\PP \VS{25}When he came
and stood
before his master,
Elisha
asked
him,
“Where
have you been, Gehazi?” He answered,
“Your servant
hasn’t
been anywhere.”
\VS{26}Elisha replied,
“I was there in spirit when a man
turned
and got down from his chariot
to meet
you. This is not
the proper
time
to accept
silver
or to accept
clothes,
olive groves,
vineyards,
sheep,
cattle,
and male and female
servants.
\VS{27}Therefore Naaman’s
skin disease
will afflict
you and your descendants
forever!” When Gehazi went out
from his presence,
his skin
was as white as snow.

\par }\Chap{6}{\PP \VerseOne{1}Some
of the prophets
said
to
Elisha,
“Look,
the place
where
we
meet with you is too cramped for us.
\VS{2}Let’s
go
to
the Jordan.
Each
of us will get
a log
from
there
and we will build a meeting place
for ourselves
there.”
He said,
“Go.”
\VS{3}One
of them said,
“Please
come
along with
your servants.”
He replied,
“All right, I’ll
come.”
\VS{4}So he went
with
them. When they arrived
at the Jordan,
they started cutting
down trees.
\VS{5}As one
of them was felling
a log,
the ax head
dropped
into
the water.
He shouted,
“Oh
no, my master! It was
borrowed!”
\VS{6}The prophet
asked,
“Where
did it drop
in?” When
he
showed him
the spot, Elisha cut off
a branch,
threw
it in at
that spot, and made the ax head
float.
\VS{7}He said,
“Lift
it out.” So he reached
out his hand
and grabbed it.
\par }{\SH Elisha Defeats an Army
\par }{\PP \VS{8}Now the king
of Syria
was at war
with Israel.
He consulted
his advisers,
who said,
“Invade
at such and such
a place.”
\VS{9}But the prophet
sent
this message to
the king
of Israel,
“Make sure
you don’t pass
through this
place
because
Syria
is invading
there.”
\VS{10}So the king
of Israel
sent
a message to
the place
the prophet
had pointed out, warning
it to be on its guard.
This happened on several occasions.
\VS{11}This
made
the king
of Syria
upset.
So he summoned
his advisers
and said
to
them, “One
of us must
be helping
the king
of Israel.”
\VS{12}One
of his advisers
said,
“No,
my master,
O king.
The prophet
Elisha
who
lives in Israel
keeps telling
the king
of Israel
the things
you say
in your bedroom.”
\VS{13}The king ordered,
“Go,
find out
where
he is, so I can send
some men to capture
him.” The king was told,
“He is in Dothan.”
\VS{14}So he sent
horses
and chariots
there,
along with a good-sized
army.
They arrived
during the night
and surrounded
the city.
\par }{\PP \VS{15}The prophet’s
attendant
got up
early
in the morning. When he went outside
there was an army
surrounding
the city,
along with horses
and chariots.
He said
to
Elisha, “Oh
no, my master! What
will we do?”
\VS{16}He replied,
“Don’t
be afraid,
for
our
side outnumbers them.”
\VS{17}Then Elisha
prayed,
“O
{\ND{Lord}}, open
his eyes
so he can
see.”
The
{\ND{Lord}}
opened
the servant’s
eyes
and he saw
that the hill
was full
of horses
and chariots
of fire
all around
Elisha.
\VS{18}As they approached
him,
Elisha
prayed
to
the {\ND{Lord}}, “Strike
these
people
with blindness.”
The
{\ND{Lord}} struck
them with blindness
as Elisha
requested.
\VS{19}Then Elisha
said
to them,
“This
is not
the right road
or
city.
Follow
me, and I will lead
you to
the man
you’re
looking
for.” He led
them to Samaria.
\par }{\PP \VS{20}When
they had entered
Samaria,
Elisha
said,
“O
{\ND{Lord}}, open
their eyes,
so they can
see.”
The
{\ND{Lord}}
opened
their eyes
and they saw
that they were
in the middle
of Samaria.
\VS{21}When the king
of Israel
saw
them, he asked
Elisha,
“Should I strike
them down,
my master?”
\VS{22}He replied,
“Do not
strike
them down! You did not capture
them with your sword
or bow,
so what
gives you
the right to strike
them down? Give
them some food
and water,
so they can eat
and drink
and then go
back to
their master.”
\VS{23}So he threw
a big
banquet
for them
and they
ate
and drank.
Then he sent
them back
to
their master.
After that no
Syrian
raiding parties
again
invaded
the land
of Israel.
\par }{\SH The Lord Saves Samaria
\par }{\PP \VS{24}Later
King
Ben Hadad
of Syria
assembled
his entire
army
and attacked
and besieged
Samaria.
\VS{25}Samaria’s
food supply ran out.
They laid siege
to
it so long
that a donkey’s
head
was
selling for eighty
shekels of silver
and a quarter
of a kab
of dove’s droppings
for five shekels
of silver.
\par }{\PP \VS{26}While
the king
of Israel
was passing
by on
the city wall,
a woman
shouted
to him,
“Help
us, my master,
O king!”
\VS{27}He replied,
“No,
let the
{\ND{Lord}}
help
you. How
can I help
you? The threshing floor
and
winepress are empty.”
\VS{28}Then the king
asked
her, “What’s
your problem?” She answered,
“This
woman
said
to me,
‘Hand
over your son;
we’ll eat
him today
and then
eat
my son
tomorrow.’
\VS{29}So we boiled
my son
and ate
him. Then I said
to
her the next
day,
‘Hand
over your son
and we’ll eat
him.’ But she hid
her son!”
\VS{30}When
the king
heard
what
the woman
said, he tore
his clothes.
As he was passing
by on
the wall,
the people
could see
he was wearing sackcloth
under his clothes.
\VS{31}Then he said,
“May
God
judge me severely
if
Elisha
son
of Shaphat
still
has his head
by the end of the day!”
\par }{\PP \VS{32}Now Elisha
was sitting
in his house
with
the community leaders.
The king sent
a messenger
on ahead,
but before
he arrived,
Elisha said
to
the leaders, “Do you realize
this
assassin
intends
to cut off
my head?” Look,
when the messenger
arrives,
shut
the door
and lean against
it. His master
will certainly be right behind him.”
\VS{33}He was still
talking
to them when
the messenger
approached
and said,
“Look,
the

{\ND{Lord}}
is responsible for this
disaster! Why
should I continue
to wait
for the
{\ND{Lord}} to help?”

\par }\Chap{7}{\PP \VerseOne{1}Elisha
replied,
“Hear
the word
of the {\ND{Lord}}! This is what
the {\ND{Lord}}
says,
‘About this time
tomorrow
a seah
of finely milled flour
will sell for a shekel
and two seahs
of barley
for a shekel
at the gate
of Samaria.’ ”
\VS{2}An officer
who
was the king’s
right-hand
man responded
to the prophet, “Look,
even if the
{\ND{Lord}}
made
it rain by opening holes
in the sky,
could this
happen
so
soon?” Elisha said,
“Look,
you will see
it happen with your own eyes,
but you will not
eat
any of the food!”
\par }{\PP \VS{3}Now four
men
with a skin disease
were sitting at the entrance
of the city gate.
They said
to
one
another,
“Why
are we
just sitting
here
waiting
to die?
\VS{4}If
we go
into the city,
we’ll die
of starvation,
and if
we
stay
here
we’ll die! So come on, let’s
defect
to
the Syrian
camp! If
they spare
us, we’ll live;
if
they kill us – well, we were going to die anyway.”
\VS{5}So they started
toward
the Syrian
camp
at dusk.
When
they reached
the edge
of the Syrian
camp,
there
was no
one
there.
\VS{6}The
{\ND{Lord}}
had caused the Syrian
camp
to hear
the sound
of chariots
and horses
and a large
army. Then they said
to one
another,
“Look,
the king
of Israel
has paid
the
kings
of the Hittites
and Egypt
to
attack us!”
\VS{7}So they got
up and fled
at dusk,
leaving
behind their tents,
horses,
and donkeys.
They left
the camp
as
it was and ran
for their lives.
\VS{8}When the men with a skin disease
reached
the edge
of the camp,
they entered
a
tent
and had a meal.
They also took
some silver,
gold,
and clothes
and went
and hid
it all. Then they went back
and entered
another
tent.
They looted
it and went
and hid what they had taken.
\VS{9}Then they said
to
one
another,
“It’s not
right
what we’re
doing! This
is a day
to celebrate,
but we
haven’t
told anyone. If we wait
until
dawn,
we’ll be
punished. So come on, let’s
go
and inform
the royal
palace.”
\VS{10}So they went
and called
out to
the gatekeepers
of the city.
They told
them,
“We entered
the Syrian
camp
and there
was no
one
there.
We didn’t even hear a man’s
voice.
But
the horses
and donkeys
are still tied up,
and the tents remain up.”
\VS{11}The gatekeepers
relayed
the news to the royal
palace.
\par }{\PP \VS{12}The king
got
up in the night
and said
to
his advisers, “I will tell
you what the Syrians
have
done
to us. They know
we
are starving,
so they left
the camp
and hid
in the field,
thinking,
‘When
they come out
of the city,
we will capture
them alive
and enter
the city.’ ”
\VS{13}One
of his advisers
replied,
“Pick
some men
and have them take
five
of the horses
that
are left
in the city. (Even
if they are
killed, their fate
will be
no different
than
that
of all
the Israelite people – we’re all going to die!) Let’s send them out so we can know for sure what’s going on.”
\VS{14}So they picked
two
horsemen
and the king
sent
them out
to track
the Syrian
army.
He ordered
them, “Go
and find out what’s going on.”
\VS{15}So
they tracked
them as far
as the Jordan.
The road
was filled
with clothes
and equipment
that
the Syrians
had discarded
in their haste.
The scouts
went back
and told
the king.
\VS{16}Then the people
went out
and looted
the Syrian
camp.
A seah
of finely milled flour
sold for a shekel,
and two seahs
of barley
for a shekel,
just as the
{\ND{Lord}}
had said they would.
\par }{\PP \VS{17}Now the king
had placed
the officer
who
was his right-hand
man at the city gate.
When the people
rushed out, they trampled
him to death
in the gate.
This
fulfilled the
prophet’s
word
which
he had spoken
when the king
tried to arrest him.
\VS{18}The prophet
told
the king,
“Two seahs
of barley
will sell for a shekel,
and a seah
of finely milled flour
for a shekel;
this will happen
about this time
tomorrow
in the gate
of Samaria.”
\VS{19}But
the officer
replied
to the prophet,
“Look,
even if the
{\ND{Lord}}
made
it rain by opening holes
in the sky,
could
this
happen so soon?” Elisha said,
“Look,
you will see
it happen with your own eyes,
but you will not
eat
any of the food!”
\VS{20}This is exactly
what happened
to him.
The people
trampled
him to death
in the city gate.

\par }\Chap{8}{\PP \VerseOne{1}Now Elisha
advised
the woman
whose
son
he had
brought back
to
life,
“You
and your family
should go
and live
somewhere
else for
a while, for the
{\ND{Lord}}
has
decreed that a famine
will overtake
the land
for seven
years.”
\VS{2}So
the woman
did
as
the prophet
said.
She
and her family
went
and lived
in the land
of the Philistines
for seven
years.
\VS{3}After
seven
years
the woman
returned
from the land
of the Philistines
and went
to ask
the king
to
give her
back her
house
and field.
\VS{4}Now the king
was talking
to
Gehazi,
the prophet’s
servant,
and said,
“Tell
me
all
the great
things which
Elisha
has done.”
\VS{5}While Gehazi was telling
the king
how
Elisha
had brought the dead
back to life,
the woman
whose
son
he had brought back to life came to
ask
the king
for her house
and field.
Gehazi
said,
“My master,
O king,
this
is the very woman
and this
is her son
whom
Elisha
brought back to life!”
\VS{6}The king
asked
the woman
about
it, and she gave
him the details. The king
assigned a eunuch
to take care of her request and ordered
him, “Give her back
everything
she owns,
as well as the amount of crops
her field
produced from the day
she left
the
land
until
now.”
\par }{\SH Elisha Meets with Hazael
\par }{\PP \VS{7}Elisha
traveled
to Damascus
while King
Ben Hadad
of Syria
was sick.
The king was told,
“The prophet
has
come here.”
\VS{8}So the king
told
Hazael,
“Take
a gift
and go
visit
the prophet.
Request from him an oracle
from the
{\ND{Lord}}. Ask
him, ‘Will I recover
from this
sickness?’ ”
\VS{9}So
Hazael
went
to visit
Elisha. He took
along a gift,
as well as forty
camel
loads
of all
the fine things
of Damascus.
When he arrived,
he stood
before
him and said,
“Your son,
King
Ben Hadad
of Syria,
has sent
me to
you with this
question, ‘Will I recover
from this
sickness?’ ”
\VS{10}Elisha
said
to him,
“Go
and tell
him, ‘You will surely
recover,’
but
the {\ND{Lord}}
has revealed
to me that
he will surely
die.”
\VS{11}Elisha just stared at him until
Hazael
became uncomfortable.
Then the prophet
started crying.
\VS{12}Hazael
asked,
“Why
are you crying,
my master?” He replied,
“Because
I know
the
trouble
you will cause the Israelites.
You will set
fire
to their fortresses,
kill
their
young men
with the sword,
smash their children to bits,
and rip
open
their pregnant women.”
\VS{13}Hazael
said,
“How
could your servant,
who is as insignificant as a dog,
accomplish
this
great
military
victory?” Elisha
answered,
“The
{\ND{Lord}}
has revealed
to me
that you
will be the king
of Syria.”
\VS{14}He left
Elisha
and went
to his master.
Ben Hadad asked
him, “What
did Elisha
tell
you?” Hazael replied,
“He told
me you would surely
recover.”
\VS{15}The next
day Hazael took
a piece of cloth,
dipped
it in water,
and spread
it over
Ben Hadad’s face
until he died.
Then Hazael
replaced
him as king.
\par }{\SH Jehoram’s Reign over Judah
\par }{\PP \VS{16}In the fifth
year
of the reign
of Israel’s
King
Joram,
son
of Ahab,
Jehoshaphat’s
son
Jehoram
became king over
Judah.
\VS{17}He was thirty-two
years
old when he became
king and he reigned
for eight
years
in Jerusalem.
\VS{18}He followed
in the footsteps
of the kings
of Israel,
just
as Ahab’s
dynasty
had done,
for
he married
Ahab’s
daughter.
He did
evil
in the sight
of the
{\ND{Lord}}.
\VS{19}But the
{\ND{Lord}}
was unwilling
to destroy
Judah.
He preserved Judah
for the sake
of his servant
David
to whom
he had promised
a perpetual
dynasty.
\par }{\PP \VS{20}During
his reign Edom
freed
themselves from Judah’s
control and set up
their own king.
\VS{21}Joram
crossed
over to Zair
with all
his chariots.
The Edomites,
who had surrounded
him,
attacked
at night
and defeated
him
and his chariot
officers.
The Israelite army
retreated
to their homeland.
\VS{22}So Edom
has remained free
from
Judah’s
control
to
this very day.
At that same time
Libnah
also rebelled.
\par }{\PP \VS{23}The rest
of the events
of Joram’s
reign, including
a record
of his accomplishments,
are recorded
in the scroll
called the Annals
of the Kings
of Judah.
\VS{24}Joram
passed away
and was buried
with
his ancestors
in the city
of David.
His son
Ahaziah
replaced
him as king.
\par }{\SH Ahaziah Takes the Throne of Judah
\par }{\PP \VS{25}In the twelfth
year
of the reign
of Israel’s
King
Joram,
son
of Ahab,
Jehoram’s
son
Ahaziah
became king
over Judah.
\VS{26}Ahaziah
was twenty-two
years
old
when he became king
and he reigned
for one
year
in Jerusalem.
His mother
was
Athaliah,
the granddaughter
of King
Omri
of Israel.
\VS{27}He followed
in the footsteps
of Ahab’s
dynasty
and did
evil
in the sight
of the
{\ND{Lord}}, like Ahab’s
dynasty,
for
he was
related
to Ahab’s
family.
\par }{\PP \VS{28}He joined
Ahab’s
son
Joram
in a battle
against
King
Hazael
of Syria
at Ramoth
Gilead
in which the Syrians
defeated
Joram.
\VS{29}King
Joram
returned
to Jezreel
to recover
from
the wounds
he received
from the Syrians
in Ramah
when he fought
against King
Hazael
of Syria.
King
Ahaziah
son
of Jehoram
of Judah
went down
to visit
Joram
son
of Ahab
in Jezreel,
for
he was ill.

\par }\Chap{9}{\PP \VerseOne{1}Now Elisha
the prophet
summoned
a member
of the prophetic guild
and told
him, “Tuck
your robes into your belt,
take
this
container
of olive oil
in your hand,
and go
to Ramoth
Gilead.
\VS{2}When you arrive
there,
look
for Jehu
son
of Jehoshaphat
son
of Nimshi
and take
him aside
into an inner room.
\VS{3}Take
the container
of olive oil,
pour
it over
his head,
and say,
‘This is what
the {\ND{Lord}}
says,
“I have designated
you as king
over Israel.” ’
Then open
the door
and run
away quickly!”
\par }{\PP \VS{4}So the young
prophet
went
to Ramoth
Gilead.
\VS{5}When he arrived,
the officers
of the army
were sitting
there. So he said,
“I have a message
for you, O officer.”
Jehu
asked,
“For which
one of us?” He replied,
“For you, O officer.”
\VS{6}So Jehu got
up and went
inside. Then the prophet poured
the olive oil
on his head
and said
to him, “This is what
the {\ND{Lord}}
God
of Israel
says, ‘I have designated
you as king
over the
{\ND{Lord}}’s
people
Israel.
\VS{7}You will destroy
the
family
of your master
Ahab.
I will get revenge
against Jezebel
for the shed blood
of my servants
the prophets
and for the shed blood
of all
the
{\ND{Lord}}’s
servants.
\VS{8}Ahab’s
entire
family
will die.
I will cut off
every last male
belonging to Ahab
in Israel,
including even the weak
and incapacitated.
\VS{9}I will make
Ahab’s
dynasty
like those of Jeroboam
son
of Nebat
and Baasha
son
of Ahijah.
\VS{10}Dogs
will devour
Jezebel
on the plot
of ground in Jezreel;
she will not
be buried.’ ”
Then he opened
the door
and ran away.
\par }{\PP \VS{11}When Jehu
rejoined
his master’s
servants,
they asked
him, “Is everything
all right? Why
did this
madman
visit
you?” He replied,
“Ah, it’s not important. You
know
what kind of man
he is and the kinds of things
he says.”
\VS{12}But they said,
“You’re lying! Tell
us
what he said.” So
he told them
what he had said. He also related how he had said, “This is what
the {\ND{Lord}}
says, ‘I have designated
you as king
over Israel.’ ”
\VS{13}Each
of them
quickly
took
off his cloak
and they spread them out at Jehu’s
feet on the steps.
The trumpet
was blown and they shouted,
“Jehu
is king!”
\VS{14}Then Jehu
son
of Jehoshaphat
son
of Nimshi
conspired
against Joram.
\par }{\SH Jehu the Assassin
\par }{\PP Now Joram
had been
in Ramoth
Gilead
with the whole
Israelite
army, guarding
against an invasion
by King
Hazael
of Syria.
\VS{15}But King
Joram
had returned
to Jezreel
to recover
from
the wounds
he received
from the Syrians
when he fought
against King
Hazael
of Syria.
Jehu
told
his supporters, “If
you really
want
me to be king, then
don’t
let anyone
escape
from
the city
to go
and warn
Jezreel.”
\VS{16}Jehu
drove
his chariot
to Jezreel,
for
Joram
was recuperating
there.
(Now King
Ahaziah
of Judah
had come down
to visit
Joram.)
\par }{\PP \VS{17}Now the watchman
was standing
on
the tower
in Jezreel
and saw
Jehu’s
troops
approaching.
He said,
“I
see
troops!” Jehoram
ordered, “Send
a rider
out to meet
them and have him ask,
‘Is everything all right?’ ”
\VS{18}So
the horseman
went to meet
him and said,
“This is what
the king
says,
‘Is everything all right?’ ” Jehu
replied,
“None of your business! Follow
me.” The watchman
reported,
“The messenger
reached
them,
but hasn’t
started back.”
\VS{19}So he sent
a second
horseman
out
to
them and he said,
“This is what
the king
says,
‘Is everything all right?’ ” Jehu
replied,
“None of your business! Follow me.”
\VS{20}The watchman
reported,
“He reached
them, but hasn’t
started back.
The one who drives
the lead chariot drives
like Jehu
son
of Nimshi;
he drives
recklessly.”
\VS{21}Jehoram
ordered,
“Hitch up
my chariot.”
When his chariot had been hitched up,
King
Jehoram
of Israel
and King
Ahaziah
of Judah
went out
in their respective
chariots
to meet
Jehu.
They met up
with him in the plot of land
that had once belonged to Naboth
of Jezreel.
\par }{\PP \VS{22}When
Jehoram
saw
Jehu,
he asked,
“Is everything all right,
Jehu?” He replied,
“How
can everything be all right
as long as
your mother
Jezebel
promotes idolatry
and pagan practices?”
\VS{23}Jehoram
turned
his chariot around and took off.
He said
to
Ahaziah,
“It’s a trap,
Ahaziah!”
\VS{24}Jehu
aimed
his bow
and shot
an arrow right between
Jehoram’s
shoulders. The arrow
went
through his heart
and he fell to his knees
in his chariot.
\VS{25}Jehu ordered
his officer
Bidkar,
“Pick
him up
and throw
him into the part
of the field
that
once belonged to Naboth
of Jezreel.
Remember,
you
and I
were riding
together behind
his father
Ahab,
when the
{\ND{Lord}}
pronounced
this
judgment on him,
\VS{26}‘ “Know for sure
that I saw
the
shed
blood
of Naboth
and his sons
yesterday,”
says
the {\ND{Lord}}, “and that I will give
you what
you deserve
right here in this
plot
of land,” says
the {\ND{Lord}}.’ So now
pick
him up and throw
him into this plot
of land, just
as the
{\ND{Lord}} said.”
\par }{\PP \VS{27}When King
Ahaziah
of Judah
saw
what happened, he took off
up the road
to Beth Haggan. Jehu
chased
him and ordered,
“Shoot him too.”
They shot
him while he was driving his chariot
up the ascent
of Gur
near Ibleam.
He fled
to Megiddo
and died
there.
\VS{28}His servants
took
his body
back to Jerusalem
and buried
him in his tomb
with
his ancestors
in the city
of David.
\VS{29}Ahaziah
had become king
over
Judah
in the eleventh
year
of Joram
son
of Ahab.
\par }{\PP \VS{30}Jehu
approached
Jezreel.
When Jezebel
heard
the news, she put
on some eye
liner,
fixed up
her hair,
and leaned out
the window.
\VS{31}When Jehu
came
through the gate,
she said,
“Is everything
all right, Zimri,
murderer
of his master?”
\VS{32}He looked
up at
the window
and said,
“Who
is on my side? Who?” Two
or three
eunuchs
looked
down at him.
\VS{33}He said,
“Throw
her down!” So they threw
her down,
and when
she hit the ground,
her blood
splattered against the wall
and the horses, and Jehu drove his chariot over her.
\VS{34}He went
inside and had a meal.
Then he said,
“Dispose
of this
accursed
woman’s corpse. Bury
her, for
after all, she was a king’s
daughter.”
\VS{35}But when they went
to bury
her, they found
nothing
left but
the skull,
feet,
and palms
of the hands.
\VS{36}When they went back
and told
him, he said,
“The
{\ND{Lord}}’s
word
through
his servant,
Elijah
the Tishbite,
has come to pass. He warned, ‘In the plot
of land at Jezreel,
dogs
will devour
Jezebel’s
flesh.
\VS{37}Jezebel’s
corpse
will be
like manure
on
the surface
of the ground
in the plot
of land at Jezreel.
People
will not
be able to even recognize
her.’ ”

\par }\Chap{10}{\PP \VerseOne{1}Ahab
had seventy
sons
living in Samaria.
So Jehu
wrote
letters
and sent
them to
Samaria
to the leading
officials
of Jezreel
and to
the guardians
of Ahab’s
dynasty. This is what the letters said,
\VS{2}“You have with
you the sons
of your master,
chariots
and horses,
a fortified
city,
and weapons.
So when this
letter
arrives,
\VS{3}pick
the best
and most capable
of your master’s
sons,
place
him on
his father’s
throne,
and defend
your master’s
dynasty.”
\par }{\PP \VS{4}They were absolutely
terrified
and said,
“Look,
two
kings
could not
stop
him! How
can
we?”
\VS{5}So the palace
supervisor, the city
commissioner, the leaders,
and the guardians
sent
this message to
Jehu,
“We
are your subjects! Whatever
you say,
we will do.
We will not
make anyone
king.
Do
what you consider
proper.”
\par }{\PP \VS{6}He wrote
them
a second
letter,
saying,
“If
you
are really
on my side and are willing
to obey
me, then take
the
heads
of
your master’s
sons
and come
to
me in Jezreel
at this time
tomorrow.”
Now the king
had seventy
sons,
and the prominent
men
of the city
were raising
them.
\VS{7}When
they received
the letter,
they seized
the king’s
sons
and executed
all seventy
of them. They
put
their heads
in baskets
and sent
them to
him in Jezreel.
\VS{8}The messenger
came
and told
Jehu, “They have brought
the heads
of the king’s
sons.”
Jehu said,
“Stack
them
in two
piles
at the entrance
of the city gate
until
morning.”
\VS{9}In the morning
he went out
and stood
there. Then he said
to
all
the people,
“You
are innocent.
I
conspired
against
my master
and killed
him. But who
struck
down all
of these men?
\VS{10}Therefore take note
that
not
one of the judgments the
{\ND{Lord}}
announced
against
Ahab’s
dynasty
has failed to materialize.
The
{\ND{Lord}}
had done
what
he announced
through
his servant
Elijah.”
\VS{11}Then Jehu
killed
all
who were left
of Ahab’s
family
in Jezreel,
and all
his nobles,
close friends,
and priests.
He left
no survivors.
\par }{\PP \VS{12}Jehu then left
there and set out
for Samaria.
While he
was traveling
through Beth Eked
of the Shepherds,
\VS{13}Jehu
encountered
the relatives
of King
Ahaziah
of Judah.
He asked,
“Who
are you?” They replied,
“We are Ahaziah’s
relatives.
We
have come down
to see how
the king’s
sons
and the queen mother’s
sons are doing.”
\VS{14}He said,
“Capture
them alive!” So they captured
them alive
and then executed
all forty-two
of them
in
the cistern
at Beth Eked.
He left
no
survivors.
\par }{\PP \VS{15}When he left
there,
he
met
Jehonadab,
son
of Rekab,
who had
been looking for him. Jehu greeted him
and asked, “Are you
as
committed
to me
as
I am to you?” Jehonadab
answered, “I
am!” Jehu replied,
“If
so,
give
me your hand.”
So he
offered
his hand
and Jehu pulled him
up into
the chariot.
\VS{16}Jehu said,
“Come
with
me and see
how zealous
I am for the
{\ND{Lord}}’s
cause.”
So he took him along in his chariot.
\VS{17}He went
to Samaria
and exterminated
all
the members of Ahab’s
family who were still alive in Samaria,
just
as the
{\ND{Lord}}
had
announced
to
Elijah.
\par }{\SH Jehu Executes the Prophets and Priests of Baal
\par }{\PP \VS{18}Jehu
assembled
all
the people
and said
to
them, “Ahab
worshiped
Baal
a little;
Jehu
will worship
him with great devotion.
\VS{19}So now,
bring
to me all
the prophets
of Baal,
as well as all
his servants
and priests.
None
of them must be absent,
for
I am offering a great
sacrifice
to Baal.
Any
of them who fail to appear
will lose their lives.”
But Jehu
was tricking
them so
he could destroy
the
servants
of Baal.
\VS{20}Then Jehu
ordered, “Make arrangements
for a celebration
for Baal.”
So
they announced it.
\VS{21}Jehu
sent
invitations
throughout
Israel,
and all
the servants
of Baal
came;
not
one
was absent.
They arrived
at the temple
of Baal
and filled it up from end to end.
\VS{22}Jehu ordered
the one who was in charge
of the wardrobe, “Bring out
robes
for all
the servants of Baal.”
So he brought out
robes for them.
\VS{23}Then Jehu
and Jehonadab
son
of Rekab
went
to the temple
of Baal.
Jehu said
to the servants
of Baal,
“Make sure
there
are no servants
of the {\ND{Lord}}
here
with you; there must be only
servants
of Baal.”
\VS{24}They went
inside to offer
sacrifices
and burnt offerings.
Now Jehu
had stationed
eighty
men
outside.
He
had told
them, “If any of the men
inside
get away,
you will pay with
your lives!”
\par }{\PP \VS{25}When
he finished
offering the burnt sacrifice,
Jehu
ordered
the royal guard
and officers,
“Come
in and strike
them down! Don’t
let any escape!” So the royal guard
and officers
struck
them down with the sword
and left their bodies lying
there. Then
they
entered
the inner sanctuary of the temple
of Baal.
\VS{26}They hauled out
the sacred pillar
of the temple
of Baal
and burned it.
\VS{27}They demolished
the sacred pillar
of Baal
and the temple
of Baal;
it is used as
a latrine
to
this very
day.
\VS{28}So Jehu
eradicated
Baal
worship from Israel.
\par }{\SH A Summary of Jehu’s Reign
\par }{\PP \VS{29}However,
Jehu
did not
repudiate
the sins
which
Jeroboam
son
of Nebat
had
encouraged
Israel
to commit;
the golden
calves
remained in Bethel
and Dan.
\VS{30}The
{\ND{Lord}}
said
to
Jehu,
“You have
done
well.
You have
accomplished
my will and carried out my wishes
with regard
to Ahab’s
dynasty.
Therefore four generations
of your descendants
will rule
over
Israel.”
\VS{31}But Jehu
did not
carefully
and wholeheartedly
obey
the law
of the {\ND{Lord}}
God
of Israel.
He did not
repudiate
the sins
which Jeroboam
had
encouraged
Israel
to commit.
\par }{\PP \VS{32}In those
days
the {\ND{Lord}}
began
to reduce
the size of Israel’s
territory. Hazael
attacked
their eastern border.
\VS{33}He conquered
all
the land
of Gilead,
including the territory of Gad,
Reuben,
and Manasseh,
extending
all the way from the Aroer
in
the Arnon
Valley
through Gilead
to Bashan.
\par }{\PP \VS{34}The rest
of the events
of Jehu’s
reign, including all
his accomplishments
and successes, are
recorded
in the scroll
called the Annals
of the Kings
of Israel.
\VS{35}Jehu
passed away
and was buried
in Samaria.
His son
Jehoahaz
replaced
him as king.
\VS{36}Jehu
reigned
over
Israel
for twenty-eight
years
in Samaria.

\par }\Chap{11}{\PP \VerseOne{1}When Athaliah
the mother
of Ahaziah
saw
that
her son
was dead,
she was determined
to destroy
the entire
royal
line.
\VS{2}So Jehosheba,
the daughter
of King
Joram
and sister
of Ahaziah,
took
Ahaziah’s
son
Joash
and sneaked
him away
from the rest
of the royal
descendants
who were to be executed.
She hid him and his nurse
in the room
where the bed covers
were stored. So he was hidden
from
Athaliah
and escaped execution.
\VS{3}He hid
out with
his nurse in the
{\ND{Lord}}’s
temple
for six
years,
while Athaliah
was ruling
over
the land.
\par }{\PP \VS{4}In the seventh
year
Jehoiada
summoned
the officers
of the units of hundreds
of the Carians
and the royal bodyguard.
He met
with
them in the
{\ND{Lord}}’s
temple.
He made
an agreement
with
them and made them swear
an oath of allegiance in the
{\ND{Lord}}’s
temple.
Then he showed
them the king’s
son.
\VS{5}He ordered
them, “This
is what
you must do.
One third
of the unit
that is on duty
during the Sabbath
will guard
the royal
palace.
\VS{6}Another third
of you will be stationed at the Foundation
Gate.
Still another third
of you will be stationed at the gate
behind
the royal guard.
You will take turns
guarding
the palace.
\VS{7}The two
units
who are off duty
on the Sabbath
will guard
the
{\ND{Lord}}’s
temple
and protect
the king.
\VS{8}You must surround
the king.
Each
of you must hold his weapon
in his hand.
Whoever approaches
your ranks
must be killed.
You must accompany
the king
wherever he goes.”
\par }{\PP \VS{9}The officers
of the units of hundreds
did
just
as Jehoiada
the priest
ordered.
Each
of them took
his men,
those who were on duty
during the Sabbath
as well as
those who were off duty
on the Sabbath,
and reported
to
Jehoiada
the priest.
\VS{10}The priest
gave
to the officers
of the units of hundreds
King
David’s
spears
and the shields
that
were kept in the
{\ND{Lord}}’s
temple.
\VS{11}The royal bodyguard
took
their stations,
each holding his weapon
in his hand.
They lined
up from the south
side
of the temple
to the north
side
and stood near the altar
and the temple,
surrounding
the
king.
\VS{12}Jehoiada led out
the king’s
son
and placed
on
him the crown
and the royal insignia.
They proclaimed him
king
and poured
olive oil on his head. They clapped
their hands
and cried
out, “Long live
the king!”
\par }{\PP \VS{13}When Athaliah
heard
the royal guard
shout,
she joined
the crowd
at the
{\ND{Lord}}’s
temple.
\VS{14}Then she saw
the king
standing
by the pillar,
according
to custom.
The officers
stood beside the king
with their trumpets
and all
the people
of the land
were celebrating
and blowing
trumpets.
Athaliah
tore
her clothes
and screamed,
“Treason,
treason!”
\VS{15}Jehoiada
the priest
ordered
the
officers
of the units of hundreds,
who were
in charge of the army, “Bring
her outside
the temple
to the guards.
Put
the sword
to
anyone
who follows her.” The priest
gave this order because
he had decided she should not
be executed
in the
{\ND{Lord}}’s
temple.
\VS{16}They seized
her and took
her into the precincts
of the royal
palace
through the horses’
entrance.
There
she was executed.
\par }{\PP \VS{17}Jehoiada
then drew
up a covenant
between
the {\ND{Lord}}
and the king
and people,
stipulating
that they should
be
loyal to the
{\ND{Lord}}.
\VS{18}All
the people
of the land
went
and demolished
the temple
of Baal.
They smashed
its altars
and idols
to bits.
They killed
Mattan
the priest
of Baal
in front
of the altar.
Jehoiada the priest
then placed
guards
at
the
{\ND{Lord}}’s
temple.
\VS{19}He took
the
officers
of the
units of hundreds,
the Carians,
the royal bodyguard,
and all
the people
of land,
and together they led the king
down
from the
{\ND{Lord}}’s
temple.
They entered
the royal
palace
through
the Gate
of the Royal Bodyguard,
and the king
sat
down on
the royal
throne.
\VS{20}All
the people
of the land
celebrated,
for the city
had rest
now that they had killed
Athaliah
with the sword
in the royal
palace.
\par }{\SH Joash’s Reign over Judah
\par }{\PP \VS{21} Jehoash
was seven
years
old when he began to reign.

\par }\Chap{12}{\PP \VerseOne{1}In Jehu’s
seventh
year
Jehoash
became king;
he reigned
for forty
years
in Jerusalem.
His mother
was Zibiah,
who was from Beer Sheba.
\VS{2}Throughout
his lifetime
Jehoash
did what the
{\ND{Lord}}
approved,
just
as Jehoiada
the priest
taught him.
\VS{3}But
the high places
were not
eliminated;
the people
continued
to offer sacrifices and burn incense
on the high places.
\par }{\PP \VS{4}Jehoash
said
to
the priests,
“I place at your disposal all
the consecrated
silver
that has been
brought
to the
{\ND{Lord}}’s
temple,
including the silver
collected
from the census
tax, the silver
received from
those who have made vows,
and all
the silver
that people
have voluntarily
contributed
to the
{\ND{Lord}}’s
temple.
\VS{5}The priests
should receive
the silver they need
from the treasurers
and repair
any
damage
to the temple
they discover.”
\par }{\PP \VS{6}By the twenty-third
year
of King
Jehoash’s
reign the priests
had still not
repaired
the damage
to the temple.
\VS{7}So King
Jehoash
summoned
Jehoiada
the priest
along with the other priests,
and said
to
them, “Why
have you not
repaired
the damage
to the temple? Now,
take
no
more silver
from your treasurers
unless
you intend to use it to repair
the damage.”
\VS{8}The priests
agreed
not
to collect
silver
from the people
and relieved
themselves of personal responsibility for the temple
repairs.
\par }{\PP \VS{9}Jehoiada
the priest
took
a chest
and drilled
a hole
in its lid.
He placed
it on the right side
of the altar
near
the entrance
of the
{\ND{Lord}}’s
temple.
The priests
who
guarded
the
entrance would put into
it all
the silver
brought
to the
{\ND{Lord}}’s
temple.
\VS{10}When
they saw
the chest
was full of silver,
the royal
secretary
and the high priest
counted
the silver
that had been
brought to the
{\ND{Lord}}’s
temple
and bagged
it up.
\VS{11}They would then hand
over the
silver
that had been weighed
to the construction
foremen
assigned
to the
{\ND{Lord}}’s
temple.
They hired
carpenters
and builders
to work
on the
{\ND{Lord}}’s
temple,
\VS{12}as well as masons
and stonecutters.
They bought
wood
and chiseled
stone
to repair
the damage
to the
{\ND{Lord}}’s
temple
and also paid
for
all the other expenses.
\VS{13}The silver
brought
to the
{\ND{Lord}}’s
temple
was not
used
for silver
bowls,
trimming
shears, basins,
trumpets,
or any
kind of gold
or silver
implements.
\VS{14}It
was handed
over to the foremen
who used
it to repair
the
{\ND{Lord}}’s
temple.
\VS{15}They did not
audit
the treasurers who disbursed
the funds
to the foremen,
for
they were
honest.
\VS{16}(The silver
collected in conjunction with reparation
offerings and sin offerings was not
brought
to the
{\ND{Lord}}’s
temple;
it belonged to the priests.)
\par }{\PP \VS{17}At that time
King
Hazael
of Syria
attacked
Gath
and captured
it.
Hazael
then decided to attack
Jerusalem.
\VS{18}King
Jehoash
of Judah
collected
all
the sacred
items
that
his ancestors
Jehoshaphat,
Jehoram,
and Ahaziah,
kings
of Judah,
had
consecrated,
as well as
his own sacred items
and all
the gold
that could be found
in the treasuries
of the
{\ND{Lord}}’s
temple
and the royal
palace.
He sent
it all to King
Hazael
of Syria,
who then withdrew from
Jerusalem.
\par }{\PP \VS{19}The rest
of the events
of Joash’s
reign, including all
his accomplishments,
are recorded
in the scroll
called the Annals
of the Kings
of Judah.
\VS{20}His servants
conspired against
him and murdered
Joash
at Beth-Millo,
on the road that goes down
to Silla.
\VS{21}His servants
Jozabad
son
of Shimeath
and Jehozabad
son
of Shomer
murdered him.
He was buried
with
his ancestors
in the city
of David.
His son
Amaziah
replaced him as king.

\par }\Chap{13}{\PP \VerseOne{1}In the twenty-third
year
of the reign
of Judah’s
King
Joash
son
of Ahaziah,
Jehu’s
son
Jehoahaz
became king over Israel.
He reigned in Samaria
for seventeen
years.
\VS{2}He did
evil
in the sight
of the
{\ND{Lord}}. He continued
in the sinful
ways of Jeroboam
son
of Nebat
who had
encouraged
Israel
to sin; he did not
repudiate those sins.
\VS{3}The
{\ND{Lord}}
was furious
with Israel
and handed
them over
to King
Hazael
of Syria
and to Hazael’s
son
Ben Hadad
for many years.
\par }{\PP \VS{4}Jehoahaz
asked for the
{\ND{Lord}}’s
mercy and the
{\ND{Lord}}
responded
favorably, for
he saw
that
Israel
was oppressed
by the king
of Syria.
\VS{5}The
{\ND{Lord}}
provided
a deliverer
for Israel
and they were freed
from
Syria’s
power.
The Israelites
once
more
lived in
security.
\VS{6}But
they did not
repudiate
the sinful
ways of the family
of Jeroboam,
who
encouraged
Israel
to sin; they continued
in those
sins. There was even
an Asherah pole
standing
in Samaria.
\VS{7}Jehoahaz
had no
army
left
except
for
fifty
horsemen,
ten
chariots,
and 10,000
foot
soldiers. The king
of Syria
had destroyed
his troops and trampled
on them like dust.
\par }{\PP \VS{8}The rest
of the events
of Jehoahaz’s
reign, including all
his accomplishments
and successes,
are
recorded
in the scroll
called the Annals
of the Kings
of Israel.
\VS{9}Jehoahaz
passed away
and was buried
in Samaria.
His son
Joash
replaced
him as king.
\par }{\SH Jehoash’s Reign over Israel
\par }{\PP \VS{10}In the thirty-seventh
year
of King
Joash’s
reign over
Judah,
Jehoahaz’s
son
Jehoash
became king over
Israel.
He reigned in Samaria
for sixteen
years.
\VS{11}He did
evil
in the sight
of the
{\ND{Lord}}. He did not
repudiate
the sinful
ways of Jeroboam
son
of Nebat
who encouraged
Israel
to sin; he continued in those sins.
\VS{12}The rest
of the events
of Joash’s
reign, including all
his accomplishments
and his successful war
with
King
Amaziah
of Judah,
are
recorded
in the scroll
called the Annals
of the Kings
of Israel.
\VS{13}Joash
passed away
and Jeroboam
succeeded
him on
the throne.
Joash
was buried
in Samaria
with
the kings
of Israel.
\par }{\SH Elisha Makes One Final Prophecy
\par }{\PP \VS{14}Now Elisha
had a terminal
illness.
King
Joash
of Israel
went down
to visit him.
He wept
before
him and said,
“My father,
my father! The chariot
and horsemen
of Israel!”
\VS{15}Elisha
told
him, “Take
a bow
and some
arrows,”
and he did so.
\VS{16}Then Elisha told
the king
of Israel,
“Aim
the bow.”
He did so, and Elisha
placed
his hands
on
the king’s
hands.
\VS{17}Elisha said,
“Open
the east
window,”
and he did so.
Elisha
said,
“Shoot!” and
\par }{\PP he did so.
Elisha said,
“This arrow
symbolizes the victory
the
{\ND{Lord}}
will give you over Syria.
You will annihilate
Syria
in Aphek!”
\VS{18}Then Elisha said,
“Take
the arrows,”
and he did so.
He told
the king
of Israel,
“Strike
the ground!” He struck
the ground three
times
and stopped.
\VS{19}The prophet
got angry
at him and said,
“If you had struck
the ground five
or
six
times,
you would have annihilated
Syria!
 But now,
you will defeat
Syria
only three
times.”
\par }{\PP \VS{20}Elisha
died
and was buried.
Moabite
raiding parties
invaded
the land
at the beginning of the year.
\VS{21}One
day some men were
burying
a man
when
they spotted
a raiding party.
So they threw
the
dead man
into Elisha’s
tomb.
When the body touched
Elisha’s
bones,
the dead man
came
to life
and stood
on
his feet.
\par }{\PP \VS{22}Now King
Hazael
of Syria
oppressed
Israel
throughout
Jehoahaz’s reign.
\VS{23}But the
{\ND{Lord}}
had mercy
on
them and felt pity
for them. He extended
his favor to
them because
of the promise
he had made to Abraham,
Isaac,
and Jacob.
He has been unwilling
to destroy
them or
remove
them from
his presence
to this very day.
\VS{24}When King
Hazael
of Syria
died,
his son
Ben Hadad
replaced
him as king.
\VS{25}Jehoahaz’s
son
Jehoash
took
back
from Ben Hadad
son
of Hazael
the cities
that he had
taken
from his father
Jehoahaz
in war.
Joash
defeated
him three
times
and recovered
the Israelite
cities.

\par }\Chap{14}{\PP \VerseOne{1}In the second
year
of the reign
of Israel’s
King
Joash
son
of Joahaz,
Joash’s
son
Amaziah
became king
over Judah.
\VS{2}He was twenty-five
years
old when
he began to reign, and he reigned
for twenty-nine
years
in Jerusalem.
His mother
was Jehoaddan,
who was from
Jerusalem.
\VS{3}He did
what the
{\ND{Lord}}
approved,
but
not
like David
his father.
He followed the example
of his father
Joash.
\VS{4}But
the high places
were not
eliminated;
the people
continued
to offer sacrifices and burn incense
on the high places.
\par }{\PP \VS{5}When
he had
secured
control
of the kingdom,
he executed
the servants
who had assassinated
his father.
\VS{6}But
he did not
execute
the
sons
of the assassins.
He obeyed the
{\ND{Lord}}’s
commandment as
recorded
in the law
scroll
of Moses, “Fathers
must not
be put to death
for what their sons
do, and sons
must not
be put to death
for what their fathers
do. A man
must be put to death
only for
his own sin.”
\par }{\PP \VS{7}He defeated
10,000
Edomites
in the Salt
Valley;
he captured
Sela
in battle
and renamed
it Joktheel,
a name
it has retained to
this
very day.
\VS{8}Then
Amaziah
sent
messengers
to
Jehoash
son
of Jehoahaz
son
of Jehu,
king
of Israel.
He said,
“Come, let’s
meet face to face.”
\VS{9}King
Jehoash
of Israel
sent
this message
back to
King
Amaziah
of Judah,
“A thornbush
in Lebanon
sent
this message to
a cedar
in Lebanon,
‘Give
your daughter
to my son
as a wife.’
Then
a wild
animal
of Lebanon
came by and trampled
down the
thorn.
\VS{10}You thoroughly defeated
Edom
and it
has gone to your head! Gloat
over your success, but
stay
in your palace.
Why
bring calamity
on
yourself? Why bring down
yourself
and Judah
along with you?”
\VS{11}But Amaziah
would not
heed
the warning, so
King
Jehoash
of Israel
attacked. He and King
Amaziah
of Judah
met face
to face
in Beth Shemesh
of Judah.
\VS{12}Judah
was defeated
by Israel,
and each man
ran
back home.
\VS{13}King
Jehoash
of Israel
captured
King
Amaziah
of Judah,
son
of Jehoash
son
of Ahaziah,
in Beth Shemesh.
He attacked
Jerusalem
and broke down
the wall
of Jerusalem
from the Gate
of Ephraim
to
the Corner
Gate
– a distance of about six hundred feet.
\VS{14}He took
away all
the gold
and silver,
all
the items
found
in the
{\ND{Lord}}’s
temple
and in the treasuries
of the royal
palace,
and some hostages.
Then he went back
to Samaria.
\par }{\PP (
\VS{15}The rest
of the events
of Jehoash’s
reign, including
all his accomplishments
and his successful war
with
King
Amaziah
of Judah,
are
recorded
in the scroll
called the Annals
of the Kings
of Israel.
\VS{16}Jehoash
passed away
and was buried
in Samaria
with
the kings
of Israel.
His son
Jeroboam
replaced
him as king.)
\par }{\PP \VS{17}King
Amaziah
son
of Joash
of Judah
lived
for fifteen
years
after
the death
of King
Jehoash
son
of Jehoahaz
of Israel.
\VS{18}The rest
of the events
of Amaziah’s
reign are
recorded
in the scroll
called the Annals
of the Kings
of Judah.
\VS{19}Conspirators
plotted against
him in
Jerusalem,
so he fled
to Lachish.
But they sent
assassins after
him
and they killed
him there.
\VS{20}His body was carried
back by horses
and he was buried
in Jerusalem
with
his ancestors
in the city
of David.
\VS{21}All
the people
of Judah
took
Azariah,
who was
sixteen
years
old, and made him king
in his father
Amaziah’s
place.
\VS{22}Azariah built up
Elat
and restored
it to Judah
after
the king
had passed away.
\par }{\SH Jeroboam II’s Reign over Israel
\par }{\PP \VS{23}In the fifteenth
year
of the reign
of Judah’s
King
Amaziah,
son
of Joash,
Jeroboam
son
of Joash
became king
over Israel.
He reigned for forty-one
years
in Samaria.
\VS{24}He did
evil
in the sight
of the
{\ND{Lord}}; he did not
repudiate
the sinful
ways of Jeroboam
son
of Nebat
who encouraged
Israel to sin.
\VS{25}He
restored
the
border
of Israel
from Lebo Hamath
in the north to
the sea
of the Arabah
in the south, in accordance with the word
of the {\ND{Lord}}
God
of Israel
announced
through
his servant
Jonah
son
of Amittai,
the prophet
from
Gath Hepher.
\VS{26}The
{\ND{Lord}}
saw
Israel’s
intense
suffering;
everyone
was weak and incapacitated
and Israel
had
no
deliverer.
\VS{27}The
{\ND{Lord}}
had not
decreed
that he would blot out
Israel’s
memory
from under
heaven,
so he delivered
them through
Jeroboam
son
of Joash.
\par }{\PP \VS{28}The rest
of the events
of Jeroboam’s
reign, including all
his accomplishments,
his military success
in restoring
Israelite
control over Damascus
and Hamath,
are
recorded
in the scroll
called the Annals
of the Kings
of Israel.
\VS{29}Jeroboam
passed away
and was buried
in Samaria with
the kings
of Israel.
His son
Zechariah
replaced him as king.

\par }\Chap{15}{\PP \VerseOne{1}In the twenty-seventh
year
of King
Jeroboam’s
reign over
Israel,
Amaziah’s
son
Azariah
became king
over Judah.
\VS{2}He was sixteen
years
old
when he began
to reign,
and he reigned
for fifty-two
years
in Jerusalem.
His mother’s
name
was Jecholiah,
who was from Jerusalem.
\VS{3}He did
what the
{\ND{Lord}}
approved,
just
as his father
Amaziah
had
done.
\VS{4}But
the high places
were not
eliminated;
the people
continued
to offer sacrifices and burn incense
on the high places.
\VS{5}The
{\ND{Lord}}
afflicted
the
king
with an illness; he suffered from a skin disease
until
the day
he died.
He lived
in separate
quarters, while his son
Jotham
was in charge
of the palace
and ruled
over the
people
of the land.
\par }{\PP \VS{6}The rest
of the events
of Azariah’s
reign, including all
his accomplishments,
are recorded
in the scroll
called the Annals
of the Kings
of Judah.
\VS{7}Azariah
passed away
and was buried
with
his ancestors
in the city
of David.
His son
Jotham
replaced
him as king.
\par }{\SH Zechariah’s Reign over Israel
\par }{\PP \VS{8}In the thirty-eighth
year
of King
Azariah’s
reign
over Judah,
Jeroboam’s
son
Zechariah
became king over
Israel.
He reigned in Samaria
for six
months.
\VS{9}He did
evil
in the sight
of the
{\ND{Lord}}, as
his ancestors
had
done.
He did not
repudiate
the sinful
ways of Jeroboam
son
of Nebat
who encouraged
Israel to sin.
\VS{10}Shallum
son
of Jabesh
conspired
against
him; he assassinated
him in Ibleam and took his place
as king.
\VS{11}The rest
of the events
of Zechariah’s
reign are recorded
in the scroll
called the Annals
of the Kings
of Israel.
\VS{12}His assassination brought to fulfillment
the
{\ND{Lord}}’s
word
to
Jehu, “Four generations
of your descendants
will rule
over
Israel.”
That is exactly
what happened.
\par }{\PP \VS{13}Shallum
son
of Jabesh
became king
in the thirty-ninth
year
of King
Uzziah’s
reign over Judah.
He reigned
for one month
in Samaria.
\VS{14}Menahem
son
of Gadi
went up
from Tirzah
to Samaria
and attacked
Shallum
son
of Jabesh.
He killed
him and took his place
as king.
\VS{15}The rest
of the events
of Shallum’s
reign, including the conspiracy
he organized,
are recorded
in
the scroll
called the Annals
of the Kings
of Israel.
\VS{16}At that time
Menahem
came from Tirzah
and attacked
Tiphsah.
He struck down
all
who
lived in the city and the surrounding territory,
because
they would not
surrender.
He even ripped
open
the pregnant women.
\par }{\SH Menahem’s Reign over Israel
\par }{\PP \VS{17}In the thirty-ninth
year
of King
Azariah’s
reign over Judah,
Menahem
son
of Gadi
became king
over
Israel.
He reigned for twelve
years
in Samaria.
\VS{18}He did
evil
in the sight
of the
{\ND{Lord}}; he did not
repudiate
the sinful
ways of Jeroboam
son
of Nebat
who encouraged
Israel
to sin.

\par }{\PP During his reign,
\VS{19}Pul
king
of Assyria
invaded
the land,
and Menahem
paid
him
a thousand
talents
of silver
to gain
his support
and to solidify
his control
of the kingdom.
\VS{20}Menahem
got
this silver
by
taxing all
the wealthy men
in Israel;
he took
fifty
shekels
of silver
from each
one
of them and paid it to the king
of Assyria.
Then the king
of Assyria
left;
he did not
stay
there
in the land.
\par }{\PP \VS{21}The rest
of the events
of Menahem’s
reign, including all
his accomplishments,
are recorded
in the scroll
called the Annals
of the Kings
of Israel.
\VS{22}Menahem
passed away
and his son
Pekahiah
replaced
him as king.
\par }{\SH Pekahiah’s Reign over Israel
\par }{\PP \VS{23}In the fiftieth
year
of King
Azariah’s
reign
over Judah,
Menahem’s
son
Pekahiah
became king over
Israel.
He reigned in Samaria
for two years.
\VS{24}He did
evil
in the sight
of the
{\ND{Lord}}; he did not
repudiate
the sinful
ways of Jeroboam
son
of Nebat
who encouraged
Israel to sin.
\VS{25}His officer
Pekah
son
of Remaliah
conspired
against
him. He and fifty
Gileadites
assassinated
Pekahiah,
as well as Argob
and Arieh,
in Samaria
in the fortress
of the royal
palace.
Pekah then took his place
as king.
\par }{\PP \VS{26}The rest
of the events
of Pekahiah’s
reign, including all
his accomplishments,
are recorded
in the scroll
called the Annals
of the Kings
of Israel.
\par }{\SH Pekah’s Reign over Israel
\par }{\PP \VS{27}In the fifty-second
year
of King
Azariah’s
reign
over Judah,
Pekah
son
of Remaliah
became king over Israel.
He reigned in Samaria
for twenty
years.
\VS{28}He did
evil
in the sight
of the
{\ND{Lord}}; he did not
repudiate
the sinful
ways of Jeroboam
son
of Nebat
who encouraged
Israel to sin.
\VS{29}During
Pekah’s
reign over
Israel,
King
Tiglath-pileser
of Assyria
came
and captured
Ijon,
Abel Beth Maacah,
Janoah,
Kedesh,
Hazor,
Gilead,
and Galilee,
including all
the territory
of Naphtali.
He deported
the people to Assyria.
\VS{30}Hoshea
son
of Elah
conspired against
Pekah
son
of Remaliah.
He assassinated
him and took his place
as king,
in the twentieth
year
of the reign of Jotham
son
of Uzziah.
\par }{\PP \VS{31}The rest
of the events
of Pekah’s
reign, including all
his accomplishments,
are recorded
in the scroll
called the Annals
of the Kings
of Israel.
\par }{\SH Jotham’s Reign over Judah
\par }{\PP \VS{32}In the second
year
of the reign
of Israel’s
King
Pekah
son
of Remaliah,
Uzziah’s
son
Jotham
became king
over Judah.
\VS{33}He was twenty-five
years
old
when he began
to reign,
and he reigned
for sixteen
years
in Jerusalem.
His mother
was Jerusha
the daughter
of Zadok.
\VS{34}He did
what the
{\ND{Lord}}
approved,
just as
his father
Uzziah
had done.
\VS{35}But
the high places
were not
eliminated;
the people
continued
to offer sacrifices and burn incense
on the high places.
He built
the Upper
Gate
to the
{\ND{Lord}}’s
temple.
\par }{\PP \VS{36}The rest
of the events
of Jotham’s
reign, including
his accomplishments,
are recorded
in the scroll
called the Annals
of the Kings
of Judah.
\VS{37}In those
days
the {\ND{Lord}}
prompted
King
Rezin
of Syria
and Pekah
son
of Remaliah
to attack Judah.
\VS{38}Jotham
passed away
and was buried
with
his ancestors
in the city
of his ancestor
David.
His son
Ahaz
replaced
him as king.

\par }\Chap{16}{\PP \VerseOne{1}In the seventeenth
year
of the reign of Pekah
son
of Remaliah,
Jotham’s
son
Ahaz
became
king
over Judah.
\VS{2}Ahaz
was twenty
years
old
when he began to reign,
and he reigned
for sixteen
years
in Jerusalem.
He did not
do
what pleased
the {\ND{Lord}}
his God,
in contrast to his ancestor
David.
\VS{3}He followed
in the footsteps
of the kings
of Israel.
He passed his son
through
the fire,
a horrible
sin practiced by the nations
whom
the {\ND{Lord}}
drove
out from before
the Israelites.
\VS{4}He offered sacrifices
and burned incense
on the high places,
on
the hills,
and under
every
green
tree.
\par }{\PP \VS{5}At that time
King
Rezin
of Syria
and King
Pekah
son
of Remaliah
of Israel
attacked
Jerusalem.
They besieged
Ahaz,
but
were unable
to conquer him.
\VS{6}(At that time
King
Rezin
of Syria
recovered
Elat
for Syria;
he drove
the Judahites
from there.
Syrians
arrived
in Elat
and live
there
to this
very
day.)
\VS{7}Ahaz
sent
messengers
to
King
Tiglath-pileser
of Assyria,
saying,
“I am
your servant
and your dependent.
March up
and rescue
me from the power
of the king
of Syria
and the king
of Israel,
who have attacked me.”
\VS{8}Then Ahaz
took
the silver
and gold
that were
in the
{\ND{Lord}}’s
temple
and in the treasuries
of the royal
palace
and sent
it as tribute
to the king
of Assyria.
\VS{9}The king
of Assyria
responded
favorably to
his request; he
attacked
Damascus
and captured
it. He deported
the people to Kir
and executed
Rezin.
\par }{\PP \VS{10}When King
Ahaz
went
to meet
with King
Tiglath-pileser
of Assyria
in Damascus,
he saw
the
altar
there. King
Ahaz
sent
to
Uriah
the priest
a drawing
of the altar
and a blueprint
for its design.
\VS{11}Uriah
the priest
built
an altar
in conformity
to the plans King
Ahaz
had
sent
from Damascus.
Uriah
the priest
finished it before
King
Ahaz
arrived
back from Damascus.
\VS{12}When
the king
arrived
back from Damascus
and saw
the altar,
he approached
it and offered
a sacrifice on it.
\VS{13}He
offered
his burnt sacrifice
and his grain offering.
He poured
out his libation
and sprinkled
the
blood
from his peace offerings
on
the altar.
\VS{14}He moved the
bronze
altar
that
stood in the
{\ND{Lord}}’s
presence
from the front
of the temple
(between
the altar
and the
{\ND{Lord}}’s
temple) and put
it on
the north
side
of the new altar.
\VS{15}King
Ahaz
ordered
Uriah
the priest,
“On
the large
altar
offer
the
morning
burnt sacrifice,
the
evening
grain offering,
the
royal
burnt sacrifices
and grain offering,
the burnt sacrifice
for all
the people
of Israel, their grain offering,
and their libations.
Sprinkle
all
the blood
of the burnt sacrifice
and other
sacrifices
on
it. The bronze
altar
will be for my personal use.”
\VS{16}So Uriah
the priest
did exactly
as King
Ahaz
ordered.
\par }{\PP \VS{17}King
Ahaz
took off
the
frames
of the movable stands,
and removed
the basins from them. He took “The Sea”
down
from the bronze
bulls
that
supported it and put
it on
the pavement.
\VS{18}He also removed the Sabbath
awning
that had
been built
in the temple
and the king’s
outer
entranceway,
on account of the king
of Assyria.
\par }{\PP \VS{19}The rest
of the events
of Ahaz’s
reign, including
his accomplishments,
are recorded
in the scroll
called the Annals
of the Kings
of Judah.
\VS{20}Ahaz
passed away
and was buried
with
his ancestors
in the city
of David.
His son
Hezekiah
replaced
him as king.

\par }\Chap{17}{\PP \VerseOne{1}In the twelfth
year
of King
Ahaz’s
reign over Judah,
Hoshea
son
of Elah
became king over
Israel.
He reigned
in Samaria
for nine
years.
\VS{2}He did
evil
in the sight
of the
{\ND{Lord}}, but not
to the same degree
as the Israelite
kings
who
preceded him.
\VS{3}King
Shalmaneser
of Assyria
threatened him; Hoshea
became his subject
and paid
him tribute.
\VS{4}The king
of Assyria
discovered
that Hoshea
was planning a revolt.
Hoshea had
sent
messengers
to
King
So
of Egypt
and had not
sent
his annual
tribute
to the king
of Assyria.
So the king
of Assyria
arrested
him and imprisoned him.
\VS{5}The king
of Assyria
marched
through the whole
land.
He attacked
Samaria
and besieged
it for three
years.
\VS{6}In the ninth
year
of Hoshea’s
reign, the king
of Assyria
captured
Samaria
and deported
the people of Israel
to Assyria.
He settled
them in Halah,
along the Habor
(the river
of Gozan), and in the cities
of the Medes.
\par }{\SH A Summary of Israel’s Sinful History
\par }{\PP \VS{7}This happened
because
the Israelites
sinned
against the
{\ND{Lord}}
their God,
who brought
them up
from the land
of Egypt
and freed them from
the power
of Pharaoh
king
of Egypt.
They worshiped
other
gods;
\VS{8}they observed
the practices
of the nations
whom
the {\ND{Lord}}
had driven
out from before
Israel,
and followed the example of the kings
of Israel.
\VS{9}The Israelites
said
things
about the
{\ND{Lord}}
their God
that
were not
right. They built
high places
in all
their cities,
from the watchtower
to
the fortress.
\VS{10}They set
up sacred pillars
and Asherah poles
on
every
high
hill
and under
every
green
tree.
\VS{11}They burned incense
on all
the high places
just like the nations
whom
the {\ND{Lord}}
had driven
away from before
them. Their evil
practices made the
{\ND{Lord}}
angry.
\VS{12}They worshiped
the disgusting idols
in blatant disregard
of the
{\ND{Lord}}’s
command.
\par }{\PP \VS{13}The
{\ND{Lord}}
solemnly warned
Israel
and Judah
through
all
his prophets
and all
the seers,
“Turn back
from your evil
ways;
obey
my commandments
and rules
that
are recorded in the law.
I ordered
your ancestors
to keep this law and sent
my servants
the prophets
to remind you of its demands.”
\VS{14}But they did not
pay attention
and were as stubborn
as their ancestors,
who
had not
trusted
the {\ND{Lord}}
their God.
\VS{15}They rejected
his rules,
the
covenant
he had
made
with
their ancestors,
and the laws
he had
commanded them
to obey. They paid
allegiance
to worthless idols,
and so became worthless
to the
{\ND{Lord}}. They copied the practices
of the surrounding
nations
in blatant disregard
of the
{\ND{Lord}}’s
command.
\VS{16}They abandoned
all
the commandments
of the {\ND{Lord}}
their God;
they made
two
metal
calves
and an Asherah pole,
bowed
down to all
the stars
in the sky,
and worshiped
Baal.
\VS{17}They passed
their sons
and daughters
through the fire,
and practiced divination
and omen
reading.
They committed themselves
to doing
evil
in the sight
of the {\ND{Lord}}
and made him angry.
\par }{\PP \VS{18}So the
{\ND{Lord}}
was furious
with Israel
and rejected
them; only
the tribe
of Judah
was left.
\VS{19}Judah
also
failed to keep
the commandments
of the {\ND{Lord}}
their God;
they followed
Israel’s
example.
\VS{20}So the
{\ND{Lord}}
rejected
all
of Israel’s
descendants;
he humiliated
them and handed
them over
to robbers,
until
he had
thrown
them from his presence.
\VS{21}He tore
Israel
away from David’s
dynasty,
and Jeroboam
son
of Nebat
became their king.
Jeroboam
drove
Israel
away
from the
{\ND{Lord}}
and encouraged
them to commit a serious sin.
\VS{22}The Israelites
followed
in the sinful
ways of Jeroboam
son
of Nebat and did
not
repudiate them.
\VS{23}Finally
the {\ND{Lord}}
rejected
Israel
just
as he had warned
he would do through
all
his servants
the prophets.
Israel
was deported
from its land
to Assyria
and remains there
to this
very
day.
\par }{\SH The King of Assyria Populates Israel with Foreigners
\par }{\PP \VS{24}The king
of Assyria
brought
foreigners from Babylon,
Cuthah,
Avva,
Hamath,
and Sepharvaim
and settled
them in the cities
of Samaria
in place
of the Israelites.
They took possession
of Samaria
and lived
in its cities.
\VS{25}When
they first
moved in,
they did not
worship
the {\ND{Lord}}. So the
{\ND{Lord}}
sent
lions among them and the lions
were killing them.
\VS{26}The king
of Assyria
was told, “The nations
whom
you deported
and settled
in the cities
of Samaria
do not
know
the
requirements
of the God
of the land,
so he has sent
lions
among them. They are killing
the
people because
they do not
know
the
requirements
of the God
of the land.”
\VS{27}So
the king
of Assyria
ordered,
“Take back
one
of the priests
whom
you deported
from there.
He
must settle
there
and teach
them the requirements
of the God
of the land.”
\VS{28}So one
of the priests
whom
they had deported
from Samaria
went back
and settled
in Bethel.
He taught
them how
to worship
the {\ND{Lord}}.
\par }{\PP \VS{29}But
each
of these nations
made
its own gods
and put
them in the shrines
on the high places
that
the people of Samaria
had made.
Each
nation
did this in the cities
where
they
lived.
\VS{30}The people
from Babylon
made
Succoth Benoth,
the people
from Cuth
made
Nergal,
the people
from Hamath
made
Ashima,
\VS{31}the Avvites
made
Nibhaz
and Tartak,
and the Sepharvites
burned
their sons
in the fire
as an offering to Adrammelech
and Anammelech,
the gods
of Sepharvaim.
\VS{32}At the
same time they worshiped
the
{\ND{Lord}}. They
appointed
some of their own people to serve as priests
in the shrines
on the high places.
\VS{33}They were worshiping
the

{\ND{Lord}}
and at the same
time
serving
their own gods
in accordance with the practices
of the nations
from which
they had been deported.
\par }{\PP \VS{34}To
this
very
day
they
observe
their earlier
practices.
They do not
worship
the {\ND{Lord}}; they do
not
obey
the rules,
regulations,
law,
and commandments
that
the {\ND{Lord}}
gave
the descendants
of Jacob,
whom
he renamed
Israel.
\VS{35}The
{\ND{Lord}}
made
an agreement
with
them and instructed
them, “You must not
worship
other
gods.
Do not
bow
down to them, serve
them, or
offer sacrifices to them.
\VS{36}Instead
you must worship
the {\ND{Lord}}, who
brought you up
from the land
of Egypt
by his great
power
and military
ability;
bow
down to him and offer sacrifices to him.
\VS{37}You must carefully
obey
at
all
times
the rules,
regulations,
law,
and commandments
he wrote
down for you. You must not
worship
other
gods.
\VS{38}You must never
forget
the agreement
I made
with
you, and you must not
worship
other
gods.
\VS{39}Instead
you must worship
the {\ND{Lord}}
your God;
then he
will rescue
you from the power
of all
your enemies.”
\VS{40}But they pay no
attention;
instead
they
observe
their earlier
practices.
\VS{41}These
nations
are worshiping
the {\ND{Lord}}
and at the same time serving
their idols;
their sons
and grandsons
do
just
as their
fathers
have done,
to this very
day. ‘span class=”footnote” id=”footnote-56” ’‘span class=”key” ’56‘a href=”\#note-56” class=”backref” ’17:31‘/a’‘span class=”text” ’
{\IT{sn}}:
{\IT{Adrammelech and Anammelech, the gods of the Sepharvaim}} are unknown in extra-biblical literature. See M. Cogan and H. Tadmor,
{\IT{II Kings}} (AB), 212.

\par }\Chap{18}{\PP \VerseOne{1}In the third
year
of the reign
of Israel’s
King
Hoshea
son
of Elah,
Ahaz’s
son
Hezekiah
became king
over Judah.
\VS{2}He was twenty-five
years
old
when he began
to reign,
and he reigned
twenty-nine
years
in Jerusalem.
His mother
was Abi,
the daughter
of Zechariah.
\VS{3}He did
what the
{\ND{Lord}}
approved,
just
as his ancestor
David
had
done.
\VS{4}He
eliminated
the high places,
smashed
the sacred pillars
to bits, and cut down
the
Asherah pole.
He also demolished
the bronze
serpent
that
Moses
had made,
for
up
to that time
the Israelites
had been offering incense
to it; it was called
Nehushtan.
\VS{5}He trusted
in the
{\ND{Lord}}
God
of Israel;
in this regard there was
none
like
him among the kings
of Judah
either before
or
after.
\VS{6}He was loyal
to the
{\ND{Lord}}
and did not
abandon
him.
He obeyed
the commandments
which
the {\ND{Lord}}
had given
to Moses.
\VS{7}The
{\ND{Lord}}
was with
him; he succeeded in all
his endeavors.
He rebelled
against the king
of Assyria
and refused
to submit to him.
\VS{8}He defeated
the Philistines
as far
as Gaza
and its territory,
from the watchtower
to
the city
fortress.
\par }{\PP \VS{9}In the fourth
year
of King
Hezekiah’s
reign (it was
the seventh
year
of the reign of Israel’s
King
Hoshea,
son
of Elah), King
Shalmaneser
of Assyria
marched up
against
Samaria
and besieged it.
\VS{10}After
three
years
he captured
it (in the sixth
year
of Hezekiah’s
reign); in the ninth
year
of King
Hoshea’s
reign over Israel
Samaria
was captured.
\VS{11}The king
of Assyria
deported
the people of Israel
to Assyria.
He settled
them in Halah,
along the Habor
(the river
of Gozan), and in the cities
of the Medes.
\VS{12}This happened because
they did not
obey
the
{\ND{Lord}}
their God
and broke
his agreement
with them. They did not
pay attention
to and obey
all
that
Moses,
the
{\ND{Lord}}’s
servant,
had commanded.
\par }{\SH Sennacherib Invades Judah
\par }{\PP \VS{13}In the fourteenth
year
of King
Hezekiah’s
reign, King
Sennacherib
of Assyria
marched up
against
all
the fortified
cities
of Judah
and captured them.
\VS{14}King
Hezekiah
of Judah
sent
this message to
the king
of Assyria,
who
was at Lachish,
“I have violated
our treaty. If you
leave, I will
do whatever
you demand.” So
the king
of Assyria
demanded that King
Hezekiah
of Judah
pay three
hundred
talents
of silver
and thirty
talents
of gold.
\VS{15}Hezekiah
gave
him
all
the silver
in the
{\ND{Lord}}’s
temple
and in the treasuries
of the royal
palace.
\VS{16}At that time
King
Hezekiah
of Judah
stripped
the metal overlays from the doors
of the
{\ND{Lord}}’s
temple
and from the posts
which
he
had plated
and gave
them to the king
of Assyria.
\par }{\PP \VS{17}The king
of Assyria
sent
his commanding general,
the
chief eunuch,
and the
chief adviser
from
Lachish
to
King
Hezekiah
in Jerusalem,
along with a large
army.
They went up
and arrived at
Jerusalem.
They went
and stood
at the conduit
of the upper
pool
which
is located on the road
to the field
where they wash and dry cloth.
\VS{18}They summoned
the king,
so
Eliakim
son
of Hilkiah,
the palace
supervisor,
accompanied by Shebna
the scribe
and Joah
son
of Asaph,
the secretary,
went out
to
meet them.
\par }{\PP \VS{19}The chief adviser
said
to
them, “Tell
Hezekiah: ‘This is what
the great
king,
the king
of Assyria,
says: “What
is your source of confidence?
\VS{20}Your claim
to have a strategy
and military
strength
is just empty talk.
In
whom
are you trusting
that
you would dare to rebel against me?
\VS{21}Now
look,
you must be
trusting
in Egypt,
that
splintered
reed
staff.
If a man
leans
for support
on
it, it punctures
his hand
and wounds him. That is what
Pharaoh
king
of Egypt
does to all
who trust
in him.
\VS{22}Perhaps
you will tell
me,
‘We are trusting
in the
{\ND{Lord}}
our God.’
But
Hezekiah
is
the one
who
eliminated
his high places
and altars
and then told
the people of Judah
and Jerusalem,
‘You must worship
at this
altar
in Jerusalem.’
\VS{23}Now
make a deal
with
my master
the king
of Assyria,
and I will give
you two thousand
horses,
provided you can
find enough riders for them.
\VS{24}Certainly
you will not refuse
one
of my master’s
minor
officials
and trust
in
Egypt
for chariots
and horsemen.
\VS{25}Furthermore
it was by the command of the
{\ND{Lord}}
that I marched up
against
this
place
to destroy
it. The
{\ND{Lord}}
told
me, ‘March up
against
this
land
and destroy it.’ ” ’ ”
\par }{\PP \VS{26}Eliakim
son
of Hilkiah,
Shebna,
and Joah
said to
the chief adviser,
“Speak
to
your servants
in Aramaic,
for
we
understand
it. Don’t
speak
with
us in the Judahite dialect
in the hearing
of the people
who
are on
the wall.”
\VS{27}But the chief adviser
said
to
them, “My master
did not send
me to
speak
these
words
only to your master
and to you. His message is also
for
the men
who sit
on
the wall,
for they will eat
their own excrement
and drink
their own urine
along with you.”
\par }{\PP \VS{28}The chief adviser
then stood
there and called out
loudly
in the Judahite dialect, “Listen
to the message
of the great
king,
the king
of Assyria.
\VS{29}This is what
the king
says: ‘Don’t
let Hezekiah
mislead
you, for
he is not
able
to rescue
you from my hand!
\VS{30}Don’t
let Hezekiah
talk you into
trusting
in the
{\ND{Lord}}
when he says,
“The
{\ND{Lord}}
will certainly rescue
us; this
city
will not
be handed
over to the king
of Assyria.”
\VS{31}Don’t
listen
to
Hezekiah!’ For
this is what
the king
of Assyria
says,
‘Send
me a token of your submission
and surrender
to
me. Then each
of you may eat
from his own
vine
and fig tree
and drink
water
from his own
cistern,
\VS{32}until
I come
and take
you to
a land just like your own – a land of grain and new wine, a land of bread and vineyards, a land of olive trees and honey. Then you will live and not die. Don’t listen to Hezekiah, for he is misleading you when he says, “The
{\ND{Lord}} will rescue us.”
\VS{33}Have
any of the gods
of the nations
actually
rescued
his land
from the power
of the king
of Assyria?
\VS{34}Where
are the gods
of Hamath
and Arpad? Where
are the gods
of Sepharvaim,
Hena,
and Ivvah? Indeed,
did any gods rescue
Samaria
from my power?
\VS{35}Who
among all
the gods
of the lands
has
rescued
their lands
from my power? So how can the
{\ND{Lord}}
rescue
Jerusalem
from my power?’ ”
\VS{36}The people
were silent
and did not
respond,
for
the king
had ordered,
“Don’t
respond to him.”
\par }{\PP \VS{37}Eliakim
son
of Hilkiah,
the palace
supervisor, accompanied by Shebna
the scribe
and Joah
son
of Asaph,
the secretary,
went to
Hezekiah
with their clothes
torn
and reported
to him what
the chief adviser had said.

\par }\Chap{19}{\PP \VerseOne{1}When
King
Hezekiah
heard
this, he tore
his clothes,
put on sackcloth,
and went
to the
{\ND{Lord}}’s
temple.
\VS{2}He sent
Eliakim
the palace
supervisor,
Shebna
the scribe,
and the leading
priests,
clothed in sackcloth,
with this message to
the prophet
Isaiah
son
of Amoz:
\VS{3}“This is what
Hezekiah
says: ‘This is a day
of distress,
insults,
and humiliation,
as when
a baby
is ready to leave
the birth
canal,
but the mother lacks
the strength to push it through.
\VS{4}Perhaps
the {\ND{Lord}}
your God
will hear
all
these things
the chief adviser
has spoken on behalf
of his master,
the king
of Assyria,
who sent
him to taunt
the living
God.
When the
{\ND{Lord}}
your God
hears,
perhaps
he will punish
him for the things
he has said. So pray
for this remnant
that remains.’ ”
\par }{\PP \VS{5}When King
Hezekiah’s
servants
came
to
Isaiah,
\VS{6}Isaiah
said
to them, “Tell
your master
this: ‘This is what
the {\ND{Lord}}
says: “Don’t
be afraid
because
of the things
you have heard
– these insults
the king
of Assyria’s
servants have hurled against me.
\VS{7}Look,
I will
take control
of his mind; he will receive
a report
and return
to his own land.
I will cut him down
with a sword
in his own land.” ’ ”
\par }{\PP \VS{8}When the chief adviser
heard
the king
of Assyria
had departed from Lachish,
he left and went to Libnah, where the king was campaigning.
\VS{9}The king heard
that King
Tirhakah
of Ethiopia
was marching
out
to fight
him.
He again
sent
messengers
to
Hezekiah,
ordering them:
\VS{10}“Tell
King
Hezekiah
of Judah
this: ‘Don’t
let your God
in whom
you
trust
mislead
you when he says,
“Jerusalem
will not
be handed
over to the king
of Assyria.”
\VS{11}Certainly
you
have heard
how
the kings
of Assyria
have annihilated
all
lands.
Do you really
think you will be rescued?
\VS{12}Were
the nations
whom
my ancestors
destroyed
– the nations
of Gozan,
Haran,
Rezeph,
and the people
of Eden
in Telassar
– rescued
by their gods?
\VS{13}Where
are the king
of Hamath,
the king
of Arpad,
and the king
of Lair, Sepharvaim,
Hena,
and Ivvah?’ ”
\par }{\PP \VS{14}Hezekiah
took
the letter
from
the messengers
and read
it. Then Hezekiah
went up
to the
{\ND{Lord}}’s
temple
and spread
it out before
the
{\ND{Lord}}.
\VS{15}Hezekiah
prayed
before
the {\ND{Lord}}: “{\ND{Lord}}
God
of Israel,
who
is enthroned
on the cherubs! You
alone
are God
over all
the kingdoms
of the earth.
You
made
the sky
and the
earth.
\VS{16}Pay
attention,

{\ND{Lord}}, and hear! Open
your eyes,

{\ND{Lord}}, and observe! Listen
to the message
Sennacherib
sent
and how he taunts
the living
God!
\VS{17}It is true,

{\ND{Lord}}, that the kings
of Assyria
have destroyed
the nations
and their lands.
\VS{18}They
have burned
the gods
of the nations,
for
they
are not
really gods,
but only
the product
of human
hands
manufactured from wood
and stone.
That
is why the Assyrians could destroy
them.
\VS{19}Now,
O
{\ND{Lord}}
our God,
rescue
us
from his power,
so that all
the kingdoms
of the earth
will know
that
you,

{\ND{Lord}}, are the only
God.”
\par }{\PP \VS{20}Isaiah
son
of Amoz
sent
this message to
Hezekiah: “This is what
the {\ND{Lord}}
God
of Israel
says: ‘I have
heard
your prayer
concerning King
Sennacherib
of Assyria.
\VS{21}This
is what
the {\ND{Lord}}
says
about him:

\par }{\Q “The virgin
daughter
Zion
\par }{\Q despises
you, she makes fun
of you;
\par }{\Q Daughter
Jerusalem
\par }{\Q shakes
her head
after you.
\par }{\Q \VS{22}Whom
have you taunted
and hurled insults
at?
\par }{\Q At whom
have you shouted,
\par }{\Q and looked
so arrogantly?

\par }{\Q At the Holy One
of Israel!
\par }{\Q \VS{23}Through
your messengers
you taunted
the sovereign
master,

\par }{\Q ‘With my many chariots
\par }{\Q I
climbed up
the high
mountains,
\par }{\Q the slopes
of Lebanon.
\par }{\Q I cut down
its tall
cedars,
\par }{\Q and its best
evergreens.
\par }{\Q I invaded
its most remote
regions,
\par }{\Q its thickest
woods.
\par }{\Q \VS{24}I
dug wells
and drank
\par }{\Q water
in foreign
lands.
\par }{\Q With the soles
of my feet I dried up
\par }{\Q all
the rivers
of Egypt.’
\VS{25}\par }{\Q Certainly
you must have heard!

\par }{\Q Long
ago
I worked
it out,
\par }{\Q In ancient
times I planned
it;
\par }{\Q and now
I am bringing
it to pass.
\par }{\Q The plan is this:
\par }{\Q Fortified
cities
will crash
\par }{\Q into heaps
of ruins.
\par }{\Q \VS{26}Their residents
are powerless,
\par }{\Q they are terrified
and ashamed.
\par }{\Q They are as short-lived as plants
in the field,
\par }{\Q or green
vegetation.
\par }{\Q They are as short-lived as grass
on the rooftops
\par }{\Q when it is scorched
by
the east wind.
\par }{\Q \VS{27}I know
where you live,
\par }{\Q and everything you do.
\par }{\Q \VS{28}Because
you rage
against me,
\par }{\Q and the uproar
you create
has reached my ears;
\par }{\Q I
will put
my hook
in your nose,
\par }{\Q and my bridle
between your lips,
\par }{\Q and I will lead you back
the way
\par }{\Q you came.”
\VS{29}\par }{\PP This
will be your confirmation
that I have spoken the truth: This year
you will eat
what grows wild,
and next
year
what grows on its own
from that. But in the third
year
you will plant seed
and harvest
crops; you will plant
vines
and consume
their produce.
\VS{30}Those who remain
in Judah
will take root
in the ground
and bear
fruit.
\par }{\Q \VS{31}For
a remnant
will leave Jerusalem;
\par }{\Q survivors will
come out
of Mount
Zion.
\par }{\Q The intense
devotion of the sovereign
{\ND{Lord}}
to his people will accomplish
this.
\par }{\Q \VS{32}So
this is what
the {\ND{Lord}}
says about the king
of Assyria:
\par }{\Q “He will not
enter
this
city,
\par }{\Q nor
will he shoot
an
arrow
here.

\par }{\Q He will not
attack
it with his shield-carrying
warriors,

\par }{\Q nor
will he build siege
works against it.
\par }{\Q \VS{33}He will go back
the way
he came.
\par }{\Q He will not
enter
this
city,”
says
the {\ND{Lord}}.
\par }{\PP \VS{34}I will shield
this
city
and rescue
it for
the sake
of my reputation
and because of my promise to David
my servant.’ ”
\par }{\PP \VS{35}That very night
the
{\ND{Lord}}’s
messenger
went out
and killed
185,000
men in the Assyrian
camp.
When they got up early
the next morning,
there
were all
the corpses.
\VS{36}So King
Sennacherib
of Assyria
broke
camp and went
on his way. He went home
and stayed
in Nineveh.
\VS{37}One day, as he was worshiping
in the temple
of his god
Nisroch,
his sons Adrammelech
and Sharezer
struck
him down
with the sword.
They
escaped
to the land
of Ararat;
his son
Esarhaddon
replaced
him as king.

\par }\Chap{20}{\PP \VerseOne{1}In those
days
Hezekiah
was stricken
with a terminal
illness. The prophet
Isaiah
son
of Amoz
visited him
and told
him,
“This is what
the {\ND{Lord}}
says,
‘Give your household
instructions,
for
you are about to die;
you will not
get well.’ ”
\VS{2}He turned
his face
to
the wall
and prayed
to
the {\ND{Lord}},
\VS{3}“Please,

{\ND{Lord}}. Remember
how I have served you faithfully
and with wholehearted
devotion, and how I have carried
out your will.” Then Hezekiah
wept bitterly.
\par }{\PP \VS{4}Isaiah
was still
in the middle
courtyard
when the
{\ND{Lord}}
told
him,
\VS{5}“Go back
and tell
Hezekiah,
the leader
of my people: ‘This is what
the {\ND{Lord}}
God
of your ancestor
David
says: “I have heard
your prayer;
I have seen
your tears.
Look,
I will heal
you. The day
after tomorrow
you will go up
to the
{\ND{Lord}}’s
temple.
\VS{6}I will add
fifteen
years
to your life
and rescue
you and this
city
from the king
of Assyria.
I will shield
this
city
for
the sake
of my reputation
and because of my promise to David
my servant.” ’ ”
\VS{7}Isaiah
ordered,
“Get
a fig
cake.”
So they did as he ordered
and placed
it on
the ulcerated sore,
and he recovered.
\par }{\PP \VS{8}Hezekiah
had said
to
Isaiah,
“What
is the confirming sign
that
the {\ND{Lord}}
will heal
me and that I will go up
to the
{\ND{Lord}}’s
temple
the day
after tomorrow?”
\VS{9}Isaiah
replied,
“This
is your sign
from the

{\ND{Lord}}
confirming that
the

{\ND{Lord}}
will do
what
he has
said.
Do you want the shadow
to move ahead
ten
steps
or
to go back
ten
steps?”
\VS{10}Hezekiah
answered,
“It is easy
for the shadow
to lengthen
ten
steps,
but not
for
it
to go
back
ten
steps.”
\VS{11}Isaiah
the prophet
called out
to
the {\ND{Lord}}, and the
{\ND{Lord}} made
the shadow
go
back
ten
steps
on the stairs
of Ahaz.
\par }{\SH Messengers from Babylon Visit Hezekiah
\par }{\PP \VS{12}At that time
Merodach-Baladan
son
of Baladan,
king
of Babylon,
sent letters
and a gift
to
Hezekiah,
for
he had heard
that
Hezekiah
was ill.
\VS{13}Hezekiah
welcomed
them and
showed
them his whole
storehouse,
with its
silver,
gold,
spices,
and
high quality olive oil,
as
well
as his armory
and everything
in his treasuries.
Hezekiah
showed
them everything
in his palace
and in his whole
kingdom.
\VS{14}Isaiah
the prophet
visited
King
Hezekiah
and asked
him, “What
did these
men
say? Where
do they come
from?” Hezekiah
replied,
“They come
from the distant
land
of Babylon.”
\VS{15}Isaiah asked,
“What
have they seen
in your palace?” Hezekiah
replied,
“They have
seen
everything
in my palace.
I showed
them everything
in my treasuries.”
\VS{16}Isaiah
said
to
Hezekiah,
“Listen
to the word
of the {\ND{Lord}},
\VS{17}‘Look,
a time
is coming
when everything
in your palace
and the things your ancestors
have accumulated
to this
day
will be carried
away to Babylon;
nothing
will be left,’
says
the {\ND{Lord}}.
\VS{18}‘Some
of your very own descendants
whom
you father
will be taken
away and will be
made eunuchs
in the palace
of the king
of Babylon.’ ”
\VS{19}Hezekiah
said
to
Isaiah,
“The
{\ND{Lord}}’s
word
which
you have announced
is appropriate.”
Then he added, “At least
there will be peace
and stability
during my lifetime.”
\par }{\PP \VS{20}The rest
of the events
of Hezekiah’s
reign and all
his accomplishments,
including how he built
a pool
and conduit
to bring
water
into the city,
are recorded
in
the scroll
called the Annals
of the Kings
of Judah.
\VS{21}Hezekiah
passed away
and his son
Manasseh
replaced
him as king.

\par }\Chap{21}{\PP \VerseOne{1}Manasseh
was twelve
years
old
when he became king,
and he reigned
for fifty-five
years
in Jerusalem.
His mother
was Hephzibah.
\VS{2}He did
evil
in the sight
of the
{\ND{Lord}}
and committed the same horrible
sins practiced by the nations
whom
the {\ND{Lord}}
drove
out from before
the Israelites.
\VS{3}He rebuilt
the high places
that
his father
Hezekiah
had destroyed;
he set
up altars
for Baal
and made
an Asherah pole
just
like King
Ahab
of Israel
had done.
He bowed
down to all
the stars
in the sky
and worshiped them.
\VS{4}He built
altars
in the
{\ND{Lord}}’s
temple,
about which
the {\ND{Lord}}
had said,
“Jerusalem
will be
my home.”
\VS{5}In the two
courtyards
of the
{\ND{Lord}}’s
temple
he built
altars
for all
the stars
in the sky.
\VS{6}He passed
his son
through the
fire
and practiced divination
and omen
reading. He
set up a ritual pit to conjure
up underworld spirits, and appointed magicians
to supervise
it. He did
a great amount of evil
in the sight
of the {\ND{Lord}}, provoking him to anger.
\VS{7}He put
an idol
of Asherah
he had
made
in the temple,
about which
the {\ND{Lord}}
had said
to
David
and to
his son
Solomon,
“This
temple
in Jerusalem,
which
I have chosen
out of all
the tribes
of Israel,
will be my
permanent
home.
\VS{8}I will not
make
Israel
again leave
the land
I gave
to their ancestors,
provided
that they carefully
obey
all
I commanded
them, the whole
law
my servant
Moses
ordered them to obey.”
\VS{9}But they did not
obey,
and Manasseh
misled
them so
that they sinned
more than
the nations
whom
the {\ND{Lord}}
had destroyed
from before
the Israelites.
\par }{\PP \VS{10}So the
{\ND{Lord}}
announced
through
his servants
the prophets:
\VS{11}“King
Manasseh
of Judah
has committed
horrible
sins. He has sinned
more than the Amorites
before
him and has encouraged
Judah
to sin by worshiping his disgusting idols.
\VS{12}So
this is what
the {\ND{Lord}}
God
of Israel
says, ‘I am
about to bring
disaster
on
Jerusalem
and Judah.
The news will reverberate
in the ears of those who hear about it.
\VS{13}I will destroy
Jerusalem
the same way I did Samaria
and the dynasty
of Ahab.
I will wipe
Jerusalem
clean, just
as one wipes
a plate
on both sides.
\VS{14}I will abandon
this last
remaining tribe among
my people
and hand
them over to their enemies;
they will be
plundered
and robbed
by all
their enemies,
\VS{15}because
they have
done
evil
in my sight
and have angered
me from
the time
their ancestors
left
Egypt
right up
to this
very
day!’ ”
\par }{\PP \VS{16}Furthermore
Manasseh
killed so many
innocent
people, he stained
Jerusalem
with
their blood
from end to end, in addition
to encouraging
Judah
to sin
by
doing
evil
in the sight
of the {\ND{Lord}}.
\par }{\PP \VS{17}The rest
of the events
of Manasseh’s
reign and all
his accomplishments,
as well as
the sinful
acts he committed,
are
recorded
in the scroll
called the Annals
of the Kings
of Judah.
\VS{18}Manasseh
passed away
and was buried
in his palace
garden,
the garden
of Uzzah,
and his son
Amon
replaced
him as king.
\par }{\SH Amon’s Reign over Judah
\par }{\PP \VS{19}Amon
was twenty-two
years
old
when he became king,
and he reigned
for two
years
in Jerusalem.
His mother
was Meshullemeth,
the daughter
of Haruz,
from
Jotbah.
\VS{20}He did
evil
in the sight
of the
{\ND{Lord}}, just
like his father
Manasseh
had done.
\VS{21}He followed
in the footsteps
of his father
and worshiped
and bowed down
to the disgusting idols
which
his father
had worshiped.
\VS{22}He abandoned
the {\ND{Lord}}
God
of his ancestors
and did not
follow
the
{\ND{Lord}}’s
instructions.
\VS{23}Amon’s
servants
conspired
against
him and killed
the king
in his palace.
\VS{24}The people
of the land
executed
all
those who had conspired
against
King
Amon,
and they made his son
Josiah
king
in his place.
\par }{\PP \VS{25}The rest
of Amon’s
accomplishments
are recorded
in the scroll
called the Annals
of the Kings
of Judah.
\VS{26}He was buried
in his tomb
in the garden
of Uzzah,
and his son
Josiah
replaced him as king.

\par }\Chap{22}{\PP \VerseOne{1}Josiah
was eight
years
old when he became king, and he reigned
for thirty-one
years
in Jerusalem.
His mother
was Jedidah,
daughter
of Adaiah,
from Bozkath.
\VS{2}He did
what the
{\ND{Lord}}
approved
and followed
in his ancestor
David’s
footsteps;
he did not
deviate
to the right
or the left.
\par }{\PP \VS{3}In the eighteenth
year
of King
Josiah’s
reign, the king
sent
the scribe
Shaphan
son
of Azaliah,
son
of Meshullam,
to the
{\ND{Lord}}’s
temple
with these orders:
\VS{4}“Go up
to
Hilkiah
the high
priest
and have him melt down
the silver
that has been brought
by the people
to the
{\ND{Lord}}’s
temple
and has been collected
by the guards
at the door.
\VS{5}Have them hand
it over
to the construction
foremen
assigned
to the
{\ND{Lord}}’s
temple.
They in turn should pay
the temple
workers
to repair it,
\VS{6}including craftsmen,
builders,
and masons,
and should buy
wood
and chiseled
stone
for the repair work.
\VS{7}Do not
audit
the foremen who disburse
the silver,
for
they are
honest.”
\par }{\PP \VS{8}Hilkiah
the high priest
informed
Shaphan
the scribe,
“I found
the law
scroll
in the
{\ND{Lord}}’s
temple.”
Hilkiah
gave
the scroll
to
Shaphan
and he read it.
\VS{9}Shaphan
the scribe
went
to
the king
and reported, “Your servants
melted
down the
silver
in the temple
and handed
it over
to the construction
foremen
assigned
to the
{\ND{Lord}}’s
temple.”
\VS{10}Then Shaphan
the scribe
told the king,
“Hilkiah
the priest
has given
me a scroll.”
Shaphan
read
it out loud before
the king.
\VS{11}When
the king
heard
the words
of the law
scroll,
he tore
his clothes.
\VS{12}The king
ordered
Hilkiah
the priest,
Ahikam
son
of Shaphan,
Acbor
son
of Micaiah,
Shaphan
the scribe,
and Asaiah
the king’s
servant,
\VS{13}“Go,
seek
an oracle from the

{\ND{Lord}}
for
me
and the people
– for
all
Judah.
Find out
about the words
of this
scroll
that has been discovered.
For
the
{\ND{Lord}}’s
fury
has
been ignited
against
us, because
our ancestors
have
not
obeyed
the words
of this
scroll
by doing
all
that it instructs
us to do.”
\par }{\PP \VS{14}So
Hilkiah
the priest,
Ahikam,
Acbor,
Shaphan,
and Asaiah
went to
Huldah
the prophetess,
the wife
of
Shullam
son
of Tikvah,
the son
of Harhas,
the supervisor of the wardrobe.
(She
lived in
Jerusalem
in
the Mishneh
district.) They stated their business,
\VS{15}and she said
to
them: “This is what
the {\ND{Lord}}
God
of Israel
says: ‘Say
this to the man
who
sent
you to me:
\VS{16}“This is what
the {\ND{Lord}}
says: ‘I am
about to bring
disaster
on
this
place
and its residents,
the details
of which are recorded in the scroll
which
the king
of Judah
has read.
\VS{17}This will happen because
they have abandoned
me and offered sacrifices
to other
gods,
angering
me with all
the idols
they have made.
My anger
will ignite
against this
place
and will not
be extinguished!’ ”
\VS{18}Say
this
to
the king
of Judah,
who sent
you to seek
an oracle from the
{\ND{Lord}}: “This is what
the

{\ND{Lord}}
God
of Israel
says concerning the words
you have
heard:
\VS{19}‘You displayed a sensitive
spirit and humbled
yourself before
the {\ND{Lord}}
when you heard
how I intended
to make this
place
and its residents
into an appalling
example
of an accursed people. You tore
your clothes
and wept
before
me, and I
have heard
you,’ says
the {\ND{Lord}}.
\VS{20}‘Therefore
I will allow
you to
die
and be buried
in peace.
You will not
have to witness
all
the disaster
I
will bring
on
this
place.’ ”’ ” Then they reported
back
to the king.

\par }\Chap{23}{\PP \VerseOne{1}The king
summoned
all
the leaders
of Judah
and Jerusalem.
\VS{2}The king
went up
to the
{\ND{Lord}}’s
temple,
accompanied by all
the people
of Judah,
all
the residents
of Jerusalem,
the priests,
and the prophets.
All
the people
were there, from the youngest
to
the oldest.
He read
aloud
all
the words
of the scroll
of the covenant
that had been discovered
in the
{\ND{Lord}}’s
temple.
\VS{3}The king
stood
by
the pillar
and renewed the
covenant
before
the {\ND{Lord}}, agreeing
to follow
the {\ND{Lord}}
and to obey
his commandments,
laws,
and rules
with all
his heart
and being,
by carrying out
the
terms
of this
covenant
recorded
on
this
scroll.
All
the people
agreed
to keep
the covenant.
\par }{\PP \VS{4}The king
ordered
Hilkiah
the high
priest,
the high-ranking
priests,
and the guards
to bring out
of the
{\ND{Lord}}’s
temple
all
the items
that were used
in the worship of Baal,
Asherah,
and all
the stars
of the sky.
The king burned
them outside
of Jerusalem
in the terraces
of Kidron,
and carried
their ashes
to Bethel.
\VS{5}He eliminated
the pagan priests
whom
the kings
of Judah
had appointed to offer
sacrifices
on the high places
in the cities
of Judah
and in the area right around
Jerusalem.
(They offered sacrifices
to Baal,
the sun
god, the moon
god, the constellations,
and all
the stars
in the sky.)
\VS{6}He removed
the Asherah pole
from the
{\ND{Lord}}’s
temple
and took it outside
Jerusalem
to
the Kidron
Valley,
where he burned
it. He smashed
it to dust
and then threw
the
dust
in
the public
graveyard.
\VS{7}He tore down
the quarters
of the male cultic prostitutes
in the
{\ND{Lord}}’s
temple,
where
women
were weaving
shrines
for Asherah.
\par }{\PP \VS{8}He brought
all
the priests
from the cities
of Judah
and ruined
the high places
where
the priests
had
offered sacrifices,
from
Geba
to
Beer Sheba.
He tore down
the high place
of the goat idols situated at
the entrance
of the gate
of Joshua,
the city
official,
on
the left side
of the city
gate.
\VS{9}(Now the priests
of the high places
did not
go up
to
the altar
of the {\ND{Lord}}
in Jerusalem,
but
they did eat
unleavened
cakes among
their fellow priests.)
\VS{10}The king ruined
Topheth
in the Valley
of Ben Hinnom
so that no
one
could pass
his son
or his daughter
through the fire
to Molech.
\VS{11}He removed
from the entrance
to
the
{\ND{Lord}}’s
temple
the statues of horses
that
the kings
of Judah
had placed
there in honor of the sun
god. (They were kept near the room
of Nathan Melech
the eunuch,
which
was situated among the courtyards.) He burned up
the chariots
devoted to the sun god.
\VS{12}The
king
tore
down the altars
the kings
of Judah
had
set up on
the roof
of Ahaz’s
upper
room, as well as
the altars
Manasseh
had
set up in the two
courtyards
of the
{\ND{Lord}}’s
temple.
He crushed
them up and threw
the
dust
in the Kidron
Valley.
\VS{13}The king
ruined the high places
east of Jerusalem,
south
of the Mount
of Destruction,
that
King
Solomon
of Israel
had built
for the detestable
Sidonian
goddess Astarte,
the detestable
Moabite
god Chemosh,
and the horrible
Ammonite
god Milcom.
\VS{14}He smashed
the sacred pillars
to bits, cut down
the Asherah pole,
and filled
those shrines
with human
bones.
\par }{\PP \VS{15}He
also
tore down
the
altar
in Bethel
at the high place
made
by Jeroboam
son
of Nebat,
who encouraged
Israel
to sin.
He
burned
all the combustible
items at that high place
and crushed
them to dust;
including
the Asherah pole.
\VS{16}When Josiah
turned
around, he saw
the
tombs
there
on the hill.
So
he ordered
the bones
from
the tombs
to be brought; he burned
them on
the altar
and defiled
it. This fulfilled
the
{\ND{Lord}}’s
announcement made by the prophet while Jeroboam
stood by the altar during a festival. King Josiah turned and saw the grave of the prophet
who
had
foretold this.
\VS{17}He asked, “What
is this
grave marker
I
see?” The men
from the city
replied,
“It’s
the grave
of the prophet
who
came
from Judah
and foretold
these
very things you have
done
to the altar
of Bethel.”
\VS{18}The king said,
“Leave
it alone! No
one
must touch
his bones.”
So they left
his bones
undisturbed,
as well as
the bones
of the Israelite prophet buried beside him.
\par }{\PP \VS{19}Josiah
also
removed
all
the shrines
on the high places
in the cities
of Samaria.
The kings
of Israel
had
made them and angered
the
{\ND{Lord}}. He did
to them
what
he had
done
to the high place in Bethel.
\VS{20}He sacrificed
all
the priests
of the high places
on
the altars
located
there,
and burned
human
bones
on
them. Then he returned
to Jerusalem.
\par }{\PP \VS{21}The king
ordered
all
the people,
“Observe
the Passover
of the {\ND{Lord}}
your God,
as prescribed
in this
scroll
of the covenant.”
\VS{22}He
issued this edict because
a Passover
like this
had
not
been observed
since the days
of the judges;
it was neglected for the entire
period
of the kings
of Israel
and Judah.
\VS{23}But
in the eighteenth
year
of King
Josiah’s
reign, such a Passover
of the {\ND{Lord}}
was observed in Jerusalem.
\par }{\PP \VS{24}Josiah also
got rid of the ritual pits used to conjure up spirits,
the magicians,
personal idols,
disgusting images,
and all
the detestable idols
that had appeared
in the land
of Judah
and in Jerusalem.
In this way he
carried out
the terms
of the law
recorded
on
the scroll
that
Hilkiah
the priest
had discovered
in the
{\ND{Lord}}’s
temple.
\VS{25}No
king
before
or
after
repented
before the
{\ND{Lord}}
as he did, with his whole
heart,
soul,
and being
in accordance
with the whole
law
of Moses.
\par }{\PP \VS{26}Yet
the
{\ND{Lord}}’s
great
anger
against Judah
did not
subside;
he was still infuriated
by all
the things
Manasseh
had done.
\VS{27}The
{\ND{Lord}}
announced,
“I will also
spurn
Judah,
just
as I spurned
Israel.
I will reject
this
city
that
I chose
– both Jerusalem
and the temple,
about which
I said,
“I will
live
there.”
\par }{\PP \VS{28}The rest
of the events
of Josiah’s
reign and all
his accomplishments
are recorded
in the scroll
called the Annals
of the Kings
of Judah.
\VS{29}During
Josiah’s
reign Pharaoh Necho
king
of Egypt
marched
toward the Euphrates
River
to help the king
of Assyria.
King
Josiah
marched out to fight
him, but Necho killed
him at Megiddo
when he saw him.
\VS{30}His servants
transported
his dead
body from Megiddo
in a chariot and brought
it to Jerusalem,
where they buried
him in his tomb.
The people
of the land
took
Josiah’s
son
Jehoahaz,
poured
olive oil on his head, and made
him king
in his father’s
place.
\par }{\SH Jehoahaz’s Reign over Judah
\par }{\PP \VS{31}Jehoahaz
was twenty-three
years
old when he became king, and he reigned
three
months
in Jerusalem.
His mother
was Hamutal
the daughter
of Jeremiah,
from Libnah.
\VS{32}He did
evil
in the sight
of the
{\ND{Lord}}
as
his ancestors
had
done.
\VS{33}Pharaoh Necho
imprisoned
him in Riblah
in the land
of Hamath
and prevented him from ruling
in Jerusalem.
He imposed
on
the land
a special tax of one hundred
talents
of silver
and a talent
of gold.
\VS{34}Pharaoh Necho
made Josiah’s
son
Eliakim
king
in Josiah’s
place,
and changed
his name
to Jehoiakim.
He took
Jehoahaz
to Egypt,
where
he died.
\VS{35}Jehoiakim
paid
Pharaoh
the required amount of silver
and gold,
but
to meet
Pharaoh’s
demands
Jehoiakim had to tax
the land.
He collected
an assessed amount
from each man
among the people
of the land
in order to pay
Pharaoh Necho.
\par }{\SH Jehoiakim’s Reign over Judah
\par }{\PP \VS{36}Jehoiakim
was twenty-five
years
old when he became king, and he reigned
for eleven
years
in Jerusalem.
His mother
was Zebidah
the daughter
of Pedaiah,
from
Rumah.
\VS{37}He did
evil
in the sight
of the
{\ND{Lord}}
as
his ancestors
had
done.

\par }\Chap{24}{\PP \VerseOne{1}During
Jehoiakim’s
reign, King
Nebuchadnezzar
of Babylon
attacked. Jehoiakim
was his subject
for three
years,
but then he rebelled against him.
\VS{2}The
{\ND{Lord}}
sent
against him Babylonian,
Syrian,
Moabite,
and Ammonite
raiding bands;
he sent
them to destroy
Judah,
as he had
warned
he would do through
his servants
the prophets.
\VS{3}Just
as the
{\ND{Lord}}
had announced,
he rejected
Judah
because
of all
the sins
which Manasseh
had
committed.
\VS{4}Because
he killed
innocent
people and
stained
Jerusalem
with their blood,
the {\ND{Lord}}
was unwilling
to forgive them.
\par }{\PP \VS{5}The rest
of the events
of Jehoiakim’s
reign and all
his accomplishments,
are recorded
in the scroll
called the Annals
of the Kings
of Judah.
\VS{6}He
passed away
and his son
Jehoiachin
replaced
him as king.
\VS{7}The king
of Egypt
did not
march out from
his land
again,
for
the king
of Babylon
conquered
all
the territory that
the king
of Egypt
had formerly controlled
between
the Brook
of Egypt
and the Euphrates
River.
\par }{\SH Jehoiachin’s Reign over Judah
\par }{\PP \VS{8}Jehoiachin
was eighteen
years
old when he became king, and he reigned
three
months
in Jerusalem.
His mother
was Nehushta
the daughter
of Elnathan,
from Jerusalem.
\VS{9}He did
evil
in the sight
of the
{\ND{Lord}}
as
his ancestors
had
done.
\par }{\PP \VS{10}At that time
the generals
of King
Nebuchadnezzar
of Babylon
marched to Jerusalem
and besieged
the city.
\VS{11}King
Nebuchadnezzar
of Babylon
came
to the city
while his generals
were besieging it.
\VS{12}King
Jehoiachin
of Judah,
along with his mother,
his servants,
his officials,
and his eunuchs
surrendered
to the
king
of Babylon.
The king
of Babylon,
in the eighth
year
of his reign,
took Jehoiachin prisoner.
\VS{13}Nebuchadnezzar took from there
all
the riches in the treasuries
of the
{\ND{Lord}}’s
temple
and of the royal
palace.
He removed
all
the gold
items
which
King
Solomon
of Israel
had made
for the
{\ND{Lord}}’s
temple,
just as the
{\ND{Lord}}
had
warned.
\VS{14}He deported
all
the
residents of Jerusalem,
including all
the officials
and all
the soldiers
(10,000
people
in all). This included all
the craftsmen
and those who worked with metal.
No
one was left
except
for the poorest
among the people
of the land.
\VS{15}He deported
Jehoiachin
from Jerusalem
to Babylon,
along with the king’s
mother
and wives,
his eunuchs,
and the high-ranking officials
of the land.
\VS{16}The king
of Babylon
deported
to Babylon
all
the soldiers
(there were 7,000), as well as 1,000
craftsmen
and metal workers.
This included all
the best warriors.
\VS{17}The king
of Babylon
made Mattaniah,
Jehoiachin’s uncle,
king in Jehoiachin’s place.
He renamed him
Zedekiah.
\par }{\SH Zedekiah’s Reign over Judah
\par }{\PP \VS{18}Zedekiah
was twenty-one
years
old when he became king, and he ruled
for eleven
years
in Jerusalem.
His mother
was Hamutal,
the daughter
of Jeremiah,
from Libnah.
\VS{19}He did
evil
in the sight
of the
{\ND{Lord}}, as
Jehoiakim
had done.
\par }{\PP \VS{20}What follows is a record of what happened
to Jerusalem
and Judah
because
of the
{\ND{Lord}}’s
anger;
he finally
threw
them out of his presence.
Zedekiah
rebelled
against the king
of Babylon.

\par }\Chap{25}{\PP \VerseOne{1}So
King
Nebuchadnezzar
of Babylon
came against
Jerusalem
with his whole
army
and set up camp
outside it. They built
siege ramps
all around
it. He arrived
on the tenth
day of the tenth
month
in the ninth
year
of Zedekiah’s reign.
\VS{2}The city
remained under siege
until
King
Zedekiah’s
eleventh
year.
\VS{3}By the ninth
day of the fourth month
the famine
in the city
was so severe
the residents
had no
food.
\VS{4}The enemy broke
through the city
walls, and all
the soldiers tried
to escape. They left the city during the night.
They went through
the gate
between
the two walls
that
is near the king’s
garden.
(The Babylonians
were all around
the city.) Then
they headed for the Jordan Valley.
\VS{5}But the Babylonian
army
chased
after
the king.
They caught
up with him
in the plains
of Jericho,
and his entire
army
deserted
him.
\VS{6}They captured
the king
and brought him up
to
the king
of Babylon
at Riblah,
where he passed sentence on him.
\VS{7}Zedekiah’s sons
were executed
while Zedekiah
was forced to watch.
The king of Babylon then had Zedekiah’s
eyes
put out,
bound
him in bronze
chains, and carried
him off
to Babylon.
\par }{\SH Nebuchadnezzar Destroys Jerusalem
\par }{\PP \VS{8}On the seventh
day of the fifth
month,
in the nineteenth
year
of King
Nebuchadnezzar
of Babylon,
Nebuzaradan,
the captain
of the royal guard
who served
the king
of Babylon,
arrived
in Jerusalem.
\VS{9}He burned down
the
{\ND{Lord}}’s
temple,
the royal
palace,
and all
the houses
in Jerusalem,
including
every
large
house.
\VS{10}The whole
Babylonian
army
that
came with the captain
of the royal guard
tore
down the walls
that surrounded
Jerusalem.
\VS{11}Nebuzaradan,
the
captain
of the royal guard,
deported
the rest
of the people
who were left
in the city,
those who
had deserted
to the king
of Babylon,
and the
rest
of the craftsmen.
\VS{12}But he left behind
some of the poor
of the land
and gave them fields
and vineyards.
\par }{\PP \VS{13}The Babylonians
broke
the two bronze
pillars
in the
{\ND{Lord}}’s
temple,
as well as
the movable stands
and the big bronze
basin called the
“The Sea.”
They took
the
bronze
to Babylon.
\VS{14}They also took
the pots,
shovels,
trimming
shears, pans,
and all
the bronze
utensils
used by the priests.
\VS{15}The captain
of the royal guard
took
the golden
and silver
censers
and basins.
\VS{16}The bronze
of the items
that King Solomon
made
for the
{\ND{Lord}}’s
temple
– including the two
pillars,
the big bronze basin called “The Sea,” the twelve bronze bulls under “The Sea,”
and
the movable stands
– was too heavy to
be
weighed.
\VS{17}Each of the pillars
was about twenty-seven feet
high.
The bronze
top of
one
pillar
was about four and a half feet
high
and had bronze
latticework
and pomegranate shaped
ornaments
all
around
it. The second
pillar
with its latticework was like it.
\par }{\PP \VS{18}The captain
of the royal guard
took
Seraiah
the chief
priest
and Zephaniah,
the priest
who was second
in rank, and the
three
doorkeepers.
\VS{19}From
the city
he took
a eunuch
who
was in charge of
the soldiers,
five
of the king’s
advisers
who
were discovered
in the city,
an official
army
secretary
who drafted
citizens
for military service,
and sixty
citizens
from the people
of the land
who were discovered
in the city.
\VS{20}Nebuzaradan,
captain
of the royal guard,
took
them and brought
them to the king
of Babylon
at Riblah.
\VS{21}The king
of Babylon
ordered
them to be executed
at Riblah
in the territory
of Hamath.
So Judah
was deported
from its land.
\par }{\SH Gedaliah Appointed Governor
\par }{\PP \VS{22}Now
King
Nebuchadnezzar
of Babylon
appointed
Gedaliah
son
of Ahikam,
son
of Shaphan,
as governor over the people
whom
he allowed
to remain
in the land
of Judah.
\VS{23}All
of the officers
of the Judahite army
and their
troops
heard
that
the king
of Babylon
had appointed
Gedaliah
to govern. So they came
to
Gedaliah
at Mizpah.
The officers who came were Ishmael
son
of Nethaniah,
Johanan
son
of Kareah,
Seraiah
son
of Tanhumeth
the Netophathite,
and Jaazaniah
son
of the Maacathite.
\VS{24}Gedaliah
took an oath
so as to give them and their troops some assurance
of safety. He said,
“You don’t
need to be afraid
to submit to the Babylonian
officials.
Settle
down in the land
and submit
to the
king
of Babylon.
Then things will go well for you.”
\VS{25}But
in the seventh
month
Ishmael
son
of Nethaniah,
son
of Elishama,
who was a member
of the royal
family, came with
ten
of his men
and murdered
Gedaliah,
as well as
the
Judeans
and Babylonians
who were with
him at Mizpah.
\VS{26}Then
all
the people,
from the youngest
to the oldest, as
well as the army
officers,
left
for
Egypt,
because
they were afraid
of what the Babylonians might do.
\par }{\SH Jehoiachin in Babylon
\par }{\PP \VS{27}In the thirty-seventh
year
of the exile
of King
Jehoiachin
of Judah,
on the twenty-seventh
day of the twelfth
month,
King
Evil-Merodach
of Babylon,
in the first year
of his reign,
pardoned
King
Jehoiachin
of Judah
and released him from prison.
\VS{28}He spoke
kindly
to him
and gave
him a more prestigious
position
than the other kings
who
were with
him in Babylon.
\VS{29}Jehoiachin took off
his prison
clothes
and ate
daily
in the king’s presence
for the rest
of his life.
\VS{30}He was given
daily
provisions
by the king
for the rest of his life
until the day he died.
\par }