\NormalFont\ShortTitle{1 Samuel}
{\MT 1 Samuel

\par }\ChapOne{1}{\SH Hannah Gives Birth to Samuel
\par }{\PP \VerseOne{1}There was
a man
from
Ramathaim Zophim,
from the hill country
of Ephraim,
whose name
was Elkanah.
He was the son
of Jeroham,
the son
of Elihu,
the son
of Tohu,
the son
of Zuph,
an Ephraimite.
\VS{2}He had two
wives;
the name
of the first
was Hannah
and the name
of the second
was Peninnah.
Now Peninnah
had children,
but Hannah
was childless.
\par }{\PP \VS{3}Year after year this man
would go up
from his city
to worship
and to sacrifice
to the
{\ND{Lord}}
of hosts
at Shiloh.
It was there
that the two
sons
of Eli,
Hophni
and Phineas,
served as the
{\ND{Lord}}’s
priests.
\VS{4}Whenever
the day
came for Elkanah
to sacrifice,
he used to give
meat portions
to his wife
Peninnah
and to all
her sons
and daughters.
\VS{5}But he would give
a double
portion
to Hannah,
because
he especially loved
her.
Now the
{\ND{Lord}}
had not enabled
her to have children.
\VS{6}Her rival
wife used to upset
her and make her worry,
for
the {\ND{Lord}}
had not enabled
her to have children.
\VS{7}Peninnah would behave
this way
year
after year.
Whenever
Hannah went up
to the
{\ND{Lord}}’s
house,
Peninnah would upset
her so that
she would weep
and refuse
to eat.
\VS{8}Finally her husband
Elkanah
said
to her, “Hannah,
why
do you weep
and not
eat? Why
are you so sad? Am
I not
better
to you than ten
sons?”
\par }{\PP \VS{9}On
one occasion
in Shiloh,
after
they had finished
eating
and drinking,
Hannah
got up.
(Now at the time Eli
the priest
was sitting
in his chair
by the doorpost
of the
{\ND{Lord}}’s
temple.)
\VS{10}She
was very upset
as she prayed
to
the {\ND{Lord}},
and she was weeping uncontrollably.
\VS{11}She made a vow
saying,
“O
{\ND{Lord}}
of hosts,
if
you will look
with compassion on the suffering
of your female servant,
remembering
me and not
forgetting
your servant,
and give
a male
child
to your servant,
then I will dedicate him
to the
{\ND{Lord}}
all
the days
of his life.
His hair will never
be cut.”
\par }{\PP \VS{12}As she continued
praying
to the
{\ND{Lord}}, Eli
was watching
her mouth.
\VS{13}Now Hannah
was speaking
from her heart.
Although
her lips
were moving,
her voice
was inaudible.
Eli
therefore thought
she was drunk.
\VS{14}So he said
to
her, “How
often
do you intend to get drunk? Put away
your wine!”
\par }{\PP \VS{15}But Hannah
replied,
“That’s not
the way it is, my lord! I am under a great deal of stress.
I
have drunk
neither wine
nor beer.
Rather, I have poured out
my soul
to
the {\ND{Lord}}.
\VS{16}Don’t
consider
your servant
a wicked
woman,
for until
now
I have spoken
from my deep
pain
and anguish.”
\par }{\PP \VS{17}Eli
replied,
“Go
in peace,
and may the God
of Israel
grant
the request
that
you have asked of him.”
\VS{18}She said,
“May I, your servant,
find
favor
in your sight.”
So the woman
went
her way
and got something to eat.
Her face
no
longer looked sad.
\par }{\PP \VS{19}They got up early
the next morning
and after worshiping
the {\ND{Lord}}, they returned
to
their home
at Ramah.
Elkanah
had marital relations
with his wife
Hannah,
and the
{\ND{Lord}}
remembered her.
\VS{20}After some time
Hannah
became pregnant
and gave birth
to a son.
She named
him Samuel,
thinking, “I asked
the {\ND{Lord}}
for him.
\par }{\SH Hannah Dedicates Samuel to the Lord
\par }{\PP \VS{21}This man
Elkanah
went up
with all
his family
to make the yearly
sacrifice
to the
{\ND{Lord}}
and to keep his vow,
\VS{22}but Hannah
did not
go up
with them. Instead
she told
her husband,
“Once the boy
is weaned,
I will bring
him and appear
before
the {\ND{Lord}}, and he will remain
there
from then on.”
\par }{\PP \VS{23}So her husband
Elkanah
said
to her, “Do
what you think
best.
Stay
until
you have weaned
him. May
the {\ND{Lord}}
fulfill
his promise.”
\par }{\PP So
the woman
stayed
and nursed
her son
until
she had weaned him.
\VS{24}Once she had weaned
him, she took
him up
with
her, along with three
bulls,
an ephah
of flour,
and a container
of wine.
She brought
him to the
{\ND{Lord}}’s
house
at Shiloh,
even though he was young.
\VS{25}Once the bull
had been slaughtered,
they brought
the boy
to
Eli.
\VS{26}She said,
“Just as surely
as you
are alive,
my lord,
I am
the woman
who previously stood
here
with
you in order to pray
to
the {\ND{Lord}}.
\VS{27}I prayed
for this
boy,
and the
{\ND{Lord}}
has given
me the request
that
I asked
of him.
\VS{28}Now
I
dedicate
him to the
{\ND{Lord}}. From this time
on he is
dedicated
to the
{\ND{Lord}}.” Then they worshiped
the {\ND{Lord}}
there.

\par }\Chap{2}{\PP \VerseOne{1}Hannah
prayed,
\par }{\Q “My heart
rejoices
in the
{\ND{Lord}};
\par }{\Q my horn
is exalted
high because
of the {\ND{Lord}}.
\par }{\Q I loudly denounce
my enemies,
\par }{\Q for
I am happy
that you delivered me.
\par }{\Q \VS{2}No
one is holy
like the
{\ND{Lord}}!
\par }{\Q There is no
one other than
you!
\par }{\Q There is no
rock
like our God!
\par }{\Q \VS{3}Don’t
keep speaking
so arrogantly,

\par }{\Q letting
proud talk come out
of your mouth!
\par }{\Q For
the {\ND{Lord}}
is
a God
who knows;
\par }{\Q he evaluates what people do.
\par }{\Q \VS{4}The bows
of warriors
are shattered,
\par }{\Q but those who stumble
find their strength
reinforced.
\par }{\Q \VS{5}Those who are well-fed
hire
themselves out to earn food,
\par }{\Q but the hungry
no longer lack.
\par }{\Q Even
the barren
woman gives birth
to seven,
\par }{\Q but the one with many
children
withers away.
\par }{\Q \VS{6}The
{\ND{Lord}}
both kills
and gives life;
\par }{\Q he brings down
to the grave
and raises up.
\par }{\Q \VS{7}The
{\ND{Lord}}
impoverishes
and makes wealthy;
\par }{\Q he humbles
and he exalts.
\par }{\Q \VS{8}He lifts
the weak
from the dust;
\par }{\Q he raises
the poor
from the ash heap
\par }{\Q to seat
them with
princes
\par }{\Q and to bestow
on them an honored position.
\par }{\Q The foundations
of the earth
belong to the
{\ND{Lord}},
\par }{\Q and he has placed
the world
on them.
\par }{\Q \VS{9}He watches
over
his holy ones,
\par }{\Q but the wicked
are made speechless
in the darkness,
\par }{\Q for
it is not
by one’s own strength
that one
prevails.
\par }{\Q \VS{10}The
{\ND{Lord}}
shatters
his adversaries;
\par }{\Q he thunders
against
them
from the heavens.
\par }{\Q The
{\ND{Lord}}
executes
judgment to the ends
of the earth.
\par }{\Q He will
strengthen his
king
\par }{\Q and exalt
the power
of his anointed one.”
\par }{\PP \VS{11}Then Elkanah
went
back home
to Ramah.
But the boy
was
serving
the {\ND{Lord}}
under the supervision
of Eli
the priest.
\par }{\SH Eli’s Sons Misuse Their Sacred Office
\par }{\PP \VS{12}The sons
of Eli
were wicked
men. They did not
recognize
the
{\ND{Lord}}’s authority.
\VS{13}Now the priests
would always treat the people
in the following way: Whenever anyone
was making
a sacrifice,
while the meat
was boiling,
the priest’s
attendant
would come
with a three-pronged
fork
in his hand.
\VS{14}He would jab
it into the basin,
kettle,
caldron,
or
pot,
and everything
that
the fork
brought up
the priest
would take
for himself. This is what
they used to do
to all
the Israelites
when they came
there
to Shiloh.
\par }{\PP \VS{15}Even
before
they burned
the
fat,
the priest’s
attendant
would come
and say
to the person
who was making the sacrifice,
“Hand
over some meat
for the priest
to roast! He won’t
take
boiled
meat
from
you, but only
raw.”
\VS{16}If the individual
said
to him,
“First
let the fat
be burned away,
and then take
for yourself whatever
you
wish,”
he would say,
“No! Hand it over
right now! If
you don’t,
I will take
it forcibly!”
\par }{\PP \VS{17}The sin
of these young men
was very
great
in the
{\ND{Lord}}’s
sight,
for
they treated
the
{\ND{Lord}}’s
offering
with contempt.
\par }{\PP \VS{18}Now Samuel
was ministering
before
the {\ND{Lord}}. The boy
was dressed
in a linen
ephod.
\VS{19}His mother
used to make
him a small
robe
and bring it up
to him at
regular intervals when
she would go up
with
her husband
to make
the annual sacrifice.
\VS{20}Eli
would bless
Elkanah
and his wife
saying,
“May the
{\ND{Lord}}
raise up
for you descendants
from
this
woman
to replace
the one that she dedicated
to the
{\ND{Lord}}.” Then they would go
to their home.
\VS{21}So
the {\ND{Lord}}
graciously
attended
to Hannah,
and she was able to conceive
and gave birth
to three
sons
and two
daughters.
The boy
Samuel
grew
up at the
{\ND{Lord}}’s sanctuary.
\par }{\PP \VS{22}Now Eli
was very
old
when he heard
about everything
that
his sons
used to do to all
the people of Israel
and how they used to have
sex
with
the women
who were stationed
at the entrance
to the tent
of meeting.
\VS{23}He said
to them, “Why
do
you behave
in this way? For I
hear
about
these
evil
things
from all
these
people.
\VS{24}This ought not
to be, my sons! For
the report
that
I
hear
circulating
among the
{\ND{Lord}}’s
people
is not
good.
\VS{25}If
a man
sins
against a man,
one
may appeal
to God
on his behalf. But if
a man
sins
against the
{\ND{Lord}}, who
then will intercede
for him?” But Eli’s sons would not
listen
to their father,
for
the {\ND{Lord}}
had decided
to kill them.
\par }{\PP \VS{26}Now the boy
Samuel
was growing
up and finding
favor both
with
the {\ND{Lord}}
and with
people.
\par }{\SH The Lord Judges the House of Eli
\par }{\PP \VS{27}A man
of God
came
to
Eli
and said
to
him, “This is what
the {\ND{Lord}}
says: ‘Did I not plainly
reveal
myself to
your ancestor’s
house
when they were
in Egypt
in the house
of Pharaoh?
\VS{28}I chose
your ancestor from all
the tribes
of Israel
to be my priest,
to offer
sacrifice on
my altar,
to burn
incense,
and to bear
the ephod
before
me. I gave
to your ancestor’s
house
all
the fire offerings
made by the Israelites.
\VS{29}Why
are you scorning
my sacrifice
and my offering
that
I commanded
for my dwelling place? You have honored
your sons
more than
you have me by having made yourselves fat
from the best parts
of all
the offerings
of my people
Israel.’
\par }{\PP \VS{30}Therefore
the {\ND{Lord}}, the God
of Israel,
says,
‘I really did say
that your house
and your ancestor’s
house
would
serve
me forever.’
But now
the {\ND{Lord}}
says,
‘May it never be! For
I will honor
those who honor
me, but those who despise
me will be cursed!
\VS{31}In fact,
days
are coming
when I will remove
your strength
and the strength
of your father’s
house.
There will not be
an old man
in your house!
\VS{32}You will see
trouble
in my dwelling place! Israel
will experience blessings,
but there will not
be
an old man
in your house
for all
time.
\VS{33}Any one
of you that I do not
cut off
from
my altar,
I will cause
your eyes
to fail
and will cause you grief.
All
of those born
to your family
will die
in the prime
of life.
\VS{34}This
will be a confirming sign
for you that
will be fulfilled
through your two
sons,
Hophni
and Phinehas: in a single
day
they both
will die!
\VS{35}Then I will raise up
for myself a faithful
priest.
He will do
what is in my heart
and soul.
I will build
for him a secure
dynasty
and he will serve
my chosen one
for all
time.
\VS{36}Everyone
who remains
in your house
will come
to bow
before him for a little
money
and for a scrap
of bread.
Each
will say,
‘Assign
me to
a priestly task
so I can eat
a scrap
of bread.’ ”

\par }\Chap{3}{\PP \VerseOne{1}Now the boy
Samuel
continued serving
the {\ND{Lord}}
under Eli’s
supervision.
Word
from the
{\ND{Lord}}
was rare
in those days;
revelatory
visions
were infrequent.
\par }{\PP \VS{2}Eli’s
eyes
had begun
to fail,
so that he was
unable
to see
well. At that time
he was lying
down in his place,
\VS{3}and the lamp
of God
had not yet
been extinguished.
Samuel
was lying
down in the temple
of the {\ND{Lord}}
as well;
the ark
of God
was also there.
\VS{4}The
{\ND{Lord}}
called
to
Samuel,
and he replied,
“Here I am!”
\VS{5}Then he ran
to
Eli
and said,
“Here I am,
for
you called
me.” But Eli said,
“I didn’t
call
you. Go back
and lie
down.” So he went
back and lay down.
\VS{6}The
{\ND{Lord}}
again
called,
“Samuel!” So Samuel
got
up and went
to
Eli
and said,
“Here I am,
for
you called
me.” But Eli said,
“I didn’t
call
you, my son.
Go back
and lie down.”
\par }{\PP \VS{7}Now Samuel
did not yet
know
the {\ND{Lord}}; the word
of the {\ND{Lord}}
had not yet
been revealed
to him.
\VS{8}Then
the {\ND{Lord}}
called
Samuel
a third
time. So he got
up and went
to
Eli
and said,
“Here I am,
for
you called
me!” Eli
then
realized that
it was the
{\ND{Lord}}
who
was calling
the boy.
\VS{9}So Eli
said
to Samuel,
“Go
back and lie
down. When
he calls
you, say,
“Speak,

{\ND{Lord}}, for
your servant
is listening.”
So Samuel
went
back and lay
down in his place.
\par }{\PP \VS{10}Then the
{\ND{Lord}}
came
and stood
nearby, calling
as he had previously done, “Samuel! Samuel!” Samuel
replied,
“Speak,
for
your servant
is listening!”
\VS{11}The
{\ND{Lord}}
said
to
Samuel,
“Look! I am
about to do
something
in Israel;
when
anyone
hears
about it, both
of his ears
will tingle.
\VS{12}On that day
I will carry out
against Eli
everything
that
I spoke
about his house
– from start
to finish!
\VS{13}You should tell
him that
I am
about to judge
his house
forever
because of the sin
that
he knew
about. For
his sons
were cursing
God, and he did not
rebuke them.
\VS{14}Therefore
I swore
an oath to the house
of Eli,
‘The sin
of the house
of Eli
can never
be forgiven
by sacrifice
or by grain offering.’ ”
\par }{\PP \VS{15}So Samuel
lay
down until
morning.
Then he opened
the doors
of the
{\ND{Lord}}’s
house.
But Samuel
was afraid
to tell
Eli
about the vision.
\VS{16}However, Eli
called
Samuel
and said,
“Samuel,
my son!” He replied,
“Here I am.”
\VS{17}Eli said,
“What
message
did
he speak
to
you? Don’t
conceal
it from
me. God
will judge you severely
if
you conceal
from
me anything
that
he said
to you!”
\par }{\PP \VS{18}So Samuel
told
him everything.
He did not
hold
back anything from
him. Eli said,
“The
{\ND{Lord}}
will do
what he
pleases.”
\VS{19}Samuel
continued to grow,
and the
{\ND{Lord}}
was
with
him. None
of his prophecies
fell
to the ground
unfulfilled.
\VS{20}All
Israel
from
Dan
to
Beer Sheba
realized
that
Samuel
was confirmed
as a prophet
of the {\ND{Lord}}.
\VS{21}Then the
{\ND{Lord}}
again
appeared
in Shiloh,
for it was in Shiloh
that
the {\ND{Lord}}
had revealed
himself to
Samuel
through the word
of the {\ND{Lord}}.

\par }\Chap{4}{\PP \VerseOne{1}Samuel
revealed the word
of the
{\ND{Lord}} to all
Israel.
\par }{\SH The Ark of the Covenant is Lost to the Philistines
\par }{\PP Then the Israelites
went out
to fight
the Philistines.
They camped
at Ebenezer,
and the Philistines
camped
at Aphek.
\VS{2}The Philistines
arranged
their forces to fight
Israel.
As the battle
spread
out, Israel
was defeated
by the Philistines,
who killed
about four
thousand
men
in the battle line
in the field.
\par }{\PP \VS{3}When the army
came
back to
the camp,
the elders
of Israel
said,
“Why
did the
{\ND{Lord}}
let us be defeated
today
by
the Philistines? Let’s take
with us the ark
of the covenant
of the {\ND{Lord}}
from Shiloh.
When it is with
us, it will save
us from the hand
of our enemies.
\par }{\PP \VS{4}So the army
sent
to Shiloh,
and they took
from there
the
ark
of the covenant
of the
{\ND{Lord}}
of hosts
who sits
between the cherubim.
Now the two
sons
of Eli,
Hophni
and Phineas,
were there
with
the ark
of the covenant
of God.
\VS{5}When
the ark
of the covenant
of the {\ND{Lord}}
arrived
at the camp,
all
Israel
shouted
so loudly
that the ground
shook.
\par }{\PP \VS{6}When the Philistines
heard
the sound
of the shout,
they said,
“What
is this loud
shout
in the camp
of the Hebrews?” Then they realized
that
the ark
of the {\ND{Lord}}
had arrived
at
the camp.
\VS{7}The Philistines
were scared
because
they thought
that gods
had come
to
the camp.
They said,
“Too bad
for
us! We’ve never
seen anything like this!
\VS{8}Too bad
for us! Who can
deliver
us from the hand
of these
mighty
gods? These
are
the gods
who struck
the Egyptians
with all
sorts
of plagues in the desert!
\VS{9}Be strong
and act like
men,
you Philistines,
or else
you will wind up serving
the Hebrews
the way they have
served
you! Act like
men
and fight!”
\par }{\PP \VS{10}So the Philistines
fought.
Israel
was defeated;
they
all ran
home.
The slaughter
was
very
great;
thirty
thousand
foot soldiers
fell in battle.
\VS{11}The ark
of God
was taken,
and the two
sons
of Eli,
Hophni
and Phineas,
were killed.
\par }{\SH Eli Dies
\par }{\PP \VS{12}On that day
a Benjaminite
ran
from the battle lines
and came
to Shiloh.
His clothes
were torn
and dirt
was on
his head.
\VS{13}When he arrived
in Shiloh,
Eli
was sitting
in his chair
watching
by the side of the road,
for
he was
very worried
about the ark
of God.
As the man
entered
the city
to give his report,
the whole
city
cried out.
\par }{\PP \VS{14}When
Eli
heard
the outcry,
he said,
“What
is this
commotion?” The man
quickly
came
and told
Eli.
\VS{15}Now Eli
was ninety-eight
years
old
and his eyes
looked straight ahead;
he was unable
to see.
\par }{\PP \VS{16}The man
said
to
Eli,
“I am
the one who came
from
the battle lines! Just today
I
fled
from
the battle lines!” Eli asked,
“How
did things
go, my son?”
\VS{17}The messenger
replied,
“Israel
has fled
from
the Philistines! The army
has suffered
a great
defeat! Your two
sons,
Hophni
and Phineas,
are dead! The ark
of God
has been captured!”
\par }{\PP \VS{18}When
he mentioned
the ark
of God,
Eli fell
backward
from his chair
beside
the gate.
He broke
his neck
and died,
for
he was old
and heavy.
He
had judged
Israel
for forty
years.
\par }{\PP \VS{19}His daughter-in-law,
the wife
of Phineas,
was pregnant
and close to
giving birth.
When she heard
that the ark
of God
was captured
and that her father-in-law
and her husband
were dead,
she doubled over
and gave birth.
But her labor pains
were too much for her.
\VS{20}As she was dying,
the women who were there with her said, “Don’t
be afraid! You have given birth
to a son!” But she did not
reply
or
pay
any attention.
\par }{\PP \VS{21}She named
the boy
Ichabod,
saying,
“The glory
has departed from
Israel,”
referring to
the capture
of the ark
of God
and the deaths
of her father-in-law
and her husband.
\VS{22}She said,
“The glory
has departed
from Israel,
because
the ark
of God
has been captured.”

\par }\Chap{5}{\PP \VerseOne{1}Now the Philistines
had captured
the ark
of God
and brought
it from Ebenezer
to Ashdod.
\VS{2}The Philistines
took
the ark
of God
and brought
it into the temple
of Dagon,
where they positioned it
beside
Dagon.
\VS{3}When the residents of Ashdod
got up early
the next
day, Dagon
was lying
on the ground
before
the ark
of the {\ND{Lord}}. So they took
Dagon
and set him back
in his place.
\VS{4}But when they got
up early
the following
day, Dagon
was again lying
on the ground
before
the ark
of the {\ND{Lord}}. The head
of Dagon
and his two
hands
were sheared
off and were lying at the threshold.
Only
Dagon’s
body was left
intact.
\VS{5}(For
this reason,
to
this very
day,
neither
Dagon’s
priests
nor anyone
else who enters
Dagon’s
temple
step on
Dagon’s
threshold
in Ashdod.)
\par }{\PP \VS{6}The
{\ND{Lord}}
attacked
the residents
of Ashdod
severely, bringing devastation
on them. He struck
the people of both Ashdod
and the surrounding area
with sores.
\VS{7}When
the people
of Ashdod
saw what
was happening,
they said,
“The ark
of the God
of Israel
should
not
remain
with
us, for
he has attacked
both us
and our god
Dagon!”
\par }{\PP \VS{8}So they assembled
all
the leaders
of the Philistines
and asked,
“What
should we do
with the ark
of the God
of Israel?” They replied,
“The ark
of the God
of Israel
should be moved
to Gath.”
So they moved
the ark
of the God
of Israel.
\par }{\PP \VS{9}But after
it had been moved
the {\ND{Lord}}
attacked
that city
as well, causing a great deal
of panic.
He struck
all
the people
of that city
with sores.
\VS{10}So they sent
the
ark
of God
to Ekron.
\par }{\PP But when
the ark
of God
arrived
at Ekron,
the residents of Ekron
cried out
saying,
“They have brought
the
ark
of the God
of Israel
here to kill
our people!”
\VS{11}So they assembled
all
the leaders
of the Philistines
and said,
“Get
the ark
of the God
of Israel
out of here! Let it go back
to its own place
so that
it won’t
kill
us and our people!” The terror
of death
was throughout the entire
city;
God
was attacking
them very
severely
there.
\VS{12}The people
who
did not
die
were struck
with sores;
the city’s
cry for help
went all the way up
to heaven.

\par }\Chap{6}{\PP \VerseOne{1}When
the ark
of the {\ND{Lord}}
had been in the land
of the Philistines
for seven
months,
\VS{2}the Philistines
called
the priests
and the omen readers,
saying,
“What
should we do
with the ark
of the {\ND{Lord}}? Advise
us as to how
we should send
it back to its place.”
\par }{\PP \VS{3}They replied,
“If
you are going to send
the
ark
of the God
of Israel
back,
don’t
send
it away empty.
Be sure to return
it with a guilt offering.
Then
you will be healed,
and you will understand
why
his hand
is not
removed
from you.”
\VS{4}They inquired,
“What
is the guilt offering
that
we should send to him?”
\par }{\PP They replied,
“The Philistine
leaders
number
five.
So send five gold
sores
and five
gold
mice,
for
it is the same
plague
that has afflicted both
you and your leaders.
\VS{5}You
should make
images
of the sores
and images
of the mice
that are destroying
the land.
You should honor
the God
of Israel.
Perhaps
he will release his
grip
on
you, your gods,
and your land.
\VS{6}Why
harden
your hearts
like
the Egyptians
and Pharaoh
did? When God treated
them harshly,
didn’t the Egyptians
send
the Israelites on their way?
\VS{7}So now
go and make
a new
cart.
Get
two
cows
that have calves
and that
have never
had a yoke
placed
on
them. Harness
the cows
to the cart
and take their calves
from them back
to their stalls.
\VS{8}Then take
the
ark
of the {\ND{Lord}}
and place
it on the cart,
and put
in a chest
beside
it the gold
objects
you are sending to him as a guilt offering.
You should then send
it on its
way.
\VS{9}But
keep an eye
on it. If
it should go up by the way
of its own border
to Beth Shemesh,
then
he has brought this
great
calamity
on us. But if
that is not
the case, then we will know
that
it was not
his hand
that
struck
us; rather, it just happened
to us by accident.”
\par }{\PP \VS{10}So
the men
did as instructed.
They took
two
cows
that had calves
and harnessed
them to a cart;
they also removed
their calves
to their stalls.
\VS{11}They put
the ark
of the {\ND{Lord}}
on the cart,
along with the chest,
the gold
mice,
and the
images
of the sores.
\VS{12}Then the cows
went
directly
on
the road
to Beth Shemesh.
They went
along, mooing
as they went;
they turned
neither to the right
nor to the left.
The leaders
of the Philistines
were walking
along behind
them all the way
to the border
of Beth Shemesh.
\par }{\PP \VS{13}Now the residents of Beth Shemesh
were harvesting
wheat
in the valley.
When
they looked
up and saw
the ark,
they were pleased
at
the sight.
\VS{14}The cart
was coming
to
the field
of Joshua,
who was from Beth Shemesh.
It paused
there
near
a big
stone.
Then they cut up
the
wood
of the cart
and offered
the cows
as a burnt offering
to the
{\ND{Lord}}.
\VS{15}The Levites
took down
the ark
of the {\ND{Lord}}
and the
chest
that
was with
it, which
contained
the gold
objects. They placed
them
near the big
stone.
At that time
the people
of Beth Shemesh
offered
burnt offerings
and made sacrifices
to the
{\ND{Lord}}.
\VS{16}The five
leaders
of the Philistines
watched
what was happening and then returned
to Ekron
on the same day.
\par }{\PP \VS{17}These
are the gold
sores
that
the Philistines
brought as a guilt offering
to the
{\ND{Lord}} –
one
for each
of the following cities: Ashdod,
Gaza,
Ashkelon,
Gath,
and Ekron.
\VS{18}The gold
mice
corresponded in number
to all
the Philistine
cities
of the five
leaders,
from the fortified
cities
to
hamlet
villages,
to
greater
Abel,
where
they positioned
the
ark
of the {\ND{Lord}}
until
this very
day
in the field
of Joshua
who was from Beth Shemesh.
\par }{\PP \VS{19}But the
{\ND{Lord}} struck down
some of the people
of Beth Shemesh
because
they had looked
into the ark
of the {\ND{Lord}}; he struck down
50,070
of the men.
The people
grieved
because
the {\ND{Lord}}
had struck
the people
with a hard
blow.
\VS{20}The residents
of Beth Shemesh
asked, “Who
is able
to stand
before
the {\ND{Lord}},
this
holy
God? To
whom
will the ark go up from here?”
\par }{\PP \VS{21}So they sent
messengers
to
the residents
of Kiriath Jearim,
saying,
“The Philistines
have returned
the
ark
of the

{\ND{Lord}}. Come down
here and take it back home with you.”

\par }\Chap{7}{\PP \VerseOne{1}Then the people
of Kiriath Jearim
came
and took
the ark
of the {\ND{Lord}};
they brought
it to
the house
of Abinadab
located on the hill.
They consecrated
Eleazar
his son
to guard
the ark
of the {\ND{Lord}}.
\par }{\SH Further Conflict with the Philistines
\par }{\PP \VS{2}It was
quite
a long
time
– some
twenty
years in all – that the ark stayed at Kiriath Jearim. All the people of Israel longed for the
{\ND{Lord}}.
\VS{3}Samuel
said
to
all
the people
of Israel,
“If
you
are really
turning to
the {\ND{Lord}}
with all
your hearts,
remove
from among
you the foreign
gods
and the images of Ashtoreth.
Give your hearts
to
the {\ND{Lord}}
and serve
only
him. Then he will deliver
you from the hand
of the Philistines.”
\VS{4}So the Israelites
removed
the Baals
and images of Ashtoreth.
They served
only
the {\ND{Lord}}.
\par }{\PP \VS{5}Then Samuel
said,
“Gather
all
Israel
to Mizpah,
and I will pray
to
the {\ND{Lord}}
on
your behalf.”
\VS{6}After they had assembled
at Mizpah,
they drew
water
and poured
it out before
the {\ND{Lord}}. They fasted
on that day,
and they confessed
there,
“We have sinned
against the
{\ND{Lord}}.” So Samuel
led
the people of Israel
at Mizpah.
\par }{\PP \VS{7}When the Philistines
heard
that
the Israelites
had gathered
at Mizpah,
the leaders
of the Philistines
went up
against Israel.
When the Israelites
heard
about this, they were afraid
of the Philistines.
\VS{8}The Israelites
said
to
Samuel,
“Keep
crying
out to
the {\ND{Lord}}
our God
so that he may save
us from the hand
of the Philistines!”
\VS{9}So Samuel
took
a nursing
lamb
and offered
it as a whole
burnt offering
to the
{\ND{Lord}}. Samuel
cried
out to
the {\ND{Lord}}
on Israel’s
behalf,
and the
{\ND{Lord}}
answered him.
\par }{\PP \VS{10}As Samuel
was offering
burnt offerings,
the Philistines
approached
to do battle
with Israel.
But on that day
the {\ND{Lord}}
thundered
loudly
against
the Philistines.
He caused them to panic,
and they were defeated
by Israel.
\VS{11}Then
the men
of Israel
left Mizpah
and chased
the Philistines,
striking
them down all the way to an area
below
Beth Car.
\par }{\PP \VS{12}Samuel
took
a
stone
and placed
it between
Mizpah
and Shen.
He named
it Ebenezer,
saying,
“Up to
here
the {\ND{Lord}}
has helped us.”
\VS{13}So the Philistines
were defeated;
they did not
invade
Israel
again.
The hand
of the {\ND{Lord}}
was against the Philistines
all
the days
of Samuel.
\par }{\PP \VS{14}The cities
that
the Philistines
had captured
from Israel
were returned
to Israel,
from Ekron
to
Gath.
Israel
also delivered
their territory
from the control
of the Philistines.
There was
also peace
between
Israel
and the Amorites.
\VS{15}So Samuel
led
Israel
all
the days
of his life.
\VS{16}Year
after
year
he used to travel
the circuit
of Bethel,
Gilgal,
and Mizpah;
he used to judge
Israel
in all
of these
places.
\VS{17}Then he would return
to Ramah,
because
his home
was there.
He also judged
Israel
there
and built
an altar
to the
{\ND{Lord}}
there.

\par }\Chap{8}{\PP \VerseOne{1}In his old
age Samuel
appointed
his sons
as judges
over Israel.
\VS{2}The name
of his firstborn
son
was Joel,
and the name
of his second
son was Abijah.
They were
judges
in Beer Sheba.
\VS{3}But his sons
did not
follow
his ways.
Instead, they made money dishonestly,
accepted
bribes,
and perverted
justice.
\par }{\PP \VS{4}So all
the elders
of Israel
gathered together
and approached
Samuel
at Ramah.
\VS{5}They said
to him,
“Look,
you
are old,
and your sons
don’t
follow
your ways.
So now
appoint
over us a king
to lead
us, just like all
the other nations have.”
\par }{\PP \VS{6}But this request
displeased
Samuel,
for they said,
“Give
us a king
to lead
us.” So Samuel
prayed
to
the {\ND{Lord}}.
\VS{7}The
{\ND{Lord}}
said
to
Samuel,
“Do
everything
the people
request
of you. For
it is not
you that
they have rejected,
but it is me
that
they have rejected
as their king.
\VS{8}Just as
they have done
from the day
that I brought them up
from Egypt
until
this
very day,
they have rejected
me and have served
other
gods.
This
is what they
are also
doing to you.
\VS{9}So now
do
as they say. But
seriously
warn
them and make them aware of the policies
of the king
who
will rule
over them.”
\par }{\PP \VS{10}So Samuel
spoke all
the words
of the {\ND{Lord}}
to
the people
who were asking
him for a king.
\VS{11}He said,
“Here
are the policies
of the king
who
will rule over
you: He will conscript
your sons
and put
them in his chariot forces
and in his cavalry;
they will run
in front
of his chariot.
\VS{12}He will appoint
for himself leaders
of thousands
and leaders
of fifties,
as well as those who plow
his ground,
reap
his harvest,
and make
his weapons
of war
and his chariot
equipment.
\VS{13}He will take
your daughters
to be ointment makers,
cooks,
and bakers.
\VS{14}He will take
your best
fields
and vineyards
and give
them to his own servants.
\VS{15}He will demand a tenth
of your seed
and of the produce of your vineyards
and give
it to his administrators
and his servants.
\VS{16}He will take
your male
and female
servants,
as well as your best
cattle
and your donkeys,
and assign
them for his own use.
\VS{17}He will demand a tenth
of your flocks,
and you yourselves
will be
his servants.
\VS{18}In that day
you will cry
out because
of your king
whom
you have chosen
for yourselves, but the
{\ND{Lord}}
won’t answer
you in that day.”
\par }{\PP \VS{19}But the people
refused
to heed
Samuel’s
warning. Instead they said,
“No! There will be
a king
over us!
\VS{20}We
will be like
all
the other nations.
Our king
will judge
us and lead
us and fight
our battles.”
\par }{\PP \VS{21}So Samuel
listened
to everything
the people
said
and then reported
it to the
{\ND{Lord}}.
\VS{22}The
{\ND{Lord}}
said
to
Samuel,
“Do
as they say and install
a king
over them.”
Then Samuel
said
to
the men
of Israel,
“Each
of you go
back to his own
city.”

\par }\Chap{9}{\PP \VerseOne{1}There was
a Benjaminite
man
named
Kish
son
of Abiel,
the son
of Zeror,
the son
of Becorath,
the son
of Aphiah
of Benjamin. He was a prominent
person.
\VS{2}He had
a son
named
Saul,
a handsome
young man.
There was no
one
among the Israelites
more
handsome
than
he was; he stood head
and shoulders
above
all
the people.
\par }{\PP \VS{3}The donkeys
of Saul’s
father
Kish
wandered off,
so Kish
said
to
his son
Saul,
“Take
one
of the servants
with you and go
look for
the donkeys.”
\VS{4}So Saul crossed
through the hill country
of Ephraim,
passing through
the land
of Shalisha,
but they did not
find
them. So they crossed
through
the land
of Shaalim,
but they were not
there. Then he crossed through
the land
of Benjamin, and still they did not
find them.
\par }{\PP \VS{5}When they
came
to the land
of Zuph,
Saul
said
to his servant
who
was with
him, “Come
on, let’s
head back
before
my father
quits
worrying
about the donkeys
and becomes anxious about us!”
\VS{6}But the servant said
to him, “Look,
there is a man
of God
in this
town.
He
is highly respected.
Everything
that
he says
really happens.
Now
let’s go
there.
Perhaps
he will tell
us where
we should go
from
here.”
\VS{7}So Saul
said
to his servant,
“All right,
we can go.
But what
can we bring
the man,
since
the food
in our bags
is used up? We have no
gift
to take
to the man
of God.
What
do we have?”
\VS{8}The servant
went
on to answer
Saul,
“Look,
I happen to have
in my hand
a quarter
shekel
of silver.
I will give
it to the man
of God
and he will tell
us where we should go.”
\VS{9}(Now it used to be in Israel
that
whenever someone
went
to inquire
of God
he would say, “Come
on, let’s
go
to
the seer.”
For
today’s
prophet
used to be called
a seer.)
\VS{10}So Saul
said
to his servant,
“That’s a good
idea! Come
on.
Let’s go.”
So they went
to
the town
where
the man
of God was.
\par }{\PP \VS{11}As they
were going up
the ascent
to the town,
they
met
some
girls
coming out
to draw
water.
They said
to them, “Is
this
where the seer is?”
\VS{12}They replied,
“Yes, straight
ahead! But hurry
now,
for
he came
to the town
today,
and the people
are making
a sacrifice
at the high place.
\VS{13}When you enter
the town,
you can find
him before
he goes up
to the high place
to eat.
The people
won’t
eat
until
he arrives,
for
he must bless
the sacrifice.
Once
that
happens,
those who have been
invited
will eat.
Now
go on up,
for
this
is the time
when you can find him!”
\par }{\PP \VS{14}So they went up
to the town.
As they
were heading
for the middle
of the town,
Samuel
was coming
in their direction
to go up
to the high place.
\VS{15}Now the day
before
Saul
arrived,
the {\ND{Lord}}
had told
Samuel:
\VS{16}“At this time
tomorrow
I will send
to
you a man
from the land
of Benjamin.
You must consecrate
him as a leader
over
my people
Israel.
He will save
my people
from the hand
of the Philistines.
For
I have looked
with favor on
my people.
Their cry
has reached
me!”
\par }{\PP \VS{17}When Samuel
saw
Saul,
the {\ND{Lord}}
said, “Here
is the man
that
I told
you about! He will rule
over my people.”
\VS{18}As Saul
approached
Samuel
in the middle
of the gate,
he said,
“Please
tell
me where
the seer’s
house is.”
\par }{\PP \VS{19}Samuel
replied
to Saul,
“I am
the seer! Go up
in front
of me to the high place! Today
you will eat
with
me and in the morning
I will send
you away. I will tell
you everything
that
you are thinking.
\VS{20}Don’t
be concerned
about the donkeys
that you lost
three
days
ago, for
they have been found.
Whom
does all
Israel
desire? Is it not
you, and all
your father’s
family?”
\par }{\PP \VS{21}Saul
replied,
“Am I
not
a Benjaminite,
from the smallest
of Israel’s
tribes,
and is not my family
clan
the smallest
of all
the tribes
of Benjamin? Why
do you speak
to me
in this
way?”
\par }{\PP \VS{22}Then Samuel
brought
Saul
and his servant
into the room
and gave
them a place
at the head
of those who had been invited.
There were about thirty
people present.
\VS{23}Samuel
said
to the cook,
“Give
me the portion
of meat that
I gave to you – the one I asked you to keep with you.”
\par }{\PP \VS{24}So
the cook
picked
up the leg
and brought it
and set
it in front
of Saul.
Samuel said,
“What
was kept
is now
set
before
you! Eat,
for
it has been kept
for you for this meeting
time, from the time I said,
‘I have invited
the people.’ ”
So Saul
ate
with
Samuel
that day.
\par }{\PP \VS{25}When they came down
from the high place
to the town,
Samuel spoke
with
Saul
on
the roof.
\VS{26}They got
up
at dawn
and Samuel
called
to
Saul
on the roof,
“Get
up, so I can send
you on your way.” So Saul
got
up and the two
of them
– he
and Samuel
– went
outside.
\VS{27}While they
were going down
to the edge
of town,
Samuel
said to
Saul,
“Tell
the servant
to go
on ahead
of us.” So
he did. Samuel
then said, “You
remain
here awhile,
so I can inform
you of God’s
message.”

\par }\Chap{10}{\PP \VerseOne{1}Then Samuel
took
a small container
of olive oil
and poured
it on
Saul’s head.
Samuel kissed
him and said,
“The
{\ND{Lord}} has chosen you to lead his people Israel! You will rule over the
{\ND{Lord}}’s people and you will deliver them from the power of the enemies who surround them. This
will be your sign that
the {\ND{Lord}}
has chosen you as leader
over
his inheritance.
\VS{2}When you leave
me
today,
you will find
two
men
near Rachel’s
tomb
at Zelzah
on Benjamin’s
border.
They will say
to
you, ‘The donkeys
you have gone
looking for
have been found.
Your father
is no longer concerned
about the donkeys
but has become anxious
about you two! He is asking,
“What
should I do
about my son?” ’
\par }{\PP \VS{3}“As you continue
on
from there,
you will come
to
the tall tree
of Tabor.
At that point
three
men
who are going up
to
God
at Bethel
will meet
you. One
of them will be carrying
three
young goats,
one
of them will be carrying
three
round loaves
of bread,
and one
of them will be carrying
a container
of wine.
\VS{4}They will ask
you how you’re
doing and will give
you two
loaves of bread.
You will accept
them.
\VS{5}Afterward
you will go
to Gibeah
of God,
where
there
are Philistine
officials.
When
you enter
the town,
you will meet
a company
of prophets
coming down
from the high place.
They will have harps,
tambourines,
flutes,
and lyres,
and they
will be prophesying.
\VS{6}Then the spirit
of the {\ND{Lord}}
will rush
upon
you and you will prophesy
with
them. You will be changed
into a different
person.
\par }{\PP \VS{7}“When
these
signs
have taken place,
do
whatever
your hand
finds
to do, for
God
will be with you.
\VS{8}You will go down
to Gilgal
before
me. I am
going to
join
you there
to
offer
burnt offerings
and to make peace offerings.
You should wait
for seven
days,
until
I arrive
and tell you what
to do.”
\par }{\SH Saul Becomes King
\par }{\PP \VS{9}As Saul turned
to leave
Samuel,
God
changed
his inmost
person.
All
these
signs
happened
on that very day.
\VS{10}When Saul and his servant arrived
at Gibeah,
a company
of prophets
was coming out to meet
him. Then the spirit
of God
rushed
upon
Saul and he prophesied
among them.
\VS{11}When
everyone who had known
him previously
saw
him prophesying
with
the prophets,
the people
all
asked
one
another,
“What
on earth has happened
to the son
of Kish? Does even
Saul
belong with the prophets?”
\par }{\PP \VS{12}A man
who was from there
replied,
“And who
is their father?” Therefore
this became
a proverb: “Is even
Saul
among the prophets?”
\VS{13}When Saul had finished
prophesying,
he went
to the high place.
\par }{\PP \VS{14}Saul’s
uncle
asked
him
and his servant,
“Where
did you go?” Saul replied,
“To look
for the donkeys.
But when
we
realized
they were lost,
we went
to
Samuel.”
\VS{15}Saul’s
uncle
said,
“Tell
me
what
Samuel
said to you.”
\VS{16}Saul
said
to
his uncle,
“He assured
us that
the donkeys
had been found.”
But Saul did not
tell
him what
Samuel
had
said
about the matter
of kingship.
\par }{\PP \VS{17}Then Samuel
called
the people
together before
the {\ND{Lord}}
at Mizpah.
\VS{18}He said
to
the Israelites,
“This is what
the {\ND{Lord}}
God
of Israel
says, ‘I
brought
Israel
up from Egypt
and I delivered
you from the
power
of the Egyptians
and from the power
of all
the kingdoms
that oppressed
you.
\VS{19}But
today
you
have rejected
your God
who
saves
you from all
your trouble
and distress.
You have said,
“No! Appoint
a king
over
us.” Now
take your positions
before
the {\ND{Lord}}
by your tribes and by your clans.’ ”
\par }{\PP \VS{20}Then
Samuel
brought all
the tribes
of Israel
near,
and the tribe
of Benjamin
was chosen by lot.
\VS{21}Then he brought
the tribe
of Benjamin
near by its families,
and the family
of Matri
was chosen by lot. At last Saul
son
of Kish
was chosen by
lot. But when they looked
for him, he was nowhere to be found.
\VS{22}So they inquired
again
of the {\ND{Lord}}, “Has the man
arrived
here
yet?” The
{\ND{Lord}}
said,
“He
has
hidden
himself among the equipment.”
\par }{\PP \VS{23}So they ran
and brought
him from there.
When
he took
his position
among
the people,
he stood head and shoulders
above
them all.
\VS{24}Then Samuel
said
to
all
the people,
“Do you see
the one whom
the {\ND{Lord}}
has chosen? Indeed,
there is no
one like
him among all
the people!” All
the people
shouted
out, “Long live
the king!”
\par }{\PP \VS{25}Then
Samuel
talked
to
the people
about how the kingship
would work. He wrote
it all down on a scroll
and set
it before
the {\ND{Lord}}. Then Samuel
sent
all
the people
away to their homes.
\VS{26}Even
Saul
went
to his home
in Gibeah.
With him
went
some brave
men whose
hearts
God
had
touched.
\VS{27}But some wicked
men said,
“How
can this
man save
us?” They despised
him and did not
even bring
him a gift.
But Saul said nothing about it.

\par }\Chap{11}{\PP \VerseOne{1}Nahash
the Ammonite
marched
against
Jabesh
Gilead.
All
the men
of Jabesh
Gilead
said to
Nahash,
“Make
a treaty
with us and we will serve you.”
\par }{\PP \VS{2}But Nahash
the Ammonite
said
to
them, “The only way I will make a treaty with you is if you let me gouge
out the right
eye
of every
one of you and in so
doing humiliate
all
Israel!”
\par }{\PP \VS{3}The elders
of Jabesh
said
to him,
“Leave
us alone for seven
days
so that we can send
messengers
throughout
the territory
of Israel.
If
there is no
one who can deliver
us, we will come out
voluntarily to you.”
\par }{\PP \VS{4}When the messengers
went
to Gibeah
(where Saul
lived) and informed
the people
of these matters, all
the people
wept
loudly.
\VS{5}Now
Saul
was walking behind
the oxen
as he came
from
the field.
Saul
asked,
“What
has happened to the people? Why are they weeping?” So they told
him about
the men
of Jabesh.
\par }{\PP \VS{6}The Spirit
of God
rushed
upon
Saul
when he heard
these
words,
and he became very
angry.
\VS{7}He took
a pair
of oxen
and cut
them up. Then he sent
the pieces throughout
the territory
of Israel
by the hand
of messengers,
who said,
“Whoever
does not
go out
after
Saul
and after
Samuel
should expect this
to be done
to his oxen!” Then the terror
of the {\ND{Lord}}
fell
on
the people,
and they went out
as one army.
\VS{8}When Saul counted
them at Bezek,
the Israelites
were
300,000
strong and the men
of Judah
numbered 30,000.
\par }{\PP \VS{9}They said
to the messengers
who had come,
“Here’s what
you should say
to the men
of Jabesh
Gilead: ‘Tomorrow
deliverance
will come
to you when the sun
is fully up.’ ”
When the messengers
went and told
the men
of Jabesh
Gilead, they were happy.
\VS{10}The men
of Jabesh
said,
“Tomorrow
we will come out
to
you and you can do
with us whatever
you wish.”
\par }{\PP \VS{11}The next day
Saul
placed
the
people
in three
groups.
They went
to
the Ammonite
camp
during the morning
watch
and struck
them down
until
the hottest part
of the day.
The survivors
scattered;
no
two
of them remained
together.
\par }{\SH Saul Is Established as King
\par }{\PP \VS{12}Then the people
said to
Samuel,
“Who
were the ones asking, ‘Will Saul
reign
over
us?’ Hand
over those men
so we may execute them!”
\VS{13}But Saul
said,
“No
one will be killed
on
this
day.
For
today
the {\ND{Lord}}
has given Israel
a victory!”
\VS{14}Samuel
said
to
the people,
“Come on! Let’s
go
to Gilgal
and renew
the kingship
there.”
\VS{15}So
all
the people
went to Gilgal,
where
they established
Saul
as king
in the
{\ND{Lord}}’s
presence.
They offered
up peace offerings
there
in the
{\ND{Lord}}’s
presence.
Saul
and all
the Israelites
were very
happy.


\par }\Chap{12}{\PP \VerseOne{1}Samuel
said
to
all
Israel,
“I
have done
everything
you requested.
I have
given you a king.
\VS{2}Now
look! This king
walks
before
you. As for me, I am
old
and gray,
though my sons
are here
with
you. I
have walked
before
you from the time of my youth
till
the present day.
\VS{3}Here I am.
Bring a charge
against me before
the {\ND{Lord}}
and before
his chosen king.
Whose
ox
have I taken? Whose
donkey
have I taken? Whom
have I wronged? Whom
have I oppressed? From whose
hand
have I taken
a bribe
so
that I would overlook
something? Tell me, and I will return it to you!”
\par }{\PP \VS{4}They replied,
“You have
not
wronged
us or oppressed
us. You have not
taken
anything
from the hand
of anyone.”
\VS{5}He said
to
them, “The
{\ND{Lord}}
is witness
against you, and his chosen king
is witness
this
day,
that
you have not
found
any reason
to accuse
me.” They said,
“He is witness!”
\par }{\PP \VS{6}Samuel
said
to
the people,
“The
{\ND{Lord}}
is the one who
chose
Moses
and Aaron
and who
brought
your ancestors
up from the land
of Egypt.
\VS{7}Now
take your positions,
so I may confront
you before
the
{\ND{Lord}}
regarding
all
the
{\ND{Lord}}’s
just actions
toward you
and your ancestors.
\VS{8}When
Jacob
entered
Egypt,
your ancestors
cried
out to
the {\ND{Lord}}. The
{\ND{Lord}}
sent
Moses
and Aaron,
and they led
your ancestors
out of Egypt
and settled
them in this
place.
\par }{\PP \VS{9}“But they forgot
the {\ND{Lord}}
their God,
so he gave
them into the hand
of Sisera,
the general in command
of Hazor’s
army,
and into the hand
of the Philistines
and into the hand
of the king
of Moab,
and they fought against them.
\VS{10}Then they cried
out to
the {\ND{Lord}}
and admitted, ‘We have sinned,
for
we have forsaken
the

{\ND{Lord}}
and have served
the
Baals
and the images of Ashtoreth.
Now
deliver
us from the hand
of our enemies
so that we may serve you.’
\VS{11}So the
{\ND{Lord}}
sent
Jerub-Baal,
Barak,
Jephthah,
and Samuel,
and he delivered
you from the
hand
of the enemies
all around
you, and you were able to live
securely.
\par }{\PP \VS{12}“When you saw
that
King
Nahash
of the Ammonites
was advancing
against
you, you said
to me, ‘No! A king
will rule over us’ – even though the
{\ND{Lord}} your God is your king!
\VS{13}Now
look! Here is the king
you have
chosen
– the one that
you asked
for! Look,
the {\ND{Lord}}
has given
you a king!
\VS{14}If
you fear
the {\ND{Lord}}, serving
him and obeying him
and not
rebelling
against what he says, and if both
you
and the king
who
rules
over
you follow
the {\ND{Lord}}
your God,
all
will be
well.
\VS{15}But if
you don’t
obey
the {\ND{Lord}}
and rebel
against
what the
{\ND{Lord}}
says,
the hand
of the {\ND{Lord}}
will be
against both you and your king.
\par }{\PP \VS{16}“So now,
take your positions
and watch this
great
thing
that
the {\ND{Lord}}
is about
to do
in your sight.
\VS{17}Is this not
the time
of the wheat
harvest? I will call
on the
{\ND{Lord}}
so
that he makes
it thunder
and rain.
Realize
and see
what
a great
sin you have
committed
before
the {\ND{Lord}}
by asking
for a king for yourselves.”
\par }{\PP \VS{18}So Samuel
called
to
the {\ND{Lord}}, and the
{\ND{Lord}}
made
it thunder
and rain
that day.
All
the people
were very
afraid
of both the
{\ND{Lord}}
and Samuel.
\VS{19}All
the people
said
to
Samuel,
“Pray
to
the {\ND{Lord}}
your God
on behalf of us – your servants – so we won’t die, for we have added to all our sins by asking for a king.”
\par }{\PP \VS{20}Then Samuel
said
to
the people,
“Don’t
be afraid.
You
have indeed sinned.
However,
don’t
turn
aside
from the
{\ND{Lord}}. Serve
the {\ND{Lord}}
with all
your heart.
\VS{21}You should not
turn aside
after
empty
things that
can’t
profit
and can’t
deliver,
since
they are
empty.
\VS{22}The
{\ND{Lord}}
will not
abandon
his people
because
he wants to uphold
his great
reputation.
The
{\ND{Lord}}
was pleased to make
you his own people.
\VS{23}As
far as
I am
concerned, far be it from me
to sin
against the
{\ND{Lord}}
by ceasing
to pray
for
you! I will instruct
you in the
way
that is good
and upright.
\VS{24}However,
fear
the {\ND{Lord}}
and serve
him faithfully
with all
your heart.
Just
look
at the great
things he has done for you!
\VS{25}But if
you continue to do evil,
both
you
and your king
will be swept away.”

\par }\Chap{13}{\PP \VerseOne{1}Saul
was [thirty] years
old when he began to reign;
he ruled
over
Israel
for [forty] years.
\VS{2}Saul
selected
for himself three
thousand
men from Israel.
Two thousand
of these were with
Saul
at Micmash
and in the hill country
of Bethel;
the remaining thousand
were with
Jonathan
at Gibeah
in the territory of Benjamin.
He sent
all the rest
of the people
back home.
\par }{\PP \VS{3}Jonathan
attacked
the Philistine
outpost
that
was at Geba
and the Philistines
heard
about it. Then Saul
alerted
all
the land
saying,
“Let the Hebrews
pay attention!”
\VS{4}All
Israel
heard
this message, “Saul
has attacked
the Philistine
outpost,
and now Israel
is repulsive
to the Philistines!” So the people
were summoned
to join
Saul
at Gilgal.
\par }{\PP \VS{5}For the battle
with
Israel
the Philistines
had amassed
3,000
chariots,
6,000
horsemen,
and an army
as numerous as the sand
on
the seashore.
They went up
and camped
at Micmash,
east
of Beth Aven.
\VS{6}The men
of Israel
realized
they had a problem
because
their army
was hard pressed.
So the army
hid
in caves,
thickets,
cliffs,
strongholds,
and cisterns.
\VS{7}Some of the Hebrews
crossed
over the
Jordan River
to the land
of Gad
and Gilead.
But Saul
stayed
at Gilgal;
the entire
army
that was with him
was terrified.
\VS{8}He waited
for seven
days,
the time
period indicated
by Samuel.
But Samuel
did not
come
to Gilgal,
and the army
began to abandon Saul.
\par }{\PP \VS{9}So Saul
said,
“Bring
me
the burnt offering
and the peace offerings.”
Then
he offered
a burnt offering.
\VS{10}Just when
he had finished
offering
the burnt offering,
Samuel
appeared
on the scene. Saul
went out
to meet
him and to greet him.
\par }{\PP \VS{11}But Samuel
said,
“What
have you done?” Saul
replied,
“When I saw
that
the army
had started
to abandon me and that you
didn’t
come
at the appointed
time
and that the Philistines
had assembled
at Micmash,
\VS{12}I thought, ‘Now
the Philistines
will come down
on
me at Gilgal
and I have not
sought
the
{\ND{Lord}}’s
favor.’
So I felt obligated
to offer
the burnt offering.”
\par }{\PP \VS{13}Then Samuel
said
to
Saul,
“You have made a foolish
choice! You have not
obeyed
the
commandment
that the
{\ND{Lord}}
your God
gave
you. Had you done that,
the {\ND{Lord}}
would have established
your kingdom
over Israel
forever!
\VS{14}But now
your kingdom
will not
continue! The
{\ND{Lord}}
has sought
out for himself
a man
who is loyal
to him
and the
{\ND{Lord}}
has appointed
him to be leader
over
his people,
for
you have not
obeyed
what
the
{\ND{Lord}}
commanded you.”
\par }{\PP \VS{15}Then
Samuel
set out and went up
from
Gilgal
to Gibeah
in the territory of Benjamin.
Saul
mustered
the army
that remained
with
him; there were about six
hundred
men.
\VS{16}Saul,
his son
Jonathan,
and the army
that remained
with
them stayed
in Gibeah
in the territory of Benjamin,
while the Philistines
camped
in Micmash.
\VS{17}Raiding bands
went out
from the camp
of the Philistines
in three
groups.
One
band
turned
toward
the road
leading to Ophrah
by the land
of Shual;
\VS{18}another band
turned
toward the road
leading to Beth Horon;
and yet another band
turned
toward the road
leading to the border
that overlooks
the valley
of Zeboim
in the direction of the desert.
\par }{\PP \VS{19}A blacksmith
could not
be found
in all
the land
of Israel,
for
the Philistines
had
said,
“This will prevent
the Hebrews
from making swords
and
spears.”
\VS{20}So all
Israel
had to go down
to the Philistines
in order to get their plowshares,
cutting instruments,
axes,
and sickles
sharpened.
\VS{21}They charged two-thirds of a shekel
to sharpen
plowshares
and cutting instruments,
and a third
of a shekel to sharpen picks
and axes,
and to set
ox goads.
\VS{22}So
on the day
of the battle
no
sword
or spear
was to be found
in the hand
of anyone
in the army
that
was with
Saul
and
Jonathan.
No
one but Saul
and his son
Jonathan had them.
\par }{\SH Jonathan Ignites a Battle
\par }{\PP \VS{23}A garrison
of the Philistines
had gone out to
the pass
at Micmash.

\par }\Chap{14}{\PP \VerseOne{1}Then
one day
Jonathan
son
of Saul
said
to
his armor
bearer, “Come
on, let’s go over
to
the Philistine
garrison
that
is opposite
us.” But he did not
let
his father
know.
\par }{\PP \VS{2}Now Saul
was sitting
under
a pomegranate
tree in Migron,
on the outskirts
of Gibeah.
The army
that
was with
him numbered about six
hundred
men.
\VS{3}Now Ahijah
was carrying
an ephod.
He was the son
of Ahitub,
who was the brother
of Ichabod
and a son
of Phineas,
son
of Eli,
the priest
of the {\ND{Lord}}
in Shiloh.
The army
was unaware
that
Jonathan
had left.
\par }{\PP \VS{4}Now there was a steep
cliff
on each side
of the pass
through
which
Jonathan
intended to go to reach the Philistine
garrison.
One cliff
was named
Bozez,
the other
Seneh.
\VS{5}The cliff
to the north
was closer
to Micmash,
the one
to the south
closer
to Geba.
\par }{\PP \VS{6}Jonathan
said
to
his armor
bearer,
“Come on,
let’s go over
to
the garrison
of these
uncircumcised men.
Perhaps
the {\ND{Lord}}
will intervene
for
us. Nothing
can prevent
the {\ND{Lord}}
from delivering,
whether by many
or by
a few.”
\VS{7}His armor
bearer
said
to him, “Do
everything
that
is on your mind.
Do as you’re inclined.
I’m with
you all the way!”
\par }{\PP \VS{8}Jonathan
replied,
“All right! We’ll
go over
to
these men
and fight them.
\VS{9}If
they say
to
us, ‘Stay
put until
we approach
you,’ we will stay
right there
and not
go up
to them.
\VS{10}But if
they say,
‘Come up
against
us,’ we will go up.
For
in that case the
{\ND{Lord}}
has given
them into our hand
– it will be a sign to us.”
\par }{\PP \VS{11}When
they made themselves known
to
the Philistine
garrison,
the Philistines
said,
“Look! The Hebrews
are coming out
of the holes
in which
they hid themselves.”
\VS{12}Then the men
of the garrison
said
to Jonathan
and his armor
bearer,
“Come on up
to us so we can teach
you a thing
or two!” Then Jonathan
said
to
his armor
bearer,
“Come up
behind
me, for
the {\ND{Lord}}
has given
them into the hand
of Israel!”
\par }{\PP \VS{13}Jonathan
crawled up
on
his hands
and feet,
with his armor
bearer
following behind
him. Jonathan
struck down
the Philistines, while his armor
bearer
came along behind
him and killed them.
\VS{14}In this initial
skirmish
Jonathan
and his armor
bearer
struck
down about twenty
men
in an area
that measured half
an acre.
\par }{\PP \VS{15}Then fear overwhelmed
those who were in the camp,
those who were in the field,
all
the army
in the garrison,
and the raiding bands.
They
trembled
and the ground
shook.
This fear
was
caused by God.
\par }{\PP \VS{16}Saul’s
watchmen
at Gibeah
in the territory of Benjamin
looked
on
as the crowd
of soldiers seemed to melt
away
first in one direction
and then in another.
\VS{17}So Saul
said
to the army
that
was with
him, “Muster
the troops and see
who
is no longer with
us.” When
they mustered
the troops,
Jonathan
and his armor
bearer
were not there.
\VS{18}So Saul
said
to Ahijah,
“Bring near
the ephod,” for
he was
at that time wearing the ephod.
\VS{19}While
Saul
spoke to
the priest,
the panic
in the Philistines’
camp
was becoming
greater and greater.
So Saul
said
to
the priest,
“Withdraw
your hand!”
\par }{\PP \VS{20}Saul
and all
the army
that
was with
him assembled and marched
into battle,
where they found
the Philistines
in total panic
killing
one
another
with
their swords.
\VS{21}The Hebrews
who had earlier
gone
over to the Philistine
side
joined
the Israelites
who
were with
Saul
and Jonathan.
\VS{22}When
all
the Israelites
who had hidden
themselves in the hill country
of Ephraim
heard
that
the Philistines
had fled,
they too
pursued
them
in battle.
\VS{23}So the
{\ND{Lord}}
delivered
Israel
that day,
and the battle
shifted
over to Beth Aven.
\par }{\SH Jonathan Violates Saul’s Oath
\par }{\PP \VS{24}Now the men
of Israel
were hard pressed
that day,
for Saul
had made
the army
agree to this oath: “Cursed
be the man
who
eats
food
before
evening! I will get my vengeance
on my enemies!” So no
one
in the army
ate
anything.
\par }{\PP \VS{25}Now the whole
army
entered
the forest
and there was
honey
on
the ground.
\VS{26}When the army
entered
the forest,
they saw
the honey
flowing,
but no
one ate
any of it,
for
the army
was afraid
of the oath.
\VS{27}But Jonathan
had not
heard
about the oath
his father
had made the army
take. He extended
the
end
of his staff
that
was in his hand
and dipped
it in the
honeycomb.
When
he ate it, his eyes gleamed.
\VS{28}Then
someone
from the army
informed him, “Your father
put the army
under a strict
oath
saying,
‘Cursed
be the man
who eats
food
today!’ That is why the army
is tired.”
\VS{29}Then Jonathan
said,
“My father
has caused
trouble for the land.
See
how
my eyes
gleamed
when
I tasted
just a little
of this
honey.
\VS{30}Certainly if
the army
had eaten
some of the enemies’
provisions
that
they came across
today,
would not
the slaughter
of the Philistines
have been even greater?”
\par }{\PP \VS{31}On that day
the army
struck
down the Philistines
from Micmash
to Aijalon,
and they became very
tired.
\VS{32}So
the army
rushed greedily
on the plunder,
confiscating
sheep,
cattle,
and calves.
They slaughtered
them right on the ground,
and the army
ate
them
blood and all.
\par }{\PP \VS{33}Now it was reported
to Saul,
“Look,
the army
is sinning
against the
{\ND{Lord}}
by eating
even the blood.”
He said,
“All of you have broken the covenant! Roll
a large
stone
over here to me.”
\VS{34}Then Saul
said,
“Scatter out
among the army
and say
to them, ‘Each
of you bring
to
me your ox
and sheep
and slaughter
them in this
spot and eat.
But don’t
sin
against
the {\ND{Lord}}
by eating
the blood.”
So
that night
each
one
brought his ox
and slaughtered
it there.
\VS{35}Then Saul
built
an altar
for the
{\ND{Lord}}; it was the first
time he had built
an altar
for the
{\ND{Lord}}.
\par }{\PP \VS{36}Saul
said,
“Let’s go down
after
the Philistines
at night;
we will rout
them until
the break of day.
We won’t leave
any of them alive!” They replied,
“Do
whatever
seems
best to
you.” But the priest
said,
“Let’s approach
God
here.”
\VS{37}So Saul
asked
God,
“Should I go down
after
the Philistines? Will
you deliver
them into the hand
of Israel?” But he did not
answer
him that day.
\par }{\PP \VS{38}Then Saul
said,
“All
you leaders
of the army
come
here. Find
out how
this
sin
occurred
today.
\VS{39}For
as surely
as the
{\ND{Lord}}, the deliverer
of Israel,
lives, even
if
it turns
out to be my own son
Jonathan,
he will certainly
die!” But no
one from the army
said
anything.
\par }{\PP \VS{40}Then he said
to
all
Israel,
“You
will be
on one
side,
and I
and my son
Jonathan
will be
on the other
side.”
The army
replied
to
Saul,
“Do
whatever
you think is best.”
\par }{\PP \VS{41}Then Saul
said,
“O
{\ND{Lord}}
God
of Israel! If this sin has been committed by me or by my son Jonathan, then, O
{\ND{Lord}} God of Israel, respond with Urim. But if this sin has been committed by your people Israel, respond with Thummim.” Then Jonathan
and Saul
were indicated by lot, while the army
was exonerated.
\VS{42}Then Saul
said,
“Cast
the lot between
me and my son
Jonathan!” Jonathan
was indicated by lot.
\par }{\PP \VS{43}So Saul
said
to
Jonathan,
“Tell
me what
you have done.”
Jonathan
told
him, “I used the end
of the staff
that
was in my hand
to taste
a little
honey.
I
must die!”
\VS{44}Saul
said,
“God
will punish me severely
if
Jonathan
doesn’t die!”
\par }{\PP \VS{45}But the army
said to
Saul,
“Should Jonathan,
who won
this
great
victory in Israel,
die? May it never be! As surely
as the
{\ND{Lord}}
lives, not
a single
hair
of his head
will fall
to the ground! For
it is with
the help of God
that he has acted
today.”
So the army
rescued
Jonathan
from death.
\par }{\PP \VS{46}Then
Saul
stopped
chasing
the Philistines,
and the Philistines
went
back home.
\VS{47}After Saul
had secured
his royal
position over
Israel,
he fought
against all
their enemies
on all sides
– the Moabites,
Ammonites,
Edomites,
the kings
of Zobah,
and the Philistines.
In every
direction that he turned
he was victorious.
\VS{48}He fought bravely,
striking down
the Amalekites
and delivering
Israel
from the hand
of its enemies.
\par }{\SH Members of Saul’s Family
\par }{\PP \VS{49}The sons
of Saul
were Jonathan,
Ishvi,
and Malki-Shua.
He had two
daughters;
the older one
was named
Merab
and the younger
Michal.
\VS{50}The name
of Saul’s
wife
was Ahinoam,
the daughter
of Ahimaaz.
The name
of the general
in command of his army
was Abner
son
of Ner,
Saul’s
uncle.
\VS{51}Kish
was the father
of Saul,
and Ner
the father
of Abner
was the son
of Abiel.
\par }{\PP \VS{52}There was
fierce
war
with the Philistines
all
the days
of Saul.
So whenever Saul
saw
anyone
who was a warrior
or a brave individual,
he would conscript
him.

\par }\Chap{15}{\PP \VerseOne{1}Then Samuel
said
to
Saul,
“I was the one the

{\ND{Lord}}
sent
to anoint
you as king
over
his people
Israel.
Now
listen
to what
the {\ND{Lord}}
says.
\VS{2}Here is what
the {\ND{Lord}}
of hosts
says: ‘I carefully
observed
how the Amalekites
opposed
Israel
along the way
when Israel came up
from Egypt.
\VS{3}So go
now
and strike down
the Amalekites.
Destroy
everything
that
they have. Don’t
spare
them. Put them to
death
– man,
woman,
child,
infant,
ox,
sheep,
camel,
and donkey alike.’ ”
\par }{\PP \VS{4}So Saul
assembled
the army
and mustered
them at Telaim.
There were 200,000
foot
soldiers and 10,000
men
of Judah.
\VS{5}Saul
proceeded
to the city
of Amalek,
where he set an ambush
in the wadi.
\VS{6}Saul
said
to
the Kenites,
“Go
on and leave! Go down
from among
the Amalekites! Otherwise
I will sweep
you away
with
them! After all, you
were kind
to
all
the Israelites
when they came up
from Egypt.”
So the Kenites
withdrew
from among
the Amalekites.
\par }{\PP \VS{7}Then Saul
struck down
the Amalekites
all the way from Havilah
to Shur,
which
is next to
Egypt.
\VS{8}He captured
King
Agag
of the Amalekites
alive,
but he executed all
Agag’s
people
with the sword.
\VS{9}However,
Saul
and the army
spared Agag,
along with the best
of the flock,
the cattle,
the fatlings,
and the lambs, as well as
everything
else that was of value. They were not
willing
to slaughter them.
But they did slaughter
everything
that was
despised
and worthless.
\par }{\PP \VS{10}Then
the word
of the {\ND{Lord}}
came to
Samuel:
\VS{11}“I regret
that
I have made
Saul
king,
for
he has turned
away
from me
and has not
done what
I told
him to do.” Samuel
became angry
and he cried
out to
the {\ND{Lord}}
all
that night.
\par }{\PP \VS{12}Then Samuel
got up early
to meet
Saul
the next morning.
But Samuel
was informed,
“Saul
has gone
to Carmel
where
he is setting up
a monument
for himself. Then Samuel left
and went down
to Gilgal.”
\VS{13}When Samuel
came
to
him, Saul
said
to him, “May the
{\ND{Lord}}
bless
you! I have done
what the
{\ND{Lord}}
said.”
\par }{\PP \VS{14}Samuel
replied,
“If that is the case, then what
is this
sound
of sheep
in my ears
and the sound
of cattle
that
I
hear?”
\VS{15}Saul
said,
“They were brought
from the Amalekites;
the army
spared
the best
of the flocks
and cattle
to sacrifice
to the
{\ND{Lord}}
our God.
But everything else we slaughtered.”
\par }{\PP \VS{16}Then Samuel
said
to
Saul,
“Wait a minute! Let me tell
you what
the {\ND{Lord}}
said
to
me last night.”
Saul said
to him, “Tell me.”
\VS{17}Samuel
said,
“Is it not
true that when you were insignificant
in your
own eyes,
you became head
of the tribes
of Israel? The
{\ND{Lord}}
chose
you
as king
over
Israel.
\VS{18}The
{\ND{Lord}}
sent
you on a campaign
saying,
‘Go
and exterminate
those sinful
Amalekites! Fight
against them until
you have destroyed them.’
\VS{19}Why
haven’t
you obeyed
the {\ND{Lord}}? Instead you have greedily
rushed upon the plunder! You have done
what is wrong
in the
{\ND{Lord}}’s
estimation.”
\par }{\PP \VS{20}Then Saul
said to
Samuel,
“But I have
obeyed
the {\ND{Lord}}! I went
on
the campaign
the {\ND{Lord}}
sent
me on. I brought
back King
Agag
of the
Amalekites
after exterminating
the Amalekites.
\VS{21}But
the army
took
from the plunder
some of the sheep
and cattle
– the best
of what was to be slaughtered
– to sacrifice
to the
{\ND{Lord}}
your God
in Gilgal.”
\par }{\PP \VS{22}Then Samuel
said,
\par }{\Q “Does the
{\ND{Lord}}
take pleasure
in burnt offerings
and sacrifices
\par }{\Q as much as he does in obedience?

\par }{\Q Certainly,
obedience
is better than sacrifice;
\par }{\Q paying attention
is better
than the fat
of rams.
\par }{\Q \VS{23}For
rebellion
is like the sin
of divination,
\par }{\Q and presumption
is like the evil
of idolatry.
\par }{\Q Because
you have rejected
the word
of the {\ND{Lord}},
\par }{\Q he has rejected
you as king.”
\par }{\PP \VS{24}Then Saul
said to
Samuel,
“I have sinned,
for
I have
disobeyed
what the
{\ND{Lord}}
commanded
and what
you said as well. For
I was afraid
of the
army, and I followed their wishes.
\VS{25}Now
please
forgive
my sin! Go back
with
me so I can worship
the {\ND{Lord}}.”
\par }{\PP \VS{26}Samuel
said
to
Saul,
“I will not
go back
with
you, for
you have rejected
the
word
of the {\ND{Lord}}, and the
{\ND{Lord}}
has rejected
you from being king
over
Israel!”
\par }{\PP \VS{27}When Samuel
turned
to leave,
Saul grabbed
the edge
of his robe
and it tore.
\VS{28}Samuel
said
to him,
“The
{\ND{Lord}}
has torn
the
kingdom
of Israel
from you this day
and has given
it to one
of your colleagues
who is better
than you!
\VS{29}The Preeminent One
of Israel
does not
go back on his word
or
change
his mind, for
he is
not
a human
being who changes
his mind.”
\VS{30}Saul again replied,
“I have sinned.
But
please
honor
me before
the elders
of my people
and before
Israel.
Go back
with
me so I may worship
the {\ND{Lord}}
your God.”
\VS{31}So Samuel
followed
Saul
back,
and Saul
worshiped
the {\ND{Lord}}.
\par }{\SH Samuel Puts Agag to Death
\par }{\PP \VS{32}Then Samuel
said,
“Bring
me King
Agag
of the Amalekites.”
So Agag
came
to
him trembling,
thinking
to himself, “Surely
death
is bitter!”
\VS{33}Samuel
said,
“Just
as your sword
left women
childless,
so
your
mother
will be the most bereaved
among women!” Then Samuel
hacked
Agag
to pieces there in Gilgal
before
the {\ND{Lord}}.
\par }{\PP \VS{34}Then Samuel
went
to Ramah,
while Saul
went up
to
his home
in Gibeah
of Saul.
\VS{35}Until
the day
he died
Samuel
did not
see
Saul
again.
Samuel
did, however, mourn
for
Saul,
but the
{\ND{Lord}}
regretted
that
he had made
Saul
king over
Israel.

\par }\Chap{16}{\PP \VerseOne{1}The
{\ND{Lord}}
said
to
Samuel,
“How long
do you
intend to
mourn
for Saul? I
have rejected
him as king
over
Israel.
Fill
your horn
with olive oil
and go! I am sending
you to
Jesse
in Bethlehem,
for
I have selected
a king
for myself from among his sons.”
\par }{\PP \VS{2}Samuel
replied,
“How
can I go? Saul
will hear
about it and kill
me!” But the
{\ND{Lord}}
said,
“Take
a heifer
with you and say,
‘I have come
to sacrifice
to the
{\ND{Lord}}.’
\VS{3}Then invite
Jesse
to the sacrifice,
and I
will show you what
you should do.
You will anoint
for me the one I point out to you.”
\par }{\PP \VS{4}Samuel
did
what
the {\ND{Lord}}
told
him. When he arrived
in Bethlehem,
the elders
of the city
were afraid
to meet
him. They said,
“Do you come
in peace?”
\VS{5}He replied,
“Yes, in peace.
I have come
to sacrifice
to the
{\ND{Lord}}. Consecrate
yourselves and come
with
me to the sacrifice.”
So he consecrated
Jesse
and his sons
and invited
them to the sacrifice.
\par }{\PP \VS{6}When
they arrived,
Samuel noticed
Eliab
and said
to himself, “Surely,
here before
the {\ND{Lord}}
stands his chosen king!”
\VS{7}But the
{\ND{Lord}}
said
to
Samuel,
“Don’t
be impressed
by
his appearance
or his height,
for
I have rejected
him. God does not
view
things the way men
do. People
look on
the outward appearance,
but the
{\ND{Lord}}
looks at
the heart.”
\par }{\PP \VS{8}Then Jesse
called
Abinadab
and presented
him to Samuel.
But Samuel said,
“The
{\ND{Lord}}
has not
chosen
this
one, either.”
\VS{9}Then
Jesse
presented
Shammah.
But Samuel said,
“The
{\ND{Lord}}
has not
chosen
this
one either.”
\VS{10}Jesse
presented
seven
of his sons
to Samuel.
But Samuel
said
to
Jesse,
“The
{\ND{Lord}}
has not
chosen
any of these.”
\VS{11}Then Samuel
said
to
Jesse,
“Is that all of the young men?” Jesse replied,
“There
is still
the youngest
one, but he’s
taking care
of the flock.”
Samuel
said
to
Jesse,
“Send
and get
him,
for
we cannot
turn
our attention to other things until
he comes
here.”
\par }{\PP \VS{12}So Jesse had him brought
in. Now he was
ruddy,
with
attractive
eyes
and a handsome
appearance.
The
{\ND{Lord}}
said,
“Go
and anoint
him. This
is the one!”
\VS{13}So
Samuel
took
the horn
full of olive oil
and anointed
him in the
presence
of his brothers.
The Spirit
of the {\ND{Lord}}
rushed
upon David
from that day
onward.
Then Samuel
got
up and went
to Ramah.
\par }{\SH David Appears before Saul
\par }{\PP \VS{14}Now the Spirit
of the {\ND{Lord}}
had turned
away
from Saul,
and an evil
spirit
from the
{\ND{Lord}}
tormented him.
\VS{15}Then Saul’s
servants
said to him,
“Look,
an evil
spirit
from God
is tormenting you!”
\VS{16}Let
our lord
instruct his servants
who are here before
you to look
for a man
who knows
how to play
the lyre.
Then whenever
the evil
spirit
from God
comes upon you, he can play
the lyre
and you will feel better.”
\VS{17}So Saul
said
to
his servants,
“Find
me
a man
who plays
well
and bring
him to me.”
\VS{18}One
of his attendants
replied, “I have seen
a son
of Jesse
in Bethlehem
who knows
how to play
the lyre. He is a brave warrior
and is articulate
and handsome,
for the
{\ND{Lord}}
is with him.”
\par }{\PP \VS{19}So Saul
sent
messengers
to
Jesse
and said,
“Send
me
your son
David,
who
is out with the sheep.
\VS{20}So Jesse
took
a donkey
loaded with bread,
a container
of wine,
and a young
goat
and sent
them to
Saul
with his
son
David.
\VS{21}David
came
to
Saul
and stood
before
him. Saul liked
him a great
deal, and he became
his armor
bearer.
\VS{22}Then Saul
sent
word to
Jesse
saying,
“Let
David
be my servant,
for
I really like him.”
\par }{\PP \VS{23}So whenever
the spirit
from God
would come upon
Saul,
David
would take
his lyre
and play
it. This would bring relief
to Saul
and make him feel
better. Then the evil
spirit
would leave him alone.

\par }\Chap{17}{\PP \VerseOne{1}The Philistines
gathered
their troops
for battle.
They assembled
at Socoh
in Judah.
They camped
in Ephes Dammim,
between
Socoh
and Azekah.
\VS{2}Saul
and the Israelite
army assembled
and camped
in the valley
of Elah,
where they arranged
their battle lines
to fight against
the Philistines.
\VS{3}The Philistines
were standing
on one
hill,
and the Israelites
on another
hill,
with the valley
between them.
\par }{\PP \VS{4}Then a champion
came out
from the camp
of the Philistines.
His name
was Goliath;
he was from Gath.
He was close to seven feet
tall.
\VS{5}He had a bronze
helmet
on
his head
and was wearing
scale
body armor.
The weight
of his bronze
body armor
was five
thousand
shekels.
\VS{6}He had bronze
shin guards
on
his legs,
and a bronze
javelin
was slung over his shoulders.
\VS{7}The shaft
of his spear
was like a weaver’s
beam,
and the iron
point
of his spear
weighed six
hundred
shekels.
His shield
bearer
was walking
before him.
\par }{\PP \VS{8}Goliath stood
and called
to
Israel’s
troops, “Why
do you come out
to prepare
for battle? Am
I not
the Philistine,
and are you
not the servants
of Saul? Choose
for yourselves a man
so he may come down
to me!
\VS{9}If
he is able
to fight
with
me and strike
me down, we will become
your servants.
But if
I
prevail
against him and strike
him down, you will become
our servants
and will serve us.”
\VS{10}Then the Philistine
said,
“I
defy
Israel’s
troops
this
day! Give
me a man
so we can fight
each other!”
\VS{11}When Saul
and all
the Israelites
heard
these
words
of the Philistine,
they were
upset
and very
afraid.
\VS{12}\par }{\PP Now David
was the son
of this
Ephrathite
named
Jesse
from Bethlehem
in Judah.
He had eight
sons,
and in Saul’s
days
he was old
and well advanced in years.
\VS{13}Jesse’s
three
oldest sons
had followed
Saul
to war.
The names
of the three
sons
who
went
to war
were Eliab,
his firstborn,
Abinadab,
the second
oldest, and Shammah,
the third
oldest.
\VS{14}Now David
was the youngest.
While the three
oldest
sons followed
Saul,
\VS{15}David
was going
back
and forth from Saul
in order to care
for his father’s
sheep
in Bethlehem.
\par }{\PP \VS{16}Meanwhile
for forty
days
the Philistine
approached every morning
and evening
and took
his position.
\VS{17}Jesse
said
to his son
David,
“Take
your
brothers
this
ephah
of roasted
grain and these
ten
loaves of bread;
go quickly
to the camp
to your brothers.
\VS{18}Also take
these
ten
portions
of cheese
to their commanding officer.
Find out
how your brothers
are doing
and bring back their pledge
that they received the goods.
\VS{19}They are
with Saul
and the whole
Israelite
army in the valley
of Elah,
fighting
with
the Philistines.”
\par }{\PP \VS{20}So David
got up early
in the morning
and entrusted
the
flock
to someone else who would watch
over
it.
After loading up,
he went
just
as Jesse
had instructed
him. He arrived
at the camp
as the army
was going out
to
the battle lines
shouting
its battle
cry.
\VS{21}Israel
and the Philistines
drew up their battle lines opposite one another.
\VS{22}After David
had entrusted
his
cargo
to the care
of the supply officer, he ran
to the battlefront.
When he arrived,
he asked
his brothers
how they were doing.
\VS{23}As he
was speaking
with
them, the champion
named
Goliath,
the Philistine
from Gath,
was coming up
from the battle lines
of the Philistines.
He spoke
the way
he usually did, and David
heard it.
\VS{24}When all
the men
of Israel
saw
this man,
they retreated
from his presence
and were very
afraid.
\par }{\PP \VS{25}The men
of Israel
said,
“Have you seen
this
man
who is coming up? He does so
to defy
Israel.
But the king
will make
the man
who
can strike
him down
very
wealthy! He will give
him his daughter
in marriage, and he will make
his father’s
house
exempt
from tax obligations in Israel.”
\par }{\PP \VS{26}David
asked
the men
who were standing
near him,
“What
will be done
for the man
who
strikes
down this Philistine
and frees
Israel
from
this humiliation? For
who
is this
uncircumcised
Philistine,
that
he defies
the armies
of the living
God?”
\VS{27}The soldiers
told
him what
had been promised,
saying, “This
is what
will be done
for the man
who
can strike him down.”
\par }{\PP \VS{28}When David’s oldest
brother
Eliab
heard
him speaking
to
the men,
he
became
angry
with David
and said,
“Why
have you come down
here? To
whom
did you entrust
those few
sheep
in the desert? I am
familiar
with your pride
and deceit! You have come down
here to watch
the battle!”
\par }{\PP \VS{29}David
replied,
“What
have I done
now? Can’t
I say anything?”
\VS{30}Then he turned
from those
who were
nearby
to someone else
and asked the same question,
but they gave
him the same
answer
as
before.
\VS{31}When
David’s
words
were overheard
and reported
to Saul,
he called for him.
\par }{\PP \VS{32}David
said
to
Saul,
“Don’t
let anyone
be discouraged.
Your servant
will go
and fight
this
Philistine!”
\VS{33}But Saul
replied
to
David,
“You aren’t
able
to go
against
this
Philistine
and fight
him! You’re just
a boy! He has been a warrior
from his youth!”
\par }{\PP \VS{34}David
replied
to
Saul,
“Your servant
has been a shepherd
for his father’s
flock.
Whenever a lion
or bear
would come
and carry off
a sheep
from the flock,
\VS{35}I would go out
after
it, strike
it down, and rescue
the sheep from its mouth.
If it rose
up against
me,
I would grab
it by its jaw,
strike
it, and kill it.
\VS{36}Your servant
has struck
down both
the lion
and the bear.
This
uncircumcised
Philistine
will be
just like one
of them.
For
he has defied
the armies
of the living
God!”
\VS{37}David
went on to say,
“The
{\ND{Lord}}
who
delivered
me from the lion
and the bear
will also deliver
me from the hand
of this
Philistine!” Then Saul
said to
David,
“Go! The
{\ND{Lord}}
will be
with you.”
\par }{\PP \VS{38}Then Saul
clothed
David
with his own fighting attire
and put
a bronze
helmet
on
his head.
He also put
body armor on him.
\VS{39}David
strapped
on his sword
over
his fighting attire
and tried
to walk
around, but
he was not
used
to them. David
said
to
Saul,
“I can’t
walk
in these
things, for
I’m not
used
to them.” So David
removed them.
\VS{40}He took
his staff
in his hand,
picked
out five
smooth
stones
from
the stream,
placed
them in the pouch
of his shepherd’s
bag,
took his sling
in
hand,
and approached
the Philistine.
\VS{41}\par }{\PP The Philistine
kept coming closer
to
David,
with his shield
bearer
walking in front of him.
\VS{42}When
the Philistine
looked
carefully at David,
he despised
him, for
he was
only a ruddy
and
handsome
boy.
\VS{43}The Philistine
said
to
David,
“Am
I a dog,
that
you
are coming
after me with sticks?” Then the Philistine
cursed
David
by his gods.
\VS{44}The Philistine
said
to
David,
“Come
here to me,
so I can give
your flesh
to the birds
of the sky
and the wild animals
of the field!”
\par }{\PP \VS{45}But David
replied
to
the Philistine,
“You
are coming
against me
with sword
and spear
and javelin.
But I am
coming
against you in the name
of the {\ND{Lord}}
of hosts,
the God
of Israel’s
armies,
whom
you have defied!
\VS{46}This very
day
the {\ND{Lord}}
will deliver
you into my hand! I will strike
you down and cut off
your head.
This
day
I will give
the corpses
of the Philistine
army
to the birds
of the sky
and the wild animals
of the land.
Then all
the land
will realize
that
Israel
has a God
\VS{47}and all
this
assembly
will know
that
it is not
by sword
or spear
that
the {\ND{Lord}}
saves! For
the battle
is
the
{\ND{Lord}}’s,
and he will deliver
you into our hand.”
\par }{\PP \VS{48}The Philistine
drew
steadily closer to David to attack
him,
while David
quickly
ran
toward the battle line
to attack
the Philistine.
\VS{49}David
reached
his hand
into
the bag
and took
out a stone.
He slung
it, striking
the
Philistine
on the forehead.
The stone
sank
deeply into
his forehead,
and he fell
down with his face
to the ground.
\VS{50}\par }{\PP David
prevailed
over the Philistine
with just the sling
and the stone.
He struck
down the
Philistine
and killed
him. David
did not
even have a sword
in his hand.
\VS{51}David
ran
and stood
over
the Philistine.
He grabbed
Goliath’s sword,
drew
it from its sheath,
killed
him, and cut off
his head
with it. When the Philistines
saw
their champion
was dead,
they ran away.
\par }{\PP \VS{52}Then the men
of Israel
and Judah
charged forward,
shouting
a battle cry. They chased
the Philistines
to
the valley
and to the very gates
of Ekron.
The Philistine
corpses
lay fallen
along the Shaaraim
road
to
Gath
and Ekron.
\VS{53}When the Israelites
returned
from their hot
pursuit
of the Philistines,
they looted
their camp.
\VS{54}David
took
the
head
of the Philistine
and brought
it to Jerusalem,
and he put Goliath’s
weapons
in his tent.
\VS{55}\par }{\PP Now as Saul
watched
David
going out
to fight
the Philistine,
he asked
Abner,
the general in command
of the army,
“Whose
son
is this
young man,
Abner?” Abner
replied,
“As surely
as you live,
O king,
I don’t know.”
\VS{56}The king
said,
“Find
out whose
son
this
boy is!”
\par }{\PP \VS{57}So when David
returned
from striking
down the Philistine,
Abner
took
him and brought
him before
Saul.
He still had the head
of the Philistine
in his hand.
\VS{58}Saul
said
to him,
“Whose
son
are you,
young man?” David
replied,
“I am the son
of your servant
Jesse
in Bethlehem.”


\par }\Chap{18}{\PP \VerseOne{1}When
David had finished
talking
with Saul,
Jonathan
and David became bound together
in close friendship.
Jonathan
loved
David
as much as he did his own life.
\VS{2}Saul
retained
David on that day
and did not
allow
him to return
to his father’s
house.
\VS{3}Jonathan
made
a covenant
with David,
for he loved
him as much as he did his own life.
\VS{4}Jonathan
took off
the robe
he was wearing
and gave
it to David,
along
with the rest of his gear,
including
his sword,
his bow,
and even
his belt.
\par }{\PP \VS{5}On every
mission on which
Saul
sent
him, David
achieved
success. So
Saul
appointed him over
the men
of war.
This pleased
not only all
the army,
but also
Saul’s
servants.
\par }{\PP \VS{6}When
the men arrived
after David
returned
from striking down
the Philistine,
the women
from all
the cities
of Israel
came out
singing
and dancing
to meet
King
Saul.
They were happy
as they played their tambourines
and three-stringed instruments.
\VS{7}The women
who were playing
the music sang,
\par }{\Q “Saul
has struck
down his thousands,
\par }{\Q but David
his tens of thousands!”
\par }{\PP \VS{8}This
made Saul
very
angry.
The statement
displeased
him and he thought, “They have attributed
to David
tens of thousands,
but to me they have attributed only
thousands.
What
does he lack, except the kingdom?”
\VS{9}So Saul
was keeping an eye on David
from that day
onward.
\par }{\PP \VS{10}The next
day an evil
spirit
from God
rushed
upon Saul
and he prophesied
within
his house.
Now David
was playing
the lyre
that
day.
There was a spear
in Saul’s
hand,
\VS{11}and Saul
threw
the spear,
thinking,
“I’ll nail
David
to the wall!” But David
escaped
from him on two different occasions.
\par }{\PP \VS{12}So Saul
feared
David,
because
the {\ND{Lord}}
was with
him
but had departed
from Saul.
\VS{13}Saul
removed
David from
his presence and made
him a commanding officer.
David led
the army
out to battle and back.
\VS{14}Now
David
achieved
success in all
he did, for the
{\ND{Lord}}
was with him.
\VS{15}When Saul
saw
how
very
successful
he was,
he was afraid of him.
\VS{16}But all
Israel
and Judah
loved
David,
for
he
was the one leading
them out
to battle and back.
\VS{17}\par }{\PP Then Saul
said to
David,
“Here’s
my oldest
daughter,
Merab.
I want to give
her to you in marriage.
Only
be
a
brave
warrior for me and fight
the battles
of the {\ND{Lord}}.” For Saul
thought,
“There’s no
need for me to raise my hand
against him. Let it be
the hand
of the Philistines!”
\par }{\PP \VS{18}David
said
to
Saul,
“Who
am
I? Who
are my relatives or the clan
of my father
in Israel
that
I should
become
the king’s
son-in-law?”
\VS{19}When
the time
came for Merab,
Saul’s
daughter,
to be given to David,
she
instead was given
in marriage
to Adriel,
who was from Meholah.
\par }{\PP \VS{20}Now Michal,
Saul’s
daughter,
loved
David.
When they told
Saul
about this, it pleased him.
\VS{21}Saul
said,
“I will give
her to him so that she may become a snare
to him and the hand
of the Philistines
may be against him.” So Saul
said to
David,
“Today
is the second
time for you to become my son-in-law.”
\par }{\PP \VS{22}Then Saul
instructed
his servants,
“Tell
David
secretly,
‘The king
is
pleased
with you, and all
his servants
like
you. So now
become the king’s
son-in-law.”
\VS{23}So Saul’s
servants
spoke
these
words
privately
to David.
David
replied,
“Is becoming the king’s
son-in-law
something insignificant
to you? I’m
just a poor
and lightly-esteemed
man!”
\par }{\PP \VS{24}When Saul’s
servants
reported
what David
had said,
\VS{25}Saul
replied,
“Here is what
you should say
to David: ‘There is nothing
that the king
wants
as a price
for
the bride
except a hundred
Philistine
foreskins,
so that he can be avenged
of his enemies.’ ”
(Now Saul
was thinking
that he could kill David
by the hand
of the Philistines.)
\par }{\PP \VS{26}So his servants
told
David
these
things
and David
agreed
to become the king’s
son-in-law.
Now the specified time
had not
yet expired
\VS{27}when David,
along
with his men,
went out
and struck down
two hundred
Philistine
men.
David
brought
their foreskins
and presented all of them to the king
so he could become the king’s
son-in-law.
Saul
then gave
him his daughter
Michal
in marriage.
\par }{\PP \VS{28}When Saul
realized
that
the {\ND{Lord}}
was with
David
and that his
daughter
Michal
loved David,
\VS{29}Saul
became even more
afraid
of him.
Saul
continued
to be at odds
with David
from then on.
\VS{30}Then
the leaders
of the Philistines
would march out,
and as often
as they did
so,
David
achieved
more success than all
of Saul’s
servants.
His name
was
held in high esteem.

\par }\Chap{19}{\PP \VerseOne{1}Then
Saul
told
his son
Jonathan
and all
his servants
to kill
David.
But Saul’s
son
Jonathan
liked
David
very much.
\VS{2}So Jonathan
told
David,
“My father
Saul
is trying
to kill
you. So
be careful
tomorrow
morning.
Find
a hiding place
and stay
in seclusion.
\VS{3}I
will go out
and stand
beside
my father
in the field
where
you
are. I
will speak
about you to
my father.
When I find
out what the problem is, I will let you know.”
\par }{\PP \VS{4}So Jonathan
spoke
on David’s
behalf
to
his father
Saul.
He said
to him,
“The king
should not
sin
against his servant
David,
for
he has not
sinned
against you. On the contrary, his actions
have been very
beneficial for you.
\VS{5}He risked
his life when he struck
down the
Philistine
and the
{\ND{Lord}}
gave all
Israel
a great
victory.
When you saw
it, you were happy.
So why
would you sin
against innocent
blood
by putting
David
to death
for no reason?”
\par }{\PP \VS{6}Saul
accepted
Jonathan’s
advice
and took an oath,
“As surely as the
{\ND{Lord}}
lives,
he will not
be put to death.”
\VS{7}Then Jonathan
called
David
and told
him
all
these
things.
Jonathan
brought
David
to
Saul,
and he served
him as he had done formerly.
\par }{\PP \VS{8}Now once again
there was
war.
So David
went out
to fight
the Philistines.
He defeated
them thoroughly
and they ran away from him.
\VS{9}Then
an evil
spirit
from the
{\ND{Lord}}
came
upon Saul.
He was
sitting
in his house
with his spear
in his hand,
while David
was playing
the lyre.
\VS{10}Saul
tried
to nail
David
to the wall
with the spear,
but
he escaped from Saul’s
presence
and the spear
drove into the wall.
David
escaped
quickly that night.
\par }{\PP \VS{11}Saul
sent
messengers
to
David’s
house
to guard
it and to kill
him in the morning.
Then David’s
wife
Michal
told
him, “If
you do not
save
yourself
tonight,
tomorrow
you
will be dead!”
\VS{12}So Michal
lowered
David
through
the window,
and he ran away
and escaped.
\par }{\PP \VS{13}Then Michal
took
a household idol
and put
it on
the bed.
She put
a quilt
made of goat’s
hair over its head
and then covered
the idol with a garment.
\VS{14}When
Saul
sent
messengers
to arrest
David,
she said,
“He’s sick.”
\par }{\PP \VS{15}Then Saul
sent
the messengers
back to see
David,
saying,
“Bring
him up
to me on his bed
so I can kill him.”
\VS{16}When the messengers
came,
they found
only the idol
on
the bed
and the quilt
made of goat’s
hair at its head.
\par }{\PP \VS{17}Saul
said
to
Michal,
“Why
have you deceived
me this way
by sending
my enemy
away? Now he has escaped!” Michal
replied
to
Saul,
“He said
to me, ‘Help me get
away or else
I will kill you!’ ”
\par }{\PP \VS{18}Now David
had run
away and escaped.
He went
to
Samuel
in Ramah
and told
him everything
that
Saul
had done
to him. Then
he and Samuel
went and stayed
at Naioth.
\VS{19}It was reported
to Saul
saying,
“David
is at Naioth
in Ramah.”
\VS{20}So
Saul
sent
messengers
to capture
David.
When they saw
a company
of prophets
prophesying
with Samuel
standing
there as their leader, the spirit
of God
came upon Saul’s
messengers,
and they also
prophesied.
\VS{21}When it was reported
to Saul,
he sent
more
messengers,
but they prophesied
too.
So Saul
sent
messengers
a third
time, but they
also
prophesied.
\VS{22}Finally
Saul himself
went
to Ramah.
When he arrived
at the large
cistern
that
is in Secu,
he asked,
“Where
are Samuel
and David?” They said,
“At Naioth
in Ramah.”
\par }{\PP \VS{23}So Saul went
to
Naioth
in Ramah.
The Spirit
of God
came upon
him as well,
and he walked
along prophesying
until
he came
to Naioth
in Ramah.
\VS{24}He even
stripped
off his clothes
and prophesied
before
Samuel.
He lay
there naked
all
that day
and night.
(For that reason
it is asked,
“Is Saul
also
among the prophets?”)

\par }\Chap{20}{\PP \VerseOne{1}David
fled
from Naioth
in Ramah.
He came
to Jonathan
and asked, “What
have I done? What
is my offense? How
have I sinned
before
your father? For
he is seeking
my life!”
\par }{\PP \VS{2}Jonathan said
to him, “By no means
are you going to die! My father
does
nothing
large
or
small
without
making
me aware
of it. Why
would my father
hide
this
matter
from me? It just won’t happen!”
\par }{\PP \VS{3}Taking an oath,
David
again
said,
“Your father
is very much aware of the fact
that
I have found
favor
with you, and he has thought, ‘Don’t
let Jonathan
know
about this,
or
he will be upset.’
But as surely as
the {\ND{Lord}}
lives
and you
live,
there is about one step
between
me and death!”
\VS{4}Jonathan
replied
to
David,
“Tell
me what
I
can do for you.”
\par }{\PP \VS{5}David
said
to
Jonathan,
“Tomorrow
is the new moon,
and I
am certainly expected
to
join
the king
for a meal.
You must send
me away so I can hide
in the field
until
the third
evening from now.
\VS{6}If
your father
happens
to miss me,
you should say,
‘David
urgently
requested me to let
him go to his city
Bethlehem,
for
there is an annual
sacrifice
there
for his entire
family.’
\VS{7}If
he should
then say,
‘That’s fine,’ then your servant
is safe.
But if
he becomes very angry,
be assured
that
he has decided
to harm
me.
\VS{8}You must be loyal
to your servant,
for
you have made
a covenant
with
your servant
in the
{\ND{Lord}}’s
name. If
I am guilty,
you yourself kill
me! Why
bother taking
me to your father?”
\par }{\PP \VS{9}Jonathan
said,
“Far be it
from you to suggest this! If
I were at all aware
that
my father
had decided
to harm
you, wouldn’t
I tell you about it?”
\VS{10}David
said
to
Jonathan,
“Who
will tell
me if your father
answers
you harshly?”
\VS{11}Jonathan
said
to
David,
“Come on. Let’s
go out
to the field.”
\par }{\PP When
the two
of them had gone out
into the field,
\VS{12}Jonathan
said
to
David,
“The
{\ND{Lord}}
God
of Israel
is my witness. I will feel out
my father
about this time
the day after tomorrow.
If he is favorably
inclined toward
David,
will I not
then
send
word to
you and let you know?
\VS{13}But if
my father
intends
to
do you harm,
may the
{\ND{Lord}}
do
all this
and more
to Jonathan,
if I don’t
let
you know
and send
word to you so you can go
safely
on
your way. May the
{\ND{Lord}}
be with
you, as
he was
with
my father.
\VS{14}While I
am still
alive,
extend
to me
the loyalty
of the {\ND{Lord}}, or
else I will die!
\VS{15}Don’t
ever
cut off
your loyalty
to my family,
not
even when the
{\ND{Lord}}
has cut off
every one
of David’s
enemies
from the face
of the earth
\VS{16}and called David’s
enemies
to account.”
So Jonathan
made a covenant with
the house
of David.
\VS{17}Jonathan
once again
took an oath
with David,
because he loved
him. In fact
Jonathan loved
him as much as he did his own life.
\VS{18}Jonathan
said
to him, “Tomorrow
is the new moon,
and you will be missed,
for your seat will be empty.
\VS{19}On the third
day you should go down
quickly
and come
to
the place
where
you hid
yourself the day
this all started.
Stay
near
the stone
Ezel.
\VS{20}I
will shoot three
arrows
near
it, as though I were shooting
at a target.
\VS{21}When I
send
a boy
after them, I will say,
“Go
and find
the
arrows.”
If
I say
to the boy,
‘Look,
the arrows
are on this side
of you; get
them,’ then come
back. For
as surely
as the
{\ND{Lord}}
lives, you will be safe
and there will no
problem.
\VS{22}But if
I
say
to the boy,
“Look,
the arrows
are on
the other side
of you,’ get away.
For
in that case the
{\ND{Lord}}
has sent you away.
\VS{23}With regard to the matter
that
you and I discussed,
the {\ND{Lord}}
is the witness between
us forever!”
\par }{\PP \VS{24}So David
hid
in the field.
When
the new moon
came,
the king
sat
down to eat his meal.
\VS{25}The king
sat down
in
his usual
place
by
the wall,
with Jonathan
opposite him and Abner
at his side.
But David’s
place
was vacant.
\VS{26}However, Saul
said
nothing
about it that
day,
for
he thought, “Something has happened
to make him ceremonially unclean.
Yes,
he must be unclean.”
\VS{27}But
the next
morning, the second
day of the new moon,
David’s
place
was still vacant.
So
Saul
said
to
his son
Jonathan,
“Why
has Jesse’s
son
not
come
to the meal
yesterday
or today?”
\par }{\PP \VS{28}Jonathan
replied
to Saul,
“David
urgently
requested
that he be allowed
to go to Bethlehem.
\VS{29}He
said,
‘Permit
me
to go, for
we are having a family
sacrifice
in the city,
and my brother
urged
me to be there. So now,
if
I have found
favor
with you, let
me go to see
my brothers.’
For that reason
he has not
come
to
the king’s
table.”
\par }{\PP \VS{30}Saul
became angry
with Jonathan
and said
to him, “You stupid
traitor! Don’t
I realize
that
to your own disgrace
and to the disgrace
of your mother’s
nakedness
you
have chosen
this son
of Jesse?
\VS{31}For
as long
as this
son
of Jesse
is alive
on
the earth,
you
and your kingdom
will not
be established.
Now,
send
some men
and bring
him to
me. For
he is
as good as dead!”
\par }{\PP \VS{32}Jonathan
responded
to his father
Saul,
“Why
should he be put to death? What
has he done?”
\VS{33}Then
Saul
threw
his spear
at Jonathan in order to strike
him down. So Jonathan
was convinced
that
his father
had decided to kill
David.
\VS{34}Jonathan
got
up from the table
enraged.
He did not
eat
any food
on that second
day
of the new moon,
for
he was upset
that
his father
had
humiliated
David.
\par }{\PP \VS{35}The next
morning
Jonathan,
along with
a young
servant,
went out
to the field
to meet
David.
\VS{36}He said
to his servant,
“Run,
find
the arrows
that
I am
about to shoot.”
As the servant
ran,
Jonathan shot
the arrow
beyond him.
\VS{37}When the servant
came
to
the place
where
Jonathan
had shot
the arrow,
Jonathan
called out
to
the servant,
“Isn’t
the arrow
further beyond you?”
\VS{38}Jonathan
called
out to the servant,
“Hurry! Go faster! Don’t
delay!” Jonathan’s
servant
retrieved
the arrow
and came
back to
his master.
\VS{39}(Now the servant
did not
understand
any
of this.
Only
Jonathan
and David
knew
what was going on.)
\VS{40}Then Jonathan
gave
his equipment
to
the servant
who
was with him. He said
to him, “Go,
take these things back
to the city.”
\par }{\PP \VS{41}When
the servant
had left, David
got up
from beside
the mound,
knelt
with his face
to the ground,
and bowed
three
times.
Then they kissed
each
other
and they both wept,
especially
David.
\VS{42}Jonathan
said
to David,
“Go
in peace,
for the two
of us
have
sworn
together in the name
of the {\ND{Lord}}
saying,
‘The
{\ND{Lord}}
will be
between
me and you and between
my descendants
and your descendants
forever.’ ”

\par }{\PP Then David got up and left, while Jonathan went back to the city.

\par }\Chap{21}{\PP \VerseOne{1}David
went
to
Ahimelech
the priest
in Nob.
Ahimelech
was shaking with fear
when he met
David,
and said
to him, “Why
are you
by yourself
with no
one accompanying you?”
\VS{2}David
replied
to Ahimelech
the priest,
“The king
instructed
me to do something,
but he said
to
me, ‘Don’t
let anyone
know
the reason
I am
sending
you or the instructions
I have given you.’ I have told
my soldiers
to
wait at a certain
place.
\VS{3}Now
what
do you have at your disposal? Give
me five
loaves of bread,
or
whatever can be found.”
\par }{\PP \VS{4}The priest
replied
to David,
“I don’t
have any ordinary
bread
at
my disposal. Only
holy
bread
is
available, and then only if
your soldiers
have abstained
from sexual relations with women.”
\VS{5}David
said
to the priest,
“Certainly
women
have been kept away
from us, just as on previous occasions
when
I have set out.
The soldiers’
equipment
is holy,
even on an ordinary
journey.
How much more so
will they be holy
today,
along with their equipment!”
\par }{\PP \VS{6}So
the priest
gave
him holy
bread, for
there
was no
bread
there
other than
the bread
of the Presence.
It had been removed
from before
the {\ND{Lord}}
in order to replace it
with hot
bread
on the day
it had been taken away.
\VS{7}(One
of Saul’s
servants
was there
that day,
detained
before
the
{\ND{Lord}}. His
name
was Doeg
the Edomite,
who
was in charge of Saul’s
shepherds.)
\VS{8}David
said
to Ahimelech,
“Is
there no
sword
or
spear
here
at your disposal? I don’t have my own sword
or
equipment
in hand
due to the urgency
of the king’s
instructions.”
\par }{\SH David Goes to Gath
\par }{\PP \VS{9}The priest
replied,
“The sword
of Goliath
the Philistine,
whom
you struck
down in the valley
of Elah,
is
wrapped
in a garment
behind
the ephod.
If
you wish, take
it for yourself. Other
than that,
there’s
nothing
here.”
David
said,
“There’s nothing
like
it! Give it to me!”
\VS{10}So on that day
David
arose
and fled
from Saul.
He went
to
King
Achish
of Gath.
\VS{11}The servants
of Achish
said
to him,
“Isn’t
this
David,
the king
of the land? Isn’t
he the one that they sing
about when they dance,
saying,
\par }{\Q ‘Saul
struck
down his thousands,
\par }{\Q But David
his tens of thousands’?”
\par }{\PP \VS{12}David
thought
about what
they said and was very
afraid
of King
Achish
of Gath.
\VS{13}He altered
his behavior
in their presence.
Since he was in their power,
he pretended to be insane,
making marks
on
the doors
of the gate
and letting his saliva
run down
his beard.
\par }{\PP \VS{14}Achish
said
to
his servants,
“Look
at this madman! Why
did you bring
him to me?
\VS{15}Do I have
a shortage
of fools,
that
you have brought
me this
man to display his insanity
in front of
me? Should this
man enter
my house?”

\par }\Chap{22}{\PP \VerseOne{1}So
David
left
there
and escaped
to
the cave
of Adullam.
When
his brothers
and the rest of his father’s
family
learned about it, they went down
there
to him.
\VS{2}All
those
who
were in trouble or owed
someone
money or were discontented
gathered around him, and he became
their leader.
He had about four
hundred
men
with him.
\par }{\PP \VS{3}Then David
went
from there
to Mizpah
in Moab,
where he said
to
the king
of Moab,
“Please
let my father
and mother
stay with
you until
I know
what
God
is going to do for me.”
\VS{4}So he had them stay
with
the king
of Moab;
they stayed
with
him the whole
time
that David
was in the stronghold.
\VS{5}Then Gad
the prophet
said to
David,
“Don’t
stay
in the stronghold.
Go
to the land
of Judah.”
So David
left
and went
to the forest
of Hereth.
\par }{\SH Saul Executes the Priests
\par }{\PP \VS{6}But Saul
found out the whereabouts
of David
and the men
who
were with
him. Now Saul
was sitting
at Gibeah
under
the tamarisk
tree at an elevated location
with his spear
in hand
and all
his servants
stationed
around him.
\VS{7}Saul
said
to his servants
who were stationed
around
him, “Listen up,
you Benjaminites! Is Jesse’s
son
giving
fields
and vineyards
to all
of you? Or is he making
all
of you commanders
and officers?
\VS{8}For
all
of you have conspired
against
me! No
one informs
me when my own son
makes an agreement
with
this son
of Jesse! Not
one of you feels sorry
for me or informs
me that
my own son
has commissioned
my own servant
to hide in
ambush
against me, as is the case today!”
\par }{\PP \VS{9}But
Doeg
the Edomite,
who had stationed
himself with the servants
of Saul,
replied,
“I saw
this son
of Jesse
come
to
Ahimelech
son
of Ahitub
at Nob.
\VS{10}He inquired
of the {\ND{Lord}}
for him and gave him provisions.
He also gave
him the sword
of Goliath
the Philistine.”
\par }{\PP \VS{11}Then
the king
arranged
for a meeting with the priest
Ahimelech
son
of Ahitub
and all
the priests
of his father’s
house
who
were at Nob.
They all
came
to
the king.
\VS{12}Then Saul
said, “Listen,
son
of Ahitub.”
He replied,
“Here I am,
my lord.”
\VS{13}Saul
said
to
him, “Why
have you conspired
against
me, you
and this
son
of Jesse? You gave
him bread
and a sword
and inquired
of God
on his behalf, so that he opposes
me and waits in ambush,
as is the case today!”
\par }{\PP \VS{14}Ahimelech
replied
to the king,
“Who
among all
your servants
is faithful
like David? He is the king’s
son-in-law,
the leader of your bodyguard,
and honored
in your house!
\VS{15}Was it just today
that I began
to inquire
of God
on his behalf? Far be it
from me! The king
should not
accuse
his servant
or any
of my father’s
house.
For
your servant
is not
aware
of all
this
– not in whole or
in part!”
\par }{\PP \VS{16}But the king
said,
“You will surely
die,
Ahimelech,
you
and all
your father’s
house!
\VS{17}Then the king
said
to the messengers
who were stationed
beside him, “Turn
and kill
the priests
of the {\ND{Lord}}, for
they too
have sided
with
David! They knew
he was fleeing,
but they did not
inform
me.” But the king’s
servants
refused
to harm
the priests
of the {\ND{Lord}}.
\par }{\PP \VS{18}Then the king
said
to Doeg,
“You
turn
and strike
down the priests!” So
Doeg
the Edomite
turned
and struck down
the priests.
He killed
on that day
eighty-five
men
who wore
the linen
ephod.
\VS{19}As for Nob,
the city
of the priests,
he struck down
with the sword
men
and women,
children
and infants,
oxen,
donkeys,
and sheep
– all with the sword.
\par }{\PP \VS{20}But one
of the sons
of Ahimelech
son
of Ahitub
escaped
and fled
to David.
His name
was Abiathar.
\VS{21}Abiathar
told
David
that
Saul
had killed
the priests
of the {\ND{Lord}}.
\VS{22}Then David
said
to Abiathar,
“I knew
that day
when
Doeg
the Edomite
was there
that
he would certainly
tell
Saul! I am
guilty
of all
the deaths
in your father’s
house!
\VS{23}Stay
with
me. Don’t
be afraid! Whoever
seeks
my life
is seeking
your life
as well. You
are secure
with me.”

\par }\Chap{23}{\PP \VerseOne{1}They told
David,
“The Philistines
are fighting
in Keilah
and are
looting
the threshing floors.”
\VS{2}So David
asked
the {\ND{Lord}}, “Should I go
and strike
down these
Philistines?” The
{\ND{Lord}}
said
to
David,
“Go,
strike
down the Philistines
and deliver
Keilah.”
\par }{\PP \VS{3}But David’s
men
said
to him,
“We are afraid
while
we
are still here
in Judah! What will it be like
if
we go
to Keilah
against the armies
of the Philistines?”
\VS{4}So David
asked
the {\ND{Lord}}
once again.
But
again
the {\ND{Lord}}
replied,
“Arise,
go down
to Keilah,
for
I
will give
the Philistines
into your hand.”
\par }{\PP \VS{5}So
David
and his men
went to Keilah
and fought
the Philistines.
He took away
their cattle
and thoroughly
defeated
them. David
delivered
the inhabitants
of Keilah.
\par }{\SH David Eludes Saul Again
\par }{\PP \VS{6}Now when
Abiathar
son
of Ahimelech
had fled
to David
at Keilah,
he had brought
with him an ephod.
\VS{7}When Saul
was told
that
David
had come
to Keilah,
Saul
said,
“God
has delivered him
into
my hand,
for
he has boxed
himself into a corner by entering
a city
with two barred
gates.”
\VS{8}So
Saul
mustered
all
his army
to go
down
to Keilah
and besiege
David
and his men.
\par }{\PP \VS{9}When David
realized
that
Saul
was planning
to harm
him, he told
Abiathar
the priest,
“Bring
the ephod!”
\VS{10}Then David
said,
“O
{\ND{Lord}}
God
of Israel,
your servant
has clearly
heard
that
Saul
is planning to come
to
Keilah
to destroy
the city
because of me.
\VS{11}Will the leaders
of Keilah
deliver
me into his hand? Will Saul
come down
as
your servant
has heard? O
{\ND{Lord}}
God
of Israel,
please
inform
your servant!”
\par }{\PP Then the
{\ND{Lord}}
said,
“He will come down.”
\VS{12}David
asked,
“Will the leaders
of Keilah
deliver
me and my men
into Saul’s
hand?” The
{\ND{Lord}}
said,
“They will deliver you over.”
\par }{\PP \VS{13}So
David
and his men,
who numbered about six
hundred,
set out
and left Keilah;
they moved
around from one place to another. When told
that
David
had
escaped
from Keilah,
Saul
called a halt to his expedition.
\VS{14}David
stayed
in the strongholds
that were in the desert
and in
the hill country
of the desert
of Ziph.
Saul
looked
for him all
the time,
but God
did not
deliver
David into his hand.
\VS{15}David
realized
that
Saul
had come out
to seek
his life;
at that time David
was in Horesh
in the desert
of Ziph.
\par }{\PP \VS{16}Then
Jonathan
son
of Saul
left and went
to
David
at Horesh.
He encouraged him
through
God.
\VS{17}He said
to him,
“Don’t
be afraid! For
the hand
of my father
Saul
cannot
find
you.
You will rule
over
Israel,
and I
will be
your second
in command. Even
my father
Saul
realizes
this.”
\VS{18}When
the two
of them had made
a covenant
before
the {\ND{Lord}}, David
stayed
on at Horesh,
but Jonathan
went
to his house.
\par }{\PP \VS{19}Then the Ziphites
went up
to
Saul
at Gibeah
and said,
“Isn’t
David
hiding
among us in the strongholds
at Horesh
on the hill
of Hakilah,
south
of Jeshimon?
\VS{20}Now
at your own discretion,
O king,
come down.
Delivering him
into
the king’s
hand will be our responsibility.”
\par }{\PP \VS{21}Saul
replied,
“May you
be blessed
by the
{\ND{Lord}}, for
you have had compassion
on me.
\VS{22}Go
and make further
arrangements.
Determine
precisely
where
he is
and who
has seen
him there,
for
I am told
that he
is extremely cunning.
\VS{23}Locate
precisely
all
the places
where
he hides
and return
to
me with
dependable
information. Then I will go
with
you. If
he is
in the land,
I will find
him
among all
the thousands
of Judah.”
\par }{\PP \VS{24}So they left
and went
to Ziph
ahead
of Saul.
Now David
and his men
were in the desert
of Maon,
in the Arabah
to
the south
of Jeshimon.
\VS{25}Saul
and his men
went to look
for him. But David
was informed
and went down
to the rock
and stayed
in the desert
of Maon.
When Saul
heard
about it, he pursued
David
in the desert
of Maon.
\VS{26}Saul
went
on one side
of the mountain,
while
David
and his men
went on the other side
of the mountain.
David
was hurrying
to get away
from Saul,
but Saul
and his men
were surrounding
David
and his men
so they could capture them.
\VS{27}But a messenger
came
to
Saul
saying,
“Come
quickly,
for
the Philistines
have raided
the land!”
\par }{\PP \VS{28}So Saul
stopped
pursuing
David
and went
to confront
the Philistines.
Therefore
that place
is called
Sela Hammahlekoth.
\VS{29}Then David
went up
from there
and stayed
in the strongholds
of En Gedi.

\par }\Chap{24}{\PP \VerseOne{1}When
Saul
returned
from pursuing
the Philistines,
they told
him, “Look,
David
is in the desert
of En Gedi.”
\VS{2}So Saul
took
three
thousand
select
men
from all
Israel
and went
to find
David
and his men
in the region
of the rocks
of the mountain goats.
\VS{3}He came
to
the sheepfolds
by
the road,
where there
was a cave.
Saul
went
into it to relieve
himself.
\par }{\PP Now David
and his men
were sitting
in the recesses
of the cave.
\VS{4}David’s
men
said
to him,
“This is
the day
about which
the {\ND{Lord}}
said
to
you, ‘I
will give
your enemy
into your hand,
and you can do
to him whatever
seems
appropriate
to you.’ ” So David
got
up and quietly
cut off
an edge
of Saul’s
robe.
\VS{5}Afterward
David’s
conscience bothered
him because
he had
cut off
an edge
of Saul’s robe.
\VS{6}He said
to his men,
“May the
{\ND{Lord}}
keep me far away from
doing
such
a thing
to my lord,
who is the
{\ND{Lord}}’s
chosen one,
by extending
my hand
against him. After all,
he
is the
{\ND{Lord}}’s
chosen one.”
\VS{7}David
restrained
his men
with these words
and did not
allow
them to
rise
up against Saul.
Then Saul
left
the cave
and started
down the road.
\par }{\PP \VS{8}Afterward
David
got
up and went out
of the cave.
He called
out after
Saul,
“My lord,
O king!” When Saul
looked
behind
him, David
kneeled
down
and bowed
with his face
to the ground.
\VS{9}David
said
to Saul,
“Why
do you pay attention
when men
say,
‘David
is seeking
to do you harm’?
\VS{10}Today
your own eyes
see
how
the {\ND{Lord}}
delivered you – this very day – into my hands in the cave. Some told me to kill you, but I had pity on you and said, ‘I will not extend my hand against my lord, for he is the
{\ND{Lord}}’s chosen one.’
\VS{11}Look,
my father,
and see
the
edge
of your robe
in my hand! When
I cut off
the
edge
of your robe,
I didn’t
kill
you. So realize
and understand that
I am not planning
evil
or rebellion.
Even though I have not
sinned
against you, you
are waiting
in ambush to take
my life.
\VS{12}May the
{\ND{Lord}}
judge
between
the two
of us, and may the
{\ND{Lord}}
vindicate
me
over you, but my hand
will not
be against you.
\VS{13}It’s like the old
proverb
says: ‘From evil people
evil
proceeds.’
But my hand
will not
be against you.
\VS{14}Who
has the king
of Israel
come out
after? Who
is it that you
are pursuing? A dead
dog? A single
flea?
\VS{15}May
the {\ND{Lord}}
be our judge
and arbiter.
May he see
and arbitrate
my case
and deliver
me from your hands!”
\par }{\PP \VS{16}When
David
finished
speaking
these
words
to
Saul,
Saul
said,
“Is that your voice,
my son
David?” Then
Saul
wept
loudly.
\VS{17}He said
to
David,
“You
are more innocent
than
I, for
you
have treated
me well,
even though I
have tried
to harm you!
\VS{18}You
have explained
today
how
you have
treated
me well.
The
{\ND{Lord}}
delivered
me into your hand,
but you did not
kill me.
\VS{19}Now if
a man
finds
his enemy,
does he send
him on his way
in good
shape? May the
{\ND{Lord}}
repay
you with good
this
day
for what
you have
done to me.
\VS{20}Now
look,
I realize
that
you will in fact be
king
and that
the kingdom
of Israel
will be established
in your hand.
\VS{21}So now
swear
to me in the
{\ND{Lord}}’s
name that you will not
kill
my descendants
after
me or
destroy
my name
from the house
of my father.”
\par }{\PP \VS{22}David
promised
Saul
this on oath. Then Saul
went
to
his house,
and David
and his men
went up
to the stronghold.

\par }\Chap{25}{\PP \VerseOne{1}Samuel
died,
and all
Israel
assembled
and mourned
him. They buried
him at his home
in Ramah.
Then
David
left and went down
to
the desert
of Paran.
\par }{\SH David Marries Abigail the Widow of Nabal
\par }{\PP \VS{2}There was a man
in Maon
whose business
was in Carmel.
This man
was very
wealthy;
he owned three
thousand
sheep
and a thousand
goats.
At that time he was shearing
his sheep
in Carmel.
\VS{3}The man’s
name
was Nabal,
and his wife’s
name
was Abigail.
She was both wise
and beautiful,
but the man
was harsh
and his deeds
were evil.
He was a Calebite.
\par }{\PP \VS{4}When David
heard
in the desert
that
Nabal
was shearing
his sheep,
\VS{5}he
sent
ten
servants,
saying
to them, “Go up
to Carmel
to
see
Nabal
and give
him greetings
in my name.
\VS{6}Then you will say to my brother, “Peace
to you
and your house! Peace
to all
that
is yours!
\VS{7}Now
I hear
that
they are shearing
sheep for you. When your shepherds
were with
us, we
neither
insulted
them
nor
harmed them the whole
time
they were in Carmel.
\VS{8}Ask
your own servants;
they can tell
you! May my servants
find
favor
in your sight,
for
we
have come
at the time
of a holiday.
Please
provide
us
– your
servants
and your son
David
– with whatever
you can spare.”
\par }{\PP \VS{9}So David’s
servants
went
and spoke
all
these
words
to
Nabal
in David’s
name.
Then they paused.
\VS{10}But Nabal
responded
to David’s
servants,
“Who
is David,
and who
is this son
of Jesse? This is a time
when many
servants
are breaking away
from their masters!
\VS{11}Should I take
my bread
and my water
and my meat
that
I have slaughtered
for my shearers
and give
them to these
men? I don’t
even know
where
they came from!”
\par }{\PP \VS{12}So David’s
servants
went on their way.
When they had returned,
they came
and told
David
all
these
things.
\VS{13}Then David
instructed his men,
“Each
of you strap
on your sword!” So each
one strapped
on his sword,
and David
also
strapped
on his sword.
About four
hundred
men
followed
David
up,
while two hundred
stayed
behind with the equipment.
\par }{\PP \VS{14}But one
of the servants
told
Nabal’s
wife
Abigail,
“David
sent
messengers
from the desert
to greet
our lord,
but he screamed at them.
\VS{15}These men
were very
good
to us. They did not
insult
us, nor
did we sustain
any loss
during the entire
time
we
were together
in the field.
\VS{16}Both
night
and day
they were a protective wall
for us
the entire
time
we were with
them,
while we were tending
our flocks.
\VS{17}Now
be aware
of this, and see
what
you can do.
For
disaster
has been planned
for our lord
and his entire
household.
He is
such a wicked
person that no one tells him anything!”
\par }{\PP \VS{18}So Abigail
quickly
took
two hundred
loaves of bread,
two
containers
of wine,
five
prepared sheep,
five
seahs
of roasted
grain, a hundred
bunches of raisins,
and two hundred
lumps
of pressed
figs.
She loaded
them on
donkeys
\VS{19}and said
to her servants,
“Go
on ahead
of me. I
will come
after
you.” But
she did not
tell
her husband
Nabal.
\par }{\PP \VS{20}Riding
on
her donkey,
she went down
under cover
of the mountain.
David
and his men
were coming down
to meet
her, and she encountered them.
\VS{21}Now David
had been thinking, “In vain
I guarded everything
that
belonged to this
man in the desert.
I didn’t take
anything
from
him. But he has repaid
my good
with evil.
\VS{22}God
will severely
punish David,
if
I leave
alive until
morning
even one male
from all
those who belong to him!”
\par }{\PP \VS{23}When Abigail
saw
David,
she got down
quickly
from
the donkey,
threw
herself down before
David,
and bowed
to the ground.
\VS{24}Falling
at his feet,
she said,
“My lord,
I
accept all the guilt! But please
let your female servant
speak
with my lord! Please listen
to the words
of your servant!
\VS{25}My lord
should not
pay
attention
to
this
wicked
man
Nabal.
He simply lives up to
his name! His name means
‘fool,’
and he
is indeed foolish! But
I,
your servant,
did not
see
the servants
my lord
sent.
\par }{\PP \VS{26}“Now,
my lord,
as surely
as the
{\ND{Lord}}
lives
and as surely as you live,
it is the
{\ND{Lord}}
who
has kept
you from shedding
blood
and taking matters
into your own hands.
Now
may your enemies
and those who seek
to
harm
my lord
be like Nabal.
\VS{27}Now
let this
present
that
your servant
has brought
to my lord
be given
to the servants
who follow
my lord.
\VS{28}Please
forgive
the sin
of your servant,
for
the {\ND{Lord}}
will certainly
establish
the house
of my lord,
because
my lord
fights
the battles
of the {\ND{Lord}}. May no
evil
be found
in you all your days!
\VS{29}When someone
sets out
to chase
you and to take
your life,
the life
of my lord
will be
wrapped
securely in the bag
of the living
by
the {\ND{Lord}}
your God.
But he will sling away
the lives
of your enemies
from
the sling’s
pocket!
\VS{30}The
{\ND{Lord}}
will
do
for
my lord
everything
that
he promised
you, and he will make you a leader
over
Israel.
\VS{31}Your conscience will not
be
overwhelmed
with guilt
for having poured out
innocent
blood
and for having taken matters
into your own hands. When the
{\ND{Lord}}
has granted
my lord
success, please remember
your servant.”
\par }{\PP \VS{32}Then David
said
to Abigail,
“Praised
be the
{\ND{Lord}},
the God
of Israel,
who
has sent
you this
day
to meet me!
\VS{33}Praised
be your good judgment! May you
yourself be rewarded
for having prevented
me this
day
from shedding
blood
and taking matters
into my own hands!
\VS{34}Otherwise, as surely as
the {\ND{Lord}}, the God
of Israel,
lives
– he who has
prevented
me from harming you – if you had not come so quickly to meet me, by morning’s light not even one male belonging to Nabal would have remained alive!”
\VS{35}Then David
took
from her hand
what she had brought
to him. He said
to her, “Go
back to your home
in peace.
Be assured
that I have listened
to you and responded
favorably.”
\par }{\PP \VS{36}When Abigail
went back
to
Nabal,
he was holding
a banquet
in
his
house
like that of the king.
Nabal
was having a good
time
and was very
intoxicated.
She told
him absolutely
nothing
until
morning’s
light.
\VS{37}In the morning,
when Nabal
was sober,
his wife
told
him about these
matters.
He had a stroke
and was
paralyzed.
\VS{38}After
about ten
days
the {\ND{Lord}}
struck
Nabal
down and he died.
\par }{\PP \VS{39}When David
heard
that
Nabal
had died,
he said,
“Praised
be the
{\ND{Lord}}
who
has vindicated
me and avenged
the insult
that
I suffered from Nabal! The
{\ND{Lord}}
has kept
his servant
from doing evil,
and he has repaid
Nabal
for his evil
deeds.”
Then David
sent
word
to Abigail
and asked
her to become
his wife.
\par }{\PP \VS{40}So the servants
of David
went
to
Abigail
at Carmel
and said
to
her, “David
has sent
us to
you to bring
you back
to be his wife.”
\VS{41}She arose,
bowed
her face
toward the ground,
and said,
“Your female servant,
like a lowly servant,
will wash
the feet
of the servants
of my lord.”
\VS{42}Then
Abigail
quickly
went and mounted
her donkey,
with five
of her female
servants accompanying
her. She followed
David’s
messengers
and became
his wife.
\par }{\PP \VS{43}David
had also
married
Ahinoam
from Jezreel;
the two
of them became his wives.
\VS{44}(Now Saul
had given
his daughter
Michal,
David’s
wife,
to Paltiel
son
of Laish,
who
was from Gallim.)

\par }\Chap{26}{\PP \VerseOne{1}The Ziphites
came
to Saul
at Gibeah
and said,
“Isn’t
David
hiding
on the hill
of Hakilah
near
Jeshimon?”
\VS{2}So Saul
arose
and
\par }{\PP went down
to
the desert
of Ziph,
accompanied
by three
thousand
select
men
of Israel,
to look
for David
in the desert
of Ziph.
\VS{3}Saul
camped
by the road
on the hill
of Hakilah
near
Jeshimon,
but David
was staying in
the desert.
When he realized
that
Saul
had come
to the desert
to find him,
\VS{4}David
sent
scouts
and verified that
Saul
had indeed
arrived.
\par }{\PP \VS{5}So David
set out
and went
to
the place
where
Saul
was camped.
David
saw
the place
where
Saul
and Abner
son
of Ner,
the general in command
of his army,
were sleeping.
Now Saul
was lying
in the entrenchment,
and the army
was camped
all around him.
\VS{6}David
said
to
Ahimelech
the Hittite
and Abishai
son
of
Zeruiah,
Joab’s
brother,
“Who
will go down
with
me to
Saul
in
the camp?” Abishai
replied,
“I
will go down
with you.”
\par }{\PP \VS{7}So
David
and Abishai
approached
the army
at night
and found
Saul
lying
asleep
in the entrenchment
with his spear
stuck
in the ground
by his head.
Abner
and the army
were lying
all around him.
\VS{8}Abishai
said
to
David,
“Today
God
has delivered
your enemy
into your hands.
Now
let
me drive the spear
right through him into the ground
with one
swift
jab! A second jab won’t
be necessary!”
\par }{\PP \VS{9}But David
said
to
Abishai,
“Don’t
kill
him! Who
can extend
his hand
against the
{\ND{Lord}}’s
chosen one
and remain guiltless?”
\VS{10}David
went on to say,
“As the
{\ND{Lord}}
lives,
the {\ND{Lord}}
himself will
strike
him down. Either
his day
will come
and he will die,
or
he will go down
into battle
and be swept away.
\VS{11}But may the
{\ND{Lord}}
prevent me
from extending
my hand
against the
{\ND{Lord}}’s
chosen one! Now
take
the spear
by Saul’s head
and the jug
of water,
and let’s get out of here!”
\VS{12}So David
took
the
spear
and the
jug
of water
by Saul’s
head,
and they got out
of there. No one
saw
them or was
aware
of their presence or
woke up.
All
of them were asleep,
for
the {\ND{Lord}}
had caused a deep sleep
to fall
on them.
\par }{\PP \VS{13}Then David
crossed
to the other side
and stood
on
the top
of the hill
some distance
away; there
was a considerable
distance between them.
\VS{14}David
called
to
the army
and to
Abner
son
of Ner,
“Won’t
you answer,
Abner?” Abner
replied,
“Who
are you, that you
have called
to
the king?”
\VS{15}David
said
to
Abner,
“Aren’t you
a man? After all, who
is like
you in Israel? Why
then haven’t
you protected
your lord
the king? One
of the soldiers
came
to kill
your lord
the king.
\VS{16}This
failure
on your part isn’t
good! As surely
as the
{\ND{Lord}}
lives,
you people who
have not
protected
your lord,
the
{\ND{Lord}}’s
chosen one,
are as good as dead! Now
look
where
the king’s
spear
and the
jug
of water
that
was by his head are!”
\par }{\PP \VS{17}When Saul
recognized
David’s
voice,
he said,
“Is that your voice,
my son
David?” David
replied,
“Yes, it’s my voice,
my lord
the king.”
\VS{18}He went on to say,
“Why
is
my lord
chasing
his servant? What
have I done? What
wrong have I done?
\VS{19}So let
my lord
the king
now
listen
to the words
of his servant.
If
the {\ND{Lord}}
has incited
you against me, may he take delight
in an offering.
But if
men
have instigated this, may they
be cursed
before
the {\ND{Lord}}! For
they have driven
me away
this day
from being united
with the
{\ND{Lord}}’s
inheritance,
saying,
‘Go
on, serve
other
gods!’
\VS{20}Now
don’t
let my blood
fall
to the ground
away from the
{\ND{Lord}}’s
presence,
for
the king
of Israel
has gone out
to look
for a flea
the way one
looks for a partridge
in the hill country.”
\par }{\PP \VS{21}Saul
replied,
“I have sinned.
Come back,
my son
David.
I won’t
harm
you, for you treated my life
with value
this
day.
I have behaved foolishly
and have made a very
terrible
mistake!”
\VS{22}David
replied,
“Here
is the king’s
spear! Let one
of your servants
cross
over and get it.
\VS{23}The
{\ND{Lord}}
rewards each man
for his integrity
and loyalty.
Even though today
the {\ND{Lord}}
delivered
you into my hand,
I was not
willing
to extend
my hand
against the
{\ND{Lord}}’s
chosen one.
\VS{24}In the same way that
I valued
your life
this
day,
may
the {\ND{Lord}}
value
my life
and deliver
me from all
danger.”
\VS{25}Saul
replied
to
David,
“May you
be rewarded,
my son
David! You will without question
be successful!” So David
went
on his way,
and Saul
returned
to his place.

\par }\Chap{27}{\PP \VerseOne{1}David
thought
to himself, “One
of these days
I’m going to be swept
away by the hand
of Saul! There is nothing
better
for
me than to escape
to
the land
of the Philistines.
Then Saul
will despair
of searching
for me
through all
the territory
of Israel
and I will escape
from his hand.”
\par }{\PP \VS{2}So
David
left and crossed
over to King
Achish
son
of Maoch
of Gath
accompanied by his six
hundred
men.
\VS{3}David
settled
with
Achish
in Gath,
along with his
men
and their families.
David
had with him his two
wives,
Ahinoam
the Jezreelite
and Abigail
the Carmelite,
Nabal’s
widow.
\VS{4}When Saul
learned that
David
had fled
to Gath,
he did not
mount a new search for him.
\par }{\PP \VS{5}David
said
to
Achish,
“If
I have found
favor
with you, let me
be given
a place
in one
of the country
towns
so that I can live
there.
Why
should your servant
settle
in the royal
city
with you?”
\VS{6}So Achish
gave
him Ziklag
on that day.
(For that
reason
Ziklag
has belonged to the kings
of Judah
until
this
very day.)
\VS{7}The length of time
that
David
lived
in the Philistine
countryside
was a year
and four
months.
\par }{\PP \VS{8}Then David
and his men
went up
and raided
the Geshurites,
the Girzites,
and the Amalekites.
(They
had been living
in that land
for
a long time,
from the approach
to Shur
as far as
the land
of Egypt.)
\VS{9}When David
would attack
a district,
he would leave neither
man
nor woman
alive.
He would take
sheep,
cattle,
donkeys,
camels,
and clothing
and would then go back
to
Achish.
\VS{10}When Achish
would ask,
“Where did you raid
today?” David
would say,
“The Negev
of Judah”
or “The Negev
of Jeharmeel”
or “The Negev
of the Kenites.”
\VS{11}Neither man
nor
woman
would David
leave alive
so as to bring
them back to Gath.
He was thinking,
“This way they can’t
tell
on
us,
saying,
‘This is what
David
did.’ ”
Such was his practice
the
entire
time
that
he lived
in the country
of the Philistines.
\VS{12}So Achish
trusted
David,
thinking
to himself, “He is really hated
among his own people
in Israel! From now on
he will be
my servant.”

\par }\Chap{28}{\PP \VerseOne{1}In those
days
the Philistines
gathered
their troops
for war
in order
to fight
Israel.
Achish
said
to
David,
“You should fully
understand
that
you
and your men
must
go with
me into the battle.”
\VS{2}David
replied
to
Achish,
“That being
the case, you
will come to know
what your servant
can do!” Achish
said
to
David,
“Then
I will make you my bodyguard
from now on.”
\par }{\PP \VS{3}Now Samuel
had died,
and all
Israel
had lamented
over him and had buried
him in Ramah,
his hometown.
In the meantime Saul
had removed
the mediums
and magicians
from the land.
\VS{4}The Philistines
assembled;
they came
and camped
at Shunem.
Saul
mustered all
Israel
and camped
at Gilboa.
\VS{5}When Saul
saw
the camp
of the Philistines,
he was absolutely
terrified.
\VS{6}So Saul
inquired
of the {\ND{Lord}}, but the
{\ND{Lord}}
did not
answer him – not by dreams nor by Urim nor by the prophets.
\VS{7}So Saul
instructed
his servants,
“Find
me a woman
who is a medium,
so that I may go
to
her and inquire
of her.” His servants
replied
to
him, “There
is a woman
who is a medium
in Endor.”
\par }{\PP \VS{8}So Saul
disguised
himself and put on
other
clothing
and left,
accompanied by two
of his men.
They came
to
the woman
at night
and said,
“Use
your ritual
pit to conjure
up
for me the
one I tell you.”
\par }{\PP \VS{9}But the woman
said
to
him, “Look,
you
are aware
of what Saul
has
done;
he has
removed the mediums
and magicians
from
the land! Why
are you
trapping
me so you can put me to death?”
\VS{10}But Saul
swore
an oath to her by the
{\ND{Lord}}, “As surely
as the
{\ND{Lord}}
lives,
you will not
incur
guilt
in this
matter!”
\VS{11}The woman
replied,
“Who
is it that I should bring up
for you?” He said,
“Bring up
for me Samuel.”
\par }{\PP \VS{12}When the woman
saw
Samuel,
she cried
out loudly.
The woman
said
to
Saul,
“Why
have you deceived
me? You
are Saul!”
\VS{13}The king
said
to her, “Don’t
be afraid! What
have you seen?” The woman
replied
to
Saul,
“I have seen
one like a god
coming up
from
the ground!”
\VS{14}He said
to her, “What
about his appearance?” She said,
“An old
man
is coming up! He is wrapped
in a robe!”
\par }{\PP Then Saul
realized
it was Samuel,
and he bowed
his face
toward the ground
and kneeled down.
\VS{15}Samuel
said
to
Saul,
“Why
have you disturbed
me by bringing me up?” Saul
replied,
“I am terribly
troubled! The Philistines
are fighting
against me and God
has turned
away from me. He does not
answer me – not by the prophets nor by dreams. So I have called on you to tell me what I should do.”
\par }{\PP \VS{16}Samuel
said,
“Why
are you asking
me, now that the
{\ND{Lord}}
has turned
away from you and has become
your enemy?
\VS{17}The
{\ND{Lord}}
has
done
exactly
as I prophesied! The
{\ND{Lord}}
has torn
the kingdom
from your hand
and has given
it to your neighbor
David!
\VS{18}Since
you did not
obey
the {\ND{Lord}}
and did not
carry
out his fierce
anger
against the Amalekites,
the {\ND{Lord}}
has done
this
thing
to you today.
\VS{19}The
{\ND{Lord}}
will hand
you
and Israel
over
to the Philistines! Tomorrow
both you
and your sons
will be with
me. The
{\ND{Lord}}
will also
hand
the
army
of Israel
over
to the Philistines!”
\par }{\PP \VS{20}Saul
quickly
fell
full
length
on the ground
and was very
afraid
because of Samuel’s
words.
He was
completely drained
of energy,
not
having
eaten
anything
all
that day
and night.
\VS{21}When the woman
came
to
Saul
and saw
how terrified
he was, she said
to
him, “Your servant
has done
what you asked.
I took
my life
into my own hands
and did what
you told
me.
\VS{22}Now
it’s your turn to listen
to your
servant! Let me
set
before
you
a bit
of bread
so that you can eat.
When
you regain your strength,
you can go
on your way.”
\par }{\PP \VS{23}But he refused,
saying,
“I won’t
eat!” Both
his servants
and the woman
urged
him to eat, so he gave in.
He got up
from the ground
and sat
down on
the bed.
\VS{24}Now the woman
had a well-fed
calf
at her home
that she quickly
slaughtered.
Taking
some flour,
she kneaded
bread
and baked
it without leaven.
\VS{25}She brought
it to Saul
and his servants,
and they ate.
Then they arose
and left
that same night.

\par }\Chap{29}{\PP \VerseOne{1}The Philistines
assembled
all
their troops
at Aphek,
while Israel
camped
at the spring
that
is in Jezreel.
\VS{2}When the leaders
of the Philistines
were passing
in review at the head of their units of hundreds
and thousands,
David
and his men
were passing
in review
in the rear with
Achish.
\par }{\PP \VS{3}The leaders
of the Philistines
asked,
“What
about these
Hebrews?” Achish
said
to
the leaders
of the Philistines,
“Isn’t
this
David,
the servant
of King
Saul
of Israel,
who
has
been with
me for quite some time? I have found
no
fault
with him from the day
of his defection
until
the present time!”
\par }{\PP \VS{4}But the leaders
of the Philistines
became angry
with him and said
to him, “Send
the man
back! Let him return
to
the place
that
you assigned
him! Don’t
let him go down
with
us into the battle,
for he might become
our adversary
in the battle.
What
better
way to
please his lord
than with the heads
of these men?
\VS{5}Isn’t
this
David,
of whom
they sang
as they danced,
\par }{\Q ‘Saul
has struck
down his thousands,
\par }{\Q but David
his tens of thousands’?”
\par }{\PP \VS{6}So Achish
summoned
David
and said
to
him, “As surely
as the
{\ND{Lord}}
lives, you are an honest
man, and I am glad to
have you serving
with
me in the army.
I have found
no
fault
with you from the day
that you first came
to me
until
the present
time.
But in
the opinion
of the leaders,
you
are not
reliable.
\VS{7}So
turn
and leave
in peace.
You must not
do
anything that the leaders
of the Philistines
consider improper!”
\par }{\PP \VS{8}But
David
said
to
Achish,
“What
have I done? What
have you found
in your servant
from the day
that
I first came into your presence
until
the present
time,
that
I shouldn’t go
and fight
the enemies
of my lord
the king?”
\VS{9}Achish
replied
to
David,
“I am convinced
that
you
are as reliable
as the angel
of God! However,
the leaders
of the Philistines
have said,
‘He must not
go up
with
us in the battle.’
\VS{10}So get
up early
in the morning
along with the servants
of your lord
who have
come
with
you. When you get up early
in the morning,
as soon as it is light
enough to see, leave.”
\par }{\PP \VS{11}So David
and his men
got
up early
in the morning
to return
to
the land
of the Philistines,
but the Philistines
went up
to Jezreel.

\par }\Chap{30}{\PP \VerseOne{1}On the third
day
David
and his men
came
to Ziklag.
Now the Amalekites
had raided
the Negev
and Ziklag.
They attacked
Ziklag
and burned it.
\VS{2}They took captive
the
women
who
were in it, from the youngest
to the oldest,
but they did not
kill
anyone.
They simply
carried them off and went
on their way.
\par }{\PP \VS{3}When David
and his men
came
to
the city,
they found
it burned.
Their wives,
sons,
and daughters
had been taken captive.
\VS{4}Then
David
and the men
who
were with
him wept loudly
until
they could weep
no
more.
\VS{5}David’s
two
wives
had been taken captive
– Ahinoam
the Jezreelite
and Abigail
the Carmelite,
Nabal’s
widow.
\VS{6}David
was very
upset,
for
the men
were thinking
of stoning
him; each man
grieved bitterly over
his sons
and daughters.
But David
drew strength
from the
{\ND{Lord}}
his God.
\par }{\PP \VS{7}Then David
said
to the priest
Abiathar
son
of Ahimelech,
“Bring
me
the ephod.”
So
Abiathar
brought the ephod
to
David.
\VS{8}David
inquired
of the {\ND{Lord}}, saying,
“Should I pursue
this
raiding band? Will I overtake
them?” He said
to him, “Pursue,
for
you will certainly
overtake
them and carry out
a rescue!”
\par }{\PP \VS{9}So David
went,
accompanied by his six
hundred
men.
When he came
to the Wadi
Besor,
those who were in the rear
stayed there.
\VS{10}David
and four
hundred
men
continued the pursuit,
but two hundred
men
who
were too exhausted
to cross
the Wadi
Besor
stayed there.
\par }{\PP \VS{11}Then they found
an
Egyptian
in the field
and brought
him to
David.
They gave
him bread
to eat
and water
to drink.
\VS{12}They gave
him a slice
of pressed figs
and two
bunches of raisins
to eat.
This greatly refreshed
him,
for
he had not
eaten
food
or
drunk
water
for three
days
and three
nights.
\VS{13}David
said
to him, “To whom
do you
belong, and where
are you
from?” The young man
said,
“I am
an Egyptian,
the servant
of an Amalekite
man.
My master
abandoned
me when
I was ill
for three
days.
\VS{14}We
conducted a raid
on the Negev
of the Kerethites,
on
the area of Judah,
and on
the Negev
of Caleb.
We burned
Ziklag.”
\VS{15}David
said
to him,
“Can you take us down
to
this
raiding party?” He said,
“Swear
to me by God
that you will not
kill
me or
hand
me over to my master,
and I will take you down
to
this
raiding party.”
\par }{\PP \VS{16}So he took David down,
and they found
them spread
out over
the land.
They were eating
and drinking
and enjoying
themselves because of all
the loot
they had
taken
from the land
of the Philistines
and from the land
of Judah.
\VS{17}But David
struck
them down from twilight
until
the following
evening.
None
of them escaped,
with the exception
of four
hundred
young
men
who
got away
on
camels.
\VS{18}David
retrieved
everything
the Amalekites
had taken;
he
also rescued
his two
wives.
\VS{19}There was nothing
missing,
whether
small
or great.
He retrieved
sons
and daughters,
the plunder,
and everything
else they had
taken.
David
brought
everything
back.
\VS{20}David
took
all
the flocks
and herds
and drove
them in front
of the rest of the animals.
People were saying,
“This
is David’s
plunder!”
\par }{\PP \VS{21}Then David
approached
the two hundred
men
who
had been too exhausted
to go
with
him,
those whom they had left
at the Wadi
Besor.
They went out
to meet
David
and the people
who
were with
him. When David
approached
the people,
he asked
how they were doing.
\VS{22}But
all
the evil
and worthless
men
among those who had
gone
with
David
said,
“Since
they didn’t
go
with
us, we won’t give
them any
of the loot
we
retrieved! They may take only
their wives
and children.
Let them lead
them away and be gone!”
\par }{\PP \VS{23}But David
said,
“No! You shouldn’t do
this,
my brothers.
Look at what
the {\ND{Lord}}
has given
us! He has protected
us and has delivered
into our hands
the raiding party
that came
against us.
\VS{24}Who
will listen
to you in this
matter? The portion
of the one who went down
into the battle
will be the same as the portion
of the one who remained
with the equipment! Let their portions be the same!”
\par }{\PP \VS{25}From that time
onward
it was
a binding
ordinance
for Israel,
right
up to the present
time.
\par }{\PP \VS{26}When David
came
to
Ziklag,
he sent
some of the plunder
to the elders
of Judah
who were his friends,
saying,
“Here’s
a gift
for you from the looting
of the
{\ND{Lord}}’s
enemies!”
\VS{27}The gift was for those in the following locations: for those in Bethel,
Ramoth Negev,
and Jattir;
\VS{28}for those in Aroer,
Siphmoth,
Eshtemoa,
\VS{29}and Racal;
for those in the cities
of the Jerahmeelites
and Kenites;
\VS{30}for those in Hormah,
Bor Ashan,
Athach,
\VS{31}and Hebron;
and for those in whatever
other places
David
and his men
had
traveled.

\par }\Chap{31}{\PP \VerseOne{1}Now the Philistines
were fighting
against Israel.
The men
of Israel
fled
from the Philistines
and many of them fell
dead
on Mount
Gilboa.
\VS{2}The Philistines
stayed right on the heels
of Saul
and his sons.
They struck
down Saul’s
sons
Jonathan,
Abinadab,
and Malki-Shua.
\VS{3}Saul
himself was in the thick
of the battle;
the archers
spotted
him and wounded
him severely.
\par }{\PP \VS{4}Saul
said
to his armor
bearer,
“Draw
your sword
and stab
me with it! Otherwise
these
uncircumcised people
will come,
stab
me, and torture
me.” But his armor
bearer
refused
to do it, because
he was very
afraid.
So Saul
took
his sword
and fell
on it.
\VS{5}When his armor
bearer
saw
that
Saul
was dead,
he also
fell
on
his own sword
and died
with him.
\VS{6}So
Saul,
his three
sons,
his armor
bearer,
and all
his men
died
together
that day.
\par }{\PP \VS{7}When
the men
of Israel
who
were in the valley
and across
the Jordan
saw that
the men
of Israel
had fled
and that
Saul
and his sons
were dead,
they abandoned
the
cities
and fled.
The Philistines
came
and occupied them.
\par }{\PP \VS{8}The next
day, when the Philistines
came
to strip
loot from the corpses,
they discovered
Saul
and his three
sons
lying dead
on Mount
Gilboa.
\VS{9}They cut off
Saul’s head
and stripped
him
of his armor.
They sent
messengers to announce
the news in the temple
of their idols
and among their people
throughout the
surrounding
land
of the Philistines.
\VS{10}They placed
Saul’s armor
in the temple
of the Ashtoreths
and hung
his corpse
on the city wall
of Beth Shan.
\par }{\PP \VS{11}When the residents
of Jabesh
Gilead
heard
what
the Philistines
had
done
to Saul,
\VS{12}all
their warriors
set out
and traveled
throughout
the night.
They took
Saul’s
corpse
and the corpses
of his sons
from the city wall
of Beth Shan
and went
to Jabesh,
where
they burned
them.
\VS{13}They took
the bones
and buried
them under
the tamarisk
tree at Jabesh;
then they fasted
for seven
days.
\par }