\NormalFont\ShortTitle{Romans}
{\MT Romans

\par }\ChapOne{1}{\SH Salutation
\par }{\PP \VerseOne{1}From Paul,
a slave
of Christ
Jesus,
called
to be an apostle,
set apart
for
the gospel
of God.
\VS{2}This gospel he promised beforehand
through
his
prophets
in
the holy
scriptures,
\VS{3}concerning
his
Son
who was
a descendant
of David
with reference to
the flesh,
\VS{4}who was appointed
the Son-of-God-in-power
according to
the Holy
Spirit
by
the resurrection
from the dead,
Jesus
Christ
our
Lord.
\VS{5}Through
him we have received
grace
and
our apostleship
to
bring about the obedience
of faith
among
all
the Gentiles
on behalf
of his
name.
\VS{6}You
also
are
among
them, called
to belong to Jesus
Christ.
\VS{7}To all
those loved
by God
in
Rome,
called
to be saints: Grace
and
peace
to you
from
God
our
Father
and
the Lord
Jesus
Christ!
\par }{\SH Paul’s Desire to Visit Rome
\par }{\PP \VS{8}First of all,
I thank
my
God
through
Jesus
Christ
for
all
of you,
because
your
faith
is proclaimed
throughout
the whole
world.
\VS{9}For
God,
whom
I serve
in
my
spirit
by
preaching the gospel
of his
Son,
is
my
witness
that
I
continually
remember
you
\VS{10}and I always
ask
in
my
prayers,
if
perhaps
now
at last I may succeed
in visiting
you
according to
the will
of God.
\VS{11}For
I long
to see
you,
so that
I may impart
to you
some
spiritual
gift
to
strengthen
you,
\VS{12}that
is,
that
we
may be mutually comforted
by
one another’s
faith,
both
yours
and
mine.
\VS{13}I do
not
want
you
to be unaware,
brothers and sisters,
that
I
often
intended
to come
to
you
(and
was prevented
until
now), so that
I may have
some
fruit
even
among
you,
just as
I already have among
the rest
of the Gentiles.
\VS{14}I am
a debtor
both
to the Greeks
and
to the barbarians,
both
to the wise
and
to the foolish.
\VS{15}Thus
I
am eager
also
to preach the gospel
to you
who are in
Rome.
\par }{\SH The Power of the Gospel
\par }{\PP \VS{16}For
I am
not
ashamed
of the gospel,
for
it is
God’s
power
for
salvation
to everyone
who believes,
to the Jew
first
and
also
to the Greek.
\VS{17}For
the righteousness
of God
is revealed
in
the gospel
from
faith
to
faith,
just as
it is written, “{\QT{
The righteous
by
faith
will live}}.”
\par }{\SH The Condemnation of the Unrighteous
\par }{\PP \VS{18}For
the wrath
of God
is revealed
from
heaven
against
all
ungodliness
and
unrighteousness
of people
who suppress
the truth
by
their unrighteousness,
\VS{19}because
what can be known
about God
is
plain
to
them,
because
God
has made
it plain
to them.
\VS{20}For
since
the creation
of the world
his
invisible attributes
– his
eternal
power
and
divine nature – have been clearly seen, because they are understood through what has been made. So people are without excuse.
\VS{21}For
although they knew
God,
they did
not
glorify
him as
God
or
give
him thanks,
but
they became futile
in
their
thoughts
and
their
senseless
hearts
were darkened.
\VS{22}Although they claimed
to be
wise,
they became fools
\VS{23}and
exchanged
the glory
of the immortal
God
for
an image
resembling
mortal
human beings
or
birds
or
four-footed animals
or
reptiles.
\par }{\PP \VS{24}Therefore
God
gave
them
over
in
the desires
of their
hearts
to
impurity,
to dishonor
their
bodies
among
themselves.
\VS{25}They
exchanged
the truth
of God
for
a lie
and
worshiped
and
served
the creation
rather than
the Creator,
who
is
blessed
forever! Amen.
\par }{\PP \VS{26}For
this reason
God
gave
them
over
to
dishonorable
passions.
For
their
women
exchanged
the natural
sexual relations
for
unnatural ones,
\VS{27}and
likewise
the men
also
abandoned
natural
relations
with women
and were inflamed
in
their
passions
for one another.
Men
committed
shameless acts
with
men
and
received
in
themselves
the due penalty
for their
error.
\par }{\PP \VS{28}And
just as
they did
not
see
fit
to
acknowledge
God,
God
gave
them
over
to
a depraved
mind,
to do
what should
not
be done.
\VS{29}They are filled
with every kind
of unrighteousness,
wickedness,
covetousness,
malice.
They are rife
with envy,
murder,
strife,
deceit,
hostility.
They are gossips,
\VS{30}slanderers,
haters of God,
insolent,
arrogant,
boastful,
contrivers
of all sorts of evil,
disobedient
to parents,
\VS{31}senseless,
covenant-breakers,
heartless,
ruthless.
\VS{32}Although
they fully know
God’s
righteous decree
that
those who practice
such things
deserve
to die,
they
not
only
do
them
but
also
approve
of those who practice them.

\par }\Chap{2}{\PP \VerseOne{1}Therefore
you are
without excuse,
whoever you are, when you judge
someone else. For
on whatever grounds you judge
another,
you condemn
yourself,
because
you
who judge
practice
the same things.
\VS{2}Now
we know
that
God’s
judgment
is
in accordance with
truth
against
those who practice
such things.
\VS{3}And
do you think,
whoever you are,
when you judge
those who practice
such things
and yet
do
them
yourself,
that
you will escape
God’s
judgment?
\VS{4}Or
do you have contempt
for the wealth
of his
kindness,
forbearance,
and
patience,
and yet do not know
that
God’s
kindness
leads
you
to
repentance?
\VS{5}But
because of
your
stubbornness
and
your unrepentant
heart,
you are storing up
wrath
for yourselves
in
the day
of wrath,
when God’s
righteous judgment
is revealed!
\VS{6}He
{\QT{
will reward
each one
according to
his
works}}:
\VS{7}eternal
life
to those who by
perseverance
in good
works
seek
glory
and
honor
and
immortality,
\VS{8}but
wrath
and
anger
to those who live in
selfish ambition
and
do not obey
the truth
but
follow
unrighteousness.
\VS{9}There will be affliction
and
distress
on
everyone
who does
evil,
on the Jew
first
and
also
the Greek,
\VS{10}but
glory
and
honor
and
peace
for everyone
who does
good,
for
the Jew
first
and
also
the Greek.
\VS{11}For
there is
no
partiality
with
God.
\VS{12}For
all
who have sinned
apart from the law
will
also
perish apart
from the law,
and all who have sinned
under the law will be judged
by
the law.
\VS{13}For
it is not
those who hear
the law
who are righteous
before
God,
but
those who do
the law
will be declared righteous.
\VS{14}For
whenever
the Gentiles,
who do
not
have
the law,
do
by nature
the things required by the law,
these
who do not
have
the law
are
a law
to themselves.
\VS{15}They show
that the work
of the law
is written
in
their
hearts,
as their
conscience
bears witness
and
their conflicting thoughts
accuse
or
else
defend them,
\VS{16}on
the day
when God
will judge
the secrets
of human
hearts, according to
my
gospel
through
Christ
Jesus.
\par }{\SH The Condemnation of the Jew
\par }{\PP \VS{17}But
if
you
call
yourself a Jew
and
rely
on the law
and
boast
of your relationship to
God
\VS{18}and
know
his will
and
approve
the superior things
because you receive instruction
from
the law,
\VS{19}and
if you are convinced
that you yourself
are
a guide
to the blind,
a light
to those who are in
darkness,
\VS{20}an educator
of
the senseless,
a teacher
of little children,
because you have
in
the law
the essential features
of knowledge
and
of the truth –
\VS{21}therefore
you who teach
someone else,
do you
not
teach
yourself? You who preach
against
stealing,
do you steal?
\VS{22}You who tell
others not
to commit adultery,
do you commit adultery? You who abhor
idols,
do you rob temples?
\VS{23}You
who
boast
in
the law
dishonor
God
by
transgressing
the law!
\VS{24}For
just as
it is written, “{\QT{
the name
of God
is being blasphemed
among
the Gentiles
because of
you}}.”
\par }{\PP \VS{25}For
circumcision
has
its value
if
you practice
the law,
but
if
you
break
the law,
your
circumcision
has become
uncircumcision.
\VS{26}Therefore
if
the uncircumcised man
obeys
the righteous requirements
of the law,
will
not
his
uncircumcision
be regarded
as
circumcision?
\VS{27}And
will not the physically
uncircumcised man
who keeps
the law
judge
you
who, despite the written code
and
circumcision,
transgress
the law?
\VS{28}For
a person is
not
a Jew
who is one outwardly,
nor
is circumcision
something that is outward
in
the flesh,
\VS{29}but
someone is a Jew
who is one inwardly,
and
circumcision
is of the heart
by
the Spirit
and
not
by the written code.
This person’s praise
is not
from
people
but
from
God.


\par }\Chap{3}{\PP \VerseOne{1}Therefore
what
advantage
does the Jew
have, or
what
is the value
of circumcision?
\VS{2}Actually, there are many
advantages.
First
of all,
the Jews were entrusted
with the oracles
of God.
\VS{3}What
then? If
some
did
not
believe,
does their
unbelief
nullify
the faithfulness
of God?
\VS{4}Absolutely not! Let God
be proven true,
and
every
human being
shown up as a liar,
just as
it is written: “{\BD{
{\IT{
so that
you will be justified
in
your
words
and
will prevail
when you
are judged}}}}.”
\par }{\PP \VS{5}But
if
our
unrighteousness
demonstrates
the righteousness
of God,
what
shall we say? The God
who inflicts
wrath
is not
unrighteous,
is he? (I am speaking
in
human terms.)
\VS{6}Absolutely
not! For otherwise
how could
God
judge
the world?
\VS{7}For if
by
my
lie
the truth
of God
enhances
his
glory,
why
am
I
still
actually being judged
as
a sinner?
\VS{8}And
why not
say, “Let us
do
evil
so that
good
may come of it”? – as some who slander us allege that we say. (Their condemnation is deserved!)
\par }{\SH The Condemnation of the World
\par }{\PP \VS{9}What
then? Are we better off? Certainly not,
for
we have already charged
that Jews
and
Greeks
alike are
all
under
sin,
\VS{10}just as
it is written:
\par }{\Q “{\BD{
{\IT{
There is
no one
righteous, not even
one}}}},
\par }{\Q \VS{11}{\BD{
{\IT{
there is
no one
who understands,}}}}
\par }{\Q {\BD{
{\IT{
there is
no one
who seeks
God}}}}.
\par }{\Q \VS{12}{\BD{
{\IT{
All
have turned away,}}}}
\par }{\Q {\BD{
{\IT{
together
they have become worthless;}}}}
\par }{\Q {\BD{
{\IT{
there is
no one
who shows
kindness, not
even one.}}}}”
\par }{\Q \VS{13}“{\BD{
{\IT{
Their
throats
are open
graves,}}}}
\par }{\Q {\BD{
{\IT{
they deceive
with their
tongues,}}}}
\par }{\Q {\BD{
{\IT{
the poison
of asps
is under
their
lips}}}}.”
\par }{\Q \VS{14}“{\BD{
{\IT{
Their mouths
are full
of cursing
and
bitterness}}}}.”
\par }{\Q \VS{15}“{\IT{
{\BD{Their}}
feet
are swift
to shed
blood,}}
\par }{\Q \VS{16}{\IT{
{\BD{
ruin
and
misery
are in
their
paths}},}}
\par }{\Q \VS{17}{\IT{
{\BD{
and
the way
of peace
they have
not
known}}.}}”
\par }{\Q \VS{18}“{\IT{
{\BD{There is}}
no
fear
of God
before
their
eyes.}}”
\par }{\PP \VS{19}Now
we know
that
whatever
the law
says,
it says
to those who are under
the law,
so that
every
mouth
may be silenced
and
the whole
world
may be
held accountable
to God.
\VS{20}For
{\IT{
no one
is declared righteous
before
him}}
by
the works
of the law,
for
through
the law
comes the knowledge
of sin.
\VS{21}But
now
apart
from the law
the righteousness
of God
(which is attested
by
the law
and
the prophets) has been disclosed –
\VS{22}namely, the righteousness
of God
through
the faithfulness
of Jesus
Christ
for
all
who believe.
For
there is
no distinction,
\VS{23}for
all
have sinned
and
fall short
of the glory
of God.
\VS{24}But they are justified
freely
by his
grace
through
the redemption
that is in
Christ
Jesus.
\VS{25}God
publicly displayed
him at
his
death
as the mercy seat
accessible through
faith.
This was to
demonstrate
his
righteousness,
because
God in his forbearance had passed over
the sins
previously committed.
\VS{26}This was also to demonstrate
his
righteousness
in
the present
time,
so that he would be
just
and
the justifier
of the one who lives because of
Jesus’
faithfulness.
\par }{\PP \VS{27}Where,
then,
is boasting? It is excluded! By
what
principle? Of works? No,
but
by
the principle
of faith!
\VS{28}For
we consider
that a person
is declared righteous
by faith
apart from
the works
of the law.
\VS{29}Or
is God the God of the Jews
only? Is
he not the God
of the Gentiles
too? Yes,
of the Gentiles too!
\VS{30}Since
God
is one,
he will justify
the circumcised
by
faith
and
the uncircumcised
through
faith.
\VS{31}Do we
then
nullify
the law
through
faith? Absolutely
not! Instead
we uphold
the law.

\par }\Chap{4}{\PP \VerseOne{1}What
then
shall we say
that Abraham,
our
ancestor
according
to the flesh, has discovered regarding this matter?
\VS{2}For
if
Abraham
was declared righteous
by
the works
of the law, he has
something to boast about – but not before God.
\VS{3}For
what
does
the scripture
say? “{\QT{
Abraham
believed
God, and
it was credited
to him
as
righteousness}}*.”
\VS{4}Now
to the one who works,
his pay
is
not
credited
due
to grace
but
due
to obligation.
\VS{5}But
to the one who does
not
work,
but
believes
in
the one who declares
the ungodly
righteous, his
faith
is credited
as
righteousness.
\par }{\PP \VS{6}So
even David
himself speaks
regarding the blessedness
of the man
to whom
God
credits
righteousness
apart from
works:
\par }{\Q \VS{7}“{\BD{
{\QT{
{\IT{Blessed}}}}}}
{\QT{are those whose}}
lawless deeds
are forgiven,
and
whose
sins
are covered;
\par }{\Q \VS{8}{\QT{
blessed
is the one
against whom the Lord
will
never
count
sin}}*.”
\par }{\PP \VS{9}Is this
blessedness
then
for
the circumcision
or
also
for
the uncircumcision? For
we say, “{\IT{
faith
{\BD{
was credited
to}}
Abraham
{\BD{
as
righteousness}}}}.”
\VS{10}How
then
was it credited
to him? Was he circumcised
at the time, or
not? No, he was not
circumcised
but
uncircumcised!
\VS{11}And
he received
the sign
of circumcision
as a seal
of the righteousness
that he had by faith
while he was still uncircumcised,
so that he
would become
the
father
of all
those who believe
but have never been circumcised,
that they too could have righteousness
credited
to them.
\VS{12}And
he is also the father
of the circumcised,
who are not
only
circumcised,
but
who also walk
in the footsteps
of the faith
that our
father
Abraham
possessed when he was still uncircumcised.
\par }{\PP \VS{13}For
the promise
to Abraham
or
to his
descendants
that he
would inherit
the world
was not fulfilled through
the law,
but
through
the righteousness
that comes by faith.
\VS{14}For
if
they become heirs
by
the law,
faith
is empty
and
the promise
is nullified.
\VS{15}For
the law
brings
wrath,
because where
there is
no
law
there is no
transgression
either.
\VS{16}For this reason
it is by
faith
so that
it may be
by
grace,
with the result
that the promise
may be certain to all
the descendants
– not
only to those who are under the law,
but
also
to those who have the faith
of Abraham,
who
is
the father
of us
all
\VS{17}(as
it is written, “{\QT{
I have made
you
the father
of many
nations}}”). He is our father in the presence
of God whom
he believed
– the God
who makes
the dead
alive
and
summons
the things that do not
yet exist
as though
they already do.
\VS{18}Against hope
Abraham believed
in
hope
with the result that he
became
{\QT{
the father
of many
nations}}
according to
the pronouncement, “{\QT{
so
will
your
descendants
be}}.”
\VS{19}Without
being weak
in faith,
he considered
his own
body
as dead
(because he was
about
one hundred years old) and
the deadness
of Sarah’s
womb.
\VS{20}He did
not
waver
in unbelief
about
the promise
of God
but
was strengthened
in faith,
giving
glory
to God.
\VS{21}He
was fully convinced
that
what
God promised
he was
also
able
to do.
\VS{22}So indeed
it was credited
to Abraham
as
righteousness.
\par }{\PP \VS{23}But
the statement
{\QT{
it was credited
to him}}
was
not
written
only
for
Abraham’s
sake,
\VS{24}but
also
for
our
sake,
to whom
it will
be credited,
those who believe
in
the one who raised
Jesus
our
Lord
from
the dead.
\VS{25}He was given over
because
of our
transgressions
and
was raised
for the sake
of our
justification.

\par }\Chap{5}{\PP \VerseOne{1}Therefore,
since we have been declared righteous
by
faith,
we have
peace
with
God
through
our
Lord
Jesus
Christ,
\VS{2}through
whom
we have
also
obtained
access
by faith
into
this
grace
in
which
we stand,
and
we rejoice
in
the hope
of God’s
glory.
\VS{3}Not only
this, but
we
also
rejoice
in
sufferings,
knowing
that
suffering
produces
endurance,
\VS{4}and
endurance,
character,
and
character,
hope.
\VS{5}And
hope
does not
disappoint,
because
the love
of God
has been poured out
in
our
hearts
through
the Holy
Spirit
who was given
to us.
\par }{\PP \VS{6}For while
we
were
still
helpless,
at
the right time
Christ
died
for
the ungodly.
\VS{7}(For
rarely
will
anyone
die
for
a righteous person,
though for
a good person
perhaps
someone
might possibly dare
to die.)
\VS{8}But
God
demonstrates
his own
love
for
us,
in that
while
we
were
still
sinners,
Christ
died
for
us.
\VS{9}Much
more
then,
because we have
now
been declared righteous
by
his
blood,
we will be saved
through
him
from
God’s wrath.
\VS{10}For
if
while
we were
enemies
we were reconciled
to God
through
the death
of his
Son,
how much
more,
since we have been reconciled,
will we be saved
by
his
life?
\VS{11}Not
only
this, but
we
also
rejoice
in
God
through
our
Lord
Jesus
Christ,
through
whom
we have
now
received
this reconciliation.
\par }{\SH The Amplification of Justification
\par }{\PP \VS{12}So then,
just as
sin
entered
the world
through
one
man
and
death
through
sin,
and
so
death
spread
to
all
people
because
all
sinned –
\VS{13}for
before the law
was
given, sin
was in
the world,
but
there is
no
accounting
for sin
when
there is
no
law.
\VS{14}Yet
death
reigned
from
Adam
until
Moses
even
over
those who did not
sin
in
the same way
that Adam
(who
is
a type
of the coming
one) transgressed.
\VS{15}But
the gracious gift
is not
like
the transgression.
For
if
the many
died
through the transgression
of the one man,
how much
more
did the grace
of God
and
the
gift
by
the grace
of the
one
man
Jesus
Christ
multiply
to
the many!
\VS{16}And
the gift is not
like
the one
who sinned.
For
judgment,
resulting from
the one
transgression, led to
condemnation,
but
the gracious gift
from
the many
failures
led to
justification.
\VS{17}For
if,
by the transgression
of the one man,
death
reigned
through
the one,
how much
more
will those who receive
the abundance
of grace
and
of the gift
of righteousness
reign
in
life
through
the one,
Jesus
Christ!
\par }{\PP \VS{18}Consequently,
just as
condemnation
for
all
people
came through
one
transgression,
so
too
through
the one
righteous act
came righteousness
leading to
life
for
all
people.
\VS{19}For
just as
through
the disobedience
of the one
man
many were made sinners, so also through the obedience of one man many will be made righteous.
\VS{20}Now
the law
came in
so that
the transgression
may increase,
but where
sin
increased,
grace
multiplied all the more,
\VS{21}so that
just as
sin
reigned
in
death,
so
also
grace
will reign
through
righteousness
to
eternal
life
through
Jesus
Christ
our
Lord.

\par }\Chap{6}{\PP \VerseOne{1}What
shall we say
then? Are we to remain
in sin
so that
grace
may increase?
\VS{2}Absolutely
not! How
can we
who
died
to sin
still
live
in
it?
\VS{3}Or
do you not know
that
as many as
were baptized
into
Christ
Jesus
were baptized
into
his
death?
\VS{4}Therefore
we have been buried
with him
through
baptism
into
death,
in order that
just as
Christ
was raised
from
the dead
through
the glory
of the Father,
so
we
too
may live
a new
life.
\par }{\PP \VS{5}For
if
we have become
united
with him in the likeness
of his
death,
we will
certainly also be
united in the likeness of his resurrection.
\VS{6}We know
that
our
old
man
was crucified
with him so that
the body
of sin
would no longer dominate
us,
so that we would no longer
be enslaved
to sin.
\VS{7}(For
someone who has died
has been freed
from
sin.)
\par }{\PP \VS{8}Now if
we died
with
Christ,
we believe
that
we will
also
live
with him.
\VS{9}We know
that
since Christ
has been raised
from
the dead,
he is never
going to die
again; death
no longer
has mastery
over him.
\VS{10}For
the death he died,
he died
to sin
once
for all, but
the life he lives,
he lives
to God.
\VS{11}So
you
too
consider
yourselves
dead
to sin,
but
alive
to God
in
Christ
Jesus.
\par }{\PP \VS{12}Therefore
do not
let sin
reign
in
your
mortal
body
so that
you obey
its
desires,
\VS{13}and
do not
present
your
members
to sin
as instruments
to be used for unrighteousness,
but
present
yourselves
to God
as
those who are alive
from
the dead
and
your
members
to God
as instruments
to be used for righteousness.
\VS{14}For
sin
will have
no
mastery
over you,
because
you are
not
under
law
but
under
grace.
\par }{\SH The Believer’s Enslavement to God’s Righteousness
\par }{\PP \VS{15}What
then? Shall we sin
because
we are
not
under
law
but
under
grace? Absolutely
not!
\VS{16}Do you
not
know
that
if you present
yourselves
as obedient
slaves,
you are
slaves
of the one you obey,
either
of sin
resulting in
death,
or
obedience
resulting in
righteousness?
\VS{17}But
thanks
be to God
that
though you were
slaves
to sin,
you obeyed
from
the heart
that
pattern
of teaching
you were entrusted
to,
\VS{18}and
having been freed
from
sin,
you became enslaved
to righteousness.
\VS{19}(I am speaking
in human terms
because of
the weakness
of your
flesh.)
 For
just
as you
once presented
your
members
as slaves
to impurity
and
lawlessness
leading to
more lawlessness,
so
now
present
your
members
as slaves
to righteousness
leading to
sanctification.
\VS{20}For
when
you were
slaves
of sin,
you were
free
with regard to righteousness.
\par }{\PP \VS{21}So
what
benefit
did you
then
reap
from those things that
you are
now
ashamed
of? For
the end
of those things
is death.
\VS{22}But
now,
freed
from
sin
and
enslaved
to God,
you have
your
benefit
leading to
sanctification,
and
the end
is eternal
life.
\VS{23}For
the payoff
of sin
is death,
but
the gift
of God
is eternal
life
in
Christ
Jesus
our
Lord.

\par }\Chap{7}{\PP \VerseOne{1}Or
do you not know,
brothers and sisters
(for
I am speaking
to those who know
the law), that
the law
is lord
over
a person
as long as
he lives?
\VS{2}For
a married
woman
is bound
by law
to
her husband
as long as he lives,
but
if
her husband
dies,
she is released
from
the law of the marriage.
\VS{3}So then, if
she is joined
to another
man
while her husband
is alive,
she will be called
an adulteress.
But
if
her husband
dies,
she is
free
from
that law,
and if she is joined to another
man,
she is
not
an adulteress.
\VS{4}So,
my
brothers and sisters,
you
also
died
to the law
through
the body
of Christ,
so that
you
could be joined
to another,
to the one who was raised
from
the dead,
to
bear fruit
to God.
\VS{5}For
when
we were
in
the flesh,
the sinful desires,
aroused
by
the law,
were active
in
the members
of our
body to
bear fruit
for death.
\VS{6}But
now
we have been released
from
the law,
because we have died
to
what
controlled
us, so that
we
may serve
in
the new life
of the Spirit
and
not
under the old
written code.
\par }{\PP \VS{7}What
shall we say
then? Is the law
sin? Absolutely
not! Certainly, I would not
have known
sin
except
through
the law.
For
indeed I would not
have known what it means to desire
something belonging to someone else if
the law
had
not
said, “{\QT{
Do
not
covet}}.”
\VS{8}But
sin,
seizing
the opportunity
through
the commandment,
produced
in
me
all kinds
of wrong desires.
For
apart from
the law,
sin
is dead.
\VS{9}And
I
was
once
alive
apart from
the law,
but
with the coming
of the commandment
sin
became alive
\VS{10}and I died. So I
found
that the very commandment
that was intended to bring
life
brought
death!
\VS{11}For
sin,
seizing
the opportunity
through
the commandment,
deceived
me
and
through
it
I died.
\VS{12}So then,
the law
is holy,
and
the commandment
is holy,
righteous,
and
good.
\par }{\PP \VS{13}Did that which is good,
then,
become
death
to me? Absolutely
not! But
sin,
so that
it would be shown
to be sin,
produced
death
in me
through
what is good,
so that
through
the commandment
sin
would become
utterly
sinful.
\VS{14}For
we know
that
the law
is
spiritual
– but I
am
unspiritual,
sold
into slavery to sin.
\VS{15}For
I
don’t
understand
what I am doing.
For
I do
not
do
what
I want
– instead,
I do
what
I hate.
\VS{16}But
if
I do
what
I
don’t
want,
I agree
that the law
is good.
\VS{17}But
now
it is no longer
me
doing
it,
but
sin
that lives
in
me.
\VS{18}For
I know
that
nothing
good
lives
in
me,
that
is,
in
my
flesh.
For
I
want
to do
the good,
but I cannot do it.
\VS{19}For
I do
not
do
the good
I want,
but
I do
the very evil
I do
not
want!
\VS{20}Now if
I do what
I do
not
want,
it is no longer
me
doing
it
but
sin
that lives
in
me.
\par }{\PP \VS{21}So,
I find
the law
that when I
want
to do
good,
evil
is present
with me.
\VS{22}For
I delight
in the law
of God
in my inner
being.
\VS{23}But
I see
a different
law
in
my
members
waging war against
the law
of my
mind
and
making
me
captive
to the law
of sin
that is
in
my
members.
\VS{24}Wretched
man
that I
am! Who
will rescue
me
from
this
body
of death?
\VS{25}Thanks
be to God
through
Jesus
Christ
our
Lord! So
then,
I
myself
serve
the law
of God
with my mind,
but
with my flesh
I serve the law
of sin.

\par }\Chap{8}{\PP \VerseOne{1}There is therefore
now
no
condemnation
for those who are in
Christ
Jesus.
\VS{2}For
the law
of the life-giving
Spirit
in
Christ
Jesus
has set
you
free
from
the law
of sin
and
death.
\VS{3}For
God
achieved
what
the law
could not do because it was weakened
through
the flesh.
By sending
his own
Son
in
the likeness
of sinful
flesh
and
concerning
sin,
he condemned
sin
in
the flesh,
\VS{4}so that
the righteous requirement
of the law
may be fulfilled
in
us,
who do
not
walk
according
to the flesh
but
according
to the Spirit.
\par }{\PP \VS{5}For
those who live
according
to the flesh
have their outlook shaped
by the things of the flesh,
but
those who live according to
the Spirit
have their outlook shaped by the things of the Spirit.
\VS{6}For
the outlook
of the flesh
is death,
but
the outlook
of the Spirit
is life
and
peace,
\VS{7}because
the outlook
of the flesh
is hostile
to
God,
for
it does
not
submit
to the law
of God,
nor
is it able to do so.
\VS{8}Those
who are
in
the flesh
cannot
please
God.
\VS{9}You,
however,
are
not
in
the flesh
but
in
the Spirit,
if indeed
the Spirit
of God
lives
in
you.
Now if
anyone
does not
have
the Spirit
of Christ,
this
person does not
belong
to him.
\VS{10}But
if
Christ
is in
you,
your body is dead because of sin, but the Spirit is your life because of righteousness.
\VS{11}Moreover if
the Spirit
of the one who raised
Jesus
from
the dead
lives
in
you,
the one who raised
Christ
from
the dead
will
also
make
your
mortal
bodies
alive
through
his
Spirit
who lives
in
you.
\par }{\PP \VS{12}So
then,
brothers and sisters,
we are
under obligation,
not
to the flesh,
to live
according to
the flesh
\VS{13}(for
if
you live
according to
the flesh,
you will
die), but
if
by the Spirit
you put to death
the deeds
of the body
you will live.
\VS{14}For
all
who are led
by the Spirit
of God
are
the sons
of God.
\VS{15}For
you did
not
receive
the spirit
of slavery
leading
again
to
fear,
but
you received
the Spirit
of adoption,
by
whom
we cry,
“Abba,
Father.”
\VS{16}The Spirit
himself
bears witness to
our
spirit
that
we are
God’s
children.
\VS{17}And
if
children,
then
heirs
(namely, heirs
of God
and also fellow heirs
with Christ) – if indeed we suffer with him so we may also be glorified with him.
\par }{\PP \VS{18}For
I consider
that
our present
sufferings
cannot
even be compared
to
the glory
that will
be revealed
to
us.
\VS{19}For
the creation
eagerly
waits for
the revelation
of the sons
of God.
\VS{20}For
the creation
was subjected
to futility
– not
willingly
but
because
of God who subjected it – in hope
\VS{21}that
the creation
itself
will
also
be set free
from
the bondage
of decay
into
the glorious
freedom
of God’s
children.
\VS{22}For
we know
that
the whole
creation
groans
and
suffers together
until
now.
\VS{23}Not
only
this, but
we
ourselves
also,
who have
the firstfruits
of the Spirit,
groan
inwardly
as we eagerly await
our adoption,
the redemption
of our
bodies.
\VS{24}For
in hope
we were saved.
Now hope
that is seen
is
not
hope,
because
who hopes
for what he sees?
\VS{25}But
if
we hope
for what
we do
not
see,
we
eagerly wait for
it with
endurance.
\par }{\PP \VS{26}In the same way,
the Spirit
helps
us in our
weakness,
for
we do
not
know
how we should
pray,
but
the Spirit
himself
intercedes
for us with inexpressible
groanings.
\VS{27}And
he who searches
our hearts
knows
the mind
of the Spirit,
because
the Spirit intercedes
on behalf
of the saints
according to
God’s will.
\VS{28}And
we know
that
all things
work together
for
good
for those who love
God,
who are
called
according
to his purpose,
\VS{29}because
those whom
he foreknew
he
also
predestined
to be conformed
to the image
of his
Son,
that his Son
would be
the firstborn
among
many
brothers and sisters.
\VS{30}And
those
he predestined,
he
also
called;
and
those he called,
he
also justified;
and
those
he justified,
he
also
glorified.
\par }{\PP \VS{31}What
then
shall we say
about
these things? If
God
is for
us,
who can be against
us?
\VS{32}Indeed,
he
who did
not
spare
his own
Son,
but
gave
him
up
for
us
all
– how
will he not
also,
along with
him,
freely give
us
all things?
\VS{33}Who
will bring any charge
against
God’s
elect? It is God
who justifies.
\VS{34}Who
is the one who will condemn? Christ
is the one who died
(and
more
than that, he was raised), who
is
at
the right hand
of God,
and
who
also is interceding
for
us.
\VS{35}Who
will separate
us
from
the love
of Christ? Will trouble,
or
distress,
or
persecution,
or
famine,
or
nakedness,
or
danger,
or
sword?
\VS{36}As
it is written, “{\QT{
For
your
sake
we encounter death
all
day
long; we were considered
as
sheep
to be slaughtered}}.”
\VS{37}No,
in
all
these things
we have complete victory
through
him who loved
us!
\VS{38}For
I am convinced
that
neither
death,
nor
life,
nor
angels,
nor
heavenly rulers,
nor
things that are present,
nor
things to come,
nor
powers,
\VS{39}nor
height,
nor
depth,
nor
anything
else
in creation
will be able
to separate
us
from
the love
of God
in
Christ
Jesus
our
Lord.

\par }\Chap{9}{\PP \VerseOne{1}I am telling
the truth
in
Christ
(I am
not
lying!), for my
conscience
assures
me
in
the Holy
Spirit –
\VS{2}I
have
great
sorrow
and
unceasing
anguish
in my
heart.
\VS{3}For
I
could wish
that I myself were
accursed
– cut off from
Christ
– for the sake
of my
people,
my
fellow countrymen,
\VS{4}who
are
Israelites.
To them
belong the adoption as sons,
the glory,
the covenants,
the giving of the law,
the temple worship,
and
the promises.
\VS{5}To them belong the patriarchs,
and
from
them,
by
human
descent, came the Christ,
who is
God
over
all,
blessed
forever! Amen.
\par }{\PP \VS{6}It is not
as though
the word
of God
had failed.
For
not
all
those who are descended from
Israel
are truly Israel,
\VS{7}nor
are
all
the children
Abraham’s
true descendants;
rather “{\QT{
through
Isaac
will
your
descendants
be counted}}.”
\VS{8}This
means it is
not
the children
of the flesh
who are the children
of God;
rather,
the children
of promise
are counted
as
descendants.
\VS{9}For
this
is what the promise
declared: “{\QT{
About
a year from now
I will return
and
Sarah
will have
a son}}.”
\VS{10}Not
only
that, but
when Rebekah
had
conceived
children by
one man,
our
ancestor
Isaac –
\VS{11}even before
they were born
or
had done
anything
good
or
bad
(so
that God’s
purpose
in
election
would stand,
not
by
works
but
by
his calling) –
\VS{12}it was said
to her, “{\QT{
The older
will serve
the younger}},”
\VS{13}just as
it is written: “{\QT{
Jacob
I loved, but
Esau
I hated}}.”
\par }{\PP \VS{14}What
shall we say
then? Is there injustice
with
God? Absolutely
not!
\VS{15}For
he says
to Moses: “{\QT{
I will have mercy
on whom
I have mercy, and
I will have compassion
on whom
I have compassion}}.”
\VS{16}So
then,
it does not
depend on human desire
or
exertion,
but
on God
who shows mercy.
\VS{17}For
the scripture
says
to Pharaoh: “{\QT{
For
this
very purpose I have raised
you
up,
that
I may demonstrate
my
power
in
you, and
that
my
name
may be proclaimed
in
all
the earth}}.”
\VS{18}So then,
God has mercy
on whom he chooses
to have mercy, and
he hardens
whom he chooses to harden.
\par }{\PP \VS{19}You will say
to me
then,
“Why
does he
still
find fault? For
who
has ever resisted
his
will?”
\VS{20}But who
indeed
are you
– a mere human being –
to talk back
to God?
{\QT{
Does what is molded
say
to the molder, “Why
have you made
me
like this?}}”
\VS{21}Has the potter
no
right
to make
from the same
lump
of clay
one vessel
for
special use
and
another for
ordinary use?
\VS{22}But
what if
God,
willing
to demonstrate
his wrath
and
to make known
his
power,
has endured
with
much
patience
the objects
of wrath
prepared
for
destruction?
\VS{23}And what if he is willing to make known
the wealth
of his
glory
on
the objects
of mercy
that
he has prepared beforehand
for
glory –
\VS{24}even
us,
whom
he has called,
not
only
from
the Jews
but
also
from
the Gentiles?
\VS{25}As
he also says in
Hosea:
\par }{\Q “{\QT{
I will call
those who were not
my
people, ‘My
people,’ and
I will call her who was unloved,
‘My beloved.}}*’ ”
\par }{\Q \VS{26}“{\QT{
And
in
the very place
where
it was said
to them, ‘You
are not
my
people,}}*’
\par }{\Q {\QT{
there
they will be called ‘sons
of the living
God}}.’ ”
\par }{\PP \VS{27}And
Isaiah
cries out
on behalf of
Israel,
{\IT{
“Though
the number
of the children
of Israel
are as
the sand
of the sea, only the remnant
will be saved}},
\VS{28}{\IT{
for
the Lord
will execute
his sentence
on
the earth
completely
and
quickly}}.”
\VS{29}Just as
Isaiah
predicted,
\par }{\Q “{\QT{
If
the Lord
of armies
had
not
left
us
descendants,}}*
\par }{\Q {\QT{
we would have become
like
Sodom,}}
\par }{\Q {\QT{
and
we
would
have resembled
Gomorrah.”}}
\par }{\SH Israel’s Rejection Culpable
\par }{\PP \VS{30}What
shall we say then? – that the Gentiles who did not pursue righteousness obtained it, that is, a righteousness that is by faith,
\VS{31}but
Israel
even though pursuing
a law
of righteousness
did
not
attain
it.
\VS{32}Why
not? Because
they pursued it not
by
faith
but
(as
if it were possible) by
works.
They stumbled over
the stumbling
stone,
\VS{33}just as
it is written,
\par }{\Q “{\QT{
Look, I am laying
in
Zion
a stone
that will cause people to stumble}}
\par }{\Q {\QT{
and
a rock
that will make them fall,}}
\par }{\Q {\QT{
yet the one who believes
in
him
will
not
be put to shame.}}”


\par }\Chap{10}{\PP \VerseOne{1}Brothers and sisters,
my
heart’s
desire
and
prayer
to
God
on behalf
of my fellow Israelites is for
their
salvation.
\VS{2}For
I can testify
that
they are
zealous
for God,
but
their zeal is not
in line with
the truth.
\VS{3}For
ignoring
the righteousness
that comes from God,
and
seeking
instead to establish
their own
righteousness,
they did
not
submit
to God’s righteousness.
\VS{4}For
Christ
is the end
of the law,
with the result that there is righteousness
for everyone
who believes.
\par }{\PP \VS{5}For
Moses
writes
about the righteousness
that is by
the law: “{\QT{
The one
who does
these things will live
by
them}}.”
\VS{6}But
the righteousness
that is by
faith
says: “{\QT{
Do
not
say
in
your
heart}}, ‘{\IT{
Who
will ascend
into
heaven}}?’ ” (that
is,
to bring Christ
down)
\VS{7}or “{\QT{
Who
will descend
into
the abyss?}}”
(that
is,
to bring
Christ
up
from
the dead).
\VS{8}But
what
does it say? “{\QT{
The word
is
near
you, in
your
mouth
and
in
your
heart}}”
(that
is,
the word
of faith
that
we preach),
\VS{9}because
if
you confess
with
your
mouth
that
Jesus
is Lord
and
believe
in
your
heart
that
God
raised
him
from
the dead,
you will be saved.
\VS{10}For
with the heart
one believes
and thus has righteousness
and
with the mouth
one confesses
and thus has salvation.
\VS{11}For
the scripture
says, “{\QT{
Everyone
who believes
in
him
will
not
be put to shame}}.”
\VS{12}For
there is
no
distinction
between the Jew
and
the Greek,
for
the same
Lord
is Lord of all,
who richly blesses
all
who call
on him.
\VS{13}For
{\QT{
everyone
who
calls
on the name
of the Lord
will be saved}}.
\par }{\PP \VS{14}How
are they to call on
one
they have
not
believed in? And
how
are they to believe in
one
they have
not
heard of? And
how
are they to hear
without
someone preaching to them ?
\VS{15}And
how
are they to preach
unless
they are sent? As
it is written, “{\QT{
How timely
is the arrival
of those who proclaim the good news}}*.”
\VS{16}But
not
all
have obeyed
the good news,
for
Isaiah
says, “{\QT{
Lord, who
has believed
our
report}}?”
\VS{17}Consequently
faith
comes from
what is heard,
and what is heard
comes through
the preached word
of Christ.
\par }{\PP \VS{18}But
I ask,
have they
not
heard? Yes, they have:

{\QT{
Their
voice
has gone out
to
all
the earth, and
their
words
to
the ends
of the world}}.
\VS{19}But
again I ask,
didn’t
Israel
understand? First
Moses
says, “{\QT{
I
will make
you
jealous
by those who are not
a nation;
with a senseless
nation
I will provoke
you to anger}}.”
\VS{20}And
Isaiah
is even bold enough
to say, “{\QT{
I was found
by those who did
not
seek
me;
I became
well known
to those who did
not
ask for
me}}.”
\VS{21}But
about Israel
he says, “{\QT{
All
day
long I held out
my
hands
to
this disobedient
and
stubborn
people!}}”

\par }\Chap{11}{\PP \VerseOne{1}So
I ask,
God
has not rejected
his
people,
has he? Absolutely
not! For
I
too am
an Israelite,
a descendant
of Abraham,
from the tribe
of Benjamin.
\VS{2}God
has not
rejected
his
people
whom
he foreknew! Do you
not
know
what
the scripture
says
about
Elijah,
how he pleads
with God
against
Israel?
\VS{3}“Lord,
{\QT{
they have killed
your
prophets, they have demolished
your
altars;
I
alone
am left
and
they are seeking
my
life!}}”
\VS{4}But
what
was the divine response
to him? “{\QT{
I have kept
for myself
seven thousand
people
who
have
not
bent
the knee
to Baal}}*.”
\par }{\PP \VS{5}So
in the same way
at
the present
time
there is
a remnant
chosen
by
grace.
\VS{6}And
if
it is by grace,
it is no longer
by
works,
otherwise
grace
would
no longer
be
grace.
\VS{7}What
then? Israel
failed
to obtain
what
it was diligently seeking,
but
the elect
obtained
it. The rest
were hardened,
\VS{8}as
it is written,
\par }{\Q {\QT{
“God
gave
them
a spirit
of stupor,}}
\par }{\Q {\QT{
eyes
that would
not
see
and
ears
that would
not
hear,}}
\par }{\Q {\QT{
to
this very day.”}}
\par }{\PP \VS{9}And
David
says,
\par }{\Q {\QT{
“Let
their
table
become
a snare
and
trap,}}
\par }{\Q {\QT{
a stumbling block
and
a retribution
for them}};
\par }{\Q \VS{10}{\QT{
let
their
eyes
be darkened
so that they may not
see,}}
\par }{\Q {\QT{
and
make their
backs
bend continually.”}}
\par }{\PP \VS{11}I ask
then,
they did
not
stumble
into an irrevocable fall,
did they? Absolutely not! But
by their
transgression
salvation
has come to the Gentiles,
to
make
Israel
jealous.
\VS{12}Now
if
their
transgression
means riches
for the world
and
their
defeat
means riches
for the Gentiles,
how much
more
will their
full restoration bring?
\par }{\PP \VS{13}Now
I am speaking
to you
Gentiles.
Seeing that I
am
an apostle
to the Gentiles,
I magnify
my
ministry,
\VS{14}if
somehow
I could provoke
my
people
to jealousy
and
save
some
of
them.
\VS{15}For
if
their
rejection
is the reconciliation
of the world,
what
will their acceptance
be but
life
from
the dead?
\VS{16}If
the first portion of the dough offered
is holy,
then
the whole batch
is holy, and
if
the root
is holy,
so too
are the branches.
\par }{\PP \VS{17}Now
if
some
of the branches
were broken off,
and
you,
a wild olive
shoot, were grafted
in among
them
and
participated
in the richness
of the olive
root,
\VS{18}do
not
boast over
the branches.
But
if
you boast,
remember that you
do not
support
the root,
but
the root
supports you.
\VS{19}Then
you will say,
“The branches
were broken off
so that
I
could be grafted in.”
\VS{20}Granted! They were broken off
because of their unbelief,
but you stand
by faith.
Do
not
be arrogant,
but
fear!
\VS{21}For
if
God
did not spare the natural
branches,
perhaps he will
not
spare
you.
\VS{22}Notice
therefore
the kindness
and
harshness
of God
– harshness
toward
those who have fallen,
but
God’s
kindness
toward
you,
provided you continue
in his kindness;
otherwise
you
also
will be cut off.
\VS{23}And
even they
– if
they do
not
continue
in their unbelief
– will be grafted in,
for
God
is
able
to graft
them
in
again.
\VS{24}For
if
you
were cut off
from what is by
nature
a wild olive tree,
and
grafted,
contrary
to nature,
into
a cultivated olive tree,
how much
more
will
these
natural
branches be grafted back
into their own
olive tree?
\par }{\PP \VS{25}For
I do
not
want
you
to be ignorant
of this
mystery,
brothers and sisters,
so that
you may
not
be
conceited: A partial
hardening
has happened
to Israel
until
the full
number of the Gentiles
has come in.
\VS{26}And
so
all
Israel
will be saved,
as
it is written:
\par }{\Q {\QT{
“The Deliverer
will come out
of
Zion;}}
\par }{\Q {\QT{
he will remove
ungodliness
from
Jacob}}.
\par }{\Q \VS{27}{\QT{
And
this
is my
covenant
with them}},
\par }{\Q {\QT{
when
I take away
their
sins.”}}
\par }{\PP \VS{28}In regard to
the gospel
they are enemies
for
your
sake,
but
in regard to
election
they are dearly loved
for the sake
of the fathers.
\VS{29}For
the gifts
and
the call
of God
are irrevocable.
\VS{30}Just as
you
were
formerly
disobedient
to God,
but
have
now
received mercy
due to their
disobedience,
\VS{31}so
they
too
have
now
been disobedient
in order that,
by the mercy
shown to you,
they
too
may
now
receive mercy.
\VS{32}For
God
has consigned
all people
to disobedience
so that
he may show mercy
to them all.
\par }{\PP \VS{33}Oh,
the depth
of the riches
and
wisdom
and
knowledge
of God! How unsearchable
are his
judgments
and
how fathomless
his
ways!
\par }{\Q \VS{34}{\QT{
For
who
has known
the mind
of the Lord,}}
\par }{\Q {\QT{
or
who
has been
his
counselor?}}
\par }{\Q \VS{35}{\QT{
Or
who
has first given
to God,}}
\par }{\Q {\QT{
that God needs to repay
him?}}
\par }{\PP \VS{36}For
from
him
and
through
him
and
to
him
are all things.
To him
be glory
forever! Amen.

\par }\Chap{12}{\PP \VerseOne{1}Therefore
I exhort
you,
brothers and sisters,
by
the mercies
of God,
to present
your
bodies
as a sacrifice
– alive,
holy,
and pleasing
to God –
which is your
reasonable
service.
\VS{2}Do
not
be conformed
to this
present world, but
be transformed
by the renewing
of your mind,
so that
you
may test and approve
what
is the will
of God
– what is good
and
well-pleasing
and
perfect.
\par }{\SH Conduct in Humility
\par }{\PP \VS{3}For
by
the grace
given
to me
I say to every
one of you
not
to think more highly
of yourself than you ought
to think,
but
to think
with
sober discernment,
as
God
has distributed
to each
of you a measure
of faith.
\VS{4}For
just as
in
one
body
we have
many
members,
and
not
all
the members
serve
the same
function,
\VS{5}so
we who are many
are
one
body
in
Christ,
and
individually we are members
who belong to
one
another.
\VS{6}And
we have
different
gifts
according
to the grace
given
to us.
If the gift is prophecy,
that individual must use it in proportion
to his faith.
\VS{7}If
it is service,
he must serve;
if
it is teaching,
he must teach;
\VS{8}if
it is exhortation,
he must exhort;
if it is contributing,
he must do so with sincerity;
if it is leadership,
he must do so with diligence;
if it is showing mercy,
he must do so with cheerfulness.
\par }{\SH Conduct in Love
\par }{\PP \VS{9}Love
must be without hypocrisy.
Abhor
what is evil,
cling
to what is good.
\VS{10}Be devoted
to
one another
with mutual love,
showing eagerness
in honoring
one another.
\VS{11}Do not
lag
in zeal,
be enthusiastic
in spirit,
serve
the Lord.
\VS{12}Rejoice
in hope,
endure
in suffering,
persist
in prayer.
\VS{13}Contribute
to the needs
of the saints,
pursue
hospitality.
\VS{14}Bless
those who persecute
you, bless
and
do
not
curse.
\VS{15}Rejoice
with
those who rejoice,
weep
with
those who weep.
\VS{16}Live in harmony with one another;
do not
be haughty
but
associate with
the lowly.
Do
not
be
conceited.
\VS{17}Do
not repay
anyone
evil
for
evil;
consider
what is good
before
all
people.
\VS{18}If
possible,
so far as it depends on
you,
live peaceably
with
all
people.
\VS{19}Do not
avenge
yourselves,
dear friends,
but
give
place
to God’s wrath,
for
it is written, “{\QT{
Vengeance
is mine, I
will repay}},” says
the Lord.
\VS{20}Rather,
{\QT{
if
your
enemy
is hungry, feed
him;
if
he is thirsty,
give
him
a drink;
for
in doing
this
you will be heaping
burning
coals
on
his
head}}.
\VS{21}Do
not
be overcome
by
evil,
but
overcome
evil
with
good.

\par }\Chap{13}{\PP \VerseOne{1}Let every
person
be subject
to the governing authorities.
For
there is
no
authority
except
by
God’s
appointment, and
the authorities that exist
have been instituted
by
God.
\VS{2}So
the person who resists
such authority
resists the ordinance
of God,
and
those who resist
will incur
judgment
\VS{3}(for
rulers
cause no
fear
for good
conduct
but
for bad). Do you desire
not
to fear
authority? Do
good
and
you will receive
its
commendation,
\VS{4}for
it is
God’s
servant
for
your
good.
But
if
you do
wrong,
be in fear,
for
it does
not
bear
the sword
in vain.
It is
God’s
servant
to
administer
retribution
on the wrongdoer.
\VS{5}Therefore
it is necessary
to be in subjection,
not
only
because
of the wrath
of the authorities but
also
because
of your conscience.
\VS{6}For
this reason
you
also
pay
taxes,
for
the authorities
are
God’s servants
devoted
to
governing.
\VS{7}Pay
everyone
what is owed: taxes
to whom taxes
are due,
revenue
to whom revenue
is due, respect
to whom respect
is due, honor
to whom honor is due.
\par }{\SH Exhortation to Love Neighbors
\par }{\PP \VS{8}Owe
no one
anything, except
to love one another,
for
the one who loves
his neighbor
has fulfilled
the law.
\VS{9}For
the commandments, “{\QT{
Do
not
commit adultery, do
not
murder, do
not
steal, do
not
covet,}}”
(and
if
there is any
other
commandment) are summed up
in
this, “{\QT{
Love
your
neighbor
as
yourself}}.”
\VS{10}Love
does
no
wrong
to a neighbor.
Therefore
love
is the fulfillment
of the law.
\par }{\SH Motivation to Godly Conduct
\par }{\PP \VS{11}And
do this
because we know
the time,
that
it is already
the hour
for us
to awake
from
sleep,
for
our
salvation
is now
nearer
than
when
we became believers.
\VS{12}The night
has advanced toward dawn;
the day
is near.
So then
we must lay aside
the works
of darkness,
and
put on
the weapons
of light.
\VS{13}Let us live
decently
as
in
the daytime,
not
in carousing
and
drunkenness,
not
in sexual immorality
and
sensuality,
not
in discord
and
jealousy.
\VS{14}Instead,
put
on the Lord
Jesus
Christ,
and
make
no
provision
for the flesh
to arouse
its desires.

\par }\Chap{14}{\PP \VerseOne{1}Now
receive
the one who is weak
in the faith,
and do not
have disputes
over differing opinions.
\VS{2}One
person believes
in eating
everything,
but
the weak
person eats
only vegetables.
\VS{3}The one who eats
everything must not
despise
the one who does
not,
and
the one who abstains
must
not
judge
the one who eats
everything, for
God
has accepted
him.
\VS{4}Who
are
you
to pass judgment
on another’s
servant? Before his own
master
he stands
or
falls.
And
he will stand,
for
the Lord
is able
to make
him
stand.
\par }{\PP \VS{5}One
person regards
one day
holier than
other days,
and
another regards
them
all
alike. Each
must be fully convinced
in
his own
mind.
\VS{6}The one who observes
the day
does it for the Lord.
The one who eats,
eats
for the Lord
because
he gives thanks
to God,
and
the one who abstains
from eating
abstains
for the Lord,
and
he gives thanks
to God.
\VS{7}For
none
of us
lives
for himself
and
none
dies
for himself.
\VS{8}If
we live,
we live
for
the Lord;
if
we die,
we die
for the Lord.
Therefore,
whether
we live
or
die,
we are
the Lord’s.
\VS{9}For
this reason
Christ
died
and
returned to life,
so that
he may be the Lord
of both
the dead
and
the living.
\par }{\PP \VS{10}But
you who eat vegetables only – why do you judge your brother or sister? And you who eat everything – why do you despise your brother or sister? For we will all stand before the judgment seat of God.
\VS{11}For
it is written, “{\QT{
As I
live, says
the Lord, every
knee
will bow
to me, and
every
tongue
will give praise
to God}}.”
\VS{12}Therefore,
each
of us
will give
an account
of
himself
to God.
\par }{\SH Exhortation for the Strong not to Destroy the Weak
\par }{\PP \VS{13}Therefore
we must
not
pass judgment
on one another,
but
rather
determine
never
to place
an obstacle
or
a trap
before a brother or sister.
\VS{14}I know
and
am convinced
in
the Lord
Jesus
that
there is nothing
unclean
in
itself;
still, it is
unclean
to the one
who considers
it unclean.
\VS{15}For
if
your
brother or sister
is distressed
because of
what you eat,
you are
no longer
walking
in
love.
Do
not
destroy
by your
food
someone for
whom
Christ
died.
\VS{16}Therefore
do not
let what you
consider good
be spoken of as evil.
\VS{17}For
the kingdom
of God
does
not
consist of
food
and
drink,
but
righteousness,
peace,
and
joy
in
the Holy
Spirit.
\VS{18}For
the one who serves
Christ
in
this
way is pleasing
to God
and
approved
by people.
\par }{\PP \VS{19}So then,
let us pursue
what makes for peace
and
for building up
one another.
\VS{20}Do
not
destroy
the work
of God
for the sake
of food.
For although
all things
are clean,
it is wrong
to cause anyone
to stumble
by
what you eat.
\VS{21}It is good
not
to eat
meat
or
drink
wine
or
to do anything that causes your
brother
to stumble.
\VS{22}The faith
you
have,
keep to
yourself
before
God.
Blessed
is the one who does
not
judge
himself
by
what
he approves.
\VS{23}But
the man who doubts
is condemned
if
he eats,
because
he does not
do so from
faith,
and whatever
is not
from
faith
is
sin.

\par }\Chap{15}{\PP \VerseOne{1}But
we
who are strong
ought
to bear
with the failings
of the weak,
and
not
just please
ourselves.
\VS{2}Let each
of us
please
his neighbor
for
his good
to
build him up.
\VS{3}For
even Christ
did
not
please
himself,
but
just as
it is written, “{\QT{
The insults
of those who insult
you
have fallen
on
me}}.”
\VS{4}For
everything
that was written in former times
was written
for
our
instruction,
so that
through
endurance
and
through
encouragement
of the scriptures
we may have
hope.
\VS{5}Now
may the God
of endurance
and
comfort
give
you
unity
with
one another
in accordance with
Christ
Jesus,
\VS{6}so that
together
you may
with
one
voice
glorify
the God
and
Father
of our
Lord
Jesus
Christ.
\par }{\SH Exhortation to Mutual Acceptance
\par }{\PP \VS{7}Receive
one another,
then, just as
Christ
also
received
you,
to
God’s
glory.
\VS{8}For
I tell
you that Christ
has become
a servant
of the circumcised
on behalf
of God’s
truth
to
confirm
the promises
made to the fathers,
\VS{9}and
thus the Gentiles
glorify
God
for
his mercy.
As
it is written, “{\QT{
Because
of this
I will confess
you
among
the Gentiles, and
I will sing praises
to your
name}}.”
\VS{10}And
again
it says: “{\QT{
Rejoice, O Gentiles, with
his
people}}.”
\VS{11}And
again, “{\QT{
Praise
the Lord
all
you Gentiles, and
let all
the peoples
praise
him}}.”
\VS{12}And
again
Isaiah
says, “{\QT{
The root
of Jesse
will come, and
the one who rises
to rule over
the Gentiles, in
him
will
the Gentiles
hope}}.”
\VS{13}Now
may the God
of hope
fill
you
with all
joy
and
peace
as you believe
in
him, so that you
may abound
in
hope
by
the power
of the Holy
Spirit.
\par }{\SH Paul’s Motivation for Writing the Letter
\par }{\PP \VS{14}But
I
myself am fully convinced
about
you,
my
brothers and sisters,
that
you
yourselves are
full
of goodness,
filled
with all
knowledge,
and able
to instruct
one another.
\VS{15}But
I have written
more boldly
to you
on
some points
so as to remind
you,
because
of the grace
given
to me
by
God
\VS{16}to be
a minister
of Christ
Jesus
to
the Gentiles.
I serve
the gospel
of God
like a priest,
so that
the Gentiles
may become
an acceptable
offering,
sanctified
by
the Holy
Spirit.
\par }{\PP \VS{17}So
I boast
in
Christ
Jesus
about the things that pertain to
God.
\VS{18}For
I will
not
dare
to speak
of anything
except what Christ
has accomplished
through
me
in order to bring about the obedience
of the Gentiles,
by word
and
deed,
\VS{19}in
the power
of signs
and
wonders,
in
the power
of the Spirit
of God. So
from
Jerusalem
even
as far as
Illyricum
I have fully preached
the gospel
of Christ.
\VS{20}And
in this way
I desire
to preach
where
Christ
has
not
been named,
so
as not
to build
on
another person’s
foundation,
\VS{21}but
as
it is written: “{\QT{
Those who
were
not
told
about
him
will see, and
those who have
not
heard
will understand}}.”
\par }{\SH Paul’s Intention of Visiting the Romans
\par }{\PP \VS{22}This is the reason
I was
often
hindered
from coming
to
you.
\VS{23}But
now
there is nothing
more to keep me in
these
regions,
and
I have
for
many
years
desired
to come
to
you
\VS{24}when
I go
to
Spain.
For
I hope
to visit
you
when
I pass through
and
that you
will help
me on
my journey
there,
after I have enjoyed
your company for a while.
\par }{\PP \VS{25}But
now
I go
to
Jerusalem
to minister
to the saints.
\VS{26}For
Macedonia
and
Achaia
are pleased
to make
some
contribution
for
the poor
among the saints
in
Jerusalem.
\VS{27}For
they were pleased
to do this, and
indeed they are
indebted
to the Jerusalem saints.
For
if
the Gentiles
have shared
in their
spiritual things,
they are obligated
also
to minister
to them
in
material things.
\VS{28}Therefore
after
I have completed
this
and
have safely delivered
this
bounty
to them,
I will set out
for
Spain
by way
of you,
\VS{29}and
I know
that
when
I come
to
you
I will come
in
the fullness
of Christ’s
blessing.
\par }{\PP \VS{30}Now I urge
you,
brothers and sisters,
through
our
Lord
Jesus
Christ
and
through
the love
of the Spirit,
to join fervently
with me
in
prayer
to
God
on
my behalf.
\VS{31}Pray
that I may be rescued
from
those who are disobedient
in
Judea
and
that my
ministry
in
Jerusalem
may be
acceptable
to the saints,
\VS{32}so that
by
God’s
will
I may come
to
you
with
joy
and be refreshed
in your company.
\VS{33}Now
may the God
of peace
be with
all
of you.
Amen.

\par }\Chap{16}{\PP \VerseOne{1}Now I commend
to you
our
sister
Phoebe,
who is
a servant
of the church
in
Cenchrea,
\VS{2}so that
you may welcome
her
in
the Lord
in a way worthy
of the saints
and
provide
her
with
whatever
help she may need
from you,
for
she
has been
a great help to many,
including
me.
\par }{\PP \VS{3}Greet
Prisca
and
Aquila,
my
fellow workers
in
Christ
Jesus,
\VS{4}who
risked
their own
necks
for
my
life.
Not
only
I,
but
all
the churches
of the Gentiles are grateful to them.
\VS{5}Also
greet the church
in their
house.
Greet
my
dear friend
Epenetus,
who
was
the first convert
to
Christ
in the province of Asia.
\VS{6}Greet
Mary,
who
has worked
very
hard
for
you.
\VS{7}Greet
Andronicus
and
Junia,
my
compatriots
and
my
fellow prisoners.
They
are
well known
to
the apostles,
and
they were
in
Christ
before
me.
\VS{8}Greet
Ampliatus,
my
dear friend
in
the Lord.
\VS{9}Greet
Urbanus,
our
fellow worker
in
Christ,
and
my
good friend
Stachys.
\VS{10}Greet
Apelles,
who is approved
in
Christ.
Greet
those who belong to the household of Aristobulus.
\VS{11}Greet
Herodion,
my
compatriot.
Greet
those in the household of Narcissus
who are
in
the Lord.
\VS{12}Greet
Tryphena
and
Tryphosa,
laborers
in
the Lord.
Greet
my dear friend
Persis,
who
has worked
hard
in
the Lord.
\VS{13}Greet
Rufus,
chosen
in
the Lord,
and
his
mother
who was also
a mother to me.
\VS{14}Greet
Asyncritus,
Phlegon,
Hermes,
Patrobas,
Hermas,
and
the brothers and sisters
with
them.
\VS{15}Greet
Philologus
and
Julia,
Nereus
and
his
sister,
and
Olympas,
and
all
the believers
who are with
them.
\VS{16}Greet
one another
with
a holy
kiss.
All
the churches
of Christ
greet
you.
\par }{\PP \VS{17}Now
I urge
you,
brothers and sisters,
to watch out
for those who create dissensions
and
obstacles
contrary
to the teaching
that
you
learned.
Avoid
them!
\VS{18}For
these are the kind
who do
not
serve
our
Lord
Christ,
but
their own
appetites.
By
their smooth talk
and
flattery
they deceive
the minds
of the naive.
\VS{19}Your
obedience
is known
to
all
and thus I rejoice
over
you.
But
I want
you
to be
wise
in
what is good
and
innocent
in
what is evil.
\VS{20}The
God
of peace
will
quickly
crush
Satan
under
your
feet.
The grace
of our
Lord
Jesus
be with
you.
\par }{\PP \VS{21}Timothy,
my
fellow worker,
greets
you;
so do Lucius,
Jason,
and
Sosipater,
my
compatriots.
\VS{22}I,
Tertius,
who am writing
this letter,
greet
you
in
the Lord.
\VS{23}Gaius,
who is host
to me
and
to the whole
church,
greets
you.
Erastus
the city
treasurer
and
our brother
Quartus greet you.
\VS{24}[[EMPTY]]
\VS{25}\par }{\PP Now to him who is able
to strengthen
you
according
to my
gospel
and
the proclamation
of Jesus
Christ,
according to
the revelation
of the mystery
that had been kept secret
for long
ages,
\VS{26}but
now
is disclosed,
and through
the prophetic
scriptures
has been made known
to
all
the nations,
according to
the command
of the eternal
God,
to bring about the obedience
of faith –
\VS{27}to the only
wise
God,
through
Jesus
Christ,
be glory
forever! Amen.
\par }